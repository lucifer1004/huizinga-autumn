\chapter{THE FAILURE OF IMAGINATION}

SYMBOLISM WAS VERY NEARLY THE LIFE'S BREATH OF medieval thought. The
habit of seeing all things in their meaningful interrelationships and
their relationship to the eternal both muted the boundaries between
things and kept the world of thought alive with radiant, glowing color.
Once, however, the symbolizing function has disappeared or become merely
mechanical, the grand edifice of God-willed dependencies becomes a
necropolis. A systematic idealism that everywhere presupposes a
relationship between things as a result of their assumed essential
general characteristics leads to a rigid and barren cataloguing in which
the division and subdivision of terms, carried out purely deductively,
is all too convenient. Ideas can be made to fit into the vault of the
world edifice so readily. However, with the exception of the rules of
abstract logic, there is no corrective that could ever point to an error
in classification, so that the mind is deceived as to the value of its
conclusions and the infallibility of the system itself is overestimated.
All terms, precise and imprecise, stand like stars in the firmament and
in order to come to know the nature of a thing one does not inquire into
its internal construction or into the long shadow of its historical
development, but looks towards the heavens where it shines as an idea.

The habit of always extending things towards the ideal along an
imaginary line is shown continuously in the medieval treatment of
political, social, and ethical disputes. Even what is most mundane and
common must be viewed in a universal context. For example, there was an
ongoing controversy at the University of Paris as to whether any kind of
payment should be asked for the degree of licentiate. Pierre d'Ailly
himself took the floor to oppose the fee in opposition to the chancellor
of the university. Instead of debating whether the demand was
historically justified or debating its validity according to the legal
code, d'Ailly framed his argument entirely
\protect\hypertarget{17_Chapter_Ten__THE_FAILURE_OF_IMAG.xhtmlux5cux23page_250}{}{}in
a scholastic manner: based on the text ``Radix omnium malorum
cupiditas,'' d'Ailly took on the task of proving three things: that to
demand the payment constituted simony; that it went against natural and
divine law; and that it was a
heresy.\textsuperscript{\protect\hypertarget{17_Chapter_Ten__THE_FAILURE_OF_IMAG.xhtmlux5cux23id_734}{\protect\hyperlink{23_NOTES.xhtmlux5cux23id_735}{1}}}
In order to criticize specific excesses that had disgraced a certain
procession, Denis the Carthusian puts together everything concerning the
procession from its beginning: as things happened under the old law,
etc.\textsuperscript{\protect\hypertarget{17_Chapter_Ten__THE_FAILURE_OF_IMAG.xhtmlux5cux23id_732}{\protect\hyperlink{23_NOTES.xhtmlux5cux23id_733}{2}}}
without dealing with the issues themselves. This is the reason for the
tedious and disappointing nature of almost any medieval proof; it points
immediately to the sky above and loses itself from the very beginning in
cases from Holy Scripture and in moral generalities.

This perfect idealism reveals itself everywhere. Every mode of life,
every social estate or occupation found itself circumscribed by a
religious-ethical ideal according to which everyone has to reform
himself to meet the requirements of his profession in order to serve the
Lord
properly.\textsuperscript{\protect\hypertarget{17_Chapter_Ten__THE_FAILURE_OF_IMAG.xhtmlux5cux23id_730}{\protect\hyperlink{23_NOTES.xhtmlux5cux23id_731}{3}}}
The emphasis given by Denis the Carthusian to the sanctity of the
earthly profession has been interpreted as a sign of the new times,
something characteristic of the Reformation. In the tract \emph{De vita
et regimine, episcoporum, archidiaconorum}, etc., which he ultimately
summarized, for his friend Brugman, in two volumes collectively entitled
\emph{De doctrina et regulis vital christianorium}, he held up to every
profession the ideal of the sanctifying fulfillment of duty; to the
bishop, the prelate, the archdeacon, the canon, the pastor, the
disciple, the prince, the nobleman, the knight, the merchants, the
married, the widows, the young maidens, the brothers in the
monasteries.\textsuperscript{\protect\hypertarget{17_Chapter_Ten__THE_FAILURE_OF_IMAG.xhtmlux5cux23id_728}{\protect\hyperlink{23_NOTES.xhtmlux5cux23id_729}{4}}}
But there is something truly medieval about just this separation of each
estate into something independent, and the detailed description and
teaching of duties has about it only abstract and general qualities and
does not touch the real character of the profession.

This universal tracing back to the general is a quality that, under the
label ``typism,''
Lamprecht\textsuperscript{\protect\hypertarget{17_Chapter_Ten__THE_FAILURE_OF_IMAG.xhtmlux5cux23id_726}{\protect\hyperlink{23_NOTES.xhtmlux5cux23id_727}{5}}}
singled out as the very special characteristic feature of the medieval
mind. But this feature is a mental consequence, a need, that arises from
deeply rooted idealism. It is not so much an inability to see things in
their own individuality as much as it is the deliberate desire to
indicate the relationship of things to the highest point of reference,
to their ethical ideality and their general significance. It was
precisely the impersonal element that was sought out in everything; the
value of anything was taken to be its value as a normal and model case.
This lack of interest in
\protect\hypertarget{17_Chapter_Ten__THE_FAILURE_OF_IMAG.xhtmlux5cux23page_251}{}{}a
thing's individuality and uniqueness, to a certain degree intentional,
is a universalizing habit of thought characteristic of a low degree of
intellectual development.

The medieval mind busied itself to the highest degree with dissecting
the entire world and all of life into independent ideas, only to arrange
these ideas into numerous large feudal relationships or intellectual
hierarchies. Thus the medieval mind was able to separate every concept
from the context to which it belonged and see it in its essential
independent existence. When Bishop Fulco of Toulouse was rebuked for
giving alms to an Albigensian woman, he answered, ``I do not give it to
the heretic, but to the poor woman.'' And the French Queen Margareta of
Scotland, who had kissed the sleeping poet Alain Chartier on the mouth,
excused her behavior, ``Je n'ay pas baisé l'homme mais la précieuse
bouche de laquelle sont yssuz et sortis tant de bons mots et vertueuses
paroles.''\textsuperscript{\protect\hypertarget{17_Chapter_Ten__THE_FAILURE_OF_IMAG.xhtmlux5cux23id_724}{\protect\hyperlink{23_NOTES.xhtmlux5cux23id_725}{6}}}\protect\hypertarget{17_Chapter_Ten__THE_FAILURE_OF_IMAG.xhtmlux5cux23id_2583}{\protect\hyperlink{23_NOTES.xhtmlux5cux23id_2584}{*\textsuperscript{1}}}
A saying had it, ``Haereticare potero, sed haereticus non
ero.''\textsuperscript{\protect\hypertarget{17_Chapter_Ten__THE_FAILURE_OF_IMAG.xhtmlux5cux23id_722}{\protect\hyperlink{23_NOTES.xhtmlux5cux23id_723}{7}}}\protect\hypertarget{17_Chapter_Ten__THE_FAILURE_OF_IMAG.xhtmlux5cux23id_2585}{\protect\hyperlink{23_NOTES.xhtmlux5cux23id_2586}{†\textsuperscript{2}}}
Does all this---in these examples from the realm of ordinary
thought---correspond to what in the highest speculation of theology was
meant to distinguish between God's \emph{voluntas
antecedens},\protect\hypertarget{17_Chapter_Ten__THE_FAILURE_OF_IMAG.xhtmlux5cux23id_2587}{\protect\hyperlink{23_NOTES.xhtmlux5cux23id_2588}{‡\textsuperscript{3}}}
desiring the salvation of all, and his \emph{voluntas
consequens},\protect\hypertarget{17_Chapter_Ten__THE_FAILURE_OF_IMAG.xhtmlux5cux23id_2589}{\protect\hyperlink{23_NOTES.xhtmlux5cux23id_2590}{§\textsuperscript{4}}}
which extends salvation only to the
elect?\textsuperscript{\protect\hypertarget{17_Chapter_Ten__THE_FAILURE_OF_IMAG.xhtmlux5cux23id_720}{\protect\hyperlink{23_NOTES.xhtmlux5cux23id_721}{8}}}

It all became an insomniac's gnawing mulling over of all things
unrestrained by the causal connections seen in reality; a virtually
automatic analysis that finally amounted to nothing more than an endless
exercise in numbering. No arena was more tempting for such elaborations
than that of virtues and sins. Every sin has its fixed number of causes,
its derivatives, its daughters, and its harmful effects. Twelve errors,
said Denis, cheat the sinner: he deceives himself, he surrenders to the
devil, he takes his own life, he rejects his wealth (his virtue), he
sells himself for nothing (while he himself has been bought with the
blood of Christ), he turns away from his most faithful lover, he thinks
he is resisting the Almighty, he serves the devil, he acquires absence
of peace, he opens for himself access to hell, he blocks his path to
heaven, and he follows that to hell.
\protect\hypertarget{17_Chapter_Ten__THE_FAILURE_OF_IMAG.xhtmlux5cux23page_252}{}{}Each
one of these errors is illustrated, depicted, and, in a way, defined
with passages of scripture, images, and details. It is so defined that
it acquires the decided certainty and independence of a figure on a
church portal. Then the same sequence is given anew with a deeper
meaning: the seriousness of a sin has to be measured from seven
standpoints: from the standpoint of God, from that of the sinner, the
content, circumstances, purpose, from the standpoint of the nature of
sin itself, and from that of its consequences. Several of these points
are, in turn, again subdivided into eight or fourteen others; for
example, the second into fourteen: the sin is heavier or lighter
depending on the received benefits, on knowledge, earlier virtues, the
office, the consecration, the ability to offer resistance, faith, age.
There are six weaknesses of the spirit that make one prone to
sin.\textsuperscript{\protect\hypertarget{17_Chapter_Ten__THE_FAILURE_OF_IMAG.xhtmlux5cux23id_718}{\protect\hyperlink{23_NOTES.xhtmlux5cux23id_719}{9}}}
This process can be compared to Buddhism: there too, we find this kind
of systematic morality that is designed to provide guidance for the
exercise of virtue.

This anatomizing of sin could easily have weakened the feeling of guilt
that it was supposed to strengthen, deflecting it into acts of squeezing
the classifications for all they were worth, if it had not at the same
time stimulated the imagination of sin and the notion of punishment. No
human can perfectly grasp or completely understand the enormity of sin
in his present
life.\textsuperscript{\protect\hypertarget{17_Chapter_Ten__THE_FAILURE_OF_IMAG.xhtmlux5cux23id_716}{\protect\hyperlink{23_NOTES.xhtmlux5cux23id_717}{10}}}
All moral conceptions are made to carry an intolerably heavy burden by
directly linking them over and over again to the majesty of God. Every
sin, no matter how trivial, affects the entire universe. Just as
Buddhist literature, encountering the great deed of a bodhisattva, hears
the applause of the heavenly beings in the form of rains of flowers,
shining light, and gentle earthquakes, Denis---in his more somber
mood---hears how all the blessed and the just, the heavenly spheres, all
the elements, and even unintelligent beings and soulless objects shout
the condemnation of the
unjust.\textsuperscript{\protect\hypertarget{17_Chapter_Ten__THE_FAILURE_OF_IMAG.xhtmlux5cux23id_714}{\protect\hyperlink{23_NOTES.xhtmlux5cux23id_715}{11}}}
His attempt to sharpen the fear of sin, death, justice, and hell in a
most painful manner by offering detailed descriptions and dreadful
images does not fall short of a terrifying effect, perhaps precisely
because of the nonpoetic way his mind worked. Dante touched the darkness
and cruelty of hell with beauty: Farinata and Ugolino are heroic in
their corruption and Lucifer, flapping his wings, impresses us with his
majesty. The monk Denis remains totally unpoetic in spite of his mystic
intensity. He presents hell to us exclusively in terms of highest dread
and misery. Pain and suffering are described in acid colors
\protect\hypertarget{17_Chapter_Ten__THE_FAILURE_OF_IMAG.xhtmlux5cux23page_253}{}{}that
the sinner should make every effort to imagine as realistically as
possible. ``Let us always keep before our mind's eye,'' says Denis, ``an
overheated and glowing stove and inside a naked man, supine, who will
never be released from such pain. Does not his pain appear unbearable to
us for even a single moment? How lost he appears to us! Just imagine how
he is writhing in the stove, how he screams, cries, lives, what dread he
suffers, what sufferings pierce him, particularly when he realizes his
unbearable pain will never
end!''\textsuperscript{\protect\hypertarget{17_Chapter_Ten__THE_FAILURE_OF_IMAG.xhtmlux5cux23id_712}{\protect\hyperlink{23_NOTES.xhtmlux5cux23id_713}{12}}}

The question may occur to us, how those who kept these images of hell
before them could have burnt people alive on earth. Denis presents the
heat of the fire, the gruesome cold, the disgusting worms, the stench,
hunger, thirst, the chains and the darkness and the unspeakable filth of
hell. The endless echo of wailing and screaming, the sight of the
devil---all this spreads like a suffocating nightmarish shroud over the
soul and senses of the readers. But even more cutting is the dread of
the cerebral pain, of repentance, fear, the empty feeling of infinite
deprivation and condemnation, of the unspeakable hatred of God and the
envy of the bliss enjoyed by his chosen. In the minds of the damned
there is nothing but confusion, compulsion, and a consciousness filled
with error and delusions, blindness and frenzy. The knowledge that all
this will remain so for all times and all eternity raises the awareness
to a height of dizzy
terror.\textsuperscript{\protect\hypertarget{17_Chapter_Ten__THE_FAILURE_OF_IMAG.xhtmlux5cux23id_710}{\protect\hyperlink{23_NOTES.xhtmlux5cux23id_711}{13}}}

There is no need to substantiate the fact that the fear of eternal pain,
either as a sudden ``divine dread'' or as a gnawing and prolonged
sickness, is frequently cited as motivation for a life of quiet
contemplation and
devotion.\textsuperscript{\protect\hypertarget{17_Chapter_Ten__THE_FAILURE_OF_IMAG.xhtmlux5cux23id_708}{\protect\hyperlink{23_NOTES.xhtmlux5cux23id_709}{14}}}
Everything was geared to this end. A tract about the four last things,
death, judgment, hell, and eternal life, perhaps borrowed from the one
authored by Denis, provided the customary reading at mealtime for the
guests of the Windesheim
convent.\textsuperscript{\protect\hypertarget{17_Chapter_Ten__THE_FAILURE_OF_IMAG.xhtmlux5cux23id_706}{\protect\hyperlink{23_NOTES.xhtmlux5cux23id_707}{15}}}
Truly a bitter seasoning! But such spicy means served to motivate people
to continuously seek ethical perfection. Medieval man resembles somebody
who has been treated for too long with strong medicines, now he will
react only to the most potent stimulants. In order to let the
praiseworthy quality of a particular virtue shine in its fullest glory,
nothing but the most extreme examples would do for the medieval mind. In
these examples, less extreme notions of ethics would already have been
sufficient to turn virtue into its own caricature. For patience we are
\protect\hypertarget{17_Chapter_Ten__THE_FAILURE_OF_IMAG.xhtmlux5cux23page_254}{}{}offered
the example of St. Giles who, wounded by an arrow, asked God that his
wound never heal as long as he lived; for temperance, the example of
those saints who mixed ashes with their food; for chastity, the model of
those who took a woman to bed with them to test their fortitude or that
of the pitiful fantasies of virgins who, in order to escape their
virtue's enemies, grew a beard or thick body hair. The attraction of the
example can be vested just as well in the extreme youth of the saint:
St. Nicholas refusing his mother's milk on high holy days. For
steadfastness, Gerson recommends St. Quiriçus---a martyr aged three
years or even only nine months---who refused to be consoled by the
prefect and was tossed into the
abyss.\textsuperscript{\protect\hypertarget{17_Chapter_Ten__THE_FAILURE_OF_IMAG.xhtmlux5cux23id_704}{\protect\hyperlink{23_NOTES.xhtmlux5cux23id_705}{16}}}

The need to experience the glory of virtue in such strong potions is
again linked to the all dominating idealism. To view virtue as an idea
pulled, so to speak, the ground of everyday life from under its
appreciation; its beauty was seen in the utmost perfection of its
independent existence and not in the daily round of failure and new
beginnings.

Medieval realism (that is, hyperidealism) has to be regarded as a
primitive mode of thought, all the impact of Christianized neo-Platonism
notwithstanding. We are dealing with the attitude of primitive man (but
for the medievals freely sublimated by philosophy), who assigns essence
and being to all abstract matters. While we may be justified in
regarding the hyperbolic veneration of virtue as a product of high
religion, in its counterpart---the contempt of the world---we clearly
recognize the link that ties medieval thought to the thought forms of a
distant past. I have in mind the fact that the tract \emph{De contemptu
mundi\protect\hypertarget{17_Chapter_Ten__THE_FAILURE_OF_IMAG.xhtmlux5cux23id_2329}{\protect\hyperlink{23_NOTES.xhtmlux5cux23id_2330}{*\textsuperscript{5}}}}
cannot avoid placing too much weight on the evil of everything material.
There is no greater motivation to despise the world than disgust over
bodily functions; that is, secretion and procreation. This is the most
pitiful part of medieval ethics: the disgust over man as ``formatus de
spurcissimo spermate, conceptus in puritu carnis sanguine menstruo
nutritus, qui fertur esse tarn detestabilis et immundus, ut ex ejus
contactu fruges non germinent, arescant arbusta .~.~. et si canes inde
comederint, in rabiem
efferantur.''\protect\hypertarget{17_Chapter_Ten__THE_FAILURE_OF_IMAG.xhtmlux5cux23id_2331}{\protect\hyperlink{23_NOTES.xhtmlux5cux23id_2332}{†\textsuperscript{6}}}
What else is this sensuality bent into its
oppo\protect\hypertarget{17_Chapter_Ten__THE_FAILURE_OF_IMAG.xhtmlux5cux23page_255}{}{}site
but a remnant of primitive realism in which savages fear magic
substances in excrements and in everything accompanying conception and
birth. There is a straight and rather short line that links the magic
fear, which prompted primitive peoples to turn away from women and the
most female of their female functions, to the ascetic hatred and cursing
of women that disfigures Christian literature since Tertullian and
Jerome.

Everything is thought of as having substance. This is nowhere expressed
as clearly as in the teaching of the \emph{thesaurus
ecclesiae}\textsuperscript{\protect\hypertarget{17_Chapter_Ten__THE_FAILURE_OF_IMAG.xhtmlux5cux23id_702}{\protect\hyperlink{23_NOTES.xhtmlux5cux23id_703}{17}}}
about the treasure of supererogation \emph{(operum superogationum)} of
Christ and all the
saints.\textsuperscript{\protect\hypertarget{17_Chapter_Ten__THE_FAILURE_OF_IMAG.xhtmlux5cux23id_700}{\protect\hyperlink{23_NOTES.xhtmlux5cux23id_701}{18}}}
Even though the concept of such a treasury and the notion that all the
faithful as members of the \emph{corpus mysticum Christi}---the
church---partake of this treasury is very old, the teaching that these
good works constitute an inexhaustible store that can be distributed by
the church and particularly by the pope appeared only during the
thirteenth century. Alexander of Hales is the first who uses the
\emph{thesaurus} in that technical sense of the word that it has
retained ever
since.\textsuperscript{\protect\hypertarget{17_Chapter_Ten__THE_FAILURE_OF_IMAG.xhtmlux5cux23id_698}{\protect\hyperlink{23_NOTES.xhtmlux5cux23id_699}{19}}}
The teaching met resistance until it finally found its complete
description and explanation in the bull \emph{Unigenitis} of Pope
Clement VI in 1343. The treasury is treated in this bull just like a
capital account that Christ entrusted to Peter and his disciples. It not
only increases day by day, but the more people are made to follow the
right path through its application, the larger the treasury of merits
becomes.\textsuperscript{\protect\hypertarget{17_Chapter_Ten__THE_FAILURE_OF_IMAG.xhtmlux5cux23id_696}{\protect\hyperlink{23_NOTES.xhtmlux5cux23id_697}{20}}}

If good works were so substantial, sins, perhaps even more intensely,
could be so regarded. Even though the church emphatically taught that
sin was neither an entity nor a
thing,\textsuperscript{\protect\hypertarget{17_Chapter_Ten__THE_FAILURE_OF_IMAG.xhtmlux5cux23id_694}{\protect\hyperlink{23_NOTES.xhtmlux5cux23id_695}{21}}}
it was inevitable that ignorant minds came to be convinced---given the
technique of forgiving sin on the part of the church, in conjunction
with the colorful presentation and the elaborate systematization of
sin---that sin was a substance (just as it is viewed in the
\emph{Artharva-Veda)}. The perception of sin as an infectious substance
could only be reinforced when Denis---even though he intended these
examples only as metaphors---calls sin a fever, a cold and corrupted
humor, and the
like.\textsuperscript{\protect\hypertarget{17_Chapter_Ten__THE_FAILURE_OF_IMAG.xhtmlux5cux23id_692}{\protect\hyperlink{23_NOTES.xhtmlux5cux23id_693}{22}}}
That the law, not so timidly concerned with dogmatic purity, reflects
such perception is shown by the fact that English jurists employed the
idea that a felony involved the corruption of the
\protect\hypertarget{17_Chapter_Ten__THE_FAILURE_OF_IMAG.xhtmlux5cux23page_256}{}{}blood.\textsuperscript{\protect\hypertarget{17_Chapter_Ten__THE_FAILURE_OF_IMAG.xhtmlux5cux23id_690}{\protect\hyperlink{23_NOTES.xhtmlux5cux23id_691}{23}}}
The blood of the savior, too, was subjected to the same hypersubstantial
view: it was a real substance; one drop would have been sufficient to
save the world, but we are given it in abundance, says St. Bernard, and
St. Thomas Aquinas waxes poetic:

\emph{Pie Pelicane, Jesu domine},

\emph{me immundum munda tuo sanguine}

\emph{cuius una stilla salvum facere}

\emph{Totum mundum quit ab omni
scelere}.\textsuperscript{\protect\hypertarget{17_Chapter_Ten__THE_FAILURE_OF_IMAG.xhtmlux5cux23id_688}{\protect\hyperlink{23_NOTES.xhtmlux5cux23id_689}{24}}}\protect\hypertarget{17_Chapter_Ten__THE_FAILURE_OF_IMAG.xhtmlux5cux23id_2591}{\protect\hyperlink{23_NOTES.xhtmlux5cux23id_2592}{*\textsuperscript{7}}}

In Denis the Carthusian we observe a desperate struggle to define the
conceptions of eternal life in spatial terms. The eternal light is of
immeasurable dignity; to enjoy God within oneself is infinite
perfection; the Redeemer was necessarily of infinite majesty and
effectiveness \emph{(efficacia)}; sin is of infinite enormity because it
is a transgression against infinite holiness; for this reason the act of
atonement requires a subject with infinite
ability.\textsuperscript{\protect\hypertarget{17_Chapter_Ten__THE_FAILURE_OF_IMAG.xhtmlux5cux23id_686}{\protect\hyperlink{23_NOTES.xhtmlux5cux23id_687}{25}}}
The negative space-adjective ``infinite'' here has in every instance the
function of making conceivable the importance, the potential of the
holy. In order to convey to his reader a sense of eternity, Denis
employs an image: imagine a sand hill as large as the universe. Every
hundred thousand years a grain of sand is removed from the hill. It will
be leveled, but even after such an incomprehensible length of time, the
punishment of hell will not have been shortened, nor will it be any
closer to its end than it was when the first grain of sand was removed.
Nonetheless, it would greatly console the damned if they knew that they
would be liberated as soon as the mountain
disappeared.\textsuperscript{\protect\hypertarget{17_Chapter_Ten__THE_FAILURE_OF_IMAG.xhtmlux5cux23id_684}{\protect\hyperlink{23_NOTES.xhtmlux5cux23id_685}{26}}}

If the attempt is made to express the joys of heaven or the majesty of
God in a similar manner, all that happens is that the idea itself is
presented in ever higher pitched clamor. The expression of the joys of
heaven remain extremely primitive. Human language is unable to evoke a
vision of bliss equally drastic as the one it conceives of terror. One
has only to delve deeply into the low dens of mankind to find raw
material for the description of ugliness and misery; but to describe the
highest bliss would require one to strain one's neck trying to look to
the heavens. Denis exhausts himself
\protect\hypertarget{17_Chapter_Ten__THE_FAILURE_OF_IMAG.xhtmlux5cux23page_257}{}{}in
desperate superlatives; that is, in a purely mathematical reinforcement
of the idea of bliss, without, however, rendering it clearer or more
profound. ``Trinitas super substantialis, superadoranda et superbona
.~.~. dirige nos ad superlucidam tui ipsius contemplationem.''Godis,
``supermisericordissimus, superdignissimus, superamabilissimus,
supersplendidissimus, superomnipotens et supersapiens,
supergloriosissimus.''\textsuperscript{\protect\hypertarget{17_Chapter_Ten__THE_FAILURE_OF_IMAG.xhtmlux5cux23id_682}{\protect\hyperlink{23_NOTES.xhtmlux5cux23id_683}{27}}}\protect\hypertarget{17_Chapter_Ten__THE_FAILURE_OF_IMAG.xhtmlux5cux23id_2593}{\protect\hyperlink{23_NOTES.xhtmlux5cux23id_2594}{*\textsuperscript{8}}}

But of what use is it to pile up superlatives, or qualitative visions of
height, width, immeasurability, and inexhaustability? These are mere
images, exercises in reducing the idea of infinity to images born of the
finite world. This leads inevitably to a weakening and externalization
of the concept of eternity. Eternity is not immeasurable time. Every
sensation, once expressed, loses its directness; every quality ascribed
to God takes something away from his majesty.

At this point begins the gigantic struggle to climb with the help of the
powers of the human mind to the absolute imagelessness of the deity: a
struggle that remains everywhere and everytime the same and is not tied
to any particular culture or era. ``There is about mystical utterances
an eternal unanimity which ought to make a critic stop and think, and
which brings it about that the mystical classics have, as has been said,
neither birthday nor native
land.''\textsuperscript{\protect\hypertarget{17_Chapter_Ten__THE_FAILURE_OF_IMAG.xhtmlux5cux23id_680}{\protect\hyperlink{23_NOTES.xhtmlux5cux23id_681}{28}}}---The
props of the imagination cannot be immediately dispensed with. One by
one the shortcomings of the means of expression become evident. The
concrete embodiments of the idea and the colorful garb of symbolism are
the first to go. Once this has happened, there is no longer any mention
of blood or atonement, of the Eucharist, of Father, Son, and Holy Ghost.
There is almost no mention of Christ in Eckhart's mysticism and just as
little of the church and the sacraments. But still the expressions for
the mystic vision of being, for truth, for the deity remain tied to
natural concepts, those of light and expansion. Later, these turn into
negatives, into silence, emptiness, darkness. Thereupon, the
shortcomings of these terms, devoid of form and content, is realized,
and the attempt is made to remove their deficiencies by continuously
cou\protect\hypertarget{17_Chapter_Ten__THE_FAILURE_OF_IMAG.xhtmlux5cux23page_258}{}{}pling
them with their opposites. Ultimately, there is nothing left but pure
negation: the deity that is recognized in the Nothingness of what
exists, because it stands above all, is called by the mystics, Nothing.
This is what Scotus
Eriugena\textsuperscript{\protect\hypertarget{17_Chapter_Ten__THE_FAILURE_OF_IMAG.xhtmlux5cux23id_678}{\protect\hyperlink{23_NOTES.xhtmlux5cux23id_679}{29}}}
does, and Angelus Silesius when he says:

God is a pure Nothing, unperturbed by Now and Here. The more you try to
grasp him, the more he is lost to
you.\textsuperscript{\protect\hypertarget{17_Chapter_Ten__THE_FAILURE_OF_IMAG.xhtmlux5cux23id_676}{\protect\hyperlink{23_NOTES.xhtmlux5cux23id_677}{30}}}

This progression of the viewing mind, by stages, to the abandonment of
all concepts did not, of course, take place in this strict sequence.
Most mystic statements show all the phases synchronically, mixed and
blended with one another. They already existed in India, were already
fully developed in Pseudo-Dionysius the Areopagite, who is the source of
all Christian mysticism, and are revived in the German mysticism of the
fourteenth century.

The following passage from the revelation of Denis the Carthusian may
serve as an
example.\textsuperscript{\protect\hypertarget{17_Chapter_Ten__THE_FAILURE_OF_IMAG.xhtmlux5cux23id_674}{\protect\hyperlink{23_NOTES.xhtmlux5cux23id_675}{31}}}
He is talking to God, who is angry: ``At this answer the friar, turning
inward, saw himself transposed into a sphere of infinite light and most
sweet. In a tremendous silence he called out with a secret voice that
did not sound outside of himself to the most secret and truly hidden,
incomprehensible God: `O Thou, super-loveable God, Thou Thyself art the
light and the sphere of light in which Thy chosen ones go sweetly to
their rest, to regain their strength, find peaceful slumber and true
sleep. Thou art like an ever-level, immeasurable desert in which the
truly pious spirit---entirely purified by special love, enlightened from
above and vibrantly inflamed---roams without becoming lost and becomes
lost without roaming, succumbs in bliss and recovers without having been
weakened.'\,'' In this passage there is first the image of light---still
positive---followed by that of sleep, then the desert, and finally by
the opposites that cancel one another.

The image of the desert, the horizontal notion of space, alternates with
the vertical notion of the abyss. The latter was a tremendous store of
mystic formulations. The expression of the absence of any particular
qualities of the deity, in Eckhart's words, ``the mannerless and
formless abyss of the silent, empty deity,'' unites the infinite
horizontal and vertical extensions to create a sensation of vertigo. It
is said of Pascal that he constantly envisioned an abyss at his side;
such a sensation is here reduced to a standard mystic
\protect\hypertarget{17_Chapter_Ten__THE_FAILURE_OF_IMAG.xhtmlux5cux23page_259}{}{}expression.
In these visions of the abyss and the silence, the most vibrant
descriptions of the indescribable mystic experience are reached. Susa
jubilantly exclaims, ``Wol uf dar, herz und sin und muot, in daz
grundlos abgründ aller lieplichen
dingen!''\textsuperscript{\protect\hypertarget{17_Chapter_Ten__THE_FAILURE_OF_IMAG.xhtmlux5cux23id_672}{\protect\hyperlink{23_NOTES.xhtmlux5cux23id_673}{32}}}\protect\hypertarget{17_Chapter_Ten__THE_FAILURE_OF_IMAG.xhtmlux5cux23id_2595}{\protect\hyperlink{23_NOTES.xhtmlux5cux23id_2596}{*\textsuperscript{9}}}
Master Eckhart, in his breathless fixation, says,

``Dirre funke (the mystic nucleus of the individual being) .~.~.
engnüeget an vater noch an sune noch an heiligem geiste noch an den drin
personen, als verre als ieclîchiu bestêt in ir eigenschaft. Ich spriche
wêrlîche, daz diseme selben liehte niht begnüeget an der einberkeit der
fruhtberlîchen art gotlîcher natûre. Ich wil noch mê sprechen, daz noch
wunderlîcher lûtet: ich spriche bî guoter wârheit, daz disem liehte niht
genueget an dem einveltigen stillestânden gotlîchen wesenne, daz weder
gît noch ennimet, mêr: ez wil wizzen, wannen diz wesen har kome, ez wil
in den einveltigen grunt, in die stillen wüeste, dâ nie unterscheit in
geluogete weder vater noch sun noch heiligeist; in dem innegen, da
nieman heime ist, da benüeget ez inme liehte, unt da ist ez einiger dan
in ime selber; want dirre grunt ist ein einvetic stille, diu in ir
selber unbegeglich ist.'' Only in this way will the soul come completely
to blessedness, ``daz sie sich wirfet in die wüesten gotheit, dâ noch
were noch bilde enist, daz si sich da verliese unde versenke in die
wüestenunge.''\textsuperscript{\protect\hypertarget{17_Chapter_Ten__THE_FAILURE_OF_IMAG.xhtmlux5cux23id_670}{\protect\hyperlink{23_NOTES.xhtmlux5cux23id_671}{33}}}\protect\hypertarget{17_Chapter_Ten__THE_FAILURE_OF_IMAG.xhtmlux5cux23id_2597}{\protect\hyperlink{23_NOTES.xhtmlux5cux23id_2598}{†\textsuperscript{10}}}

Tauler says, ``In this, the beatified and purified spirit plunges into
the divine darkness, into calm silence and an incomprehensible
\protect\hypertarget{17_Chapter_Ten__THE_FAILURE_OF_IMAG.xhtmlux5cux23page_260}{}{}and
inexpressible union. In this immersion is lost all notion of equal and
unequal; in this abyss the spirit loses itself and knows nothing of God
nor of itself nor of equal and unequal nor of utility .~.~. because it
is joined to God's unity and has lost all ability to
separate.''\textsuperscript{\protect\hypertarget{17_Chapter_Ten__THE_FAILURE_OF_IMAG.xhtmlux5cux23id_669}{\protect\hyperlink{23_NOTES.xhtmlux5cux23page_428}{34}}}

Ruusbroec uses all these means of expressing the mystic experience even
more realistically than the Germans:

\emph{Roept dan alle met openre herten}:

\emph{O gheweldich slont!}

\emph{Al sonder mont},

\emph{Voere ons in dinen afgront}:

\emph{Ende make ons dine minne
cont\protect\hypertarget{17_Chapter_Ten__THE_FAILURE_OF_IMAG.xhtmlux5cux23id_2599}{\protect\hyperlink{23_NOTES.xhtmlux5cux23id_2600}{*\textsuperscript{11}}}}

The enjoyment of the bliss of the union with God ``is wild and chaotic,
like losing oneself since there is neither guidance, nor way, nor path,
nor statutes, nor measure.'' ``We shall be de-elevated, de-immersed,
de-widened, de-lengthened (the cancelation of all notions of space),
losing ourselves in an eternal state, which knows no
return.''\textsuperscript{\protect\hypertarget{17_Chapter_Ten__THE_FAILURE_OF_IMAG.xhtmlux5cux23id_667}{\protect\hyperlink{23_NOTES.xhtmlux5cux23id_668}{35}}}
The enjoyment of bliss is such ``that God and all saints and elevated
men who experience it are devoured into an undefinable state that is one
of not-knowing and of eternal
immersion.''\textsuperscript{\protect\hypertarget{17_Chapter_Ten__THE_FAILURE_OF_IMAG.xhtmlux5cux23id_665}{\protect\hyperlink{23_NOTES.xhtmlux5cux23id_666}{36}}}
God gives the fullness of bliss to all alike, ``but those who receive it
are not alike, and yet there is something for everyone.'' That is, in
the union with God, they cannot hold themselves against the wealth of
bliss offered to them. ``But after being lost in the darkness of the
desert, there is nothing left. There is neither giving nor receiving,
but simple pure being. In it, God and all those united with him are
immersed and lost and they shall never more find Him in this formless
mode of
being.''\textsuperscript{\protect\hypertarget{17_Chapter_Ten__THE_FAILURE_OF_IMAG.xhtmlux5cux23id_663}{\protect\hyperlink{23_NOTES.xhtmlux5cux23id_664}{37}}}

All negations have been united in the following passage:

Thereupon follows the seventh step (of love), the noblest, the highest
that can be experienced in time and eternity. It comes at the moment
when we find within ourselves a groundless not-knowing that is beyond
all confessing and knowing; when we die and lose ourselves in an eternal
\protect\hypertarget{17_Chapter_Ten__THE_FAILURE_OF_IMAG.xhtmlux5cux23page_261}{}{}namelessness
beyond all those names that we bestow on God or on creatures; when we
see, then find within ourselves, an eternal state of not-doing beyond
any desire for practicing virtues, and where no one is able to assert
his individual desire, and when, beyond all blessed spirits, we find a
bottomless bliss where we are all one and are the same one that is bliss
itself in its own selfhood, and when we see all the blessed spirits,
their being immersed, departed and lost, into their supra-existence,
into an unknown formless
darkness.\textsuperscript{\protect\hypertarget{17_Chapter_Ten__THE_FAILURE_OF_IMAG.xhtmlux5cux23id_661}{\protect\hyperlink{23_NOTES.xhtmlux5cux23id_662}{38}}}

In this simple, artless blissfulness all difference of creatures is
dissolved. ``They take leave of themselves. losing themselves in a
bottomless state of groundless ignorance; there all clarity is returned
to darkness and the three personages give way to the essential
unity.''\textsuperscript{\protect\hypertarget{17_Chapter_Ten__THE_FAILURE_OF_IMAG.xhtmlux5cux23id_659}{\protect\hyperlink{23_NOTES.xhtmlux5cux23id_660}{39}}}

Always there is the futile effort to do away with all images, to express
``our empty state which is mere formlessness''---which only God can
grant. ``He rids us of all images and pulls us back to our origin. There
we find nothing but a wild, void, unformed emptiness forever
corresponding with
eternity.''\textsuperscript{\protect\hypertarget{17_Chapter_Ten__THE_FAILURE_OF_IMAG.xhtmlux5cux23id_657}{\protect\hyperlink{23_NOTES.xhtmlux5cux23id_658}{40}}}

In these quotations from Ruusbroec the two last mentioned elements of
description are exhausted: light transforming itself into darkness and
pure negation, the abandonment of all positive knowledge. The practice
of calling the innermost secret essence of God his darkness originates
with the Pseudo-Areopagite. His namesake, admirer, and commentator,
Denis the Carthusian, elaborates on this expression. ``And the most
unexcelled, immeasurable, invisible fullness of Your eternal light is
called darkness in which, as it is said, You dwell, who makes darkness
his refuge. And the divine darkness itself is veiled from all light and
hidden from all sight because of the indescribable, impenetrable
splendor of its own
clarity.''\textsuperscript{\protect\hypertarget{17_Chapter_Ten__THE_FAILURE_OF_IMAG.xhtmlux5cux23id_655}{\protect\hyperlink{23_NOTES.xhtmlux5cux23id_656}{41}}}
Darkness is not knowing, the ending of every concept. ``The more the
spirit approaches Your super-shining divine light, the more Your
unapproachability and incomprehensibility become apparent and as soon as
the spirit has entered into this darkness all names and all thought soon
succumb entirely {[}\emph{omne mox nomen omnisque cognitioprorsus
deficient}{]}. But this will be granted the spirit: to see You, to see
that you are entirely invisible. The clearer the spirit sees this, the
clearer it will see you. We ask to
\protect\hypertarget{17_Chapter_Ten__THE_FAILURE_OF_IMAG.xhtmlux5cux23page_262}{}{}become
like this super-light darkness, O Blessed Trinity, and to see You by
virtue of invisibility and not-knowing and to recognize that You are
above all seeing and knowing. You appear to those above who have
overcome and left behind all that which can be perceived and
comprehended and everything that is created, including themselves. They
enter into the darkness in which you truly
are.''\textsuperscript{\protect\hypertarget{17_Chapter_Ten__THE_FAILURE_OF_IMAG.xhtmlux5cux23id_653}{\protect\hyperlink{23_NOTES.xhtmlux5cux23id_654}{42}}}

Just as light becomes darkness, the highest life changes itself into
death. Once the soul has understood, says Master Eckhart, that no
creature can enter God's kingdom, then the soul will go its own way and
no longer seek God. ``Und allhie so stirbit si iren eigenen hohsten tot.
In disem tot verleuset di sele alle begerung und alle bild und alle
vestentnüzz und alle form und wirt beraubt aller wesen. Und dez seit
sicher als got lebt: als wenik mag di sele, di also geistlich tot ist,
einik weis oder einik bild vorgetragen einigen menschen. Wann diser
geist ist tot und is begraben in der gotheit.''
\protect\hypertarget{17_Chapter_Ten__THE_FAILURE_OF_IMAG.xhtmlux5cux23id_2601}{\protect\hyperlink{23_NOTES.xhtmlux5cux23id_2602}{*\textsuperscript{12}}}
Soul, if you cannot drown yourself in this bottomless sea of the Deity,
you cannot confess this divine
death.\textsuperscript{\protect\hypertarget{17_Chapter_Ten__THE_FAILURE_OF_IMAG.xhtmlux5cux23id_651}{\protect\hyperlink{23_NOTES.xhtmlux5cux23id_652}{43}}}

Denis says elsewhere that viewing God through negations is more perfect
than doing so through affirmations. ``Because if I say, God is kindness,
essence \emph{{[}essentia{]})}, life, I seem to hint at what God is as
if that which He is had anything in common with creation or resembled it
to any degree. It is certain that He is incomprehensible and unknown,
unfathomable and inexpressible, and is separated from everything he
creates by an immeasurable and totally incomparable difference and
uniqueness.''\textsuperscript{\protect\hypertarget{17_Chapter_Ten__THE_FAILURE_OF_IMAG.xhtmlux5cux23id_649}{\protect\hyperlink{23_NOTES.xhtmlux5cux23id_650}{44}}}
He calls the unifying wisdom \emph{(sapientia unitiva)} unreasonable,
meaningless, and
foolish.\textsuperscript{\protect\hypertarget{17_Chapter_Ten__THE_FAILURE_OF_IMAG.xhtmlux5cux23id_647}{\protect\hyperlink{23_NOTES.xhtmlux5cux23id_648}{45}}}

How similarly and yet how differently these sounds echo from ancient
India.

The disciple came to the master and said, ``Teach me the Brahma, O
Honored One! But the master remained silent. When the other repeated for
the second and third time, ``Teach me the Brahma, O Honored One!,'' the
master said,
\protect\hypertarget{17_Chapter_Ten__THE_FAILURE_OF_IMAG.xhtmlux5cux23page_263}{}{}``I'll
teach you, but you won't understand it. This âtman {[}the self{]} is
quiet.''\textsuperscript{\protect\hypertarget{17_Chapter_Ten__THE_FAILURE_OF_IMAG.xhtmlux5cux23id_645}{\protect\hyperlink{23_NOTES.xhtmlux5cux23id_646}{46}}}

The Gods wanted to know âtman from Prajâpati. They lived with him for
thirty-two years as Brahma students. Then he taught them that the little
man one sees in another's eye or the reflection in the water is the
self. Then looking after them as they departed, he said to himself,
There they go without having comprehended the self. After another
thirty-two years he revealed to Indra, in response to his objections,
that he who walks in a dream, he is âtman. And after once more the same
interval, ``That which, when man has fallen asleep, is immersed, has
come entirely to rest, is no longer seen in any dream,---that is the
self.'' ``But he, the âtman is neither this nor
that.''\textsuperscript{\protect\hypertarget{17_Chapter_Ten__THE_FAILURE_OF_IMAG.xhtmlux5cux23id_643}{\protect\hyperlink{23_NOTES.xhtmlux5cux23id_644}{47}}}

The entire sequence of opposing negations is now exhausted to explain
the self's nature.

Like someone who, embraced by a beloved woman, is not conscious of what
is external or internal, thus the spirit, embraced by the self, which is
cognition, is not conscious of what is external or what is internal.
That is the form of its being: craving satisfied, he himself is his
craving, being without craving, and divorced from suffering. Then father
is not-father, mother is not-mother, world is not-world .~.~.
\textsuperscript{\protect\hypertarget{17_Chapter_Ten__THE_FAILURE_OF_IMAG.xhtmlux5cux23id_641}{\protect\hyperlink{23_NOTES.xhtmlux5cux23id_642}{48}}}

Had the power of images been overcome? Not a single thought can be
expressed without image or metaphor. When we speak of the
incomprehensible essence of things, every word is image. The mind is not
satisfied with speaking only in negation of that which is highest and
most fervently desired, and the poet has to come to the rescue whenever
the wise man with his definitions and terms reaches an impasse. From the
snowy peaks of his formal visions, the sweet lyrical mind of Suso always
found the way back to the flowery fantasies of the older mysticism of
St. Bernard. In the midst of the ecstasy of the highest contemplation,
all the color and form of allegory return. Suso sees his betrothed,
Eternal Wisdom: ``si swepte hoh ob ime in einem gewültigen throne
(Heaven): sie luhte als der morgensterne, und schein als diu splindiu
sunne; ire krone waz ewikeit, ihr wat seliket, ihr wort süzzkeit, ihr
umbfang alles lustes gnuhsamkeit: si waz verr und nahe, hoh und neider;
si\protect\hypertarget{17_Chapter_Ten__THE_FAILURE_OF_IMAG.xhtmlux5cux23page_264}{}{}waz
gegenwürtig und doch verborgen; si liess mit ir umbgan, und moht si noch
nieman
begriffen.''\textsuperscript{\protect\hypertarget{17_Chapter_Ten__THE_FAILURE_OF_IMAG.xhtmlux5cux23id_639}{\protect\hyperlink{23_NOTES.xhtmlux5cux23id_640}{49}}}\protect\hypertarget{17_Chapter_Ten__THE_FAILURE_OF_IMAG.xhtmlux5cux23id_2603}{\protect\hyperlink{23_NOTES.xhtmlux5cux23id_2604}{*\textsuperscript{13}}}

There were still other ways back from the lonely heights of an
individual, solitary, formless, and imageless mysticism. Those heights
could be reached only by exhausting the mystery of the sacraments and
liturgy; only if one had completely exhausted the symbolic aesthetic
miracle of the dogmas and sacraments could one shake off the forms of
images and ascend to the nonconceptual vision of the mystics. But the
mind was incapable of enjoying its clarity at any time and as often as
it desired; clarity was restricted to moments of unusual grace and short
duration. Moreover, the church was always waiting below with its wise
and economic system of mysteries. In its liturgy the church had
concentrated the contact of the mind with the divine into the experience
of definite moments and had imposed on the mystery form and color. That
is why ritual always survived unbridled mysticism: it saved energy. With
equanimity the church tolerated aesthetic mysticism's wild flowers of
the imagination, but it feared true, radical mysticism, which put to the
flames everything on which the church was built: its harmonious
symbolism, its dogmas and sacraments.

``Unitive wisdom is unreasonable, meaningless and foolish.'' The path of
the mystic leads to the infinite and to unconsciousness. By denying any
connection between the Deity and anything created, transcendence is
destroyed. The bridge back to life is burnt. ``Alle creatûre sint ein
lûter niht. Ich sprich niht, daz sie kleine sin oder iht sin, sie sind
ein lûter niht. Swaz niht wesens hat daz ist niht. Alle creatûre hânt
kein wesen, wan ir wesen swebet an der gegenwertigkeit
gotes.''\textsuperscript{\protect\hypertarget{17_Chapter_Ten__THE_FAILURE_OF_IMAG.xhtmlux5cux23id_637}{\protect\hyperlink{23_NOTES.xhtmlux5cux23id_638}{50}}}\protect\hypertarget{17_Chapter_Ten__THE_FAILURE_OF_IMAG.xhtmlux5cux23id_2605}{\protect\hyperlink{23_NOTES.xhtmlux5cux23id_2606}{†\textsuperscript{14}}}

Intensive mysticism means a return to a pre-intellectual spiritual life.
All intellectualism is lost in it, is overcome and rendered superfluous.
But if, all this notwithstanding, mysticism has borne rich fruit for
culture, this is the result of the fact that mysticism always
\protect\hypertarget{17_Chapter_Ten__THE_FAILURE_OF_IMAG.xhtmlux5cux23page_265}{}{}proceeds
through preparatory stages and only gradually discards the forms of
custom and culture. Its fruits for civilization are born in its first
stages below the upper limit of vegetation. This is where the orchard of
ethical perfection blossoms as the required preparation for anyone who
wishes to achieve the vision: peace and gentility of mind, the
suppression of desire, the virtues of simplicity, moderation,
industriousness, seriousness, and fervor. This was the case in India and
the same thing is true here: the initial impact of mysticism is moral
and practical, consisting above all in the practice of practical
charity. All the great mystics have lavishly praised practicality.
Master Eckhart himself ranked Martha above
Mary\textsuperscript{\protect\hypertarget{17_Chapter_Ten__THE_FAILURE_OF_IMAG.xhtmlux5cux23id_635}{\protect\hyperlink{23_NOTES.xhtmlux5cux23id_636}{51}}}
and said that one should even abandon the ecstasy of Paul if one could
help a pauper with a bowl of soup. The history of mysticism, beginning
with Eckhart and continuing with his disciple Tauler, points more and
more in the direction of dignifying the practical element. Ruusbroec,
too, praises quiet unassuming work, and Denis the Carthusian represents
in person the union of a practical sense of daily religious life and the
most intense individual mysticism. In the Netherlands began that
movement in which these concomitant elements of mysticism---moralism,
pietism, charity, industriousness---became the main focus. This meant
that from the intense mysticism of the remote moments of a few flow the
extensive mysticism of the everyday life of the many, the ongoing
communal fervor of modern devotees, in place of lonely and rare ecstasy.
The sober mysticism, one is tempted to say.

In the Fraterhouses and the monasteries of the Windesheim Congregation,
we find, constantly poured over quiet daily work, the radiance of
religious fervor that was constantly present in the mind of the
congregation. The flexible lyrical and the unrestrained striving
elements have both been abandoned and, together with them, have
evaporated the danger of faith gone wrong. The brothers and sisters are
perfectly orthodox and conservative. It was mysticism \emph{en detail}:
one had not been struck by lightning, one had only received a little
spark, and experienced in the small, quiet, unassuming circle the
transport of ecstasy in the form of intimate spiritual communion, the
exchange of letters and self-contemplation. Emotional and spiritual life
was cultivated like a greenhouse plant; there was much narrow
puritanism, much moral exercise, a stifling of laughter and of basic
human drives, and much pietist simplemindedness.

But the most powerful and beautiful work of that period, the
\emph{\protect\hypertarget{17_Chapter_Ten__THE_FAILURE_OF_IMAG.xhtmlux5cux23page_266}{}{}Imitatio
Christi}, arose in those circles. Here we meet the man, no theologian,
no humanist, no philosopher, no poet, and actually also no mystic, who
wrote the book destined to become for centuries a source of solace.
Thomas à Kempis, quiet, introverted, full of tenderness for the miracle
of the mass and with a most narrow perception of divine guidance, knew
nothing about the outrage over church administration or secular life,
such as inspired the preachers, or of the multifaceted ambitions of a
Gerson, Denis, or Nicholas of Cusa, or of the wild fantasies of a John
Brugman or of the colorful symbolism of an Alain de la Roche. He looked
only for the element of quietude in all things and found it ``in angello
cum li bello'': ``O quam salubre quam iucundum et suave est sedere in
solitudine et tacere ei loqui cum Deo!'' (``O how wholesome, how
pleasant and sweet it is to sit in solitude and to be silent and speak
with
God!'').\textsuperscript{\protect\hypertarget{17_Chapter_Ten__THE_FAILURE_OF_IMAG.xhtmlux5cux23id_633}{\protect\hyperlink{23_NOTES.xhtmlux5cux23id_634}{52}}}
And his book, of simple wisdom for living and dying, addressed to
resigned minds, became a book for all the ages. In this book all
neo-Platonic mysticism has been abandoned. Its only basis is the voice
of the beloved Bernard of Clairvaux. It does not present any development
of philosophical thought, it only contains a number of the most simple
ideas grouped in the form of sayings around a central point. Every one
of them is couched in short, straightforward sentences; there is no
subordination and hardly any correlation of ideas. There is none of the
lyric resonance of a Henry Suso or the tense sparkle of a Ruusbroec.
Ringing with the sound of parallel sentences and weak assonances, the
\emph{Imitatio} would appear to be prose, if it were not for that
monotonous rhythm that makes it resemble the ocean on a soft rainy
evening or the autumnal sigh of the wind. There is something miraculous
about the effect of the \emph{Imitatio}. The thinker does not captivate
us with his power or élan, as for example, Augustine, or by flowering
prose, as St. Bernard, nor with the depth or fullness of his thought.
Everything is even and melancholy, everything is kept in a minor key.
There is only peace, calm, a quiet, resigned expectation, and solace.
``Taedet me vitae temporalis'' (``Earthly life is a burden to me''),
Thomas says somewhere
else.\textsuperscript{\protect\hypertarget{17_Chapter_Ten__THE_FAILURE_OF_IMAG.xhtmlux5cux23id_631}{\protect\hyperlink{23_NOTES.xhtmlux5cux23id_632}{53}}}
And yet, the words of this man, removed from the world, are able to
strengthen us for life in this world as are those of no other. There is
something this book for the tired of all ages shares with the
expressions of intense mysticism: here too, the power of images is
overcome as far as possible and the colorful garb of glittering symbols
is discarded. For this
\protect\hypertarget{17_Chapter_Ten__THE_FAILURE_OF_IMAG.xhtmlux5cux23page_267}{}{}very
reason, the \emph{Imitatio} is not limited to one cultural epoch; like
ecstatic contemplations of the All-One, it departs from all culture and
belongs to no culture in particular. This explains its two thousand
editions as well as the different suppositions concerning its author and
its time of composition that fall into a range of three hundred years.
Thomas did not speak that ``Ama nesciri'' (``You must love to remain
alone'') in vain.
