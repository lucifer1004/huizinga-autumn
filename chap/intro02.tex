\chapter{PREFACE TO THE FIRST AND SECOND DUTCH EDITIONS}

IN MOST INSTANCES IT IS THE ORIGIN OF THE NEW that attracts the
attention of the mind to the past. We want to know how the new ideas and
the forms of life that shine in their fullness during later times came
to be. We view past ages primarily in terms of the promise they hold for
those that follow. How eagerly the Middle Ages have been scrutinized for
evidence of the first sprouts of modern culture, so eagerly that it
sometimes must appear as if the intellectual history of the Middle Ages
was nothing but the advent of the Renaissance. Did we not see everywhere
in this age, which was once regarded as rigid and dead, new growths that
all seemed to point to future perfection? Yet in our search for newly
arising life it is easily forgotten that in history, as in nature, the
processes of death and birth are eternally in step with one another. Old
forms of thought die out while, at the same time and on the same soil, a
new crop begins to bloom.

This book is an attempt to view the time around the fourteenth and
fifteenth centuries, not as announcing the Renaissance, but as the end
of the Middle Ages, as the age of medieval thought in its last phase of
life, as a tree with overripe fruits, fully unfolded and developed. The
luxuriant growth of old compelling forms over the living core of
thought, the drying and rigidifying of a previously valid store of
thought: this is the main content of the following pages. In writing
this text, my eye was trained on the depth of the evening sky, a sky
steeped blood red, desolate with threatening leaden clouds, full of the
false glow of copper. Looking back at what I have written, the question
arises whether, if my eye had dwelt still longer on the evening sky, the
turbid colors may yet have dissolved into utter clarity. It also seems
quite possible that the image, now that I have given it contours and
colors, may yet have become more gloomy and less serene than I had
perceived it
\protect\hypertarget{06_PREFACE_TO_THE_DUTCH_EDITION.xhtmlux5cux23page_xx}{}{}when
I started my labors. It can easily happen to one who has his vision
trained downward that what he perceives becomes too decrepit and wilted,
that too much of the shadow of death has been allowed to fall upon his
work.

The point of departure for this work was the attempt to better
understand the work of the van Eycks and that of their successors and to
understand it within the context of the entire life of that age. The
Burgundian community was the frame of reference that I had in mind: it
seemed possible to view this community as a civilization in its own
right, just like the Italian community of the fourteenth century; the
title of the work was first set as \emph{The Century of Burgundy}. But
as the scope of this civilization was viewed in a wider perspective,
certain limitations had to be abandoned. Just to retain the notion of a
postulated unity of Burgundian culture meant that non-Burgundian France
had to be given at least as much attention. Thus the place of Burgundy
was taken by the dual entities of France and the Netherlands and that in
a very different way. While in viewing the dying medieval culture the
Dutch element lags behind the French, there are areas where that element
has its own significance: in the life of piety and that of art. These
are given the opportunity to speak in greater detail.

There is no need to defend the crossing of the fixed geographic
boundaries in the tenth chapter so as to call on, next to Ruusbroec and
Denis the Carthusian, on Eckhardt, Suso, and Tauler as witnesses. How
little my story is justified by the writings I have studied from the
fourteenth and fifteenth centuries compared to all those I wanted to
read. How much I would have liked to place, next to the evolution of the
main types of the different intellectual traditions on which some of the
notions of these figures are often based, yet still others. But if I
relied among the historiographers on Froissart and Chastellain more than
on others, among the poets on Eustache Deschamps, among the theologians
on Jean de Gerson and Denis the Carthusian, among the painters on Jan
van Eyck---so is this not only the result of the limitation of my
material, but even more so the result of the richness of their works and
the singularly keen way in which their expressions are the preeminent
mirror of the spirit of their age.

It is the forms of life and thought that are used as evidence here. To
capture the essential content that rests in the form: is this not the
proper task of historical study?
