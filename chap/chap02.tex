\chapter{THE CRAVING FOR A MORE BEAUTIFUL LIFE}

EVERY AGE YEARNS FOR A MORE BEAUTIFUL WORLD. The deeper the desperation
and the depression about the confusing present, the more intense that
yearning. Towards the end of the Middle Ages the ground tone underlying
life is one of bitter despondency. The note of an assertive joy of life
and of a strong confidence in an individual's powers, which permeates
the history of the Renaissance and that of the age of Enlightenment, is
barely audible in the French-Burgundian world of the fifteenth century.
Was life really more unhappy then than usual? It may, at times, seem to
be the case. Wherever one looks in the sources of that period, in the
chronicles, in poetry, in sermons and religious tracts and even official
documents---with few exceptions, only the traces of strife, hatred and
malevolence, greed and poverty seem to have survived. One may well ask,
was this age incapable of enjoying nothing but cruelty, arrogant pride,
and intemperance? Is joyfulness and quiet happiness nowhere to be found?
To be sure, the age left in its records more traces of its suffering
than of its happiness. Its misfortunes became its history. But an
instinctive conviction tells us that the sum total of happiness, serene
joy, and sweet rest given to man cannot differ very much in one period
from that in another. The splendor of late medieval happiness has still
not completely vanished; it survives in folk song, in music, in the
quiet horizons of landscape paintings and in the sober faces seen in the
portraits.

But in the fifteenth century, it is tempting to say, it was not yet
customary, it was not in good taste, to loudly praise life and the
world. Those given to the serious contemplation of the course of daily
events, and who subsequently pronounced judgment on life, were
accustomed to dwell on only suffering and despair. They saw time coming
to an end and everything earthly inclining to ruin. The optimism that
was to rise beginning with the Renaissance, and
\protect\hypertarget{09_Chapter_Two__THE_CRAVING_FOR_A_M.xhtmlux5cux23page_31}{}{}to
fully bloom during the eighteenth century, was still unknown to the
French mind of the fifteenth century. Which group is it who are the
first to speak, full of hope and satisfaction, about their own times?
Not poets, much less religious thinkers; not even statesmen, but rather
scholars, the humanists. It is the exultation over rediscovered antique
wisdom that first elicits jubilation about the present; this is an
intellectual triumph. Ulrich von Hutten's well-known dictum ``O
saeculum, O literae! Juvat Vivere!'' (``O century, O literature! It is a
joy to live!'') is usually taken in much too wide a sense. It is the
enthusiastic man of letters rather than the whole man who is jubilating
here. One could easily cite, from the beginning of the sixteenth
century, a number of familiar shouts of joy about the splendor of the
times, but the facts would make one notice that they are almost
exclusively directed towards the regained intellectual world and are by
no means dithyrambic expressions of the joy of life in all its fullness.
Even the mood of the humanists is tempered by the old, pious turning
from the world. Better than from Hutten's too often cited dictum, this
can be ascertained from the letters Erasmus wrote around 1517---but no
longer from those written only a little later, because the optimism that
had prompted this joyous mood soon leaves him.

Erasmus writes, early in 1517, to Wolfgang Fabricius
Capito:\textsuperscript{\protect\hypertarget{09_Chapter_Two__THE_CRAVING_FOR_A_M.xhtmlux5cux23id_2084}{\protect\hyperlink{23_NOTES.xhtmlux5cux23id_2085}{1}}}
``I am truly no longer so keen on life, perhaps because I have already
lived almost too long as far as I am concerned---I have already begun my
51st year---perhaps because I see in this life nothing so glorious or
pleasant as to be worthy of pursuit for someone whom the Christian faith
has taught to truly believe that for those who devote their strength to
piety a much happier life awaits. Yet now, I could almost fancy becoming
young again for a short while if only because I can almost sense that a
golden age is about to arise in the near future.'' He then describes how
all the princes of Europe are in agreement and lean towards peace (so
dear to him), and continues, ``I cannot but hold the firm expectation
that there will be in part a new revival and in part a new unfolding not
only of law-abiding customs and Christian piety but also of a cleansed
and
genuine\textsuperscript{\protect\hypertarget{09_Chapter_Two__THE_CRAVING_FOR_A_M.xhtmlux5cux23id_2082}{\protect\hyperlink{23_NOTES.xhtmlux5cux23id_2083}{2}}}
literature and a very beautiful science.'' Through protection by the
princes, needless to say. ``We owe it to their pious minds that we
witness the awakening and arising of glorious minds---as if in response
to a given signal---all pledging to each other the restoration of good
literature {[}\emph{ad restituendas optimas litems}{]}.''

\protect\hypertarget{09_Chapter_Two__THE_CRAVING_FOR_A_M.xhtmlux5cux23page_32}{}{}Here
we have a pure expression of what the sixteenth century knew of
optimism. The basic sentiment of the Renaissance and humanism is
actually something entirely different from the unrestrained lust for
life that is usually held to be its basic tone. The affirmation of life
on the part of Erasmus is shy and a little stiff and, above all, very
intellectual. Nevertheless, it is a voice that could not yet be heard,
during the fifteenth century, outside of Italy. Intellectuals in France
and in the Burgundian provinces around 1400 still prefer to pile their
scorn of life and the times on rather thickly and, in a peculiar way
(but not without parallel; consider Byronism), the closer they are to
secular life, the darker their mood. Those who express that deep
melancholy, so characteristic of that time, most vigorously, are not
primarily those who have permanently retired from the world into
monasteries or scholarship. Mostly they are the chroniclers and the
fashionable court poets, given their lack of higher culture and their
inability to gain from the joys they perceive any expectations of a turn
for the better, who never tire of lamenting the debilities of an aged
world and despairing of peace and justice. No one has repeated the
lamentation, that all good things have left the world, more interminably
than Eustache Deschamps.

\emph{Temps de doleur et de temptacion},

\emph{Aages de plour, d'envie et de tourment},

\emph{Temps de langour et de dampnacion},

\emph{Aages meneur près du definement},

\emph{Temps plains d'orreur qui tout fait faussement},

\emph{Aages menteur, plain d'orgueil et d'envie},

\emph{Temps sanz honeur et sanz vray jugement},

\emph{Aage en tristour qui abrege la
vie}.\textsuperscript{\protect\hypertarget{09_Chapter_Two__THE_CRAVING_FOR_A_M.xhtmlux5cux23id_2080}{\protect\hyperlink{23_NOTES.xhtmlux5cux23id_2081}{3}}}\emph{\protect\hypertarget{09_Chapter_Two__THE_CRAVING_FOR_A_M.xhtmlux5cux23id_2411}{\protect\hyperlink{23_NOTES.xhtmlux5cux23id_2412}{*\textsuperscript{1}}}}

Dozens of his ballads were composed in this spirit---monotonous, weak
variations of the same dull theme. A pronounced melancholy must have
dominated the higher estates; why else would
\protect\hypertarget{09_Chapter_Two__THE_CRAVING_FOR_A_M.xhtmlux5cux23page_33}{}{}the
nobility have allowed its favorite poet to repeat these sentiments with
such frequency?

\emph{Toute léesse deffaut},

\emph{Tous cueurs ont prins par assaut}

\emph{Tristesse et
merencolie}.\textsuperscript{\protect\hypertarget{09_Chapter_Two__THE_CRAVING_FOR_A_M.xhtmlux5cux23id_2078}{\protect\hyperlink{23_NOTES.xhtmlux5cux23id_2079}{4}}}\protect\hypertarget{09_Chapter_Two__THE_CRAVING_FOR_A_M.xhtmlux5cux23id_2413}{\protect\hyperlink{23_NOTES.xhtmlux5cux23id_2414}{*\textsuperscript{2}}}

Three quarters of a century after Deschamps, Jean Merchinot still sings
in the same key:

\emph{O miserable et très dolente vie!} .~.~.

\emph{La guerre avons, mortalité, famine};

\emph{Le froid, le chaud, le jour, le nuit nous mine};

\emph{Puces, cirons et tant d'autre vermine}

\emph{Nous guerroyent. Bref, miserere domine}

\emph{Noz meschans corps, dont le vivre est très
court}.\protect\hypertarget{09_Chapter_Two__THE_CRAVING_FOR_A_M.xhtmlux5cux23id_2415}{\protect\hyperlink{23_NOTES.xhtmlux5cux23id_2416}{†\textsuperscript{3}}}

He, too, endlessly repeats the bitter conviction that everything in the
world is going badly; justice is mislaid, the powerful plunder the weak,
and the weak, in turn, plunder each other. According to his own
confession, his hypochondria even takes him to the brink of suicide. He
describes himself:

\emph{Et je, le pouvre escrivain},

\emph{Au cueur triste, faible et vain},

\emph{Voyant de chascun le dueil},

\emph{Soucy me tient en sa main};

\emph{Toujours les larmes à l'oeil},

\emph{Rien fors mourir je ne
vueil}.\textsuperscript{\protect\hypertarget{09_Chapter_Two__THE_CRAVING_FOR_A_M.xhtmlux5cux23id_2076}{\protect\hyperlink{23_NOTES.xhtmlux5cux23id_2077}{5}}}\protect\hypertarget{09_Chapter_Two__THE_CRAVING_FOR_A_M.xhtmlux5cux23id_2417}{\protect\hyperlink{23_NOTES.xhtmlux5cux23id_2418}{‡\textsuperscript{4}}}

All the examples of the nobility's mood of life testify to a
senti\protect\hypertarget{09_Chapter_Two__THE_CRAVING_FOR_A_M.xhtmlux5cux23page_34}{}{}mental
need for a dark costume for the soul. Nearly everyone declares that he
had seen nothing but misery, that one had to be prepared for something
worse and that he would not want to repeat the life he had lived so far.
``Moi douloreux homme, né en eclipse de ténèbres en espesses bruynes de
lamentation,''\protect\hypertarget{09_Chapter_Two__THE_CRAVING_FOR_A_M.xhtmlux5cux23id_2419}{\protect\hyperlink{23_NOTES.xhtmlux5cux23id_2420}{*\textsuperscript{5}}}
so Chastellain announces
himself.\textsuperscript{\protect\hypertarget{09_Chapter_Two__THE_CRAVING_FOR_A_M.xhtmlux5cux23id_2074}{\protect\hyperlink{23_NOTES.xhtmlux5cux23id_2075}{6}}}
``Tant a souffert La
Marchent''\protect\hypertarget{09_Chapter_Two__THE_CRAVING_FOR_A_M.xhtmlux5cux23id_2421}{\protect\hyperlink{23_NOTES.xhtmlux5cux23id_2422}{†\textsuperscript{6}}}
the court poet and chronicler of Charles the Bold selects as his motto;
life has a bitter taste for him and his portrait shows us those morose
features that attract our attention in so many pictures of that
period.\textsuperscript{\protect\hypertarget{09_Chapter_Two__THE_CRAVING_FOR_A_M.xhtmlux5cux23id_2072}{\protect\hyperlink{23_NOTES.xhtmlux5cux23id_2073}{7}}}

Is there a life---equally full of earthly arrogant pride and boastful
pleasure-seeking and at the same time crowned by so much success---as
that of Philip the Good? But even it reveals the despair of the time
lurking below its facade. When informed of the death of his one-year-old
infant boy he says, ``If only God deigned to let me die so young, I
would have considered myself
fortunate.''\textsuperscript{\protect\hypertarget{09_Chapter_Two__THE_CRAVING_FOR_A_M.xhtmlux5cux23id_2070}{\protect\hyperlink{23_NOTES.xhtmlux5cux23id_2071}{8}}}

Isn't it strange that during this time, in the word ``melancholy,'' the
meanings of depression, serious contemplation, and imagination come
together? This shows how any serious endeavor of the mind would, of
necessity, take it into somber moods. Froissart tells us about Philip
van Artevelde, who is musing over a message just received, ``quant il
eut merancoliet une espasse, il s'avisa que il rescriproit aus
commissaires dou roi de
France,''\protect\hypertarget{09_Chapter_Two__THE_CRAVING_FOR_A_M.xhtmlux5cux23id_2423}{\protect\hyperlink{23_NOTES.xhtmlux5cux23id_2424}{‡\textsuperscript{7}}}
etc. Deschamps says about something so ugly that it is beyond all power
of imagination, ``No painter is so `merencolieux' that he would be able
to paint
it.''\textsuperscript{\protect\hypertarget{09_Chapter_Two__THE_CRAVING_FOR_A_M.xhtmlux5cux23id_2068}{\protect\hyperlink{23_NOTES.xhtmlux5cux23id_2069}{9}}}

In the pessimism of all these overburdened, disappointed, and fatigued
individuals there is a religious element, but only a very weak one.
Their world-weariness certainly echoes the expectation of the
approaching end of the world that was poured into minds everywhere by
the popular preaching of the revived mendicant orders, with renewed
threats and intensified imaginative power. The dark and confusing times,
the chronic misery of war were well suited to reinforce these thoughts.
There seems to have been, during the last years of the fourteenth
century, a popular belief that nobody had been admitted to paradise
since the Great Schism
be\protect\hypertarget{09_Chapter_Two__THE_CRAVING_FOR_A_M.xhtmlux5cux23page_35}{}{}gan.\textsuperscript{\protect\hypertarget{09_Chapter_Two__THE_CRAVING_FOR_A_M.xhtmlux5cux23id_2066}{\protect\hyperlink{23_NOTES.xhtmlux5cux23id_2067}{10}}}
Just turning away from the vain glitter of courtly life made people
ready to bid farewell to the world. For all that, the mood of depression
as expressed by nearly all princely liegemen and courtiers had hardly
any religious substance. At best, religious notions slightly colored the
general sense of life's malaise. This penchant for scorning life and the
world is a far cry from an essentially religious conviction. The world,
says Deschamps, is like a childlike old man; he was innocent in the
beginning, then for a long time wise, just, virtuous and brave:

\emph{Or est laches, chetis et molz},

\emph{Vieulx convoiteus et mal parlant};

\emph{Je ne voy que foies et folz} .~.~.

\emph{La fin s'approche, en verité .} ..

\emph{Tout va mal} .~.~.
\textsuperscript{\protect\hypertarget{09_Chapter_Two__THE_CRAVING_FOR_A_M.xhtmlux5cux23id_2064}{\protect\hyperlink{23_NOTES.xhtmlux5cux23id_2065}{11}}}\protect\hypertarget{09_Chapter_Two__THE_CRAVING_FOR_A_M.xhtmlux5cux23id_2425}{\protect\hyperlink{23_NOTES.xhtmlux5cux23id_2426}{*\textsuperscript{8}}}

There is not only weariness with the world, but also an actual dread of
life, a fearful shrinking away because of life's inevitable suffering;
this is the mental attitude that underlies Buddhism: an irresolute
turning away from the effort of everyday life, fear and disgust in
anticipation of disease and old age. Blasé individuals shared this dread
of life with those who had never succumbed to the temptations of the
world because they had always shied away from life.

The poems of Deschamps overflow with miserable aspersions about life. He
is fortunate who has no children because small children are nothing but
wails and stinks, trouble and worry. They have to be clothed, given
shoes, fed, and are always in danger of falling or hurting themselves.
They become sick and die, or else they grow up and turn bad; they are
put in jail. Nothing but trouble and disappointment, there is no
happiness to reward all the worries, efforts, and expenses of their
education. There is no greater misfortune than to have deformed
children. The poet has no loving words for them; deformed people have
black hearts, he has the scripture say. He who is unmarried is fortunate
because it is terrible to live with a bad woman and one has to be
constantly afraid of losing a
\protect\hypertarget{09_Chapter_Two__THE_CRAVING_FOR_A_M.xhtmlux5cux23page_36}{}{}good
one. As well as fleeing from misfortune one must shy away from good
fortune. In old age the poet sees nothing but evil and disgust, a
miserable physical and mental decay, laughable and calamitous. Old age
comes early, for woman at thirty, for men at fifty, and sixty is the
normal end of their life
span.\textsuperscript{\protect\hypertarget{09_Chapter_Two__THE_CRAVING_FOR_A_M.xhtmlux5cux23id_2062}{\protect\hyperlink{23_NOTES.xhtmlux5cux23id_2063}{12}}}
How far one is here from the pure ideality with which Dante described
the dignity of the noble elder in his
\emph{Convivio}.\textsuperscript{\protect\hypertarget{09_Chapter_Two__THE_CRAVING_FOR_A_M.xhtmlux5cux23id_2060}{\protect\hyperlink{23_NOTES.xhtmlux5cux23id_2061}{13}}}

A pious tendency, rarely found in Deschamps, may on occasion elevate
reflections on the dread of life, but the basic mood of discouraged
failure is always more strongly felt than genuine piety. Serious
admonitions to saintliness echo these negative elements more than they
reflect a genuine will for sanctification. The irreproachable Jean de
Gerson, the Chancellor of the University of Paris, writing a treatise
for his sister about the superiority of virginity, cites among his
arguments a long list of sufferings and pains bound up with marriage. A
husband may turn out to be a drunkard, or be extravagant or miserly. But
even if he is a solid and good individual there may be a bad harvest, or
epidemic, or shipwrecks may rob him of his worldly possessions. How
miserable is pregnancy, how many women die in childbirth! Does a nursing
mother ever enjoy undisturbed sleep, what about merriment and joy? Her
children may turn out to be malformed or disobedient, her husband may
die and the widowed mother be left to face a life of worry and
poverty.\textsuperscript{\protect\hypertarget{09_Chapter_Two__THE_CRAVING_FOR_A_M.xhtmlux5cux23id_2059}{\protect\hyperlink{23_NOTES.xhtmlux5cux23page_402}{14}}}

Daily reality is viewed in terms of the deepest depression whenever the
childlike joy of life or blind hedonism gives way to meditation. Where
is that more beautiful world for which every age is bound to yearn?

Those yearning for a better life, at all times, have seen three paths to
the distant goal before them. The first of these ordinarily leads away
from the world: the path of denial. The more beautiful life seems to be
attainable only in the world beyond; it will prove to be a deliverance
from all earthly concerns. All the attention wasted on the world delays
the promised bliss. This path has been followed in every higher culture.
Christianity had impressed this struggle on consciousness, both as the
purpose of an individual life and as the basis of culture, to such a
degree that it almost entirely prevented people from following the
second path for a long time.

The second path was that leading to the improvement and perfection of
the world itself. The Middle Ages hardly knew this way.
\protect\hypertarget{09_Chapter_Two__THE_CRAVING_FOR_A_M.xhtmlux5cux23page_37}{}{}To
them, the world was as good and as bad as it could be; that is, all
arrangements, since God had made them, were good: it was man's
sinfulness that made the world miserable. For this age, a conscious
striving for the improvement or reform of social and political
institutions was not the mainspring of thought and deed. To be virtuous
in the practice of one's own profession is the only way to benefit the
world, and even given this fact, the real goal is still the other life.
Moreover, wherever a new social form is actually created, it is seen in
principle as a restoration of good old tradition, or as a fight against
abuses by virtue of a deliberate delegation of power from the proper
authorities. The conscious creation of structures, thought of as truly
new, is rare even in the many-faceted legislative work carried out by
the French monarchy after Saint Louis. This work was imitated by the
dukes of Burgundy in their hereditary territories, but that those labors
actually constituted a development of the organization of the state in
the direction of more functional forms is a fact of which they were not
yet, or barely, aware. They issued ordinances or created offices because
this was in tune with their immediate task of promoting the general
welfare, not out of a serious vision on their part of a political
future.

Nothing contributed so much to the general mood of fearfulness and
pessimism about future times than this lack of a firm determination on
the part of all to make the world a better and happier place. This world
was not included in the promise of better things to come. To those
yearning for something better and yet unwilling to bid farewell to the
world and all its splendor, nothing was left but despair; nowhere could
they see hope or joy anymore. The world would only endure for a short
time and only misery remained for those in it.

Once the path of the positive improvement of the world is taken, a new
era is born in which the dread of life can give way to courage and hope.
This insight waits until the eighteenth century to appear. The
Renaissance owes its energetic affirmation of life to different sorts of
satisfactions. It is only the eighteenth century that makes the
perfectibility of man and society its chief dogma, and the social
struggle of the following century lost only the naiveté of its
predecessor, not its courage and optimism.

The third path to a better world leads through a land of dreams. It is
the most comfortable, but one in which the goal remains at an unchanging
distance. If earthly reality is so hopelessly miserable
\protect\hypertarget{09_Chapter_Two__THE_CRAVING_FOR_A_M.xhtmlux5cux23page_38}{}{}and
the denial of the world so difficult, this leaves us to color life with
lustrous tones, to live in a dreamland of shining fantasies, and to
soften reality in the ecstasy of the ideal. It requires only a simple
theme, a single chord, to begin the heart-stirring fugue: a glance at
the dreamy bliss of a more beautiful past time suffices, one glimpse of
its heroism and its virtue, or just the gay sunshine of life in nature
and its enjoyment. All literary culture since antiquity was based on two
themes: the heroic and the bucolic. The Middle Ages, the Renaissance,
and both the eighteenth and nineteenth centuries managed nothing more
than new variations on the old song.

But is this third path to a better life, this fleeing from harsh reality
into a beautiful illusion, only a concern of literary culture? Surely it
is more than that. Just as the other two paths, it affects the form and
content of communal life; and it affects that life the more strongly the
more primitive the culture is.

The impact of the three above-mentioned intellectual attitudes on real
life itself differs considerably. The most intimate and consistent
contact between the labor of life and the ideal goal is found when the
idea points to the improvement and perfection of the world itself. In
these instances both man's inspirational strength and his confidence
flow into material work. Immediate reality is charged with energy. To
follow one's life's calling means striving to attain the ideal of a
better world. If you wish, here too a blissful dream is the motivating
element. To a certain degree, every culture strives towards the creation
of a dream world in reality through the transformation of social forms.
But while in the other instances we encounter only a mental
transformation, the setting up of imaginary perfections in place of the
harsh reality one wants to forget, here the object of the dream is
reality itself. The idea is to transform reality, to cleanse and improve
it. The world appears to be on the good path towards the ideal, if only
people would go on working. The ideal form of life seems to be only
slightly distant from the life of labor; there is only a minute tension
between reality and the dream. Wherever striving for the highest
production and cheapest distribution of goods suffices, where the ideal
consists of welfare, freedom, and culture, there are comparatively few
demands placed on the art of life. There is no longer any need for men
to playact the roles of nobleman or hero, wise man or refined courtier.

The first of these three intellectual attitudes, that of world denial,
exercises an entirely different influence on real life. Homesickness
\protect\hypertarget{09_Chapter_Two__THE_CRAVING_FOR_A_M.xhtmlux5cux23page_39}{}{}for
eternal bliss renders us indifferent towards the events and forms of
earthly existence, desiring only that virtue be generated and maintained
in them. The forms of life and society are left as they are, but one
strives to permeate them with transcendent morality. This prevents the
turning away from the world from having an entirely negative effect on
the earthly community as merely denial and abstinence, but allows it to
radiate back on society in the form of godly work and practical charity.

But what is the impact of the third attitude on life? Does the yearning
for a better life correspond to a dreamed-of ideal? This attitude
changes the forms of life into forms of art. But this path does not
express its dream of beauty only in artworks as such: it aims at
ennobling life itself with beauty and fills communal life with play and
forms. Here are found the highest demands on the personal art of living,
demands that only an elite can try to meet with an artful life of
play.\textsuperscript{\protect\hypertarget{09_Chapter_Two__THE_CRAVING_FOR_A_M.xhtmlux5cux23id_2057}{\protect\hyperlink{23_NOTES.xhtmlux5cux23id_2058}{15}}}
The imitation of heroes and sages is not for everyone; painting life
with either heroic or idyllic colors is an expensive pastime and, as a
rule, is only partially successful. The struggle to realize the dream of
beauty within the forms of society itself has an aristocratic character
in its \emph{vitium originis}.

Now we have come to the point from which we intend to view the culture
of late medieval times: the point of the beautification of aristocratic
life with the forms of the ideal---the artistic light of chivalric
romanticism spread over life, with the world costumed in the garb of the
round table. The tension between the forms of life and reality is
extremely high; the light is false and overdone.

The desire for the beautiful life is generally held to be the most
characteristic feature of the Renaissance. Then we witness the greatest
harmony in satisfying the thirst for beauty, equally in works of art and
in life itself. Art served life and life served art as never before. But
here the line between the Middle Ages and the Renaissance is too sharply
drawn. The passionate desire to dress life in beauty, the refinement of
the art of living, the colorful products of a life lived in imitation of
an ideal are much older than the Italian quattrocento. The very motifs
of the beautification of life that the Florentines expanded upon are
nothing but old medieval forms; Lorenzo de'Medici, even as did Charles
the Bold, paid homage to the old knightly ideal as the noble form of
life. He even saw in it a model of sorts, its barbarian splendor
notwithstanding. Italy discovered new aspects of the beauty of life and
gave life a new
\protect\hypertarget{09_Chapter_Two__THE_CRAVING_FOR_A_M.xhtmlux5cux23page_40}{}{}tone,
but the attitude toward life that is usually seen as characteristic of
the Renaissance---the striving to transform or even elevate one's own
life to a higher level of artistic form---was by no means invented by
the Renaissance.

The great divide in the perception of the beauty of life comes much more
between the Renaissance and the modern period than between the Middle
Ages and the Renaissance. The turnabout occurs at the point where art
and life begin to diverge. It is the point where art begins to be no
longer in the midst of life, as a noble part of the joy of life itself,
but outside of life as something to be highly venerated, as something to
turn to in moments of edification or rest. The old dualism separating
God and world has thus returned in another form, that of the separation
of art and life. Now a line has been drawn right through the enjoyments
offered by life. Henceforth they are separated into two halves---one
lower, one higher. For medieval man they were all sinful without
exception; now they are all considered permissible, but their ethical
evaluation differs according to their greater or lesser degree of
spirituality.

The things that can make life enjoyable remain the same. They are, now
as before, reading, music, fine arts, travel, the enjoyment of nature,
sports, fashion, social vanity (knightly orders, honorary offices,
gatherings), and the intoxication of the senses. For the majority, the
border between the higher and lower levels seems now to be located
between the enjoyment of nature and sports. But this border is not firm.
Most likely sport will sooner or later again be counted among the higher
enjoyments---at least insofar as it is the art of physical strength and
courage. For medieval man the border lay, in the best of cases, right
after reading; the enjoyment of reading could only be sanctified through
striving for virtue or wisdom. For music and the fine arts, it was their
service to faith alone that was recognized as being good. Enjoyment
\emph{per se} was sinful. The Renaissance had managed to free itself
from the rejection of all the joy of life as something sinful, but had
not yet found a new way of separating the higher and lower enjoyments of
life; the Renaissance wanted an unencumbered enjoyment of all of life.
The new distinction is the result of the compromise between the
Renaissance and Puritanism that is at the base of modern spiritual
attitudes. It amounted to a mutual capitulation in which the one side
insisted on saving beauty while the other insisted on the condemnation
of sin. Strict Puritanism, just as did the Middle Ages, still condemned
\protect\hypertarget{09_Chapter_Two__THE_CRAVING_FOR_A_M.xhtmlux5cux23page_41}{}{}as
basically sinful and worldly the entire sphere of the beautification of
life with an exception being made in cases where such efforts assumed
expressly religious forms and sanctified themselves through their use in
the service of faith. Only after the Puritan worldview lost its
intensity did the Renaissance receptiveness to all the joys of life gain
ground again; perhaps even more ground than before, because beginning
with the eighteenth century there is a tendency to regard the natural
\emph{per se} as an element of the ethically good. Anyone attempting to
draw the dividing line between the higher and lower enjoyment of life
according to the dictates of ethical consciousness would no longer
separate art from sensuous enjoyment, the enjoyment of nature from the
cult of the body, the elevated from the natural, but would only separate
egotism, lies, and vanity from purity.

Towards the end of the medieval period, even as a new spirit began to
stir, there was, in principle, still only the old choice between God and
the world: the total rejection of all the splendor and beauty of earthly
life or a daring acceptance of it that ran the risk of harming the soul.
The beauty of the world became twice as tempting because its sinfulness
was recognized; surrendering oneself to it meant, therefore, to enjoy it
with unbridled passion. But those who could not do without beauty and
yet were unwilling to surrender to the world had no choice but to
ennoble beauty. They were able to sanctify the entire sector of art and
literature---where admiration constituted the essence of enjoyment---by
putting it in the service of faith. And if it was actually the enjoyment
of color and line that inspired the connoisseurs of painting and
miniatures, the stamp of sinfulness was removed from the enjoyment of
these objects because of their sacred subject matter. But what about
beauty with a high degree of sinfulness? How could all that, the cult of
the body of the knightly sports, courtly life, pride and the avidity for
office and honor, and the mesmerizing mystery of love, how could these
be made noble and elevated after faith had scorned and condemned them?
Here the middle path that led to the land of dreams helped: one dressed
everything in the beautiful light of the old fantastic ideals.

The strict cultivation of the beautiful life in the form of a heroic
ideal is the characteristic that ties French knightly culture after the
twelfth century to the Renaissance. The worship of nature was still too
weak to take the beauty of the world in all its nakedness into
\protect\hypertarget{09_Chapter_Two__THE_CRAVING_FOR_A_M.xhtmlux5cux23page_42}{}{}its
service with full conviction as the Greek mind had done: the idea of sin
was too powerful for that. Only in so much as people could wrap
themselves in the garment of virtue could beauty be brought to culture.

The whole aristocratic life of the later Middle Ages, whether one thinks
of France and Burgundy or of Florence, is an attempt to play out a
dream. It is always the same dream, that of the old heroes and sages, of
knight and maid, of simple and amusing shepherds. France and Burgundy
always play the piece in the old style; Florence composes on the set
theme a new and more beautiful variation.

Noble and princely life has reached up to its highest possible
expression; all the forms of life are equally elevated to the level of
mysteries, embellished with color and adornment, masked as virtue. The
events of life and the changes of emotion they trigger in us are here
framed in beautiful and elevating forms. I well understand that all this
is not specifically medieval; it had already arisen in the primitive
stages of culture, one can denominate it in chinoiserie and
Byzantianism, and it did not die with the Middle Ages, as the Sun King
proves.

The stateliness of the court is the arena wherein the aesthetic of the
form of life can unfold most fully. It is well known how much importance
the Burgundian princes attached to everything that bore on the splendor
and stateliness of their courts. Next to military glory, says
Chastellain, the courtly ritual is the most important thing demanding
attention and its regulation and maintenance are of highest
necessity.\textsuperscript{\protect\hypertarget{09_Chapter_Two__THE_CRAVING_FOR_A_M.xhtmlux5cux23id_2055}{\protect\hyperlink{23_NOTES.xhtmlux5cux23id_2056}{16}}}
Olivier de la Marche, the Master of Ceremonies of Charles the Bold, at
the instigation of the English king Edward IV, wrote a tract about
ritual at the court of Charles, urging this model of ceremony and
etiquette as worthy of
emulation.\textsuperscript{\protect\hypertarget{09_Chapter_Two__THE_CRAVING_FOR_A_M.xhtmlux5cux23id_2053}{\protect\hyperlink{23_NOTES.xhtmlux5cux23id_2054}{17}}}
The Hapsburgs inherited the beautifully elaborate court life of Burgundy
and later exported it to Spain and Austria, at which courts it remained
the bulwark of this high artificiality until recently. The court of
Burgundy was praised by all as the wealthiest and best
ordered.\textsuperscript{\protect\hypertarget{09_Chapter_Two__THE_CRAVING_FOR_A_M.xhtmlux5cux23id_2051}{\protect\hyperlink{23_NOTES.xhtmlux5cux23id_2052}{18}}}
Charles the Bold, above all, known as a man of violent disposition,
given to discipline and order but leaving nothing but disorder behind
him, had a passion for the most formal forms of life. The old illusion
that the prince himself heard the grievances of the poor and powerless
and adjudicated them on the spot was dressed by him in a beautiful form.
Two or three times
\protect\hypertarget{09_Chapter_Two__THE_CRAVING_FOR_A_M.xhtmlux5cux23page_43}{}{}a
week, after lunch, he had a public audience during which anyone could
approach him with petitions. All the noblemen of his house had to be
present; none dared to be absent. Carefully ordered according to rank,
they were seated on both sides of the passage leading to the high seat
of the duke. Kneeling at his feet were the two \emph{maistres des
requestes}, the \emph{audiencier}, and a secretary. They read the
petitions or dealt with them as instructed by the prince. Behind
balustrades placed around the hall were the lower ranking members of the
court. On the surface, says Chastellain, it was ``un chose magnifique et
de grand
los,''\protect\hypertarget{09_Chapter_Two__THE_CRAVING_FOR_A_M.xhtmlux5cux23id_2427}{\protect\hyperlink{23_NOTES.xhtmlux5cux23id_2428}{*\textsuperscript{9}}}
but the involuntary spectators were thoroughly bored and he doubted that
this method of administering justice was successful. It was,
nonetheless, something that he had never seen done by any other
prince.\textsuperscript{\protect\hypertarget{09_Chapter_Two__THE_CRAVING_FOR_A_M.xhtmlux5cux23id_2049}{\protect\hyperlink{23_NOTES.xhtmlux5cux23id_2050}{19}}}

Recreation, too, at the court of Charles the Bold had to take on
beautiful forms. ``Tournoit toutes ses manières et ses moeurs à sens une
part du jour, et avecques jeux et ris entremeslés, se delitoit en beau
parler et en amonester ses nobles à vertu, comme un orateur. Et en
cestuy regart, plusieurs fois, s'est trouvé assis en un hautdos paré et
ses nobles devant luy, là où il leur fit diverses remonstrances selon
les divers temps et causes. Et toujours, comme prince et chef sur tous,
fut richement et magnifiquement habitué sur tous les
autres.''\textsuperscript{\protect\hypertarget{09_Chapter_Two__THE_CRAVING_FOR_A_M.xhtmlux5cux23id_2047}{\protect\hyperlink{23_NOTES.xhtmlux5cux23id_2048}{20}}}\protect\hypertarget{09_Chapter_Two__THE_CRAVING_FOR_A_M.xhtmlux5cux23id_2429}{\protect\hyperlink{23_NOTES.xhtmlux5cux23id_2430}{†\textsuperscript{10}}}
This conscious effort to make an art form of life is actually a perfect
realization of the Renaissance, its stiff and naive forms
notwithstanding. What Chastellain calls his ``haute magnificence de
coeur pour estre vu et regardé en singulières
choses''\protect\hypertarget{09_Chapter_Two__THE_CRAVING_FOR_A_M.xhtmlux5cux23id_2431}{\protect\hyperlink{23_NOTES.xhtmlux5cux23id_2432}{‡\textsuperscript{11}}}
is the characteristic quality of Burckhardt's Renaissance man.

The hierarchical arrangements of the courtly household have a
Rabelaisian exuberance wherever they involve meals and the kitchen. The
courtly table of Charles the Bold, with all the panetiers and carvers
and wine pourers and chefs, whose services were regulated with nearly
liturgical dignity, resembled the performance
\protect\hypertarget{09_Chapter_Two__THE_CRAVING_FOR_A_M.xhtmlux5cux23page_44}{}{}of
a grand and solemn play. The entire court ate in groups of ten in
separate rooms, served and attended, as was the duke, with scrupulous
observance of rank and standing. Everything was so well regulated that
all these groups, after finishing their meal, still had time to greet
the duke who was still sitting at his table, ``pour luy donner
gloire.''\textsuperscript{\protect\hypertarget{09_Chapter_Two__THE_CRAVING_FOR_A_M.xhtmlux5cux23id_2045}{\protect\hyperlink{23_NOTES.xhtmlux5cux23id_2046}{21}}}

In the kitchen (one should try to imagine the heroic kitchen with its
seven giant hearths, now all that is left of the ducal palace in Dijon)
sits the on-duty chef in an armchair, located between hearth and buffet,
from which he is able to overlook the entire room. In his hand he must
hold a large wooden spoon ``which serves him two purposes: one, to taste
soup and sauces; the other to spur the kitchen boys to their duty and,
if necessary, to spank them.'' On rare occasions, as for example, when
the first truffles or the first herring are served, the chef
himself---holding a torch---may do the honors.

To the stilted courtier who describes all this for us, these are sacred
mysteries about which he speaks with respect and in a kind of scholastic
scientific manner. When I was a page boy, La Marche says, I was still
too young to understand questions of \emph{préséance} and
ceremonial.\textsuperscript{\protect\hypertarget{09_Chapter_Two__THE_CRAVING_FOR_A_M.xhtmlux5cux23id_2043}{\protect\hyperlink{23_NOTES.xhtmlux5cux23id_2044}{22}}}
He puts before his readers important questions of precedence and court
service in order to answer them on the basis of his mature insights. Why
must the cook and not the \emph{écuyer de cuisine} be present at the
master's meal? In what manner must the cook come into employment at the
court? Who should replace him in case of his absence: the \emph{hateur}
or the \emph{potagier}? Here I answer, says the wise man: when a cook is
to be employed at the court of a prince, the \emph{maîtres d'hôte}, the
\emph{écuyers de cuisine}, and all those employed in the kitchen speak
up one by one and by solemn choice made by everyone of them under his
oath, the cook takes his position. And to the second question: neither
the \emph{hateur} nor the \emph{potagier} may replace him but only an
individual chosen by a similar election may substitute for the cook. Why
do the panetiers and cupbearers hold the first and second ranks above
the meat carvers and cooks? Because their office concerns bread and
wine---holy objects glorified by virtue of the
sacrament.\textsuperscript{\protect\hypertarget{09_Chapter_Two__THE_CRAVING_FOR_A_M.xhtmlux5cux23id_2041}{\protect\hyperlink{23_NOTES.xhtmlux5cux23id_2042}{23}}}

One can see in this instance that there is an actual connection between
the sphere of faith and that of court etiquette. It does not overstate
the case to claim that every means of beautifying and ennobling the
forms of life contain a liturgical element that raises the observance of
these forms almost to a religious realm. Only
\protect\hypertarget{09_Chapter_Two__THE_CRAVING_FOR_A_M.xhtmlux5cux23page_45}{}{}this
can explain the extraordinary importance people give to all questions of
precedence and etiquette, and not only during late medieval times.

Quarrels over royal precedence resulted in the establishment of a
regular department of state service in the pre-Romanov days of the
Russian empire. Though the western states of medieval times did not
create departments, envy about precedence played an important role. It
would be easy to gather examples to illustrate this, but here we need
only to show how the forms of life were elaborated into beautiful and
uplifting games and the wild growth of these games into empty display.
Here are just a few examples to demonstrate this. Formal beauty may
occasionally completely push practical action aside. Immediately before
the battle of Crécy, four French knights reconnoitered the English order
of battle. The king, quite impatiently awaiting their report and slowly
riding across the field, halts his horse when he sees them returning.
They manage to make their way to the presence of the king through the
throng of warriors. ``What news do you have, messiers?'' asks the king.
They look at each other without speaking a word because none of them
wants to speak ahead of his comrades. And one says to the other, ``Sire,
you tell it, you speak to the King, I will not speak ahead of you.'' So
they debate a while, back and forth, because no one wants to speak
first, ``par honneur.'' Finally, the king orders one of them to
report.\textsuperscript{\protect\hypertarget{09_Chapter_Two__THE_CRAVING_FOR_A_M.xhtmlux5cux23id_2039}{\protect\hyperlink{23_NOTES.xhtmlux5cux23id_2040}{24}}}
Practicality had to give way to beautiful form even more so in the case
of Messire Gaultier Rallart, \emph{Chevalier du guet} in Paris in 1418.
This chief of police never went on his rounds unless he was accompanied
by three or four musicians who preceded him. They played so lustily that
people said he was practically warning the crooks, ``Flee, for I am
coming!''\textsuperscript{\protect\hypertarget{09_Chapter_Two__THE_CRAVING_FOR_A_M.xhtmlux5cux23id_2037}{\protect\hyperlink{23_NOTES.xhtmlux5cux23id_2038}{25}}}
This is not an isolated case. There is another case in 1465. The bishop
of Evreux, Jean Balure, makes his nightly round in Paris accompanied by
clarinets, trumpets and other musical instruments, ``qui n'estoit pas
acoustumé de faire à gens faisans
guet.''\textsuperscript{\protect\hypertarget{09_Chapter_Two__THE_CRAVING_FOR_A_M.xhtmlux5cux23id_2035}{\protect\hyperlink{23_NOTES.xhtmlux5cux23id_2036}{26}}}\protect\hypertarget{09_Chapter_Two__THE_CRAVING_FOR_A_M.xhtmlux5cux23id_2433}{\protect\hyperlink{23_NOTES.xhtmlux5cux23id_2434}{*\textsuperscript{12}}}
The honors due rank and status were strictly observed even at the
scaffold: that of the \emph{connétable} de Saint Pol is richly decorated
with embroidered lilies, the prayer pillow and the blindfold are of
crimson velvet, and the hangman is one who has never hanged anyone
before---a rather dubious privilege for the
condemned.\textsuperscript{\protect\hypertarget{09_Chapter_Two__THE_CRAVING_FOR_A_M.xhtmlux5cux23id_2033}{\protect\hyperlink{23_NOTES.xhtmlux5cux23id_2034}{27}}}

\protect\hypertarget{09_Chapter_Two__THE_CRAVING_FOR_A_M.xhtmlux5cux23page_46}{}{}Competition
in courtliness and politeness---now characteristically \emph{petit
bourgeois}---was extraordinarily strongly developed in the life of the
courts of the fifteenth century. It was regarded as a personal and
unbearable disgrace not to yield to the higher ranking their proper
place. Burgundian dukes gave painfully correct precedence to their royal
relations of France. John the Fearless treated his young
daughter-in-law, Michelle de France, at all times with exaggerated
respect; he called her madame, always knelt before her and offered to
serve her constantly---something she was, however, not prepared to
tolerate.\textsuperscript{\protect\hypertarget{09_Chapter_Two__THE_CRAVING_FOR_A_M.xhtmlux5cux23id_2031}{\protect\hyperlink{23_NOTES.xhtmlux5cux23id_2032}{28}}}
When Duke Philip the Good learns that his cousin the dauphin, heir to
the throne of France, has escaped to Brabant during a conflict with his
father the king, he interrupts the siege of Deventer, part of an
expedition to bring Frisia under his control, and hurries back to
Brussels to welcome his noble guest. The closer the time of their
meeting comes, the greater the competition over who will outdo the other
in paying homage. Philip is in mortal fear that the dauphin will come
out to meet him. He travels posthaste and sends messenger after
messenger to make the dauphin wait in place for him. If the dauphin
comes to meet him in person, he vows, he'll turn around and travel so
far that the dauphin will never find him because the duke will be so
ridiculed and shamed that the whole world will never let him forget it.
Philip enters Brussels modestly departing from the usual pomp; he
hastily dismounts in front of the palace and enters it. He runs forward,
then he sees the dauphin who, with the duchess, has left his chamber and
who approaches Philip in the courtyard with open arms. Immediately the
old duke bares his head, kneels down for a short moment, and runs
forward in great haste. The duchess holds on to the dauphin to keep him
from taking another step while the dauphin tries in vain to hold the
duke on his feet and to keep him from kneeling down. Failing this, he
tries to get the duke to stand up. Both weep with emotion, says
Chastellain, and all the bystanders weep with them.

For the duration of the stay by this man who, as king, was soon to
become the worst enemy of his house, the duke outdoes himself in
displays of Chinese servility. He calls himself and his son ``de si
meschans
gens,''\protect\hypertarget{09_Chapter_Two__THE_CRAVING_FOR_A_M.xhtmlux5cux23id_2307}{\protect\hyperlink{23_NOTES.xhtmlux5cux23id_2308}{*\textsuperscript{13}}}
he exposes his sixty-year-old head to the rain, and offers the dauphin
all his
lands.\textsuperscript{\protect\hypertarget{09_Chapter_Two__THE_CRAVING_FOR_A_M.xhtmlux5cux23id_2029}{\protect\hyperlink{23_NOTES.xhtmlux5cux23id_2030}{29}}}
``Celuy qui se humilie devant
\protect\hypertarget{09_Chapter_Two__THE_CRAVING_FOR_A_M.xhtmlux5cux23page_47}{}{}son
plus grand, celuy accroist et multiplie son honneur envers soymesme, et
de quoy la bonté mesme luy resplend et redonde en
face'':\protect\hypertarget{09_Chapter_Two__THE_CRAVING_FOR_A_M.xhtmlux5cux23id_2435}{\protect\hyperlink{23_NOTES.xhtmlux5cux23id_2436}{*\textsuperscript{14}}}
with these words Chastellain ends his report about how the Count of
Charolais stubbornly refused before a meal to use the same wash basin as
Queen Margaret of England and her young son. The noblemen talked about
it all day: the case was brought before the old duke, who had two
noblemen argue the pros and cons of Charles's attitude. The feudal sense
of honor was still alive to the degree that such things were apparently
still held to be meaningful, beautiful, and edifying. How else are we to
understand that refusals to accept precedence could, as a rule, be
continued for a quarter of an
hour?\textsuperscript{\protect\hypertarget{09_Chapter_Two__THE_CRAVING_FOR_A_M.xhtmlux5cux23id_2027}{\protect\hyperlink{23_NOTES.xhtmlux5cux23id_2028}{30}}}
The longer the refusal, the more impressed the bystanders. Someone
entitled to have his hand kissed hides his hand to avoid the honor. The
queen of Spain hides her hand in this manner to thwart the young
archduke Philip the Beautiful, but the latter, after having waited for a
while, unexpectedly seizes her hand and kisses it. The entire Spanish
court broke out in laughter on this occasion because the queen was no
longer expecting this
gesture.\textsuperscript{\protect\hypertarget{09_Chapter_Two__THE_CRAVING_FOR_A_M.xhtmlux5cux23id_2025}{\protect\hyperlink{23_NOTES.xhtmlux5cux23id_2026}{31}}}

All spontaneous displays of tenderness in social relations are carefully
turned into form. It is precisely prescribed which of the ladies at
court had to go about holding hands, and even which one or the other had
to take the initiative. The invitation, a wave or a call, is a technical
term \emph{(hucher)} in the vocabulary of the old court lady who
describes Burgundian court
etiquette.\textsuperscript{\protect\hypertarget{09_Chapter_Two__THE_CRAVING_FOR_A_M.xhtmlux5cux23id_2023}{\protect\hyperlink{23_NOTES.xhtmlux5cux23id_2024}{32}}}
The formality of preventing a departing guest from leaving is carried to
the most vexing extremes. The wife of Louis XI is for a few days the
guest of Philip of Burgundy; Louis has set a certain day for her return,
but the duke refuses to let her go, disregarding the fervent pleas of
her attendants and even the queen's own fear of her husband's
rage.\textsuperscript{\protect\hypertarget{09_Chapter_Two__THE_CRAVING_FOR_A_M.xhtmlux5cux23id_2021}{\protect\hyperlink{23_NOTES.xhtmlux5cux23id_2022}{33}}}
Goethe said: ``There is no external sign of courtesy without a deep
ethical cause,'' but Emerson called courtesy ``virtue gone to seed.'' It
is perhaps not justifiable to claim that this ethical cause was still
any longer felt during the fifteenth century, but surely the aesthetic
value was located somewhere between the honest display of affection and
barren social form.

\protect\hypertarget{09_Chapter_Two__THE_CRAVING_FOR_A_M.xhtmlux5cux23page_48}{}{}It
goes without saying that these overelaborated embellishments of life
took place above all at the princely courts, where sufficient time and
space were available. But that they also permeated the lower spheres of
society is proven by the fact that these forms are preserved today
precisely among the \emph{petite bourgeoisie} (not to speak of the
courts themselves). The customs of urging guests repeatedly to have
still another helping of a particular dish, of encouraging them to stay
a little longer, of refusing to go ahead of someone, have for the most
part disappeared during the last half-century from the etiquette of the
higher bourgeoisie. But during the fifteenth century these forms were
still in full bloom. Yet while they are most painfully observed, they
are the target of biting satire. Above all, it is at church where the
stage for beautiful and lengthy displays of civility is found. Most
obviously during the \emph{offrande}, because nobody wants to be the
first to put his alms on the altar.

\emph{``Passez.''---``Non feray.''---``Or avant!}

\emph{Certes si ferez, ma cousine.''}

---\emph{``Non feray.''}---\emph{``Huchez no voisine},

\emph{Qu'elle doit mieux devant offrir.''}

---\emph{``Vous ne le devriez souffrir.''}

\emph{Dist la voisine: ``n'appartient}

\emph{A moy: offrez, qu'à vous ne tient}

\emph{Que li prestres ne se
delivre}.''\textsuperscript{\protect\hypertarget{09_Chapter_Two__THE_CRAVING_FOR_A_M.xhtmlux5cux23id_2019}{\protect\hyperlink{23_NOTES.xhtmlux5cux23id_2020}{34}}}\emph{\protect\hypertarget{09_Chapter_Two__THE_CRAVING_FOR_A_M.xhtmlux5cux23id_2437}{\protect\hyperlink{23_NOTES.xhtmlux5cux23id_2438}{*\textsuperscript{15}}}}

Finally after the social superior among them had at last taken the lead,
all the time humbly protesting that he did so only to end the stalemate,
the quarrel starts over again about who will first kiss \emph{la paix},
the wooden, silver, or ivory plate that had found its way into the mass,
following the Agnus Dei, during late medieval times replacing the kiss
of peace that had been given mouth to
mouth.\textsuperscript{\protect\hypertarget{09_Chapter_Two__THE_CRAVING_FOR_A_M.xhtmlux5cux23id_2017}{\protect\hyperlink{23_NOTES.xhtmlux5cux23id_2018}{35}}}
Because the \emph{paix} was passed from hand to hand among the prominent
members of the congregation, who most courteously refused
\protect\hypertarget{09_Chapter_Two__THE_CRAVING_FOR_A_M.xhtmlux5cux23page_49}{}{}to
kiss it first, it became a standard and protracted disturbance of church
services.

\emph{Respondre doit la jeune fame}:

---\emph{Prenez, je ne prendray pas, dame}.

---\emph{Si ferez, prenez, douce amie}.

---\emph{Certes, je ne le prandray mie};

\emph{L'en me tendroit pour un sote}

---\emph{Baillez, damoiselle Masrote}.

---\emph{Non feray, Jhesucrist m'en gart}!

\emph{Portez à ma dame Ermagart}.

---\emph{Dame, prenez}.---\emph{Saincte Marie},

\emph{Portez la paix a la baille}.

---\emph{Non, mais à la gouverner
esse}.\textsuperscript{\protect\hypertarget{09_Chapter_Two__THE_CRAVING_FOR_A_M.xhtmlux5cux23id_2015}{\protect\hyperlink{23_NOTES.xhtmlux5cux23id_2016}{36}}}\protect\hypertarget{09_Chapter_Two__THE_CRAVING_FOR_A_M.xhtmlux5cux23id_2439}{\protect\hyperlink{23_NOTES.xhtmlux5cux23id_2440}{*\textsuperscript{16}}}

At last, she accepts it. Even a saintly, world-renouncing man like Franz
von Paula considers it his duty to participate in this affectation, and
his pious admirers credit this as a sign of true humility, proving that
the ethical content had not altogether vanished from these
formalities.\textsuperscript{\protect\hypertarget{09_Chapter_Two__THE_CRAVING_FOR_A_M.xhtmlux5cux23id_2013}{\protect\hyperlink{23_NOTES.xhtmlux5cux23id_2014}{37}}}
The importance of these forms, incidentally, is clearly evident in the
fact that the precedence that people so civilly forced upon one another
in church, was, on the other hand, the cause of volatile and stubborn
quarrels.\textsuperscript{\protect\hypertarget{09_Chapter_Two__THE_CRAVING_FOR_A_M.xhtmlux5cux23id_2011}{\protect\hyperlink{23_NOTES.xhtmlux5cux23id_2012}{38}}}
Yielding precedence was a beautiful and praiseworthy denial of a lively
noble or bourgeois arrogance.

The entire church visit thus became a kind of minuet, since the quarrel
resumed upon leaving the church. Then came the competition to get the
higher ranking individuals to walk on the right side, the question of
who would cross a plank bridge or enter a narrow alley. Arriving at
home, one had to invite the entire company to come inside for a
drink---something that Spanish custom still requires---while the
invitees, in turn, were obliged to refuse in a most polite manner;
whereupon the would-be host had to accompany
\protect\hypertarget{09_Chapter_Two__THE_CRAVING_FOR_A_M.xhtmlux5cux23page_50}{}{}them
part of their way: all this again amidst displays of polite refusal by
those
accompanied.\textsuperscript{\protect\hypertarget{09_Chapter_Two__THE_CRAVING_FOR_A_M.xhtmlux5cux23id_2009}{\protect\hyperlink{23_NOTES.xhtmlux5cux23id_2010}{39}}}

There is something touching about these beautiful forms, particularly if
we remind ourselves that they are the blossoms that arise from the
serious struggle with its own arrogance and rage of a race prone to
violence and passion. The formal denial of pride frequently fails and,
time and again, crass rudeness breaks through the ornate forms. John of
Bavaria is a guest in Paris; the luminaries of the city entertain him
lavishly but the elector of Liege takes all their money in a game of
chance. One of the princes can stand it no longer and cries out: ``What
the devil kind of a priest is this? How? Shall he take all of our
money?'' Whereupon John replies, ``I am no priest and I don't need your
money.'' And he took the money and tossed it all over the room. ``Dont y
pluseurs orent grant mervelle de sa grant
liberaliteit.''\textsuperscript{\protect\hypertarget{09_Chapter_Two__THE_CRAVING_FOR_A_M.xhtmlux5cux23id_2008}{\protect\hyperlink{23_NOTES.xhtmlux5cux23page_403}{40}}}\protect\hypertarget{09_Chapter_Two__THE_CRAVING_FOR_A_M.xhtmlux5cux23id_2441}{\protect\hyperlink{23_NOTES.xhtmlux5cux23id_2442}{*\textsuperscript{17}}}
Hue de Lannoy hits someone with an iron glove while the victim is
kneeling in accusation before the duke; the Cardinal of Bar calls a
priest a liar and a low dog in the presence of the
king.\textsuperscript{\protect\hypertarget{09_Chapter_Two__THE_CRAVING_FOR_A_M.xhtmlux5cux23id_2006}{\protect\hyperlink{23_NOTES.xhtmlux5cux23id_2007}{41}}}

The formal sense of honor is so strong that an affront against
etiquette, as is still the case among many Oriental people, wounds like
a mortal insult because it causes the beautiful illusion that one's own
life is high and pure---something found at the bottom of any unveiled
reality---to collapse. To John the Fearless it is a matter of unerasable
shame that he has greeted Capeluche, the hangman of Paris, who meets him
dressed in full regalia, like a nobleman and touched his hand; only the
death of the hangman will redress this
outrage.\textsuperscript{\protect\hypertarget{09_Chapter_Two__THE_CRAVING_FOR_A_M.xhtmlux5cux23id_2004}{\protect\hyperlink{23_NOTES.xhtmlux5cux23id_2005}{42}}}
During the state banquet at the coronation of Charles VI in 1380, Philip
of Burgundy forces his way to the seat between the king and the duke of
Anjou to which he is entitled as the senior of the two; both their
entourages approach with shouts and threats to settle the dispute with
force, but the king settles it by giving in to the demands of the
Burgundian.\textsuperscript{\protect\hypertarget{09_Chapter_Two__THE_CRAVING_FOR_A_M.xhtmlux5cux23id_2002}{\protect\hyperlink{23_NOTES.xhtmlux5cux23id_2003}{43}}}
Amidst the serious life on the campaigns, too, violations of forms are
not tolerated. The King of England resents that I'lsle Adam appears
before him in a garb of ``blanc gris'' and looks at him face to
face.\textsuperscript{\protect\hypertarget{09_Chapter_Two__THE_CRAVING_FOR_A_M.xhtmlux5cux23id_2000}{\protect\hyperlink{23_NOTES.xhtmlux5cux23id_2001}{44}}}
An English commander sends a peace emissary from the besieged city of
Sens to a barber for a shave before receiving
him.\textsuperscript{\protect\hypertarget{09_Chapter_Two__THE_CRAVING_FOR_A_M.xhtmlux5cux23id_1998}{\protect\hyperlink{23_NOTES.xhtmlux5cux23id_1999}{45}}}

The splendid order of the court of Burgundy that was praised
\protect\hypertarget{09_Chapter_Two__THE_CRAVING_FOR_A_M.xhtmlux5cux23page_51}{}{}by
contemporaries\textsuperscript{\protect\hypertarget{09_Chapter_Two__THE_CRAVING_FOR_A_M.xhtmlux5cux23id_1996}{\protect\hyperlink{23_NOTES.xhtmlux5cux23id_1997}{46}}}
reveals its true significance only if it is viewed side by side with the
confusion customary at the much older French court. Deschamps in a
number of ballads decries the misery of court life. His laments mean
something more than the usual disapproval of the life of a courtier,
about which we will talk later on. Poor food and poor lodgings, constant
clamor and confusion, curses and quarrels, envy and mockery: it is a
cesspool of sin, a gateway to
hell.\textsuperscript{\protect\hypertarget{09_Chapter_Two__THE_CRAVING_FOR_A_M.xhtmlux5cux23id_1994}{\protect\hyperlink{23_NOTES.xhtmlux5cux23id_1995}{47}}}
In spite of the pious veneration of royalty and the proud edifice of
grand ceremonies, the decorum of the most significant of events is
pitifully lost on more than one occasion. During the funeral of Charles
VI at St. Denis in 1422, massive disputes arise between the monks of the
abbey and the guild of the salt-weighers \emph{(henouars)} over the
state robe and other items of clothing covering the royal corpse; each
of the parties insist that it has a claim to them and they engage in a
tug-of-war and nearly come to blows. But the duke of Bedford turned the
case over to the courts, ``et fut le corps
enterré!''\textsuperscript{\protect\hypertarget{09_Chapter_Two__THE_CRAVING_FOR_A_M.xhtmlux5cux23id_1992}{\protect\hyperlink{23_NOTES.xhtmlux5cux23id_1993}{48}}}\protect\hypertarget{09_Chapter_Two__THE_CRAVING_FOR_A_M.xhtmlux5cux23id_2443}{\protect\hyperlink{23_NOTES.xhtmlux5cux23id_2444}{*\textsuperscript{18}}}
The same quarrel is repeated in 1461 during the funeral of Charles VII.
Having arrived at the Croix aux Fiens on their way to St. Denis, the
salt-weighers, after an exchange of words with the monks of the abbey,
refuse to carry the royal corpse any further unless they are paid ten
Parisian pounds to which they claim to be entitled. They leave the bier
in the middle of the road and the funeral procession is held up for
considerable time. The burghers of St. Denis are on the verge of
assuming the duties themselves when the \emph{grand écuyer} promises
payment out of his own pocket to the \emph{henouars}. Thereupon the
procession continues and finally reaches the church at nearly eight
o'clock in the evening. Immediately following the funeral a new dispute
over the state robe ensues between the monks and the royal \emph{grand
écuyer}
himself.\textsuperscript{\protect\hypertarget{09_Chapter_Two__THE_CRAVING_FOR_A_M.xhtmlux5cux23id_1990}{\protect\hyperlink{23_NOTES.xhtmlux5cux23id_1991}{49}}}
Similar tumultuous confrontations over the ownership of the utensils of
a festive event were a regular part of the festivals, so to speak;
disrupting a form had itself become a
form.\textsuperscript{\protect\hypertarget{09_Chapter_Two__THE_CRAVING_FOR_A_M.xhtmlux5cux23id_1988}{\protect\hyperlink{23_NOTES.xhtmlux5cux23id_1989}{50}}}

The general public, which even during the seventeenth century was still
a mandatory participant in all the important events of royal life, cause
the largest festive occasions, in particular, to frequently lack any
semblance of order. During the 1380 coronation banquet the throng of
spectators, participants, and servants was so great that the king's
waiters, especially hired for the purpose, the
\emph{\protect\hypertarget{09_Chapter_Two__THE_CRAVING_FOR_A_M.xhtmlux5cux23page_52}{}{}connétable}
and the marshal of Sancerre had to serve the dishes from
horseback.\textsuperscript{\protect\hypertarget{09_Chapter_Two__THE_CRAVING_FOR_A_M.xhtmlux5cux23id_1986}{\protect\hyperlink{23_NOTES.xhtmlux5cux23id_1987}{51}}}
When Henry VI of England is crowned king in Paris in 1431, the people
crowd into the great hall of the palace at early morning, some to watch,
some to pilfer, and some to sneak a bite or two. The lords of
parliament, those of the university, the \emph{prévôt des marchands},
and the aldermen are barely able to push their way to the banquet and,
when they get there, find that the tables meant for them have been taken
by a number of craftsmen. Attempts are made to remove them, ``mais quant
on en faisoit lever ung ou deux, il s'en asseoit VI ou VIII d'autre
costé.''\textsuperscript{\protect\hypertarget{09_Chapter_Two__THE_CRAVING_FOR_A_M.xhtmlux5cux23id_1984}{\protect\hyperlink{23_NOTES.xhtmlux5cux23id_1985}{52}}}\protect\hypertarget{09_Chapter_Two__THE_CRAVING_FOR_A_M.xhtmlux5cux23id_2445}{\protect\hyperlink{23_NOTES.xhtmlux5cux23id_2446}{*\textsuperscript{19}}}
At the coronation of Louis XI in 1461 the Cathedral of Reims is closed
early and carefully guarded as a precaution that only as many people be
admitted to the church as the choir can safely hold. However, the place
near the high altar where the anointment takes place is so crowded that
there is hardly any room for the prelate assisting the bishop to move
and the princes of the blood on their seats of honor are in acute
physical
danger.\textsuperscript{\protect\hypertarget{09_Chapter_Two__THE_CRAVING_FOR_A_M.xhtmlux5cux23id_1982}{\protect\hyperlink{23_NOTES.xhtmlux5cux23id_1983}{53}}}

The church in Paris only reluctantly tolerated the fact that it was
(until 1622) the \emph{suffragan} of the Archbishopric of Sens. The
archbishop was made to feel in every way that his authority was not
appreciated and there was constant reference to the exemption granted by
the pope. On February 2, 1492, the archbishop of Sens celebrates a mass
in Notre Dame in Paris in the presence of the king. Before the king
leaves the church, the archbishop, blessing the crowd, retreats with the
priest's cross carried ahead of him. Two of the canons advance with a
large number of ecclesiastics, get their hands on the cross and damage
it, twist the hand of the man carrying it, and start a tumultuous scene
during which the servants of the archbishop have some of their hair
pulled out. When the archbishop attempts to end the quarrel, ``sans lui
mot dire, vinrent prés de lui; Lhuillier {[}Dean of the cathedral{]} lui
baille du coude dans l'estomac, les autres romprient le chapeau
pontifical et les cordons
d'icelluy.''\protect\hypertarget{09_Chapter_Two__THE_CRAVING_FOR_A_M.xhtmlux5cux23id_2448}{\protect\hyperlink{23_NOTES.xhtmlux5cux23id_2447}{†\textsuperscript{20}}}
The other canon chases the archbishop ``disant plusieurs injures en luy
mectant le doigt au visage, et prenant son bras tant que dessira son
rochet, et n'eust esté que n'eust mis
\protect\hypertarget{09_Chapter_Two__THE_CRAVING_FOR_A_M.xhtmlux5cux23page_53}{}{}sa
main au devant, l'eust frappé au
visage.''\protect\hypertarget{09_Chapter_Two__THE_CRAVING_FOR_A_M.xhtmlux5cux23id_2449}{\protect\hyperlink{23_NOTES.xhtmlux5cux23id_2450}{*\textsuperscript{21}}}
This resulted in a lawsuit that lasted thirteen
years.\textsuperscript{\protect\hypertarget{09_Chapter_Two__THE_CRAVING_FOR_A_M.xhtmlux5cux23id_1980}{\protect\hyperlink{23_NOTES.xhtmlux5cux23id_1981}{54}}}

The passionate and violent mind of the time, hardened and at the same
time prone to tears; on the one side despairing of the world, yet on the
other reveling in its colorful beauty, could not exist without the
strictest formalized behavior. It was essential that the excitement be
fixed in a firm frame of standardized forms. Only in this way could life
attain a regulated ordering. Thus one's own experiences and those of
others were turned into a beautiful, intellectually pleasing
presentation; people enjoyed the exaggerated spectacle of suffering and
joy under stage lights. The means for a purely spiritual expression was
still lacking. Only the aesthetic shaping of emotions allowed that high
degree of expression demanded by the times.

This does not mean, of course, that these life forms, above all those
relating to the great holy events of birth, marriage, and death, were
implemented with such meaning in mind. Customs and ceremonies grow out
of primitive beliefs and cults. But the original meaning that
constituted their essence has long been lost from consciousness. In its
place the forms have been filled with new aesthetic value.

The dressing of sentiment in the garb of a suggestive form reaches its
highest development in mourning. There were unlimited possibilities for
a splendid exaggeration of sorrow, the counterpart of the hyperbolic
expressions of joy during the grandiose court festivities. We do not
intend to offer at this point a detailed description of all that somber
splendor of black dresses and the lavish display of the funeral
ceremonies that accompanied the death of every prince. This is not a
characteristic exclusively belonging to the later Middle Ages;
monarchies preserve it in our time, and the bourgeois hearse is one of
its products. The suggestiveness of the black, used for the clothing not
only of the court, but also of the magistrates, the members of the
guilds, and ordinary people on the occasion of a princely death, must
have been made much stronger by the contrast to the ordinarily rich and
varied colors of medieval city life. The funeral pomp displayed over the
murdered
\protect\hypertarget{09_Chapter_Two__THE_CRAVING_FOR_A_M.xhtmlux5cux23page_54}{}{}John
the Fearless was tailored with the most deliberate of intentions for
maximum (and in part political) effect. The retinue of warriors
accompanying Philip to greet the Kings of England and France displays
two thousand black pennons with black standards and banners seven yards
long, the fringes of black lace, all embroidered or painted with golden
escutcheons. The duke's throne and coach of state have been painted
black for the
occasion.\textsuperscript{\protect\hypertarget{09_Chapter_Two__THE_CRAVING_FOR_A_M.xhtmlux5cux23id_1978}{\protect\hyperlink{23_NOTES.xhtmlux5cux23id_1979}{55}}}
At a splendid meeting at Troyes, Philip accompanies the Queens of France
and England in a velvet mourning garb that trails across the back of his
horse and down to the
ground.\textsuperscript{\protect\hypertarget{09_Chapter_Two__THE_CRAVING_FOR_A_M.xhtmlux5cux23id_1976}{\protect\hyperlink{23_NOTES.xhtmlux5cux23id_1977}{56}}}
He and his entourage continued to wear black for a considerable period
after
that.\textsuperscript{\protect\hypertarget{09_Chapter_Two__THE_CRAVING_FOR_A_M.xhtmlux5cux23id_1974}{\protect\hyperlink{23_NOTES.xhtmlux5cux23id_1975}{57}}}

On occasion an exception in the midst of all that black could enhance
the impact: while the entire French court, including the queen, wears
black, the king mourns wearing
red.\textsuperscript{\protect\hypertarget{09_Chapter_Two__THE_CRAVING_FOR_A_M.xhtmlux5cux23id_1972}{\protect\hyperlink{23_NOTES.xhtmlux5cux23id_1973}{58}}}
And in 1393 the Parisians viewed with consternation the all-white
funeral procession for the King of Armenia, Leo of Lusignan, who had
died in
exile.\textsuperscript{\protect\hypertarget{09_Chapter_Two__THE_CRAVING_FOR_A_M.xhtmlux5cux23id_1970}{\protect\hyperlink{23_NOTES.xhtmlux5cux23id_1971}{59}}}

There is no doubt that black mourning dress frequently enclosed a large
measure of genuine and passionate grief. Given the medievals' fear of
death, strong family attachments, and intense loyalty to their lord, the
death of a prince was a truly depressing event. Add to this an injury to
the honor of a proud family that made revenge a sacred duty, as was the
case with respect to the murder of the duke of Burgundy in 1419, and the
expressions of pain and pomp in all their exaggerated forms could well
be appropriate to the intensity of the mood. Chastellain deals profusely
with the aesthetics of the way the news of the duke's death was
transmitted; he invents the long speech, and the weighty and halting
style of its dignified rhetoric, with which, at Ghent, the bishop of
Tournay gradually prepares the young duke for the terrible news, and he
invents the dignified expressions of lament by Philip and his wife,
Michelle of France. But we have no reason to doubt the heart of his
report: that the news leads to a nervous breakdown on the part of the
young duke and that his wife, too, fell into a swoon. The terrible
confusion at court, the loud laments in the city---in short, all the
intense, unbridled pain with which the news was received---is not to be
doubted.\textsuperscript{\protect\hypertarget{09_Chapter_Two__THE_CRAVING_FOR_A_M.xhtmlux5cux23id_1968}{\protect\hyperlink{23_NOTES.xhtmlux5cux23id_1969}{60}}}
Chastellain's report about the expression of pain on the part of Charles
the Bold at the passing of Philip in 1467 has elements of truth in it.
In this instance the blow was less violent; the old and nearly childish
duke had deteriorated
\protect\hypertarget{09_Chapter_Two__THE_CRAVING_FOR_A_M.xhtmlux5cux23page_55}{}{}for
a long time. Relations between the duke and his son had been anything
but cordial during the last years. This prompted Chastellain himself to
remark that he was astonished to see Charles break down in tears, cry,
wring his hands and fall to the ground at the deathbed, ``et ne tenoit
règle, ne mesure, et tellement qu'il fit chacun s'esmerveiller de sa
démesurée
douleur.''\protect\hypertarget{09_Chapter_Two__THE_CRAVING_FOR_A_M.xhtmlux5cux23id_2451}{\protect\hyperlink{23_NOTES.xhtmlux5cux23id_2452}{*\textsuperscript{22}}}
In the city of Bruges where the duke had died, there too, ``estoit pitié
de oyr toutes manières de gens crier et plorer et faire leurs diverses
lamentaions et
regrets.''\textsuperscript{\protect\hypertarget{09_Chapter_Two__THE_CRAVING_FOR_A_M.xhtmlux5cux23id_1966}{\protect\hyperlink{23_NOTES.xhtmlux5cux23id_1967}{61}}}\protect\hypertarget{09_Chapter_Two__THE_CRAVING_FOR_A_M.xhtmlux5cux23id_2453}{\protect\hyperlink{23_NOTES.xhtmlux5cux23id_2454}{†\textsuperscript{23}}}

It is difficult to tell, from this and similar reports, how far the
court style went and how much a noisy display of suffering was
considered appropriate and beautiful and how profound the really intense
emotions, characteristic of these times, were. There is certainly still
a primitive element in it: the loud lament over the dead person that is
formalized in the cries of the hired women mourners, that becomes art in
the \emph{plourants}, and that bestows something so deeply moving upon
grave sculptures, particularly during this period, is a very ancient
cultural element.

The combination of primitivity, high sensitivity, and beautiful form can
also be sensed in the great fear of conveying the news of a death to a
great prince of the Middle Ages. The news of the death of her father is
kept from the duchess of Charolais as long as she is pregnant with the
future Mary of Burgundy; any news of a death remotely of concern to him
is kept from Philip the Good on his sickbed, which means, among other
things, that Adolf of Cleve is not permitted to wear mourning dress
after the death of his wife. When the duke managed to ``get wind''
(Chastellain himself uses the term ``avoit esté en vent un peu de ceste
mort'') of the death of his chancellor, Nicolas Rolin, he asked the
bishop of Tournay, who had come to see him, whether it was true that the
chancellor had died. ``Monseigneur,'' says the bishop, ``in truth he may
be dead because he is old and broken in body and spirit and will hardly
live for long.'' ``Déa!'' says the duke, ``I don't ask that, I ask
whether he is `mort de mort et
trespassé.''\protect\hypertarget{09_Chapter_Two__THE_CRAVING_FOR_A_M.xhtmlux5cux23id_2455}{\protect\hyperlink{23_NOTES.xhtmlux5cux23id_2456}{‡\textsuperscript{24}}}
``Well,
monsei\protect\hypertarget{09_Chapter_Two__THE_CRAVING_FOR_A_M.xhtmlux5cux23page_56}{}{}gneur,''
the bishop replies, ``he has not died, but he is paralyzed on one side
and is as good as dead.'' The duke gets angry, ``Vechy
merveilles!\protect\hypertarget{09_Chapter_Two__THE_CRAVING_FOR_A_M.xhtmlux5cux23id_2457}{\protect\hyperlink{23_NOTES.xhtmlux5cux23id_2458}{*\textsuperscript{25}}}
Tell me clearly now, whether he is dead.'' Only then does the bishop
admit: ``Yes, truly, monseigneur, he has really
died.''\textsuperscript{\protect\hypertarget{09_Chapter_Two__THE_CRAVING_FOR_A_M.xhtmlux5cux23id_1964}{\protect\hyperlink{23_NOTES.xhtmlux5cux23id_1965}{62}}}
Does not this strange way of conveying the news of a death reveal more
of an old superstitious form than mere consideration for a sick person
whom all this hesitation could only irritate? All this is part of that
sort of thinking that prompted Louis XI never again to wear clothes he
had worn when he received any kind of bad news; nor to ride again a
horse on which he had been mounted on one of those occasions. Indeed, he
even had a section of the Forest of Loches cut down because it was there
that he had learned of the death of his newborn
son.\textsuperscript{\protect\hypertarget{09_Chapter_Two__THE_CRAVING_FOR_A_M.xhtmlux5cux23id_1962}{\protect\hyperlink{23_NOTES.xhtmlux5cux23id_1963}{63}}}
``M. le chancellier,'' he writes on May 25, 1483, ``je vous mercye des
lettres etc. mais je vous pry que m'en envoyés plus par celluy qui les
m'a aportées, car je luy ay trouvé le visage terriblement changé depuis
que je ne le vitz, et vous prometz par ma foy qu'il m'a fait grant peur;
et
adieu.''\textsuperscript{\protect\hypertarget{09_Chapter_Two__THE_CRAVING_FOR_A_M.xhtmlux5cux23id_1960}{\protect\hyperlink{23_NOTES.xhtmlux5cux23id_1961}{64}}}\protect\hypertarget{09_Chapter_Two__THE_CRAVING_FOR_A_M.xhtmlux5cux23id_2459}{\protect\hyperlink{23_NOTES.xhtmlux5cux23id_2460}{†\textsuperscript{26}}}

No matter what old taboo notions may be hidden in the mourning customs,
their living cultural value is that they bestow on sorrow a form and
turn it into something beautiful and lofty. They bestow a rhythm on
pain, transpose real life into the sphere of drama and dress it in the
cothurnus.\textsuperscript{\protect\hypertarget{09_Chapter_Two__THE_CRAVING_FOR_A_M.xhtmlux5cux23id_1958}{\protect\hyperlink{23_NOTES.xhtmlux5cux23id_1959}{65}}}
In a primitive culture---I have, for example, the Irish in
mind---mourning customs and funeral poetry are still an unbroken whole.
Court mourning during Burgundian times can only be understood if viewed
in relation to elegy. The displays of mourning demonstrated in beautiful
form how totally powerless the affected individual is in the face of
suffering. The higher the rank the more heroic the display of pain. The
Queen of France had to stay an entire year in the room where she was
told of the death of her husband. In the case of princesses, six weeks
were the norm. After Madame de Charolais, Isabella de Bourbon, had been
told of the death of her father, she did attend the funeral at
Couwenberg Castle but thereafter remained for six
\protect\hypertarget{09_Chapter_Two__THE_CRAVING_FOR_A_M.xhtmlux5cux23page_57}{}{}weeks
in her room---all the time lying on her bed, propped up by pillows, but
clothed in \emph{barbette}, cap, and overcoat. The room is draped
entirely in black. On the floor is a large black sheet in place of a
soft carpet, and the antechamber is similarly draped in black. Noble
women are confined to bed solely for the death of their husband for six
weeks, only ten days for father or mother, but for the rest of the six
weeks they remain seated before the bed on a large sheet of black cloth.
The death of the eldest brother requires six weeks of confinement to a
room, but not to the
bed.\textsuperscript{\protect\hypertarget{09_Chapter_Two__THE_CRAVING_FOR_A_M.xhtmlux5cux23id_1956}{\protect\hyperlink{23_NOTES.xhtmlux5cux23id_1957}{66}}}
This makes it clear why, in a time that held this kind of high
ceremonial in such honor, one of the most mentioned of the shocking
circumstances surrounding the murder of John the Fearless in 1419 was
that he was buried dressed only in vest, trousers, and
shoes.\textsuperscript{\protect\hypertarget{09_Chapter_Two__THE_CRAVING_FOR_A_M.xhtmlux5cux23id_1954}{\protect\hyperlink{23_NOTES.xhtmlux5cux23id_1955}{67}}}

The emotion of grief, dressed in beautiful forms and assimilated in this
manner, is easily dealt with; the urge to dramatize life leaves room
``behind the scene'' where nobly embellished pathos can be denied. There
is a naive separation between ``state'' and real life that is revealed
in the writings of the old court lady Alienor de Poitiers, who still
venerates all these external displays as if they were high mysteries.
Following the description of Isabella of Bourbon's magnificent mourning
she declares, ``Quand Madame estoit en son particulier, elle n'estoit
point toujours couchee, ni en une
chambre.''\protect\hypertarget{09_Chapter_Two__THE_CRAVING_FOR_A_M.xhtmlux5cux23id_2461}{\protect\hyperlink{23_NOTES.xhtmlux5cux23id_2462}{*\textsuperscript{27}}}
The princess receives in this state, but only as a beautiful formality.
Alienor adds in a similar vein, ``It is proper to wear mourning clothes
for two years in memory of a husband if you can't avoid remarriage.''
Speedy remarriage was frequent, particularly among the highest estates,
the princes with the most famous names. The duke of Bedford, Regent of
France for the young Henry VI, remarried after only five months.

Next to mourning, confinement during childbirth offered ample
opportunities for serious pomp and hierarchical distinctions of
ostentation. There, colors have meaning. Green, which was the usual
color for the middle-class crib and the
\emph{vuurmand}\textsuperscript{\protect\hypertarget{09_Chapter_Two__THE_CRAVING_FOR_A_M.xhtmlux5cux23id_1953}{\protect\hyperlink{23_NOTES.xhtmlux5cux23page_404}{68}}}
as late as the nineteenth century, was in the fifteenth century the
prerogative of queens and princesses. The confinement room of the Queen
of France was of green silk (earlier on, it was entirely in white). Even
countesses were not permitted to have ``la chambre verde.'' Fabrics,
\protect\hypertarget{09_Chapter_Two__THE_CRAVING_FOR_A_M.xhtmlux5cux23page_58}{}{}furs,
and the colors of blankets and bedspreads were prescribed. On the
dressing table in the room of Isabella of Bourbon, two large candles in
silver holders burn continuously because the shutters of the lying-in
room are kept closed for fourteen days. Most remarkable however are the
stately beds that, like the carriages at the burial of the King of
Spain, remain empty. The young mother lies on a couch in front of the
fire and the child, Mary of Burgundy, in a cradle in the nursery. In
addition, in the confinement room there are two large beds in an
artistic ensemble with green drapes. They are made, with covers turned,
as if for someone to sleep in them. In the nursery, there are in
addition two large bedsteads in green and violet and still another large
bed in an antechamber, or \emph{chambre de parement}, that is entirely
draped in crimson-colored satin which was donated to John the Fearless
by the city of Utrecht; the room was therefore called ``la chambre
d'Utrecht.'' During the baptism ceremonies the beds served ceremonial
functions.\textsuperscript{\protect\hypertarget{09_Chapter_Two__THE_CRAVING_FOR_A_M.xhtmlux5cux23id_1951}{\protect\hyperlink{23_NOTES.xhtmlux5cux23id_1952}{69}}}

The aesthetics of formality revealed itself in the everyday look of town
and countryside: the strict hierarchy of fabrics, colors, and furs
placed the different estates in an eternal frame of reference that both
enhanced and protected a sense of their dignity. But the aesthetic of
emotional swings was not limited to festive rejoicing and sorrowing on
the occasions of birth, marriage, and death, where processions were a
function of the necessary ceremonies. Every ethical action was
preferably seen in terms of a beautifully embellished form. There is
such an element in the admiration for the humility and self-flagellation
of a saint, for the repentance of a sinner such as the ``moult belle
contrition de ses
Péchés''\protect\hypertarget{09_Chapter_Two__THE_CRAVING_FOR_A_M.xhtmlux5cux23id_2463}{\protect\hyperlink{23_NOTES.xhtmlux5cux23id_2464}{*\textsuperscript{28}}}
of Agnes
Sorel.\textsuperscript{\protect\hypertarget{09_Chapter_Two__THE_CRAVING_FOR_A_M.xhtmlux5cux23id_1949}{\protect\hyperlink{23_NOTES.xhtmlux5cux23id_1950}{70}}}
Every relationship in life is stylized. In contrast to the modern
preoccupation with hiding and obscuring intimate relations, medieval man
strove to express them as a form and as a spectacle for others. Thus
friendship had its elaborated form in the life of the fifteenth century.
Side by side with the older brotherhood of blood and the brotherhood of
arms, honored by commoners and nobles alike, a form of sentimental
friendship, known as ``mignon,''
existed.\textsuperscript{\protect\hypertarget{09_Chapter_Two__THE_CRAVING_FOR_A_M.xhtmlux5cux23id_1947}{\protect\hyperlink{23_NOTES.xhtmlux5cux23id_1948}{71}}}
The princely ``mignon'' is a formal institution that survived during all
of the sixteenth and part of the seventeenth century. The term applies
to the relationship between James I of England and Robert Carr and
George Villiers; William of Orange
\protect\hypertarget{09_Chapter_Two__THE_CRAVING_FOR_A_M.xhtmlux5cux23page_59}{}{}at
the time of the abdication of Charles V should be seen from this vantage
point. ``Twelfth Night'' can only be understood if we keep this
particular form of sentimental friendship in mind while considering the
behavior of the duke towards the pretender Cesario. This relationship is
a parallel to courtly love. ``Sy n'as dame ne mignon,'' says
Chastellain.\textsuperscript{\protect\hypertarget{09_Chapter_Two__THE_CRAVING_FOR_A_M.xhtmlux5cux23id_1945}{\protect\hyperlink{23_NOTES.xhtmlux5cux23id_1946}{72}}}
But any signs that would place it in the tradition of Greek friendship
fail to appear. The openness with which mignoncy is treated in an age
that was horrified by the \emph{crimen nefandum} silences any suspicion.
Bernardino of Siena holds up as models to his Italian compatriots, among
whom sodomy was widely
spread,\textsuperscript{\protect\hypertarget{09_Chapter_Two__THE_CRAVING_FOR_A_M.xhtmlux5cux23id_1943}{\protect\hyperlink{23_NOTES.xhtmlux5cux23id_1944}{73}}}
France and Germany, where it was unknown. Only princes who are very much
hated are on occasion charged with illicit relations with an official
favorite, as in the case of Richard II of England with Robert de
Vere.\textsuperscript{\protect\hypertarget{09_Chapter_Two__THE_CRAVING_FOR_A_M.xhtmlux5cux23id_1941}{\protect\hyperlink{23_NOTES.xhtmlux5cux23id_1942}{74}}}
Under these circumstances, mignonism is a harmless relationship, which
honors those so favored and is freely admitted by them. Commines himself
recounts how he had enjoyed the honor of having the distinction of
receiving royal favors from Louis XI and how he went about dressed like
him.\textsuperscript{\protect\hypertarget{09_Chapter_Two__THE_CRAVING_FOR_A_M.xhtmlux5cux23id_1939}{\protect\hyperlink{23_NOTES.xhtmlux5cux23id_1940}{75}}}
Because it is the clear mark of the relationship, the king always has at
his side a \emph{mignon en titre}, wearing the same dress as he, on whom
he may lean for support during
receptions.\textsuperscript{\protect\hypertarget{09_Chapter_Two__THE_CRAVING_FOR_A_M.xhtmlux5cux23id_1937}{\protect\hyperlink{23_NOTES.xhtmlux5cux23id_1938}{76}}}
Frequently two friends of same age but different rank also dress alike,
sleep in the same room, and occasionally even in the same
bed.\textsuperscript{\protect\hypertarget{09_Chapter_Two__THE_CRAVING_FOR_A_M.xhtmlux5cux23id_1935}{\protect\hyperlink{23_NOTES.xhtmlux5cux23id_1936}{77}}}
Such inseparable friendship exists between the young Gaston de Foix and
his bastard brother, which friendship came to such a tragic end; between
Louis of Orléans (then still of Touraine) and Pierre de
Craon,\textsuperscript{\protect\hypertarget{09_Chapter_Two__THE_CRAVING_FOR_A_M.xhtmlux5cux23id_1933}{\protect\hyperlink{23_NOTES.xhtmlux5cux23id_1934}{78}}}
between the young duke of Cleve and Jacques de Lalaing. In the same
mode, princesses have a trusted female friend, dressed like them and
called
\emph{mignonne}.\textsuperscript{\protect\hypertarget{09_Chapter_Two__THE_CRAVING_FOR_A_M.xhtmlux5cux23id_1931}{\protect\hyperlink{23_NOTES.xhtmlux5cux23id_1932}{79}}}

All these beautifully stylized life forms, serving the task of lifting
harsh reality into the sphere of noble harmonies, were part of the great
art of living without having any direct impact on art itself in the
narrow sense of the word. The forms of social etiquette with their
friendly appearance of unforced altruism and accommodating recognition
of others, the splendor of the court and court etiquette with all its
hieratic majesty and seriousness, the gay adornments of marriage and
confinement---all this passed in beauty without leaving any traces in
art and literature. The means of expression joining them to each other
is not art but fashion. Actually, fashion generally is much closer to
art than academic aesthetics are willing to
\protect\hypertarget{09_Chapter_Two__THE_CRAVING_FOR_A_M.xhtmlux5cux23page_60}{}{}admit.
As an artificial emphasis on physical beauty and movement, it is
intimately linked to one of the arts, i.e. dance, but in other respects
the realm of fashion, or better, that of ensemble fitted to an occasion,
borders in the fifteenth century much more directly on art than we are
inclined to assume. It does so not only because the frequent use of
jewels and the use of metals in the fashioning of the garb of warriors
added a direct craft element to costumes; fashion shares essential
qualities with art itself: style and rhythm are just as indispensable to
it as they are to art. During late medieval times, fashion in costume
constantly expressed a measure of the style of life that finds only a
pale reflection even in today's coronation festivities. In daily life
the differences in fur and color, cap and bonnet indicated the strict
order of the estates, the splendid dignities, states of joy and sadness,
and tender relations between friends and lovers.

The aesthetics of all of life's circumstances and life's conditions were
elaborated with the greatest possible emphasis. The higher the substance
of beauty and ethicality of these relationships, so much the better
could they be expressed as a true art. Courtliness and etiquette could
only express themselves in life itself, in clothing and jewelry.
Mourning, on the other hand, found still another emphatic means of
expression in a durable and powerful art form---the tomb monument; the
cultural value of mourning was enhanced by its relation to liturgy. But
still richer were the aesthetic flowers of the three elements of life:
courage, honor, and love.