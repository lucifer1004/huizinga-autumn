\chapter{THE PIOUS PERSONALITY}

THE PEOPLE USUALLY LIVED IN THE LACKADAISICAL corruption of an entirely
externalized religion. Their firm belief engendered both fear and
delight, but the ordinary religious form did not involve the
unsophisticated in any questions or spiritual struggles such as
Protestantism was destined to do. A comfortable lack of religious awe
and the complacencies of everyday life alternated with periods of the
most intense displays of the passionate piety that spasmodically seized
the people. The continuous contrast between the strong and weak states
of religious tension cannot be explained by dividing the herd into two
groups, the pious and children of the world, as if a part of the people
consistently led lives of strict religiosity while others were only
externally devout. Our perception of late medieval northern Dutch and
lower German pietism could easily lead us to such mistaken conclusions.
To be sure, pietist circles separated themselves from secular life in
the \emph{devotio moderna} of the Fraterhouses and the Windesheim
convents and among them sustained religious tension became normal, but
as pious people \emph{par excellence} they formed a contrast to the
large majority. France and the southern Netherlands, on the other hand,
hardly experienced this phenomenon in the form of a movement at all. Yet
here too, the emotions that were the basis of the \emph{devotio moderna}
had the same effect as in the calm lands along the Yssel. In the south,
such a formal separation from secular life never occurred; passionate
devotion remained a part of general religious life, but peaked, from
time to time, in more intense and shorter outbursts. In our own time the
same difference in temperament separates the Latin peoples from their
northern neighbors; those in the south accept contradictions more
readily. They feel less need to go the whole way and find it easier to
combine the easy skeptical attitude of daily life with the high
emotional stirrings of blessed moments.

\protect\hypertarget{14_Chapter_Seven__THE_PIOUS_PERSONA.xhtmlux5cux23page_204}{}{}The
low esteem in which the clergy was held, which throughout the Middle
Ages parallels the high veneration of the priestly estate, may be
explained in part as the result of the worldly behavior of the higher
clergy and the considerable loss in status of the lower clergy, or as
the result of old pagan instincts. The mind of the people, only
incompletely Christianized, had never quite lost its disgust for men who
were not allowed to fight and had to be chaste. Knightly pride, rooted
in courage and love, just like the crude mind of the people, rejected
the spiritual. The corruption of the clergy contributed its share. For
centuries the higher and lower estates alike had reveled in the figure
of the unchaste monk and debauched fat clergyman. A latent hatred of the
clergy had always existed. The more a preacher railed against the sins
of his estate, the greater his appeal to the
people.\textsuperscript{\protect\hypertarget{14_Chapter_Seven__THE_PIOUS_PERSONA.xhtmlux5cux23id_1021}{\protect\hyperlink{23_NOTES.xhtmlux5cux23id_1022}{1}}}
As soon as a preacher attacks the clergy, we are told by Bernardinus of
Siena, the audience is prone to forget everything else; there is no
better way to keep the service lively at times when the congregation
tends to get sleepy or uncomfortable because they are too warm or too
cold. Instantly all those in attendance become wide-awake and in good
spirits.\textsuperscript{\protect\hypertarget{14_Chapter_Seven__THE_PIOUS_PERSONA.xhtmlux5cux23id_1019}{\protect\hyperlink{23_NOTES.xhtmlux5cux23id_1020}{2}}}
While, on the one hand, the dramatic religious movement caused by the
itinerant popular preachers of the fourteenth and fifteenth centuries
originated in the revival of the mendicant orders, on the the other
hand, these same mendicants became the objects of ridicule because of
their dissolute lifestyle. The unworthy priest of popular literature
who, like a lowly servant, reads mass for three grooten or who serves as
father confessor on a regular retainer, ``pour absoudre du
tout,''\protect\hypertarget{14_Chapter_Seven__THE_PIOUS_PERSONA.xhtmlux5cux23id_3077}{\protect\hyperlink{23_NOTES.xhtmlux5cux23id_3078}{*\textsuperscript{1}}}
is usually a mendicant
monk.\textsuperscript{\protect\hypertarget{14_Chapter_Seven__THE_PIOUS_PERSONA.xhtmlux5cux23id_1017}{\protect\hyperlink{23_NOTES.xhtmlux5cux23id_1018}{3}}}
Molinet, who is otherwise very pious in every respect, expresses the
facile mockery directed at the mendicant orders in a New Year's wish:

\emph{Prions Dieu que les Jacobins}

\emph{Puissent manger les Augustins}

\emph{Et les Carmes soient pendus}

\emph{Des cordes des Frères
Menus}.\textsuperscript{\protect\hypertarget{14_Chapter_Seven__THE_PIOUS_PERSONA.xhtmlux5cux23id_1015}{\protect\hyperlink{23_NOTES.xhtmlux5cux23id_1016}{4}}}\protect\hypertarget{14_Chapter_Seven__THE_PIOUS_PERSONA.xhtmlux5cux23id_3079}{\protect\hyperlink{23_NOTES.xhtmlux5cux23id_3080}{†\textsuperscript{2}}}

\protect\hypertarget{14_Chapter_Seven__THE_PIOUS_PERSONA.xhtmlux5cux23page_205}{}{}The
dogmatic conception of poverty that was incorporated in the mendicant
orders was no longer intellectually satisfying. The formal symbolism of
poverty as a spiritual idea had been replaced by the issue of real
social misery. The new insight occurs towards the end of the fourteenth
century in England, where, earlier than in other countries, eyes were
opened to an appreciation of the economic factors in life. The author of
that strangely dreamy and misty poem \emph{The Vision Concerning Piers
Plowman} is the first to focus on the troubles of the hardworking masses
and, filled with hatred of the mendicants, of the idle, the wasteful,
the phony cripples, of the \emph{validx rnendicantes}, who were the bane
of the Middle Ages, to praise the sacred nature of ordinary labor. But
even in the highest theological circles, some, such as Pierre d'Ailly,
do not shy away from contrasting the \emph{vere pauperes}, the truly
poor, with the mendicant orders. There is no doubt that the serious
approach to faith taken by the \emph{devotio moderna} puts its adherents
somewhat in contrast to the mendicant orders.

All we hear of day-to-day religious life shows abrupt alternations of
nearly diametrically opposed extremes. The ridicule heaped on priests
and monks and the hatred felt for them are merely the opposite side of a
general and profound attachment and veneration. A naive perception of
religious duties gives way just as quickly to an excess of devotion. In
1437, upon the return of the French king to his capital, there was a
solemn service for the repose of the soul of the duke of
Armagnac,\textsuperscript{\protect\hypertarget{14_Chapter_Seven__THE_PIOUS_PERSONA.xhtmlux5cux23id_1013}{\protect\hyperlink{23_NOTES.xhtmlux5cux23id_1014}{5}}}
whose murder had begun the sad period of years just endured. The people
flock to witness the occasion, but there is disappointment because no
alms are distributed. The Burgher of Paris casually reports that nearly
four thousand of those in attendance would not have gone if they had
known that nothing was to be given out. ``Et le maudirent qui avant
prièrent pour
lui.''\textsuperscript{\protect\hypertarget{14_Chapter_Seven__THE_PIOUS_PERSONA.xhtmlux5cux23id_1011}{\protect\hyperlink{23_NOTES.xhtmlux5cux23id_1012}{6}}}\protect\hypertarget{14_Chapter_Seven__THE_PIOUS_PERSONA.xhtmlux5cux23id_2545}{\protect\hyperlink{23_NOTES.xhtmlux5cux23id_2546}{*\textsuperscript{3}}}
But these are the same Parisians who shed floods of tears at the
numerous processions and squirm at the words of itinerant preachers.
Ghillebert de Lannoy, when in Rotterdam, saw a riot calmed by a priest
who held up the Corpus
Domine.\textsuperscript{\protect\hypertarget{14_Chapter_Seven__THE_PIOUS_PERSONA.xhtmlux5cux23id_1009}{\protect\hyperlink{23_NOTES.xhtmlux5cux23id_1010}{7}}}

The great contradictions and the strong shifts in religious tension are
as well revealed in the lives of the educated as they are in the lives
of the ignorant masses. Religious illumination comes time and
\protect\hypertarget{14_Chapter_Seven__THE_PIOUS_PERSONA.xhtmlux5cux23page_206}{}{}again
with the force of a sudden blow. It is always a watered-down repetition
of the experience of St. Francis when he took the words of the gospels
to be direct orders. A knight heard the reading of the baptismal formula
for perhaps the twentieth time, but suddenly realized the full sanctity
and wonderful utility of the words and resolved to turn the Devil away,
without making the sign of the cross, merely by remembering his own
baptism.\textsuperscript{\protect\hypertarget{14_Chapter_Seven__THE_PIOUS_PERSONA.xhtmlux5cux23id_1007}{\protect\hyperlink{23_NOTES.xhtmlux5cux23id_1008}{8}}}---Le
Jouvencel witnesses a duel. The parties stand ready to swear the justice
of their cause on the Host. Suddenly the knight realizes the absolute
necessity that one of the oaths must be false, that one of the two must
of necessity be damned, and says, Don't swear. Fight for a stake of five
hundred schillings, but don't take an
oath.\textsuperscript{\protect\hypertarget{14_Chapter_Seven__THE_PIOUS_PERSONA.xhtmlux5cux23id_1005}{\protect\hyperlink{23_NOTES.xhtmlux5cux23id_1006}{9}}}

The piety of the upper crust, with their heavy load of excessive
ostentation and pleasure seeking, has, for that reason, something of a
forced quality, like that found in the piety of the people. Charles V of
France is wont to abandon a hunt just as it reaches its most exciting
moment in order to attend a
mass.\textsuperscript{\protect\hypertarget{14_Chapter_Seven__THE_PIOUS_PERSONA.xhtmlux5cux23id_1004}{\protect\hyperlink{23_NOTES.xhtmlux5cux23page_421}{10}}}
The young Anne of Burgundy, the bride of Bedford, the English regent in
conquered France, angers the Burgher of Paris on one occasion by
splashing excrement on a procession during one of her wild outings on
horseback. On another occasion, however, she leaves the gay festivities
of the court at midnight in order to hear matins with Celestine nuns.
Her early death was caused by an illness she contracted during a visit
to the poor sick in the Hotel
Dieu.\textsuperscript{\protect\hypertarget{14_Chapter_Seven__THE_PIOUS_PERSONA.xhtmlux5cux23id_1002}{\protect\hyperlink{23_NOTES.xhtmlux5cux23id_1003}{11}}}

The contrast between piety and sinfulness are found in their puzzling
extremes in the person of Louis d'Orléans, who, among the prominent
servants of luxury and indulgence, was the most overindulged and
passionate man in the world. He had even taken up witchcraft and refused
to
recant.\textsuperscript{\protect\hypertarget{14_Chapter_Seven__THE_PIOUS_PERSONA.xhtmlux5cux23id_1000}{\protect\hyperlink{23_NOTES.xhtmlux5cux23id_1001}{12}}}
This same Orléans is, nonetheless, so devout that he has a cell in the
regular dormitory of the Celestines where he participates in the
cloistered life, hears matins at midnight and, on occasion, mass five or
six times a
day.\textsuperscript{\protect\hypertarget{14_Chapter_Seven__THE_PIOUS_PERSONA.xhtmlux5cux23id_998}{\protect\hyperlink{23_NOTES.xhtmlux5cux23id_999}{13}}}
There is a cruel mixture of religion and crime in the life of Gilles de
Rais, who, in the middle of his murder of children at Machecoul,
sponsored a service in honor of the Blessed Innocents for the bliss of
his soul. He was astonished when his judges accused him of heresy. Many
join piety with less bloody sins; there are many examples of devout
worldliness: the barbaric Gaston Phébus, Count of Foix; the frivolous
King René; the refined Charles d'Orléans. John of Bavaria, most feared
and most ambitious, pays a visit in
\protect\hypertarget{14_Chapter_Seven__THE_PIOUS_PERSONA.xhtmlux5cux23page_207}{}{}disguise
to Lidwina van Schiedam, to consult about the state of his
soul.\textsuperscript{\protect\hypertarget{14_Chapter_Seven__THE_PIOUS_PERSONA.xhtmlux5cux23id_996}{\protect\hyperlink{23_NOTES.xhtmlux5cux23id_997}{14}}}
Jean Coustain, the traitorous servant of Philip the Good, a godless man
who hardly ever attended mass and never gave alms, when in the hands of
his executioner gave himself to God in a passionate plea voiced in his
coarse Burgundian
dialect.\textsuperscript{\protect\hypertarget{14_Chapter_Seven__THE_PIOUS_PERSONA.xhtmlux5cux23id_994}{\protect\hyperlink{23_NOTES.xhtmlux5cux23id_995}{15}}}

Philip the Good, himself, is one of the most striking examples of the
intertwining of piety and worldliness. This man of extravagant
festivities and numerous bastards, of political calculation, of
tremendous pride and rage, is an earnest pietist. He remains on his
knees long after mass is over. For four days a week, and during all the
vigils of Our Lady and the apostles, he fasts on bread and water.
Sometimes he does not eat anything until four in the afternoon. He gives
many alms, always
secretly.\textsuperscript{\protect\hypertarget{14_Chapter_Seven__THE_PIOUS_PERSONA.xhtmlux5cux23id_992}{\protect\hyperlink{23_NOTES.xhtmlux5cux23id_993}{16}}}
After the surprise attack on Luxembourg he remained so long after mass
immersed in his breviary and, after that, in special prayers of
thanksgiving, that his entourage, waiting on horseback because the
battle was not over, became impatient: the duke, they insisted, could
easily make up saying his Our Fathers at another time. Warned that delay
was dangerous, the duke responded merely, ``Si Dieu m'a donné victoire,
il la me
gardera.''\textsuperscript{\protect\hypertarget{14_Chapter_Seven__THE_PIOUS_PERSONA.xhtmlux5cux23id_990}{\protect\hyperlink{23_NOTES.xhtmlux5cux23id_991}{17}}}\protect\hypertarget{14_Chapter_Seven__THE_PIOUS_PERSONA.xhtmlux5cux23id_2547}{\protect\hyperlink{23_NOTES.xhtmlux5cux23id_2548}{*\textsuperscript{4}}}

We should not see hypocrisy or conceited bigotry in all this, but rather
a state of tension between two spiritual poles that is no longer
possible for the modern mind. For them, it is possible because of the
perfect dualism between the sinful world and the Kingdom of God. In the
medieval mind, all the higher, purer feelings were absorbed by religion
so that the natural and sensuous drives were bound to be consciously
rejected and allowed to sink to the level of sinful worldliness. Two
views of life took shape side by side in the medieval mind: the piously
ascetic view that pulled all ethical conceptions into itself and the
worldly mentality, completely left to the devil, that took revenge with
ever greater abandon. If one of the two dominates, then one encounters
either saints or dissolute sinners. As a rule, they remain in balance,
although the scales oscillate violently. One sees passionate human
beings come into view whose fully blooming sinfulness makes their
overflowing pity break out all the more vehemently.

When we observe how medieval poets compose the most pious songs of
praise alongside all kinds of profane and obscene pieces,
\protect\hypertarget{14_Chapter_Seven__THE_PIOUS_PERSONA.xhtmlux5cux23page_208}{}{}as
do so many poets, such as Deschamps, Antoine de la Salle, and Jean
Molinet, then we have even less cause to attribute these productions to
hypothetical periods of worldliness and introspection as we do in the
case of modern poets. The contradiction, no matter how incomprehensible
to us, must be accepted.

There occur bizarre blends of the love of ostentation and strong
devotion. The unrestrained desire to decorate and depict all aspects of
life and thought with colorful embellishments and forms is not limited
to the overburdening of religion with paintings, the work of the
goldsmith, and sculpture. Even spiritual life itself is occasionally
embellished because of the hunger for color and glamor. Brother Thomas
complains bitterly of all the luxury and ostentation, but the platform
from which he speaks has been draped by the people with the most
splendid tapestries that could be
found.\textsuperscript{\protect\hypertarget{14_Chapter_Seven__THE_PIOUS_PERSONA.xhtmlux5cux23id_988}{\protect\hyperlink{23_NOTES.xhtmlux5cux23id_989}{18}}}
Philipe de Mézières is the perfect type of the splendor-loving pietist.
He decided the most minute details of the clothing for the Order of the
Passion that he intended to found. The object of his dream resembles a
festival of color. The knights should wear red, green, scarlet, or azure
depending on their rank; the Grand Master, white. White was also the
color of the ceremonial dress. The cross should be red, the belts of
leather or silk with horn buckles and ornaments of gilded brass. Boots
were to be black and capes red. The dress of the brothers, servants,
priests, and women were also
described.\textsuperscript{\protect\hypertarget{14_Chapter_Seven__THE_PIOUS_PERSONA.xhtmlux5cux23id_986}{\protect\hyperlink{23_NOTES.xhtmlux5cux23id_987}{19}}}
Nothing came of the order; Philippe de Mézières remained all his life
the great dreamer of crusades and maker of plans. But in the cloister of
the Celestines in Paris he found the place that could satisfy him; as
strict as the order was, so the church and cloister sparkled with gold
and precious stones, a mausoleum for princes and
princesses.\textsuperscript{\protect\hypertarget{14_Chapter_Seven__THE_PIOUS_PERSONA.xhtmlux5cux23id_984}{\protect\hyperlink{23_NOTES.xhtmlux5cux23id_985}{20}}}
Christine de Pisan regarded this church as beauty perfected. Mézières
spent some time there as a lay brother, took part in the strict life of
the cloister, but remained at the same time in contact with the great
lords and artistic minds of his time; a mundane artistic counterpart to
Gerard
Groote.\textsuperscript{\protect\hypertarget{14_Chapter_Seven__THE_PIOUS_PERSONA.xhtmlux5cux23id_982}{\protect\hyperlink{23_NOTES.xhtmlux5cux23id_983}{21}}}
His princely friend Orléans was also attracted to this place, where he
found the moments of reflection that punctuated his debauched life, and
there too, he found his early grave. It is certainly no accident that
those two lovers of splendor Louis d'Orléans and his uncle Philip the
Bold of Burgundy chose as the places to indulge their love of art the
houses of the strictest cloistered orders, where the contrast between
the lives of the monks and the splendor of the decorations
\protect\hypertarget{14_Chapter_Seven__THE_PIOUS_PERSONA.xhtmlux5cux23page_209}{}{}could
be felt most strongly: Orléans in those of the Celestines, Burgundy in
those of the Carthusians at Champmol near Dijon.

Old King René discovered a hermit while on a hunt near Angers: a priest
who had given up his sinecure and lived on black bread and berries. The
king was moved by his virtue and had a hut and small chapel built for
him. For himself, he made a garden and built a modest garden house,
which he decorated with paintings and allegories. He frequently went to
``son cher ermitage de
Reculée''\protect\hypertarget{14_Chapter_Seven__THE_PIOUS_PERSONA.xhtmlux5cux23id_2549}{\protect\hyperlink{23_NOTES.xhtmlux5cux23id_2550}{*\textsuperscript{5}}}
to converse with his artists and
scholars.\textsuperscript{\protect\hypertarget{14_Chapter_Seven__THE_PIOUS_PERSONA.xhtmlux5cux23id_980}{\protect\hyperlink{23_NOTES.xhtmlux5cux23id_981}{22}}}
Is this medieval, is it Renaissance, or is it not eighteenth century?

A duke of Savoy becomes a hermit with a gilded belt, red cap, golden
cross, and good
wine.\textsuperscript{\protect\hypertarget{14_Chapter_Seven__THE_PIOUS_PERSONA.xhtmlux5cux23id_978}{\protect\hyperlink{23_NOTES.xhtmlux5cux23id_979}{23}}}

It is only a step from that devotional splendor to expressions of
hyperbolic humility, which in turn are themselves full-fledged
extravagance. Olivier de la Marche retained from his boyhood memories
the arrival of King Jacques de Bourbon of Naples, who, under the
influence of the saintly Colette, had renounced the world. The king,
shabbily dressed, was carried in a cart, ``telle sans aultre difference
que les civières en quoy l'on porte les fiens et les ordures
communement.'' Behind came an elegant courtly escort. ``Et ouys
racompter et dire''---says La Marche, full of admiration---``que en
toutes les villes où il venoit, il faisoit semblables entrées par
humilité.''\textsuperscript{\protect\hypertarget{14_Chapter_Seven__THE_PIOUS_PERSONA.xhtmlux5cux23id_976}{\protect\hyperlink{23_NOTES.xhtmlux5cux23id_977}{24}}}\protect\hypertarget{14_Chapter_Seven__THE_PIOUS_PERSONA.xhtmlux5cux23id_2551}{\protect\hyperlink{23_NOTES.xhtmlux5cux23id_2552}{†\textsuperscript{6}}}

Such picturesque self-deprecation is not found in the prescriptions,
recommended by many holy examples, for funerals, which are expected to
be fitting representations of the deceased's unworthiness. The holy
Pierre Thomas, bosom friend and spiritual teacher of Philippe de
Mézières, feeling his approaching death, had himself put in a sack, a
rope put around his neck and placed on the ground. This was his
imitation, much exaggerated, of St. Francis, who had himself put on the
ground as he lay dying. Bury me, said Pierre Thomas, in the entrance to
the choir, if possible, so that everyone will have to step on my body,
even goats and
dogs.\textsuperscript{\protect\hypertarget{14_Chapter_Seven__THE_PIOUS_PERSONA.xhtmlux5cux23id_974}{\protect\hyperlink{23_NOTES.xhtmlux5cux23id_975}{25}}}
Mézières, his admiring disciple, takes his turn at outdoing his master
in fantastic humility. A heavy iron chain is to be put around
\protect\hypertarget{14_Chapter_Seven__THE_PIOUS_PERSONA.xhtmlux5cux23page_210}{}{}his
neck during his last hours. As soon as he has given up the ghost, he is
to be dragged by his feet, naked, to the choir. There he is to be left
until his burial with his arms spread in the form of a cross, tied with
three ropes to a board that is to take the place of an expensively
ornamented coffin upon which someone might have been tempted to paint
his vain worldly motto, ``se Dieu l'eust tant hay qu'il fust mors ès
cours des princes de ce
monde.''\protect\hypertarget{14_Chapter_Seven__THE_PIOUS_PERSONA.xhtmlux5cux23id_2553}{\protect\hyperlink{23_NOTES.xhtmlux5cux23id_2554}{*\textsuperscript{7}}}
The board, covered with two ells of canvas or coarse black linen, is to
be dragged in the same manner to the burial pit into which the naked
body of the poor pilgrim is to be thrown as it is. A small grave marker
is to be erected. Only his good friend in God, Martin, and the executors
of his last will are to be notified of his death.

It is almost self-evident that a mind given so much to protocol and
ceremony and the ever fashioning of new plans with greater and greater
details would leave many testaments. There is no mention, in the later
documents, of the provisions of 1392 and when he died in 1405 he was
given an ordinary funeral, dressed in the garb of his beloved Celestine
order; there were two tomb inscriptions, which most likely were composed
by
him.\textsuperscript{\protect\hypertarget{14_Chapter_Seven__THE_PIOUS_PERSONA.xhtmlux5cux23id_972}{\protect\hyperlink{23_NOTES.xhtmlux5cux23id_973}{26}}}

To the ideal of holiness, one could almost say to the romanticism of
holiness, the fifteenth century did not yet contribute anything that
heralded the new age. Even the Renaissance did not change the ideal of
holiness. Unaffected by the strong currents guiding culture into new
paths, the ideal of holiness remained, both before and after the great
crisis of the Reformation, what it always was. The saint is as timeless
as the mystic. The types of saints in the Counter-Reformation are the
same as those of the later Middle Ages, and these do not differ in any
special way from those of the earlier Middle Ages. There are, in the one
or the other period, some who are great activists, saints of the fiery
word or the passionately inspired deed: including, on the one hand, such
as Ignatius Loyola, Francis Xavier, and Karl Borromeus; on the other,
Bernardine of Sienna, Vincent Ferrer, and John of Capistrano. These are
joined by the mystics who find rapture in contemplation, similar to the
types of saints found in Islam and Buddhism, Aloysius Gonzaga in the
sixteenth century; Francis de Paola, Colette, and Peter of Luxembourg in
the fourteenth and fifteenth centuries. Between
\protect\hypertarget{14_Chapter_Seven__THE_PIOUS_PERSONA.xhtmlux5cux23page_211}{}{}these
two types are all those who share something of both extremes; as a
matter of fact, they may on occasion even combine these extreme
characteristics in their highest degree.

It might even be possible to place the romanticism of saintliness on an
equal footing with the romanticism of knighthood; both arise from a need
to realize certain aspects of an ideal life form in the life of an
individual or in literature. It is remarkable that the romanticism of
holiness has at all times taken much more delight in the fantastically
exciting extremes of abstinence and humility than in great elevating
deeds of religious culture. Holiness is not attained by churchly social
service, no matter how great, but rather through wondrous piety. The
great energetic figures only gain a holy reputation when their deeds are
bathed in the glow of the supernatural. This rules out Nicholas of Cusa,
but not his fellow spirit, Denis the
Carthusian.\textsuperscript{\protect\hypertarget{14_Chapter_Seven__THE_PIOUS_PERSONA.xhtmlux5cux23id_970}{\protect\hyperlink{23_NOTES.xhtmlux5cux23id_971}{27}}}

In this context, it is of greatest interest for us to observe how the
circles of refined splendor, those circles that venerated the knightly
ideal and continued to do so after the Middle Ages were over, dealt with
the ideal of holiness. Though their contacts with this ideal form were
not so numerous, they did occur. The princely circles managed a few
times to produce a saint. One of these is Charles de Blois. On his
mother's side he sprang from the house of Valois and, through his
marriage with the heir of Brittany, Jeanne de Penthièvre, became
involved in a dispute about succession that took the greater part of his
life. Under the terms of his marriage contract, he was obligated to
adopt the coat of arms and battle cry of the dukedom. He found himself
confronted by another pretender, Jean de Montfort, and the ensuing
conflict over Brittany coincided with the beginning of the Hundred Years
War. The defense of Montfort's claim was one of the complications that
prompted Edward III to come to France. The count of Blois accepted
battle like a true knight and fought as well as the best leaders of his
time. Taken prisoner in 1347, just prior to the siege of Calais, he was
held in England until 1356. He resumed the fight for the dukedom in 1362
and was killed in 1364 near Aurai while fighting bravely at the side of
Bertrand du Guesclin and Beaumanoir.

This war hero, whose life differed in none of its external features from
those of so many princely pretenders and leaders of his time, had led a
life of strict austerity since the days of his youth. When he was a boy,
his father had kept him away from edifying books
\protect\hypertarget{14_Chapter_Seven__THE_PIOUS_PERSONA.xhtmlux5cux23page_212}{}{}because
such books would be inappropriate for someone of his calling. He slept
on straw on the ground next to the bed of his wife, and a hair shirt was
found under his armor at the time of his death in battle. He took
confession each evening before going to bed, because, as he said, no
Christian should go to sleep with his sins unforgiven. During his
captivity in London, he was wont to visit cemeteries and, on his knees,
recite the \emph{De profundis}. The Breton page whom he asked to recite
the responses refused, arguing that these locations were the burial
grounds of those who had killed his parents and friends and had burned
their houses.

After his liberation, he intends to walk barefoot from La Roche-Derrien,
where he began his imprisonment, to Tréguier, the site of a shrine of
St. Ives, the patron of Brittany, whose biography he had written while a
captive. The people hear about his plans and strew his path with straw
and blankets. The count of Blois, however, takes a different route and
ends up with feet so sore that he cannot walk for fifteen
weeks.\textsuperscript{\protect\hypertarget{14_Chapter_Seven__THE_PIOUS_PERSONA.xhtmlux5cux23id_968}{\protect\hyperlink{23_NOTES.xhtmlux5cux23id_969}{28}}}
Immediately following his death, his princely relatives, among them his
brother-in-law, Louis of Anjou, attempt to have him canonized. The
proceedings, which resulted in beatification, took place in Angers in
the year 1371.

The strange thing, if we can rely on Froissart, is that this same
Charles de Blois had a bastard. ``Là fu occis en bon couvenant li dis
messires Charles de Blois, le viaire sus ses ennemis, et uns siens filz
bastars qui s'appeloit messires Jehans de Blois, et pluiseur aultre
chevalier et escuier de
Bretagne.''\textsuperscript{\protect\hypertarget{14_Chapter_Seven__THE_PIOUS_PERSONA.xhtmlux5cux23id_966}{\protect\hyperlink{23_NOTES.xhtmlux5cux23id_967}{29}}}\protect\hypertarget{14_Chapter_Seven__THE_PIOUS_PERSONA.xhtmlux5cux23id_2555}{\protect\hyperlink{23_NOTES.xhtmlux5cux23id_2556}{*\textsuperscript{8}}}
Are we to reject this as an evident
falsehood?\textsuperscript{\protect\hypertarget{14_Chapter_Seven__THE_PIOUS_PERSONA.xhtmlux5cux23id_964}{\protect\hyperlink{23_NOTES.xhtmlux5cux23id_965}{30}}}
Or should we assume that the combination of piety and sensuality that
was present in figures such as Louis d'Orléans and Philip the Good was
even more noticeably present in the count de Blois?

Such a question does not arise about the life of another nobleman of
that time, Pierre of Luxembourg. This scion of the house of the Dukes of
Luxembourg, which during the fourteenth century held such a respectable
place in the German empire as well as in the courts of France and
Burgundy, is a fitting example of what William James calls ``the
under-witted
saint,''\textsuperscript{\protect\hypertarget{14_Chapter_Seven__THE_PIOUS_PERSONA.xhtmlux5cux23id_962}{\protect\hyperlink{23_NOTES.xhtmlux5cux23id_963}{31}}}
whose narrow mind can only exist in a fearfully closed-in little world
of pious thinking.
\protect\hypertarget{14_Chapter_Seven__THE_PIOUS_PERSONA.xhtmlux5cux23page_213}{}{}He
was born in 1369, not long before his father was killed in the fighting
near Baesweiler (1371) between Brabant and Geldern. His spiritual
history takes us back again to the cloister of the Celestines in Paris,
where the eight-year-old boy came in contact with Philippe de Mézières.
He was already overburdened with church offices as a mere boy, first
with different cathedral sinecures, and then, at the age of fifteen,
with the Bishopric of Metz and still later with a cardinalship. He died
in 1387, not yet eighteen, and Avignon immediately went to work to
secure his canonization. The most important authorities were pressed
into service for this task: The King of France issued the petition and
it was supported by the cathedral chapter of Paris and by the University
of Paris. During the proceedings of 1389 the greatest notables of France
appeared as witnesses: Pierre's brother André of Luxembourg, Louis of
Bourbon, and Enguerrand de Coucy. Owing to the negligence of the Avignon
pope, sainthood was not bestowed (beatitude was proclaimed in 1527), but
the veneration justified by the petition had been recognized long before
this and developed without interference. At the spot in Avignon where
the body was buried and where daily the most remarkable miracles were
reported, the king founded a Celestine monastery in imitation of the
monastery in Paris that was the preferred sanctuary of the princely
circles in those days. The dukes of Orléans, Berry and Burgundy, came to
lay the first stones for the
king.\textsuperscript{\protect\hypertarget{14_Chapter_Seven__THE_PIOUS_PERSONA.xhtmlux5cux23id_960}{\protect\hyperlink{23_NOTES.xhtmlux5cux23id_961}{32}}}
Pierre Salmon tells us how he heard mass in the chapel of the holy one a
few years
later.\textsuperscript{\protect\hypertarget{14_Chapter_Seven__THE_PIOUS_PERSONA.xhtmlux5cux23id_958}{\protect\hyperlink{23_NOTES.xhtmlux5cux23id_959}{33}}}

There is something pitiful about the image of the princely ascetic who
died so young, conveyed by the witnesses during the proceedings about
his canonization. Peter of Luxembourg was an unusually tall boy, sickly,
who even as a child knew nothing but the seriousness of a fearful and
strict faith. He reproached his little brother who had laughed, because
while it was written that Our Lord had cried, it was not recorded that
he ever laughed. ``Douls, courtois et debonnaire,'' Froissart calls him,
``vierge de son corps, moult large aumosnier. Le plus du jour et de la
nuit il estoit en orisons. En toute sa vye il n'y ot fors
humilité.''\textsuperscript{\protect\hypertarget{14_Chapter_Seven__THE_PIOUS_PERSONA.xhtmlux5cux23id_956}{\protect\hyperlink{23_NOTES.xhtmlux5cux23id_957}{34}}}\protect\hypertarget{14_Chapter_Seven__THE_PIOUS_PERSONA.xhtmlux5cux23id_2557}{\protect\hyperlink{23_NOTES.xhtmlux5cux23id_2558}{*\textsuperscript{9}}}
Initially his aristocratic elders attempted to make him give up his
world-renouncing plans.
\protect\hypertarget{14_Chapter_Seven__THE_PIOUS_PERSONA.xhtmlux5cux23page_214}{}{}When
he spoke about his desire to become an itinerant priest, he was told,
you are much too tall; everyone would instantly recognize you and you
wouldn't be able to stand the cold. How could you preach in favor of a
crusade? For a moment we hear the groundtone of that small, rigid mind:
``Je vous bien,'' says Peter, ``qu'on me veut faire venir de bonne voye
à la malvaise: certes, certes, si je m'y mets, je feray tant que tout le
monde parlera de
moy.''\protect\hypertarget{14_Chapter_Seven__THE_PIOUS_PERSONA.xhtmlux5cux23id_2559}{\protect\hyperlink{23_NOTES.xhtmlux5cux23id_2560}{*\textsuperscript{10}}}
Sire, responds Master Jean de Marche, his confessor, there is no one who
wants you to do evil, only good.

It is evident that the noble relatives begin to feel admiration and
pride about the case once the ascetic inclinations of the youngster
prove to be irrevocable. A saint, and such a young saint, of their kind
and dwelling among them! Try to imagine the poor sickly youth weighed
down by the burden of church offices, living in the midst of the
extravagant splendor and arrogance of the court life of Berry and
Burgundy, himself covered with dirt and parasites and always concerned
with his small miserable sins. Confession itself became a bad habit with
him. Every day he recorded his sins on a piece of paper and, when
prevented from doing so on a journey, he made up for it by long hours
spent recording sins after the travels were completed. He was observed
writing at night and checking his list by candlelight. He would get up
in darkness to take confession from one of his chaplains. Sometimes he
knocked in vain at the door of their chambers; they pretended to be
deaf. If admitted, he would read his sins from his note sheets. These
confessions increased from two or three times a week to twice a day as
he approached the end. During his final days his confessor was not
allowed to leave his side. He finally died of consumption and having
asked to be buried like a pauper, a whole box full of pieces of paper
was found on which the sins of his little life had been recorded day by
day.\textsuperscript{\protect\hypertarget{14_Chapter_Seven__THE_PIOUS_PERSONA.xhtmlux5cux23id_954}{\protect\hyperlink{23_NOTES.xhtmlux5cux23id_955}{35}}}

There is yet another case that provides evidence illuminating the
relationship between court circles and saintliness: the stay of Saint
Francis of Paola at the court of Louis XI. The particular type of
piousness of the king is so well known that there is no need to describe
it in detail at this point. Louis, ``qui achetoit la grace de
\protect\hypertarget{14_Chapter_Seven__THE_PIOUS_PERSONA.xhtmlux5cux23page_215}{}{}Dieu
et de la Vierge Marie á plus grans deniers que oncques ne fist
roy,''\textsuperscript{\protect\hypertarget{14_Chapter_Seven__THE_PIOUS_PERSONA.xhtmlux5cux23id_952}{\protect\hyperlink{23_NOTES.xhtmlux5cux23id_953}{36}}}\protect\hypertarget{14_Chapter_Seven__THE_PIOUS_PERSONA.xhtmlux5cux23id_2561}{\protect\hyperlink{23_NOTES.xhtmlux5cux23id_2562}{*\textsuperscript{11}}}
shows all the qualities of the most overt and complacent fetishism. His
veneration for relics and passion for pilgrimages and processions seems
to lack any of the higher impulses and any shadow of awed restraint. He
treats sacred objects as if they were expensive home remedies. The cross
of St. Laud that was kept in Angers had to be brought to Nantes for no
other purpose than to have an oath taken on
it.\textsuperscript{\protect\hypertarget{14_Chapter_Seven__THE_PIOUS_PERSONA.xhtmlux5cux23id_950}{\protect\hyperlink{23_NOTES.xhtmlux5cux23id_951}{37}}}
An oath on the cross of St. Laud counted more to Louis than any other
oath. When the \emph{connétable} of Saint Pol is called into the
presence of the king and asks the king to swear to his safety on the
cross of St Laud, the king responds, any oath but that
one.\textsuperscript{\protect\hypertarget{14_Chapter_Seven__THE_PIOUS_PERSONA.xhtmlux5cux23id_948}{\protect\hyperlink{23_NOTES.xhtmlux5cux23id_949}{38}}}
When his end, which he feared above all other things, approaches, the
most precious relics are sent to him from everywhere. The pope sends
him, among other things, the \emph{corporale} of St. Peter himself; even
the Great Turk offers a collection of relics that were still in
Constantinople. On the buffet next to the king's sickbed is the sacred
Ampoule itself, which had been brought from Reims, from whence it had
never been removed before. Some said that the king wanted to test the
efficacy of the container of ointment by having his whole body
salved.\textsuperscript{\protect\hypertarget{14_Chapter_Seven__THE_PIOUS_PERSONA.xhtmlux5cux23id_946}{\protect\hyperlink{23_NOTES.xhtmlux5cux23id_947}{39}}}
Such religious impulses are usually found only in the history of the
Merovingians.

It is hardly possible to draw a line between Louis's passion for
collecting exotic animals such as reindeer and elands and his passion
for precious relics. He corresponds with Lorenzo de'Medici about the
ring of Saint Zanobi, a local Florentine saint, and about an ``agnus
dei,'' the plant-like growth also known as ``agnus scythicus,'' which
was regarded as an exotic
rarity.\textsuperscript{\protect\hypertarget{14_Chapter_Seven__THE_PIOUS_PERSONA.xhtmlux5cux23id_945}{\protect\hyperlink{23_NOTES.xhtmlux5cux23page_422}{40}}}
In the strange household in the castle of Plessis lès Tours during
Louis's last days one could find pious intercessors and musicians
wandering about together. ``At this time the king had a large number of
musicians come with their strings and wood-winds. He provided quarters
for them in Saint-Cosme near Tours. Some 120 of them gathered there,
among them many shepherds from around Poitou. Sometimes they played in
front of the royal apartments, but without seeing the king. The king was
not only to enjoy the aforementioned instruments in order to pass the
time, they were also intended to keep him awake. He also summoned a
large number of bigots,
\protect\hypertarget{14_Chapter_Seven__THE_PIOUS_PERSONA.xhtmlux5cux23page_216}{}{}both
male and female, and devotees, hermits and saintly people, to come and
pray to God without interruption that the king might not die, but go on
living.''\textsuperscript{\protect\hypertarget{14_Chapter_Seven__THE_PIOUS_PERSONA.xhtmlux5cux23id_943}{\protect\hyperlink{23_NOTES.xhtmlux5cux23id_944}{41}}}

Even Saint Francis of Paola, the Calabrian hermit, who managed to outdo
the humility of the Minorites by founding the Minims, became, in a
literal sense, the object of Louis's collecting mania. During his final
illness, the king summoned the saint with the expressed intent that the
prayers of the saint might prolong his
life.\textsuperscript{\protect\hypertarget{14_Chapter_Seven__THE_PIOUS_PERSONA.xhtmlux5cux23id_941}{\protect\hyperlink{23_NOTES.xhtmlux5cux23id_942}{42}}}
After several messages to the King of Naples had not borne fruit, the
king, through diplomatically intervening with the pope, managed to
secure the arrival, very much against Francis's will, of the miracle
man. A noble entourage accompanied the monk from
Italy.\textsuperscript{\protect\hypertarget{14_Chapter_Seven__THE_PIOUS_PERSONA.xhtmlux5cux23id_939}{\protect\hyperlink{23_NOTES.xhtmlux5cux23id_940}{43}}}---But
when he arrived, Louis was not convinced of his authenticity, ``because
he had been cheated by several persons operating under the pretense of
saintliness.'' Following suggestions from his personal physician, he had
the holy man kept under surveillance and had his virtue tested in a
variety of
ways.\textsuperscript{\protect\hypertarget{14_Chapter_Seven__THE_PIOUS_PERSONA.xhtmlux5cux23id_937}{\protect\hyperlink{23_NOTES.xhtmlux5cux23id_938}{44}}}
The saint passed all tests with distinction. His asceticism was of the
most barbaric kind, reminiscent of the practices of his countrymen of
the tenth century, St. Niles and St. Romauld. He flees at the sight of a
woman. He has not touched a coin since he was a boy. He usually sleeps
standing up or leaning on something; he never has his hair cut or his
beard shaved. He never eats meat and is served only
roots.\textsuperscript{\protect\hypertarget{14_Chapter_Seven__THE_PIOUS_PERSONA.xhtmlux5cux23id_935}{\protect\hyperlink{23_NOTES.xhtmlux5cux23id_936}{45}}}
The king is still personally engaged during his last month in writing to
secure proper food for his strange holy man: ``Monsieur de Genas, je
vous prie de m'envoyer des citrons et des oranges douces et des poires
muscadelles et des pastenargues, et c'est pour le saint homme qui ne
mange ny chair ny poisson: et vous me ferés ung fort grant
plaisir.''\textsuperscript{\protect\hypertarget{14_Chapter_Seven__THE_PIOUS_PERSONA.xhtmlux5cux23id_933}{\protect\hyperlink{23_NOTES.xhtmlux5cux23id_934}{46}}}\protect\hypertarget{14_Chapter_Seven__THE_PIOUS_PERSONA.xhtmlux5cux23id_2319}{\protect\hyperlink{23_NOTES.xhtmlux5cux23id_2320}{*\textsuperscript{12}}}
He never refers to him other than as ``le saint homme,'' so that
Commines, who met the saint on several occasions, does not seem to have
known his
name.\textsuperscript{\protect\hypertarget{14_Chapter_Seven__THE_PIOUS_PERSONA.xhtmlux5cux23id_931}{\protect\hyperlink{23_NOTES.xhtmlux5cux23id_932}{47}}}
But he was also called ``le saint homme'' by those who ridiculed the
arrival of this weird guest or did not believe in his holiness, such as,
for instance, the king's physician, Jacques
Coitier.\textsuperscript{\protect\hypertarget{14_Chapter_Seven__THE_PIOUS_PERSONA.xhtmlux5cux23id_929}{\protect\hyperlink{23_NOTES.xhtmlux5cux23id_930}{48}}}
Commines couches his reports in terms of sober reservations. ``Il est
encores vif,'' he concludes, ``par quoy se pourrait bien changer ou en
my\protect\hypertarget{14_Chapter_Seven__THE_PIOUS_PERSONA.xhtmlux5cux23page_217}{}{}eulx
ou in pis, par quoy me tays, pour se que plusieurs se mocquoient de la
venue de ce hermite, qu'ilz appelloient `sainct
homme.'\,''\protect\hypertarget{14_Chapter_Seven__THE_PIOUS_PERSONA.xhtmlux5cux23id_2321}{\protect\hyperlink{23_NOTES.xhtmlux5cux23id_2322}{*\textsuperscript{13}}}
However, Commines himself testifies that no one had seen ``de si saincte
vie, ne où il semblast myeulx que le Sainct Esperit parlast par sa
bouche.''\protect\hypertarget{14_Chapter_Seven__THE_PIOUS_PERSONA.xhtmlux5cux23id_2323}{\protect\hyperlink{23_NOTES.xhtmlux5cux23id_2324}{†\textsuperscript{14}}}
And the learned theologians of Paris, Jan Standonck and Jean Quintin,
who had been dispatched to talk to the saintly man about founding a
convent of Minims in Paris, were most profoundly moved and returned to
Paris cured of their
prejudices.\textsuperscript{\protect\hypertarget{14_Chapter_Seven__THE_PIOUS_PERSONA.xhtmlux5cux23id_927}{\protect\hyperlink{23_NOTES.xhtmlux5cux23id_928}{49}}}

The interest the dukes of Burgundy take in the saints of their time is
less self-seeking than that displayed by Louis XI in Saint Francis of
Paola. It is noticeable that more than one of the great visionaries and
ascetics regularly appears as intermediary or adviser in political
matters. This is the case with St. Colette, the blessed Denis of Ryckel
and the Carthusian. Colette was treated by the house of Burgundy with
particular distinction; Philip the Good and his mother, Margarita of
Bavaria, knew her personally and sought her
advice.\textsuperscript{\protect\hypertarget{14_Chapter_Seven__THE_PIOUS_PERSONA.xhtmlux5cux23id_925}{\protect\hyperlink{23_NOTES.xhtmlux5cux23id_926}{50}}}
She negotiated complicated matters between the houses of France, Savoy,
and Burgundy. Charles the Bold, Mary and Maximilian, and Margaret of
Austria repeatedly pressed for her canonization. More important yet is
the role played by Denis the Carthusian in the public life of his time.
He, too, was in repeated contact with the house of Burgundy and acted as
adviser to Philip the Good. Along with Cardinal Nicholas of Cusa, whom
he had accompanied on his famous journey throughout the German empire,
he was received, in 1451, by the duke in Brussels. Denis, who is
constantly depressed by the feeling that things are going badly for the
church and Christendom and that the great calamity is imminent, asks in
a vision, Lord, will the Turks reach Rome? He reminds the duke of the
crusade.\textsuperscript{\protect\hypertarget{14_Chapter_Seven__THE_PIOUS_PERSONA.xhtmlux5cux23id_923}{\protect\hyperlink{23_NOTES.xhtmlux5cux23id_924}{51}}}
The ``inclytus devotus ac optimus princeps et
dux,''\protect\hypertarget{14_Chapter_Seven__THE_PIOUS_PERSONA.xhtmlux5cux23id_2325}{\protect\hyperlink{23_NOTES.xhtmlux5cux23id_2326}{‡\textsuperscript{15}}}
to whom he dedicates his tract on the princely life, cannot be anyone
other than Philip. Charles the Bold joins Denis in his efforts to found
a Carthusian house at
Hertogen\protect\hypertarget{14_Chapter_Seven__THE_PIOUS_PERSONA.xhtmlux5cux23page_218}{}{}bosch
in honor of St. Sophia of Constantinople, whom the duke understandably
regards as a saint whereas she is really the figure of eternal
wisdom.\textsuperscript{\protect\hypertarget{14_Chapter_Seven__THE_PIOUS_PERSONA.xhtmlux5cux23id_921}{\protect\hyperlink{23_NOTES.xhtmlux5cux23id_922}{52}}}
Duke Arnold of Geldern asks Denis for advice about his quarrel with his
son
Adolf.\textsuperscript{\protect\hypertarget{14_Chapter_Seven__THE_PIOUS_PERSONA.xhtmlux5cux23id_919}{\protect\hyperlink{23_NOTES.xhtmlux5cux23id_920}{53}}}

Not only princes, but also numerous noblemen, clerics, and burghers came
for advice to his cell at Roermond; he was constantly busy resolving
innumerable difficulties, doubts and questions of conscience.

Denis the Carthusian is the perfect type of the powerful religious
enthusiast produced by the waning Middle Ages. His life was incredibly
energetic; he combined the ecstasies of the great mystics, the wildest
asceticism, the continuous visions and revelations of a spiritual seer
with a vast activity as a theological writer and practical spiritual
adviser. He was as close to the great mystics as he was to the practical
Windesheimers such as Brugman, for whom he writes his famous guide for
the Christian
life,\textsuperscript{\protect\hypertarget{14_Chapter_Seven__THE_PIOUS_PERSONA.xhtmlux5cux23id_917}{\protect\hyperlink{23_NOTES.xhtmlux5cux23id_918}{54}}}
or to Nicholas of Cusa or even to the witch hunters or those who
enthusiastically labored for the abolition of clerical
abuses.\textsuperscript{\protect\hypertarget{14_Chapter_Seven__THE_PIOUS_PERSONA.xhtmlux5cux23id_915}{\protect\hyperlink{23_NOTES.xhtmlux5cux23id_916}{55}}}
His energies must have been inexhaustible. His writings fill forty-five
quarto volumes. It is as if through him the entire stream of medieval
theology flows once again. ``Qui Dionysium legit, nihil non
legit''\protect\hypertarget{14_Chapter_Seven__THE_PIOUS_PERSONA.xhtmlux5cux23id_2327}{\protect\hyperlink{23_NOTES.xhtmlux5cux23id_2328}{*\textsuperscript{16}}}
was said by the theologians of the sixteenth century. Responding to a
request from an old lay brother, Willem, he writes about the mutual
recognition of souls in the hereafter with the same touch with which he
handles the most profound questions of a philosophical nature. He
promises Brother Willem that he will write as simply as possible and
says that Willem can translate it into
Dutch.\textsuperscript{\protect\hypertarget{14_Chapter_Seven__THE_PIOUS_PERSONA.xhtmlux5cux23id_913}{\protect\hyperlink{23_NOTES.xhtmlux5cux23id_914}{56}}}
Everything his great predecessors had thought, he expresses in an
endless flood of simply expressed thoughts. It has all the
characteristics of a late work: summarizing, concluding, breaking no new
ground. The quotations from Bernard of Clairvaux or Hugo of Saint Victor
sparkle like jewels on the simple unicolor garment of Denis's prose. All
of his works were written, proofread, improved, indexed, and illuminated
by himself until at the end of his life he ended his writing with a
well-chosen quotation: ``Ad securae taciturnitatis portum me transferre
intendo,---I will go now to the haven of secure
taciturnity.''\textsuperscript{\protect\hypertarget{14_Chapter_Seven__THE_PIOUS_PERSONA.xhtmlux5cux23id_911}{\protect\hyperlink{23_NOTES.xhtmlux5cux23id_912}{57}}}

He knew no rest. Daily he recited nearly all the Psalms; at least
\protect\hypertarget{14_Chapter_Seven__THE_PIOUS_PERSONA.xhtmlux5cux23page_219}{}{}half
are necessary, he declares. During every activity, dressing and
undressing, he prays. After midnight mass, when others go to rest, he
remains awake. He is strong and tall and his body can withstand
everything. I have an iron head and a copper stomach, he says. Without
disgust, indeed by preference, he enjoys spoiled food, such as wormy
butter, cherries partially consumed by snails; these kinds of parasites
have no deadly poisons, he says, one can eat them with confidence. He
hangs oversalted herrings out until they rot; I would rather eat food
that stinks than that which is too
salty.\textsuperscript{\protect\hypertarget{14_Chapter_Seven__THE_PIOUS_PERSONA.xhtmlux5cux23id_909}{\protect\hyperlink{23_NOTES.xhtmlux5cux23id_910}{58}}}

He accomplishes the entire mental work of the deepest philosophical
speculation and definition, not in the context of an even-tempered and
undisturbed scholarly life, but with a mind subject to the constant
upheavals of receptiveness to every dramatic stirring of the
supernatural. As a boy he got up by the light of the moon because he
thought it was time to go to
school.\textsuperscript{\protect\hypertarget{14_Chapter_Seven__THE_PIOUS_PERSONA.xhtmlux5cux23id_907}{\protect\hyperlink{23_NOTES.xhtmlux5cux23id_908}{59}}}
He stutters; he is called ``Taterbek'' by a devil whom he tried to
exorcise. He sees that the room of the dying Lady of Vlodrop is full of
devils; they knock his stick out of his hand. No one has experienced the
dread of the ``Four Last Things'' to the extent he has. The violent
attacks of the devil upon the dying is a repeated subject of his
sermons. He constantly communicates with the deceased. Have spirits
often appeared to him? asks one of the brothers. O, hundreds and
hundreds of times, he answers. He sees his father in Purgatory and
resists the impulse to free him. He is constantly confronted by
apparitions, revelations, and visions, but is reluctant to speak about
them. He is ashamed of the ecstasies he experiences as a result of
external stimuli: above all music, which sometimes seizes him in the
midst of a noble gathering listening to his wisdom and exhortations.
Among the honorary names of the great theologians, his is Doctor
Ecstaticus.

We should not think that such a great figure as Denis the Carthusian was
spared the sort of suspicions and ridicule heaped on the strange miracle
man of Louis XI. He, too, had to wage a constant battle against the
denunciations and mockeries of the world. In the mentality of the
fifteenth century we already see the stirrings of resentment and
rejection of the highest expressions of medieval faith; stirrings that
exist side by side with unrestrained devotion and enthusiasm.
