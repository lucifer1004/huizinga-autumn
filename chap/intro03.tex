\chapter{PREFACE TO THE GERMAN TRANSLATION}

THE NEED TO BETTER UNDERSTAND THE ART OF THE van Eyck brothers and that
of their successors and to view these artists in the context of the life
of their time provided the first impetus for this book. But a different,
in many respects more comprehensive image emerged during the course of
the investigation. It became evident that the fourteenth and fifteenth
centuries in France and in the Netherlands in particular are much more
suited to give us a sense of the end of the Middle Ages and of the last
manifestation of medieval culture than they are to demonstrate to us the
awakening Renaissance.

Our minds prefer to concern themselves with ``origins'' and
``beginnings.'' In most instances the promise that ties one age to its
successor appears to be more important than the memories that link it to
its predecessor. As a result, the search to find the first sprouts of
modern culture in medieval culture was carried out so eagerly and to the
point that the term medieval period itself came to be questioned and it
appeared as if this epoch was barely something other than the age that
ushered in the Renaissance. But dying and becoming keep just as much
pace with each other in history as in nature. To trace the vanishing of
overripe cultural forms is not less significant---and by no means less
fascinating---than to trace the arising of new forms. We do more
justice, not only to artists like the van Eycks, but also to {[}poets
such as{]} Eustache Deschamps, historiographers such as Froissart and
Chastellain, theologians such as Jean de Gerson and Denis the
Carthusian, and to all representatives of the spirit of this age if we
view them not as initiating and heralding what is to come, but rather as
completing the forms of an age in its final stage.

The author was, at the time he wrote this book, less aware than now of
the danger of comparing historical periods to the seasons of the year;
he asks therefore that the title of the book be taken
\protect\hypertarget{07_PREFACE_TO_THE_GERMAN_TRANSLATIO.xhtmlux5cux23page_xxii}{}{}only
as a figure of speech that is intended to capture the general mood of
the whole.

The translation follows exactly the second revised Dutch edition of 1921
(the first appeared in 1919). If the German tongue still tastes in
places the flavor of the Dutch original, we should remind ourselves that
a translation in the strict sense of the word is an impossibility even
in so closely related languages such as German and Dutch. Why should we
be so eager to obliterate fearfully the traces of what is foreign in
that which is of foreign origin?

Many have supported this work of translation in a valuable way. We owe a
debt of gratitude, next to the translator, primarily to our friends
Prof. André Jolles (Leipzig), Prof. W. Vogelsang (Utrecht), and Paul
Lehman (Munich). My sincere expression of thanks for his valuable
contribution to this work go to Prof. Eugene Lerch, who took it upon
himself to translate the French quotations found in the appended
section.

Leiden

November 1923