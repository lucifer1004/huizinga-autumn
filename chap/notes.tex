\chapter{NOTES}

\textbf{\emph{Translator's Introduction}}

\protect\hypertarget{23_NOTES.xhtmlux5cux23id_2248}{\protect\hyperlink{05_TRANSLATOR_S_INTRODUCTION.xhtmlux5cux23id_2247}{1}}.
As Professor Weintraub characterizes it in his \emph{Visions of Culture}
(Chicago: The University of Chicago Press, 1966), p. 212.

\protect\hypertarget{23_NOTES.xhtmlux5cux23id_2246}{\protect\hyperlink{05_TRANSLATOR_S_INTRODUCTION.xhtmlux5cux23id_2245}{2}}.
Johan Huizinga, \emph{The Waning of the Middle Ages}, trans. F. Hopman
(Garden City, N.Y.: Doubleday and Company, Inc., 1954).

\protect\hypertarget{23_NOTES.xhtmlux5cux23id_2244}{\protect\hyperlink{05_TRANSLATOR_S_INTRODUCTION.xhtmlux5cux23id_2243}{3}}.{~~~~~~~~},
1989. \emph{Briefwisseling I 1894--1924} (Veen: Tjeenkwillink).

\protect\hypertarget{23_NOTES.xhtmlux5cux23id_2242}{\protect\hyperlink{05_TRANSLATOR_S_INTRODUCTION.xhtmlux5cux23id_2241}{4}}.{~~~~~~~~},
1972. \emph{America: A Dutch Historian's Vision, from Afar and Near},
trans. and ed. Herbert H. Rowen (New York: Harper \& Row). This volume
contains both \emph{Man and the Masses in America} and \emph{Life and
Thought in America}.

\protect\hypertarget{23_NOTES.xhtmlux5cux23id_2240}{\protect\hyperlink{05_TRANSLATOR_S_INTRODUCTION.xhtmlux5cux23id_2239}{5}}.
Pesch, A. J. van, 1932. ``Levensbericht van F. J. Hopman,'' in
\emph{Handelingen en Levensberichten van de Maatschappij der
Nederlandsche Letterkunde} (Leiden: E. J. Brill), pp. 177--92.

\protect\hypertarget{23_NOTES.xhtmlux5cux23id_2238}{\protect\hyperlink{05_TRANSLATOR_S_INTRODUCTION.xhtmlux5cux23id_2237}{6}}.
\emph{In Johan Huizinga 1872--1972. Papers delivered to the Johan
Huizinga Conference Groningen 11--13 December 1972}, eds. W. H. R. Coops
\emph{et. al}. (The Hague: Martinus Nijhoff, 1973), pp. 91--103.

\protect\hypertarget{23_NOTES.xhtmlux5cux23id_2236}{\protect\hyperlink{05_TRANSLATOR_S_INTRODUCTION.xhtmlux5cux23id_2235}{7}}.
As we have translated the book's title. We refer to the Hopman
translation as \emph{Waning}.

\protect\hypertarget{23_NOTES.xhtmlux5cux23id_2234}{\protect\hyperlink{05_TRANSLATOR_S_INTRODUCTION.xhtmlux5cux23id_2233}{8}}.
Available in English as \emph{My Path to History}, in \emph{Dutch
Civilization in the Seventeenth Century and Other Essays}, ed. Pieter
Geyl and F. W. N. Hugenholtz, trans. Arnold J. Pomerans (New York:
Frederick Unger Publishing Co., 1968), pp. 244--76. This translation,
too, leaves much to be desired.

\protect\hypertarget{23_NOTES.xhtmlux5cux23id_2232}{\protect\hyperlink{05_TRANSLATOR_S_INTRODUCTION.xhtmlux5cux23id_2231}{9}}.
\emph{In the Shadow of Tomorrow}, trans. J. H. Huizinga (New York: W. W.
Norton \& Company, Inc. n.d.).

\protect\hypertarget{23_NOTES.xhtmlux5cux23id_2230}{\protect\hyperlink{05_TRANSLATOR_S_INTRODUCTION.xhtmlux5cux23id_2229}{10}}.
\emph{Homo Ludens: A Study of the Play-Element in Culture} (Boston: The
Beacon Press, 1955).

\textbf{\emph{Chapter 1}}

\protect\hypertarget{23_NOTES.xhtmlux5cux23id_2304}{\protect\hyperlink{08_Chapter_One__THE_PASSIONATE_INTE.xhtmlux5cux23id_2303}{*\textsuperscript{1}}}
``which is hideous to hear''

\protect\hypertarget{23_NOTES.xhtmlux5cux23id_2251}{\protect\hyperlink{08_Chapter_One__THE_PASSIONATE_INTE.xhtmlux5cux23id_2252}{*\textsuperscript{2}}}
``the most touching processions that had been seen in the memory of
men''

\protect\hypertarget{23_NOTES.xhtmlux5cux23id_2249}{\protect\hyperlink{08_Chapter_One__THE_PASSIONATE_INTE.xhtmlux5cux23id_2250}{†\textsuperscript{3}}}
``with great weeping, with many tears, with great devotion.''

\protect\hypertarget{23_NOTES.xhtmlux5cux23id_2255}{\protect\hyperlink{08_Chapter_One__THE_PASSIONATE_INTE.xhtmlux5cux23id_2256}{‡\textsuperscript{4}}}
``and he so touched their hearts that everyone burst into tears and his
death was commended as the finest that was ever seen.''

\protect\hypertarget{23_NOTES.xhtmlux5cux23id_2253}{\protect\hyperlink{08_Chapter_One__THE_PASSIONATE_INTE.xhtmlux5cux23id_2254}{*\textsuperscript{5}}}
``There was a great multitude of people there, almost all of whom wept
hot tears.''

\protect\hypertarget{23_NOTES.xhtmlux5cux23id_2258}{\protect\hyperlink{08_Chapter_One__THE_PASSIONATE_INTE.xhtmlux5cux23id_2257}{†\textsuperscript{6}}}
``to the bread of adversity, to the water of affliction'' (Isaiah
30:20).

\protect\hypertarget{23_NOTES.xhtmlux5cux23id_2261}{\protect\hyperlink{08_Chapter_One__THE_PASSIONATE_INTE.xhtmlux5cux23id_2259}{‡\textsuperscript{7}}}
``in the style of the members of parliament''

\protect\hypertarget{23_NOTES.xhtmlux5cux23id_2263}{\protect\hyperlink{08_Chapter_One__THE_PASSIONATE_INTE.xhtmlux5cux23id_2260}{*\textsuperscript{8}}}
``the people, great and small, wept from the bottom of their hearts as
if they were watching their best friends being put into the ground, and
so did he.''

\protect\hypertarget{23_NOTES.xhtmlux5cux23id_2264}{\protect\hyperlink{08_Chapter_One__THE_PASSIONATE_INTE.xhtmlux5cux23id_2262}{†\textsuperscript{9}}}
``sobbing and crying loudly at his departure.''

\protect\hypertarget{23_NOTES.xhtmlux5cux23id_2266}{\protect\hyperlink{08_Chapter_One__THE_PASSIONATE_INTE.xhtmlux5cux23id_2267}{*\textsuperscript{10}}}
``Meanwhile, they behaved like snails who pull in their horns when
people come near and put them out again when they don't hear anything
anymore. After the said preacher had left the neighborhood, they began,
in a very short space of time, to behave as before and gradually to
resume wearing their old finery as large or larger than they had been.''

\protect\hypertarget{23_NOTES.xhtmlux5cux23id_2268}{\protect\hyperlink{08_Chapter_One__THE_PASSIONATE_INTE.xhtmlux5cux23id_2269}{†\textsuperscript{11}}}
``dressed in the deepest mourning, most pitiful to see; and because of
the great sorrow and grief they showed at the death of their said
master, many tears were shed and lamentations uttered throughout the
said town.''

\protect\hypertarget{23_NOTES.xhtmlux5cux23id_2274}{\protect\hyperlink{08_Chapter_One__THE_PASSIONATE_INTE.xhtmlux5cux23id_2271}{‡\textsuperscript{12}}}
``And God knows what doleful and piteous plaints they made, mourning for
their master.''

\protect\hypertarget{23_NOTES.xhtmlux5cux23id_2306}{\protect\hyperlink{08_Chapter_One__THE_PASSIONATE_INTE.xhtmlux5cux23id_2305}{*\textsuperscript{13}}}
``and even the wisest would lose his patience.''

\protect\hypertarget{23_NOTES.xhtmlux5cux23id_2272}{\protect\hyperlink{08_Chapter_One__THE_PASSIONATE_INTE.xhtmlux5cux23id_2273}{*\textsuperscript{14}}}
``Then were heard voices and cries and tears flowed and with one accord
they shouted: `We all, we all, my lord, will live and die with you.'\,''

\protect\hypertarget{23_NOTES.xhtmlux5cux23id_2276}{\protect\hyperlink{08_Chapter_One__THE_PASSIONATE_INTE.xhtmlux5cux23id_2277}{*\textsuperscript{15}}}
``So stay then and suffer, and I will suffer for you, rather than see
you in want.''

\protect\hypertarget{23_NOTES.xhtmlux5cux23id_2280}{\protect\hyperlink{08_Chapter_One__THE_PASSIONATE_INTE.xhtmlux5cux23id_2275}{†\textsuperscript{16}}}
``says the one, `I have a thousand,' the other, `Ten thousand,' the
third, `I have this or that to put at your service and I am willing to
share all that might befall you.'\,''

\protect\hypertarget{23_NOTES.xhtmlux5cux23id_2278}{\protect\hyperlink{08_Chapter_One__THE_PASSIONATE_INTE.xhtmlux5cux23id_2279}{*\textsuperscript{17}}}
``Good morning, your majesty, good morning; and what is this? Are you
playing at King Arthur, now, or is it Sir Lancelot?''

\protect\hypertarget{23_NOTES.xhtmlux5cux23id_2282}{\protect\hyperlink{08_Chapter_One__THE_PASSIONATE_INTE.xhtmlux5cux23id_2283}{*\textsuperscript{18}}}
``who, half reluctantly and with regret, took a Scottish groat out of
his purse and lent it to her.''

\protect\hypertarget{23_NOTES.xhtmlux5cux23id_2285}{\protect\hyperlink{08_Chapter_One__THE_PASSIONATE_INTE.xhtmlux5cux23id_2281}{†\textsuperscript{19}}}
``A certain little treatise on fortune, based on its inconstancy and
deceptive nature.''

\protect\hypertarget{23_NOTES.xhtmlux5cux23id_2289}{\protect\hyperlink{08_Chapter_One__THE_PASSIONATE_INTE.xhtmlux5cux23id_2284}{‡\textsuperscript{20}}}
``a duke and a count and ten men, all on horseback.''

\protect\hypertarget{23_NOTES.xhtmlux5cux23id_2287}{\protect\hyperlink{08_Chapter_One__THE_PASSIONATE_INTE.xhtmlux5cux23id_2288}{*\textsuperscript{21}}}
``by the art of magic or in other ways.''

\protect\hypertarget{23_NOTES.xhtmlux5cux23id_2291}{\protect\hyperlink{08_Chapter_One__THE_PASSIONATE_INTE.xhtmlux5cux23id_2286}{†\textsuperscript{22}}}
``For princes are men, and their affairs are high and dangerous, and
their natures are subject to many passions such as hatred and envy and
their hearts are veritable dwelling places for these because of their
pride in reigning.''

\protect\hypertarget{23_NOTES.xhtmlux5cux23id_2292}{\protect\hyperlink{08_Chapter_One__THE_PASSIONATE_INTE.xhtmlux5cux23id_2290}{*\textsuperscript{23}}}
``he, who to avenge the outrage done to the person of the duke Jean
sustained the war for sixteen years.''

\protect\hypertarget{23_NOTES.xhtmlux5cux23id_2295}{\protect\hyperlink{08_Chapter_One__THE_PASSIONATE_INTE.xhtmlux5cux23id_2296}{†\textsuperscript{24}}}
``in the most violent and deadly rage he would give himself up to
revenge the dead, in so far as God would permit him, and he would risk
body and soul, possessions and lands, staking everything on the game and
on inconstant fortune, because he considered it more salutary and
agreeable to God to undertake the task than to leave it.''

\protect\hypertarget{23_NOTES.xhtmlux5cux23id_2293}{\protect\hyperlink{08_Chapter_One__THE_PASSIONATE_INTE.xhtmlux5cux23id_2294}{*\textsuperscript{25}}}
``in defiance of him.''

\protect\hypertarget{23_NOTES.xhtmlux5cux23id_2386}{\protect\hyperlink{08_Chapter_One__THE_PASSIONATE_INTE.xhtmlux5cux23id_2385}{†\textsuperscript{26}}}
``because they said I was a schismatic and believed in Benedict, the
antipope.''

\protect\hypertarget{23_NOTES.xhtmlux5cux23id_2388}{\protect\hyperlink{08_Chapter_One__THE_PASSIONATE_INTE.xhtmlux5cux23id_2387}{*\textsuperscript{27}}}
``at which the people were more delighted than if a new holy body had
been resurrected.''

\protect\hypertarget{23_NOTES.xhtmlux5cux23id_2390}{\protect\hyperlink{08_Chapter_One__THE_PASSIONATE_INTE.xhtmlux5cux23id_2389}{*\textsuperscript{28}}}
``and there was a great deal of laughter because they were all poor
men.''

\protect\hypertarget{23_NOTES.xhtmlux5cux23id_2392}{\protect\hyperlink{08_Chapter_One__THE_PASSIONATE_INTE.xhtmlux5cux23id_2391}{*\textsuperscript{29}}}
``the female dwarf of Mademoiselle of Burgundy''

\protect\hypertarget{23_NOTES.xhtmlux5cux23id_2394}{\protect\hyperlink{08_Chapter_One__THE_PASSIONATE_INTE.xhtmlux5cux23id_2393}{†\textsuperscript{30}}}
``to the father of Belon, the fool, who came to visit his daughter .~.~.
''

\protect\hypertarget{23_NOTES.xhtmlux5cux23id_2396}{\protect\hyperlink{08_Chapter_One__THE_PASSIONATE_INTE.xhtmlux5cux23id_2395}{‡\textsuperscript{31}}}
``to chain up Belon, the fool, and the other to put around the neck of
the monkey of her grace, the Duchess.''

\protect\hypertarget{23_NOTES.xhtmlux5cux23id_2398}{\protect\hyperlink{08_Chapter_One__THE_PASSIONATE_INTE.xhtmlux5cux23id_2397}{*\textsuperscript{32}}}
``as if it had been washed in rose water.''

\protect\hypertarget{23_NOTES.xhtmlux5cux23id_2400}{\protect\hyperlink{08_Chapter_One__THE_PASSIONATE_INTE.xhtmlux5cux23id_2399}{*\textsuperscript{33}}}
``hot-tempered Picardy''

\protect\hypertarget{23_NOTES.xhtmlux5cux23id_2402}{\protect\hyperlink{08_Chapter_One__THE_PASSIONATE_INTE.xhtmlux5cux23id_2401}{†\textsuperscript{34}}}
Pride gives rise to every evil.

\protect\hypertarget{23_NOTES.xhtmlux5cux23id_2404}{\protect\hyperlink{08_Chapter_One__THE_PASSIONATE_INTE.xhtmlux5cux23id_2403}{‡\textsuperscript{35}}}
Greed is the root of all evil.

\protect\hypertarget{23_NOTES.xhtmlux5cux23id_2406}{\protect\hyperlink{08_Chapter_One__THE_PASSIONATE_INTE.xhtmlux5cux23id_2405}{*\textsuperscript{36}}}
``a very pompous man, grasping, more worldly than his station
required.''

\protect\hypertarget{23_NOTES.xhtmlux5cux23id_2408}{\protect\hyperlink{08_Chapter_One__THE_PASSIONATE_INTE.xhtmlux5cux23id_2407}{†\textsuperscript{37}}}
``a man who showed very little pity to people, if he did not receive
money or another gift which was worthwhile; and it was told for truth
that he had more than fifty lawsuits in process, since nothing could be
gotten out of him without going to court.''

\protect\hypertarget{23_NOTES.xhtmlux5cux23id_2410}{\protect\hyperlink{08_Chapter_One__THE_PASSIONATE_INTE.xhtmlux5cux23id_2409}{*\textsuperscript{38}}}
``as a robber and murderer''

\protect\hypertarget{23_NOTES.xhtmlux5cux23id_2228}{\protect\hyperlink{08_Chapter_One__THE_PASSIONATE_INTE.xhtmlux5cux23id_2227}{1}}.
{[}Trans. (Throughout, all translators' comments will be preceded by
this abbreviation in brackets){]} Huizinga's title is \emph{`s Levens
felheid}, literally, Life's Facets, and the \emph{fel} carries the sense
of something that affects the senses strongly. Perhaps there is a bit of
the idea of tension between states of mind or emotion, \emph{spanning},
as the span of a bridge, as well. The implication is that medieval life,
lived in such suspended tension between sharply, even crassly,
contrasting states, strongly affected people's senses. This is the
beginning of a subtle, but forceful development of metaphor that is an
important part of Huizinga's narrative technique.

\protect\hypertarget{23_NOTES.xhtmlux5cux23page_398}{\protect\hyperlink{08_Chapter_One__THE_PASSIONATE_INTE.xhtmlux5cux23id_2226}{2}}.
Oeuvres de Georges Chastellain, ed. Kervyn de Lettenhove, Bruxelles
1863--66; 8 vols., III, p. 44.

\protect\hypertarget{23_NOTES.xhtmlux5cux23id_2225}{\protect\hyperlink{08_Chapter_One__THE_PASSIONATE_INTE.xhtmlux5cux23id_2224}{3}}.
Chastellain, II, p. 267; Mémoires d'Olivier de la Marche, ed. Beaune et
d'Arbaumont (Soc. de l'historie de France), 1883--88; 4 vols., II, p.
248.

\protect\hypertarget{23_NOTES.xhtmlux5cux23id_2223}{\protect\hyperlink{08_Chapter_One__THE_PASSIONATE_INTE.xhtmlux5cux23id_2222}{4}}.
Journal d'un bourgeois de Paris, ed. A. Tuetey (Publ. de la soc. de
l'historie de Paris, Doc. no. III), 1881, pp. 5, 56.

\protect\hypertarget{23_NOTES.xhtmlux5cux23id_2221}{\protect\hyperlink{08_Chapter_One__THE_PASSIONATE_INTE.xhtmlux5cux23id_2220}{5}}.
Journal d'un bourgeois, pp. 20--24. See Journal de Jean de Roye, dite
Chronique scandaleuse, ed. B. de Mandrot (Soc. de l'historie de France),
1894--96, 2 vols., I, p. 330.

\protect\hypertarget{23_NOTES.xhtmlux5cux23id_2219}{\protect\hyperlink{08_Chapter_One__THE_PASSIONATE_INTE.xhtmlux5cux23id_2218}{6}}.
Chastellain, III, p. 461; see V, p. 403.

\protect\hypertarget{23_NOTES.xhtmlux5cux23id_2217}{\protect\hyperlink{08_Chapter_One__THE_PASSIONATE_INTE.xhtmlux5cux23id_2216}{7}}.
Jean Juvenal des Ursins, Chronique, ed. Michaud et Poujoulat, Nouvelle
collection des mémoires, II, 1412, p. 474.

\protect\hypertarget{23_NOTES.xhtmlux5cux23id_2215}{\protect\hyperlink{08_Chapter_One__THE_PASSIONATE_INTE.xhtmlux5cux23id_2214}{8}}.
{[}Trans.{]} The image of the life of people as a dance to death:
frequently painted and depicted in poetry. See below, chap. 5.

\protect\hypertarget{23_NOTES.xhtmlux5cux23id_2213}{\protect\hyperlink{08_Chapter_One__THE_PASSIONATE_INTE.xhtmlux5cux23id_2212}{9}}.
See Journal d'un bourgeois, pp. 6, 70; Jean Molinet, Chronique, ed.
Buchon (Coll. de chron. nat.), 1827--28, 5 vols., II, p. 23; Lettres de
Louis XI, ed. Vaesen, Charavay, de Mandrot (Soc. de l'hist. de France),
1883--1909, 11 vols., 20. Apr. 1477, VI, p. 158; Chronique scandaleuse,
II, p. 47; Chronique scandaleuse, Interpolations, II, p. 364.

\protect\hypertarget{23_NOTES.xhtmlux5cux23id_2211}{\protect\hyperlink{08_Chapter_One__THE_PASSIONATE_INTE.xhtmlux5cux23id_2210}{10}}.
Journal d'un bourgeois, p. 234--37.

\protect\hypertarget{23_NOTES.xhtmlux5cux23id_2209}{\protect\hyperlink{08_Chapter_One__THE_PASSIONATE_INTE.xhtmlux5cux23id_2208}{11}}.
Chron. scand., II, p. 70, 72.

\protect\hypertarget{23_NOTES.xhtmlux5cux23id_2207}{\protect\hyperlink{08_Chapter_One__THE_PASSIONATE_INTE.xhtmlux5cux23id_2206}{12}}.
Vita auct. Petro Ranzano O. P. (1455), Acta sanctorum Apr. t. I, pp.
494ff.

\protect\hypertarget{23_NOTES.xhtmlux5cux23id_2205}{\protect\hyperlink{08_Chapter_One__THE_PASSIONATE_INTE.xhtmlux5cux23id_2204}{13}}.
J. Soyer, Notes pour servir à l'histoire littéraire. Du succes de la
prédication de frère Olivier Maillart à Orléans en 1485, Bulletin de la
société archéologique et historique de l'Orléanais, t. XVIII, 1919,
according to Revue historique, t. 131, p. 351.

\protect\hypertarget{23_NOTES.xhtmlux5cux23id_2203}{\protect\hyperlink{08_Chapter_One__THE_PASSIONATE_INTE.xhtmlux5cux23id_2202}{14}}.
{[}Trans.{]} \emph{Hennin}. A style of coiffure in the shape of a cone
rising very high from which veils were suspended.

\protect\hypertarget{23_NOTES.xhtmlux5cux23id_2201}{\protect\hyperlink{08_Chapter_One__THE_PASSIONATE_INTE.xhtmlux5cux23id_2200}{15}}.
{[}Trans.{]} A mystical order for lay women that performed good works
and was given to publicly reading the Bible aloud in French. Not
officially sanctioned by the church. See: Barbara Tuchmann, \emph{The
Distant Mirror}. Alfred A. Knopf, New York, 1978, p. 317.

\protect\hypertarget{23_NOTES.xhtmlux5cux23id_2199}{\protect\hyperlink{08_Chapter_One__THE_PASSIONATE_INTE.xhtmlux5cux23id_2198}{16}}.
Enguerrand de Monstrelet, Chroniques, ed. Douët d'Arcq. (Soc. de l'hist.
de France) 1857--62, 6 vols., IV, pp. 302--6.

\protect\hypertarget{23_NOTES.xhtmlux5cux23id_2197}{\protect\hyperlink{08_Chapter_One__THE_PASSIONATE_INTE.xhtmlux5cux23id_2196}{17}}.
Wadding, Annales Minorum, X, p. 72; K. Hefele, Der h. Bernhardin von
Siena und die franziskanische wanderpredigt in Italien. Freiburg 1912,
S. 47, 80.

\protect\hypertarget{23_NOTES.xhtmlux5cux23id_2195}{\protect\hyperlink{08_Chapter_One__THE_PASSIONATE_INTE.xhtmlux5cux23id_2194}{18}}.
Chron. scand., I, p. 22, 1461; Jean Chartier, Hist. de Charles VII, ed.
N. Godefroy, 1661, p. 320.

\protect\hypertarget{23_NOTES.xhtmlux5cux23id_2193}{\protect\hyperlink{08_Chapter_One__THE_PASSIONATE_INTE.xhtmlux5cux23id_2192}{19}}.
Chastellain, III, pp. 36, 98, 124, 125, 210, 238, 239, 247, 474; Jacques
du Clercq, Mémoires (1448--1467), ed. de Reiffenberg, Bruxelles 1823, 4
vols., IV, p. 40, II, p. 280, 355, III, p. 100; Juvenal des Ursins, pp.
405, 407, 420; Molinet, III, pp. 36, 314.

\protect\hypertarget{23_NOTES.xhtmlux5cux23id_2191}{\protect\hyperlink{08_Chapter_One__THE_PASSIONATE_INTE.xhtmlux5cux23id_2190}{20}}.
Jean Germain, Liber de virtutibus Philippi ducis Burgundiae, ed. Kervyn
de Lettenhove, Chron. rel. à l'hist. de la Belg. sous la dom. des ducs
de Bourg. (Coll. des chron. belges), 1876, II, p. 50.

\protect\hypertarget{23_NOTES.xhtmlux5cux23id_2189}{\protect\hyperlink{08_Chapter_One__THE_PASSIONATE_INTE.xhtmlux5cux23id_2188}{21}}.
La Marche, I, p. 61.

\protect\hypertarget{23_NOTES.xhtmlux5cux23id_2187}{\protect\hyperlink{08_Chapter_One__THE_PASSIONATE_INTE.xhtmlux5cux23id_2186}{22}}.
Chastellain, IV, pp. 333f.

\protect\hypertarget{23_NOTES.xhtmlux5cux23page_399}{\protect\hyperlink{08_Chapter_One__THE_PASSIONATE_INTE.xhtmlux5cux23id_2185}{23}}.
Chastellain, III, p. 92.

\protect\hypertarget{23_NOTES.xhtmlux5cux23id_2184}{\protect\hyperlink{08_Chapter_One__THE_PASSIONATE_INTE.xhtmlux5cux23id_2183}{24}}.
Jean Froissart, Chroniques, ed. S. Luce et G. Raynaud (Soc. de l'hist.
de France), 1869--1899, 11 vols. (only up to 1385), IV, pp. 89--93.

\protect\hypertarget{23_NOTES.xhtmlux5cux23id_2182}{\protect\hyperlink{08_Chapter_One__THE_PASSIONATE_INTE.xhtmlux5cux23id_2181}{25}}.
Chastellain, III, pp. 85ff.

\protect\hypertarget{23_NOTES.xhtmlux5cux23id_2180}{\protect\hyperlink{08_Chapter_One__THE_PASSIONATE_INTE.xhtmlux5cux23id_2179}{26}}.
Chastellain, III, p. 279.

\protect\hypertarget{23_NOTES.xhtmlux5cux23id_2178}{\protect\hyperlink{08_Chapter_One__THE_PASSIONATE_INTE.xhtmlux5cux23id_2177}{27}}.
La Marche, II, p. 421.

\protect\hypertarget{23_NOTES.xhtmlux5cux23id_2176}{\protect\hyperlink{08_Chapter_One__THE_PASSIONATE_INTE.xhtmlux5cux23id_2175}{28}}.
Juvenal des Ursins, p. 379.

\protect\hypertarget{23_NOTES.xhtmlux5cux23id_2174}{\protect\hyperlink{08_Chapter_One__THE_PASSIONATE_INTE.xhtmlux5cux23id_2173}{29}}.
Martin Le Franc, Le Champion des dames, See G. Doutrepont, La
littérature française à la cour des ducs de Bourgogne (Bibl. de XVe
siecle t. VIII), Paris, Champion, 1909, p. 304.

\protect\hypertarget{23_NOTES.xhtmlux5cux23id_2172}{\protect\hyperlink{08_Chapter_One__THE_PASSIONATE_INTE.xhtmlux5cux23id_2171}{30}}.
Acta sanctorum Apr. t. I, p. 496; A. Renaudet, Préréforme et humanisme à
Paris 1494--1517, Paris, Champion, 1916, p. 163.

\protect\hypertarget{23_NOTES.xhtmlux5cux23id_2170}{\protect\hyperlink{08_Chapter_One__THE_PASSIONATE_INTE.xhtmlux5cux23id_2169}{31}}.
{[}Trans.{]} \emph{Spanning}. See note 1.

\protect\hypertarget{23_NOTES.xhtmlux5cux23id_2168}{\protect\hyperlink{08_Chapter_One__THE_PASSIONATE_INTE.xhtmlux5cux23id_2167}{32}}.
Chastellain, IV, pp. 3oof, VII, p. 75; see Thomas Basin, De rebus gestis
Caroli VII. et Lud. XL historiarum libri XII, ed. Quicherat (Soc. de
l'hist. de France), 1855--1859, 4 vols., I, p. 158.

\protect\hypertarget{23_NOTES.xhtmlux5cux23id_2166}{\protect\hyperlink{08_Chapter_One__THE_PASSIONATE_INTE.xhtmlux5cux23id_2165}{33}}.
Journal d'un bourgeois, p. 219.

\protect\hypertarget{23_NOTES.xhtmlux5cux23id_2164}{\protect\hyperlink{08_Chapter_One__THE_PASSIONATE_INTE.xhtmlux5cux23id_2163}{34}}.
Chastellain III, p. 30.

\protect\hypertarget{23_NOTES.xhtmlux5cux23id_2162}{\protect\hyperlink{08_Chapter_One__THE_PASSIONATE_INTE.xhtmlux5cux23id_2161}{35}}.
La Marche, I, p. 89.

\protect\hypertarget{23_NOTES.xhtmlux5cux23id_2160}{\protect\hyperlink{08_Chapter_One__THE_PASSIONATE_INTE.xhtmlux5cux23id_2159}{36}}.
Chastellain, I, pp. 82, 79; Monstrelet, III, p. 361.

\protect\hypertarget{23_NOTES.xhtmlux5cux23id_2158}{\protect\hyperlink{08_Chapter_One__THE_PASSIONATE_INTE.xhtmlux5cux23id_2157}{37}}.
La Marche, I, p. 201.

\protect\hypertarget{23_NOTES.xhtmlux5cux23id_2156}{\protect\hyperlink{08_Chapter_One__THE_PASSIONATE_INTE.xhtmlux5cux23id_2155}{38}}.
On the Treaty of Arras see among others La Marche, I, p. 207.

\protect\hypertarget{23_NOTES.xhtmlux5cux23id_2154}{\protect\hyperlink{08_Chapter_One__THE_PASSIONATE_INTE.xhtmlux5cux23id_2153}{39}}.
Chastellain, I, p. 196.

\protect\hypertarget{23_NOTES.xhtmlux5cux23id_2152}{\protect\hyperlink{08_Chapter_One__THE_PASSIONATE_INTE.xhtmlux5cux23id_2151}{40}}.
Basin, III, p. 74, {[}Trans.{]} 40 {[}Trans.{]} \emph{Hoecken and
Kabeljauen}: The names of two political parties that formed during the
complicated struggle for succession in Holland, Zeeland, and Hainaut. In
standard interpretations the Kabeljauen (codfish) were the party of the
ascending burghers while the Hoecken (hooks) were the declining nobles
who hoped to snare the wealth of the burghers. Huizinga repeatedly
cautions against accepting such simple economic explanations.

\protect\hypertarget{23_NOTES.xhtmlux5cux23id_2150}{\protect\hyperlink{08_Chapter_One__THE_PASSIONATE_INTE.xhtmlux5cux23id_2149}{41}}.
That a perception like this by no means rules out a recognition of
economic factors, not to mention the charge that it was formulated as a
protest against the economic explanation of history, can be demonstrated
by the following quotation from Jaures: ``Mais il n'y a pas seulement
dans l'histoire des luttes de classes, il y a aussi des luttes de
partis. J'entends qu'en dehors des affinités ou des antagonismes
économiques il se forme des groupements de passions, des intérêts
d'orgueil, de domination, qui se disputent la surface de l'histoire et
qui déterminent de très vastes ébranlements.'' Histoire de la révolution
francaise, IV, p. 1458.

\protect\hypertarget{23_NOTES.xhtmlux5cux23id_2148}{\protect\hyperlink{08_Chapter_One__THE_PASSIONATE_INTE.xhtmlux5cux23id_2147}{42}}.
{[}Trans.{]} \emph{Arnold and Adolf of Geldern}: Arnold had secured the
dukedom by ceding much of the power of the position to a council of
nobles and leading burghers that led his wife and son Adolf to conspire
against him. Arnold, in retaliation, sold the succession to Charles the
Bold, who became Duke of Gelder upon Arnold's death. When Charles was
killed, Adolf was released from prison. He mounted a campaign to regain
the dukedom, but was killed at the siege of Tournai.

\protect\hypertarget{23_NOTES.xhtmlux5cux23id_2146}{\protect\hyperlink{08_Chapter_One__THE_PASSIONATE_INTE.xhtmlux5cux23id_2145}{43}}.
Chastellain, IV, p. 201; see my Studie uit de voorgeschiedenis van ons
nationaal besef, in De Gids 1912, I.

\protect\hypertarget{23_NOTES.xhtmlux5cux23id_2144}{\protect\hyperlink{08_Chapter_One__THE_PASSIONATE_INTE.xhtmlux5cux23id_2143}{44}}.
{[}Trans.{]} In the Hundred Years War (1337--1453). The struggle between
\protect\hypertarget{23_NOTES.xhtmlux5cux23page_400}{}{}England and
France over territory and dynastic issues which waxed and waned during
nearly the whole period covered by Huizinga's study. Though the
\emph{Autumn of the Middle Ages} is a cultural history of France and
Burgundy at the time of the war, Huizinga does not narrate the events of
that war, although an alert reader will be able to detect many of them.
In part, Huizinga assumes basic familiarity with this history on the
part of his readers, but the omission is also a deliberate break with
the narrative historical tradition. The reader might be well served by
reading a good encyclopedia article on both the Hundred Years War and
the history of Burgundy.

\protect\hypertarget{23_NOTES.xhtmlux5cux23id_2142}{\protect\hyperlink{08_Chapter_One__THE_PASSIONATE_INTE.xhtmlux5cux23id_2141}{45}}.
Journal d'un bourgeois, p. 242; see Monstrelet, IV, p. 341d.

\protect\hypertarget{23_NOTES.xhtmlux5cux23id_2140}{\protect\hyperlink{08_Chapter_One__THE_PASSIONATE_INTE.xhtmlux5cux23id_2139}{46}}.
Jan van Dixmude, ed. Lambin, Ypres 1839, p. 283.

\protect\hypertarget{23_NOTES.xhtmlux5cux23id_2138}{\protect\hyperlink{08_Chapter_One__THE_PASSIONATE_INTE.xhtmlux5cux23id_2137}{47}}.
Froissart, ed. Luce, XI, p. 52.

\protect\hypertarget{23_NOTES.xhtmlux5cux23id_2136}{\protect\hyperlink{08_Chapter_One__THE_PASSIONATE_INTE.xhtmlux5cux23id_2135}{48}}.
Mémoires de Pierre le Fruictier dit Salmon, Buchon 3\textsuperscript{e}
suppl. de Froissart, XV, p. 22.

\protect\hypertarget{23_NOTES.xhtmlux5cux23id_2134}{\protect\hyperlink{08_Chapter_One__THE_PASSIONATE_INTE.xhtmlux5cux23id_2133}{49}}.
Chronique du Religieux de Saint Denis, ed. Bellaguet (Coll. des
documents inédits) 1839--1852, 6 vols., I, p. 34; Juvenal des Ursins,
pp. 342, 467--471; Journal d'un bourgeois, pp. 12, 31, 44. {[}Trans.{]}
A St. Andrew's cross is a cross in the shape of an ``X.'' It was an
insignium of the Burgundian party, hence pro--English. It is still a
part of the Union Jack.

\protect\hypertarget{23_NOTES.xhtmlux5cux23id_2132}{\protect\hyperlink{08_Chapter_One__THE_PASSIONATE_INTE.xhtmlux5cux23id_2131}{50}}.
Molinet, III, p. 487.

\protect\hypertarget{23_NOTES.xhtmlux5cux23id_2130}{\protect\hyperlink{08_Chapter_One__THE_PASSIONATE_INTE.xhtmlux5cux23id_2129}{51}}.
Molinet, III, pp. 226, 241, 283--287; La Marche, III, pp. 289, 302.

\protect\hypertarget{23_NOTES.xhtmlux5cux23id_2128}{\protect\hyperlink{08_Chapter_One__THE_PASSIONATE_INTE.xhtmlux5cux23id_2127}{52}}.
Clementis V constitutiones. lib. V. tit. 9, c. 1.; Joannis Gersonii,
Opera omnia, ed. L. Ellies Dupin, ed. II, Hagae Comitis 1728, 5 vols.,
II, p. 427; Ordonnances des rois de France, t. VIII, p. 122; N. Jorga,
Philippe de Mézières et la croisade au XlVe siècle (Bibl. de l'ecole des
hautes études, fasc. 110), 1896, p. 438; Religieux de S. Denis, II, p.
533.

\protect\hypertarget{23_NOTES.xhtmlux5cux23id_2126}{\protect\hyperlink{08_Chapter_One__THE_PASSIONATE_INTE.xhtmlux5cux23id_2125}{53}}.
Journal d'un bourgeois, pp. 223, 229.

\protect\hypertarget{23_NOTES.xhtmlux5cux23id_2124}{\protect\hyperlink{08_Chapter_One__THE_PASSIONATE_INTE.xhtmlux5cux23id_2123}{54}}.
Jacques du Clercq, IV, p. 265. Petit--Dutaillis, Documents nouveaux sur
les moeurs populaires et le droit de venegeance das les Pays-Bas au XVe
siecle (Bibl. du XVe Siecle), Paris, Champion, 1908, pp. 7, 21.

\protect\hypertarget{23_NOTES.xhtmlux5cux23id_2122}{\protect\hyperlink{08_Chapter_One__THE_PASSIONATE_INTE.xhtmlux5cux23id_2121}{55}}.
Pierre de Fenin (Petitot, Coll. de mém. VII), p. 593; see his story of
the fool who was beaten to death, p. 619.

\protect\hypertarget{23_NOTES.xhtmlux5cux23id_2120}{\protect\hyperlink{08_Chapter_One__THE_PASSIONATE_INTE.xhtmlux5cux23id_2119}{56}}.
Journal d'un bourgeois, p. 204.

\protect\hypertarget{23_NOTES.xhtmlux5cux23id_2118}{\protect\hyperlink{08_Chapter_One__THE_PASSIONATE_INTE.xhtmlux5cux23id_2117}{57}}.
{[}Trans.{]} \emph{entremets}. Although the word comes to mean side
dishes, Huizinga uses it in an older sense meaning elaborate
entertainments held between the courses of aristocratic banquets. See
chap. 12 for descriptions of some of these \emph{entremets}.

\protect\hypertarget{23_NOTES.xhtmlux5cux23id_2116}{\protect\hyperlink{08_Chapter_One__THE_PASSIONATE_INTE.xhtmlux5cux23id_2115}{58}}.
Jean Lefèvre de Saint-Remy, Chronique, ed. F. Morand (Soc. de l'hist. de
France), 1876, 2 vols., II, p. 168; Laborde, Les ducs de Bourgogne,
Etudes sur les lettres, les arts, et l'industrie pendant le XVe siecle,
Paris 1849--1853, 3 vols., II, p. 208.

\protect\hypertarget{23_NOTES.xhtmlux5cux23id_2114}{\protect\hyperlink{08_Chapter_One__THE_PASSIONATE_INTE.xhtmlux5cux23id_2113}{59}}.
La Marche, III, p. 135; Laborde, II, p. 325.

\protect\hypertarget{23_NOTES.xhtmlux5cux23id_2112}{\protect\hyperlink{08_Chapter_One__THE_PASSIONATE_INTE.xhtmlux5cux23id_2111}{60}}.
Laborde, III, pp. 355, 398. Le Moyen-age, XX, 1907, pp. 193--201.

\protect\hypertarget{23_NOTES.xhtmlux5cux23id_2110}{\protect\hyperlink{08_Chapter_One__THE_PASSIONATE_INTE.xhtmlux5cux23id_2109}{61}}.
Juvenal des Ursins, pp. 438, 1405. See, however, Rel. de. S. Denis, III,
p. 349.

\protect\hypertarget{23_NOTES.xhtmlux5cux23id_2108}{\protect\hyperlink{08_Chapter_One__THE_PASSIONATE_INTE.xhtmlux5cux23id_2107}{62}}.
Piaget, Romania, XX, p. 417 en XXXI, 1902, pp. 597--603.

\protect\hypertarget{23_NOTES.xhtmlux5cux23id_2106}{\protect\hyperlink{08_Chapter_One__THE_PASSIONATE_INTE.xhtmlux5cux23id_2105}{63}}.
Journal d'un bourgeois, p. 95.

\protect\hypertarget{23_NOTES.xhtmlux5cux23id_2104}{\protect\hyperlink{08_Chapter_One__THE_PASSIONATE_INTE.xhtmlux5cux23id_2103}{64}}.
Jacques de Clercq, III, p. 262.

\protect\hypertarget{23_NOTES.xhtmlux5cux23page_401}{\protect\hyperlink{08_Chapter_One__THE_PASSIONATE_INTE.xhtmlux5cux23id_2102}{65}}.
Jacques du Clercq passim; Petit Dutaillis, Documents etc., p. 131.

\emph{\protect\hypertarget{23_NOTES.xhtmlux5cux23id_2101}{\protect\hyperlink{08_Chapter_One__THE_PASSIONATE_INTE.xhtmlux5cux23id_2100}{66}}}.
Hugo of St. Victor, De fructibus carnia et spiritus, Migne CLXXVI, p.
997.

\protect\hypertarget{23_NOTES.xhtmlux5cux23id_2099}{\protect\hyperlink{08_Chapter_One__THE_PASSIONATE_INTE.xhtmlux5cux23id_2098}{67}}.
Tobit 4:14. ({[}Trans.{]} In English Bibles, Tobit 4:13.)

\protect\hypertarget{23_NOTES.xhtmlux5cux23id_2097}{\protect\hyperlink{08_Chapter_One__THE_PASSIONATE_INTE.xhtmlux5cux23id_2096}{68}}.
I Timothy 6:10.

\protect\hypertarget{23_NOTES.xhtmlux5cux23id_2095}{\protect\hyperlink{08_Chapter_One__THE_PASSIONATE_INTE.xhtmlux5cux23id_2094}{69}}.
Petrus Damiani, Epist. lib. I, 15, Migne CXLIV, p. 234, id. Contra
philargyriam ib. CXLV, p. 533; Pseudo-Bernardus, Liber de modo bene
vivendi 44, 45, Migne CLXXXLV, p. 1266.

\protect\hypertarget{23_NOTES.xhtmlux5cux23id_2093}{\protect\hyperlink{08_Chapter_One__THE_PASSIONATE_INTE.xhtmlux5cux23id_2092}{70}}.
Journal d'un bourgeois, pp. 325, 343, 357; in the note on the citations
from the parliamentary records.

\protect\hypertarget{23_NOTES.xhtmlux5cux23id_2091}{\protect\hyperlink{08_Chapter_One__THE_PASSIONATE_INTE.xhtmlux5cux23id_2090}{71}}.
L. Mirot, Les d'Orgemont, leur origine, leur fortune, etc. (Bibl. du XVe
siecle), Paris, 1913; P. Champion, Francois Villon, Sa vie et son temps
(Bibl. du XVe siècle), Paris, 1913, II, pp. 230f.

\protect\hypertarget{23_NOTES.xhtmlux5cux23id_2089}{\protect\hyperlink{08_Chapter_One__THE_PASSIONATE_INTE.xhtmlux5cux23id_2088}{72}}.
Mathieu d'Escouchy, Chronique, ed. G. du Fresne de Beaucourt (Soc. de
l'hist. de France), 1863--1864, 3 vols., I, p. iv--xxiii.

\protect\hypertarget{23_NOTES.xhtmlux5cux23id_2087}{\protect\hyperlink{08_Chapter_One__THE_PASSIONATE_INTE.xhtmlux5cux23id_2086}{73}}.
P. Champion, François Villon, sa vie et son temps (Bibl. du XVe siècle),
Paris, 1913, 2 vols.

\textbf{\emph{Chapter 2}}

\protect\hypertarget{23_NOTES.xhtmlux5cux23id_2412}{\protect\hyperlink{09_Chapter_Two__THE_CRAVING_FOR_A_M.xhtmlux5cux23id_2411}{*\textsuperscript{1}}}
Time of mourning and of temptation, /Age of tears, of envy and of
torment, / Time of languor and of damnation,/Age that brings us to the
end,/Time full of horror which does all things foolishly,/Lying age,
full of pride and envy,/Time without honor and without true judgment,
/Age of sadness which shortens life.

\protect\hypertarget{23_NOTES.xhtmlux5cux23id_2414}{\protect\hyperlink{09_Chapter_Two__THE_CRAVING_FOR_A_M.xhtmlux5cux23id_2413}{*\textsuperscript{2}}}
All mirth is lost, /All hearts have been taken by storm, /By sadness and
melancholy.

\protect\hypertarget{23_NOTES.xhtmlux5cux23id_2416}{\protect\hyperlink{09_Chapter_Two__THE_CRAVING_FOR_A_M.xhtmlux5cux23id_2415}{†\textsuperscript{3}}}
O miserable and most sad life! .~.~. /There is warfare, death, and
famine;/ Cold and heat, day and night make us weak;/Fleas, scabs, and so
many other vermin/Make war on us. In short, have mercy Lord/On our
miserable persons, whose life is very short.

\protect\hypertarget{23_NOTES.xhtmlux5cux23id_2418}{\protect\hyperlink{09_Chapter_Two__THE_CRAVING_FOR_A_M.xhtmlux5cux23id_2417}{‡\textsuperscript{4}}}
And I poor writer,/With the sad heart, feeble and vain,/When I see
everyone mourning,/Then trouble holds me in her hand,/I always have
tears in my eyes,/ For I wish for nothing but to die.

\protect\hypertarget{23_NOTES.xhtmlux5cux23id_2420}{\protect\hyperlink{09_Chapter_Two__THE_CRAVING_FOR_A_M.xhtmlux5cux23id_2419}{*\textsuperscript{5}}}
``I man of sadness, born in deepest darkness and thick rain of
lamentations''

\protect\hypertarget{23_NOTES.xhtmlux5cux23id_2422}{\protect\hyperlink{09_Chapter_Two__THE_CRAVING_FOR_A_M.xhtmlux5cux23id_2421}{†\textsuperscript{6}}}
``So much had La Marche suffered.''

\protect\hypertarget{23_NOTES.xhtmlux5cux23id_2424}{\protect\hyperlink{09_Chapter_Two__THE_CRAVING_FOR_A_M.xhtmlux5cux23id_2423}{‡\textsuperscript{7}}}
``when he had reflected (merancoliet) for a while, he resolved to answer
the emissaries of the King of France''

\protect\hypertarget{23_NOTES.xhtmlux5cux23id_2426}{\protect\hyperlink{09_Chapter_Two__THE_CRAVING_FOR_A_M.xhtmlux5cux23id_2425}{*\textsuperscript{8}}}
Now he is decaying, pitiful and weak,/Old, covetous and libelous;/I see
only fools, men and women both .~.~. /The end is truly near .~.~.
/Everything is going bad .~.~.

\protect\hypertarget{23_NOTES.xhtmlux5cux23id_2428}{\protect\hyperlink{09_Chapter_Two__THE_CRAVING_FOR_A_M.xhtmlux5cux23id_2427}{*\textsuperscript{9}}}
``a magnificent and praiseworthy thing''

\protect\hypertarget{23_NOTES.xhtmlux5cux23id_2430}{\protect\hyperlink{09_Chapter_Two__THE_CRAVING_FOR_A_M.xhtmlux5cux23id_2429}{†\textsuperscript{10}}}
``He was in a habit of devoting a part of his day to serious
occupations, and, with games and laughter interspersed, pleased himself
with fine speeches and with exhorting his nobles, like an orator, to
practice virtue. And in this intention, he was often seen sitting in a
chair of state, with his nobles before him, remonstrating with them
according to time and circumstances. And always, as the prince and ruler
of all, he was richly and magnificently dressed, more so than all the
others.''

\protect\hypertarget{23_NOTES.xhtmlux5cux23id_2432}{\protect\hyperlink{09_Chapter_Two__THE_CRAVING_FOR_A_M.xhtmlux5cux23id_2431}{‡\textsuperscript{11}}}
``high magnificence of heart, because he was seen and regarded in
extraordinary things''

\protect\hypertarget{23_NOTES.xhtmlux5cux23id_2434}{\protect\hyperlink{09_Chapter_Two__THE_CRAVING_FOR_A_M.xhtmlux5cux23id_2433}{*\textsuperscript{12}}}
``which was not customary for men on watch.''

\protect\hypertarget{23_NOTES.xhtmlux5cux23id_2308}{\protect\hyperlink{09_Chapter_Two__THE_CRAVING_FOR_A_M.xhtmlux5cux23id_2307}{*\textsuperscript{13}}}
``wretched people''

\protect\hypertarget{23_NOTES.xhtmlux5cux23id_2436}{\protect\hyperlink{09_Chapter_Two__THE_CRAVING_FOR_A_M.xhtmlux5cux23id_2435}{*\textsuperscript{14}}}
``He who humbles himself before someone who is greater than he increases
and multiplies his honor himself and the good shines forth and overflows
from his face.''

\protect\hypertarget{23_NOTES.xhtmlux5cux23id_2438}{\protect\hyperlink{09_Chapter_Two__THE_CRAVING_FOR_A_M.xhtmlux5cux23id_2437}{*\textsuperscript{15}}}
``Go on.''---``I shall not.''---``Come forward!/Certainly, you will do
so, dear cousin.''/---``No, I shall not.''---``Call to our
neighbor,/That she should offer before you.''/---``You should not suffer
it.''/The neighbor responds; ``This is not proper/for me; offer, it is
only because of you/That the priest does not continue.''

\protect\hypertarget{23_NOTES.xhtmlux5cux23id_2440}{\protect\hyperlink{09_Chapter_Two__THE_CRAVING_FOR_A_M.xhtmlux5cux23id_2439}{*\textsuperscript{16}}}
The young woman should answer:/---``Take it, I shall not,
lady.''/---``Yes, do take it, dear friend.''/---``Certainly I shall not
take it;/People would take me for a fool.''/---``Pass it, Miss
Marote.''/``I shall not, Jesus Christ forbid!/Pass it to the Lady
Ermagart.''/---``Lady take it.''---``Holy Mary,/Take the pax to the
bailiff's wife.''/---``No, to the governor's wife.''

\protect\hypertarget{23_NOTES.xhtmlux5cux23id_2442}{\protect\hyperlink{09_Chapter_Two__THE_CRAVING_FOR_A_M.xhtmlux5cux23id_2441}{*\textsuperscript{17}}}
``and many marveled greatly at his liberality.''

\protect\hypertarget{23_NOTES.xhtmlux5cux23id_2444}{\protect\hyperlink{09_Chapter_Two__THE_CRAVING_FOR_A_M.xhtmlux5cux23id_2443}{*\textsuperscript{18}}}
``and had the body buried!''

\protect\hypertarget{23_NOTES.xhtmlux5cux23id_2446}{\protect\hyperlink{09_Chapter_Two__THE_CRAVING_FOR_A_M.xhtmlux5cux23id_2445}{*\textsuperscript{19}}}
``but when they had succeeded in making one or two get up, seven or
eight sat down on the other side.''

\protect\hypertarget{23_NOTES.xhtmlux5cux23id_2447}{\protect\hyperlink{09_Chapter_Two__THE_CRAVING_FOR_A_M.xhtmlux5cux23id_2448}{†\textsuperscript{20}}}
``without saying a word to him, they approached him. Lhuillier elbowed
him in the stomach, the others tore up the priest's hat and its
ribbons.''

\protect\hypertarget{23_NOTES.xhtmlux5cux23id_2450}{\protect\hyperlink{09_Chapter_Two__THE_CRAVING_FOR_A_M.xhtmlux5cux23id_2449}{*\textsuperscript{21}}}
``heaping many invectives on him, shaking his finger in the archbishop's
face, and grabbing him in such a way by the arm that he tore his
vestment; and if the archbishop had not held up his hand, he would have
hit him in the face.''

\protect\hypertarget{23_NOTES.xhtmlux5cux23id_2452}{\protect\hyperlink{09_Chapter_Two__THE_CRAVING_FOR_A_M.xhtmlux5cux23id_2451}{*\textsuperscript{22}}}
``and there was neither rule nor measure to his grief, and he astonished
everyone with the depths of his sorrow.''

\protect\hypertarget{23_NOTES.xhtmlux5cux23id_2454}{\protect\hyperlink{09_Chapter_Two__THE_CRAVING_FOR_A_M.xhtmlux5cux23id_2453}{†\textsuperscript{23}}}
``there was lamentation; all kinds of people weeping and crying and
varied shouts of pain and suffering, loudly expressed, could be heard.''

\protect\hypertarget{23_NOTES.xhtmlux5cux23id_2456}{\protect\hyperlink{09_Chapter_Two__THE_CRAVING_FOR_A_M.xhtmlux5cux23id_2455}{‡\textsuperscript{24}}}
``truly dead and gone to the grave.''

\protect\hypertarget{23_NOTES.xhtmlux5cux23id_2458}{\protect\hyperlink{09_Chapter_Two__THE_CRAVING_FOR_A_M.xhtmlux5cux23id_2457}{*\textsuperscript{25}}}
``That is humbug!''

\protect\hypertarget{23_NOTES.xhtmlux5cux23id_2460}{\protect\hyperlink{09_Chapter_Two__THE_CRAVING_FOR_A_M.xhtmlux5cux23id_2459}{†\textsuperscript{26}}}
``Monsieur the Chancellor, I thank you for the letters, etc., but I beg
you to send no more by him who brought them, for I found his face
terribly changed since I last saw him, and I tell you on my honor that
he made me much afraid; and farewell.''

\protect\hypertarget{23_NOTES.xhtmlux5cux23id_2462}{\protect\hyperlink{09_Chapter_Two__THE_CRAVING_FOR_A_M.xhtmlux5cux23id_2461}{*\textsuperscript{27}}}
``When Madame was in private, she by no means always lay in bed nor
confined herself to one room.''

\protect\hypertarget{23_NOTES.xhtmlux5cux23id_2464}{\protect\hyperlink{09_Chapter_Two__THE_CRAVING_FOR_A_M.xhtmlux5cux23id_2463}{*\textsuperscript{28}}}
``most beautiful contrition for her Sins''

\protect\hypertarget{23_NOTES.xhtmlux5cux23id_2085}{\protect\hyperlink{09_Chapter_Two__THE_CRAVING_FOR_A_M.xhtmlux5cux23id_2084}{1}}.
Allen, no. 541, Antwerpen, 26 February 1516/17; see no. 542, no. 566,
no. 862, no. 967.

\protect\hypertarget{23_NOTES.xhtmlux5cux23id_2083}{\protect\hyperlink{09_Chapter_Two__THE_CRAVING_FOR_A_M.xhtmlux5cux23id_2082}{2}}.
\emph{Germanae}, which, in this particular instance, cannot be
translated as ``German.''

\protect\hypertarget{23_NOTES.xhtmlux5cux23id_2081}{\protect\hyperlink{09_Chapter_Two__THE_CRAVING_FOR_A_M.xhtmlux5cux23id_2080}{3}}.
Eustache Deschamps, Oeuvres complètes, ed. De Queux de Saint Hilaire et
G. Raynaud (Soc. des anciens textes français) 1878--1903, 11 vols., no.
31 (I, p. 113, see nos. 85, 126, 152, 162, 176, 248, 366, 375, 386, 400,
933, 936, 1195, 1196, 1207, 1213, 1239, 1240, etc.; Chastellain, I pp.
9, 27, IV pp. 5, 56, VI pp. 206, 208, 219, 295; Alain Chartier, Oeuvres,
ed. A. Duchesne, Paris 1617, p. 262; Alanus de Rupe, Sermo, II, p. 313
(B. Alanus redivivus, ed J. A. Coppenstein, Naples, 1642).

\protect\hypertarget{23_NOTES.xhtmlux5cux23id_2079}{\protect\hyperlink{09_Chapter_Two__THE_CRAVING_FOR_A_M.xhtmlux5cux23id_2078}{4}}.
Deschamps, no. 562 (IV, p. 18).

\protect\hypertarget{23_NOTES.xhtmlux5cux23id_2077}{\protect\hyperlink{09_Chapter_Two__THE_CRAVING_FOR_A_M.xhtmlux5cux23id_2076}{5}}.
A. de la Borderie, Jean Meschinot, sa vie et ses oeuvres (Bibl. de
l'Ecole des chartes), LVI, 1895, pp. 277, 280, 305, 310, 312, 622, etc.

\protect\hypertarget{23_NOTES.xhtmlux5cux23id_2075}{\protect\hyperlink{09_Chapter_Two__THE_CRAVING_FOR_A_M.xhtmlux5cux23id_2074}{6}}.
Chastellain, I, p. 10 Prologue; see Complainte de fortune, VIII, p. 334.

\protect\hypertarget{23_NOTES.xhtmlux5cux23id_2073}{\protect\hyperlink{09_Chapter_Two__THE_CRAVING_FOR_A_M.xhtmlux5cux23id_2072}{7}}.
La Marche, I p. 186, IV p. 89; H. Stein, Etude sur Olivier de la Marche,
historien, poete et diplomate (Mém. couronnés etc., de l'Acad royale de
Belg. t. XLIX), Bruxelles 1888, frontispiece.

\protect\hypertarget{23_NOTES.xhtmlux5cux23id_2071}{\protect\hyperlink{09_Chapter_Two__THE_CRAVING_FOR_A_M.xhtmlux5cux23id_2070}{8}}.
Monstrelet, IV, p. 430.

\protect\hypertarget{23_NOTES.xhtmlux5cux23id_2069}{\protect\hyperlink{09_Chapter_Two__THE_CRAVING_FOR_A_M.xhtmlux5cux23id_2068}{9}}.
Froissart, ed. Luce, X, p. 275; Deschamps no. 810 (IV, p. 327); see Les
Quinze joyes de mariage (Paris, Marpon et Flammarion), p. 64 (quinte
joye); Le livre messire Geoffroi de Charney, Romania, XXXVI, 1897, p.
399.

\protect\hypertarget{23_NOTES.xhtmlux5cux23id_2067}{\protect\hyperlink{09_Chapter_Two__THE_CRAVING_FOR_A_M.xhtmlux5cux23id_2066}{10}}.
Joannis de Varennis responsiones ad capitula accusationum etc. §17, by
Gerson, Opera, I, p. 920.

\protect\hypertarget{23_NOTES.xhtmlux5cux23id_2065}{\protect\hyperlink{09_Chapter_Two__THE_CRAVING_FOR_A_M.xhtmlux5cux23id_2064}{11}}.
Deschamps, no. 95 (I, p. 203).

\protect\hypertarget{23_NOTES.xhtmlux5cux23id_2063}{\protect\hyperlink{09_Chapter_Two__THE_CRAVING_FOR_A_M.xhtmlux5cux23id_2062}{12}}.
Deschamps, Le miroir de mariage, IX, pp. 25, 69, 81, no. 1004 (V, p.
259), further II pp. 8, 183--188, III pp. 39, 373, VII p. 3, IX p. 209,
etc.

\protect\hypertarget{23_NOTES.xhtmlux5cux23id_2061}{\protect\hyperlink{09_Chapter_Two__THE_CRAVING_FOR_A_M.xhtmlux5cux23id_2060}{13}}.
Convivio lib., IV, cap. 27, 28.

\protect\hypertarget{23_NOTES.xhtmlux5cux23page_402}{\protect\hyperlink{09_Chapter_Two__THE_CRAVING_FOR_A_M.xhtmlux5cux23id_2059}{14}}.
Discours de l'excellence de virginité, Gerson, Opera III, p. 382; see
Dionysius Cartusianus, De vanitate mundi, Opera omnia, cura et labore
monachorum sacr. ord. Cart., Monstrolii-Tornaci 1896--1913, 41, vol.
XXXIX, p. 472.

\protect\hypertarget{23_NOTES.xhtmlux5cux23id_2058}{\protect\hyperlink{09_Chapter_Two__THE_CRAVING_FOR_A_M.xhtmlux5cux23id_2057}{15}}.
{[}Trans.{]} \emph{Levensspel}. Life game. An important element in
Huizinga's thinking about how culture arises. The forms of life, of
which chivalry is one, arise through \emph{play}, which, as Huizinga
explains in the later \emph{Homo Ludens}, is neither unconscious (people
always know when they are ``playing'' at being a knight or shepherd) nor
nonseriousness. See below, note 65.

\protect\hypertarget{23_NOTES.xhtmlux5cux23id_2056}{\protect\hyperlink{09_Chapter_Two__THE_CRAVING_FOR_A_M.xhtmlux5cux23id_2055}{16}}.
Chastellain, V, p. 364.

\protect\hypertarget{23_NOTES.xhtmlux5cux23id_2054}{\protect\hyperlink{09_Chapter_Two__THE_CRAVING_FOR_A_M.xhtmlux5cux23id_2053}{17}}.
La Marche, IV, p. cxiv.--The old Dutch translation of his Estat de la
maison du duc Charles de Bourgogne by Matthaeus, Analecta, I, pp.
357--494.

\protect\hypertarget{23_NOTES.xhtmlux5cux23id_2052}{\protect\hyperlink{09_Chapter_Two__THE_CRAVING_FOR_A_M.xhtmlux5cux23id_2051}{18}}.
Christine de Pisan, Oeuvres poétiques, ed. M. Roy (Soc. des anciens
texts français), 1886--1896, 3 vols., I, p. 251, no. 38; Leo von
Rozmitals Reise, ed. Schmeller (Bibl. des lit. Vereins zu Stuttgart, t.
VII), 1844, pp. 24, 149.

\protect\hypertarget{23_NOTES.xhtmlux5cux23id_2050}{\protect\hyperlink{09_Chapter_Two__THE_CRAVING_FOR_A_M.xhtmlux5cux23id_2049}{19}}.
La Marche, IV, pp. 4ff.; Chastellain, V, p. 370.

\protect\hypertarget{23_NOTES.xhtmlux5cux23id_2048}{\protect\hyperlink{09_Chapter_Two__THE_CRAVING_FOR_A_M.xhtmlux5cux23id_2047}{20}}.
Chastellain, V, p. 368.

\protect\hypertarget{23_NOTES.xhtmlux5cux23id_2046}{\protect\hyperlink{09_Chapter_Two__THE_CRAVING_FOR_A_M.xhtmlux5cux23id_2045}{21}}.
La Marche, IV, Estat de la maison, pp. 34ff.

\protect\hypertarget{23_NOTES.xhtmlux5cux23id_2044}{\protect\hyperlink{09_Chapter_Two__THE_CRAVING_FOR_A_M.xhtmlux5cux23id_2043}{22}}.
La Marche, I, p. 277.

\protect\hypertarget{23_NOTES.xhtmlux5cux23id_2042}{\protect\hyperlink{09_Chapter_Two__THE_CRAVING_FOR_A_M.xhtmlux5cux23id_2041}{23}}.
La Marche, IV, Estat de la maison, pp. 34, 51, 20, 31.

\protect\hypertarget{23_NOTES.xhtmlux5cux23id_2040}{\protect\hyperlink{09_Chapter_Two__THE_CRAVING_FOR_A_M.xhtmlux5cux23id_2039}{24}}.
Froissart, ed. Luce, III, p. 172.

\protect\hypertarget{23_NOTES.xhtmlux5cux23id_2038}{\protect\hyperlink{09_Chapter_Two__THE_CRAVING_FOR_A_M.xhtmlux5cux23id_2037}{25}}.
Journal d'un bourgeois, §218, p. 105.

\protect\hypertarget{23_NOTES.xhtmlux5cux23id_2036}{\protect\hyperlink{09_Chapter_Two__THE_CRAVING_FOR_A_M.xhtmlux5cux23id_2035}{26}}.
Chronique scandaleuse, I, p. 53.

\protect\hypertarget{23_NOTES.xhtmlux5cux23id_2034}{\protect\hyperlink{09_Chapter_Two__THE_CRAVING_FOR_A_M.xhtmlux5cux23id_2033}{27}}.
Molinet, I, p. 184; Basin, II, p. 376.

\protect\hypertarget{23_NOTES.xhtmlux5cux23id_2032}{\protect\hyperlink{09_Chapter_Two__THE_CRAVING_FOR_A_M.xhtmlux5cux23id_2031}{28}}.
Alienor de Poitiers, Les honneurs de la cour, ed. La Curne de Sainte
Palaye, Mémoires sur l'ancienne chevalerie, 1781, II, p. 201.

\protect\hypertarget{23_NOTES.xhtmlux5cux23id_2030}{\protect\hyperlink{09_Chapter_Two__THE_CRAVING_FOR_A_M.xhtmlux5cux23id_2029}{29}}.
Chastellain, III pp. 196--212, 290, 292, 308, IV pp. 412--414, 428;
Alienor de Poitiers, pp. 209, 212.

\protect\hypertarget{23_NOTES.xhtmlux5cux23id_2028}{\protect\hyperlink{09_Chapter_Two__THE_CRAVING_FOR_A_M.xhtmlux5cux23id_2027}{30}}.
Alienor de Poitiers, p. 210; Chastellain, IV, p. 312; Juvenal des
Ursins, p. 405; La Marche, I, p. 278, Froissart, ed. Luce, I, pp. 16,
22, etc.

\protect\hypertarget{23_NOTES.xhtmlux5cux23id_2026}{\protect\hyperlink{09_Chapter_Two__THE_CRAVING_FOR_A_M.xhtmlux5cux23id_2025}{31}}.
Molinet, V, pp. 194, 192.

\protect\hypertarget{23_NOTES.xhtmlux5cux23id_2024}{\protect\hyperlink{09_Chapter_Two__THE_CRAVING_FOR_A_M.xhtmlux5cux23id_2023}{32}}.
Alienor de Poitiers, p. 190; Deschamps, IX, p. 109.

\protect\hypertarget{23_NOTES.xhtmlux5cux23id_2022}{\protect\hyperlink{09_Chapter_Two__THE_CRAVING_FOR_A_M.xhtmlux5cux23id_2021}{33}}.
Chastellain, V, p. 27--33.

\protect\hypertarget{23_NOTES.xhtmlux5cux23id_2020}{\protect\hyperlink{09_Chapter_Two__THE_CRAVING_FOR_A_M.xhtmlux5cux23id_2019}{34}}.
Only on your account must the priest wait. Deschamps, IX, Le miroir de
mariage, pp. 109--110.

\protect\hypertarget{23_NOTES.xhtmlux5cux23id_2018}{\protect\hyperlink{09_Chapter_Two__THE_CRAVING_FOR_A_M.xhtmlux5cux23id_2017}{35}}.
There are more examples of such ``paix'' in Laborde, II, nos. 43, 45,
75, 126, 140, 5293. The English term, now rare, is ``osculatory.'' The
plate frequently had a figure of Christ or the Virgin painted on it.

\protect\hypertarget{23_NOTES.xhtmlux5cux23id_2016}{\protect\hyperlink{09_Chapter_Two__THE_CRAVING_FOR_A_M.xhtmlux5cux23id_2015}{36}}.
Deschamps, IX, Le miroir de mariage, p. 300, see VIII, p. 156 ballade
no. 1462; Molinet, V, p. 195; Les cent nouvelles nouvelles, ed. Th.
Wright, II, p. 123; see Les Quinze joy es de mariage, p. 185.

\protect\hypertarget{23_NOTES.xhtmlux5cux23id_2014}{\protect\hyperlink{09_Chapter_Two__THE_CRAVING_FOR_A_M.xhtmlux5cux23id_2013}{37}}.
Canonization procedure at Tours, Acta Sanctorum Apr. t. I, p. 152.

\protect\hypertarget{23_NOTES.xhtmlux5cux23id_2012}{\protect\hyperlink{09_Chapter_Two__THE_CRAVING_FOR_A_M.xhtmlux5cux23id_2011}{38}}.
Such quarrels over rank among Dutch nobles, which were already pointed
out by W. Moll, Kerkgeschiedenis van Nederland voor de hervorming
(Utrecht 1864--69), 2 Teile (5 Stücke), II, 3, p. 284, are described in
greater detail by H. Obreen, Bydragen voor Vaderlandsche Geschiedenis en
Oudheidkunde, X\textsuperscript{4}, p. 308.

\protect\hypertarget{23_NOTES.xhtmlux5cux23id_2010}{\protect\hyperlink{09_Chapter_Two__THE_CRAVING_FOR_A_M.xhtmlux5cux23id_2009}{39}}.
Deschamps, IX, pp. 111--114.

\protect\hypertarget{23_NOTES.xhtmlux5cux23page_403}{\protect\hyperlink{09_Chapter_Two__THE_CRAVING_FOR_A_M.xhtmlux5cux23id_2008}{40}}.
Jean de Stavelot, Chronique ed. Borgnet (Coll. des chron. belges) 1861,
p. 96.

\protect\hypertarget{23_NOTES.xhtmlux5cux23id_2007}{\protect\hyperlink{09_Chapter_Two__THE_CRAVING_FOR_A_M.xhtmlux5cux23id_2006}{41}}.
Pierre de Fenin, p. 607; Journal d'un bourgeois, p. 9.

\protect\hypertarget{23_NOTES.xhtmlux5cux23id_2005}{\protect\hyperlink{09_Chapter_Two__THE_CRAVING_FOR_A_M.xhtmlux5cux23id_2004}{42}}.
According to Jevenal des Ursins, p. 543, and Thomas Basin, I. p. 31. The
Journal d'un bourgeois gives another cause for the death sentence, as
does Le Livre des trahisons, ed. Kervyn de Lettenhove (Chron. rel. à
hist. de Belg. sous les ducs de Bourg.), II, p. 138.

\protect\hypertarget{23_NOTES.xhtmlux5cux23id_2003}{\protect\hyperlink{09_Chapter_Two__THE_CRAVING_FOR_A_M.xhtmlux5cux23id_2002}{43}}.
Rel. de S. Denis, I, p. 30; Juvenal des Ursins, p. 341.

\protect\hypertarget{23_NOTES.xhtmlux5cux23id_2001}{\protect\hyperlink{09_Chapter_Two__THE_CRAVING_FOR_A_M.xhtmlux5cux23id_2000}{44}}.
Pierre de Fenin, p. 606; Monstrelet, IV, p. 9.

\protect\hypertarget{23_NOTES.xhtmlux5cux23id_1999}{\protect\hyperlink{09_Chapter_Two__THE_CRAVING_FOR_A_M.xhtmlux5cux23id_1998}{45}}.
Pierre de Fenin. p. 604.

\protect\hypertarget{23_NOTES.xhtmlux5cux23id_1997}{\protect\hyperlink{09_Chapter_Two__THE_CRAVING_FOR_A_M.xhtmlux5cux23id_1996}{46}}.
Christine de Pisan, I, p. 251, no. 38; Chastellain, V, p. 364ff.;
Rozmitals Reise, pp. 24, 149.

\protect\hypertarget{23_NOTES.xhtmlux5cux23id_1995}{\protect\hyperlink{09_Chapter_Two__THE_CRAVING_FOR_A_M.xhtmlux5cux23id_1994}{47}}.
Deschamps, I, nos. 80, 114, 118, II, nos. 256, 266, IV, nos. 800, 803,
V, nos. 1018, 1024, 1029, VII, nos. 253, X, nos. 13, 14.

\protect\hypertarget{23_NOTES.xhtmlux5cux23id_1993}{\protect\hyperlink{09_Chapter_Two__THE_CRAVING_FOR_A_M.xhtmlux5cux23id_1992}{48}}.
Anonymous report from the fifteenth century in Journal de l'inst. hist.,
IV. p. 353; see Juvenal des Ursins, p. 569; Rel. de S. Denis, VI, p.
492.

\protect\hypertarget{23_NOTES.xhtmlux5cux23id_1991}{\protect\hyperlink{09_Chapter_Two__THE_CRAVING_FOR_A_M.xhtmlux5cux23id_1990}{49}}.
Jean Chartier, Hist. de Charles VII, ed. D. Godefroy 1661, p. 318.

\protect\hypertarget{23_NOTES.xhtmlux5cux23id_1989}{\protect\hyperlink{09_Chapter_Two__THE_CRAVING_FOR_A_M.xhtmlux5cux23id_1988}{50}}.
Entry of the Dauphin as duke of Brittany into Rennes 1532, in Th.
Godefroy, Le cérémonial françois 1649, p. 619.

\protect\hypertarget{23_NOTES.xhtmlux5cux23id_1987}{\protect\hyperlink{09_Chapter_Two__THE_CRAVING_FOR_A_M.xhtmlux5cux23id_1986}{51}}.
Rel. de S. Denis, I, p. 32.

\protect\hypertarget{23_NOTES.xhtmlux5cux23id_1985}{\protect\hyperlink{09_Chapter_Two__THE_CRAVING_FOR_A_M.xhtmlux5cux23id_1984}{52}}.
Journal d'un bourgeois, p. 277.

\protect\hypertarget{23_NOTES.xhtmlux5cux23id_1983}{\protect\hyperlink{09_Chapter_Two__THE_CRAVING_FOR_A_M.xhtmlux5cux23id_1982}{53}}.
Thomas Basin, II, p. 9.

\protect\hypertarget{23_NOTES.xhtmlux5cux23id_1981}{\protect\hyperlink{09_Chapter_Two__THE_CRAVING_FOR_A_M.xhtmlux5cux23id_1980}{54}}.
A. Renaudet, Préréforme et humanisme a Paris, p. 11. Based on the
documents of the trial.

\protect\hypertarget{23_NOTES.xhtmlux5cux23id_1979}{\protect\hyperlink{09_Chapter_Two__THE_CRAVING_FOR_A_M.xhtmlux5cux23id_1978}{55}}.
De Laborde, Les ducs de Bourgogne, I, p. 172, 177.

\protect\hypertarget{23_NOTES.xhtmlux5cux23id_1977}{\protect\hyperlink{09_Chapter_Two__THE_CRAVING_FOR_A_M.xhtmlux5cux23id_1976}{56}}.
Livre des trahisons, p. 156.

\protect\hypertarget{23_NOTES.xhtmlux5cux23id_1975}{\protect\hyperlink{09_Chapter_Two__THE_CRAVING_FOR_A_M.xhtmlux5cux23id_1974}{57}}.
Chastellain, I, p. 188.

\protect\hypertarget{23_NOTES.xhtmlux5cux23id_1973}{\protect\hyperlink{09_Chapter_Two__THE_CRAVING_FOR_A_M.xhtmlux5cux23id_1972}{58}}.
Alienor de Poitiers, Les honneurs de la cour, p. 254.

\protect\hypertarget{23_NOTES.xhtmlux5cux23id_1971}{\protect\hyperlink{09_Chapter_Two__THE_CRAVING_FOR_A_M.xhtmlux5cux23id_1970}{59}}.
Rel. de S. Denis, II, p. 114.

\protect\hypertarget{23_NOTES.xhtmlux5cux23id_1969}{\protect\hyperlink{09_Chapter_Two__THE_CRAVING_FOR_A_M.xhtmlux5cux23id_1968}{60}}.
Chastellain, I p. 49, V p. 240; see La Marche, I, p. 201; Monstrelet,
III, p. 358; Lefèvre de S. Remy, I, p. 380.

\protect\hypertarget{23_NOTES.xhtmlux5cux23id_1967}{\protect\hyperlink{09_Chapter_Two__THE_CRAVING_FOR_A_M.xhtmlux5cux23id_1966}{61}}.
Chastellain, V, p. 228; see IV, p. 210.

\protect\hypertarget{23_NOTES.xhtmlux5cux23id_1965}{\protect\hyperlink{09_Chapter_Two__THE_CRAVING_FOR_A_M.xhtmlux5cux23id_1964}{62}}.
Chastellain, III, p. 296; IV, p. 213, 216.

\protect\hypertarget{23_NOTES.xhtmlux5cux23id_1963}{\protect\hyperlink{09_Chapter_Two__THE_CRAVING_FOR_A_M.xhtmlux5cux23id_1962}{63}}.
Chronique scandaleuse, interpol., II, p. 332.

\protect\hypertarget{23_NOTES.xhtmlux5cux23id_1961}{\protect\hyperlink{09_Chapter_Two__THE_CRAVING_FOR_A_M.xhtmlux5cux23id_1960}{64}}.
Lettres de Louis XI, X, p. 110.

\protect\hypertarget{23_NOTES.xhtmlux5cux23id_1959}{\protect\hyperlink{09_Chapter_Two__THE_CRAVING_FOR_A_M.xhtmlux5cux23id_1958}{65}}.
{[}Trans.{]} In this sentence is the kernel of Huizinga's theory of the
role of play in culture that was later to be elaborated in \emph{Homo
Ludens}. Mourning customs are ``play'' in the sense that they enable us
to deal with an otherwise crushing reality. To Huizinga, such forms are
never unconsciously performed; the player always recognizes the game
just as the actor in his thick-soled \emph{corthurni} never confuses
himself with the role he is playing.

The theatrical metaphor is important. See below, chapter 13, p. 342.

\emph{\protect\hypertarget{23_NOTES.xhtmlux5cux23id_1957}{\protect\hyperlink{09_Chapter_Two__THE_CRAVING_FOR_A_M.xhtmlux5cux23id_1956}{66}}}.
Alienor de Poitiers, Les honneurs de la cour, pp. 254--256.

\protect\hypertarget{23_NOTES.xhtmlux5cux23id_1955}{\protect\hyperlink{09_Chapter_Two__THE_CRAVING_FOR_A_M.xhtmlux5cux23id_1954}{67}}.
Lefèvre de S. Remy, II, p. 11; Pierre de Fenin, pp. 599, 605;
Monstrelet, III, p. 347; Theod. Pauli, De rebus actis sub ducibus
Burgundiae compendium, ed Kervyn de Lettenhove (Chron. rel. à l'hist. de
Belg. sous dom. des ducs de Bourg, t. III), p. 267.

\protect\hypertarget{23_NOTES.xhtmlux5cux23page_404}{\protect\hyperlink{09_Chapter_Two__THE_CRAVING_FOR_A_M.xhtmlux5cux23id_1953}{68}}.
{[}Trans.{]} \emph{vuurmand}. A kind of cabinet warmed by coals and used
to dry an infant's linens and blankets. For this information we are
indebted to Helen Roozen of Mt. Vernon, Washington, and her sister
Jeanne Roozen of Heemstede, Holland.

\protect\hypertarget{23_NOTES.xhtmlux5cux23id_1952}{\protect\hyperlink{09_Chapter_Two__THE_CRAVING_FOR_A_M.xhtmlux5cux23id_1951}{69}}.
Alienor de Poitiers, pp. 217--245; Laborde, II, p. 267, Inventory of
1420.

\protect\hypertarget{23_NOTES.xhtmlux5cux23id_1950}{\protect\hyperlink{09_Chapter_Two__THE_CRAVING_FOR_A_M.xhtmlux5cux23id_1949}{70}}.
Successor to Monstrelet, 1449 (Chastellain, V, p. 367).

\protect\hypertarget{23_NOTES.xhtmlux5cux23id_1948}{\protect\hyperlink{09_Chapter_Two__THE_CRAVING_FOR_A_M.xhtmlux5cux23id_1947}{71}}.
See Petit Dutaillis, Documents nouveaux sur les moeurs populaires, etc.,
p. 14; La Curne de S. Palaye, Mémoires sur l'ancienne chevalerie, I, p.
272.

\protect\hypertarget{23_NOTES.xhtmlux5cux23id_1946}{\protect\hyperlink{09_Chapter_Two__THE_CRAVING_FOR_A_M.xhtmlux5cux23id_1945}{72}}.
Chastellain. Le Pas de la mort, VI, p. 61.

\protect\hypertarget{23_NOTES.xhtmlux5cux23id_1944}{\protect\hyperlink{09_Chapter_Two__THE_CRAVING_FOR_A_M.xhtmlux5cux23id_1943}{73}}.
Hefele, Der h. Bernhardin v. Siena etc. p. 42. On the prosecution of
sodomy in France, Jacques du Clercq, II, pp. 272, 282, 337, 338, 350,
III, p. 15.

\protect\hypertarget{23_NOTES.xhtmlux5cux23id_1942}{\protect\hyperlink{09_Chapter_Two__THE_CRAVING_FOR_A_M.xhtmlux5cux23id_1941}{74}}.
Thomas Walsingham, Historia Anglicana, II, 148 (Rolls series ed. H. T.
Riley, 1864). In the case of Henry II of France, the guilty nature of
the mignons is not to be doubted, but this happens at the end of the
sixteenth century.

\protect\hypertarget{23_NOTES.xhtmlux5cux23id_1940}{\protect\hyperlink{09_Chapter_Two__THE_CRAVING_FOR_A_M.xhtmlux5cux23id_1939}{75}}.
Philippe de Commines, Mémoires, ed. B. de Mandrot (Coll. de textes pour
servir a l'enseignement de l'histoire) 1901--1913, 2 vols., I, p. 316.

\protect\hypertarget{23_NOTES.xhtmlux5cux23id_1938}{\protect\hyperlink{09_Chapter_Two__THE_CRAVING_FOR_A_M.xhtmlux5cux23id_1937}{76}}.
La Marche, II, p. 425; Molinet, II, pp. 29, 280; Chastellain, IV, p. 41.

\protect\hypertarget{23_NOTES.xhtmlux5cux23id_1936}{\protect\hyperlink{09_Chapter_Two__THE_CRAVING_FOR_A_M.xhtmlux5cux23id_1935}{77}}.
Les cent nouvelles, II, p. 61; Froissart, ed. Kervyn, XI, p. 93.

\protect\hypertarget{23_NOTES.xhtmlux5cux23id_1934}{\protect\hyperlink{09_Chapter_Two__THE_CRAVING_FOR_A_M.xhtmlux5cux23id_1933}{78}}.
Froissart, ed. Kervyn, ib. XIV, p. 318; Le livre des faits de Jacques de
Lalaing, p. 29, 247 (Chastellain, VIII); La Marche, I, p. 268;
L'hystoire du petit Jehan de Saintré, chap. 47.

\protect\hypertarget{23_NOTES.xhtmlux5cux23id_1932}{\protect\hyperlink{09_Chapter_Two__THE_CRAVING_FOR_A_M.xhtmlux5cux23id_1931}{79}}.
Chastellain, IV, p. 237.

\textbf{\emph{Chapter 3}}

\protect\hypertarget{23_NOTES.xhtmlux5cux23id_2466}{\protect\hyperlink{10_Chapter_Three__THE_HEROIC_DREAM.xhtmlux5cux23id_2465}{*\textsuperscript{1}}}
estates of body and mouth

\protect\hypertarget{23_NOTES.xhtmlux5cux23id_2468}{\protect\hyperlink{10_Chapter_Three__THE_HEROIC_DREAM.xhtmlux5cux23id_2467}{*\textsuperscript{2}}}
``Coming to the third estate, which completes the kingdom, it is the
estate of the good towns, of merchants and of laboring men, of whom it
is not becoming to give such a long exposition as of the others, because
it is hardly possible to attribute great qualities to them, as they are
of servile degree. {[}O Flemish folk!{]}''

\protect\hypertarget{23_NOTES.xhtmlux5cux23id_2470}{\protect\hyperlink{10_Chapter_Three__THE_HEROIC_DREAM.xhtmlux5cux23id_2469}{*\textsuperscript{3}}}
``this rebellious rustic brewer''

\protect\hypertarget{23_NOTES.xhtmlux5cux23id_2472}{\protect\hyperlink{10_Chapter_Three__THE_HEROIC_DREAM.xhtmlux5cux23id_2471}{†\textsuperscript{4}}}
``and such a naughty villein too.''

\protect\hypertarget{23_NOTES.xhtmlux5cux23id_2474}{\protect\hyperlink{10_Chapter_Three__THE_HEROIC_DREAM.xhtmlux5cux23id_2473}{‡\textsuperscript{5}}}
The innocent must starve;/In this way the big wolves fill their belly
every day, /Who by thousands and hundreds/Hoard ill-gotten treasures; it
is the grain, it is the corn,/The blood and the bones with which the
soil is tilled/By the poor people, and their spirits cry/To God for
vengeance and woe to lordship .~.~.

\protect\hypertarget{23_NOTES.xhtmlux5cux23id_2476}{\protect\hyperlink{10_Chapter_Three__THE_HEROIC_DREAM.xhtmlux5cux23id_2475}{*\textsuperscript{6}}}
``the prince knows nothing of this''

\protect\hypertarget{23_NOTES.xhtmlux5cux23id_2478}{\protect\hyperlink{10_Chapter_Three__THE_HEROIC_DREAM.xhtmlux5cux23id_2477}{†\textsuperscript{7}}}
``poor sheep, poor foolish people''

\protect\hypertarget{23_NOTES.xhtmlux5cux23id_2480}{\protect\hyperlink{10_Chapter_Three__THE_HEROIC_DREAM.xhtmlux5cux23id_2479}{‡\textsuperscript{8}}}
``The poor man will not have bread to eat, except perhaps a handful of
rye or barley; his poor wife will lie in and they will have four or six
little ones about the hearth or the oven, which perchance will be warm;
they will ask for bread, they will scream, mad with hunger. The poor
mother will have but a very little salted bread to stuff between their
teeth. Now such misery ought to suffice; but no;---the plunderers will
come who will seek everything. .~.~. Everything will be taken and
snapped up; and we need not ask who pays.''

\protect\hypertarget{23_NOTES.xhtmlux5cux23id_2310}{\protect\hyperlink{10_Chapter_Three__THE_HEROIC_DREAM.xhtmlux5cux23id_2309}{*\textsuperscript{9}}}
O God, see the indigence of the common people,/Provide for it with all
speed:/Alas! with hunger, cold, fear, and misery they tremble. /If they
have sinned or have been negligent/Towards you, they pray indulgence./Is
it not a pity they have lost their goods?/They have no more corn to take
to the mill,/From them their wool and linen are taken,/Water, nothing
more, they have left to drink.

\protect\hypertarget{23_NOTES.xhtmlux5cux23id_2482}{\protect\hyperlink{10_Chapter_Three__THE_HEROIC_DREAM.xhtmlux5cux23id_2481}{*\textsuperscript{10}}}
Whence does sovereign nobility come to one?/From a gentle heart, adorned
by noble morals/ .~.~. No one is a villein unless it comes from his
heart.

\protect\hypertarget{23_NOTES.xhtmlux5cux23id_2484}{\protect\hyperlink{10_Chapter_Three__THE_HEROIC_DREAM.xhtmlux5cux23id_2483}{*\textsuperscript{11}}}
Children, children, from me, Adam, born,/Who after God am the first
father./Created by him, you are descended from me,/Naturally of my rib
and of Eve;/She was your mother. How is it that one is a villein/And the
other takes the name of gentility,/Of you brothers? Whence comes such
nobility?/I do not know, unless it comes from virtues,/And the villains
from all \emph{vice} which wounds:/You are all covered by the same
skin./When God made me out of the mud where I lay,/Mortal man, feeble,
heavy and vain,/Eve from me, He created us quite nude,/But the
imperishable spirit gave us/in abundance; We were hungry and thirsty
afterwards,/Labor, pain, and children in sorrow;/For our sins, children
are born in pain/by all women; Vilely you are conceived./Whence comes
this name: Villein, that wounds the heart?/You are all covered by the
same skin. /The mighty kings, the counts and the dukes, /The governor of
the people and sovereign,/When they are born, with what are they
clothed?/By a dirty skin./ .~.~. Prince, remember without disdaining/The
poor people, that death holds the reins.

\protect\hypertarget{23_NOTES.xhtmlux5cux23id_2486}{\protect\hyperlink{10_Chapter_Three__THE_HEROIC_DREAM.xhtmlux5cux23id_2485}{*\textsuperscript{12}}}
``knighthood and learning which go very well together.''

\protect\hypertarget{23_NOTES.xhtmlux5cux23id_2488}{\protect\hyperlink{10_Chapter_Three__THE_HEROIC_DREAM.xhtmlux5cux23id_2487}{*\textsuperscript{13}}}
``the first deed of knighthood and chivalrous prowess that was ever
achieved.''

\protect\hypertarget{23_NOTES.xhtmlux5cux23id_2490}{\protect\hyperlink{10_Chapter_Three__THE_HEROIC_DREAM.xhtmlux5cux23id_2489}{†\textsuperscript{14}}}
``terrestrial knighthood and human chivalry''

\protect\hypertarget{23_NOTES.xhtmlux5cux23id_2492}{\protect\hyperlink{10_Chapter_Three__THE_HEROIC_DREAM.xhtmlux5cux23id_2491}{*\textsuperscript{15}}}
``The glory of princes consists in pride and in undertaking exceedingly
dangerous things; all princely expressions of power converge in a single
point which we call pride.''

\protect\hypertarget{23_NOTES.xhtmlux5cux23id_2494}{\protect\hyperlink{10_Chapter_Three__THE_HEROIC_DREAM.xhtmlux5cux23id_2493}{†\textsuperscript{16}}}
``among the profound sentiments of man there is none more apt to be
transformed into probity, patriotism and conscience, for a proud man
feels the need of self-respect, and to obtain it, he is led to deserve
it.''

\protect\hypertarget{23_NOTES.xhtmlux5cux23id_2496}{\protect\hyperlink{10_Chapter_Three__THE_HEROIC_DREAM.xhtmlux5cux23id_2495}{*\textsuperscript{17}}}
Honor urges every noble nature/To love all that is noble in its own
essence. / Nobility adds uprightness to it.

\protect\hypertarget{23_NOTES.xhtmlux5cux23id_2498}{\protect\hyperlink{10_Chapter_Three__THE_HEROIC_DREAM.xhtmlux5cux23id_2497}{*\textsuperscript{18}}}
``and he maintained the discipline of chivalry very well, as did the
Romans formerly.''

\protect\hypertarget{23_NOTES.xhtmlux5cux23id_2502}{\protect\hyperlink{10_Chapter_Three__THE_HEROIC_DREAM.xhtmlux5cux23id_2501}{†\textsuperscript{19}}}
``lofty stories of Rome''

\protect\hypertarget{23_NOTES.xhtmlux5cux23id_2500}{\protect\hyperlink{10_Chapter_Three__THE_HEROIC_DREAM.xhtmlux5cux23id_2499}{‡\textsuperscript{20}}}
``whom he wished to follow and imitate.''

\protect\hypertarget{23_NOTES.xhtmlux5cux23id_2504}{\protect\hyperlink{10_Chapter_Three__THE_HEROIC_DREAM.xhtmlux5cux23id_2503}{*\textsuperscript{21}}}
``He desired the great glory of fame, which more than anything else led
him to undertake his wars; and longed to resemble those ancient princes
who have been so much talked of after their death.''

\protect\hypertarget{23_NOTES.xhtmlux5cux23id_2506}{\protect\hyperlink{10_Chapter_Three__THE_HEROIC_DREAM.xhtmlux5cux23id_2505}{†\textsuperscript{22}}}
``And then I perceived that he had set his heart on high and singular
purposes for the future, and on acquiring glory and renown by
extraordinary works.''

\protect\hypertarget{23_NOTES.xhtmlux5cux23id_2508}{\protect\hyperlink{10_Chapter_Three__THE_HEROIC_DREAM.xhtmlux5cux23id_2507}{*\textsuperscript{23}}}
``the valiant dead---Roman or otherwise.''

\protect\hypertarget{23_NOTES.xhtmlux5cux23id_2510}{\protect\hyperlink{10_Chapter_Three__THE_HEROIC_DREAM.xhtmlux5cux23id_2509}{†\textsuperscript{24}}}
``My lord, who are those two women to whom you bowed so low?''
``Huguenin,'' said he, ``I do not know.'' Then he said to him: ``My
lord, they are whores.'' ``Whores, you say,'' said he, ``Huguenin, I
would rather have saluted ten whores than to have omitted saluting one
respectable woman.''

\protect\hypertarget{23_NOTES.xhtmlux5cux23id_2512}{\protect\hyperlink{10_Chapter_Three__THE_HEROIC_DREAM.xhtmlux5cux23id_2511}{‡\textsuperscript{25}}}
``What you will''

\protect\hypertarget{23_NOTES.xhtmlux5cux23id_2514}{\protect\hyperlink{10_Chapter_Three__THE_HEROIC_DREAM.xhtmlux5cux23id_2513}{*\textsuperscript{26}}}
``It is a joyous thing, is war. .~.~. You love your comrade so in war.
When you see that your quarrel is just and your blood is fighting well,
tears come to your eyes. A great sweet feeling of loyalty and of pity
fills your heart on seeing your friend so valiantly exposing his body to
execute and accomplish the command of our creator. And then you prepare
to go and die or live with him, and for love not to abandon him. And out
of that, there arises such a delectation, that he who has not tasted it
is not fit to say what a delight it is. Do you think that a man who does
that fears death? Not at all; for he feels so strengthened, he is so
elated, that he does not know where he is. Truly he is afraid of
nothing.''

\protect\hypertarget{23_NOTES.xhtmlux5cux23id_2516}{\protect\hyperlink{10_Chapter_Three__THE_HEROIC_DREAM.xhtmlux5cux23id_2515}{*\textsuperscript{27}}}
``he served all, honored all, for the love of one. His speech was
graceful, courteous and diffident before his lady.''

\protect\hypertarget{23_NOTES.xhtmlux5cux23id_2518}{\protect\hyperlink{10_Chapter_Three__THE_HEROIC_DREAM.xhtmlux5cux23id_2517}{*\textsuperscript{28}}}
When we are in the tavern, drinking strong wine,/When the ladies pass
and look at us,/With those white throats and those tight bodices,/Those
sparkling eyes resplendent with smiling beauty,/Nature urges us to have
desiring hearts,/ .~.~. Then we could conquer Yaumont and Agoulant/And
the others would conquer Olivier and Rollant./But when we are in camp on
our trotting chargers,/ Our bucklers round our necks and our lances
lowered,/And the great cold freezes us all together,/And our limbs are
crushed before and behind,/And our enemies are approaching us,/Then we
should wish to be in a cellar so large,/That we might not be seen by any
means.

\protect\hypertarget{23_NOTES.xhtmlux5cux23id_2520}{\protect\hyperlink{10_Chapter_Three__THE_HEROIC_DREAM.xhtmlux5cux23id_2519}{*\textsuperscript{29}}}
``Alas, where are women to inspire us, to fire us to bravery, or to
charge us with tokens, insignia, scarves, and veils!''

\protect\hypertarget{23_NOTES.xhtmlux5cux23id_2522}{\protect\hyperlink{10_Chapter_Three__THE_HEROIC_DREAM.xhtmlux5cux23id_2521}{*\textsuperscript{30}}}
``and they went to the battlefield for, I don't know, whatever foolish
enterprise.''

\protect\hypertarget{23_NOTES.xhtmlux5cux23id_2524}{\protect\hyperlink{10_Chapter_Three__THE_HEROIC_DREAM.xhtmlux5cux23id_2523}{*\textsuperscript{31}}}
``the Fountain of Tears, the Tree of Charlemagne''

\protect\hypertarget{23_NOTES.xhtmlux5cux23id_2525}{\protect\hyperlink{10_Chapter_Three__THE_HEROIC_DREAM.xhtmlux5cux23id_2526}{†\textsuperscript{32}}}
``lady of the secret island''

\protect\hypertarget{23_NOTES.xhtmlux5cux23id_2528}{\protect\hyperlink{10_Chapter_Three__THE_HEROIC_DREAM.xhtmlux5cux23id_2527}{‡\textsuperscript{33}}}
``noble knight, slave and servant of the beautiful giantess with the
blonde wig, the greatest in the world.''

\protect\hypertarget{23_NOTES.xhtmlux5cux23id_2530}{\protect\hyperlink{10_Chapter_Three__THE_HEROIC_DREAM.xhtmlux5cux23id_2529}{§\textsuperscript{34}}}
``the white knight,'' ``the unknown knight,'' ``the knight with the
cape''

\protect\hypertarget{23_NOTES.xhtmlux5cux23id_2532}{\protect\hyperlink{10_Chapter_Three__THE_HEROIC_DREAM.xhtmlux5cux23id_2531}{*\textsuperscript{35}}}
Not for amusement, nor for recreation,/But for the purpose that
praise/Be given to God in the first place,/And glory and high fame to
the good.

\protect\hypertarget{23_NOTES.xhtmlux5cux23id_2534}{\protect\hyperlink{10_Chapter_Three__THE_HEROIC_DREAM.xhtmlux5cux23id_2533}{†\textsuperscript{36}}}
``did not, as is said, institute this order for vain purposes.''

\protect\hypertarget{23_NOTES.xhtmlux5cux23id_2536}{\protect\hyperlink{10_Chapter_Three__THE_HEROIC_DREAM.xhtmlux5cux23id_2535}{*\textsuperscript{37}}}
Detestable to God and to men/Is lying and treason,/For this reason, not
placed in the gallery/Of worthies is the image of Jason,/Who to carry
off the fleece/of Cholchis was willing to perjure./Larceny cannot remain
hidden.

\protect\hypertarget{23_NOTES.xhtmlux5cux23id_2538}{\protect\hyperlink{10_Chapter_Three__THE_HEROIC_DREAM.xhtmlux5cux23id_2537}{*\textsuperscript{38}}}
``a very savage rule of order''

\protect\hypertarget{23_NOTES.xhtmlux5cux23id_2540}{\protect\hyperlink{10_Chapter_Three__THE_HEROIC_DREAM.xhtmlux5cux23id_2539}{†\textsuperscript{39}}}
``and I firmly believe that these Galois and Galoises, who died in this
manner, were martyrs of love.''

\protect\hypertarget{23_NOTES.xhtmlux5cux23id_2542}{\protect\hyperlink{10_Chapter_Three__THE_HEROIC_DREAM.xhtmlux5cux23id_2541}{*\textsuperscript{40}}}
``My beauty, is it well closed?'' ``Yes, certainly.''

\protect\hypertarget{23_NOTES.xhtmlux5cux23id_2544}{\protect\hyperlink{10_Chapter_Three__THE_HEROIC_DREAM.xhtmlux5cux23id_2543}{†\textsuperscript{41}}}
Now come what may, for it is not otherwise./---Then the gentle girl took
away her finger, /And the eye remained closed, as the people saw.

\protect\hypertarget{23_NOTES.xhtmlux5cux23id_2872}{\protect\hyperlink{10_Chapter_Three__THE_HEROIC_DREAM.xhtmlux5cux23id_2871}{*\textsuperscript{42}}}
Now then, said the queen, I have well known for a long time/That I am
with child, my body has felt it;/It has already turned within my
body./And I vow, and promise God who created me .~.~. /That this, my
fruit, shall not exit my body,/Until you have taken me into the land
over there,/And fulfilled the vow which you have sworn,/And if it is to
be born before this has been done,/ Then I will kill myself with a big
steel knife./My soul will be lost and the fruit will perish.

\protect\hypertarget{23_NOTES.xhtmlux5cux23id_2874}{\protect\hyperlink{10_Chapter_Three__THE_HEROIC_DREAM.xhtmlux5cux23id_2873}{†\textsuperscript{43}}}
And when the king had heard, he thought seriously about it, /And then he
said, Now then, there will be no more vows.

\protect\hypertarget{23_NOTES.xhtmlux5cux23id_2876}{\protect\hyperlink{10_Chapter_Three__THE_HEROIC_DREAM.xhtmlux5cux23id_2875}{*\textsuperscript{44}}}
``with the desire of avoiding idleness, and with the thought, thereby,
to obtain honor and the esteem of the very beautiful whose servants we
are.''

\protect\hypertarget{23_NOTES.xhtmlux5cux23id_2878}{\protect\hyperlink{10_Chapter_Three__THE_HEROIC_DREAM.xhtmlux5cux23id_2877}{†\textsuperscript{45}}}
``to the death.''

\protect\hypertarget{23_NOTES.xhtmlux5cux23id_2880}{\protect\hyperlink{10_Chapter_Three__THE_HEROIC_DREAM.xhtmlux5cux23id_2879}{*\textsuperscript{46}}}
``It is not the pleasure of my very redoubted lord that Messire Philippe
Pot undertakes, in his company, the holy votive journey with his arm
bare; but he desires that he should travel with him well and
sufficiently armed as is becoming.''

\protect\hypertarget{23_NOTES.xhtmlux5cux23id_2882}{\protect\hyperlink{10_Chapter_Three__THE_HEROIC_DREAM.xhtmlux5cux23id_2881}{*\textsuperscript{47}}}
``if she wants.''

\protect\hypertarget{23_NOTES.xhtmlux5cux23id_2884}{\protect\hyperlink{10_Chapter_Three__THE_HEROIC_DREAM.xhtmlux5cux23id_2883}{†\textsuperscript{48}}}
``poor squire''

\protect\hypertarget{23_NOTES.xhtmlux5cux23id_2886}{\protect\hyperlink{10_Chapter_Three__THE_HEROIC_DREAM.xhtmlux5cux23id_2885}{*\textsuperscript{49}}}
``Show favor to Zion and grant her prosperity; rebuild the walls of
Jerusalem.''

\protect\hypertarget{23_NOTES.xhtmlux5cux23id_2888}{\protect\hyperlink{10_Chapter_Three__THE_HEROIC_DREAM.xhtmlux5cux23id_2887}{†\textsuperscript{50}}}
``if it had pleased God, his creator, to allow him a full span of
years.''

\protect\hypertarget{23_NOTES.xhtmlux5cux23id_2890}{\protect\hyperlink{10_Chapter_Three__THE_HEROIC_DREAM.xhtmlux5cux23id_2889}{‡\textsuperscript{51}}}
``the voyage to Turkey''

\protect\hypertarget{23_NOTES.xhtmlux5cux23id_2892}{\protect\hyperlink{10_Chapter_Three__THE_HEROIC_DREAM.xhtmlux5cux23id_2891}{*\textsuperscript{52}}}
``to prevent the shedding of the blood of Christians and the destruction
of the people on whom my heart has compassion .~.~. that by my own body
this quarrel might be settled, without proceeding by means of wars,
which would entail that many noblemen and others, both of your army and
of mine, would end their days pitifully.''

\protect\hypertarget{23_NOTES.xhtmlux5cux23id_2894}{\protect\hyperlink{10_Chapter_Three__THE_HEROIC_DREAM.xhtmlux5cux23id_2893}{†\textsuperscript{53}}}
``man to man''

\protect\hypertarget{23_NOTES.xhtmlux5cux23id_2896}{\protect\hyperlink{10_Chapter_Three__THE_HEROIC_DREAM.xhtmlux5cux23id_2895}{*\textsuperscript{54}}}
``most beautiful ceremony''

\protect\hypertarget{23_NOTES.xhtmlux5cux23id_2898}{\protect\hyperlink{10_Chapter_Three__THE_HEROIC_DREAM.xhtmlux5cux23id_2897}{†\textsuperscript{55}}}
``O, my lord of Burgundy, I have served you well in your war against
Ghent! O my lord, for God's sake, I beg for mercy, save my life!''

\protect\hypertarget{23_NOTES.xhtmlux5cux23id_2900}{\protect\hyperlink{10_Chapter_Three__THE_HEROIC_DREAM.xhtmlux5cux23id_2899}{*\textsuperscript{56}}}
``as the chief guardian of the very laudable ceremonies of honor,''

\protect\hypertarget{23_NOTES.xhtmlux5cux23id_2902}{\protect\hyperlink{10_Chapter_Three__THE_HEROIC_DREAM.xhtmlux5cux23id_2901}{*\textsuperscript{57}}}
``If we were to seek another path to the fight .~.~. we would show that
we are not proper knights.''

\protect\hypertarget{23_NOTES.xhtmlux5cux23id_2904}{\protect\hyperlink{10_Chapter_Three__THE_HEROIC_DREAM.xhtmlux5cux23id_2903}{†\textsuperscript{58}}}
``Some held it a prowess, and some held it to be a shame and a great
overbearing.''

\protect\hypertarget{23_NOTES.xhtmlux5cux23id_2906}{\protect\hyperlink{10_Chapter_Three__THE_HEROIC_DREAM.xhtmlux5cux23id_2905}{*\textsuperscript{59}}}
``From this day on, this encounter was called the struggle of Mons en
Vimeu. It was not, however, declared to be a battle because the parties
only encountered one another by chance and no flags were displayed.''

\protect\hypertarget{23_NOTES.xhtmlux5cux23id_2908}{\protect\hyperlink{10_Chapter_Three__THE_HEROIC_DREAM.xhtmlux5cux23id_2907}{†\textsuperscript{60}}}
``because all battles should take the name of the nearest fortress.''

\protect\hypertarget{23_NOTES.xhtmlux5cux23id_2910}{\protect\hyperlink{10_Chapter_Three__THE_HEROIC_DREAM.xhtmlux5cux23id_2909}{*\textsuperscript{61}}}
``Then the king fought for a very long time with Monsigneur Utsasse, and
he with him, so that it was a great pleasure to see.''

\protect\hypertarget{23_NOTES.xhtmlux5cux23id_2912}{\protect\hyperlink{10_Chapter_Three__THE_HEROIC_DREAM.xhtmlux5cux23id_2911}{†\textsuperscript{62}}}
``When he had looked at it a long time, it was taken from that place and
hanged on a tree. This was the last end of Phillippe d'Artevelle.''

\protect\hypertarget{23_NOTES.xhtmlux5cux23id_2914}{\protect\hyperlink{10_Chapter_Three__THE_HEROIC_DREAM.xhtmlux5cux23id_2913}{‡\textsuperscript{63}}}
``in which he treated him like a villain.''

\protect\hypertarget{23_NOTES.xhtmlux5cux23id_2916}{\protect\hyperlink{10_Chapter_Three__THE_HEROIC_DREAM.xhtmlux5cux23id_2915}{*\textsuperscript{64}}}
``a gentleman.''

\protect\hypertarget{23_NOTES.xhtmlux5cux23id_2918}{\protect\hyperlink{10_Chapter_Three__THE_HEROIC_DREAM.xhtmlux5cux23id_2917}{†\textsuperscript{65}}}
``because there is danger and loss of life and God knows how awful it is
when there is a storm and there is sea sickness that many people find
hard to bear. Beyond that, look at the hard life which must be endured
and does not become nobility.''

\protect\hypertarget{23_NOTES.xhtmlux5cux23id_2920}{\protect\hyperlink{10_Chapter_Three__THE_HEROIC_DREAM.xhtmlux5cux23id_2919}{‡\textsuperscript{66}}}
``in the style of Burgundy''

\protect\hypertarget{23_NOTES.xhtmlux5cux23id_2922}{\protect\hyperlink{10_Chapter_Three__THE_HEROIC_DREAM.xhtmlux5cux23id_2921}{*\textsuperscript{67}}}
``I am a poor man who desires advancement,''

\protect\hypertarget{23_NOTES.xhtmlux5cux23id_2924}{\protect\hyperlink{10_Chapter_Three__THE_HEROIC_DREAM.xhtmlux5cux23id_2923}{†\textsuperscript{68}}}
``who desire to advance themselves by arms.''

\protect\hypertarget{23_NOTES.xhtmlux5cux23id_2926}{\protect\hyperlink{10_Chapter_Three__THE_HEROIC_DREAM.xhtmlux5cux23id_2925}{‡\textsuperscript{69}}}
And when will the paymaster come?

\protect\hypertarget{23_NOTES.xhtmlux5cux23id_2928}{\protect\hyperlink{10_Chapter_Three__THE_HEROIC_DREAM.xhtmlux5cux23id_2927}{§\textsuperscript{70}}}
A nobleman of twenty thalers

\protect\hypertarget{23_NOTES.xhtmlux5cux23id_3082}{\protect\hyperlink{10_Chapter_Three__THE_HEROIC_DREAM.xhtmlux5cux23id_3081}{*\textsuperscript{71}}}
``My opinion is, if he had gone out on this night, he would have been
behaving well .~.~. but really, when honor came into question he would
not have liked to be accused of cowardice.''

\protect\hypertarget{23_NOTES.xhtmlux5cux23id_3084}{\protect\hyperlink{10_Chapter_Three__THE_HEROIC_DREAM.xhtmlux5cux23id_3083}{*\textsuperscript{72}}}
William, it is your desire to go to Hungary and Turkey and try to fight
with people and countries who have never done anything to us and you
have no reasonable ground to do this other than vain earthly glory. Let
John of Burgundy and our cousins of France go forth on this undertaking
and as for you, go to Friesland and conquer our heritage.''

\protect\hypertarget{23_NOTES.xhtmlux5cux23id_3086}{\protect\hyperlink{10_Chapter_Three__THE_HEROIC_DREAM.xhtmlux5cux23id_3085}{*\textsuperscript{73}}}
Under green leaves, on delightful grass /Near a noisy brook and a clear
fountain/I found a portable hut. /There Gontier took his meal with dame
Helayne/ On fresh cheese, milk, cheese curds,/Cream, cream cheese,
apple, nut, plum, pear,/Garlic and onions, chopped shallots/on a brown
crust, with coarse salt, the better to drink.

\protect\hypertarget{23_NOTES.xhtmlux5cux23id_3088}{\protect\hyperlink{10_Chapter_Three__THE_HEROIC_DREAM.xhtmlux5cux23id_3087}{†\textsuperscript{74}}}
``mouth as well as nose, to the smooth as well as the bearded.''

\protect\hypertarget{23_NOTES.xhtmlux5cux23id_3090}{\protect\hyperlink{10_Chapter_Three__THE_HEROIC_DREAM.xhtmlux5cux23id_3089}{*\textsuperscript{75}}}
I heard Gontier in felling his tree/Thanking God for his security:/``I
do not know,'' said he, ``what are pillars of marble,/Shining pommels,
walls decorated with paintings; /I have no fear of treason hidden/Under
friendly appearances, nor that I shall be poisoned/in a gold cup. I do
not bare my head/Before a tyrant, nor bend my knee./No usher's rod ever
turns me away,/For no covetousness, ambition, greed/entices me./Work
holds me in joyous liberty;/I love Helayne, and she me without fail,/And
that is enough. The tomb does not frighten us.''/ Then I said, ``Alas, a
serf of the court is not worth a farthing,/But free Franc Gontier is
worth a real gem set in gold.''

\protect\hypertarget{23_NOTES.xhtmlux5cux23id_3092}{\protect\hyperlink{10_Chapter_Three__THE_HEROIC_DREAM.xhtmlux5cux23id_3091}{†\textsuperscript{76}}}
Returning from a sovereign's court/Where I had long sojourned,/In a
bush, near a fountain,/I found Robin the free, his head crowned,/With
chaplets of flowers he had adorned/His head, and Marion, his love .~.~.

\protect\hypertarget{23_NOTES.xhtmlux5cux23id_3094}{\protect\hyperlink{10_Chapter_Three__THE_HEROIC_DREAM.xhtmlux5cux23id_3093}{*\textsuperscript{77}}}
.~.~. I will henceforth live/In a middle station, that is my resolve,/To
abandon war and live by labor:/Waging war is but damnation.

\protect\hypertarget{23_NOTES.xhtmlux5cux23id_3096}{\protect\hyperlink{10_Chapter_Three__THE_HEROIC_DREAM.xhtmlux5cux23id_3095}{†\textsuperscript{78}}}
I only ask of God to give me/That in the world I may serve and praise
him,/ Live for myself, my coat or doublet whole,/One horse to carry my
labor./And that I may govern my estate/To no extreme, in grace without
envy,/Without having too much, without begging my bread,/For today, this
is the safest life.

\protect\hypertarget{23_NOTES.xhtmlux5cux23id_3098}{\protect\hyperlink{10_Chapter_Three__THE_HEROIC_DREAM.xhtmlux5cux23id_3097}{‡\textsuperscript{79}}}
.~.~. A working man, a poor teamster/goes ill dressed, torn clothes, ill
shod/ But laboring he takes pleasure in his work/And merrily finishes
it. /At night he sleeps well; and therefore such a loyal heart/Sees four
kings and their reigns end.

\protect\hypertarget{23_NOTES.xhtmlux5cux23id_3100}{\protect\hyperlink{10_Chapter_Three__THE_HEROIC_DREAM.xhtmlux5cux23id_3099}{*\textsuperscript{80}}}
The court is a sea, from which comes /Waves of pride, storms of envy
.~.~. / Wrath stirs up quarrels and outrages, /Which often cause the
ships to sink:/Treason has its part here./Swim elsewhere for your
amusement.

\protect\hypertarget{23_NOTES.xhtmlux5cux23id_1930}{\protect\hyperlink{10_Chapter_Three__THE_HEROIC_DREAM.xhtmlux5cux23id_1929}{1}}.
Deschamps, II, p. 226.

\protect\hypertarget{23_NOTES.xhtmlux5cux23id_1928}{\protect\hyperlink{10_Chapter_Three__THE_HEROIC_DREAM.xhtmlux5cux23id_1927}{2}}.
Chastellain, Le miroir des nobles hommes en France, VI, p. 204.
Exposition sur vérité mal prise, VI, p. 416. L'entrée du roys Loys en
nouveau règne, VII, p. 10.

\protect\hypertarget{23_NOTES.xhtmlux5cux23id_1926}{\protect\hyperlink{10_Chapter_Three__THE_HEROIC_DREAM.xhtmlux5cux23id_1925}{3}}.
Froissart, ed. Kervyn, XIII, p. 22; Jean Germain, Liber de virtutibus
ducis Burg., p. 109; Molinet, I p. 83, III p. 100.

\protect\hypertarget{23_NOTES.xhtmlux5cux23id_1924}{\protect\hyperlink{10_Chapter_Three__THE_HEROIC_DREAM.xhtmlux5cux23id_1923}{4}}.
Monstrelet, II, p. 241.

\protect\hypertarget{23_NOTES.xhtmlux5cux23id_1922}{\protect\hyperlink{10_Chapter_Three__THE_HEROIC_DREAM.xhtmlux5cux23id_1921}{5}}.
Chastellain, VII, pp. 13--16.

\protect\hypertarget{23_NOTES.xhtmlux5cux23id_1920}{\protect\hyperlink{10_Chapter_Three__THE_HEROIC_DREAM.xhtmlux5cux23id_1919}{6}}.
Chastellain, III, p. 82; IV, p. 170; V, pp. 279, 309.

\protect\hypertarget{23_NOTES.xhtmlux5cux23id_1918}{\protect\hyperlink{10_Chapter_Three__THE_HEROIC_DREAM.xhtmlux5cux23id_1917}{7}}.
Jacques du Clercq, II, p. 245, see p. 339.

\protect\hypertarget{23_NOTES.xhtmlux5cux23id_1916}{\protect\hyperlink{10_Chapter_Three__THE_HEROIC_DREAM.xhtmlux5cux23id_1915}{8}}.
See above p. 11.

\protect\hypertarget{23_NOTES.xhtmlux5cux23id_1914}{\protect\hyperlink{10_Chapter_Three__THE_HEROIC_DREAM.xhtmlux5cux23id_1913}{9}}.
Chastellain, III, pp. 82--89.

\protect\hypertarget{23_NOTES.xhtmlux5cux23id_1912}{\protect\hyperlink{10_Chapter_Three__THE_HEROIC_DREAM.xhtmlux5cux23id_1911}{10}}.
{[}Trans.{]} \emph{Gilles de Rais}: Baron of Rais who fought bravely on
the side of Jeanne d'Arc and who was made marshal of France in 1429.
Falling upon hard times, he turned to magic and alchemy, for which he
attempted to atone by holy acts. At the same time he pursued a secret
life of kidnapping, pederasty, sodomy, and murder. His frightful acts
are described in horrific detail in Huysmans's 1891 novel \emph{Là Bas},
which Huizinga read as a young man. Some sources identify Gilles de Rais
with the figure of Bluebeard, but Gilles's crimes were committed against
young boys. Michelet calls him ``bête d'extermination.''

\protect\hypertarget{23_NOTES.xhtmlux5cux23id_1910}{\protect\hyperlink{10_Chapter_Three__THE_HEROIC_DREAM.xhtmlux5cux23id_1909}{11}}.
Chastellain, VIII, pp. 90 ff.

\protect\hypertarget{23_NOTES.xhtmlux5cux23id_1908}{\protect\hyperlink{10_Chapter_Three__THE_HEROIC_DREAM.xhtmlux5cux23id_1907}{12}}.
Chastellain, II, p. 345.

\protect\hypertarget{23_NOTES.xhtmlux5cux23id_1906}{\protect\hyperlink{10_Chapter_Three__THE_HEROIC_DREAM.xhtmlux5cux23id_1905}{13}}.
Deschamps, no. 113, I, p. 230.

\protect\hypertarget{23_NOTES.xhtmlux5cux23page_405}{\protect\hyperlink{10_Chapter_Three__THE_HEROIC_DREAM.xhtmlux5cux23id_1904}{14}}.
Nicholas de Clémanges, Opera, ed. Lydius, Leiden 1613, p. 48. cap. IX.

\protect\hypertarget{23_NOTES.xhtmlux5cux23id_1903}{\protect\hyperlink{10_Chapter_Three__THE_HEROIC_DREAM.xhtmlux5cux23id_1902}{15}}.
In the Latin translation of Gerson, Opera, IV, p. 583--622; the French
text is from 1824, the cited text by D. H. Carnahan, The Ad Deum vadit
of Jean Gerson, University of Illinois studies in language and
literature 1917, III, no. 1, p. 13. See Denifle et Chatelain,
Chartularium Univ. Paris. IV, no. 1819.

\protect\hypertarget{23_NOTES.xhtmlux5cux23id_1901}{\protect\hyperlink{10_Chapter_Three__THE_HEROIC_DREAM.xhtmlux5cux23id_1900}{16}}.
In H. Denifle, La guerre de cent Ans et la désolation des eglises etc.
en France, Paris 1897--99, 2 vols., I, pp. 497--513.

\protect\hypertarget{23_NOTES.xhtmlux5cux23id_1899}{\protect\hyperlink{10_Chapter_Three__THE_HEROIC_DREAM.xhtmlux5cux23id_1898}{17}}.
Alain Chartier, Oeuvres, ed. Duchesne, p. 402.

\protect\hypertarget{23_NOTES.xhtmlux5cux23id_1897}{\protect\hyperlink{10_Chapter_Three__THE_HEROIC_DREAM.xhtmlux5cux23id_1896}{18}}.
Rob. Gaguini Epistole et orationes, ed. L. Thuasne (Bibl. litt. de la
Renaissance), Paris 1903, 2 vols., II pp. 321, 350.

\protect\hypertarget{23_NOTES.xhtmlux5cux23id_1895}{\protect\hyperlink{10_Chapter_Three__THE_HEROIC_DREAM.xhtmlux5cux23id_1894}{19}}.
Froissart, ed. Kervyn, XII, p. 4, Le livre des trahisons, pp. 19, 26;
Chastellain, I p. xxx, III p. 325, V pp. 260, 275, 325, VII, pp.
466--480; Thomas Basin, passim, especially I, pp. 44, 56, 59, 115; see
La complainte du povre commun et des povres laboureurs de France
(Monstrelet, VI, p. 176--190).

\protect\hypertarget{23_NOTES.xhtmlux5cux23id_1893}{\protect\hyperlink{10_Chapter_Three__THE_HEROIC_DREAM.xhtmlux5cux23id_1892}{20}}.
Les Faicts de Dictz de messire Jehan Molinet, Paris, Jehan Petit, 1537,
f. 87 vso.

\protect\hypertarget{23_NOTES.xhtmlux5cux23id_1891}{\protect\hyperlink{10_Chapter_Three__THE_HEROIC_DREAM.xhtmlux5cux23id_1890}{21}}.
Ballade 19, in A. de la Borderie, Jean Meschinot, sa vie et ses oeuvres
(Bibl. de l'école des chartes), LVI, 1895, p. 296; see Les lunettes des
princes, ibid., pp. 607, 613.

\protect\hypertarget{23_NOTES.xhtmlux5cux23id_1889}{\protect\hyperlink{10_Chapter_Three__THE_HEROIC_DREAM.xhtmlux5cux23id_1888}{22}}.
Masselin, Journal des Etats Généraux de France tenus à Tours en 1484,
ed. A. Bernier (Coll. des documents inédits), p. 672.

\protect\hypertarget{23_NOTES.xhtmlux5cux23id_1887}{\protect\hyperlink{10_Chapter_Three__THE_HEROIC_DREAM.xhtmlux5cux23id_1886}{23}}.
Deschamps, VI, no. 1140, p. 67. The link between the idea of equality
and the ``nobility of the heart'' is the point of the words of Ghismonda
to her father Tancred in the first novella of the fourth day in
Boccaccio's Decameron.

\protect\hypertarget{23_NOTES.xhtmlux5cux23id_1885}{\protect\hyperlink{10_Chapter_Three__THE_HEROIC_DREAM.xhtmlux5cux23id_1884}{24}}.
Deschamps, VI, p. 124, no. 1176.

\protect\hypertarget{23_NOTES.xhtmlux5cux23id_1883}{\protect\hyperlink{10_Chapter_Three__THE_HEROIC_DREAM.xhtmlux5cux23id_1882}{25}}.
Molinet, II, p. 104--107; Jean le Maire de Belges, Les chansons de Namur
1507.

\protect\hypertarget{23_NOTES.xhtmlux5cux23id_1881}{\protect\hyperlink{10_Chapter_Three__THE_HEROIC_DREAM.xhtmlux5cux23id_1880}{26}}.
Chastellain, Le miroir des nobles hommes de France, VI, pp. 203, 211,
214.

\protect\hypertarget{23_NOTES.xhtmlux5cux23id_1879}{\protect\hyperlink{10_Chapter_Three__THE_HEROIC_DREAM.xhtmlux5cux23id_1878}{27}}.
Le Jouvencel, ed. C. Favre et L. Lecestre (Soc. de l'hist. de France)
1887--89, 2 vols., I, p. 13.

\protect\hypertarget{23_NOTES.xhtmlux5cux23id_1877}{\protect\hyperlink{10_Chapter_Three__THE_HEROIC_DREAM.xhtmlux5cux23id_1876}{28}}.
Livre des faicts du mareschal de Boucicaut, Petitot, Coll de mém., VI,
p. 375.

\protect\hypertarget{23_NOTES.xhtmlux5cux23id_1875}{\protect\hyperlink{10_Chapter_Three__THE_HEROIC_DREAM.xhtmlux5cux23id_1874}{29}}.
Philippe de Vitri, Le chapel des fleurs de lis (1335), ed. A. Piaget,
Romania XXVII, 1898, pp. 8off.

\protect\hypertarget{23_NOTES.xhtmlux5cux23id_1873}{\protect\hyperlink{10_Chapter_Three__THE_HEROIC_DREAM.xhtmlux5cux23id_1872}{30}}.
Molinet, I, p. 16--17.

\protect\hypertarget{23_NOTES.xhtmlux5cux23id_1871}{\protect\hyperlink{10_Chapter_Three__THE_HEROIC_DREAM.xhtmlux5cux23id_1870}{31}}.
N. Jorga, Philippe de Mézières, p. 469.

\protect\hypertarget{23_NOTES.xhtmlux5cux23id_1869}{\protect\hyperlink{10_Chapter_Three__THE_HEROIC_DREAM.xhtmlux5cux23id_1868}{32}}.
Jorga, Mézières, p. 506.

\protect\hypertarget{23_NOTES.xhtmlux5cux23id_1867}{\protect\hyperlink{10_Chapter_Three__THE_HEROIC_DREAM.xhtmlux5cux23id_1866}{33}}.
Froissart, ed. Luce, I, pp. 2--3; Monstrelet, I, p. 2; d'Escouchy, I, p.
1; Chastellain, I prologue, II p. 116, VI p. 266; La Marche, I, p. 187;
Molinet, I p. 17, II p. 54.

\protect\hypertarget{23_NOTES.xhtmlux5cux23id_1865}{\protect\hyperlink{10_Chapter_Three__THE_HEROIC_DREAM.xhtmlux5cux23id_1864}{34}}.
{[}Trans.{]} \emph{Heralds and Kings of Arms}: Heralds were originally
royal messengers, but not, however, trumpeters. The position evolved
into that of those in charge of tournaments and the regulations
concerning coats of arms. Kings of Arms were the chief heralds of
particular chivalric orders. See Charles MacKinnon of Dunakin,
\emph{Heraldry}.

\protect\hypertarget{23_NOTES.xhtmlux5cux23page_406}{\protect\hyperlink{10_Chapter_Three__THE_HEROIC_DREAM.xhtmlux5cux23id_1863}{35}}.
Lefèvre de S. Remy, II, p. 249; Froissart, ed Luce, I, p. 1; see Le
débat des hérauts d'armes de France et d'Angleterre, ed. L. Pannier et
P. Meyer (Soc. des anciens textes français), 1887, p. 1.

\protect\hypertarget{23_NOTES.xhtmlux5cux23id_1862}{\protect\hyperlink{10_Chapter_Three__THE_HEROIC_DREAM.xhtmlux5cux23id_1861}{36}}.
{[}Trans.{]} \emph{Lefevre de S. Remy}, Toison d'or, King of Arms of the
Order of the Golden Fleece.

\protect\hypertarget{23_NOTES.xhtmlux5cux23id_1860}{\protect\hyperlink{10_Chapter_Three__THE_HEROIC_DREAM.xhtmlux5cux23id_1859}{37}}.
Chastellain, V, p. 443.

\protect\hypertarget{23_NOTES.xhtmlux5cux23id_1858}{\protect\hyperlink{10_Chapter_Three__THE_HEROIC_DREAM.xhtmlux5cux23id_1857}{38}}.
Les origines de la France contemporaine, La révolution, I, p. 190.

\protect\hypertarget{23_NOTES.xhtmlux5cux23id_1856}{\protect\hyperlink{10_Chapter_Three__THE_HEROIC_DREAM.xhtmlux5cux23id_1855}{39}}.
Die Kultur der Renaissance in Italien, X, II, p. 155.

\protect\hypertarget{23_NOTES.xhtmlux5cux23id_1854}{\protect\hyperlink{10_Chapter_Three__THE_HEROIC_DREAM.xhtmlux5cux23id_1853}{40}}.
Burckhardt, Die Kulture, X, I, p. 152--165.

\protect\hypertarget{23_NOTES.xhtmlux5cux23id_1852}{\protect\hyperlink{10_Chapter_Three__THE_HEROIC_DREAM.xhtmlux5cux23id_1851}{41}}.
Froissart, ed. Luce, IV, p. 112; where the name Bamborough, called as
well Bembro or Brembo, is mangled into Brandebourch.

\protect\hypertarget{23_NOTES.xhtmlux5cux23id_1850}{\protect\hyperlink{10_Chapter_Three__THE_HEROIC_DREAM.xhtmlux5cux23id_1849}{42}}.
Le dit de vérité, Chastellain, VI, p. 221.

\protect\hypertarget{23_NOTES.xhtmlux5cux23id_1848}{\protect\hyperlink{10_Chapter_Three__THE_HEROIC_DREAM.xhtmlux5cux23id_1847}{43}}.
Le livre de la paix, Chastellain, VIII, p. 367.

\protect\hypertarget{23_NOTES.xhtmlux5cux23id_1846}{\protect\hyperlink{10_Chapter_Three__THE_HEROIC_DREAM.xhtmlux5cux23id_1845}{44}}.
Froissart, ed Luce, I, p. 3.

\protect\hypertarget{23_NOTES.xhtmlux5cux23id_1844}{\protect\hyperlink{10_Chapter_Three__THE_HEROIC_DREAM.xhtmlux5cux23id_1843}{45}}.
Le cuer d'amours épris, Oeuvres du roi René, ed. De Quatrebarbes, Angers
1845, 4 vols., III, p. 112.

\protect\hypertarget{23_NOTES.xhtmlux5cux23id_1842}{\protect\hyperlink{10_Chapter_Three__THE_HEROIC_DREAM.xhtmlux5cux23id_1841}{46}}.
Lefèvre de S. Remy, II, p. 68.

\protect\hypertarget{23_NOTES.xhtmlux5cux23id_1840}{\protect\hyperlink{10_Chapter_Three__THE_HEROIC_DREAM.xhtmlux5cux23id_1839}{47}}.
Doutrepont, p. 183.

\protect\hypertarget{23_NOTES.xhtmlux5cux23id_1838}{\protect\hyperlink{10_Chapter_Three__THE_HEROIC_DREAM.xhtmlux5cux23id_1837}{48}}.
La Marche, II, p. 216, 334.

\protect\hypertarget{23_NOTES.xhtmlux5cux23id_1836}{\protect\hyperlink{10_Chapter_Three__THE_HEROIC_DREAM.xhtmlux5cux23id_1835}{49}}.
Ph. Wielant, Antiquités de Flandre, ed. De Smet (Corp. chron. Flandriae,
IV), p. 56.

\protect\hypertarget{23_NOTES.xhtmlux5cux23id_1834}{\protect\hyperlink{10_Chapter_Three__THE_HEROIC_DREAM.xhtmlux5cux23id_1833}{50}}.
Commines, I, p. 390, see the anecdote in Doutrepont, p. 185.

\protect\hypertarget{23_NOTES.xhtmlux5cux23id_1832}{\protect\hyperlink{10_Chapter_Three__THE_HEROIC_DREAM.xhtmlux5cux23id_1831}{51}}.
Chastellain, V, p. 316--319.

\protect\hypertarget{23_NOTES.xhtmlux5cux23id_1830}{\protect\hyperlink{10_Chapter_Three__THE_HEROIC_DREAM.xhtmlux5cux23id_1829}{52}}.
P. Meyer, Bull. de la soc. des anc. textes français, p. 45--54.

\protect\hypertarget{23_NOTES.xhtmlux5cux23id_1828}{\protect\hyperlink{10_Chapter_Three__THE_HEROIC_DREAM.xhtmlux5cux23id_1827}{53}}.
Deschamps, nos. 12, 93, 207, 239, 362, 403, 432, 652, I pp. 86, 199, II
p. 29, X pp. xxxv, xxviff.

\protect\hypertarget{23_NOTES.xhtmlux5cux23id_1826}{\protect\hyperlink{10_Chapter_Three__THE_HEROIC_DREAM.xhtmlux5cux23id_1825}{54}}.
Journal d'un bourgeois, p. 274. In the middle of the sixteenth century
John Coke still knew them as The nyne worthyes, The debate between the
Heraides, ed. L. Pannier et P. Meyer, Le débat des hérauts d'armes, p.
108, §171, while Cervantes called them ``todos los nueve de la fama'';
Don Quijote, I, 5.

\protect\hypertarget{23_NOTES.xhtmlux5cux23id_1824}{\protect\hyperlink{10_Chapter_Three__THE_HEROIC_DREAM.xhtmlux5cux23id_1823}{55}}.
Molinet, Faictz et Dictz, f. 151 v.

\protect\hypertarget{23_NOTES.xhtmlux5cux23id_1822}{\protect\hyperlink{10_Chapter_Three__THE_HEROIC_DREAM.xhtmlux5cux23id_1821}{56}}.
La Curne de Sainte Palaye, II, p. 88.

\protect\hypertarget{23_NOTES.xhtmlux5cux23id_1820}{\protect\hyperlink{10_Chapter_Three__THE_HEROIC_DREAM.xhtmlux5cux23id_1819}{57}}.
Deschamps, nos. 206, 239, II pp. 27, 69, no. 312, II p. 324, Le lay du
très bon connestable B. du Guesclin.

\protect\hypertarget{23_NOTES.xhtmlux5cux23id_1818}{\protect\hyperlink{10_Chapter_Three__THE_HEROIC_DREAM.xhtmlux5cux23id_1817}{58}}.
S. Luce, La France pendant la querre de cent ans, p. 231: Du Guesclin,
dixième preux.

\protect\hypertarget{23_NOTES.xhtmlux5cux23id_1816}{\protect\hyperlink{10_Chapter_Three__THE_HEROIC_DREAM.xhtmlux5cux23id_1815}{59}}.
Chastellain, La mort du roy charles VII, VI, p. 440.

\protect\hypertarget{23_NOTES.xhtmlux5cux23id_1814}{\protect\hyperlink{10_Chapter_Three__THE_HEROIC_DREAM.xhtmlux5cux23id_1813}{60}}.
Laborde, II, p. 242, no. 4091; 138, no. 242; see also p. 146, no. 3343;
p. 260, no. 4220; p. 266, no. 4253. The psalter was acquired during the
War of the Spanish Succession by Joan van den Berg, the \emph{Kommissar
der Generalstaaten} in Belgium, and is today in the University of Leiden
library.

\protect\hypertarget{23_NOTES.xhtmlux5cux23id_1812}{\protect\hyperlink{10_Chapter_Three__THE_HEROIC_DREAM.xhtmlux5cux23id_1811}{61}}.
Burckhardt, \emph{Die Kultur der Renaissance in Italien}, X, I, p. 246.

\protect\hypertarget{23_NOTES.xhtmlux5cux23id_1810}{\protect\hyperlink{10_Chapter_Three__THE_HEROIC_DREAM.xhtmlux5cux23id_1809}{62}}.
Le livre des faicts du maréchal Boucicaut, ed. Petitot, Coll. de
mémoires, I. serie, VI, VII.

\protect\hypertarget{23_NOTES.xhtmlux5cux23id_1808}{\protect\hyperlink{10_Chapter_Three__THE_HEROIC_DREAM.xhtmlux5cux23id_1807}{63}}.
Le livre des faicts, VI, p. 379.

\protect\hypertarget{23_NOTES.xhtmlux5cux23id_1806}{\protect\hyperlink{10_Chapter_Three__THE_HEROIC_DREAM.xhtmlux5cux23id_1805}{64}}.
Le livre des faicts, VII, pp. 214, 185, 200--201.

\protect\hypertarget{23_NOTES.xhtmlux5cux23id_1804}{\protect\hyperlink{10_Chapter_Three__THE_HEROIC_DREAM.xhtmlux5cux23id_1803}{65}}.
Chr. de Pisan, Le débat des deux amants, Oeuvres poétiques, II, p. 96.

\emph{\protect\hypertarget{23_NOTES.xhtmlux5cux23page_407}{\protect\hyperlink{10_Chapter_Three__THE_HEROIC_DREAM.xhtmlux5cux23id_1802}{66}}}.
Antoine de la Salle, La salade, chap. 3, Paris, M. le Noir, 1521, f. 4
vso.

\protect\hypertarget{23_NOTES.xhtmlux5cux23id_1801}{\protect\hyperlink{10_Chapter_Three__THE_HEROIC_DREAM.xhtmlux5cux23id_1800}{67}}.
Le livre des cent ballades, ed. G. Raynaud (Soc. des anciens textes
français), p. lv.

\protect\hypertarget{23_NOTES.xhtmlux5cux23id_1799}{\protect\hyperlink{10_Chapter_Three__THE_HEROIC_DREAM.xhtmlux5cux23id_1798}{68}}.
Ed. C. Favre and L. Lecestre (Soc. de l'hist. de France), 1887--89.

\emph{\protect\hypertarget{23_NOTES.xhtmlux5cux23id_1797}{\protect\hyperlink{10_Chapter_Three__THE_HEROIC_DREAM.xhtmlux5cux23id_1796}{6g}}}.
{[}Trans.{]} \emph{Minnelieder}: Songs composed by the knightly class of
practitioners of the northern versions of courtly love.

\protect\hypertarget{23_NOTES.xhtmlux5cux23id_1795}{\protect\hyperlink{10_Chapter_Three__THE_HEROIC_DREAM.xhtmlux5cux23id_1794}{70}}.
Lejouvencel, I, p. 25.

\protect\hypertarget{23_NOTES.xhtmlux5cux23id_1793}{\protect\hyperlink{10_Chapter_Three__THE_HEROIC_DREAM.xhtmlux5cux23id_1792}{71}}.
Le livre des faits du bon chevalier Messire Jacques de Lalaing, ed.
Kervyn de Lettenhove, in Chastellain, Oeuvres, VIII.

\protect\hypertarget{23_NOTES.xhtmlux5cux23id_1791}{\protect\hyperlink{10_Chapter_Three__THE_HEROIC_DREAM.xhtmlux5cux23id_1790}{72}}.
Lejouvencel, II, p. 20.

\protect\hypertarget{23_NOTES.xhtmlux5cux23id_1789}{\protect\hyperlink{10_Chapter_Three__THE_HEROIC_DREAM.xhtmlux5cux23id_1788}{73}}.
W. James, The varieties of religious experience. Gifford lectures
1901--1902. London 1903, p. 318.

\protect\hypertarget{23_NOTES.xhtmlux5cux23id_1787}{\protect\hyperlink{10_Chapter_Three__THE_HEROIC_DREAM.xhtmlux5cux23id_1786}{74}}.
{[}Trans.{]} \emph{Opgeheven} (German, \emph{aufgehoben)} has the
connotation of canceling and then raising up into a higher synthesis.
Here the sensual passion is spiritualized into the heroic dream.

\protect\hypertarget{23_NOTES.xhtmlux5cux23id_1785}{\protect\hyperlink{10_Chapter_Three__THE_HEROIC_DREAM.xhtmlux5cux23id_1784}{75}}.
{[}Trans.{]} This obscure passage is probably a reflection of Huizinga's
early studies in philology. He apparently refers to a theory suggested
by Max Muller that traced the origin of all myth back to celestial
phenomena such as the rising and setting of the sun. Huizinga's
suggestion that a much more natural explanation lies in the sexual drive
of young men is a rejection of complexity in favor of common sense and
is a counterpoint to his rejection of the too facile simplifications of
economic determinism. See Richard M. Dorson (ed.), \emph{Pagan Customs
and Savage Myths} (Chicago: University of Chicago Press, 1968).

\protect\hypertarget{23_NOTES.xhtmlux5cux23id_1783}{\protect\hyperlink{10_Chapter_Three__THE_HEROIC_DREAM.xhtmlux5cux23id_1782}{76}}.
{[}Trans.{]} Huizinga uses the term for love \emph{Min} (German,
\emph{Minne)} to refer to the elevated love of the practice of courtly
love. Occasionally, he uses \emph{Min} to specify the heavenly love of
God. In this translation, we have used \emph{Minne} only where the
reference is to courtly love.

\protect\hypertarget{23_NOTES.xhtmlux5cux23id_1781}{\protect\hyperlink{10_Chapter_Three__THE_HEROIC_DREAM.xhtmlux5cux23id_1780}{77}}.
Le livre des faicts, p. 398

\protect\hypertarget{23_NOTES.xhtmlux5cux23id_1779}{\protect\hyperlink{10_Chapter_Three__THE_HEROIC_DREAM.xhtmlux5cux23id_1778}{78}}.
Ed. G. Raynaud, Société des anciens textes français, 1905.

\protect\hypertarget{23_NOTES.xhtmlux5cux23id_1777}{\protect\hyperlink{10_Chapter_Three__THE_HEROIC_DREAM.xhtmlux5cux23id_1776}{79}}.
Two heros from the Romance of Aspremont.

\protect\hypertarget{23_NOTES.xhtmlux5cux23id_1775}{\protect\hyperlink{10_Chapter_Three__THE_HEROIC_DREAM.xhtmlux5cux23id_1774}{80}}.
Les voeux du héron vs. 354--371, ed. Soc des bibliophiles de Mons, no.
8, 1839.

\protect\hypertarget{23_NOTES.xhtmlux5cux23id_1773}{\protect\hyperlink{10_Chapter_Three__THE_HEROIC_DREAM.xhtmlux5cux23id_1772}{81}}.
Letter of the Count of Chimay to Chastellain, Oeuvres, VIII, p. 266.

\protect\hypertarget{23_NOTES.xhtmlux5cux23id_1771}{\protect\hyperlink{10_Chapter_Three__THE_HEROIC_DREAM.xhtmlux5cux23id_1770}{82}}.
Perceforest, in Quatrebarbes, Oeuvres de roi René, II, p. xciv.

\protect\hypertarget{23_NOTES.xhtmlux5cux23id_1769}{\protect\hyperlink{10_Chapter_Three__THE_HEROIC_DREAM.xhtmlux5cux23id_1768}{83}}.
Des trois chevaliers et del chainse, in Jacques de Baisieux, ed.
Scheler, Trouvères belges, I, 1876, p. 162.

\protect\hypertarget{23_NOTES.xhtmlux5cux23id_1767}{\protect\hyperlink{10_Chapter_Three__THE_HEROIC_DREAM.xhtmlux5cux23id_1766}{84}}.
Rel. de S. Denis, I, p. 594ff; Juvenal des Ursins, p. 379.

\protect\hypertarget{23_NOTES.xhtmlux5cux23id_1765}{\protect\hyperlink{10_Chapter_Three__THE_HEROIC_DREAM.xhtmlux5cux23id_1764}{85}}.
Among others forbidden by the Lateran Synod of 1215; again by Pope
Nicholas II, 1279; see Raynaldus, Annales ecclesiastici, III (= Baronius
XXII), 1279, xvi--xx; Dionysii Cartusiani Opera, I, XXXVI, p. 206.

\protect\hypertarget{23_NOTES.xhtmlux5cux23id_1763}{\protect\hyperlink{10_Chapter_Three__THE_HEROIC_DREAM.xhtmlux5cux23id_1762}{86}}.
Deschamps, I p. 222, no. 108; p. 223, no. 109.

\protect\hypertarget{23_NOTES.xhtmlux5cux23id_1761}{\protect\hyperlink{10_Chapter_Three__THE_HEROIC_DREAM.xhtmlux5cux23id_1760}{87}}.
Journal d'un bourgeois de Paris, pp. 59, 56.

\protect\hypertarget{23_NOTES.xhtmlux5cux23id_1759}{\protect\hyperlink{10_Chapter_Three__THE_HEROIC_DREAM.xhtmlux5cux23id_1758}{88}}.
Adam v. Bremen, Gesta Hammaburg. eccl. pontificum, lib., II, cap. 1.

\protect\hypertarget{23_NOTES.xhtmlux5cux23id_1757}{\protect\hyperlink{10_Chapter_Three__THE_HEROIC_DREAM.xhtmlux5cux23id_1756}{89}}.
La Marche, II, pp. 119, 144; d'Escouchy, I, p. 245.

\protect\hypertarget{23_NOTES.xhtmlux5cux23id_1755}{\protect\hyperlink{10_Chapter_Three__THE_HEROIC_DREAM.xhtmlux5cux23id_1754}{90}}.
Chastellain, VIII, p. 238.

\protect\hypertarget{23_NOTES.xhtmlux5cux23id_1753}{\protect\hyperlink{10_Chapter_Three__THE_HEROIC_DREAM.xhtmlux5cux23id_1752}{91}}.
La Marche, I, p. 292.

\protect\hypertarget{23_NOTES.xhtmlux5cux23id_1751}{\protect\hyperlink{10_Chapter_Three__THE_HEROIC_DREAM.xhtmlux5cux23id_1750}{92}}.
Le livre des faits de Jacques de Lalaing, in Chastellain, VIII, pp.
188ff.

\protect\hypertarget{23_NOTES.xhtmlux5cux23page_408}{\protect\hyperlink{10_Chapter_Three__THE_HEROIC_DREAM.xhtmlux5cux23id_1749}{93}}.
Oeuvres du roi René, I, p. lxxv.

\protect\hypertarget{23_NOTES.xhtmlux5cux23id_1748}{\protect\hyperlink{10_Chapter_Three__THE_HEROIC_DREAM.xhtmlux5cux23id_1747}{94}}.
La Marche, III, p. 123; Molinet, V, p. 18.

\protect\hypertarget{23_NOTES.xhtmlux5cux23id_1746}{\protect\hyperlink{10_Chapter_Three__THE_HEROIC_DREAM.xhtmlux5cux23id_1745}{95}}.
La Marche, II, pp. 118, 121, 122, 133, 341; Chastellain, I p. 256, VIII
pp. 217, 246.

\protect\hypertarget{23_NOTES.xhtmlux5cux23id_1744}{\protect\hyperlink{10_Chapter_Three__THE_HEROIC_DREAM.xhtmlux5cux23id_1743}{96}}.
La Marche, II p. 173, I p. 285; Oeuvres du roi René, I, p. lxxv.

\protect\hypertarget{23_NOTES.xhtmlux5cux23id_1742}{\protect\hyperlink{10_Chapter_Three__THE_HEROIC_DREAM.xhtmlux5cux23id_1741}{97}}.
Oeuvres du roi René, I, pp. lxxxvi, 57.

\protect\hypertarget{23_NOTES.xhtmlux5cux23id_1740}{\protect\hyperlink{10_Chapter_Three__THE_HEROIC_DREAM.xhtmlux5cux23id_1739}{98}}.
{[}Trans.{]} \emph{Knightly orders}. Huizinga probably means as the
three orders of the Holy Land of the Templars, the Knights of Saint John
the Baptist \emph{(Hospitallers)}, and the Teutonic Knights. The three
Spanish orders are the Order of Santiago, the Knights of Calatrava, and
perhaps the Knights of Christ, although these last were Portuguese.

\protect\hypertarget{23_NOTES.xhtmlux5cux23id_1738}{\protect\hyperlink{10_Chapter_Three__THE_HEROIC_DREAM.xhtmlux5cux23id_1737}{99}}.
N. Jorga, Phil, de Mézières, p. 348.

\protect\hypertarget{23_NOTES.xhtmlux5cux23id_1736}{\protect\hyperlink{10_Chapter_Three__THE_HEROIC_DREAM.xhtmlux5cux23id_1735}{100}}.
Chastellain, II, p. 7; IV, p. 233, cf. 269; VI, p. 154.

\protect\hypertarget{23_NOTES.xhtmlux5cux23id_1734}{\protect\hyperlink{10_Chapter_Three__THE_HEROIC_DREAM.xhtmlux5cux23id_1733}{101}}.
La Marche, I, p. 109.

\protect\hypertarget{23_NOTES.xhtmlux5cux23id_1732}{\protect\hyperlink{10_Chapter_Three__THE_HEROIC_DREAM.xhtmlux5cux23id_1731}{102}}.
Statuten \emph{des ordens}, in Luc d'Achéry, Spicilegium, III, p. 730.

\protect\hypertarget{23_NOTES.xhtmlux5cux23id_1730}{\protect\hyperlink{10_Chapter_Three__THE_HEROIC_DREAM.xhtmlux5cux23id_1729}{103}}.
Chastellain, II, p. 10.

\protect\hypertarget{23_NOTES.xhtmlux5cux23id_1728}{\protect\hyperlink{10_Chapter_Three__THE_HEROIC_DREAM.xhtmlux5cux23id_1727}{104}}.
Chronique scandaleuse, I, p. 236.

\protect\hypertarget{23_NOTES.xhtmlux5cux23id_1726}{\protect\hyperlink{10_Chapter_Three__THE_HEROIC_DREAM.xhtmlux5cux23id_1725}{105}}.
Le songe de la thoison d'or, in Doutrepont, p. 154.

\protect\hypertarget{23_NOTES.xhtmlux5cux23id_1724}{\protect\hyperlink{10_Chapter_Three__THE_HEROIC_DREAM.xhtmlux5cux23id_1723}{106}}.
Fillastre, Le premier volume de la toison d'or, Paris 1515, fol. 2.

\protect\hypertarget{23_NOTES.xhtmlux5cux23id_1722}{\protect\hyperlink{10_Chapter_Three__THE_HEROIC_DREAM.xhtmlux5cux23id_1721}{107}}.
{[}Trans.{]} \emph{Bannerets}: Knights who had distinguished themselves
and been promoted to the command of a section of knights.

\protect\hypertarget{23_NOTES.xhtmlux5cux23id_1720}{\protect\hyperlink{10_Chapter_Three__THE_HEROIC_DREAM.xhtmlux5cux23id_1719}{108}}.
Boucicaut, I, p. 504; Jorga, Ph. de Mézières, pp. 83, 4638; Romania,
XXVI, pp. 3951, 3961; Deschamps, XI, p. 28; Oeuvres du roi René, I, p.
xi; Monstrelet, V, p. 449.

\protect\hypertarget{23_NOTES.xhtmlux5cux23id_1718}{\protect\hyperlink{10_Chapter_Three__THE_HEROIC_DREAM.xhtmlux5cux23id_1717}{109}}.
Froissart, Poésies, ed. A Scheler (Acad. royale de Belgique), 1870--72,
3 vols., II, p. 341.

\protect\hypertarget{23_NOTES.xhtmlux5cux23id_1716}{\protect\hyperlink{10_Chapter_Three__THE_HEROIC_DREAM.xhtmlux5cux23id_1715}{110}}.
Alain Chartier, La ballade de Fougères, p. 718.

\protect\hypertarget{23_NOTES.xhtmlux5cux23id_1714}{\protect\hyperlink{10_Chapter_Three__THE_HEROIC_DREAM.xhtmlux5cux23id_1713}{111}}.
La Marche, IV, p. 164; Jacques du Clercq, II, p. 6.

\protect\hypertarget{23_NOTES.xhtmlux5cux23id_1712}{\protect\hyperlink{10_Chapter_Three__THE_HEROIC_DREAM.xhtmlux5cux23id_1711}{112}}.
Liber Karoleidos vs. 88 (Chron. rel. a l'hist de Belg. sous la dom. des
ducs de Bourg), III.

\protect\hypertarget{23_NOTES.xhtmlux5cux23id_1710}{\protect\hyperlink{10_Chapter_Three__THE_HEROIC_DREAM.xhtmlux5cux23id_1709}{113}}.
Gen. 30, 32; 4 Kings, 3, 4; Job 31:20; Psalm 71:6.

\protect\hypertarget{23_NOTES.xhtmlux5cux23id_1708}{\protect\hyperlink{10_Chapter_Three__THE_HEROIC_DREAM.xhtmlux5cux23id_1707}{114}}.
Guillaume Fillastre, Le second volume de la toison d'or, Paris, Franc.
Regnault, 1516, fol. 1, 2.

\protect\hypertarget{23_NOTES.xhtmlux5cux23id_1706}{\protect\hyperlink{10_Chapter_Three__THE_HEROIC_DREAM.xhtmlux5cux23id_1705}{115}}.
La Marche, III p. 201, IV p. 67; Lefèvre de S. Remy, II, p. 292; the
ceremonial of such a christening is in Humphrey of Glocester's Herald
Nicholas Upton, De officio militari, ed. E. Bysshe (Bissaeus), London
1654, lib. I, c. XI, p. 19.

\protect\hypertarget{23_NOTES.xhtmlux5cux23id_1704}{\protect\hyperlink{10_Chapter_Three__THE_HEROIC_DREAM.xhtmlux5cux23id_1703}{116}}.
Presumably Deschamps is hinting at this order in the envoi of the
ballade on the love order of the leaves (as opposed to that of the
Flowers), no. 767, IV, p. 262; see 763: ``Royne sur fleurs en vertu
demourant, Galoys, Dannoy, Mornay, Pierre ensement De tremoille .~.~.
vont .~.~. vostre bien qui est grant etc.''

\protect\hypertarget{23_NOTES.xhtmlux5cux23id_1702}{\protect\hyperlink{10_Chapter_Three__THE_HEROIC_DREAM.xhtmlux5cux23id_1701}{117}}.
Le livre du chevalier de la Tour Landry, ed. A. de Montaiglon (Bibl.
elzevirenne), Paris, 1854, p. 241ff.

\protect\hypertarget{23_NOTES.xhtmlux5cux23id_1700}{\protect\hyperlink{10_Chapter_Three__THE_HEROIC_DREAM.xhtmlux5cux23id_1699}{118}}.
Voeu du héron, ed. Soc. des bibl. de Mons, p. 17.

\protect\hypertarget{23_NOTES.xhtmlux5cux23id_1698}{\protect\hyperlink{10_Chapter_Three__THE_HEROIC_DREAM.xhtmlux5cux23id_1697}{119}}.
Froissart, ed. Luce, I, p. 124.

\protect\hypertarget{23_NOTES.xhtmlux5cux23id_1696}{\protect\hyperlink{10_Chapter_Three__THE_HEROIC_DREAM.xhtmlux5cux23id_1695}{120}}.
Rel. de S. Denis, III, p. 72. Harald Harfagri took a vow not to cut his
\protect\hypertarget{23_NOTES.xhtmlux5cux23page_409}{}{}hair until he
had conquered all of Norway. Haraldar saga Harfagra, cap. 4; see Voluspa
33.

\protect\hypertarget{23_NOTES.xhtmlux5cux23id_1694}{\protect\hyperlink{10_Chapter_Three__THE_HEROIC_DREAM.xhtmlux5cux23id_1693}{121}}.
Jorga, Ph. de Mézières, p. 76.

\protect\hypertarget{23_NOTES.xhtmlux5cux23id_1692}{\protect\hyperlink{10_Chapter_Three__THE_HEROIC_DREAM.xhtmlux5cux23id_1691}{122}}.
Claude Menard, Hist. de Bertrand du Guesclin, pp. 39, 55, 410, 488, La
Curne, I, p. 240. Luther still speaks of the superstitious vows by the
soldiers of his time, Tischreden, Weimarer Ausg. II, no. 2753 b, p.
632--33.

\protect\hypertarget{23_NOTES.xhtmlux5cux23id_1690}{\protect\hyperlink{10_Chapter_Three__THE_HEROIC_DREAM.xhtmlux5cux23id_1689}{123}}.
Douet d'Arcq, Choix de pièces inédites rel. au règne de Charles VI.
(Soc. de l'hist. de France, 1863), I, p. 370.

\protect\hypertarget{23_NOTES.xhtmlux5cux23id_1688}{\protect\hyperlink{10_Chapter_Three__THE_HEROIC_DREAM.xhtmlux5cux23id_1687}{124}}.
Le livre des faits, chaps. XVIff, in Chastellain, VIII, p. 70.

\protect\hypertarget{23_NOTES.xhtmlux5cux23id_1686}{\protect\hyperlink{10_Chapter_Three__THE_HEROIC_DREAM.xhtmlux5cux23id_1685}{125}}.
Le petit Jehan de Saintré, chap. 48.

\protect\hypertarget{23_NOTES.xhtmlux5cux23id_1684}{\protect\hyperlink{10_Chapter_Three__THE_HEROIC_DREAM.xhtmlux5cux23id_1683}{126}}.
Germania, chap. 31; La Curne, I, p. 236.

\protect\hypertarget{23_NOTES.xhtmlux5cux23id_1682}{\protect\hyperlink{10_Chapter_Three__THE_HEROIC_DREAM.xhtmlux5cux23id_1681}{127}}.
Heimskringla, Olafssaga Tryggvasonar, chap. 35; Weinhold, Altnordisches
Leben, p. 462.

\protect\hypertarget{23_NOTES.xhtmlux5cux23id_1680}{\protect\hyperlink{10_Chapter_Three__THE_HEROIC_DREAM.xhtmlux5cux23id_1679}{128}}.
La Marche, II, p. 366.

\protect\hypertarget{23_NOTES.xhtmlux5cux23id_1678}{\protect\hyperlink{10_Chapter_Three__THE_HEROIC_DREAM.xhtmlux5cux23id_1677}{129}}.
La Marche, II, p. 381--387.

\protect\hypertarget{23_NOTES.xhtmlux5cux23id_1676}{\protect\hyperlink{10_Chapter_Three__THE_HEROIC_DREAM.xhtmlux5cux23id_1675}{130}}.
La Marche, loc. cit.; d'Escouchy, II, pp. 166, 218.

\protect\hypertarget{23_NOTES.xhtmlux5cux23id_1674}{\protect\hyperlink{10_Chapter_Three__THE_HEROIC_DREAM.xhtmlux5cux23id_1673}{131}}.
d'Escouchy, II, p. 189.

\protect\hypertarget{23_NOTES.xhtmlux5cux23id_1672}{\protect\hyperlink{10_Chapter_Three__THE_HEROIC_DREAM.xhtmlux5cux23id_1671}{132}}.
Doutrepont, p. 513.

\protect\hypertarget{23_NOTES.xhtmlux5cux23id_1670}{\protect\hyperlink{10_Chapter_Three__THE_HEROIC_DREAM.xhtmlux5cux23id_1669}{133}}.
Doutrepont, pp. 110, 112.

\protect\hypertarget{23_NOTES.xhtmlux5cux23id_1668}{\protect\hyperlink{10_Chapter_Three__THE_HEROIC_DREAM.xhtmlux5cux23id_1667}{134}}.
Chastellain, III, p. 376.

\protect\hypertarget{23_NOTES.xhtmlux5cux23id_1666}{\protect\hyperlink{10_Chapter_Three__THE_HEROIC_DREAM.xhtmlux5cux23id_1665}{135}}.
See above p. 87.

\protect\hypertarget{23_NOTES.xhtmlux5cux23id_1664}{\protect\hyperlink{10_Chapter_Three__THE_HEROIC_DREAM.xhtmlux5cux23id_1663}{136}}.
Chronique de Berne (Molinier no. 3103), in Kervyn, Froissart, II, p.
531.

\protect\hypertarget{23_NOTES.xhtmlux5cux23id_1662}{\protect\hyperlink{10_Chapter_Three__THE_HEROIC_DREAM.xhtmlux5cux23id_1661}{137}}.
d'Escouchy, II, p. 220.

\protect\hypertarget{23_NOTES.xhtmlux5cux23id_1660}{\protect\hyperlink{10_Chapter_Three__THE_HEROIC_DREAM.xhtmlux5cux23id_1659}{138}}.
Froissart, ed. Luce, X, pp. 240, 243.

\protect\hypertarget{23_NOTES.xhtmlux5cux23id_1658}{\protect\hyperlink{10_Chapter_Three__THE_HEROIC_DREAM.xhtmlux5cux23id_1657}{139}}.
Le livre des faits, Chastellain, VIII, pp. 158--161.

\protect\hypertarget{23_NOTES.xhtmlux5cux23id_1656}{\protect\hyperlink{10_Chapter_Three__THE_HEROIC_DREAM.xhtmlux5cux23id_1655}{140}}.
La Marche, IV, Estat de la Maison, pp. 34, 47.

\protect\hypertarget{23_NOTES.xhtmlux5cux23id_1654}{\protect\hyperlink{10_Chapter_Three__THE_HEROIC_DREAM.xhtmlux5cux23id_1653}{141}}.
{[}Trans.{]} ``Waning'' here translates \emph{laatste}.

\protect\hypertarget{23_NOTES.xhtmlux5cux23id_1652}{\protect\hyperlink{10_Chapter_Three__THE_HEROIC_DREAM.xhtmlux5cux23id_1651}{142}}.
See my essay, ``Uit de voorgeschiedenis van ons nationaal desef,'' de
Gils, 1912, I.

\protect\hypertarget{23_NOTES.xhtmlux5cux23id_1650}{\protect\hyperlink{10_Chapter_Three__THE_HEROIC_DREAM.xhtmlux5cux23id_1649}{143}}.
Psalms 50:19; in the King James and Revised eds., 51:18; and in the
Vulgate, 51:20.

\protect\hypertarget{23_NOTES.xhtmlux5cux23id_1648}{\protect\hyperlink{10_Chapter_Three__THE_HEROIC_DREAM.xhtmlux5cux23id_1647}{144}}.
Monstrelet, IV, p. 112; Pierre de Fenin, p. 363; Lefèvre de S. Remy, II,
p. \emph{63}; Chastellain, I, p. 331.

\protect\hypertarget{23_NOTES.xhtmlux5cux23id_1646}{\protect\hyperlink{10_Chapter_Three__THE_HEROIC_DREAM.xhtmlux5cux23id_1645}{145}}.
See J. D. Hintzen, De Kruistochtplannen van Philip den Goede,
Dissertation: University of Leiden, 1918.

\protect\hypertarget{23_NOTES.xhtmlux5cux23id_1644}{\protect\hyperlink{10_Chapter_Three__THE_HEROIC_DREAM.xhtmlux5cux23id_1643}{146}}.
Chastellain, III, pp. 6, 10, 34, 77, 118, 119, 178, 334; IV, pp. 125,
128, 171, 431, 437, 451, 470; V, p. 49.

\protect\hypertarget{23_NOTES.xhtmlux5cux23id_1642}{\protect\hyperlink{10_Chapter_Three__THE_HEROIC_DREAM.xhtmlux5cux23id_1641}{147}}.
La Marche, II, p. 382.

\protect\hypertarget{23_NOTES.xhtmlux5cux23id_1640}{\protect\hyperlink{10_Chapter_Three__THE_HEROIC_DREAM.xhtmlux5cux23id_1639}{148}}.
De Gids, 1912, I, Uit de voorgeschiedenis van ons nationaal besef

\protect\hypertarget{23_NOTES.xhtmlux5cux23id_1638}{\protect\hyperlink{10_Chapter_Three__THE_HEROIC_DREAM.xhtmlux5cux23id_1637}{149}}.
Rymer, Foedera III, pars 3, p. 158 = VII, p. 407.

\protect\hypertarget{23_NOTES.xhtmlux5cux23id_1636}{\protect\hyperlink{10_Chapter_Three__THE_HEROIC_DREAM.xhtmlux5cux23id_1635}{150}}.
Monstrelet, I, pp. 43ff.

\protect\hypertarget{23_NOTES.xhtmlux5cux23id_1634}{\protect\hyperlink{10_Chapter_Three__THE_HEROIC_DREAM.xhtmlux5cux23id_1633}{151}}.
Monstrelet, IV, p. 219.

\protect\hypertarget{23_NOTES.xhtmlux5cux23id_1632}{\protect\hyperlink{10_Chapter_Three__THE_HEROIC_DREAM.xhtmlux5cux23id_1631}{152}}.
Pierre de Fenin, p. 626--27; Monstrelet, IV, p. 244; Liber de
virtutibus, p. 27.

\protect\hypertarget{23_NOTES.xhtmlux5cux23id_1630}{\protect\hyperlink{10_Chapter_Three__THE_HEROIC_DREAM.xhtmlux5cux23id_1629}{153}}.
Lefèvre de S. Remy, II, p. 107.

\protect\hypertarget{23_NOTES.xhtmlux5cux23id_1628}{\protect\hyperlink{10_Chapter_Three__THE_HEROIC_DREAM.xhtmlux5cux23id_1627}{154}}.
Laborde, I, pp. 201ff.

\protect\hypertarget{23_NOTES.xhtmlux5cux23page_410}{\protect\hyperlink{10_Chapter_Three__THE_HEROIC_DREAM.xhtmlux5cux23id_1626}{155}}.
La Marche, II, pp. 27, 382.

\protect\hypertarget{23_NOTES.xhtmlux5cux23id_1625}{\protect\hyperlink{10_Chapter_Three__THE_HEROIC_DREAM.xhtmlux5cux23id_1624}{156}}.
Bandello, I, Nov. 39; Filippo duca di Burgogna si mette fuor di
proposito a grandissimo periglio.

\protect\hypertarget{23_NOTES.xhtmlux5cux23id_1623}{\protect\hyperlink{10_Chapter_Three__THE_HEROIC_DREAM.xhtmlux5cux23id_1622}{157}}.
F. von Bezold, Aus dem Briefwechsel der Markgräfin Isabella von
Este-Gonzaga, Archiv f. Kulturgesch., VIII, p. 396.

\protect\hypertarget{23_NOTES.xhtmlux5cux23id_1621}{\protect\hyperlink{10_Chapter_Three__THE_HEROIC_DREAM.xhtmlux5cux23id_1620}{158}}.
Papiers de Granvelle, I, pp. 360ff.; Geschichte Karls V, II, p. 641;
Fueter, Geschichte des europäischen staatensystems 1492--1559, p. 307.
See from Erasmus to Nicolaus Beraldus, 25 May 1522, Dedication of De
Ratione conscribendi epistolas, Leidener Ausg., I, p. 344.

\protect\hypertarget{23_NOTES.xhtmlux5cux23id_1619}{\protect\hyperlink{10_Chapter_Three__THE_HEROIC_DREAM.xhtmlux5cux23id_1618}{159}}.
Chastellain, III, pp. 38--49; La Marche, II, pp. 406ff.; d'Escouchy, II,
pp. 300ff; Corp. chron. Flandr., III, p. 525; Petit Dutaillis, Documents
nouveaux, pp. 113, 137. --For an apparently safe form of judicial duel
see Deschamps, IX, p. 21.

\protect\hypertarget{23_NOTES.xhtmlux5cux23id_1617}{\protect\hyperlink{10_Chapter_Three__THE_HEROIC_DREAM.xhtmlux5cux23id_1616}{160}}.
{[}Trans.{]} \emph{houpelande: A} tunic with a long skirt \emph{(OED)}.

\protect\hypertarget{23_NOTES.xhtmlux5cux23id_1615}{\protect\hyperlink{10_Chapter_Three__THE_HEROIC_DREAM.xhtmlux5cux23id_1614}{161}}.
Froissart, ed. Luce, IV, pp. 89--94.

\protect\hypertarget{23_NOTES.xhtmlux5cux23id_1613}{\protect\hyperlink{10_Chapter_Three__THE_HEROIC_DREAM.xhtmlux5cux23id_1612}{162}}.
Froissart, IV, pp. 127--28.

\protect\hypertarget{23_NOTES.xhtmlux5cux23id_1611}{\protect\hyperlink{10_Chapter_Three__THE_HEROIC_DREAM.xhtmlux5cux23id_1610}{163}}.
Lefèvre de S. Remy, I, p. 241.

\protect\hypertarget{23_NOTES.xhtmlux5cux23id_1609}{\protect\hyperlink{10_Chapter_Three__THE_HEROIC_DREAM.xhtmlux5cux23id_1608}{164}}.
Froissart, ed. Luce, XI, p. 3.

\protect\hypertarget{23_NOTES.xhtmlux5cux23id_1607}{\protect\hyperlink{10_Chapter_Three__THE_HEROIC_DREAM.xhtmlux5cux23id_1606}{165}}.
Rel. de S. Denis, III, p. 175.

\protect\hypertarget{23_NOTES.xhtmlux5cux23id_1605}{\protect\hyperlink{10_Chapter_Three__THE_HEROIC_DREAM.xhtmlux5cux23id_1604}{166}}.
Froissart, ed. Luce, XI, pp. 24ff.; VI, p. 156.

\protect\hypertarget{23_NOTES.xhtmlux5cux23id_1603}{\protect\hyperlink{10_Chapter_Three__THE_HEROIC_DREAM.xhtmlux5cux23id_1602}{167}}.
{[}Trans.{]} \emph{Aristies: A} useful term for which there does not
seem to be an English equivalent. A knight or group of knights who fight
in circumstances to which they and their opponents have agreed in
advance; a pitched battle.

\protect\hypertarget{23_NOTES.xhtmlux5cux23id_1601}{\protect\hyperlink{10_Chapter_Three__THE_HEROIC_DREAM.xhtmlux5cux23id_1600}{168}}.
Froissart, ed. Luce, IV, p. no. 115. Other similar combats for instance,
Molinier, Sources, IV, no. 3707; Molinet, IV, p. 294.

\protect\hypertarget{23_NOTES.xhtmlux5cux23id_1599}{\protect\hyperlink{10_Chapter_Three__THE_HEROIC_DREAM.xhtmlux5cux23id_1598}{169}}.
Rel. de S. Denis, I, p. 392.

\protect\hypertarget{23_NOTES.xhtmlux5cux23id_1597}{\protect\hyperlink{10_Chapter_Three__THE_HEROIC_DREAM.xhtmlux5cux23id_1596}{170}}.
Le Jouvencel, I, p. 209; II, pp. 99, 103.

\protect\hypertarget{23_NOTES.xhtmlux5cux23id_1595}{\protect\hyperlink{10_Chapter_Three__THE_HEROIC_DREAM.xhtmlux5cux23id_1594}{171}}.
Froissart, ed. Luce, I, p. 65; IV, p. 49; II, p. 32.

\protect\hypertarget{23_NOTES.xhtmlux5cux23id_1593}{\protect\hyperlink{10_Chapter_Three__THE_HEROIC_DREAM.xhtmlux5cux23id_1592}{172}}.
Chastellain, II, p. 140.

\protect\hypertarget{23_NOTES.xhtmlux5cux23id_1591}{\protect\hyperlink{10_Chapter_Three__THE_HEROIC_DREAM.xhtmlux5cux23id_1590}{173}}.
Monstrelet, III, p. 101; Lefèvre de S. Remy, I, p. 247.

\protect\hypertarget{23_NOTES.xhtmlux5cux23id_1589}{\protect\hyperlink{10_Chapter_Three__THE_HEROIC_DREAM.xhtmlux5cux23id_1588}{174}}.
Molinet, II, pp. 36, 48; III, pp. 98, 453; IV, p. 372.

\protect\hypertarget{23_NOTES.xhtmlux5cux23id_1587}{\protect\hyperlink{10_Chapter_Three__THE_HEROIC_DREAM.xhtmlux5cux23id_1586}{175}}.
Froissart, ed. Luce, III, p. 187; XI, p. 22.

\protect\hypertarget{23_NOTES.xhtmlux5cux23id_1585}{\protect\hyperlink{10_Chapter_Three__THE_HEROIC_DREAM.xhtmlux5cux23id_1584}{176}}.
Chastellain, II, 374.

\protect\hypertarget{23_NOTES.xhtmlux5cux23id_1583}{\protect\hyperlink{10_Chapter_Three__THE_HEROIC_DREAM.xhtmlux5cux23id_1582}{177}}.
Molinet, I, p. 65.

\protect\hypertarget{23_NOTES.xhtmlux5cux23id_1581}{\protect\hyperlink{10_Chapter_Three__THE_HEROIC_DREAM.xhtmlux5cux23id_1580}{178}}.
Monstrelet, IV, p. 65.

\protect\hypertarget{23_NOTES.xhtmlux5cux23id_1579}{\protect\hyperlink{10_Chapter_Three__THE_HEROIC_DREAM.xhtmlux5cux23id_1578}{179}}.
Monstrelet, III, p. 111; Lefèvre de S. Remy, I, p. 259.

\protect\hypertarget{23_NOTES.xhtmlux5cux23id_1577}{\protect\hyperlink{10_Chapter_Three__THE_HEROIC_DREAM.xhtmlux5cux23id_1576}{180}}.
Basin, III, p. 57.

\protect\hypertarget{23_NOTES.xhtmlux5cux23id_1575}{\protect\hyperlink{10_Chapter_Three__THE_HEROIC_DREAM.xhtmlux5cux23id_1574}{181}}.
Froissart, ed. Luce, IV, p. 80.

\protect\hypertarget{23_NOTES.xhtmlux5cux23id_1573}{\protect\hyperlink{10_Chapter_Three__THE_HEROIC_DREAM.xhtmlux5cux23id_1572}{182}}.
Chastellain, I, p. 260; La Marche, I, p. 89.

\protect\hypertarget{23_NOTES.xhtmlux5cux23id_1571}{\protect\hyperlink{10_Chapter_Three__THE_HEROIC_DREAM.xhtmlux5cux23id_1570}{183}}.
Commines, I, p. 55.

\protect\hypertarget{23_NOTES.xhtmlux5cux23id_1569}{\protect\hyperlink{10_Chapter_Three__THE_HEROIC_DREAM.xhtmlux5cux23id_1568}{184}}.
Chastellain, III, pp. 82ff.

\protect\hypertarget{23_NOTES.xhtmlux5cux23id_1567}{\protect\hyperlink{10_Chapter_Three__THE_HEROIC_DREAM.xhtmlux5cux23id_1566}{185}}.
Froissart, ed. Luce, IX, p. 220; XI, p. 202.

\protect\hypertarget{23_NOTES.xhtmlux5cux23id_1565}{\protect\hyperlink{10_Chapter_Three__THE_HEROIC_DREAM.xhtmlux5cux23id_1564}{186}}.
Ms. Chronik von Oudenarde, in Rel de S. Denis, I, p. 2291. {[}Trans.{]}
The king was fourteen years old.

\protect\hypertarget{23_NOTES.xhtmlux5cux23id_1563}{\protect\hyperlink{10_Chapter_Three__THE_HEROIC_DREAM.xhtmlux5cux23id_1562}{187}}.
Froissart, ed. Luce, XI, p. 58.

\protect\hypertarget{23_NOTES.xhtmlux5cux23id_1561}{\protect\hyperlink{10_Chapter_Three__THE_HEROIC_DREAM.xhtmlux5cux23id_1560}{188}}.
Chastellain, II, p. 259.

\protect\hypertarget{23_NOTES.xhtmlux5cux23page_411}{\protect\hyperlink{10_Chapter_Three__THE_HEROIC_DREAM.xhtmlux5cux23id_1559}{189}}.
La Marche, II, p. 324.

\protect\hypertarget{23_NOTES.xhtmlux5cux23id_1558}{\protect\hyperlink{10_Chapter_Three__THE_HEROIC_DREAM.xhtmlux5cux23id_1557}{190}}.
Chastellain, I, p. 28; Commines, I, p. 31; see Petit Dutaillis in
Lavisse, Histoire de France, IV\textsuperscript{2}, p. 33.

\protect\hypertarget{23_NOTES.xhtmlux5cux23id_1556}{\protect\hyperlink{10_Chapter_Three__THE_HEROIC_DREAM.xhtmlux5cux23id_1555}{191}}.
Deschamps, IX, p. 80; see vs. 2228, 2295; XI, p. 173.

\protect\hypertarget{23_NOTES.xhtmlux5cux23id_1554}{\protect\hyperlink{10_Chapter_Three__THE_HEROIC_DREAM.xhtmlux5cux23id_1553}{192}}.
Froissart, ed. Luce, II, p. 37.

\protect\hypertarget{23_NOTES.xhtmlux5cux23id_1552}{\protect\hyperlink{10_Chapter_Three__THE_HEROIC_DREAM.xhtmlux5cux23id_1551}{193}}.
Le débat des hérauts d'armes, §86, 87, p. 33

\protect\hypertarget{23_NOTES.xhtmlux5cux23id_1550}{\protect\hyperlink{10_Chapter_Three__THE_HEROIC_DREAM.xhtmlux5cux23id_1549}{194}}.
Livre des faits, Chastellain, VIII, p. 2522.

\protect\hypertarget{23_NOTES.xhtmlux5cux23id_1548}{\protect\hyperlink{10_Chapter_Three__THE_HEROIC_DREAM.xhtmlux5cux23id_1547}{195}}.
Froissart, ed Kervyn, XI, p. 24.

\protect\hypertarget{23_NOTES.xhtmlux5cux23id_1546}{\protect\hyperlink{10_Chapter_Three__THE_HEROIC_DREAM.xhtmlux5cux23id_1545}{196}}.
Froissart, ed. Luce, IV, p. 83, ed. Kervyn, XI, p. 24.

\protect\hypertarget{23_NOTES.xhtmlux5cux23id_1544}{\protect\hyperlink{10_Chapter_Three__THE_HEROIC_DREAM.xhtmlux5cux23id_1543}{197}}.
Deschamps, IV, no. 785, p. 289.

\protect\hypertarget{23_NOTES.xhtmlux5cux23id_1542}{\protect\hyperlink{10_Chapter_Three__THE_HEROIC_DREAM.xhtmlux5cux23id_1541}{198}}.
Chastellain, V, p. 217.

\protect\hypertarget{23_NOTES.xhtmlux5cux23id_1540}{\protect\hyperlink{10_Chapter_Three__THE_HEROIC_DREAM.xhtmlux5cux23id_1539}{199}}.
Le songe véritable, Mém. de la soc. de l'hist. de Paris, t. XVII, p.
325, in Raynaud, Les cent ballades, p. iv.

\protect\hypertarget{23_NOTES.xhtmlux5cux23id_1538}{\protect\hyperlink{10_Chapter_Three__THE_HEROIC_DREAM.xhtmlux5cux23id_1537}{200}}.
Commines, I, p. 295.

\protect\hypertarget{23_NOTES.xhtmlux5cux23id_1536}{\protect\hyperlink{10_Chapter_Three__THE_HEROIC_DREAM.xhtmlux5cux23id_1535}{201}}.
Livres messires Geoffroy de Charny, Romania XXVI.

\protect\hypertarget{23_NOTES.xhtmlux5cux23id_1534}{\protect\hyperlink{10_Chapter_Three__THE_HEROIC_DREAM.xhtmlux5cux23id_1533}{202}}.
Commines, I, pp. 36--42, 86, 164.

\protect\hypertarget{23_NOTES.xhtmlux5cux23id_1532}{\protect\hyperlink{10_Chapter_Three__THE_HEROIC_DREAM.xhtmlux5cux23id_1531}{203}}.
Froissart, ed. Luce, IV, p. 70, 302; see ed. Kervyn de Lettenhove,
Bruxelles 1869--1877, 26 vols., V, p. 513.

\protect\hypertarget{23_NOTES.xhtmlux5cux23id_1530}{\protect\hyperlink{10_Chapter_Three__THE_HEROIC_DREAM.xhtmlux5cux23id_1529}{204}}.
{[}Trans.{]} \emph{Jean de Nevers}: The name of John the Fearless before
he became duke of Burgundy. Nevers was one of the territories of the
dukes of Burgundy in central France.

\protect\hypertarget{23_NOTES.xhtmlux5cux23id_1528}{\protect\hyperlink{10_Chapter_Three__THE_HEROIC_DREAM.xhtmlux5cux23id_1527}{205}}.
Froissart, ed. Kervyn, XV, p. 227.

\protect\hypertarget{23_NOTES.xhtmlux5cux23id_1526}{\protect\hyperlink{10_Chapter_Three__THE_HEROIC_DREAM.xhtmlux5cux23id_1525}{206}}.
Doutrepont, p. 112.

\protect\hypertarget{23_NOTES.xhtmlux5cux23id_1524}{\protect\hyperlink{10_Chapter_Three__THE_HEROIC_DREAM.xhtmlux5cux23id_1523}{207}}.
Emerson, Nature, ed. Routledge, 1881, 230--31.

\protect\hypertarget{23_NOTES.xhtmlux5cux23id_1522}{\protect\hyperlink{10_Chapter_Three__THE_HEROIC_DREAM.xhtmlux5cux23id_1521}{208}}.
{[}Trans.{]} \emph{The Y of Pythagoras}: An image of the course of the
afterlife that Plato elevates to a myth in book 10 of \emph{Republic}.
Huizinga uses it simply as the image of a path which splits, one of the
branches itself having two branches.

\protect\hypertarget{23_NOTES.xhtmlux5cux23id_1520}{\protect\hyperlink{10_Chapter_Three__THE_HEROIC_DREAM.xhtmlux5cux23id_1519}{209}}.
Piaget, Romania, XXVII, 1898, p. 63.

\protect\hypertarget{23_NOTES.xhtmlux5cux23id_1518}{\protect\hyperlink{10_Chapter_Three__THE_HEROIC_DREAM.xhtmlux5cux23id_1517}{210}}.
Deschamps, no. 315, III, p. 1.

\protect\hypertarget{23_NOTES.xhtmlux5cux23id_1516}{\protect\hyperlink{10_Chapter_Three__THE_HEROIC_DREAM.xhtmlux5cux23id_1515}{211}}.
Deschamps, I, p. 161 no. 65; see I, p. 78 no. 7, p. 175 no. 75.

\protect\hypertarget{23_NOTES.xhtmlux5cux23id_1514}{\protect\hyperlink{10_Chapter_Three__THE_HEROIC_DREAM.xhtmlux5cux23id_1513}{212}}.
Deschamps, nos. 1287, 1288, 1289; VII, p. 33; see no. 178, I, p. 313.

\protect\hypertarget{23_NOTES.xhtmlux5cux23id_1512}{\protect\hyperlink{10_Chapter_Three__THE_HEROIC_DREAM.xhtmlux5cux23id_1511}{213}}.
Deschamps, no. 240, II, p. 71; see no. 196, II, p. 15.

\protect\hypertarget{23_NOTES.xhtmlux5cux23id_1510}{\protect\hyperlink{10_Chapter_Three__THE_HEROIC_DREAM.xhtmlux5cux23id_1509}{214}}.
Deschamps, no. 184, I, p. 320.

\protect\hypertarget{23_NOTES.xhtmlux5cux23id_1508}{\protect\hyperlink{10_Chapter_Three__THE_HEROIC_DREAM.xhtmlux5cux23id_1507}{215}}.
Deschamps, no. 1124, no. 307; VI, p. 41; II, p. 213, Lai de franchise.

\protect\hypertarget{23_NOTES.xhtmlux5cux23id_1506}{\protect\hyperlink{10_Chapter_Three__THE_HEROIC_DREAM.xhtmlux5cux23id_1505}{216}}.
See further Deschamps, nos. 199, 200, 201, 258, 291, 970, 973, 1017,
1018, 1021, 1201, 1258.

\protect\hypertarget{23_NOTES.xhtmlux5cux23id_1504}{\protect\hyperlink{10_Chapter_Three__THE_HEROIC_DREAM.xhtmlux5cux23id_1503}{217}}.
Deschamps, XI, p. 94.

\protect\hypertarget{23_NOTES.xhtmlux5cux23id_1502}{\protect\hyperlink{10_Chapter_Three__THE_HEROIC_DREAM.xhtmlux5cux23id_1501}{218}}.
N. de Clémanges, Opera, ed. 1613, Epistolaeno. 14, p. 57; no. 18, p. 72;
no. 104, p. 296.

\protect\hypertarget{23_NOTES.xhtmlux5cux23id_1500}{\protect\hyperlink{10_Chapter_Three__THE_HEROIC_DREAM.xhtmlux5cux23id_1499}{219}}.
Joh. de Monasteriolo, Epistolae, Martène et Durand, Ampl. Collectio, II,
c. 1398.

\protect\hypertarget{23_NOTES.xhtmlux5cux23id_1498}{\protect\hyperlink{10_Chapter_Three__THE_HEROIC_DREAM.xhtmlux5cux23id_1497}{220}}.
Joh. de Monasteriolo, Epistolae, c. 1459.

\protect\hypertarget{23_NOTES.xhtmlux5cux23id_1496}{\protect\hyperlink{10_Chapter_Three__THE_HEROIC_DREAM.xhtmlux5cux23id_1495}{221}}.
Alain Chartier, Oeuvres, ed. Duchesne, 1617, p. 391.

\protect\hypertarget{23_NOTES.xhtmlux5cux23id_1494}{\protect\hyperlink{10_Chapter_Three__THE_HEROIC_DREAM.xhtmlux5cux23id_1493}{222}}.
See Roberti Gaguini Epistole et orationes, ed. Thuasne (Paris: E.
Bouillon, 1903), I, p. 37; II, p. 202.

\protect\hypertarget{23_NOTES.xhtmlux5cux23id_1492}{\protect\hyperlink{10_Chapter_Three__THE_HEROIC_DREAM.xhtmlux5cux23id_1491}{223}}.
Oeuvres du roi René, ed. Quatrebarbes, IV, p. 73; see Thuasne, Gaguini,
II, p. 204.

\protect\hypertarget{23_NOTES.xhtmlux5cux23page_412}{\protect\hyperlink{10_Chapter_Three__THE_HEROIC_DREAM.xhtmlux5cux23id_1490}{224}}.
Meschinot, ed. 1522, f. 94, in La Borderie, Bibl. de l'Ecole des
Chartes, LVI, 1895, p. 313.

\protect\hypertarget{23_NOTES.xhtmlux5cux23id_1489}{\protect\hyperlink{10_Chapter_Three__THE_HEROIC_DREAM.xhtmlux5cux23id_1488}{225}}.
See Thuasne, Gaguini, II, p. 205.

\textbf{\emph{Chapter 4}}

\protect\hypertarget{23_NOTES.xhtmlux5cux23id_3102}{\protect\hyperlink{11_Chapter_Four__THE_FORMS_OF_LOVE.xhtmlux5cux23id_3101}{*\textsuperscript{1}}}
``sweet new style.''

\protect\hypertarget{23_NOTES.xhtmlux5cux23id_3104}{\protect\hyperlink{11_Chapter_Four__THE_FORMS_OF_LOVE.xhtmlux5cux23id_3103}{*\textsuperscript{2}}}
``art of love.''

\protect\hypertarget{23_NOTES.xhtmlux5cux23id_3106}{\protect\hyperlink{11_Chapter_Four__THE_FORMS_OF_LOVE.xhtmlux5cux23id_3105}{*\textsuperscript{3}}}
``for them and all their retinue, baths provided with everything
required for the calling of Venus, to take by choice and by election
what they liked best, and all at the expense of the duke.''

\protect\hypertarget{23_NOTES.xhtmlux5cux23id_3108}{\protect\hyperlink{11_Chapter_Four__THE_FORMS_OF_LOVE.xhtmlux5cux23id_3107}{†\textsuperscript{4}}}
``a machine to wet the ladies when they pass under it.''

\protect\hypertarget{23_NOTES.xhtmlux5cux23id_3110}{\protect\hyperlink{11_Chapter_Four__THE_FORMS_OF_LOVE.xhtmlux5cux23id_3109}{*\textsuperscript{5}}}
\emph{One Hundred New Stories}

\protect\hypertarget{23_NOTES.xhtmlux5cux23id_3112}{\protect\hyperlink{11_Chapter_Four__THE_FORMS_OF_LOVE.xhtmlux5cux23id_3111}{†\textsuperscript{6}}}
``honorable and edifying work''

\protect\hypertarget{23_NOTES.xhtmlux5cux23id_3114}{\protect\hyperlink{11_Chapter_Four__THE_FORMS_OF_LOVE.xhtmlux5cux23id_3113}{‡\textsuperscript{7}}}
``very suitable to tell in any good company.''

\protect\hypertarget{23_NOTES.xhtmlux5cux23id_3116}{\protect\hyperlink{11_Chapter_Four__THE_FORMS_OF_LOVE.xhtmlux5cux23id_3115}{*\textsuperscript{8}}}
These are the ten commandments,/True God of love .~.~.

\protect\hypertarget{23_NOTES.xhtmlux5cux23id_3118}{\protect\hyperlink{11_Chapter_Four__THE_FORMS_OF_LOVE.xhtmlux5cux23id_3117}{†\textsuperscript{9}}}
Then call me and command me lay my hands/On a book and order me to
swear,/That I will honestly do my duty/In the matters of love.

\protect\hypertarget{23_NOTES.xhtmlux5cux23id_3120}{\protect\hyperlink{11_Chapter_Four__THE_FORMS_OF_LOVE.xhtmlux5cux23id_3119}{‡\textsuperscript{10}}}
And I have the hope, that he soon/Will sit high in the Paradise of
lovers/As a martyr and highly honored saint.

\protect\hypertarget{23_NOTES.xhtmlux5cux23id_3122}{\protect\hyperlink{11_Chapter_Four__THE_FORMS_OF_LOVE.xhtmlux5cux23id_3121}{*\textsuperscript{11}}}
I have celebrated the obsequies of my lady/In the monastery of love,/And
the service for her soul/Was sung by Dolorous Thought./Many tapers of
pitiful sighs/Have burned in her illumination./Also I had the tomb made/
Of regrets .~.~.

\protect\hypertarget{23_NOTES.xhtmlux5cux23id_3124}{\protect\hyperlink{11_Chapter_Four__THE_FORMS_OF_LOVE.xhtmlux5cux23id_3123}{†\textsuperscript{12}}}
``The Lover Made Member of the Order of Love''

\protect\hypertarget{23_NOTES.xhtmlux5cux23id_3126}{\protect\hyperlink{11_Chapter_Four__THE_FORMS_OF_LOVE.xhtmlux5cux23id_3125}{*\textsuperscript{13}}}
``which was not made of virgin wax.''

\protect\hypertarget{23_NOTES.xhtmlux5cux23id_3128}{\protect\hyperlink{11_Chapter_Four__THE_FORMS_OF_LOVE.xhtmlux5cux23id_3127}{*\textsuperscript{14}}}
``of the slanderous \emph{Roman de la rose.''}

\protect\hypertarget{23_NOTES.xhtmlux5cux23id_3130}{\protect\hyperlink{11_Chapter_Four__THE_FORMS_OF_LOVE.xhtmlux5cux23id_3129}{†\textsuperscript{15}}}
``on the feathers and wings of my various thoughts, from one place to
another, to the holy court of Christianity.''

\protect\hypertarget{23_NOTES.xhtmlux5cux23id_3132}{\protect\hyperlink{11_Chapter_Four__THE_FORMS_OF_LOVE.xhtmlux5cux23id_3131}{*\textsuperscript{16}}}
``Shame, Fear, and Danger {[}Virtue on Guard{]}, the good porter who
would not dare, who would not deign to sanction even an impure kiss or
dissolute look, or attractive smile or light speech.''

\protect\hypertarget{23_NOTES.xhtmlux5cux23id_3134}{\protect\hyperlink{11_Chapter_Four__THE_FORMS_OF_LOVE.xhtmlux5cux23id_3133}{†\textsuperscript{17}}}
``how all young girls should sell their bodies early and dearly, without
fear and without shame, and that they should make light of deceit and
perjury.''

\protect\hypertarget{23_NOTES.xhtmlux5cux23id_3136}{\protect\hyperlink{11_Chapter_Four__THE_FORMS_OF_LOVE.xhtmlux5cux23id_3135}{*\textsuperscript{18}}}
``to spend a part of the time more pleasantly and to find new joys
arising---to honor, praise, recommend and serve all women and maidens.''

\protect\hypertarget{23_NOTES.xhtmlux5cux23id_3138}{\protect\hyperlink{11_Chapter_Four__THE_FORMS_OF_LOVE.xhtmlux5cux23id_3137}{*\textsuperscript{19}}}
``in the form of amorous lawsuits to defend different positions.''

\protect\hypertarget{23_NOTES.xhtmlux5cux23id_3140}{\protect\hyperlink{11_Chapter_Four__THE_FORMS_OF_LOVE.xhtmlux5cux23id_3139}{*\textsuperscript{20}}}
``signifying novelty.''

\protect\hypertarget{23_NOTES.xhtmlux5cux23id_3142}{\protect\hyperlink{11_Chapter_Four__THE_FORMS_OF_LOVE.xhtmlux5cux23id_3141}{†\textsuperscript{21}}}
Instead of in blue, lady, you dress in green.

\protect\hypertarget{23_NOTES.xhtmlux5cux23id_3144}{\protect\hyperlink{11_Chapter_Four__THE_FORMS_OF_LOVE.xhtmlux5cux23id_3143}{‡\textsuperscript{22}}}
``fools of court and changers of names,''

\protect\hypertarget{23_NOTES.xhtmlux5cux23id_3146}{\protect\hyperlink{11_Chapter_Four__THE_FORMS_OF_LOVE.xhtmlux5cux23id_3145}{§\textsuperscript{23}}}
A little darling of great courage/who carried bunches of mottos

\protect\hypertarget{23_NOTES.xhtmlux5cux23id_3148}{\protect\hyperlink{11_Chapter_Four__THE_FORMS_OF_LOVE.xhtmlux5cux23id_3147}{*\textsuperscript{24}}}
``The King Who Does Not Lie,'' ``The Castle of Love,'' ``Sales of
Love,'' and ``Games for Sale''

\protect\hypertarget{23_NOTES.xhtmlux5cux23id_3150}{\protect\hyperlink{11_Chapter_Four__THE_FORMS_OF_LOVE.xhtmlux5cux23id_3149}{†\textsuperscript{25}}}
I sell you the hollyhock./---Belle, I dare not tell/How love draws me
towards you/But you know it without a word.

\protect\hypertarget{23_NOTES.xhtmlux5cux23id_3152}{\protect\hyperlink{11_Chapter_Four__THE_FORMS_OF_LOVE.xhtmlux5cux23id_3151}{‡\textsuperscript{26}}}
Of the Castle of Love I ask you:/Tell me the first foundation!/---To
love loyally./Now mention the principal wall/Which makes it fine, strong
and sure!/---To conceal wisely. /Tell me what are the loopholes,/The
windows and the stones!/---Alluring looks. /Friend, mention the
porter!/Ill-speaking danger {[}Virtue on Guard{]}./What is the key that
can unlock it?/---Courteous request.

\protect\hypertarget{23_NOTES.xhtmlux5cux23id_3154}{\protect\hyperlink{11_Chapter_Four__THE_FORMS_OF_LOVE.xhtmlux5cux23id_3153}{*\textsuperscript{27}}}
``graceful games of love and his adventures''

\protect\hypertarget{23_NOTES.xhtmlux5cux23id_3156}{\protect\hyperlink{11_Chapter_Four__THE_FORMS_OF_LOVE.xhtmlux5cux23id_3155}{†\textsuperscript{28}}}
``Lady, I would prefer to hear her well spoken of and that I should find
her bad.''

\protect\hypertarget{23_NOTES.xhtmlux5cux23id_2312}{\protect\hyperlink{11_Chapter_Four__THE_FORMS_OF_LOVE.xhtmlux5cux23id_2311}{*\textsuperscript{29}}}
``I shall make, to your glory and praise, something that will be well
remembered.'' ``And, my very sweet heart, are you sorry because we have
begun so late? .~.~. By God, so am I; .~.~. but here is the remedy: let
us enjoy life as much as circumstances permit, so that we may make up
for the time we have lost; and that people may speak of our love a
hundred years hence, and all well and honorably; for if there were evil,
you would conceal it from God, if you could.''

\protect\hypertarget{23_NOTES.xhtmlux5cux23id_3158}{\protect\hyperlink{11_Chapter_Four__THE_FORMS_OF_LOVE.xhtmlux5cux23id_3157}{*\textsuperscript{30}}}
``very devoutly.''

\protect\hypertarget{23_NOTES.xhtmlux5cux23id_3160}{\protect\hyperlink{11_Chapter_Four__THE_FORMS_OF_LOVE.xhtmlux5cux23id_3159}{*\textsuperscript{31}}}
.~.~. When the priest said, Agnus Dei,/Faith I owe to Saint
Crepais,/Sweetly she gave me the pax, /Between two pillars of the
church. /And I needed it indeed, / For my amorous heart was/Troubled,
that we soon had to part.

\protect\hypertarget{23_NOTES.xhtmlux5cux23id_3162}{\protect\hyperlink{11_Chapter_Four__THE_FORMS_OF_LOVE.xhtmlux5cux23id_3161}{*\textsuperscript{32}}}
``false long and pensive looks and little sighs, and wonderful emotional
faces, and who have more words at hand than other people.''

\protect\hypertarget{23_NOTES.xhtmlux5cux23id_3164}{\protect\hyperlink{11_Chapter_Four__THE_FORMS_OF_LOVE.xhtmlux5cux23id_3163}{†\textsuperscript{33}}}
``\,'Mademoiselle, it would be better to fall into your hands as a
prisoner than into many another's, and I think your prison would not be
so hard as that of the English.'---She replied that she had recently
seen one whom she could wish to be her prisoner. And then I asked her,
if she would make a bad prison for him, and she said not at all, and
that she would hold him as dear as her own person, and I told her that
the man would be very fortunate in having such a sweet and noble prison.
What can I say? She could talk well enough, and it seemed, to judge from
her conversation, that she knew a great deal, and her eyes also had a
lively and lightsome expression. .~.~. And when we had departed my lord
my father said to me, `What do you think of her whom you have seen? Tell
me your opinion.' `Monseigneur, she seems to me all well and good, but I
shall never be nearer to her than I am now, if you please.'\,''

\protect\hypertarget{23_NOTES.xhtmlux5cux23id_2314}{\protect\hyperlink{11_Chapter_Four__THE_FORMS_OF_LOVE.xhtmlux5cux23id_2313}{*\textsuperscript{34}}}
``to marry for love''

\protect\hypertarget{23_NOTES.xhtmlux5cux23id_2315}{\protect\hyperlink{11_Chapter_Four__THE_FORMS_OF_LOVE.xhtmlux5cux23id_2316}{†\textsuperscript{35}}}
``in hope of marriage''

\protect\hypertarget{23_NOTES.xhtmlux5cux23id_2318}{\protect\hyperlink{11_Chapter_Four__THE_FORMS_OF_LOVE.xhtmlux5cux23id_2317}{‡\textsuperscript{36}}}
``For I have heard many women say who were in love in their youth, that
when they were in church, their thoughts and fancies made them dwell
more on those nimble imaginations and delights of their love-affairs
than on the service of God, and the art of love is of such a nature,
that just at the holiest moments of the service, that is to say, when
the priest holds our Lord on the altar, the most of these little
thoughts will come to them.''

\protect\hypertarget{23_NOTES.xhtmlux5cux23id_2930}{\protect\hyperlink{11_Chapter_Four__THE_FORMS_OF_LOVE.xhtmlux5cux23id_2929}{*\textsuperscript{37}}}
I have seen a king of Sicily/Turn shepherd/And his gentle wife/Take to
the same trade/Carrying the shepherd's pouch,/The crook and
hat,/Dwelling on the heath/Near their flock.

\protect\hypertarget{23_NOTES.xhtmlux5cux23id_2932}{\protect\hyperlink{11_Chapter_Four__THE_FORMS_OF_LOVE.xhtmlux5cux23id_2931}{†\textsuperscript{38}}}
Seigneur, you are God's shepherd;/Guard his animals loyally,/Lead them
to the field or orchard,/But do not lose them by any means,/For your
trouble you will be well paid/If you guard them well, and if you do
not,/You received this name in an evil hour.

\protect\hypertarget{23_NOTES.xhtmlux5cux23id_2934}{\protect\hyperlink{11_Chapter_Four__THE_FORMS_OF_LOVE.xhtmlux5cux23id_2933}{*\textsuperscript{39}}}
``noble shepherdesses who formerly tended and guarded the sheep of the
country over there {[}the Netherlands{]}.''

\protect\hypertarget{23_NOTES.xhtmlux5cux23id_2936}{\protect\hyperlink{11_Chapter_Four__THE_FORMS_OF_LOVE.xhtmlux5cux23id_2935}{†\textsuperscript{40}}}
``all in the style of a pastoral.''

\protect\hypertarget{23_NOTES.xhtmlux5cux23id_2938}{\protect\hyperlink{11_Chapter_Four__THE_FORMS_OF_LOVE.xhtmlux5cux23id_2937}{‡\textsuperscript{41}}}
Rest to my poor sheep,/Who here in great need,/Your shepherd shall not
sleep,/And now that you are scattered,/To God thou shall betake,/Accept
his wholesome word,/Live as pious Christians,/Soon it will be done.

\protect\hypertarget{23_NOTES.xhtmlux5cux23id_2940}{\protect\hyperlink{11_Chapter_Four__THE_FORMS_OF_LOVE.xhtmlux5cux23id_2939}{§\textsuperscript{42}}}
``the shepherd and the shepherdess''

\protect\hypertarget{23_NOTES.xhtmlux5cux23id_2942}{\protect\hyperlink{11_Chapter_Four__THE_FORMS_OF_LOVE.xhtmlux5cux23id_2941}{*\textsuperscript{43}}}
My bread is good; no one needs to clothe me;/The water is healthy, which
I desire to drink,/I do not fear either tyrant or poison.

\protect\hypertarget{23_NOTES.xhtmlux5cux23id_2944}{\protect\hyperlink{11_Chapter_Four__THE_FORMS_OF_LOVE.xhtmlux5cux23id_2943}{†\textsuperscript{44}}}
``that instrument of bestial people.''

\protect\hypertarget{23_NOTES.xhtmlux5cux23id_2946}{\protect\hyperlink{11_Chapter_Four__THE_FORMS_OF_LOVE.xhtmlux5cux23id_2945}{‡\textsuperscript{45}}}
``which causes strong breath''

\protect\hypertarget{23_NOTES.xhtmlux5cux23id_2948}{\protect\hyperlink{11_Chapter_Four__THE_FORMS_OF_LOVE.xhtmlux5cux23id_2947}{§\textsuperscript{46}}}
``All the birds from here to Babylon,''

\protect\hypertarget{23_NOTES.xhtmlux5cux23id_2950}{\protect\hyperlink{11_Chapter_Four__THE_FORMS_OF_LOVE.xhtmlux5cux23id_2949}{*\textsuperscript{47}}}
Marriage is a sweet thing,/I know that from my own experience .~.~.

\protect\hypertarget{23_NOTES.xhtmlux5cux23id_1487}{\protect\hyperlink{11_Chapter_Four__THE_FORMS_OF_LOVE.xhtmlux5cux23id_1486}{1}}.
{[}Trans.{]} \emph{La vita nuova. The New Life}---Dante's masterpiece of
courtly love poetry.

\protect\hypertarget{23_NOTES.xhtmlux5cux23id_1485}{\protect\hyperlink{11_Chapter_Four__THE_FORMS_OF_LOVE.xhtmlux5cux23id_1484}{2}}.
As the newest {[}1914{]} translator of the Roman de la rose, E.
Langlois, renders the name.

\protect\hypertarget{23_NOTES.xhtmlux5cux23id_1483}{\protect\hyperlink{11_Chapter_Four__THE_FORMS_OF_LOVE.xhtmlux5cux23id_1482}{3}}.
Chastellain, IV, p. 165.

\protect\hypertarget{23_NOTES.xhtmlux5cux23id_1481}{\protect\hyperlink{11_Chapter_Four__THE_FORMS_OF_LOVE.xhtmlux5cux23id_1480}{4}}.
Basin, II, p. 224.

\protect\hypertarget{23_NOTES.xhtmlux5cux23id_1479}{\protect\hyperlink{11_Chapter_Four__THE_FORMS_OF_LOVE.xhtmlux5cux23id_1478}{5}}.
La Marche, II, p. 350.

\protect\hypertarget{23_NOTES.xhtmlux5cux23id_1477}{\protect\hyperlink{11_Chapter_Four__THE_FORMS_OF_LOVE.xhtmlux5cux23id_1476}{6}}.
{[}Trans.{]} ``And if they passed that night together in great delight,
one can well believe it.'' In Tuchmann, \emph{Mirror}, p. 420.

\protect\hypertarget{23_NOTES.xhtmlux5cux23id_1475}{\protect\hyperlink{11_Chapter_Four__THE_FORMS_OF_LOVE.xhtmlux5cux23id_1474}{7}}.
Froissart, IX, pp. 223--236; Deschamps, VII, no. 1282.

\protect\hypertarget{23_NOTES.xhtmlux5cux23id_1473}{\protect\hyperlink{11_Chapter_Four__THE_FORMS_OF_LOVE.xhtmlux5cux23id_1472}{8}}.
Cent nouvelles nouvelles, ed. Wright, II, p. 15; see I, p. 277; II, pp.
20, 168, and so forth, and the Quinze joyes de mariage, passim.

\protect\hypertarget{23_NOTES.xhtmlux5cux23id_1471}{\protect\hyperlink{11_Chapter_Four__THE_FORMS_OF_LOVE.xhtmlux5cux23id_1470}{9}}.
Petit de Julleville, Jean Regnier, balli d'Auxerre, Revue d'hist. litt.
de la France, 1895, p. 157, in Doutrepont, p. 383; see Deschamps, VIII,
p. 43.

\protect\hypertarget{23_NOTES.xhtmlux5cux23id_1469}{\protect\hyperlink{11_Chapter_Four__THE_FORMS_OF_LOVE.xhtmlux5cux23id_1468}{10}}.
Deschamps, VI, p. 112, no. 1169, La leçon de musique.

\protect\hypertarget{23_NOTES.xhtmlux5cux23id_1467}{\protect\hyperlink{11_Chapter_Four__THE_FORMS_OF_LOVE.xhtmlux5cux23id_1466}{11}}.
Charles d'Orléans, Poésies complètes, Paris 1874, 2 vols., I, pp. 12,
42.

\protect\hypertarget{23_NOTES.xhtmlux5cux23id_1465}{\protect\hyperlink{11_Chapter_Four__THE_FORMS_OF_LOVE.xhtmlux5cux23id_1464}{12}}.
Charles d'Orléans, I, p. 88.

\protect\hypertarget{23_NOTES.xhtmlux5cux23id_1463}{\protect\hyperlink{11_Chapter_Four__THE_FORMS_OF_LOVE.xhtmlux5cux23id_1462}{13}}.
Deschamps, VI, p 82, no. 1151; see also V, p. 132, no. 926; IX, p. 94,
c. 31; VI, p. 138, no. 1184; XI, p. 18, no. 1438; XI, pp. 269, 2861.

\protect\hypertarget{23_NOTES.xhtmlux5cux23id_1461}{\protect\hyperlink{11_Chapter_Four__THE_FORMS_OF_LOVE.xhtmlux5cux23id_1460}{14}}.
Christine de Pisan, I'Epistre au dieu d'amours, Oeuvres poétiques, ed.
M. Roy, II, p. 1.

\protect\hypertarget{23_NOTES.xhtmlux5cux23id_1459}{\protect\hyperlink{11_Chapter_Four__THE_FORMS_OF_LOVE.xhtmlux5cux23id_1458}{15}}.
Joh. de Monasteriolo, Epistolae, Martène et Durand, Amplissima.
collectio, II col., p. 1409, 1421, 1422.

\protect\hypertarget{23_NOTES.xhtmlux5cux23id_1457}{\protect\hyperlink{11_Chapter_Four__THE_FORMS_OF_LOVE.xhtmlux5cux23id_1456}{16}}.
Piaget, Chronologie des épistres sur le Roman de la rose, Etudes romanes
dédiées à Gaston Paris, Paris, 1891, p. 119.

\protect\hypertarget{23_NOTES.xhtmlux5cux23id_1455}{\protect\hyperlink{11_Chapter_Four__THE_FORMS_OF_LOVE.xhtmlux5cux23id_1454}{17}}.
Gerson, Opera, III, p. 597; Gerson, Considérations sur St. Joseph, III,
p. 866; Sermo contra luxuriem, III, pp. 923, 925, 930, 968.

\protect\hypertarget{23_NOTES.xhtmlux5cux23id_1453}{\protect\hyperlink{11_Chapter_Four__THE_FORMS_OF_LOVE.xhtmlux5cux23id_1452}{18}}.
{[}Trans.{]} \emph{Old Woman}: Another allegorical figure of the
\emph{Roman de la rose}.

\protect\hypertarget{23_NOTES.xhtmlux5cux23id_1451}{\protect\hyperlink{11_Chapter_Four__THE_FORMS_OF_LOVE.xhtmlux5cux23id_1450}{19}}.
After Gerson.---The ms. letter of Pierre Col in the Bibl. nationale mss
français 1563, f. 183, was not accessible to me.

\protect\hypertarget{23_NOTES.xhtmlux5cux23id_1449}{\protect\hyperlink{11_Chapter_Four__THE_FORMS_OF_LOVE.xhtmlux5cux23id_1448}{20}}.
{[}Trans.{]} ``As it is written in the law of the Lord, Every male that
openeth the womb shall be called holy to the Lord.''

\protect\hypertarget{23_NOTES.xhtmlux5cux23id_1447}{\protect\hyperlink{11_Chapter_Four__THE_FORMS_OF_LOVE.xhtmlux5cux23id_1446}{21}}.
Bibl. de l'école des chartes, LX, 1899, p. 569.

\protect\hypertarget{23_NOTES.xhtmlux5cux23id_1445}{\protect\hyperlink{11_Chapter_Four__THE_FORMS_OF_LOVE.xhtmlux5cux23id_1444}{22}}.
E. Langlois, Le roman de la rose (Société des anciens textes français),
1914, T.I., Introduction, p. 36.

\protect\hypertarget{23_NOTES.xhtmlux5cux23id_1443}{\protect\hyperlink{11_Chapter_Four__THE_FORMS_OF_LOVE.xhtmlux5cux23id_1442}{23}}.
Ronsard, Amours, no. CLXI.

\protect\hypertarget{23_NOTES.xhtmlux5cux23id_1441}{\protect\hyperlink{11_Chapter_Four__THE_FORMS_OF_LOVE.xhtmlux5cux23id_1440}{24}}.
A. Piaget, La cour amoureuse dite de Charles VI, Romania, XX p. 417,
XXXI p. 599, Doutrepont, p. 367.

\protect\hypertarget{23_NOTES.xhtmlux5cux23id_1439}{\protect\hyperlink{11_Chapter_Four__THE_FORMS_OF_LOVE.xhtmlux5cux23id_1438}{25}}.
Leroux de Lincy, Tentative de rapt etc. en 1405, Bibl. de l'école de
chartes, 2. serie, III, 1846, p. 316.

\protect\hypertarget{23_NOTES.xhtmlux5cux23id_1437}{\protect\hyperlink{11_Chapter_Four__THE_FORMS_OF_LOVE.xhtmlux5cux23id_1436}{26}}.
Piaget, Romania, XX, p. 447.

\protect\hypertarget{23_NOTES.xhtmlux5cux23page_413}{\protect\hyperlink{11_Chapter_Four__THE_FORMS_OF_LOVE.xhtmlux5cux23id_1435}{27}}.
Oeuvres de Rabelais, ed. Abel Lefranc c.s. I., Gargantua chap. 9, p. 96.

\protect\hypertarget{23_NOTES.xhtmlux5cux23id_1434}{\protect\hyperlink{11_Chapter_Four__THE_FORMS_OF_LOVE.xhtmlux5cux23id_1433}{28}}.
Guillaume de Machaut, Le livre du voir-dit, ed. p. P. Paris (Société des
bibliophiles françois), 1875, pp. 82, 213, 214, 240, 299, 309, 347, 351.

\protect\hypertarget{23_NOTES.xhtmlux5cux23id_1432}{\protect\hyperlink{11_Chapter_Four__THE_FORMS_OF_LOVE.xhtmlux5cux23id_1431}{29}}.
Juvenal des Ursins, p. 496.

\protect\hypertarget{23_NOTES.xhtmlux5cux23id_1430}{\protect\hyperlink{11_Chapter_Four__THE_FORMS_OF_LOVE.xhtmlux5cux23id_1429}{30}}.
Rabelais, Gargantua, chap. 9

\protect\hypertarget{23_NOTES.xhtmlux5cux23id_1428}{\protect\hyperlink{11_Chapter_Four__THE_FORMS_OF_LOVE.xhtmlux5cux23id_1427}{31}}.
Coquillart, Droits nouveaux, I, p. 111.

\protect\hypertarget{23_NOTES.xhtmlux5cux23id_1426}{\protect\hyperlink{11_Chapter_Four__THE_FORMS_OF_LOVE.xhtmlux5cux23id_1425}{32}}.
Christine de Pisan, I, p. 187ff.

\protect\hypertarget{23_NOTES.xhtmlux5cux23id_1424}{\protect\hyperlink{11_Chapter_Four__THE_FORMS_OF_LOVE.xhtmlux5cux23id_1423}{33}}.
E. Hoepffner, Frage- und Antwortspiele in der franz. Literatur des 14
Jahrh., Zeitschrift f. roman. Philogie, XXXIII, 1909, pp. 695, 703.

\protect\hypertarget{23_NOTES.xhtmlux5cux23id_1422}{\protect\hyperlink{11_Chapter_Four__THE_FORMS_OF_LOVE.xhtmlux5cux23id_1421}{34}}.
Christine de Pisan, Le dit de la rose, 75, Oeuvres poétiques, II, p. 31.

\protect\hypertarget{23_NOTES.xhtmlux5cux23id_1420}{\protect\hyperlink{11_Chapter_Four__THE_FORMS_OF_LOVE.xhtmlux5cux23id_1419}{35}}.
Machaut, Remède de fortune, 3879ff. Oeuvres, ed. Hoepffner (Soc. des
anc. textes français), 1908-11, 2 vols., II, p. 142.

\protect\hypertarget{23_NOTES.xhtmlux5cux23id_1418}{\protect\hyperlink{11_Chapter_Four__THE_FORMS_OF_LOVE.xhtmlux5cux23id_1417}{36}}.
Christine de Pisan, Le livre des trois jugements, Oeuvres poétiques, II,
p. 111.

\protect\hypertarget{23_NOTES.xhtmlux5cux23id_1416}{\protect\hyperlink{11_Chapter_Four__THE_FORMS_OF_LOVE.xhtmlux5cux23id_1415}{37}}.
{[}Trans.{]} \emph{Bettina}. Bettina von Arnim was a young woman who
carried on a correspondence with the older Goethe. The
\emph{Briefwechsel Goethes mit einem Kinde} was published in 1835.

\protect\hypertarget{23_NOTES.xhtmlux5cux23id_1414}{\protect\hyperlink{11_Chapter_Four__THE_FORMS_OF_LOVE.xhtmlux5cux23id_1413}{38}}.
Le livre du voir-dit. The hypothesis that there is no reality to the
history of this affair (following Hanf, Zeitschrift fur roman. Philogie,
XXII, p. 145) lacks any proof.

\protect\hypertarget{23_NOTES.xhtmlux5cux23id_1412}{\protect\hyperlink{11_Chapter_Four__THE_FORMS_OF_LOVE.xhtmlux5cux23id_1411}{39}}.
A castle near Château Thierry.

\protect\hypertarget{23_NOTES.xhtmlux5cux23id_1410}{\protect\hyperlink{11_Chapter_Four__THE_FORMS_OF_LOVE.xhtmlux5cux23id_1409}{40}}.
Voir-Dit, lettre, II, p. 20.

\protect\hypertarget{23_NOTES.xhtmlux5cux23id_1408}{\protect\hyperlink{11_Chapter_Four__THE_FORMS_OF_LOVE.xhtmlux5cux23id_1407}{41}}.
Voir-Dit, lettre, XXVII, p. 203.

\protect\hypertarget{23_NOTES.xhtmlux5cux23id_1406}{\protect\hyperlink{11_Chapter_Four__THE_FORMS_OF_LOVE.xhtmlux5cux23id_1405}{42}}.
Voir-Dit, pp. 20, 96, 146, 154, 162.

\protect\hypertarget{23_NOTES.xhtmlux5cux23id_1404}{\protect\hyperlink{11_Chapter_Four__THE_FORMS_OF_LOVE.xhtmlux5cux23id_1403}{43}}.
The kiss separated by a leaf reoccurs: see Le grand garde derrière str.
6, W. G. C. Byvanck, Un poete inconnu de la société de François Villon,
Paris, Champion, 1891, p. 27. Compare the figure of speech: ``he held no
leaf in front of his mouth.''

\protect\hypertarget{23_NOTES.xhtmlux5cux23id_1402}{\protect\hyperlink{11_Chapter_Four__THE_FORMS_OF_LOVE.xhtmlux5cux23id_1401}{44}}.
Voir-Dit, pp. 143, 144.

\protect\hypertarget{23_NOTES.xhtmlux5cux23id_1400}{\protect\hyperlink{11_Chapter_Four__THE_FORMS_OF_LOVE.xhtmlux5cux23id_1399}{45}}.
Voir-Dit, p. no.

\protect\hypertarget{23_NOTES.xhtmlux5cux23id_1398}{\protect\hyperlink{11_Chapter_Four__THE_FORMS_OF_LOVE.xhtmlux5cux23id_1397}{46}}.
See above, p. 48.

\protect\hypertarget{23_NOTES.xhtmlux5cux23id_1396}{\protect\hyperlink{11_Chapter_Four__THE_FORMS_OF_LOVE.xhtmlux5cux23id_1395}{47}}.
Voir-Dit, pp. 98, 70.

\protect\hypertarget{23_NOTES.xhtmlux5cux23id_1394}{\protect\hyperlink{11_Chapter_Four__THE_FORMS_OF_LOVE.xhtmlux5cux23id_1393}{48}}.
Le livre du chevalier de la Tour Landry, ed. A. de Montaiglon (Bibl.
elzevirienne), 1854.

\protect\hypertarget{23_NOTES.xhtmlux5cux23id_1392}{\protect\hyperlink{11_Chapter_Four__THE_FORMS_OF_LOVE.xhtmlux5cux23id_1391}{49}}.
p. 245.

\protect\hypertarget{23_NOTES.xhtmlux5cux23id_1390}{\protect\hyperlink{11_Chapter_Four__THE_FORMS_OF_LOVE.xhtmlux5cux23id_1389}{50}}.
p. 28.

\protect\hypertarget{23_NOTES.xhtmlux5cux23id_1388}{\protect\hyperlink{11_Chapter_Four__THE_FORMS_OF_LOVE.xhtmlux5cux23id_1387}{51}}.
See above, p. 32.

\protect\hypertarget{23_NOTES.xhtmlux5cux23id_1386}{\protect\hyperlink{11_Chapter_Four__THE_FORMS_OF_LOVE.xhtmlux5cux23id_1385}{52}}.
The sentence is completely illogical (pensée .~.~. fait penser .~.~. à
pensiers); they seize upon one, but nowhere so often as in church.

\protect\hypertarget{23_NOTES.xhtmlux5cux23id_1384}{\protect\hyperlink{11_Chapter_Four__THE_FORMS_OF_LOVE.xhtmlux5cux23id_1383}{53}}.
pp. 249, 252--254.

\protect\hypertarget{23_NOTES.xhtmlux5cux23id_1382}{\protect\hyperlink{11_Chapter_Four__THE_FORMS_OF_LOVE.xhtmlux5cux23id_1381}{54}}.
Recollections des merveilles, Chastellain, VII, p. 200; see the
description of the Joutes de Saint Inglevert, mentioned by Kervyn,
Froissart, XIV, p. 406.

\protect\hypertarget{23_NOTES.xhtmlux5cux23id_1380}{\protect\hyperlink{11_Chapter_Four__THE_FORMS_OF_LOVE.xhtmlux5cux23id_1379}{55}}.
Le pastoralet, ed. Kervyn de Lettenhove (Chron. rel. à l'hist. de Belg.
sous la dom. des ducs de Bourg.), II, p. 573. For this mixture of
pastoral form and political purpose, the poet finds his parallel in no
less a man than Ariosto, whose single pastoral composition was dedicated
to the defense of his patron,
Cardinal\protect\hypertarget{23_NOTES.xhtmlux5cux23page_414}{}{}Ippolito
d'Este, on the occasion of the plot by Albertino Boschetti (1506). The
case of the cardinal was hardly better than that of John the Fearless,
and the support of Ariosto hardly more sympathetic than that of the
unknown Burgundian. See G. Bertoni, L'Orlando furioso e la rinascenza a
Ferrara, Modena, 1919, pp. 42, 247.

\protect\hypertarget{23_NOTES.xhtmlux5cux23id_1378}{\protect\hyperlink{11_Chapter_Four__THE_FORMS_OF_LOVE.xhtmlux5cux23id_1377}{56}}.
P. 2151.

\protect\hypertarget{23_NOTES.xhtmlux5cux23id_1376}{\protect\hyperlink{11_Chapter_Four__THE_FORMS_OF_LOVE.xhtmlux5cux23id_1375}{57}}.
Meschinot, Les Lunettes des princes, in La Borderie (Bibl. de l'Ec. des
chartes, LVI, 1895), p. 606.

\protect\hypertarget{23_NOTES.xhtmlux5cux23id_1374}{\protect\hyperlink{11_Chapter_Four__THE_FORMS_OF_LOVE.xhtmlux5cux23id_1373}{58}}.
La Marche, III, p. 135; see Molinet, Recollection des merveilles, about
the imprisonment of Maximillian at Bruges; ``Les moutons detenterent En
son parc le bergier,'' Faictz et dictz, f. 208 vso.

\protect\hypertarget{23_NOTES.xhtmlux5cux23id_1372}{\protect\hyperlink{11_Chapter_Four__THE_FORMS_OF_LOVE.xhtmlux5cux23id_1371}{59}}.
Molinet, IV, p. 389.

\protect\hypertarget{23_NOTES.xhtmlux5cux23id_1370}{\protect\hyperlink{11_Chapter_Four__THE_FORMS_OF_LOVE.xhtmlux5cux23id_1369}{60}}.
{[}Trans.{]} \emph{Leewendalers}. A one-act play by Vondel.

\protect\hypertarget{23_NOTES.xhtmlux5cux23id_1368}{\protect\hyperlink{11_Chapter_Four__THE_FORMS_OF_LOVE.xhtmlux5cux23id_1367}{61}}.
{[}Trans.{]} The Dutch national anthem.

\protect\hypertarget{23_NOTES.xhtmlux5cux23id_1366}{\protect\hyperlink{11_Chapter_Four__THE_FORMS_OF_LOVE.xhtmlux5cux23id_1365}{62}}.
Molinet, I, pp. 190, 194; III, p. 138; see Juvenal des Ursins. p. 382.

\emph{\protect\hypertarget{23_NOTES.xhtmlux5cux23id_1364}{\protect\hyperlink{11_Chapter_Four__THE_FORMS_OF_LOVE.xhtmlux5cux23id_1363}{63}}}.
Deschamps, II, p. 213, Lay de franchise, see Chr. de Pisan, Le dit de al
Pastoure, Le pastoralet, Roi René, Regnant et Jehanneton, Martial
d'Auvergne, vigilles du roi Charles VII, etc., etc.

\protect\hypertarget{23_NOTES.xhtmlux5cux23id_1362}{\protect\hyperlink{11_Chapter_Four__THE_FORMS_OF_LOVE.xhtmlux5cux23id_1361}{64}}.
Deschamps, no. 923, see XI, p. 322.

\protect\hypertarget{23_NOTES.xhtmlux5cux23id_1360}{\protect\hyperlink{11_Chapter_Four__THE_FORMS_OF_LOVE.xhtmlux5cux23id_1359}{65}}.
Villon, ed. Longnon, p. 83.

\emph{\protect\hypertarget{23_NOTES.xhtmlux5cux23id_1358}{\protect\hyperlink{11_Chapter_Four__THE_FORMS_OF_LOVE.xhtmlux5cux23id_1357}{66}}}.
Gerson, Opera, III, p. 302.

\protect\hypertarget{23_NOTES.xhtmlux5cux23id_1356}{\protect\hyperlink{11_Chapter_Four__THE_FORMS_OF_LOVE.xhtmlux5cux23id_1355}{67}}.
L'epistre au dieu d'amours, II, p. 14.

\protect\hypertarget{23_NOTES.xhtmlux5cux23id_1354}{\protect\hyperlink{11_Chapter_Four__THE_FORMS_OF_LOVE.xhtmlux5cux23id_1353}{68}}.
Quinze joyes de mariage, p. 222.

\protect\hypertarget{23_NOTES.xhtmlux5cux23id_1352}{\protect\hyperlink{11_Chapter_Four__THE_FORMS_OF_LOVE.xhtmlux5cux23id_1351}{69}}.
Oeuvres poétiques, I, p. 237, no. 26.

\textbf{\emph{Chapter} 5}

\protect\hypertarget{23_NOTES.xhtmlux5cux23id_2951}{\protect\hyperlink{12_Chapter_Five__THE_VISION_OF_DEAT.xhtmlux5cux23id_2952}{*\textsuperscript{1}}}
``reminder of death''

\protect\hypertarget{23_NOTES.xhtmlux5cux23id_2956}{\protect\hyperlink{12_Chapter_Five__THE_VISION_OF_DEAT.xhtmlux5cux23id_2955}{*\textsuperscript{2}}}
Where is the glory of Babylon? Where is now the terrible/Nebuchadnezzar,
and strong Darius, famous Cyrus?/Like a wheel, left to itself, so they
went away;/ Their glory remains in plenty, it is secure---they, however,
moulder/Where is now the Curia Julia, Where the Julian procession?
Caesar, you vanished!/And you were the fiercest in the whole world and
the mightiest!/ .~.~. /Where is now Marius and Fabricius, who were
strangers to gold?/Where is the honorable death and memorable deeds of
Paulus?/Where is the heavenly Phillipic voice {[}Demosthenes{]}, where
that of heavenly Cicero?/Where is Cato's peacefulness for the citizens
and his scorn for the rebels?/Where is now Regulus? Or where Romulus, or
where Remus?/The rose {[}Rome{]} of yore is but a name, mere names are
left to us.

\protect\hypertarget{23_NOTES.xhtmlux5cux23id_2954}{\protect\hyperlink{12_Chapter_Five__THE_VISION_OF_DEAT.xhtmlux5cux23id_2953}{*\textsuperscript{3}}}
Say, where is Solomon, once so splendid,/Or Sampson, where is he,
invincible chief,/And fair Absalom of the wonderful face,/Or sweet
Jonathan, the most amiable?/Where has Caesar gone, greatest in
power?/Whither the famous rich {[}Crassus{]}, whose whole soul centered
around mealtime?/Say, where is Tullius {[}Cicero{]}, famous for his
speech,/Where is Aristotle, the greatest in genius?

\protect\hypertarget{23_NOTES.xhtmlux5cux23id_2958}{\protect\hyperlink{12_Chapter_Five__THE_VISION_OF_DEAT.xhtmlux5cux23id_2957}{†\textsuperscript{4}}}
``But where are the snows of yesterday?''

\protect\hypertarget{23_NOTES.xhtmlux5cux23id_2960}{\protect\hyperlink{12_Chapter_Five__THE_VISION_OF_DEAT.xhtmlux5cux23id_2959}{‡\textsuperscript{5}}}
Alas! And the good king of Spain/Whose name I do not know?

\protect\hypertarget{23_NOTES.xhtmlux5cux23id_2962}{\protect\hyperlink{12_Chapter_Five__THE_VISION_OF_DEAT.xhtmlux5cux23id_2961}{*\textsuperscript{6}}}
Once I was beautiful above all women/But by death I became like this,/my
flesh was very beautiful, fresh and soft,/Now it has completely turned
to ashes./ My body was very pleasing and very pretty,/I used frequently
to dress in silk,/ Now I must justly be quite naked. /Clad I was in gray
fur and in miniver,/In a great palace I lived as I wished,/Now I am
lodged in this little coffin./My room was adorned with fine
tapestry,/Now my grave is enveloped by cobwebs.

\protect\hypertarget{23_NOTES.xhtmlux5cux23id_2964}{\protect\hyperlink{12_Chapter_Five__THE_VISION_OF_DEAT.xhtmlux5cux23id_2963}{†\textsuperscript{7}}}
``The Ornament and Triumph of the Ladies''

\protect\hypertarget{23_NOTES.xhtmlux5cux23id_2966}{\protect\hyperlink{12_Chapter_Five__THE_VISION_OF_DEAT.xhtmlux5cux23id_2965}{*\textsuperscript{8}}}
These sweet looks, these eyes made for pleasure,/Remember, they will
lose their luster, /Nose and brows, the eloquent mouth/Will putrefy
.~.~.

\protect\hypertarget{23_NOTES.xhtmlux5cux23id_2968}{\protect\hyperlink{12_Chapter_Five__THE_VISION_OF_DEAT.xhtmlux5cux23id_2967}{†\textsuperscript{9}}}
If you live your natural lifetime,/Of which sixty years is a great
deal,/your beauty will change into ugliness,/Your health into obscure
malady,/And you will be an encumbrance to the earth./If you have a
daughter, you will be a shadow to her,/She will be desired and asked
for,/And the mother will be abandoned by all.

\protect\hypertarget{23_NOTES.xhtmlux5cux23id_2970}{\protect\hyperlink{12_Chapter_Five__THE_VISION_OF_DEAT.xhtmlux5cux23id_2969}{*\textsuperscript{10}}}
What has become of this smooth forehead,/Fair hair, curving brows,/Large
space between the eyes, pretty looks,/With which I caught the most
subtle ones;/ That fine straight nose, neither large nor small, /These
tiny ears close to the head, / The dimpled chin, well-shaped bright
face,/And those beautiful vermillion lips?/ .~.~. /The forehead
wrinkled, hair gray,/The brows bare, lackluster eyes .~.~.

\protect\hypertarget{23_NOTES.xhtmlux5cux23id_2972}{\protect\hyperlink{12_Chapter_Five__THE_VISION_OF_DEAT.xhtmlux5cux23id_2971}{†\textsuperscript{11}}}
``abuse of abominable savagery, practised by some of the faithful in a
horrible way and inconsiderately.''

\protect\hypertarget{23_NOTES.xhtmlux5cux23id_2974}{\protect\hyperlink{12_Chapter_Five__THE_VISION_OF_DEAT.xhtmlux5cux23id_2973}{*\textsuperscript{12}}}
``I made the Dance Macabre''

\protect\hypertarget{23_NOTES.xhtmlux5cux23id_2976}{\protect\hyperlink{12_Chapter_Five__THE_VISION_OF_DEAT.xhtmlux5cux23id_2975}{*\textsuperscript{13}}}
``I am Death, known to all creatures,''

\protect\hypertarget{23_NOTES.xhtmlux5cux23id_2978}{\protect\hyperlink{12_Chapter_Five__THE_VISION_OF_DEAT.xhtmlux5cux23id_2977}{*\textsuperscript{14}}}
My friend, look at my face,/See what doleful death does,/And henceforth
do not forget;/This is she, whom you so loved,/And this body, which is
yours,/ Will become hateful and filthy to you, forever lost;/It will be
a stinking meal/ For the earth and for the worms./Hard death ends all
beauty.

\protect\hypertarget{23_NOTES.xhtmlux5cux23id_2980}{\protect\hyperlink{12_Chapter_Five__THE_VISION_OF_DEAT.xhtmlux5cux23id_2979}{*\textsuperscript{15}}}
There is not a limb nor a form/That does not smell of
putrefaction./Before the soul is outside,/The heart which wants to burst
the body,/Raises and lifts the chest,/Which nearly touches the
backbone./---The face is discolored and pale,/ And the eyes veiled in
the head./Speech fails him,/For the tongue cleaves to the palate./The
pulse trembles and he pants./ .~.~. /The bones are disjointed on all
sides;/There is not a tendon that does not stretch as to burst.

\protect\hypertarget{23_NOTES.xhtmlux5cux23id_2982}{\protect\hyperlink{12_Chapter_Five__THE_VISION_OF_DEAT.xhtmlux5cux23id_2981}{†\textsuperscript{16}}}
Death makes him shudder, pale,/The nose to curve, the veins to swell,
/The neck to inflate, the flesh to soften, /Joints and tendons to grow
and swell.

\protect\hypertarget{23_NOTES.xhtmlux5cux23id_2984}{\protect\hyperlink{12_Chapter_Five__THE_VISION_OF_DEAT.xhtmlux5cux23id_2983}{‡\textsuperscript{17}}}
O female body, which is so soft,/Smooth, suave, precious,/Do these evils
await you?/Yes, or you must go to heaven alive.

\protect\hypertarget{23_NOTES.xhtmlux5cux23id_2986}{\protect\hyperlink{12_Chapter_Five__THE_VISION_OF_DEAT.xhtmlux5cux23id_2985}{*\textsuperscript{18}}}
``beautiful bone chambers''

\protect\hypertarget{23_NOTES.xhtmlux5cux23id_2988}{\protect\hyperlink{12_Chapter_Five__THE_VISION_OF_DEAT.xhtmlux5cux23id_2987}{*\textsuperscript{19}}}
Laborer, who in care and toil/Have lived all your time,/You must die,
that is certain,/No drawing back helps, no struggling./Death should make
you happy,/ Because it frees you from great sorrow.

\protect\hypertarget{23_NOTES.xhtmlux5cux23id_1350}{\protect\hyperlink{12_Chapter_Five__THE_VISION_OF_DEAT.xhtmlux5cux23id_1349}{1}}.
Directorium vitae nobilium, Dionysii opera, t. XXXVII, p. 550; t.
XXXVIII, p. 358

\protect\hypertarget{23_NOTES.xhtmlux5cux23id_1348}{\protect\hyperlink{12_Chapter_Five__THE_VISION_OF_DEAT.xhtmlux5cux23id_1347}{2}}.
Don Juan, c. 11, 76--80, see C. H. Becker, Ubi sunt qui ante in mundo
fuere. Essays dedicated to Ernst Kuhn 11 February 1916, pp. 87--105.

\protect\hypertarget{23_NOTES.xhtmlux5cux23id_1346}{\protect\hyperlink{12_Chapter_Five__THE_VISION_OF_DEAT.xhtmlux5cux23id_1345}{3}}.
Bernardi Morlanensis, De contemptu mundi, ed. Th. Wright, the
Anglo-Latin satirical poets and epigrammatists of the twelfth century
(Rerum Britannicarum medii aevi scriptores), London, 1872, 2 vols., II,
p. 37.

\protect\hypertarget{23_NOTES.xhtmlux5cux23id_1344}{\protect\hyperlink{12_Chapter_Five__THE_VISION_OF_DEAT.xhtmlux5cux23id_1343}{4}}.
Earlier ascribed to Bernhard of Clairvaux, held by a few to be by Walter
Mapes; see H. L. Daniel, Thesaurus hymnologicus, Lipsiae 1841--1856, IV,
p. 288.

\protect\hypertarget{23_NOTES.xhtmlux5cux23id_1342}{\protect\hyperlink{12_Chapter_Five__THE_VISION_OF_DEAT.xhtmlux5cux23id_1341}{5}}.
Deschamps, III, nos. 330, 345, 368, 399; Gerson, Sermo III de defunctis,
Opera, III, p. 1568; Dion. Cart., De quatuor hominim novissimis, Opera,
t. XLI, p. 511; Chastellain, VI, p. 52.

\protect\hypertarget{23_NOTES.xhtmlux5cux23id_1340}{\protect\hyperlink{12_Chapter_Five__THE_VISION_OF_DEAT.xhtmlux5cux23id_1339}{6}}.
Villon, ed. Longnon, p. 33.

\protect\hypertarget{23_NOTES.xhtmlux5cux23id_1338}{\protect\hyperlink{12_Chapter_Five__THE_VISION_OF_DEAT.xhtmlux5cux23id_1337}{7}}.
Villon, ed. Longnon, p. 34.

\protect\hypertarget{23_NOTES.xhtmlux5cux23id_1336}{\protect\hyperlink{12_Chapter_Five__THE_VISION_OF_DEAT.xhtmlux5cux23id_1335}{8}}.
Emile Mâle, L'art religieux à la fin du moyen-âge, Paris 1908, p. 376.

\protect\hypertarget{23_NOTES.xhtmlux5cux23id_1334}{\protect\hyperlink{12_Chapter_Five__THE_VISION_OF_DEAT.xhtmlux5cux23id_1333}{9}}.
See my work De Vidûshaka in het Indisch tooneel, Groningen 1897, p. 77.

\protect\hypertarget{23_NOTES.xhtmlux5cux23id_1332}{\protect\hyperlink{12_Chapter_Five__THE_VISION_OF_DEAT.xhtmlux5cux23id_1331}{10}}.
Odo of Cluny, Collationum lib. III, Migne t. CXXXIII, p. 556.

\protect\hypertarget{23_NOTES.xhtmlux5cux23id_1330}{\protect\hyperlink{12_Chapter_Five__THE_VISION_OF_DEAT.xhtmlux5cux23id_1329}{11}}.
Innocentius III, De contemptu mundi sive de miseria conditionis humanae
libri tres, Migne t. CCXVII, p. 702.

\protect\hypertarget{23_NOTES.xhtmlux5cux23id_1328}{\protect\hyperlink{12_Chapter_Five__THE_VISION_OF_DEAT.xhtmlux5cux23id_1327}{12}}.
Innocentius III, p. 713.

\protect\hypertarget{23_NOTES.xhtmlux5cux23page_415}{\protect\hyperlink{12_Chapter_Five__THE_VISION_OF_DEAT.xhtmlux5cux23id_1326}{13}}.
Oeuvres de roi René, ed. Quatrebarbes I, p. ci. After the fifth and the
eighth lines there appears to be a verse missing. Possibly to rhyme with
``menu vair'' should be ``mangé des vers'' or something similar.

\protect\hypertarget{23_NOTES.xhtmlux5cux23id_1325}{\protect\hyperlink{12_Chapter_Five__THE_VISION_OF_DEAT.xhtmlux5cux23id_1324}{14}}.
Olivier de la Marche, Le parement et triumphe des dames, Paris, Michel
le Noir, 1520, at the end.

\protect\hypertarget{23_NOTES.xhtmlux5cux23id_1323}{\protect\hyperlink{12_Chapter_Five__THE_VISION_OF_DEAT.xhtmlux5cux23id_1322}{15}}.
La Marche, Le parement et triumphe des dames, at the end.

\protect\hypertarget{23_NOTES.xhtmlux5cux23id_1321}{\protect\hyperlink{12_Chapter_Five__THE_VISION_OF_DEAT.xhtmlux5cux23id_1320}{16}}.
Villon, Testament, vs. 453ff., ed. Longnon, p. 39.

\protect\hypertarget{23_NOTES.xhtmlux5cux23id_1319}{\protect\hyperlink{12_Chapter_Five__THE_VISION_OF_DEAT.xhtmlux5cux23id_1318}{17}}.
H. Kern, Het Lied van Ambapâlî uit de Therîgâthâ, Versl. en Meded. der
Koninkl. Akad. van Wetenschappen te Amsterdam 5, III, p. 153, 1917.

\protect\hypertarget{23_NOTES.xhtmlux5cux23id_1317}{\protect\hyperlink{12_Chapter_Five__THE_VISION_OF_DEAT.xhtmlux5cux23id_1316}{18}}.
Molinet, Faictz et dictz, fo. 4, fo. 42 v.

\protect\hypertarget{23_NOTES.xhtmlux5cux23id_1315}{\protect\hyperlink{12_Chapter_Five__THE_VISION_OF_DEAT.xhtmlux5cux23id_1314}{19}}.
Procedure concerning the beatification of Peter of Luxembourg, 1390,
Acta sanctorum Julii, I, p. 562. Compare the regular renewal of the wax
in which the bodies of the English kings and their relatives are
wrapped, Rymer, Foedera VII, 361, 433 = III\textsuperscript{3}, 140,
168, etc.

\protect\hypertarget{23_NOTES.xhtmlux5cux23id_1313}{\protect\hyperlink{12_Chapter_Five__THE_VISION_OF_DEAT.xhtmlux5cux23id_1312}{20}}.
Les Grandes chroniques de France, ed. Paulin Paris 1836--1838, 6 vols.,
VI, p. 334.

\protect\hypertarget{23_NOTES.xhtmlux5cux23id_1311}{\protect\hyperlink{12_Chapter_Five__THE_VISION_OF_DEAT.xhtmlux5cux23id_1310}{21}}.
See the detailed study of Dietrich Schäfer, Mittelalterlicher Brauch bei
de Überfuhrung von Leichen. Sitzungsberichte der preussischen Akademie
der Wissenschaften, 1920, pp. 478--498.

\protect\hypertarget{23_NOTES.xhtmlux5cux23id_1309}{\protect\hyperlink{12_Chapter_Five__THE_VISION_OF_DEAT.xhtmlux5cux23id_1308}{22}}.
Lefèvre de S. Remy, I, p. 200, wherein one must read Suffolk for Oxford.

\protect\hypertarget{23_NOTES.xhtmlux5cux23id_1307}{\protect\hyperlink{12_Chapter_Five__THE_VISION_OF_DEAT.xhtmlux5cux23id_1306}{23}}.
Juvenal des Ursins, p. 567; Journal d'un bourgeois, pp. 237, 307, 671.

\protect\hypertarget{23_NOTES.xhtmlux5cux23id_1305}{\protect\hyperlink{12_Chapter_Five__THE_VISION_OF_DEAT.xhtmlux5cux23id_1304}{24}}.
See also the extensive literature on the theme, G. Huet, Notes
d'histoire littéraire, III, Le moyen âge, XX, 1918, p. 148.

\protect\hypertarget{23_NOTES.xhtmlux5cux23id_1303}{\protect\hyperlink{12_Chapter_Five__THE_VISION_OF_DEAT.xhtmlux5cux23id_1302}{25}}.
See the above cited Emile Mâle, L'art religieux à la fin du moyen-âge,
II, 2. La Mort.

\protect\hypertarget{23_NOTES.xhtmlux5cux23id_1301}{\protect\hyperlink{12_Chapter_Five__THE_VISION_OF_DEAT.xhtmlux5cux23id_1300}{26}}.
Laborde, II, 1, 393.

\protect\hypertarget{23_NOTES.xhtmlux5cux23id_1299}{\protect\hyperlink{12_Chapter_Five__THE_VISION_OF_DEAT.xhtmlux5cux23id_1298}{27}}.
A few reproductions by Mâle, L'art religieux à la fin du moyen-âge, and
in the Gazette des beaux arts 1918, avril--juin, p. 167.

\protect\hypertarget{23_NOTES.xhtmlux5cux23id_1297}{\protect\hyperlink{12_Chapter_Five__THE_VISION_OF_DEAT.xhtmlux5cux23id_1296}{28}}.
Through the researches of Huet, Notes de l'hist. littéraire, it is
clearly seen that a round dance of the dead was the primitive source to
which Goethe returned in his \emph{Totentanz}.

\protect\hypertarget{23_NOTES.xhtmlux5cux23id_1295}{\protect\hyperlink{12_Chapter_Five__THE_VISION_OF_DEAT.xhtmlux5cux23id_1294}{29}}.
{[}Trans.{]} \emph{dubbeldanger}: German \emph{Doppelgänger}.

\protect\hypertarget{23_NOTES.xhtmlux5cux23id_1293}{\protect\hyperlink{12_Chapter_Five__THE_VISION_OF_DEAT.xhtmlux5cux23id_1292}{30}}.
Earlier and incorrectly thought to be older (1350). See G. Ticknor,
Geschichte der schönen Literatur in Spanien (original in English), I p.
77, II p. 598; Grobers Grundrisz, II\textsuperscript{1} p. 1180,
II\textsuperscript{2} p. 428.

\protect\hypertarget{23_NOTES.xhtmlux5cux23id_1291}{\protect\hyperlink{12_Chapter_Five__THE_VISION_OF_DEAT.xhtmlux5cux23id_1290}{31}}.
Oeuvres du roi René, I, p. clii.

\protect\hypertarget{23_NOTES.xhtmlux5cux23id_1289}{\protect\hyperlink{12_Chapter_Five__THE_VISION_OF_DEAT.xhtmlux5cux23id_1288}{32}}.
Chastellain, Le pas de la mort, VI, p. 59.

\protect\hypertarget{23_NOTES.xhtmlux5cux23id_1287}{\protect\hyperlink{12_Chapter_Five__THE_VISION_OF_DEAT.xhtmlux5cux23id_1286}{33}}.
See Innocentius III, De contemptu mundi, II, c. 42; Dion. Cart., De
quatuor hominum novissimis, t. XLI, p. 496.

\protect\hypertarget{23_NOTES.xhtmlux5cux23id_1285}{\protect\hyperlink{12_Chapter_Five__THE_VISION_OF_DEAT.xhtmlux5cux23id_1284}{34}}.
Chastellain, Oeuvres, VI, p. 49.

\protect\hypertarget{23_NOTES.xhtmlux5cux23id_1283}{\protect\hyperlink{12_Chapter_Five__THE_VISION_OF_DEAT.xhtmlux5cux23id_1282}{35}}.
Loc. cit., p. 60.

\protect\hypertarget{23_NOTES.xhtmlux5cux23id_1281}{\protect\hyperlink{12_Chapter_Five__THE_VISION_OF_DEAT.xhtmlux5cux23id_1280}{36}}.
Villon, Testament, XLI, vs. 321--28, ed. Longnon, p. 33.

\protect\hypertarget{23_NOTES.xhtmlux5cux23id_1279}{\protect\hyperlink{12_Chapter_Five__THE_VISION_OF_DEAT.xhtmlux5cux23id_1278}{37}}.
Champion, Villon, I, p. 303.

\protect\hypertarget{23_NOTES.xhtmlux5cux23id_1277}{\protect\hyperlink{12_Chapter_Five__THE_VISION_OF_DEAT.xhtmlux5cux23id_1276}{38}}.
Mâle, L'art religieux .~.~. , p. 389.

\protect\hypertarget{23_NOTES.xhtmlux5cux23id_1275}{\protect\hyperlink{12_Chapter_Five__THE_VISION_OF_DEAT.xhtmlux5cux23id_1274}{39}}.
Leroux de Lincy, Livre des légendes, p. 95.

\protect\hypertarget{23_NOTES.xhtmlux5cux23id_42}{\protect\hyperlink{12_Chapter_Five__THE_VISION_OF_DEAT.xhtmlux5cux23id_41}{40}}.
Le livre des faits, etc., II, p. 184.

\protect\hypertarget{23_NOTES.xhtmlux5cux23page_416}{\protect\hyperlink{12_Chapter_Five__THE_VISION_OF_DEAT.xhtmlux5cux23id_1273}{41}}.
Journal d'un bourgeois, I, pp. 233--234, 392, 276. See further Champion,
Villon, I, p. 306.

\protect\hypertarget{23_NOTES.xhtmlux5cux23id_1272}{\protect\hyperlink{12_Chapter_Five__THE_VISION_OF_DEAT.xhtmlux5cux23id_1271}{42}}.
A. de la Salle, Le reconfort de Madame du Fresne, ed. J Nève, Paris
1903.

\textbf{\emph{Chapter 6}}

\protect\hypertarget{23_NOTES.xhtmlux5cux23id_2990}{\protect\hyperlink{13_Chapter_Six__THE_DEPICTION_OF_TH.xhtmlux5cux23id_2989}{*\textsuperscript{1}}}
``from the mere fantasies of men and their melancholy powers of
imagination''

\protect\hypertarget{23_NOTES.xhtmlux5cux23id_2992}{\protect\hyperlink{13_Chapter_Six__THE_DEPICTION_OF_TH.xhtmlux5cux23id_2991}{*\textsuperscript{2}}}
``then the material semen out of which the body was composed was neither
too solid nor too fluid.''

\protect\hypertarget{23_NOTES.xhtmlux5cux23id_2994}{\protect\hyperlink{13_Chapter_Six__THE_DEPICTION_OF_TH.xhtmlux5cux23id_2993}{†\textsuperscript{3}}}
``a beautiful theological question''

\protect\hypertarget{23_NOTES.xhtmlux5cux23id_2996}{\protect\hyperlink{13_Chapter_Six__THE_DEPICTION_OF_TH.xhtmlux5cux23id_2995}{*\textsuperscript{4}}}
``to see God in passing.''

\protect\hypertarget{23_NOTES.xhtmlux5cux23id_2998}{\protect\hyperlink{13_Chapter_Six__THE_DEPICTION_OF_TH.xhtmlux5cux23id_2997}{†\textsuperscript{5}}}
``a God on a donkey.''

\protect\hypertarget{23_NOTES.xhtmlux5cux23id_3000}{\protect\hyperlink{13_Chapter_Six__THE_DEPICTION_OF_TH.xhtmlux5cux23id_2999}{‡\textsuperscript{6}}}
``and she believed she was about to die and had the loving God brought
to her.''

\protect\hypertarget{23_NOTES.xhtmlux5cux23id_3002}{\protect\hyperlink{13_Chapter_Six__THE_DEPICTION_OF_TH.xhtmlux5cux23id_3001}{*\textsuperscript{7}}}
``Let God do it, He is a man of mature years.''

\protect\hypertarget{23_NOTES.xhtmlux5cux23id_3004}{\protect\hyperlink{13_Chapter_Six__THE_DEPICTION_OF_TH.xhtmlux5cux23id_3003}{†\textsuperscript{8}}}
``and begged him with folded hands, because he was as highly placed as
God.''

\protect\hypertarget{23_NOTES.xhtmlux5cux23id_3006}{\protect\hyperlink{13_Chapter_Six__THE_DEPICTION_OF_TH.xhtmlux5cux23id_3005}{‡\textsuperscript{9}}}
``strong spirit''

\protect\hypertarget{23_NOTES.xhtmlux5cux23id_3008}{\protect\hyperlink{13_Chapter_Six__THE_DEPICTION_OF_TH.xhtmlux5cux23id_3007}{*\textsuperscript{10}}}
``So Much I Enjoy Myself,'' ``If My Face Is Pale,'' ``The Armed Man''

\protect\hypertarget{23_NOTES.xhtmlux5cux23id_3010}{\protect\hyperlink{13_Chapter_Six__THE_DEPICTION_OF_TH.xhtmlux5cux23id_3009}{†\textsuperscript{11}}}
Then to the sound of the trumpet /God shall open his general and grand
accounting office.

\protect\hypertarget{23_NOTES.xhtmlux5cux23id_3012}{\protect\hyperlink{13_Chapter_Six__THE_DEPICTION_OF_TH.xhtmlux5cux23id_3011}{‡\textsuperscript{12}}}
Hear ye, Hear ye, the honor and the glory /And the great indulgence
conferred by arms.

\protect\hypertarget{23_NOTES.xhtmlux5cux23id_3014}{\protect\hyperlink{13_Chapter_Six__THE_DEPICTION_OF_TH.xhtmlux5cux23id_3013}{*\textsuperscript{13}}}
``See here, the image of the Trinity, the Father, the Son and the Holy
Ghost.''

\protect\hypertarget{23_NOTES.xhtmlux5cux23id_3016}{\protect\hyperlink{13_Chapter_Six__THE_DEPICTION_OF_TH.xhtmlux5cux23id_3015}{*\textsuperscript{14}}}
``seins''---bosom; ``dévotion''---submission, piety; ``bénir''---bless,
blessed, pregnant.

\protect\hypertarget{23_NOTES.xhtmlux5cux23id_3018}{\protect\hyperlink{13_Chapter_Six__THE_DEPICTION_OF_TH.xhtmlux5cux23id_3017}{†\textsuperscript{15}}}
``dishonorable parts of the body and filthy and hateful sins.''

\protect\hypertarget{23_NOTES.xhtmlux5cux23id_3020}{\protect\hyperlink{13_Chapter_Six__THE_DEPICTION_OF_TH.xhtmlux5cux23id_3019}{‡\textsuperscript{16}}}
``Kiss Me''

\protect\hypertarget{23_NOTES.xhtmlux5cux23id_3022}{\protect\hyperlink{13_Chapter_Six__THE_DEPICTION_OF_TH.xhtmlux5cux23id_3021}{§\textsuperscript{17}}}
``Red Nose''

\protect\hypertarget{23_NOTES.xhtmlux5cux23id_3024}{\protect\hyperlink{13_Chapter_Six__THE_DEPICTION_OF_TH.xhtmlux5cux23id_3023}{*\textsuperscript{18}}}
``as befits the conqueror of a country, a secular prince.''

\protect\hypertarget{23_NOTES.xhtmlux5cux23id_3026}{\protect\hyperlink{13_Chapter_Six__THE_DEPICTION_OF_TH.xhtmlux5cux23id_3025}{†\textsuperscript{19}}}
Earlier people were/Very pious in church,/On their knees in
humility/Close to the altar, /And meekly uncovering their heads,/But at
present like beasts/Too often they come to the altar/With hood and hat
on their heads.

\protect\hypertarget{23_NOTES.xhtmlux5cux23id_3028}{\protect\hyperlink{13_Chapter_Six__THE_DEPICTION_OF_TH.xhtmlux5cux23id_3027}{*\textsuperscript{20}}}
``in great and high solemnity and reverence,''

\protect\hypertarget{23_NOTES.xhtmlux5cux23id_3030}{\protect\hyperlink{13_Chapter_Six__THE_DEPICTION_OF_TH.xhtmlux5cux23id_3029}{†\textsuperscript{21}}}
``a crowd of rascals and toughs''

\protect\hypertarget{23_NOTES.xhtmlux5cux23id_3032}{\protect\hyperlink{13_Chapter_Six__THE_DEPICTION_OF_TH.xhtmlux5cux23id_3031}{*\textsuperscript{22}}}
If I often go to church/it is to see the fair one/fresh as a new rose.

\protect\hypertarget{23_NOTES.xhtmlux5cux23id_3034}{\protect\hyperlink{13_Chapter_Six__THE_DEPICTION_OF_TH.xhtmlux5cux23id_3033}{†\textsuperscript{23}}}
``and there is here a good example why one should not go on pilgrimage
for the sake of silly, worldly lusts.''

\protect\hypertarget{23_NOTES.xhtmlux5cux23id_3036}{\protect\hyperlink{13_Chapter_Six__THE_DEPICTION_OF_TH.xhtmlux5cux23id_3035}{*\textsuperscript{24}}}
There is none so mean but says,/I deny God and His mother.

\protect\hypertarget{23_NOTES.xhtmlux5cux23id_3038}{\protect\hyperlink{13_Chapter_Six__THE_DEPICTION_OF_TH.xhtmlux5cux23id_3037}{†\textsuperscript{25}}}
``I deny God''

\protect\hypertarget{23_NOTES.xhtmlux5cux23id_3040}{\protect\hyperlink{13_Chapter_Six__THE_DEPICTION_OF_TH.xhtmlux5cux23id_3039}{*\textsuperscript{26}}}
``I deny boots''

\protect\hypertarget{23_NOTES.xhtmlux5cux23id_3042}{\protect\hyperlink{13_Chapter_Six__THE_DEPICTION_OF_TH.xhtmlux5cux23id_3041}{†\textsuperscript{27}}}
``My good dog whom God pardons.''

\protect\hypertarget{23_NOTES.xhtmlux5cux23id_3044}{\protect\hyperlink{13_Chapter_Six__THE_DEPICTION_OF_TH.xhtmlux5cux23id_3043}{‡\textsuperscript{28}}}
``straight to the paradise of dogs.''

\protect\hypertarget{23_NOTES.xhtmlux5cux23id_3046}{\protect\hyperlink{13_Chapter_Six__THE_DEPICTION_OF_TH.xhtmlux5cux23id_3045}{§\textsuperscript{29}}}
``Heretic.''

\protect\hypertarget{23_NOTES.xhtmlux5cux23id_3048}{\protect\hyperlink{13_Chapter_Six__THE_DEPICTION_OF_TH.xhtmlux5cux23id_3047}{*\textsuperscript{30}}}
``the young angel makes an old devil''

\protect\hypertarget{23_NOTES.xhtmlux5cux23id_3050}{\protect\hyperlink{13_Chapter_Six__THE_DEPICTION_OF_TH.xhtmlux5cux23id_3049}{*\textsuperscript{31}}}
``I have attended to my spiritual concerns and, in my conscience, I
believe I have greatly angered God, having for a long time already erred
against the faith, and I cannot believe a word about the Trinity, nor
that the Son of God has humbled Himself to such an extent to come down
from heaven into the carnal body of a woman; and I believe and say that
when we die there is no such thing as a soul. .~.~. I have held this
opinion ever since I became self-conscious, and I shall hold it to the
end.''

\protect\hypertarget{23_NOTES.xhtmlux5cux23id_3052}{\protect\hyperlink{13_Chapter_Six__THE_DEPICTION_OF_TH.xhtmlux5cux23id_3051}{*\textsuperscript{32}}}
``as imitations and small reflections of God.''

\protect\hypertarget{23_NOTES.xhtmlux5cux23id_3054}{\protect\hyperlink{13_Chapter_Six__THE_DEPICTION_OF_TH.xhtmlux5cux23id_3053}{†\textsuperscript{33}}}
``You shall neither adore them or serve them.''

\protect\hypertarget{23_NOTES.xhtmlux5cux23id_3056}{\protect\hyperlink{13_Chapter_Six__THE_DEPICTION_OF_TH.xhtmlux5cux23id_3055}{*\textsuperscript{34}}}
Woman I am, poor and old,/Who knows nothing; I never could read;/In the
church where I am a parishioner,/I see paradise painted with harps and
lutes,/ And a hell where the damned are boiled:/The one frightens me,
the other brings joy and mirth.

\protect\hypertarget{23_NOTES.xhtmlux5cux23id_3058}{\protect\hyperlink{13_Chapter_Six__THE_DEPICTION_OF_TH.xhtmlux5cux23id_3057}{*\textsuperscript{35}}}
``very often and very loudly.''

\protect\hypertarget{23_NOTES.xhtmlux5cux23id_3060}{\protect\hyperlink{13_Chapter_Six__THE_DEPICTION_OF_TH.xhtmlux5cux23id_3059}{*\textsuperscript{36}}}
You who serve a wife and children /Always keep Joseph in mind;/He served
a woman constantly, gloomily and mournfully,/And he guarded Jesus Christ
in his infancy;/He went on foot with his bundle on his staff;/In many
places he is so pictured,/Next to a mule, for their fair pleasure,/And
so he never had any amusement in this world.

\protect\hypertarget{23_NOTES.xhtmlux5cux23id_3062}{\protect\hyperlink{13_Chapter_Six__THE_DEPICTION_OF_TH.xhtmlux5cux23id_3061}{*\textsuperscript{37}}}
What poverty Joseph suffered/And hardship/And misery,/When God was
born!/Many a time he has carried him/And lifted him up out of
kindness./Together with his mother/On his mule/He led them/Into Egypt.
/I saw him/Painted thus./The good man is painted/Quite exhausted/and
dressed/In a frock and a striped garment;/Leaning on his stick/Old,
spent,/And broken./No earthly joy had he,/But of him/Goes the cry/It is
Joseph the fool.

\protect\hypertarget{23_NOTES.xhtmlux5cux23id_3166}{\protect\hyperlink{13_Chapter_Six__THE_DEPICTION_OF_TH.xhtmlux5cux23id_3165}{*\textsuperscript{38}}}
``God wished that she should marry that saintly man Joseph, who was old
and upright, for God wished to be born in wedlock, to comply with the
current legal requirements, \emph{to avoid gossip}.''

\protect\hypertarget{23_NOTES.xhtmlux5cux23id_3168}{\protect\hyperlink{13_Chapter_Six__THE_DEPICTION_OF_TH.xhtmlux5cux23id_3167}{†\textsuperscript{39}}}
``If it pleases you, I shall marry and shall have a large bevy of
children and relations.''

\protect\hypertarget{23_NOTES.xhtmlux5cux23id_3170}{\protect\hyperlink{13_Chapter_Six__THE_DEPICTION_OF_TH.xhtmlux5cux23id_3169}{‡\textsuperscript{40}}}
``I am black, but comely'' {[}Song of Songs 1:5{]}.

\protect\hypertarget{23_NOTES.xhtmlux5cux23id_3172}{\protect\hyperlink{13_Chapter_Six__THE_DEPICTION_OF_TH.xhtmlux5cux23id_3171}{§\textsuperscript{41}}}
``Even though this maiden is black, nonetheless, she is graceful and has
a beautiful body and limbs and is well suited to bear many children.''

\protect\hypertarget{23_NOTES.xhtmlux5cux23id_3174}{\protect\hyperlink{13_Chapter_Six__THE_DEPICTION_OF_TH.xhtmlux5cux23id_3173}{**\textsuperscript{42}}}
``My beloved son has said to me that she is black and brown. Certainly,
I desire that my son's bride should be young, courteous, pretty,
graceful, and beautiful, and should have beautiful limbs.''

\protect\hypertarget{23_NOTES.xhtmlux5cux23id_3176}{\protect\hyperlink{13_Chapter_Six__THE_DEPICTION_OF_TH.xhtmlux5cux23id_3175}{††\textsuperscript{43}}}
Take her, for she is pleasing,/fit to love her sweet bridegroom;/Now
take plenty of our possessions,/And give them to her in abundance.

\protect\hypertarget{23_NOTES.xhtmlux5cux23id_3064}{\protect\hyperlink{13_Chapter_Six__THE_DEPICTION_OF_TH.xhtmlux5cux23id_3063}{*\textsuperscript{44}}}
There are five saints in the genealogy,/And five female saints to whom
God has granted/Benignantly at the end of their lives,/That who ever
invokes their help with all his heart/In all dangers, that God will
hear/Their intercedence in all disorders whatever./He is wise who serves
these five,/George, Denis, Christopher, Giles, and Blaise.

\protect\hypertarget{23_NOTES.xhtmlux5cux23id_3066}{\protect\hyperlink{13_Chapter_Six__THE_DEPICTION_OF_TH.xhtmlux5cux23id_3065}{*\textsuperscript{45}}}
``O God, who hath distinguished Thy chosen saints, George, etc., etc.,
with special privileges before all others, that all those who in their
need invoke their help shall obtain the salutary fulfillment of their
prayer, according to the promise of Thy grace.''

\protect\hypertarget{23_NOTES.xhtmlux5cux23id_3068}{\protect\hyperlink{13_Chapter_Six__THE_DEPICTION_OF_TH.xhtmlux5cux23id_3067}{*\textsuperscript{46}}}
``May Saint Anthony burn me''

\protect\hypertarget{23_NOTES.xhtmlux5cux23id_3070}{\protect\hyperlink{13_Chapter_Six__THE_DEPICTION_OF_TH.xhtmlux5cux23id_3069}{†\textsuperscript{47}}}
``Saint Anthony burn the brothel, Saint Anthony burn that horse!''

\protect\hypertarget{23_NOTES.xhtmlux5cux23id_3072}{\protect\hyperlink{13_Chapter_Six__THE_DEPICTION_OF_TH.xhtmlux5cux23id_3071}{‡\textsuperscript{48}}}
``Saint Anthony sells me his evil all too dear, /He stokes the fire in
my body.''

\protect\hypertarget{23_NOTES.xhtmlux5cux23id_3074}{\protect\hyperlink{13_Chapter_Six__THE_DEPICTION_OF_TH.xhtmlux5cux23id_3073}{§\textsuperscript{49}}}
``Saint Maur will not make you tremble.''

\protect\hypertarget{23_NOTES.xhtmlux5cux23id_3076}{\protect\hyperlink{13_Chapter_Six__THE_DEPICTION_OF_TH.xhtmlux5cux23id_3075}{*\textsuperscript{50}}}
Do not make gods of silver,/Of gold, of wood, of stone, or of bronze,/
That lead the people to idolatry .~.~. /Because the work has a pleasant
shape;/ Their coloring of which I complain,/The beauty of shining
gold,/Make many ignorant people believe/That these are God for
certain,/And they serve by foolish thoughts/Such images as stand
about/In churches where they place too many of them;/That is very ill
done; in short,/Let us not adore such counterfeits./ .~.~. / Prince, let
us believe in one God/And adore him to perfection /In the fields,
everywhere, for this is right,/No false gods, of iron or stone,/Stones
which have no understanding:/Let us not adore such counterfeits.

\protect\hypertarget{23_NOTES.xhtmlux5cux23id_1270}{\protect\hyperlink{13_Chapter_Six__THE_DEPICTION_OF_TH.xhtmlux5cux23id_1269}{1}}.
J. Burckhardt, Weltgeschichtliche Betrachtungen, 1905, p. 97, 147.

\protect\hypertarget{23_NOTES.xhtmlux5cux23id_1268}{\protect\hyperlink{13_Chapter_Six__THE_DEPICTION_OF_TH.xhtmlux5cux23id_1267}{2}}.
{[}Trans.{]} \emph{Spanning}. See chap. 1, n. 1

\protect\hypertarget{23_NOTES.xhtmlux5cux23id_1266}{\protect\hyperlink{13_Chapter_Six__THE_DEPICTION_OF_TH.xhtmlux5cux23id_1265}{3}}.
{[}Trans.{]} \emph{This-worldliness in other-worldly terms}: That is, to
envision the after-life or the divine merely as an exaggerated version
of this life; a habit of thought which demeans the transcendent by
lowering it to the material. For a later, fuller use of this
terminology, see Arthur O. Lovejoy, \emph{The Great Chain of Being}.
Boston: Harvard University Press, 1964.

\protect\hypertarget{23_NOTES.xhtmlux5cux23id_1264}{\protect\hyperlink{13_Chapter_Six__THE_DEPICTION_OF_TH.xhtmlux5cux23id_1263}{4}}.
{[}Trans.{]} \emph{Suso}: 1300--1366. A follower of the great German
mystic Meister Eckhart. He was an accomplished ascetic as well as a
renowned preacher.

\protect\hypertarget{23_NOTES.xhtmlux5cux23id_1262}{\protect\hyperlink{13_Chapter_Six__THE_DEPICTION_OF_TH.xhtmlux5cux23id_1261}{5}}.
Heinrich Seuse, Leben, ed. Bihlmeyer, Deutsche Schriften, 1907, pp. 24,
25.

\protect\hypertarget{23_NOTES.xhtmlux5cux23id_1260}{\protect\hyperlink{13_Chapter_Six__THE_DEPICTION_OF_TH.xhtmlux5cux23id_1259}{6}}.
{[}Trans.{]} \emph{bonte}: German \emph{buntes}, colorful. An adjective
very often used by Huizinga in \emph{Autumn}. The translator's
temptation to replace it with synonyms should be resisted since the
``colorful'' aspect of such as the images of saints is precisely why
they are ``this-worldly.'' Huizinga would have remembered
Mephistopheles' temptation of the student in \emph{Faust}: ``Grey, my
friend, is all of theory, and green is life's golden tree'' (I, 2,
2039).

\protect\hypertarget{23_NOTES.xhtmlux5cux23id_1258}{\protect\hyperlink{13_Chapter_Six__THE_DEPICTION_OF_TH.xhtmlux5cux23id_1257}{7}}.
{[}Trans.{]} \emph{Gerson}: 1363--1429. A student of Pierre d'Ailly and
succeeded him as Chancellor of the University of Paris. An enemy of
scholastic speculation and a nominalist, he was also a delicate mystic.
Of very humble origin, he rose to his position through strength of mind.
He was very prominent in the attempts to heal the Great Schism. One
feels that Huizinga admired him greatly.

\protect\hypertarget{23_NOTES.xhtmlux5cux23id_1256}{\protect\hyperlink{13_Chapter_Six__THE_DEPICTION_OF_TH.xhtmlux5cux23id_1255}{8}}.
Gerson, Opera, III, p. 309.

\protect\hypertarget{23_NOTES.xhtmlux5cux23id_1254}{\protect\hyperlink{13_Chapter_Six__THE_DEPICTION_OF_TH.xhtmlux5cux23id_1253}{9}}.
Nic. de Clémanges, De novis festivitatibus non instituendis, Opera, ed.
Lydius Lugd. Bat. 1613, pp. 151, 159.

\protect\hypertarget{23_NOTES.xhtmlux5cux23id_1252}{\protect\hyperlink{13_Chapter_Six__THE_DEPICTION_OF_TH.xhtmlux5cux23id_1251}{10}}.
In Gerson, Opera, II, p. 911.

\protect\hypertarget{23_NOTES.xhtmlux5cux23id_1250}{\protect\hyperlink{13_Chapter_Six__THE_DEPICTION_OF_TH.xhtmlux5cux23id_1249}{11}}.
Acta sanctorum, Apr. t. III, p. 149.

\protect\hypertarget{23_NOTES.xhtmlux5cux23id_1248}{\protect\hyperlink{13_Chapter_Six__THE_DEPICTION_OF_TH.xhtmlux5cux23id_1247}{12}}.
Ac aliis vere pauperibus et miserabilibus indigentibus, quibus convenit
jus et verus titulus mendicandi.

\protect\hypertarget{23_NOTES.xhtmlux5cux23id_1246}{\protect\hyperlink{13_Chapter_Six__THE_DEPICTION_OF_TH.xhtmlux5cux23id_1245}{13}}.
Qui ecclesiam suis mendaciis maculant et earn irrisibilem reddunt.

\protect\hypertarget{23_NOTES.xhtmlux5cux23id_1244}{\protect\hyperlink{13_Chapter_Six__THE_DEPICTION_OF_TH.xhtmlux5cux23id_1243}{14}}.
Alanus Redivivus, ed. J. Coppenstein, 1642, p. 77.

\protect\hypertarget{23_NOTES.xhtmlux5cux23id_1242}{\protect\hyperlink{13_Chapter_Six__THE_DEPICTION_OF_TH.xhtmlux5cux23id_1241}{15}}.
Commines, I, p. 310; Chastellain, V, p. 27; Le Jouvencel, I, p. 82; Jean
Lud, in Deutsche Geschichtsblätter, XV, p. 248; Journal d'un bourgeois,
p. 384; Paston Letters, II, p. 18; J, H. Ramsay, Lancaster and York, II,
p. 275; Play of Sir John Oldcastle, II, p. 2 and others.

\protect\hypertarget{23_NOTES.xhtmlux5cux23id_1240}{\protect\hyperlink{13_Chapter_Six__THE_DEPICTION_OF_TH.xhtmlux5cux23id_1239}{16}}.
Contra superstitionem praesertim Innocentum, Opera, I, p. 203.

\protect\hypertarget{23_NOTES.xhtmlux5cux23id_1238}{\protect\hyperlink{13_Chapter_Six__THE_DEPICTION_OF_TH.xhtmlux5cux23id_1237}{17}}.
Gerson, Quaedam argumentatio adversus eos qui publice volunt dogmatizare
etc. Opera, II, pp. 521--522.

\protect\hypertarget{23_NOTES.xhtmlux5cux23id_1236}{\protect\hyperlink{13_Chapter_Six__THE_DEPICTION_OF_TH.xhtmlux5cux23id_1235}{18}}.
Johannis de Varennis Responsiones etc., Gerson, I, p. 909.

\protect\hypertarget{23_NOTES.xhtmlux5cux23id_1234}{\protect\hyperlink{13_Chapter_Six__THE_DEPICTION_OF_TH.xhtmlux5cux23id_1233}{19}}.
Journal d'un bourgeois, p. 259. For ``une hucque vermeille par
dessoubz,'' which is impossible, read ``par dessus.''

\protect\hypertarget{23_NOTES.xhtmlux5cux23id_1232}{\protect\hyperlink{13_Chapter_Six__THE_DEPICTION_OF_TH.xhtmlux5cux23id_1231}{20}}.
Gerson, Considérations sur Saint Joseph, III, pp. 842--68, Josephina,
IV, \protect\hypertarget{23_NOTES.xhtmlux5cux23page_417}{}{}p. 753;
Sermo de natalitate beatae Mariae Virginis, III, p. 1351; Further IV, p.
729, 731, 732, 735, 736.

\protect\hypertarget{23_NOTES.xhtmlux5cux23id_1230}{\protect\hyperlink{13_Chapter_Six__THE_DEPICTION_OF_TH.xhtmlux5cux23id_1229}{21}}.
Gerson, De distinctione verarum visionum a falsis, Opera, I, p. 50.

\protect\hypertarget{23_NOTES.xhtmlux5cux23id_1228}{\protect\hyperlink{13_Chapter_Six__THE_DEPICTION_OF_TH.xhtmlux5cux23id_1227}{22}}.
C. Schmidt, Der Prediger Olivier Maillard, Zeitschrift f. hist.
Theologie, 1856, p. 501.

\protect\hypertarget{23_NOTES.xhtmlux5cux23id_1226}{\protect\hyperlink{13_Chapter_Six__THE_DEPICTION_OF_TH.xhtmlux5cux23id_1225}{23}}.
See Thuasne, Rob. Gaguini, Ep. Or., I, pp. 72ff.

\protect\hypertarget{23_NOTES.xhtmlux5cux23id_1224}{\protect\hyperlink{13_Chapter_Six__THE_DEPICTION_OF_TH.xhtmlux5cux23id_1223}{24}}.
Les cent nouvelles nouvelles, ed. Wright, II, pp. 75ff., I2ff.

\protect\hypertarget{23_NOTES.xhtmlux5cux23id_1222}{\protect\hyperlink{13_Chapter_Six__THE_DEPICTION_OF_TH.xhtmlux5cux23id_1221}{25}}.
Le livre du chevalier de la Tour Landry, ed. de Montaiglon, p. 56.

\protect\hypertarget{23_NOTES.xhtmlux5cux23id_1220}{\protect\hyperlink{13_Chapter_Six__THE_DEPICTION_OF_TH.xhtmlux5cux23id_1219}{26}}.
Loc. cit., p. 257: ``Se elles ouyssent sonner la messe ou a veoir
Dieu.''

\protect\hypertarget{23_NOTES.xhtmlux5cux23id_1218}{\protect\hyperlink{13_Chapter_Six__THE_DEPICTION_OF_TH.xhtmlux5cux23id_1217}{27}}.
Leroux de Lincy, Le livre des Proverbes français, Paris, 1859, 2 vols.,
I, p. 21.

\protect\hypertarget{23_NOTES.xhtmlux5cux23id_1216}{\protect\hyperlink{13_Chapter_Six__THE_DEPICTION_OF_TH.xhtmlux5cux23id_1215}{28}}.
Froissart, ed. Luce, V, p. 24.

\protect\hypertarget{23_NOTES.xhtmlux5cux23id_1214}{\protect\hyperlink{13_Chapter_Six__THE_DEPICTION_OF_TH.xhtmlux5cux23id_1213}{29}}.
``Cum juramento asseruit non credere in Deum dicti episcopi,'' Rel. de
S. Denis, I, p. 102.

\protect\hypertarget{23_NOTES.xhtmlux5cux23id_1212}{\protect\hyperlink{13_Chapter_Six__THE_DEPICTION_OF_TH.xhtmlux5cux23id_1211}{30}}.
{[}Trans.{]} \emph{Hansje in den Kelder}. Hans in the cellar. A rare
antique drinking dish in which through a clever mechanism a little
figure pops up when the dish is filled. The dish was used to toast an
expectant mother. We are indebted to Willemien Rathonyi-Reusz of the
Canadian Association for the Advancement of Netherlandic Studies for
this information.

\protect\hypertarget{23_NOTES.xhtmlux5cux23id_1210}{\protect\hyperlink{13_Chapter_Six__THE_DEPICTION_OF_TH.xhtmlux5cux23id_1209}{31}}.
Laborde, II, p. 264, no. 4238, Inventory of 1420; ib. II, p. 10, no. 77.
Inventory of Charles the Bold, who well might be the source of this
specimen.

\protect\hypertarget{23_NOTES.xhtmlux5cux23id_1208}{\protect\hyperlink{13_Chapter_Six__THE_DEPICTION_OF_TH.xhtmlux5cux23id_1207}{32}}.
Gerson, Opera, III, p. 947.

\protect\hypertarget{23_NOTES.xhtmlux5cux23id_1206}{\protect\hyperlink{13_Chapter_Six__THE_DEPICTION_OF_TH.xhtmlux5cux23id_1205}{33}}.
Journal d'un bourgeois, p. 3662.

\protect\hypertarget{23_NOTES.xhtmlux5cux23id_1204}{\protect\hyperlink{13_Chapter_Six__THE_DEPICTION_OF_TH.xhtmlux5cux23id_1203}{34}}.
A Dutch letter of indulgence from the fourteenth century, ed. J. Verdam,
Ned. Archief voor Kerkgesch. 1900, pp. 117--22.

\protect\hypertarget{23_NOTES.xhtmlux5cux23id_1202}{\protect\hyperlink{13_Chapter_Six__THE_DEPICTION_OF_TH.xhtmlux5cux23id_1201}{35}}.
A. Eekhof, De questierders van den aflaat in de Noordelijke Nederl., `s
Gravenhagem 1909, p. 12.

\protect\hypertarget{23_NOTES.xhtmlux5cux23id_1200}{\protect\hyperlink{13_Chapter_Six__THE_DEPICTION_OF_TH.xhtmlux5cux23id_1199}{36}}.
Chastellain, I, pp. 187--89; entry of Henry V and Philip of Burgundy
into Paris 1420; II, p. 16: Entry of the latter into Ghent 1430.

\protect\hypertarget{23_NOTES.xhtmlux5cux23id_1198}{\protect\hyperlink{13_Chapter_Six__THE_DEPICTION_OF_TH.xhtmlux5cux23id_1197}{37}}.
Doutrepont, p. 379.

\protect\hypertarget{23_NOTES.xhtmlux5cux23id_1196}{\protect\hyperlink{13_Chapter_Six__THE_DEPICTION_OF_TH.xhtmlux5cux23id_1195}{38}}.
Deschamps, III, p. 89, no. 357; le roi René, Traicté de la forme de
devise d'un tournoy, Oeuvres, II, p. 9.

\protect\hypertarget{23_NOTES.xhtmlux5cux23id_1194}{\protect\hyperlink{13_Chapter_Six__THE_DEPICTION_OF_TH.xhtmlux5cux23id_1193}{39}}.
Olivier de la Marche, II, p. 202.

\protect\hypertarget{23_NOTES.xhtmlux5cux23id_1192}{\protect\hyperlink{13_Chapter_Six__THE_DEPICTION_OF_TH.xhtmlux5cux23id_1191}{40}}.
Monstrelet, I, p. 285, cf. 306.

\protect\hypertarget{23_NOTES.xhtmlux5cux23id_1190}{\protect\hyperlink{13_Chapter_Six__THE_DEPICTION_OF_TH.xhtmlux5cux23id_1189}{41}}.
Liber de virtutibus Philippe ducis Burgundiae, pp. 13, 16 (Chron. rel. à
l'hist de Belgique sous la dom. des ducs de Bourg., II).

\protect\hypertarget{23_NOTES.xhtmlux5cux23id_1188}{\protect\hyperlink{13_Chapter_Six__THE_DEPICTION_OF_TH.xhtmlux5cux23id_1187}{42}}.
Molinet, II pp. 84--94, III p. 98; Faictz et Dictz, f. 47, see I, p.
240, and also Chastellain, III pp. 209, 260, IV p. 48, V p. 301, VII p.
1ff.

\protect\hypertarget{23_NOTES.xhtmlux5cux23id_1186}{\protect\hyperlink{13_Chapter_Six__THE_DEPICTION_OF_TH.xhtmlux5cux23id_1185}{43}}.
Molinet, III, p. 109.

\protect\hypertarget{23_NOTES.xhtmlux5cux23id_1184}{\protect\hyperlink{13_Chapter_Six__THE_DEPICTION_OF_TH.xhtmlux5cux23id_1183}{44}}.
Gerson, Oratio ad regem Franciae, Opera, IV, p. 662.

\protect\hypertarget{23_NOTES.xhtmlux5cux23id_1182}{\protect\hyperlink{13_Chapter_Six__THE_DEPICTION_OF_TH.xhtmlux5cux23id_1181}{45}}.
Quinze joyes de Mariage, p. XIII.

\protect\hypertarget{23_NOTES.xhtmlux5cux23id_1180}{\protect\hyperlink{13_Chapter_Six__THE_DEPICTION_OF_TH.xhtmlux5cux23id_1179}{46}}.
Gerson, Opera, III, p. 299.

\protect\hypertarget{23_NOTES.xhtmlux5cux23id_1178}{\protect\hyperlink{13_Chapter_Six__THE_DEPICTION_OF_TH.xhtmlux5cux23id_1177}{47}}.
{[}Trans.{]} \emph{Agnes Sorel} 1422--50. The first ``official''
mistress of a King of France, she served Charles VII. He gave her the
castle of Beauté, where she died after childbirth and profound
repentance.

\protect\hypertarget{23_NOTES.xhtmlux5cux23id_1176}{\protect\hyperlink{13_Chapter_Six__THE_DEPICTION_OF_TH.xhtmlux5cux23id_1175}{48}}.
Friedländer, Jahrb. d. K. Preuss. Kunstsammlungen, XVII, 1896, p. 206.

\protect\hypertarget{23_NOTES.xhtmlux5cux23id_1174}{\protect\hyperlink{13_Chapter_Six__THE_DEPICTION_OF_TH.xhtmlux5cux23id_1173}{49}}.
Wetzer und Welte, Kirchenlexicon, see Musik, col. 2040; see Erasmus,
Christiani Matrimonii Institutio, Opera (ed. Lugd. Bat.), V, col. 718c:
``Nunc\protect\hypertarget{23_NOTES.xhtmlux5cux23page_418}{}{}sonis
neqissimis aptantur verba sacra, nihilo magis decore, quam si thaidis
ornatum addas Catoni. Interdum nec verba silentur impudica cantorum
licentia.'' {[}Trans.: ``Nowadays the most frivolous tunes are given
holy words, which is no better than if one put the jewelry of Thais on
Cato. And given the whore-like shamelessness of the singers, the
(secular) words are not even held back.''{]}

\protect\hypertarget{23_NOTES.xhtmlux5cux23id_1172}{\protect\hyperlink{13_Chapter_Six__THE_DEPICTION_OF_TH.xhtmlux5cux23id_1171}{50}}.
Chastellain, III, p. 155.

\protect\hypertarget{23_NOTES.xhtmlux5cux23id_1170}{\protect\hyperlink{13_Chapter_Six__THE_DEPICTION_OF_TH.xhtmlux5cux23id_1169}{51}}.
H. van den Velden, Rod. Agricola, een nederlandsch Humanist der 15 eeuw,
I, dl., Leiden 1911, p. 44.

\protect\hypertarget{23_NOTES.xhtmlux5cux23id_1168}{\protect\hyperlink{13_Chapter_Six__THE_DEPICTION_OF_TH.xhtmlux5cux23id_1167}{52}}.
Deschamps, X, no. 33, p. lxi, in the next to the last line we find
``l'ostel,'' which, of course, makes no sense.

\protect\hypertarget{23_NOTES.xhtmlux5cux23id_1166}{\protect\hyperlink{13_Chapter_Six__THE_DEPICTION_OF_TH.xhtmlux5cux23id_1165}{53}}.
Nic. de Clémanges, De novis celebritatibus non instituendis, Opera, ed.
Lydius, 1613, p. 143.

\protect\hypertarget{23_NOTES.xhtmlux5cux23id_1164}{\protect\hyperlink{13_Chapter_Six__THE_DEPICTION_OF_TH.xhtmlux5cux23id_1163}{54}}.
Le livre du chevalier de la Tour Landry, pp. \emph{66}, 70.

\protect\hypertarget{23_NOTES.xhtmlux5cux23id_1162}{\protect\hyperlink{13_Chapter_Six__THE_DEPICTION_OF_TH.xhtmlux5cux23id_1161}{55}}.
Gerson, sermo de nativitate Domini, Opera, III, pp. 946, 947.

\protect\hypertarget{23_NOTES.xhtmlux5cux23id_1160}{\protect\hyperlink{13_Chapter_Six__THE_DEPICTION_OF_TH.xhtmlux5cux23id_1159}{56}}.
Nicolas de Clémanges, De novis celebritatibus non instituendis, p. 147.

\protect\hypertarget{23_NOTES.xhtmlux5cux23id_1158}{\protect\hyperlink{13_Chapter_Six__THE_DEPICTION_OF_TH.xhtmlux5cux23id_1157}{57}}.
O. Winckelmann, Zur Kulturgeschichte des Strassburger Munsters,
Zeitschr. f. d. Geschichte des Oberrheins, N. F. XXII, 2.

\protect\hypertarget{23_NOTES.xhtmlux5cux23id_1156}{\protect\hyperlink{13_Chapter_Six__THE_DEPICTION_OF_TH.xhtmlux5cux23id_1155}{58}}.
Dionysius Cartusianus, De modo agendi processiones etc., Opera, XXXVI,
pp. I98ff.

\protect\hypertarget{23_NOTES.xhtmlux5cux23id_1154}{\protect\hyperlink{13_Chapter_Six__THE_DEPICTION_OF_TH.xhtmlux5cux23id_1153}{59}}.
Chastellain, V, pp. 253ff.

\protect\hypertarget{23_NOTES.xhtmlux5cux23id_1152}{\protect\hyperlink{13_Chapter_Six__THE_DEPICTION_OF_TH.xhtmlux5cux23id_1151}{60}}.
See above, p. 48.

\protect\hypertarget{23_NOTES.xhtmlux5cux23id_1150}{\protect\hyperlink{13_Chapter_Six__THE_DEPICTION_OF_TH.xhtmlux5cux23id_1149}{61}}.
Michel Menot, Sermones, f. 144 vso., in Champion, Villon, I, p. 202.

\protect\hypertarget{23_NOTES.xhtmlux5cux23id_1148}{\protect\hyperlink{13_Chapter_Six__THE_DEPICTION_OF_TH.xhtmlux5cux23id_1147}{62}}.
Le livre du chevalier de la Tour Landry, p. 65; Olivier de la Marche,
II, p. 89; L'amant rendu cordelier, p. 25, huitain 68; Rel. de S. Denis,
I, p. 102.

\protect\hypertarget{23_NOTES.xhtmlux5cux23id_1146}{\protect\hyperlink{13_Chapter_Six__THE_DEPICTION_OF_TH.xhtmlux5cux23id_1145}{63}}.
Christine de Pisan, Oeuvres poétiques, I, p. 172, see p. 60, l'epistre
au dieu d'Amours, II, 3; Deschamps V p. 51 no. 871, II p. 185 vs 75; Seè
above, p. 147.

\protect\hypertarget{23_NOTES.xhtmlux5cux23id_1144}{\protect\hyperlink{13_Chapter_Six__THE_DEPICTION_OF_TH.xhtmlux5cux23id_1143}{64}}.
L'amant rendu cordelier, p. 25.

\protect\hypertarget{23_NOTES.xhtmlux5cux23id_1142}{\protect\hyperlink{13_Chapter_Six__THE_DEPICTION_OF_TH.xhtmlux5cux23id_1141}{65}}.
Menot, Sermones, p. 202.

\protect\hypertarget{23_NOTES.xhtmlux5cux23id_1140}{\protect\hyperlink{13_Chapter_Six__THE_DEPICTION_OF_TH.xhtmlux5cux23id_1139}{66}}.
Gerson, Expostulatio .~.~. adversus corruptionem juventutis per lascivas
imagines et alia hujus modi, Opera, III, p. 291; cf. De parvulis
Christum trahendis, ib. p. 281; Contra tentationem blasphemiae, ib. p.
246.

\protect\hypertarget{23_NOTES.xhtmlux5cux23id_1138}{\protect\hyperlink{13_Chapter_Six__THE_DEPICTION_OF_TH.xhtmlux5cux23id_1137}{67}}.
Le livre du chevalier de la Tour Landry, pp. 80, 81; see Machaut, Livre
du voir-dit, pp. 143ff.

\protect\hypertarget{23_NOTES.xhtmlux5cux23id_1136}{\protect\hyperlink{13_Chapter_Six__THE_DEPICTION_OF_TH.xhtmlux5cux23id_1135}{68}}.
Tour Landry, pp. 55, 63, 73, 79.

\protect\hypertarget{23_NOTES.xhtmlux5cux23id_1134}{\protect\hyperlink{13_Chapter_Six__THE_DEPICTION_OF_TH.xhtmlux5cux23id_1133}{69}}.
Nicolas de Clémanges, De novis celebritatibus .~.~. , p. 145.

\protect\hypertarget{23_NOTES.xhtmlux5cux23id_1132}{\protect\hyperlink{13_Chapter_Six__THE_DEPICTION_OF_TH.xhtmlux5cux23id_1131}{70}}.
Quinze joyes de mariage, p. 127; see pp. 19, 29, 124.

\protect\hypertarget{23_NOTES.xhtmlux5cux23id_1130}{\protect\hyperlink{13_Chapter_Six__THE_DEPICTION_OF_TH.xhtmlux5cux23id_1129}{71}}.
Froissart, ed. Luce et Raynaud, XI, pp. 225ff.

\protect\hypertarget{23_NOTES.xhtmlux5cux23id_1128}{\protect\hyperlink{13_Chapter_Six__THE_DEPICTION_OF_TH.xhtmlux5cux23id_1127}{72}}.
Chron. Montis S. Agnetis, p. 341; J. C. Pool, Frederik v. Heilo en aijne
schriften, Amsterdam 1866, p. 126; see Hendrik Mande in W. Moll, Joh.
Brugman en het godsdienstig leven onzer vaderen in de
15\textsuperscript{e} eeuw, 1854, 2 vols., I, p. 264.

\protect\hypertarget{23_NOTES.xhtmlux5cux23id_1126}{\protect\hyperlink{13_Chapter_Six__THE_DEPICTION_OF_TH.xhtmlux5cux23id_1125}{73}}.
Gerson, Centilogium de impulsibus, Opera, III, p. 154.

\protect\hypertarget{23_NOTES.xhtmlux5cux23id_1124}{\protect\hyperlink{13_Chapter_Six__THE_DEPICTION_OF_TH.xhtmlux5cux23id_1123}{74}}.
Deschamps, IV, p. 322 no. 807; see I, p. 272 no. 146: ``si n'y a Si
meschant qui encor ne die Je regni Dieu .~.~. ''

\protect\hypertarget{23_NOTES.xhtmlux5cux23id_1122}{\protect\hyperlink{13_Chapter_Six__THE_DEPICTION_OF_TH.xhtmlux5cux23id_1121}{75}}.
Gerson, Adversus lascivas imagines, Opera, III, p. 292; Sermo de
nativatate Domini, III, p. 946.

\protect\hypertarget{23_NOTES.xhtmlux5cux23page_419}{\protect\hyperlink{13_Chapter_Six__THE_DEPICTION_OF_TH.xhtmlux5cux23id_1120}{76}}.
Deschamps, I, pp. 271ff. nos. 145, 146, p. 217 no. 105; see II, p. lvi,
and Gerson, III, p. 85.

\protect\hypertarget{23_NOTES.xhtmlux5cux23id_1119}{\protect\hyperlink{13_Chapter_Six__THE_DEPICTION_OF_TH.xhtmlux5cux23id_1118}{77}}.
Gerson, Considérations sur le peché de blasphème, Opera, III, p. 889.

\protect\hypertarget{23_NOTES.xhtmlux5cux23id_1117}{\protect\hyperlink{13_Chapter_Six__THE_DEPICTION_OF_TH.xhtmlux5cux23id_1116}{78}}.
Regulae morales, Opera, III, p. 85.

\protect\hypertarget{23_NOTES.xhtmlux5cux23id_1115}{\protect\hyperlink{13_Chapter_Six__THE_DEPICTION_OF_TH.xhtmlux5cux23id_1114}{79}}.
Ordonnances des rois de France, t. VIII, p. 130; Rel. de S. Denis, II,
p. 533.

\protect\hypertarget{23_NOTES.xhtmlux5cux23id_1113}{\protect\hyperlink{13_Chapter_Six__THE_DEPICTION_OF_TH.xhtmlux5cux23id_1112}{80}}.
P. d'Ailly, De reformatione, cap. 6, de reform. laicorum, in Gerson,
Opera II, p. 914.

\protect\hypertarget{23_NOTES.xhtmlux5cux23id_1111}{\protect\hyperlink{13_Chapter_Six__THE_DEPICTION_OF_TH.xhtmlux5cux23id_1110}{81}}.
Gerson, Contra foedam tentationem blasphemiae, Opera, III, p. 243.

\protect\hypertarget{23_NOTES.xhtmlux5cux23id_1109}{\protect\hyperlink{13_Chapter_Six__THE_DEPICTION_OF_TH.xhtmlux5cux23id_1108}{82}}.
Gerson, Regulae morales, Opera, III, p. 85.

\protect\hypertarget{23_NOTES.xhtmlux5cux23id_1107}{\protect\hyperlink{13_Chapter_Six__THE_DEPICTION_OF_TH.xhtmlux5cux23id_1106}{83}}.
Gerson, Contra foedam tentationem blasphemiae, Opera, III, p. 246: hi
qui audacter contra fidem loquuntur in forma joci etc.

\protect\hypertarget{23_NOTES.xhtmlux5cux23id_1105}{\protect\hyperlink{13_Chapter_Six__THE_DEPICTION_OF_TH.xhtmlux5cux23id_1104}{84}}.
Cent nouvelles nouvelles, II, p. 205.

\protect\hypertarget{23_NOTES.xhtmlux5cux23id_1103}{\protect\hyperlink{13_Chapter_Six__THE_DEPICTION_OF_TH.xhtmlux5cux23id_1102}{85}}.
Gerson, Sermo de S. Nicolao, Opera, III, p. 1577; De parvulis ad
Christum trahendis ib. p. 279. Against this same saying also Dionysius
Cart., Inter Jesum et puerum dialogus, art. 2, Opera, t. XXXVIII, p.
190.

\protect\hypertarget{23_NOTES.xhtmlux5cux23id_1101}{\protect\hyperlink{13_Chapter_Six__THE_DEPICTION_OF_TH.xhtmlux5cux23id_1100}{86}}.
Gerson, De distinctione verarum visionum a falsis, Opera, I, p. 45.

\protect\hypertarget{23_NOTES.xhtmlux5cux23id_1099}{\protect\hyperlink{13_Chapter_Six__THE_DEPICTION_OF_TH.xhtmlux5cux23id_1098}{87}}.
Ibid., p. 58.

\protect\hypertarget{23_NOTES.xhtmlux5cux23id_1097}{\protect\hyperlink{13_Chapter_Six__THE_DEPICTION_OF_TH.xhtmlux5cux23id_1096}{88}}.
Petrus Damiani, Opera, XII, 29; Migne, P. L., 145, p. 283; see for the
twelfth and thirteenth centuries Hauck, Kirchengeschichte Deutschlands,
IV, pp. 81, 898.

\protect\hypertarget{23_NOTES.xhtmlux5cux23id_1095}{\protect\hyperlink{13_Chapter_Six__THE_DEPICTION_OF_TH.xhtmlux5cux23id_1094}{89}}.
Deschamps, VI, p. 109, no. 1167, id., no. 1222; Commines, I, p. 449.

\protect\hypertarget{23_NOTES.xhtmlux5cux23id_1093}{\protect\hyperlink{13_Chapter_Six__THE_DEPICTION_OF_TH.xhtmlux5cux23id_1092}{90}}.
Froissart, ed. Kervyn, XIV, p. 67.

\protect\hypertarget{23_NOTES.xhtmlux5cux23id_1091}{\protect\hyperlink{13_Chapter_Six__THE_DEPICTION_OF_TH.xhtmlux5cux23id_1090}{91}}.
Rel. de S. Denis, I, pp. 102, 104; Jean Juvenal des Ursins, p. 346.

\protect\hypertarget{23_NOTES.xhtmlux5cux23id_1089}{\protect\hyperlink{13_Chapter_Six__THE_DEPICTION_OF_TH.xhtmlux5cux23id_1088}{92}}.
Jacques du Clercq, II, pp. 277, 340; IV, p. 59; see Molinet IV, p. 390,
Rel. de S. Denis, I, p. 643.

\protect\hypertarget{23_NOTES.xhtmlux5cux23id_1087}{\protect\hyperlink{13_Chapter_Six__THE_DEPICTION_OF_TH.xhtmlux5cux23id_1086}{93}}.
Joh. de Monasteriolo, Epistolae, II, p. 1415; see ep. 75, 76, p. 1456 of
Ambr. de Miliis to Gontier Col, in which he complains about Jean de
Montreuil.

\protect\hypertarget{23_NOTES.xhtmlux5cux23id_1085}{\protect\hyperlink{13_Chapter_Six__THE_DEPICTION_OF_TH.xhtmlux5cux23id_1084}{94}}.
Gerson, Sermo III in Sancti Ludovici, Opera, III, p. 1451.

\protect\hypertarget{23_NOTES.xhtmlux5cux23id_1083}{\protect\hyperlink{13_Chapter_Six__THE_DEPICTION_OF_TH.xhtmlux5cux23id_1082}{95}}.
Gerson, Contra impugnantes ordinem carthusiensium, Opera, II, p. 713.

\protect\hypertarget{23_NOTES.xhtmlux5cux23id_1081}{\protect\hyperlink{13_Chapter_Six__THE_DEPICTION_OF_TH.xhtmlux5cux23id_1080}{96}}.
Gerson, De decern praceptis, Opera, I, p. 245.

\protect\hypertarget{23_NOTES.xhtmlux5cux23id_1079}{\protect\hyperlink{13_Chapter_Six__THE_DEPICTION_OF_TH.xhtmlux5cux23id_1078}{97}}.
Gerson, Sermo de nativ. Domini, Opera, III, p. 947.

\protect\hypertarget{23_NOTES.xhtmlux5cux23id_1077}{\protect\hyperlink{13_Chapter_Six__THE_DEPICTION_OF_TH.xhtmlux5cux23id_1076}{98}}.
Nic. de Clémanges, De novis celebr. etc., p. 151.

\protect\hypertarget{23_NOTES.xhtmlux5cux23id_1075}{\protect\hyperlink{13_Chapter_Six__THE_DEPICTION_OF_TH.xhtmlux5cux23id_1074}{99}}.
Villon, Testament, vs. 893ff., ed Longnon, p. 57.

\protect\hypertarget{23_NOTES.xhtmlux5cux23id_1073}{\protect\hyperlink{13_Chapter_Six__THE_DEPICTION_OF_TH.xhtmlux5cux23id_1072}{100}}.
Gerson, Sermo de nativitate Domine, Opera, III, p. 947; Regulae morales,
ib. p. 86; Liber de vita spirituali animae, ib. p. 66.

\protect\hypertarget{23_NOTES.xhtmlux5cux23id_1071}{\protect\hyperlink{13_Chapter_Six__THE_DEPICTION_OF_TH.xhtmlux5cux23id_1070}{101}}.
Hist. translationis corporis sanctissimi ecclesiae doctoris divi Thorn.
de Aq., 1368, auct. fr. Raymundo Hugonis O. P., Acta sanctorum Martii,
I, p. 725.

\protect\hypertarget{23_NOTES.xhtmlux5cux23id_1069}{\protect\hyperlink{13_Chapter_Six__THE_DEPICTION_OF_TH.xhtmlux5cux23id_1068}{102}}.
Report of the papal commissioner Bishop Conrad of Hildesheim and Abbot
Hermann of Georgenthal about the testimony concerning St. Elisabeth of
Marburg in January 1235, given in Historisches Jahrbuch der
Görres-Gesellschaft, XXVIII, p. 887.

\protect\hypertarget{23_NOTES.xhtmlux5cux23id_1067}{\protect\hyperlink{13_Chapter_Six__THE_DEPICTION_OF_TH.xhtmlux5cux23id_1066}{103}}.
Rel. de S. Denis, II, p. 37.

\protect\hypertarget{23_NOTES.xhtmlux5cux23id_1065}{\protect\hyperlink{13_Chapter_Six__THE_DEPICTION_OF_TH.xhtmlux5cux23id_1064}{104}}.
See below p. 198.

\protect\hypertarget{23_NOTES.xhtmlux5cux23id_1063}{\protect\hyperlink{13_Chapter_Six__THE_DEPICTION_OF_TH.xhtmlux5cux23id_1062}{105}}.
Chastellain, III, p. 407; IV, p. 216.

\protect\hypertarget{23_NOTES.xhtmlux5cux23id_1061}{\protect\hyperlink{13_Chapter_Six__THE_DEPICTION_OF_TH.xhtmlux5cux23id_1060}{106}}.
Deschamps, I, p. 277, no. 150.

\protect\hypertarget{23_NOTES.xhtmlux5cux23id_1059}{\protect\hyperlink{13_Chapter_Six__THE_DEPICTION_OF_TH.xhtmlux5cux23id_1058}{107}}.
Deschamps, II, p. 348, no. 314.

\protect\hypertarget{23_NOTES.xhtmlux5cux23page_420}{\protect\hyperlink{13_Chapter_Six__THE_DEPICTION_OF_TH.xhtmlux5cux23id_1057}{108}}.
From Johann Ecks's Pfarrbuch for U. L. Frau in Ingolstadt, in Archiv fur
Kulturgesch., VIII, p. 103.

\protect\hypertarget{23_NOTES.xhtmlux5cux23id_1056}{\protect\hyperlink{13_Chapter_Six__THE_DEPICTION_OF_TH.xhtmlux5cux23id_1055}{109}}.
Joseph Seitz, Die Verehrung des heil. Joseph in ihrer gesch.
Entwicklung, etc., Freiburg, Herder, 1908.

\protect\hypertarget{23_NOTES.xhtmlux5cux23id_1054}{\protect\hyperlink{13_Chapter_Six__THE_DEPICTION_OF_TH.xhtmlux5cux23id_1053}{110}}.
Le livre du chevalier de la Tour Landry, p. 212.

\protect\hypertarget{23_NOTES.xhtmlux5cux23id_1052}{\protect\hyperlink{13_Chapter_Six__THE_DEPICTION_OF_TH.xhtmlux5cux23id_1051}{111}}.
B. Nat. Ms. fr. 1875, in Ch. Oulmont, Le verger, le temple et la
cellule, essai sur la sensualité dans les oeuvres de mystique
religieuse, Paris 1912, pp. 284ff.

\protect\hypertarget{23_NOTES.xhtmlux5cux23id_1050}{\protect\hyperlink{13_Chapter_Six__THE_DEPICTION_OF_TH.xhtmlux5cux23id_1049}{112}}.
See the passages about the images of saints in E. Mâle, L'art religieux
à la fin du moyen-âge, chap. IV.

\protect\hypertarget{23_NOTES.xhtmlux5cux23id_1048}{\protect\hyperlink{13_Chapter_Six__THE_DEPICTION_OF_TH.xhtmlux5cux23id_1047}{113}}.
Deschamps, I, p. 114, no. 32; VI, p. 243, no. 1237.

\protect\hypertarget{23_NOTES.xhtmlux5cux23id_1046}{\protect\hyperlink{13_Chapter_Six__THE_DEPICTION_OF_TH.xhtmlux5cux23id_1045}{114}}.
Bambergisches Missale from 1490, in Uhrig, Die 14 hl. Nothelfer (XIV.
Auxiliatores), Theol. Quartalschrift, LXX, 1888, p. 72; see an Utrecht
Missal from 1514 and a Dominican Missal of 1550, in Acta sanctorum
Aprilis, t. III, p. 149.

\protect\hypertarget{23_NOTES.xhtmlux5cux23id_1044}{\protect\hyperlink{13_Chapter_Six__THE_DEPICTION_OF_TH.xhtmlux5cux23id_1043}{115}}.
Erasmus, Ratio seu methodus compendio pervendi ad veram theologiam, ed.
Basel, 1520, p. 171. ({[}Trans.{]} Added in the German translation: see
Moriae Encomium, cap. 40; Colloquia, Militaria, LB I 642.)

\protect\hypertarget{23_NOTES.xhtmlux5cux23id_1042}{\protect\hyperlink{13_Chapter_Six__THE_DEPICTION_OF_TH.xhtmlux5cux23id_1041}{116}}.
In the just cited ballade of Deschamps we also find Martha, who
destroyed the Tarasque at Tarascon. {[}Trans.{]} That is, St. Martha,
who destroyed a monster called the Tarasque at the town of Tarascon in
southern France.

\protect\hypertarget{23_NOTES.xhtmlux5cux23id_1040}{\protect\hyperlink{13_Chapter_Six__THE_DEPICTION_OF_TH.xhtmlux5cux23id_1039}{117}}.
Oeuvres de Coquillart, ed. Ch. d'Héricault (Bibl. elzevirenne), 1857,
II, p. 281.

\protect\hypertarget{23_NOTES.xhtmlux5cux23id_1038}{\protect\hyperlink{13_Chapter_Six__THE_DEPICTION_OF_TH.xhtmlux5cux23id_1037}{118}}.
Deschamps, no. 1230, VI, p. 232.

\protect\hypertarget{23_NOTES.xhtmlux5cux23id_1036}{\protect\hyperlink{13_Chapter_Six__THE_DEPICTION_OF_TH.xhtmlux5cux23id_1035}{119}}.
Rob. Gaguini, Epistole et orationes, ed. Thuasne, II, p. 176.

\protect\hypertarget{23_NOTES.xhtmlux5cux23id_1034}{\protect\hyperlink{13_Chapter_Six__THE_DEPICTION_OF_TH.xhtmlux5cux23id_1033}{120}}.
Colloquia, Exequiae Seraphicae, ed. Elzev., p. 620. {[}Trans.{]} The
German translation adds: I, c. 869 B, see Ep. 447, line 426, Allen II,
p. 303, and cites the Colloquia as Leidener Ausg.

\protect\hypertarget{23_NOTES.xhtmlux5cux23id_1032}{\protect\hyperlink{13_Chapter_Six__THE_DEPICTION_OF_TH.xhtmlux5cux23id_1031}{121}}.
Gargantua, chap. 45.

\protect\hypertarget{23_NOTES.xhtmlux5cux23id_1030}{\protect\hyperlink{13_Chapter_Six__THE_DEPICTION_OF_TH.xhtmlux5cux23id_1029}{122}}.
Apologie pour Hérodote, chap. 38, ed. Ristelhuber, 1879, II, p. 324.

\protect\hypertarget{23_NOTES.xhtmlux5cux23id_1028}{\protect\hyperlink{13_Chapter_Six__THE_DEPICTION_OF_TH.xhtmlux5cux23id_1027}{123}}.
Deschamps, VIII, p. 201, no. 1489.

\protect\hypertarget{23_NOTES.xhtmlux5cux23id_1026}{\protect\hyperlink{13_Chapter_Six__THE_DEPICTION_OF_TH.xhtmlux5cux23id_1025}{124}}.
Gerson, de Angelis, Opera, III, p. 1481; De praeceptis decalogi, I, p.
431; Oratio ad bonum angelum suum, III, p. 511; Tractatus VIII super
Magnificat, IV, p. 370; see III, pp. 137, 553, 739.

\protect\hypertarget{23_NOTES.xhtmlux5cux23id_1024}{\protect\hyperlink{13_Chapter_Six__THE_DEPICTION_OF_TH.xhtmlux5cux23id_1023}{125}}.
Gerson, Opera, IV, p. 389.

\textbf{\emph{Chapter} 7}

\protect\hypertarget{23_NOTES.xhtmlux5cux23id_3078}{\protect\hyperlink{14_Chapter_Seven__THE_PIOUS_PERSONA.xhtmlux5cux23id_3077}{*\textsuperscript{1}}}
``in order to absolve everyone''

\protect\hypertarget{23_NOTES.xhtmlux5cux23id_3080}{\protect\hyperlink{14_Chapter_Seven__THE_PIOUS_PERSONA.xhtmlux5cux23id_3079}{†\textsuperscript{2}}}
We pray God that the Jacobins/Might eat the Augustinians/And that the
Carmelites might be hung/With the cords of the Friars Minor.

\protect\hypertarget{23_NOTES.xhtmlux5cux23id_2546}{\protect\hyperlink{14_Chapter_Seven__THE_PIOUS_PERSONA.xhtmlux5cux23id_2545}{*\textsuperscript{3}}}
``and they who had earlier prayed for him now cursed him.''

\protect\hypertarget{23_NOTES.xhtmlux5cux23id_2548}{\protect\hyperlink{14_Chapter_Seven__THE_PIOUS_PERSONA.xhtmlux5cux23id_2547}{*\textsuperscript{4}}}
``If God wants to give me the victory, He will keep it for me.''

\protect\hypertarget{23_NOTES.xhtmlux5cux23id_2550}{\protect\hyperlink{14_Chapter_Seven__THE_PIOUS_PERSONA.xhtmlux5cux23id_2549}{*\textsuperscript{5}}}
``his dear hermitage of Reculée''

\protect\hypertarget{23_NOTES.xhtmlux5cux23id_2552}{\protect\hyperlink{14_Chapter_Seven__THE_PIOUS_PERSONA.xhtmlux5cux23id_2551}{†\textsuperscript{6}}}
``not differing from the barrows in which dung and ordure are usually
carried .~.~. and I have heard it recounted and said .~.~. that in all
towns where he came, he made similar entries out of humility.''

\protect\hypertarget{23_NOTES.xhtmlux5cux23id_2554}{\protect\hyperlink{14_Chapter_Seven__THE_PIOUS_PERSONA.xhtmlux5cux23id_2553}{*\textsuperscript{7}}}
``if God had hated him so much that he let him die at the court of
princes of this world.''

\protect\hypertarget{23_NOTES.xhtmlux5cux23id_2556}{\protect\hyperlink{14_Chapter_Seven__THE_PIOUS_PERSONA.xhtmlux5cux23id_2555}{*\textsuperscript{8}}}
``There was killed in good style the aforesaid Lord Charles of Blois,
with his face to the enemy, and a bastard son of his called Jehans de
Blois, and several other knights and squires of Brittany.''

\protect\hypertarget{23_NOTES.xhtmlux5cux23id_2558}{\protect\hyperlink{14_Chapter_Seven__THE_PIOUS_PERSONA.xhtmlux5cux23id_2557}{*\textsuperscript{9}}}
``Sweet, courteous and debonair .~.~. virgin as to body, a great giver
of alms. The greater part of the day and the night he spent in prayer.
And in all his life there was nothing but humility.''

\protect\hypertarget{23_NOTES.xhtmlux5cux23id_2560}{\protect\hyperlink{14_Chapter_Seven__THE_PIOUS_PERSONA.xhtmlux5cux23id_2559}{*\textsuperscript{10}}}
``I see well .~.~. that you want to lead me from the right road to the
bad; but assuredly, if once I enter on it, I shall do so much that the
whole world will talk of me.''

\protect\hypertarget{23_NOTES.xhtmlux5cux23id_2562}{\protect\hyperlink{14_Chapter_Seven__THE_PIOUS_PERSONA.xhtmlux5cux23id_2561}{*\textsuperscript{11}}}
``who bought the grace of God and of the Virgin Mary for more money than
ever king did''

\protect\hypertarget{23_NOTES.xhtmlux5cux23id_2320}{\protect\hyperlink{14_Chapter_Seven__THE_PIOUS_PERSONA.xhtmlux5cux23id_2319}{*\textsuperscript{12}}}
``Monsieur de Genas, I beg you to send me lemons and sweet oranges, and
muscatel pears, and parsnips, and it is for the holy man who eats
neither flesh nor fish; and you will do me a very great favor.''

\protect\hypertarget{23_NOTES.xhtmlux5cux23id_2322}{\protect\hyperlink{14_Chapter_Seven__THE_PIOUS_PERSONA.xhtmlux5cux23id_2321}{*\textsuperscript{13}}}
``He is still alive .~.~. so that he may well change, for the better or
for the worst, so that I shall be silent, as many mocked at the arrival
of this hermit, whom they called `holy man.'\,''

\protect\hypertarget{23_NOTES.xhtmlux5cux23id_2324}{\protect\hyperlink{14_Chapter_Seven__THE_PIOUS_PERSONA.xhtmlux5cux23id_2323}{†\textsuperscript{14}}}
``of such saintly life, nor one in whom the Holy Spirit seemed more to
speak through his mouth.''

\protect\hypertarget{23_NOTES.xhtmlux5cux23id_2326}{\protect\hyperlink{14_Chapter_Seven__THE_PIOUS_PERSONA.xhtmlux5cux23id_2325}{‡\textsuperscript{15}}}
``famous pious man and greatest prince and duke,''

\protect\hypertarget{23_NOTES.xhtmlux5cux23id_2328}{\protect\hyperlink{14_Chapter_Seven__THE_PIOUS_PERSONA.xhtmlux5cux23id_2327}{*\textsuperscript{16}}}
``He who reads Dionysius, has read everything.''

\protect\hypertarget{23_NOTES.xhtmlux5cux23id_1022}{\protect\hyperlink{14_Chapter_Seven__THE_PIOUS_PERSONA.xhtmlux5cux23id_1021}{1}}.
Monstrelet, IV, p. 304.

\protect\hypertarget{23_NOTES.xhtmlux5cux23id_1020}{\protect\hyperlink{14_Chapter_Seven__THE_PIOUS_PERSONA.xhtmlux5cux23id_1019}{2}}.
Bernh. of Siena, Opera, I, p. 100, in Hefele, Der h. Bernhardin von
Siena .~.~. , p. 36.

\protect\hypertarget{23_NOTES.xhtmlux5cux23id_1018}{\protect\hyperlink{14_Chapter_Seven__THE_PIOUS_PERSONA.xhtmlux5cux23id_1017}{3}}.
Les cent nouvelles nouvelles, II, p. 157; Les quinze joyes de mariage,
pp. 111, 215.

\protect\hypertarget{23_NOTES.xhtmlux5cux23id_1016}{\protect\hyperlink{14_Chapter_Seven__THE_PIOUS_PERSONA.xhtmlux5cux23id_1015}{4}}.
Molinet, Faictz et dictz, f. 188 vso.

\protect\hypertarget{23_NOTES.xhtmlux5cux23id_1014}{\protect\hyperlink{14_Chapter_Seven__THE_PIOUS_PERSONA.xhtmlux5cux23id_1013}{5}}.
{[}Trans.{]} \emph{duke of Armagnac}: A title of Louis d'Orléans.

\protect\hypertarget{23_NOTES.xhtmlux5cux23id_1012}{\protect\hyperlink{14_Chapter_Seven__THE_PIOUS_PERSONA.xhtmlux5cux23id_1011}{6}}.
Journal d'un bourgeois, p. 336, see p. 242, no. 514.

\protect\hypertarget{23_NOTES.xhtmlux5cux23id_1010}{\protect\hyperlink{14_Chapter_Seven__THE_PIOUS_PERSONA.xhtmlux5cux23id_1009}{7}}.
Ghillebert de Lannoy, Oeuvres, ed. Ch. Potvin, Louvain, 1878, p. 163.

\protect\hypertarget{23_NOTES.xhtmlux5cux23id_1008}{\protect\hyperlink{14_Chapter_Seven__THE_PIOUS_PERSONA.xhtmlux5cux23id_1007}{8}}.
Les cent nouvelles nouvelles, II, p. 101.

\protect\hypertarget{23_NOTES.xhtmlux5cux23id_1006}{\protect\hyperlink{14_Chapter_Seven__THE_PIOUS_PERSONA.xhtmlux5cux23id_1005}{9}}.
Lejouvencel, II, p. 107.

\protect\hypertarget{23_NOTES.xhtmlux5cux23page_421}{\protect\hyperlink{14_Chapter_Seven__THE_PIOUS_PERSONA.xhtmlux5cux23id_1004}{10}}.
Songe de viel pelerin, bij Jorga, Phi. de Mézières, p.
423\textsuperscript{e}.

\protect\hypertarget{23_NOTES.xhtmlux5cux23id_1003}{\protect\hyperlink{14_Chapter_Seven__THE_PIOUS_PERSONA.xhtmlux5cux23id_1002}{11}}.
Journal d'un bourgeois, pp. 214, 2892.

\protect\hypertarget{23_NOTES.xhtmlux5cux23id_1001}{\protect\hyperlink{14_Chapter_Seven__THE_PIOUS_PERSONA.xhtmlux5cux23id_1000}{12}}.
Gerson, Opera, I, p. 206.

\protect\hypertarget{23_NOTES.xhtmlux5cux23id_999}{\protect\hyperlink{14_Chapter_Seven__THE_PIOUS_PERSONA.xhtmlux5cux23id_998}{13}}.
Jorga, Phil. de Mézières, p. 506.

\protect\hypertarget{23_NOTES.xhtmlux5cux23id_997}{\protect\hyperlink{14_Chapter_Seven__THE_PIOUS_PERSONA.xhtmlux5cux23id_996}{14}}.
W. Moll, Johannes Brugman, II, p. 125.

\protect\hypertarget{23_NOTES.xhtmlux5cux23id_995}{\protect\hyperlink{14_Chapter_Seven__THE_PIOUS_PERSONA.xhtmlux5cux23id_994}{15}}.
Chastellain, IV, p. 263--65.

\protect\hypertarget{23_NOTES.xhtmlux5cux23id_993}{\protect\hyperlink{14_Chapter_Seven__THE_PIOUS_PERSONA.xhtmlux5cux23id_992}{16}}.
Chastellain, II, p. 300; VII, p. 222. Jean Germain, Liber de virtutibus,
p. 10 (The less severe fasting exercise mentioned here may belong to a
different time); Jean Jouffroy, De Philippo duce oratio (Chron. rel. à
l'hist de Belg. sous la dom. des ducs de Bourg., III), p. 118.

\protect\hypertarget{23_NOTES.xhtmlux5cux23id_991}{\protect\hyperlink{14_Chapter_Seven__THE_PIOUS_PERSONA.xhtmlux5cux23id_990}{17}}.
La Marche, II, p. 40.

\protect\hypertarget{23_NOTES.xhtmlux5cux23id_989}{\protect\hyperlink{14_Chapter_Seven__THE_PIOUS_PERSONA.xhtmlux5cux23id_988}{18}}.
Monstrelet, IV, p. 302.

\protect\hypertarget{23_NOTES.xhtmlux5cux23id_987}{\protect\hyperlink{14_Chapter_Seven__THE_PIOUS_PERSONA.xhtmlux5cux23id_986}{19}}.
Jorga, Phil. de Mézières, p. 350.

\protect\hypertarget{23_NOTES.xhtmlux5cux23id_985}{\protect\hyperlink{14_Chapter_Seven__THE_PIOUS_PERSONA.xhtmlux5cux23id_984}{20}}.
See Jorga, Phil. de Mézières, p. 444; Champion, Villon, I, p. 17.

\protect\hypertarget{23_NOTES.xhtmlux5cux23id_983}{\protect\hyperlink{14_Chapter_Seven__THE_PIOUS_PERSONA.xhtmlux5cux23id_982}{21}}.
{[}Trans.{]} \emph{Gerard Groote}: 1340--84. Mystic and popular preacher
who founded both the Brethren of the Common Life and the Windesheim
Convent (although the latter did not come into existence until after his
death). Some modern opinion is that it is he rather than Thomas à Kempis
who was the author of the \emph{Imitatio Christi}.

\protect\hypertarget{23_NOTES.xhtmlux5cux23id_981}{\protect\hyperlink{14_Chapter_Seven__THE_PIOUS_PERSONA.xhtmlux5cux23id_980}{22}}.
Oeuvres du roi René, ed. Quatrebarbes, I, p. cx.

\protect\hypertarget{23_NOTES.xhtmlux5cux23id_979}{\protect\hyperlink{14_Chapter_Seven__THE_PIOUS_PERSONA.xhtmlux5cux23id_978}{23}}.
Monstrelet, V, p. 112.

\protect\hypertarget{23_NOTES.xhtmlux5cux23id_977}{\protect\hyperlink{14_Chapter_Seven__THE_PIOUS_PERSONA.xhtmlux5cux23id_976}{24}}.
La Marche, I, p. 194.

\protect\hypertarget{23_NOTES.xhtmlux5cux23id_975}{\protect\hyperlink{14_Chapter_Seven__THE_PIOUS_PERSONA.xhtmlux5cux23id_974}{25}}.
Acta sanctorum Jan., t. II, p. 1018.

\protect\hypertarget{23_NOTES.xhtmlux5cux23id_973}{\protect\hyperlink{14_Chapter_Seven__THE_PIOUS_PERSONA.xhtmlux5cux23id_972}{26}}.
Jorga, Phil. de Mézières, pp. 509, 512.

\protect\hypertarget{23_NOTES.xhtmlux5cux23id_971}{\protect\hyperlink{14_Chapter_Seven__THE_PIOUS_PERSONA.xhtmlux5cux23id_970}{27}}.
It is not important in this connection whether the church had clarified
the question of recommending persons for sainthood or only for
beatification.

\protect\hypertarget{23_NOTES.xhtmlux5cux23id_969}{\protect\hyperlink{14_Chapter_Seven__THE_PIOUS_PERSONA.xhtmlux5cux23id_968}{28}}.
André Du Chesne, Hist. de la maison de Chastillon sur Marne, Paris 1621,
Preuves, pp. 126--31, Extraict de l'enqueste faite pour la canonization
de Charles de Blois, pp. 223, 234.

\protect\hypertarget{23_NOTES.xhtmlux5cux23id_967}{\protect\hyperlink{14_Chapter_Seven__THE_PIOUS_PERSONA.xhtmlux5cux23id_966}{29}}.
Froissart, ed. Luce, VI, p. 168.

\protect\hypertarget{23_NOTES.xhtmlux5cux23id_965}{\protect\hyperlink{14_Chapter_Seven__THE_PIOUS_PERSONA.xhtmlux5cux23id_964}{30}}.
The grounds on which Dom Plaine, Revue des questions historiques, XI, p.
41, objects to Froissart's testimony do not seem cogent enough to me.

\protect\hypertarget{23_NOTES.xhtmlux5cux23id_963}{\protect\hyperlink{14_Chapter_Seven__THE_PIOUS_PERSONA.xhtmlux5cux23id_962}{31}}.
W. James, The varieties of religious experience, pp. 37of.

\protect\hypertarget{23_NOTES.xhtmlux5cux23id_961}{\protect\hyperlink{14_Chapter_Seven__THE_PIOUS_PERSONA.xhtmlux5cux23id_960}{32}}.
Ordonnances des rois de France, t. VIII, p. 398, Nov. 1400, 426, 18
March 1401.

\protect\hypertarget{23_NOTES.xhtmlux5cux23id_959}{\protect\hyperlink{14_Chapter_Seven__THE_PIOUS_PERSONA.xhtmlux5cux23id_958}{33}}.
Mémoires de Pierre Salmon, ed. Buchon, Coll. de chron. nationales,
3\textsuperscript{e} Supplément de Froissart, XV, p. 49.

\protect\hypertarget{23_NOTES.xhtmlux5cux23id_957}{\protect\hyperlink{14_Chapter_Seven__THE_PIOUS_PERSONA.xhtmlux5cux23id_956}{34}}.
Froissart, ed. Kervyn, XIII, p. 40.

\protect\hypertarget{23_NOTES.xhtmlux5cux23id_955}{\protect\hyperlink{14_Chapter_Seven__THE_PIOUS_PERSONA.xhtmlux5cux23id_954}{35}}.
Acta sanctorum Julii, t. I, p. 486--628. Prof. Wensinck has brought to
my attention that the custom of keeping a daily list of sins is a very
old saintly tradition already described by Johannes Climacus (c. 600),
Scala Paradisi, ed. Raderus, Paris 1633, p. 65; that it is also known in
Islam, by Ghazâlî, and that it is still recommended by Ignatius of
Loyola in the Exercitia spiritualia.

\protect\hypertarget{23_NOTES.xhtmlux5cux23id_953}{\protect\hyperlink{14_Chapter_Seven__THE_PIOUS_PERSONA.xhtmlux5cux23id_952}{36}}.
La Marche, I, p. 180.

\protect\hypertarget{23_NOTES.xhtmlux5cux23id_951}{\protect\hyperlink{14_Chapter_Seven__THE_PIOUS_PERSONA.xhtmlux5cux23id_950}{37}}.
Lettres de Louis XI, t. VI, p. 514, cf. V, p. 86, X, p. 65.

\protect\hypertarget{23_NOTES.xhtmlux5cux23id_949}{\protect\hyperlink{14_Chapter_Seven__THE_PIOUS_PERSONA.xhtmlux5cux23id_948}{38}}.
Commines, I, p. 291.

\protect\hypertarget{23_NOTES.xhtmlux5cux23id_947}{\protect\hyperlink{14_Chapter_Seven__THE_PIOUS_PERSONA.xhtmlux5cux23id_946}{39}}.
Commines, II, pp. 67, 68.

\protect\hypertarget{23_NOTES.xhtmlux5cux23page_422}{\protect\hyperlink{14_Chapter_Seven__THE_PIOUS_PERSONA.xhtmlux5cux23id_945}{40}}.
Commines, II, p. 57; Lettres X, p. 16; IX, p. 260. At times, there was
such an \emph{agnus scythicus} in the Colouia Museum at Haarlem.

\protect\hypertarget{23_NOTES.xhtmlux5cux23id_944}{\protect\hyperlink{14_Chapter_Seven__THE_PIOUS_PERSONA.xhtmlux5cux23id_943}{41}}.
Chron. scan., II, p. 122.

\protect\hypertarget{23_NOTES.xhtmlux5cux23id_942}{\protect\hyperlink{14_Chapter_Seven__THE_PIOUS_PERSONA.xhtmlux5cux23id_941}{42}}.
Commines, II, pp. 55, 77.

\protect\hypertarget{23_NOTES.xhtmlux5cux23id_940}{\protect\hyperlink{14_Chapter_Seven__THE_PIOUS_PERSONA.xhtmlux5cux23id_939}{43}}.
Acta sanctorum Apr. t., I, p. 115.---Lettres de Louis XI, X, pp. 76, 90.

\protect\hypertarget{23_NOTES.xhtmlux5cux23id_938}{\protect\hyperlink{14_Chapter_Seven__THE_PIOUS_PERSONA.xhtmlux5cux23id_937}{44}}.
Sed volens caute atque astute agere propterea quod a pluribus fuisset
sub umbra sanctitatis deceptus, decrevit variis modis experiri virtutem
servi Dei, Acta sanctorum, Apr., t. 1, p. 115.

\protect\hypertarget{23_NOTES.xhtmlux5cux23id_936}{\protect\hyperlink{14_Chapter_Seven__THE_PIOUS_PERSONA.xhtmlux5cux23id_935}{45}}.
Acta sanctorum, Apr., t. 1, p. 108; Commines, II, p. 55.

\protect\hypertarget{23_NOTES.xhtmlux5cux23id_934}{\protect\hyperlink{14_Chapter_Seven__THE_PIOUS_PERSONA.xhtmlux5cux23id_933}{46}}.
Lettres, X, pp. 124, 29. June 1483.

\protect\hypertarget{23_NOTES.xhtmlux5cux23id_932}{\protect\hyperlink{14_Chapter_Seven__THE_PIOUS_PERSONA.xhtmlux5cux23id_931}{47}}.
Lettres, X, p. 4 \emph{passim}; Commines, II, p. 54.

\protect\hypertarget{23_NOTES.xhtmlux5cux23id_930}{\protect\hyperlink{14_Chapter_Seven__THE_PIOUS_PERSONA.xhtmlux5cux23id_929}{48}}.
Commines, II, p. 56; Acta sanctorum, Apr., t. 1, p. 115.

\protect\hypertarget{23_NOTES.xhtmlux5cux23id_928}{\protect\hyperlink{14_Chapter_Seven__THE_PIOUS_PERSONA.xhtmlux5cux23id_927}{49}}.
A. Renaudet, Préréforme et humanisme à Paris, p. 172.

\protect\hypertarget{23_NOTES.xhtmlux5cux23id_926}{\protect\hyperlink{14_Chapter_Seven__THE_PIOUS_PERSONA.xhtmlux5cux23id_925}{50}}.
Doutrepont, p. 226.

\protect\hypertarget{23_NOTES.xhtmlux5cux23id_924}{\protect\hyperlink{14_Chapter_Seven__THE_PIOUS_PERSONA.xhtmlux5cux23id_923}{51}}.
Vita Dionysii auct. Theod. Loer, Dion. Opera, I, p. xliiff., id. De vita
et regimine principum, t. XXXVII, p. 497.

\protect\hypertarget{23_NOTES.xhtmlux5cux23id_922}{\protect\hyperlink{14_Chapter_Seven__THE_PIOUS_PERSONA.xhtmlux5cux23id_921}{52}}.
Opera, t. XLI, p. 621; D. A. Mougel, Denys le chartreux, sa vie etc.;
Montreuil, 1896, p. 63.

\protect\hypertarget{23_NOTES.xhtmlux5cux23id_920}{\protect\hyperlink{14_Chapter_Seven__THE_PIOUS_PERSONA.xhtmlux5cux23id_919}{53}}.
Opera, t. XLI, p. 617; Vita, I, p. xxxi; Mougel, p. 51; Bijdragen en
mede-deelingen van het historisch genootschap te Utrecht, XVIII, p.
331d.

\protect\hypertarget{23_NOTES.xhtmlux5cux23id_918}{\protect\hyperlink{14_Chapter_Seven__THE_PIOUS_PERSONA.xhtmlux5cux23id_917}{54}}.
Opera, t. XXXIX, p. 496; Mougel, p. 54; Moll, Johannes Brugman, I, p.
74; Kerkgesch., II 2, p. 124; K. Krogh-Tonning, Der letzte Scholastiker,
Freiburg 1904, p. 175.

\protect\hypertarget{23_NOTES.xhtmlux5cux23id_916}{\protect\hyperlink{14_Chapter_Seven__THE_PIOUS_PERSONA.xhtmlux5cux23id_915}{55}}.
Mougel, p. 58.

\protect\hypertarget{23_NOTES.xhtmlux5cux23id_914}{\protect\hyperlink{14_Chapter_Seven__THE_PIOUS_PERSONA.xhtmlux5cux23id_913}{56}}.
De mutua cogitione, Opera, t. XXXVI, p. 178.

\protect\hypertarget{23_NOTES.xhtmlux5cux23id_912}{\protect\hyperlink{14_Chapter_Seven__THE_PIOUS_PERSONA.xhtmlux5cux23id_911}{57}}.
Vita, Opera, t. I, p. xxiv, xxxviii.

\protect\hypertarget{23_NOTES.xhtmlux5cux23id_910}{\protect\hyperlink{14_Chapter_Seven__THE_PIOUS_PERSONA.xhtmlux5cux23id_909}{58}}.
Vita, Opera, t. I, p. XXVI.

\protect\hypertarget{23_NOTES.xhtmlux5cux23id_908}{\protect\hyperlink{14_Chapter_Seven__THE_PIOUS_PERSONA.xhtmlux5cux23id_907}{59}}.
De munificentia Dei beneficiis Dei, Opera, t. XXXIV, art. 26, p. 319.

\textbf{\emph{Chapter 8}}

\protect\hypertarget{23_NOTES.xhtmlux5cux23id_2564}{\protect\hyperlink{15_Chapter_Eight__RELIGIOUS_EXCITAT.xhtmlux5cux23id_2563}{*\textsuperscript{1}}}
``And he was much visited by people who came to see him from all
countries on account of the simple, very noble and most honest life he
led.''

\protect\hypertarget{23_NOTES.xhtmlux5cux23id_2566}{\protect\hyperlink{15_Chapter_Eight__RELIGIOUS_EXCITAT.xhtmlux5cux23id_2565}{†\textsuperscript{2}}}
``Saint of St. Lié''

\protect\hypertarget{23_NOTES.xhtmlux5cux23id_2568}{\protect\hyperlink{15_Chapter_Eight__RELIGIOUS_EXCITAT.xhtmlux5cux23id_2567}{‡\textsuperscript{3}}}
``fool of St. Lié''

\protect\hypertarget{23_NOTES.xhtmlux5cux23id_2570}{\protect\hyperlink{15_Chapter_Eight__RELIGIOUS_EXCITAT.xhtmlux5cux23id_2569}{§\textsuperscript{4}}}
``After the wolf, after the wolf!''

\protect\hypertarget{23_NOTES.xhtmlux5cux23id_2572}{\protect\hyperlink{15_Chapter_Eight__RELIGIOUS_EXCITAT.xhtmlux5cux23id_2571}{*\textsuperscript{5}}}
``prostitute of apostasy''

\protect\hypertarget{23_NOTES.xhtmlux5cux23id_906}{\protect\hyperlink{15_Chapter_Eight__RELIGIOUS_EXCITAT.xhtmlux5cux23id_905}{1}}.
Gerson, Tractatus VIII super Magnificat, Opera, IV, p. 386.

\protect\hypertarget{23_NOTES.xhtmlux5cux23id_904}{\protect\hyperlink{15_Chapter_Eight__RELIGIOUS_EXCITAT.xhtmlux5cux23id_903}{2}}.
Acta sanctorum Martii, t. I, p. 561, see pp. 540, 601.

\protect\hypertarget{23_NOTES.xhtmlux5cux23id_902}{\protect\hyperlink{15_Chapter_Eight__RELIGIOUS_EXCITAT.xhtmlux5cux23id_901}{3}}.
Hefele, Der h. Bernhardin von Siena .~.~. , p. 79.

\protect\hypertarget{23_NOTES.xhtmlux5cux23id_900}{\protect\hyperlink{15_Chapter_Eight__RELIGIOUS_EXCITAT.xhtmlux5cux23id_899}{4}}.
Moll, Johannes Brugman, II, pp. 74, 86.

\protect\hypertarget{23_NOTES.xhtmlux5cux23id_898}{\protect\hyperlink{15_Chapter_Eight__RELIGIOUS_EXCITAT.xhtmlux5cux23id_897}{5}}.
See above, p. 181.

\protect\hypertarget{23_NOTES.xhtmlux5cux23id_896}{\protect\hyperlink{15_Chapter_Eight__RELIGIOUS_EXCITAT.xhtmlux5cux23id_895}{6}}.
See above, p. 4.

\protect\hypertarget{23_NOTES.xhtmlux5cux23id_894}{\protect\hyperlink{15_Chapter_Eight__RELIGIOUS_EXCITAT.xhtmlux5cux23id_893}{7}}.
Acta sanctorum Apr, t. I, p. 195.---The picture which Hefele (Der h.
Bernhardin von Siena .~.~. ) gives of the preachers in Italy is in many
regards accurate for French-speaking countries.

\protect\hypertarget{23_NOTES.xhtmlux5cux23id_892}{\protect\hyperlink{15_Chapter_Eight__RELIGIOUS_EXCITAT.xhtmlux5cux23id_891}{8}}.
Opus quadragesimale Sancti Vincentii, 1482, and Oliverii Maillardi
Sermones dominicales etc., Paris, Jean Petit, 1515. In the first edition
(see p. 316, note 2) I stated that I had not found the work of these two
in the Netherlands. Dr. C. van Slee and Miss M. E. Kronenberg were kind
enough to point out to me that the DeVente Athenaeum Library owns both.

\protect\hypertarget{23_NOTES.xhtmlux5cux23id_890}{\protect\hyperlink{15_Chapter_Eight__RELIGIOUS_EXCITAT.xhtmlux5cux23id_889}{9}}.
Life of S. Petrus Thomasius, Carmeliter, in Philippe de Mézières, Acta
sanctorum Jan., t. II, p. 997; also Dionysius Cartusianus over Brugman's
style of
\protect\hypertarget{23_NOTES.xhtmlux5cux23page_423}{}{}preaching: De
vita et regimine episcoporum, nobilium, etc., etc., vol. 37ff.; Inter
Jesum et puerum dialogus, vol. 38.

\protect\hypertarget{23_NOTES.xhtmlux5cux23id_888}{\protect\hyperlink{15_Chapter_Eight__RELIGIOUS_EXCITAT.xhtmlux5cux23id_887}{10}}.
Acta sanctorum Apr., t. I, p. 513.

\protect\hypertarget{23_NOTES.xhtmlux5cux23id_886}{\protect\hyperlink{15_Chapter_Eight__RELIGIOUS_EXCITAT.xhtmlux5cux23id_885}{11}}.
James, Varieties of Religious Experience, p. 348: ``For sensitiveness
and narrowness, when they occur together, as they often do, require
above all things a simplified world to dwell in''; cf. p. 3531.

\protect\hypertarget{23_NOTES.xhtmlux5cux23id_884}{\protect\hyperlink{15_Chapter_Eight__RELIGIOUS_EXCITAT.xhtmlux5cux23id_883}{12}}.
Moll, Brugman, I, p. 52.

\protect\hypertarget{23_NOTES.xhtmlux5cux23id_882}{\protect\hyperlink{15_Chapter_Eight__RELIGIOUS_EXCITAT.xhtmlux5cux23id_881}{13}}.
Dion. Cart. De quotidiano baptimate lacrimarum, t. XXIX, p. 84; De
oratione, t. XLI, p. 31--55; Expositio hymni Audi conditor, t. XXXV, p.
34.

\protect\hypertarget{23_NOTES.xhtmlux5cux23id_880}{\protect\hyperlink{15_Chapter_Eight__RELIGIOUS_EXCITAT.xhtmlux5cux23id_879}{14}}.
Acta sanctorum Apr., t. I, pp. 485, 494.

\protect\hypertarget{23_NOTES.xhtmlux5cux23id_878}{\protect\hyperlink{15_Chapter_Eight__RELIGIOUS_EXCITAT.xhtmlux5cux23id_877}{15}}.
Chastellain, III p. 119; Antonio de Beatis (1517), L. Pastor, Die Reise
des Kardinals Luigi d'Aragona, Freiburg 1905, p. 513, 52; Polydorus
Vergilius, Anglicae historiae libri XXVI, Basileae, 1546, p. 15.

\protect\hypertarget{23_NOTES.xhtmlux5cux23id_876}{\protect\hyperlink{15_Chapter_Eight__RELIGIOUS_EXCITAT.xhtmlux5cux23id_875}{16}}.
Gerson, Epistola contra libellum Johannis de Schonhavia, Opera, I, p.
79.

\protect\hypertarget{23_NOTES.xhtmlux5cux23id_874}{\protect\hyperlink{15_Chapter_Eight__RELIGIOUS_EXCITAT.xhtmlux5cux23id_873}{17}}.
Gerson, De distinctione verarum visionum a falsis, Opera, I, p. 44.

\protect\hypertarget{23_NOTES.xhtmlux5cux23id_872}{\protect\hyperlink{15_Chapter_Eight__RELIGIOUS_EXCITAT.xhtmlux5cux23id_871}{18}}.
Ibid., p. 48.

\protect\hypertarget{23_NOTES.xhtmlux5cux23id_870}{\protect\hyperlink{15_Chapter_Eight__RELIGIOUS_EXCITAT.xhtmlux5cux23id_869}{19}}.
Gerson, De examinatione doctrinarum, Opera, I, p. 19.

\protect\hypertarget{23_NOTES.xhtmlux5cux23id_868}{\protect\hyperlink{15_Chapter_Eight__RELIGIOUS_EXCITAT.xhtmlux5cux23id_867}{20}}.
Ibid., p. 16, 17.

\protect\hypertarget{23_NOTES.xhtmlux5cux23id_866}{\protect\hyperlink{15_Chapter_Eight__RELIGIOUS_EXCITAT.xhtmlux5cux23id_865}{21}}.
Gerson, De distinctione etc., I, p. 44.

\protect\hypertarget{23_NOTES.xhtmlux5cux23id_864}{\protect\hyperlink{15_Chapter_Eight__RELIGIOUS_EXCITAT.xhtmlux5cux23id_863}{22}}.
Gerson, Tractatus II super Magnificat, Opera, IV, p. 248.

\protect\hypertarget{23_NOTES.xhtmlux5cux23id_862}{\protect\hyperlink{15_Chapter_Eight__RELIGIOUS_EXCITAT.xhtmlux5cux23id_861}{23}}.
Sixty-five useful articles on the Passion of our Lord, Moll, Brugman,
II, p. 75.

\protect\hypertarget{23_NOTES.xhtmlux5cux23id_860}{\protect\hyperlink{15_Chapter_Eight__RELIGIOUS_EXCITAT.xhtmlux5cux23id_859}{24}}.
Gerson, De monte contemplationis, Opera, III, p. 562.

\protect\hypertarget{23_NOTES.xhtmlux5cux23id_858}{\protect\hyperlink{15_Chapter_Eight__RELIGIOUS_EXCITAT.xhtmlux5cux23id_857}{25}}.
Gerson, De distinctione etc., Opera, I, p. 49.

\protect\hypertarget{23_NOTES.xhtmlux5cux23id_856}{\protect\hyperlink{15_Chapter_Eight__RELIGIOUS_EXCITAT.xhtmlux5cux23id_855}{26}}.
Ibid.

\protect\hypertarget{23_NOTES.xhtmlux5cux23id_854}{\protect\hyperlink{15_Chapter_Eight__RELIGIOUS_EXCITAT.xhtmlux5cux23id_853}{27}}.
Acta sanctorum Martii, t. I, p. 562.

\protect\hypertarget{23_NOTES.xhtmlux5cux23id_852}{\protect\hyperlink{15_Chapter_Eight__RELIGIOUS_EXCITAT.xhtmlux5cux23id_851}{28}}.
James, Varieties of religious experience, p. 343.

\protect\hypertarget{23_NOTES.xhtmlux5cux23id_850}{\protect\hyperlink{15_Chapter_Eight__RELIGIOUS_EXCITAT.xhtmlux5cux23id_849}{29}}.
Acta sanctorum, Martii, t. 1, p. 552ff

\protect\hypertarget{23_NOTES.xhtmlux5cux23id_848}{\protect\hyperlink{15_Chapter_Eight__RELIGIOUS_EXCITAT.xhtmlux5cux23id_847}{30}}.
Froissart, ed. Kervyn, XV, p. 132; Rel. de S. Denis, II, p. 124;
Johannis de Varennis, Responsiones ad capita accusationum in Gerson,
Opera, I, pp. 925, 926.

\protect\hypertarget{23_NOTES.xhtmlux5cux23id_846}{\protect\hyperlink{15_Chapter_Eight__RELIGIOUS_EXCITAT.xhtmlux5cux23id_845}{31}}.
Responsiones, Opera, I, p. 936.

\emph{\protect\hypertarget{23_NOTES.xhtmlux5cux23id_844}{\protect\hyperlink{15_Chapter_Eight__RELIGIOUS_EXCITAT.xhtmlux5cux23id_843}{32}}}.
Ibid., p. 910ff.

\protect\hypertarget{23_NOTES.xhtmlux5cux23id_842}{\protect\hyperlink{15_Chapter_Eight__RELIGIOUS_EXCITAT.xhtmlux5cux23id_841}{33}}.
Gerson, De probatione spirituum, Opera, I, p. 41.

\protect\hypertarget{23_NOTES.xhtmlux5cux23id_840}{\protect\hyperlink{15_Chapter_Eight__RELIGIOUS_EXCITAT.xhtmlux5cux23id_839}{34}}.
Gerson, Epistola contra libellum Joh. de Schonhavia (polemics over
Ruusbroec), Opera, I, p. 82.

\protect\hypertarget{23_NOTES.xhtmlux5cux23id_838}{\protect\hyperlink{15_Chapter_Eight__RELIGIOUS_EXCITAT.xhtmlux5cux23id_837}{35}}.
Gerson, Sermo contra luxuriem, Opera, III, p. 924.

\protect\hypertarget{23_NOTES.xhtmlux5cux23id_836}{\protect\hyperlink{15_Chapter_Eight__RELIGIOUS_EXCITAT.xhtmlux5cux23id_835}{36}}.
Gerson, De distinctione etc., Opera, I, p. 55.

\protect\hypertarget{23_NOTES.xhtmlux5cux23id_834}{\protect\hyperlink{15_Chapter_Eight__RELIGIOUS_EXCITAT.xhtmlux5cux23id_833}{37}}.
Opera, III, pp. 589ff.

\protect\hypertarget{23_NOTES.xhtmlux5cux23id_832}{\protect\hyperlink{15_Chapter_Eight__RELIGIOUS_EXCITAT.xhtmlux5cux23id_831}{38}}.
Ibid., p. 593.

\protect\hypertarget{23_NOTES.xhtmlux5cux23id_830}{\protect\hyperlink{15_Chapter_Eight__RELIGIOUS_EXCITAT.xhtmlux5cux23id_829}{39}}.
Gerson, De consolatione theologiae, Opera, I, p. 174.

\protect\hypertarget{23_NOTES.xhtmlux5cux23id_828}{\protect\hyperlink{15_Chapter_Eight__RELIGIOUS_EXCITAT.xhtmlux5cux23id_827}{40}}.
{[}Trans.{]} \emph{Ruusbroec}: 1293--1381. Dutch mystic, the teacher of
Groote. Unlike many other mystics, Ruusbroec did not teach the the soul
was extinguished in God at the highest ecstasy, but that it retained its
identity.

\protect\hypertarget{23_NOTES.xhtmlux5cux23id_826}{\protect\hyperlink{15_Chapter_Eight__RELIGIOUS_EXCITAT.xhtmlux5cux23id_825}{41}}.
Gerson, Epistola .~.~. super tertia parte libri Johannis Ruysbroeck, De
ornatu nupt. spir., Opera, I, pp. 59, 67 \emph{passim}.

\protect\hypertarget{23_NOTES.xhtmlux5cux23page_424}{\protect\hyperlink{15_Chapter_Eight__RELIGIOUS_EXCITAT.xhtmlux5cux23id_824}{42}}.
Gerson, Epistola contra libellum Joh. de Schonhavia, Opera, I, p. 82.

\protect\hypertarget{23_NOTES.xhtmlux5cux23id_823}{\protect\hyperlink{15_Chapter_Eight__RELIGIOUS_EXCITAT.xhtmlux5cux23id_822}{43}}.
The same feeling in a modern person: ``I committed myself to Him in the
profoundest belief that my individuality was going to be destroyed, that
he would take all from me, and that I was willing.'' James, Varieties of
religious experience, p. 223.

\protect\hypertarget{23_NOTES.xhtmlux5cux23id_821}{\protect\hyperlink{15_Chapter_Eight__RELIGIOUS_EXCITAT.xhtmlux5cux23id_820}{44}}.
Gerson, De distinctione etc., Opera, I, p. 55; De libris caute legendis,
Opera, I, p. 114.

\protect\hypertarget{23_NOTES.xhtmlux5cux23id_819}{\protect\hyperlink{15_Chapter_Eight__RELIGIOUS_EXCITAT.xhtmlux5cux23id_818}{45}}.
{[}Trans.{]} \emph{the mad love of God}: Huizinga here uses the poetic
form for love, \emph{min}.

\protect\hypertarget{23_NOTES.xhtmlux5cux23id_817}{\protect\hyperlink{15_Chapter_Eight__RELIGIOUS_EXCITAT.xhtmlux5cux23id_816}{46}}.
Gerson, De examinatione doctrinarum, Opera, I, p. 19; De distinctione,
I, p. 55; De libris caute legendis, I, p. 114; Epistola super Joh.
Ruysbroeck De ornatu, I, p. 62; De consolatione theologiae, I, p. 174;
De susceptione humanitatis Christi, I, p. 455; De nuptiis Christi et
ecclesiae, II, p. 370; De triplici theologia, III, p. 869.

\protect\hypertarget{23_NOTES.xhtmlux5cux23id_815}{\protect\hyperlink{15_Chapter_Eight__RELIGIOUS_EXCITAT.xhtmlux5cux23id_814}{47}}.
Moll, Johannes Brugman, I, p. 57.

\protect\hypertarget{23_NOTES.xhtmlux5cux23id_813}{\protect\hyperlink{15_Chapter_Eight__RELIGIOUS_EXCITAT.xhtmlux5cux23id_812}{48}}.
Gerson, De distinctione etc., I, p. 55.

\protect\hypertarget{23_NOTES.xhtmlux5cux23id_811}{\protect\hyperlink{15_Chapter_Eight__RELIGIOUS_EXCITAT.xhtmlux5cux23id_810}{49}}.
Moll, Brugman, I, pp. 234, 314.

\protect\hypertarget{23_NOTES.xhtmlux5cux23id_809}{\protect\hyperlink{15_Chapter_Eight__RELIGIOUS_EXCITAT.xhtmlux5cux23id_808}{50}}.
Ecclesiasticus 24: 29 {[}the English languages bibles: 24:21{]}; see
Meister Eckhart, Predigten no. 43, p. 146, par. 26.

\protect\hypertarget{23_NOTES.xhtmlux5cux23id_807}{\protect\hyperlink{15_Chapter_Eight__RELIGIOUS_EXCITAT.xhtmlux5cux23id_806}{51}}.
Ruusbroec, Die Spieghel der ewigher salicheit, cap. 7, Die chierheit der
gheesteleker brulocht, 1. II c 53, Werken, ed. David en Snellaert
(Maatsch. der Vlaemsche bibliophilen) 18602, 1868, III pp. 156--59, VI
p. 132.

\protect\hypertarget{23_NOTES.xhtmlux5cux23id_805}{\protect\hyperlink{15_Chapter_Eight__RELIGIOUS_EXCITAT.xhtmlux5cux23id_804}{52}}.
After the ms. in Oulmont, Le verger, le temple, et la cellule, p. 277.

\protect\hypertarget{23_NOTES.xhtmlux5cux23id_803}{\protect\hyperlink{15_Chapter_Eight__RELIGIOUS_EXCITAT.xhtmlux5cux23id_802}{53}}.
See the refutation of this opinion by James, Varieties of Religious
Experience, pp. 101, 191 276.

\protect\hypertarget{23_NOTES.xhtmlux5cux23id_801}{\protect\hyperlink{15_Chapter_Eight__RELIGIOUS_EXCITAT.xhtmlux5cux23id_800}{54}}.
Moll, Brugman, II, p. 84.

\protect\hypertarget{23_NOTES.xhtmlux5cux23id_799}{\protect\hyperlink{15_Chapter_Eight__RELIGIOUS_EXCITAT.xhtmlux5cux23id_798}{55}}.
Oulmont, Le verger, le temple, et la cellule, pp. 204, 210.

\protect\hypertarget{23_NOTES.xhtmlux5cux23id_797}{\protect\hyperlink{15_Chapter_Eight__RELIGIOUS_EXCITAT.xhtmlux5cux23id_796}{56}}.
B. Alanus redivivus, ed. J. A. Coppenstein, Neapel 1642, pp. 29, 31,
105, 108, \emph{116 passim}.

\protect\hypertarget{23_NOTES.xhtmlux5cux23id_795}{\protect\hyperlink{15_Chapter_Eight__RELIGIOUS_EXCITAT.xhtmlux5cux23id_794}{57}}.
Alanus redivivus, pp. 209, 218.

\protect\hypertarget{23_NOTES.xhtmlux5cux23id_793}{\protect\hyperlink{15_Chapter_Eight__RELIGIOUS_EXCITAT.xhtmlux5cux23id_792}{58}}.
{[}Trans.{]} \emph{The Hammer of Witches}: The most astonishing work of
pathological religious fanaticism to come out of the Middle Ages; it
prescribes techniques for the trials of accused witches that assure a
guilty verdict, yet the authors clearly believe in the validity of their
approach. It is available in a modern edition. H. Kramer and J.
Sprenger, \emph{The Malleus Maleficarum}. trans. M. Summers (New York:
Dover Publications, 1971). The translator, Montague Summers, was a
famous eccentric and his editorial comments in favor of the persecution
of witches can be ignored.

\textbf{\emph{Chapter 9}}

\protect\hypertarget{23_NOTES.xhtmlux5cux23id_2574}{\protect\hyperlink{16_Chapter_Nine__THE_DECLINE_OF_SYM.xhtmlux5cux23id_2573}{*\textsuperscript{1}}}
``In those times when speculation has become completely abstract,
defined concepts are easily in disaccord with profound intuitions.''

\protect\hypertarget{23_NOTES.xhtmlux5cux23id_2576}{\protect\hyperlink{16_Chapter_Nine__THE_DECLINE_OF_SYM.xhtmlux5cux23id_2575}{*\textsuperscript{2}}}
The slipper only gives us health/And all profit without serious
illness,/To give it a title to authority/I give it the name of humility.

\protect\hypertarget{23_NOTES.xhtmlux5cux23id_2578}{\protect\hyperlink{16_Chapter_Nine__THE_DECLINE_OF_SYM.xhtmlux5cux23id_2577}{*\textsuperscript{3}}}
``Then arose the goddess of Discord, who lived in the tower of Evil
Counsel, and Rage and Vengeance, and they took up arms of all sorts and
cast out Reason, Justice, Remembrance of God, and Moderation most
shamefully.''

\protect\hypertarget{23_NOTES.xhtmlux5cux23id_2580}{\protect\hyperlink{16_Chapter_Nine__THE_DECLINE_OF_SYM.xhtmlux5cux23id_2579}{†\textsuperscript{4}}}
``And since the time they were dead was the short time a man needs to go
a hundred paces, they had only their pants on, and they lay in heaps
like swine, covered with filth.''

\protect\hypertarget{23_NOTES.xhtmlux5cux23id_2582}{\protect\hyperlink{16_Chapter_Nine__THE_DECLINE_OF_SYM.xhtmlux5cux23id_2581}{*\textsuperscript{5}}}
``in the fashion of mummers, and to raise the mood in order to arouse
the greatest enjoyment.''

\protect\hypertarget{23_NOTES.xhtmlux5cux23id_791}{\protect\hyperlink{16_Chapter_Nine__THE_DECLINE_OF_SYM.xhtmlux5cux23id_790}{1}}.
Seuse, Leben, chap. 4, 45. Deutsche Schriften, S. 15, 154; Acta
sanctorum Jan. t. II, p. 656.

\protect\hypertarget{23_NOTES.xhtmlux5cux23id_789}{\protect\hyperlink{16_Chapter_Nine__THE_DECLINE_OF_SYM.xhtmlux5cux23id_788}{2}}.
Hefele, Der h. Bernhardin von Siena .~.~. , p. 167; see p. 259, ``Uber
den Namen Jesus,'' B's defense of the custom.

\protect\hypertarget{23_NOTES.xhtmlux5cux23id_787}{\protect\hyperlink{16_Chapter_Nine__THE_DECLINE_OF_SYM.xhtmlux5cux23id_786}{3}}.
Eug. Demole, Le soleil comme cimier des armes de Geneve, note in Revue
historique, CXXIII, p. 450.

\protect\hypertarget{23_NOTES.xhtmlux5cux23page_425}{\protect\hyperlink{16_Chapter_Nine__THE_DECLINE_OF_SYM.xhtmlux5cux23id_785}{4}}.
Rod. Hospinianus, De templis etc., ed. II a, Turgi, 1603, p. 213.

\protect\hypertarget{23_NOTES.xhtmlux5cux23id_784}{\protect\hyperlink{16_Chapter_Nine__THE_DECLINE_OF_SYM.xhtmlux5cux23id_783}{5}}.
{[}Trans.{]} \emph{Monstrance}: A container for holding the Host after
it is consecrated.

\protect\hypertarget{23_NOTES.xhtmlux5cux23id_782}{\protect\hyperlink{16_Chapter_Nine__THE_DECLINE_OF_SYM.xhtmlux5cux23id_781}{6}}.
James, Varieties of religious experience, pp. 474, 475.

\protect\hypertarget{23_NOTES.xhtmlux5cux23id_780}{\protect\hyperlink{16_Chapter_Nine__THE_DECLINE_OF_SYM.xhtmlux5cux23id_779}{7}}.
Irenaeus, Adversus haereses libri V, 1. IV c. 213.

\protect\hypertarget{23_NOTES.xhtmlux5cux23id_778}{\protect\hyperlink{16_Chapter_Nine__THE_DECLINE_OF_SYM.xhtmlux5cux23id_777}{8}}.
Concerning the necessity of such realism see James, Varieties of
religious experience, p. 56.

\protect\hypertarget{23_NOTES.xhtmlux5cux23id_776}{\protect\hyperlink{16_Chapter_Nine__THE_DECLINE_OF_SYM.xhtmlux5cux23id_775}{9}}.
{[}Trans.{]} \emph{Universals, Realism, Nominalism}: These complex
issues are much too involved to be adequately handled by a brief note.
However, the position of the church was that there existed in the world
(perhaps in the mind of God) universals of which all particulars are
imperfect examples. That is to say, that beauty existed as such
\emph{(ante rem)} fully apparent to God. Man, on the other hand, only
experiences particular expressions of beauty (in nature or art, for
instance). Necessarily, since man is fallen, man's experience of beauty
is incomplete. Nevertheless, since man's experience of beauty is truly a
part of universal beauty, God and man are tied together in a shared
reality. The church is the mediator between God and man in this reality.
This realist mentality logically gives credence to symbolic thought,
since in it any particular entity refers to the universal of which it is
a reflection. The nominalists, on the other hand, held that our only
knowledge is of particulars and that we generalize from our knowledge of
particulars to create a universal. After \emph{(post rem)} we have seen
enough particular examples of beauty, we form a general idea of beauty.
The problem with this, from the point of view of a centralized church,
is that God and man do not necessarily share the same reality and the
church is not necessarily the only means by which a person can attain
salvation. Huizinga's claim is that the nominalists still believe in
universals, although for them they are created by human thought rather
than existing \emph{a priori}. Since universals exist, symbolic modes of
thought have attraction and value for the nominalist as well as the
realist. For an elegant explanation of these issues see Steven Ozment,
\emph{The Age of Reform 1250--1550}. New Haven: Yale University Press,
1980.

\protect\hypertarget{23_NOTES.xhtmlux5cux23id_774}{\protect\hyperlink{16_Chapter_Nine__THE_DECLINE_OF_SYM.xhtmlux5cux23id_773}{10}}.
Goethe, Sprüche in Prosa, nos. 742, 743.

\protect\hypertarget{23_NOTES.xhtmlux5cux23id_772}{\protect\hyperlink{16_Chapter_Nine__THE_DECLINE_OF_SYM.xhtmlux5cux23id_771}{11}}.
St. Bernard, Libellus ad quendam sacerdotem, in Dion. Cart., De vita et
regimine curatorum, t. XXXVII, p. 222.

\protect\hypertarget{23_NOTES.xhtmlux5cux23id_770}{\protect\hyperlink{16_Chapter_Nine__THE_DECLINE_OF_SYM.xhtmlux5cux23id_769}{12}}.
Bonaventura, De reductione artium ad theologiam, Opera, ed. Paris, 1871,
t. VII, p. 502.

\protect\hypertarget{23_NOTES.xhtmlux5cux23id_768}{\protect\hyperlink{16_Chapter_Nine__THE_DECLINE_OF_SYM.xhtmlux5cux23id_767}{13}}.
P. Rousselot, Pour l'historie de probleme de l'amour (Bäumker und von
Hertling, Beitr zur Gesch. der Philosophie in Mittelalter, VI, 6),
Münster 1908.

\protect\hypertarget{23_NOTES.xhtmlux5cux23id_766}{\protect\hyperlink{16_Chapter_Nine__THE_DECLINE_OF_SYM.xhtmlux5cux23id_765}{14}}.
{[}Trans.{]} \emph{Eindigende} (German \emph{Ausgehenden)}: The use of
this term here and in similar places somewhat justifies the use of the
title \emph{Waning of the Middle Ages} in the previous translation of
the work.

\protect\hypertarget{23_NOTES.xhtmlux5cux23id_764}{\protect\hyperlink{16_Chapter_Nine__THE_DECLINE_OF_SYM.xhtmlux5cux23id_763}{15}}.
Sicard, Mitrale sive de officiis ecclesiasticis summa, Migne, t. CCXIII,
c. 232.

\protect\hypertarget{23_NOTES.xhtmlux5cux23id_762}{\protect\hyperlink{16_Chapter_Nine__THE_DECLINE_OF_SYM.xhtmlux5cux23id_761}{16}}.
Gerson, Compendium Theologiae, Opera, I, pp. 234, 303f., 325; Meditatio
super septimo psalmo poenitentiali, IV, p. 26.

\protect\hypertarget{23_NOTES.xhtmlux5cux23id_760}{\protect\hyperlink{16_Chapter_Nine__THE_DECLINE_OF_SYM.xhtmlux5cux23id_759}{17}}.
Alanus redivivus, passim.

\protect\hypertarget{23_NOTES.xhtmlux5cux23id_758}{\protect\hyperlink{16_Chapter_Nine__THE_DECLINE_OF_SYM.xhtmlux5cux23id_757}{18}}.
On page 12 Fortitudo is equated with Abstinentia, however on page 201 it
is Temperantia that falls into the place. There are still other
variations.

\protect\hypertarget{23_NOTES.xhtmlux5cux23page_426}{\protect\hyperlink{16_Chapter_Nine__THE_DECLINE_OF_SYM.xhtmlux5cux23id_756}{19}}.
Froissart, Poésies, ed. Scheler, I, p. 53.

\protect\hypertarget{23_NOTES.xhtmlux5cux23id_755}{\protect\hyperlink{16_Chapter_Nine__THE_DECLINE_OF_SYM.xhtmlux5cux23id_754}{20}}.
Chastellain, Traité par forme d'allégorie mystique sur l'entrée du roy
Loys en nouveau règne, Oeuvres, VII, p. 1; Molinet, II, p. 71, III, p.
112.

\protect\hypertarget{23_NOTES.xhtmlux5cux23id_753}{\protect\hyperlink{16_Chapter_Nine__THE_DECLINE_OF_SYM.xhtmlux5cux23id_752}{21}}.
See Coquillart, Les droits nouveaux, ed. d'Héricault, I, p. 72.

\protect\hypertarget{23_NOTES.xhtmlux5cux23id_751}{\protect\hyperlink{16_Chapter_Nine__THE_DECLINE_OF_SYM.xhtmlux5cux23id_750}{22}}.
Opera, I, p. xliv ff.

\protect\hypertarget{23_NOTES.xhtmlux5cux23id_749}{\protect\hyperlink{16_Chapter_Nine__THE_DECLINE_OF_SYM.xhtmlux5cux23id_748}{23}}.
H. Usener, Götternamen, Versuch zu einer Lehre von der religïsen
Begriffsbildung, Bonn 1896, p. 73.

\protect\hypertarget{23_NOTES.xhtmlux5cux23id_747}{\protect\hyperlink{16_Chapter_Nine__THE_DECLINE_OF_SYM.xhtmlux5cux23id_746}{24}}.
J. Mangeart, Catalogue des mss. de la bibl. de Valenciennes, 1860, p.
687.

\protect\hypertarget{23_NOTES.xhtmlux5cux23id_745}{\protect\hyperlink{16_Chapter_Nine__THE_DECLINE_OF_SYM.xhtmlux5cux23id_744}{25}}.
Journal d'un bourgeois, p. 96.

\protect\hypertarget{23_NOTES.xhtmlux5cux23id_743}{\protect\hyperlink{16_Chapter_Nine__THE_DECLINE_OF_SYM.xhtmlux5cux23id_742}{26}}.
La Marche, II, p. 378.

\protect\hypertarget{23_NOTES.xhtmlux5cux23id_741}{\protect\hyperlink{16_Chapter_Nine__THE_DECLINE_OF_SYM.xhtmlux5cux23id_740}{27}}.
Histoire littéraire de la France (XlVe siecle), t. XXIV, 1862, p. 541;
Grôbers Grundriss, II, 1, p. 877, II, 2, p. 406; see les Cent nouvelles
nouvelles, II, p. 183, Rabelais, Pantagruel, 1, IV, chap. 29.

\protect\hypertarget{23_NOTES.xhtmlux5cux23id_739}{\protect\hyperlink{16_Chapter_Nine__THE_DECLINE_OF_SYM.xhtmlux5cux23id_738}{28}}.
H. Grotefend, Korrespondenzblatt des Gesamtvereins etc., 67, 1919, p.
124, Dock = doll.

\protect\hypertarget{23_NOTES.xhtmlux5cux23id_737}{\protect\hyperlink{16_Chapter_Nine__THE_DECLINE_OF_SYM.xhtmlux5cux23id_736}{29}}.
De captivitate babylonica ecclesiae praeludium, Weimarer Ausgabe, VI, p.
562.

\textbf{\emph{Chapter 10}}

\protect\hypertarget{23_NOTES.xhtmlux5cux23id_2584}{\protect\hyperlink{17_Chapter_Ten__THE_FAILURE_OF_IMAG.xhtmlux5cux23id_2583}{*\textsuperscript{1}}}
``I did not kiss the man, but the precious mouth whence have issued and
gone forth so many good words and virtuous sayings.''

\protect\hypertarget{23_NOTES.xhtmlux5cux23id_2586}{\protect\hyperlink{17_Chapter_Ten__THE_FAILURE_OF_IMAG.xhtmlux5cux23id_2585}{†\textsuperscript{2}}}
``It may be that I err in my faith, but I will not be a heretic.''

\protect\hypertarget{23_NOTES.xhtmlux5cux23id_2588}{\protect\hyperlink{17_Chapter_Ten__THE_FAILURE_OF_IMAG.xhtmlux5cux23id_2587}{‡\textsuperscript{3}}}
``antecedent will''

\protect\hypertarget{23_NOTES.xhtmlux5cux23id_2590}{\protect\hyperlink{17_Chapter_Ten__THE_FAILURE_OF_IMAG.xhtmlux5cux23id_2589}{§\textsuperscript{4}}}
``consequent will''

\protect\hypertarget{23_NOTES.xhtmlux5cux23id_2330}{\protect\hyperlink{17_Chapter_Ten__THE_FAILURE_OF_IMAG.xhtmlux5cux23id_2329}{*\textsuperscript{5}}}
\emph{On Contempt of the World}

\protect\hypertarget{23_NOTES.xhtmlux5cux23id_2332}{\protect\hyperlink{17_Chapter_Ten__THE_FAILURE_OF_IMAG.xhtmlux5cux23id_2331}{†\textsuperscript{6}}}
``made from the dirtiest semen, conceived in the titillation of the
flesh, nourished with menstrual blood so that it is so loathsome and
impure that fruit will not grow and plants will wither when touched by
it .~.~. if dogs eat it, they will go mad.''

\protect\hypertarget{23_NOTES.xhtmlux5cux23id_2592}{\protect\hyperlink{17_Chapter_Ten__THE_FAILURE_OF_IMAG.xhtmlux5cux23id_2591}{*\textsuperscript{7}}}
Pious Pelican, Lord Jesus,/Cleanse me, an impure one, by your blood,/O
which one drop can save/the whole world from iniquity.

\protect\hypertarget{23_NOTES.xhtmlux5cux23id_2594}{\protect\hyperlink{17_Chapter_Ten__THE_FAILURE_OF_IMAG.xhtmlux5cux23id_2593}{*\textsuperscript{8}}}
``Trinity super-substantial, super-adorable and super-good .~.~. lead us
to the super-bright contemplation of Thyself.'' {[}The Lord is{]}
``super-merciful, super-dignified, super-kind, super-radiant,
super-omnipotent and super-wise, superglorious.''

\protect\hypertarget{23_NOTES.xhtmlux5cux23id_2596}{\protect\hyperlink{17_Chapter_Ten__THE_FAILURE_OF_IMAG.xhtmlux5cux23id_2595}{*\textsuperscript{9}}}
``Well be, heart and mind and soul in the bottomless abyss of all lovely
things.''

\protect\hypertarget{23_NOTES.xhtmlux5cux23id_2598}{\protect\hyperlink{17_Chapter_Ten__THE_FAILURE_OF_IMAG.xhtmlux5cux23id_2597}{†\textsuperscript{10}}}
``This spark .~.~. is not satisfied either with the Father nor with the
Son nor yet with the Holy Ghost, nor with the Trinity itself, in so far
as each one exists in its own being. Indeed, I affirm: this light is not
even satisfied when the divine nature is born in him as a generative
fruit. I will say one thing more that will sound even stranger: I
maintain in all seriousness that this light is also not satisfied with
the unified divine being, resting in itself, which neither gives nor
receives: rather, it will know whence this being comes; it wants to
enter the simple ground, the silent desert; into which never anything
distinct has ever been seen; not-Father, not-Son, not--Holy Ghost; in
the innermost where no one is at home, only there is this light
satisfied, and it belongs to it more fervently than to itself. Because
this ground is a simple (bare of all particulars) silence which rests in
itself'' .~.~. ``entering into the empty deity, where there is neither
work nor image; that it can lose itself there and immerse itself into
the wilderness.''

\protect\hypertarget{23_NOTES.xhtmlux5cux23id_2600}{\protect\hyperlink{17_Chapter_Ten__THE_FAILURE_OF_IMAG.xhtmlux5cux23id_2599}{*\textsuperscript{11}}}
Let them shout with open heart:/O tremendous abyss!/Entirely without
mouth,/Lead us into your abyss/And make known thy love to us.

\protect\hypertarget{23_NOTES.xhtmlux5cux23id_2602}{\protect\hyperlink{17_Chapter_Ten__THE_FAILURE_OF_IMAG.xhtmlux5cux23id_2601}{*\textsuperscript{12}}}
``And here it dies its highest death. In this death the soul loses all
craving and all images and all power of comprehension and all form and
is deprived of all being. And you can be sure as God lives: as little as
a dead man, who is physically dead, can move, as little can a soul,
which is spiritually dead like that, reveal any mode or any image to any
man. Because this spirit is dead and buried in the Deity.''

\protect\hypertarget{23_NOTES.xhtmlux5cux23id_2604}{\protect\hyperlink{17_Chapter_Ten__THE_FAILURE_OF_IMAG.xhtmlux5cux23id_2603}{*\textsuperscript{13}}}
``She soared high above him in a clouded sky; she was bright as the
morning star and shone like the radiant sun; her crown was eternity, her
dress was bliss, her words sweetness, her embrace satisfied all lust;
she was far and near, high and low; she was present and yet hidden; she
could be approached, but no one could hold her.''

\protect\hypertarget{23_NOTES.xhtmlux5cux23id_2606}{\protect\hyperlink{17_Chapter_Ten__THE_FAILURE_OF_IMAG.xhtmlux5cux23id_2605}{†\textsuperscript{14}}}
``All creatures are a pure nothing. I don't say they are small or they
are something, they are a pure nothing. What has no being, that is
nothing. All creatures have no being, because their being soars in the
presence of God.''

\protect\hypertarget{23_NOTES.xhtmlux5cux23id_735}{\protect\hyperlink{17_Chapter_Ten__THE_FAILURE_OF_IMAG.xhtmlux5cux23id_734}{1}}.
Petri de Alliaco, Tractatus I, adversus cancellarium Parisiensem, in
Gerson, Opera, I, p. 723.

\protect\hypertarget{23_NOTES.xhtmlux5cux23id_733}{\protect\hyperlink{17_Chapter_Ten__THE_FAILURE_OF_IMAG.xhtmlux5cux23id_732}{2}}.
Dion. Cart., Opera, t. XXXVI, p. 200.

\protect\hypertarget{23_NOTES.xhtmlux5cux23id_731}{\protect\hyperlink{17_Chapter_Ten__THE_FAILURE_OF_IMAG.xhtmlux5cux23id_730}{3}}.
Dion. Cart. Revelatio II, Opera, I, p. xiv.

\protect\hypertarget{23_NOTES.xhtmlux5cux23id_729}{\protect\hyperlink{17_Chapter_Ten__THE_FAILURE_OF_IMAG.xhtmlux5cux23id_728}{4}}.
Dion. Cart., Opera, t. XXXVII, XXXVIII, XXXIX, p. 496.

\protect\hypertarget{23_NOTES.xhtmlux5cux23id_727}{\protect\hyperlink{17_Chapter_Ten__THE_FAILURE_OF_IMAG.xhtmlux5cux23id_726}{5}}.
{[}Trans.{]} \emph{Lamprecht}: Along with Burckhardt, whose great
\emph{Kultur der Renaissance in Italien} has important interactions with
\emph{Autumn}, Lamprecht was a significant figure in the development of
historiography preceding Huizinga. For a full treatment of Burckhardt,
Lamprecht, and others see Karl J. Weintraub, \emph{Visions of Culture}.
Chicago: The University of Chicago Press, 1969.

\protect\hypertarget{23_NOTES.xhtmlux5cux23id_725}{\protect\hyperlink{17_Chapter_Ten__THE_FAILURE_OF_IMAG.xhtmlux5cux23id_724}{6}}.
Alain Chartier, Oeuvres, p. xi.

\protect\hypertarget{23_NOTES.xhtmlux5cux23id_723}{\protect\hyperlink{17_Chapter_Ten__THE_FAILURE_OF_IMAG.xhtmlux5cux23id_722}{7}}.
Gerson, Opera, I, p. 17.

\protect\hypertarget{23_NOTES.xhtmlux5cux23id_721}{\protect\hyperlink{17_Chapter_Ten__THE_FAILURE_OF_IMAG.xhtmlux5cux23id_720}{8}}.
Dion. Cart., Opera, t. XVIII, p. 433.

\protect\hypertarget{23_NOTES.xhtmlux5cux23id_719}{\protect\hyperlink{17_Chapter_Ten__THE_FAILURE_OF_IMAG.xhtmlux5cux23id_718}{9}}.
Dion. Cart., Opera, XXXIX, p. 18 ff., De vitiis et virtutibus, p.
\emph{363}; De gravitate et enormitate peccati, ibid., t. XXIX, p. 50.

\protect\hypertarget{23_NOTES.xhtmlux5cux23id_717}{\protect\hyperlink{17_Chapter_Ten__THE_FAILURE_OF_IMAG.xhtmlux5cux23id_716}{10}}.
Dion. Cart., Opera, XXXIX, p. 37.

\protect\hypertarget{23_NOTES.xhtmlux5cux23id_715}{\protect\hyperlink{17_Chapter_Ten__THE_FAILURE_OF_IMAG.xhtmlux5cux23id_714}{11}}.
Ibid. p. 56.

\protect\hypertarget{23_NOTES.xhtmlux5cux23id_713}{\protect\hyperlink{17_Chapter_Ten__THE_FAILURE_OF_IMAG.xhtmlux5cux23id_712}{12}}.
Dion Cart., De quatuor hominum novissimis, Opera, t. XLI, p. 545.

\protect\hypertarget{23_NOTES.xhtmlux5cux23id_711}{\protect\hyperlink{17_Chapter_Ten__THE_FAILURE_OF_IMAG.xhtmlux5cux23id_710}{13}}.
Dion. Cart., De quatuor hominum novissimis, t. XLI, pp. 489ff.

\protect\hypertarget{23_NOTES.xhtmlux5cux23id_709}{\protect\hyperlink{17_Chapter_Ten__THE_FAILURE_OF_IMAG.xhtmlux5cux23id_708}{14}}.
Moll, Brugman, I, pp. 20, 23, 28.

\protect\hypertarget{23_NOTES.xhtmlux5cux23id_707}{\protect\hyperlink{17_Chapter_Ten__THE_FAILURE_OF_IMAG.xhtmlux5cux23id_706}{15}}.
Moll, Brugman, I., p. 3201.

\protect\hypertarget{23_NOTES.xhtmlux5cux23id_705}{\protect\hyperlink{17_Chapter_Ten__THE_FAILURE_OF_IMAG.xhtmlux5cux23id_704}{16}}.
The example of St. Aegidius, Germanus, Quiricus in Gerson, De via
imitativa, III, p. 777; see Contra gulam sermo, ibid., p. 909.---Olivier
Maillard, Serm. de sanctis fol. 8a.

\protect\hypertarget{23_NOTES.xhtmlux5cux23id_703}{\protect\hyperlink{17_Chapter_Ten__THE_FAILURE_OF_IMAG.xhtmlux5cux23id_702}{17}}.
{[}Trans.{]} \emph{thesaurus ecclesiae}. the treasury of the church. The
doctrine that Christ's sacrifice on the cross and the works of the
saints are a source of merit (grace) from which all can draw. Here it is
explained in a modern handbook:

\protect\hypertarget{23_NOTES.xhtmlux5cux23page_427}{}{}The lives of the
saints were immensely fruitful. By their constant attention to the
interior voice of God, and by their obedience to His will and to His
Church, they became dear to Him; they knew that the best way to serve
Him lay in their service to others, and thus they offered their lives as
a supplication and a reparation for all those still laden with the
burden of the punishment due to their sins. Through their good works
they grew continuously in the love of God, but their expiation greatly
exceeded that which was required for their own shortcomings. Recognizing
their desire to share their spiritual wealth with others, the Church
uses this overflow of the merits of her saints---joined to the infinite
merit of Christ Himself---as an offering to God by which the balance of
reparation due to His justice may be paid by their sacrificial love (N.
G. M. Van Doornik, S. Jelsma, and A. Van de Lisdonk, \emph{A Handbook of
the Catholic Faith}, ed. John Greenwood, New York: Image Books, 1956,
pp. 290--91).

\protect\hypertarget{23_NOTES.xhtmlux5cux23id_701}{\protect\hyperlink{17_Chapter_Ten__THE_FAILURE_OF_IMAG.xhtmlux5cux23id_700}{18}}.
Innocentius III, De contemptu mundi 1. I, c. 1, Migne, t. CCXVII, pp.
702ff.

\protect\hypertarget{23_NOTES.xhtmlux5cux23id_699}{\protect\hyperlink{17_Chapter_Ten__THE_FAILURE_OF_IMAG.xhtmlux5cux23id_698}{19}}.
Wetzer und Welte, Kirchenlexikon, XI, 1601, Freiburg im Breisgau,
Herder, 1882--1903.

\protect\hypertarget{23_NOTES.xhtmlux5cux23id_697}{\protect\hyperlink{17_Chapter_Ten__THE_FAILURE_OF_IMAG.xhtmlux5cux23id_696}{20}}.
Extravag. commun. lib. V, tit. IX, cap. 2---``Quanto plures ex eius
applicatione trahuntur ad iustitiam, tanto magis accrescit ipsorum
cumulus meritorum.''

\protect\hypertarget{23_NOTES.xhtmlux5cux23id_695}{\protect\hyperlink{17_Chapter_Ten__THE_FAILURE_OF_IMAG.xhtmlux5cux23id_694}{21}}.
Bonaventura, In secundum librum sententiarum, dist. 41, art. 1, qu. 2;
ibid. 30, 2, 1, 34; in quart, lib. sent. d. 34, a. 1, qu. 2, Breviloquii
pars II, Opera, ed. Paris, 1871, t. III, pp. 577a, 335, 438, VI, p.
327b, VII, p. 271ab.

\protect\hypertarget{23_NOTES.xhtmlux5cux23id_693}{\protect\hyperlink{17_Chapter_Ten__THE_FAILURE_OF_IMAG.xhtmlux5cux23id_692}{22}}.
Dion. Cart., De vitiis et virtutibus, Opera, t. XXXIX, p. 20.

\protect\hypertarget{23_NOTES.xhtmlux5cux23id_691}{\protect\hyperlink{17_Chapter_Ten__THE_FAILURE_OF_IMAG.xhtmlux5cux23id_690}{23}}.
McKechnie, William Sharp, Magna Carta, p. 401, Glasgow, J. Maclehose and
Sons, 1905.

\protect\hypertarget{23_NOTES.xhtmlux5cux23id_689}{\protect\hyperlink{17_Chapter_Ten__THE_FAILURE_OF_IMAG.xhtmlux5cux23id_688}{24}}.
From the hymn ``Adore te devoto.'' The same thought is in the earlier
mentioned Bull Unigenitus. See Marlow, Faustus: ``See, where Christ's
blood streams in the firmament! One drop of blood will save me.''

\protect\hypertarget{23_NOTES.xhtmlux5cux23id_687}{\protect\hyperlink{17_Chapter_Ten__THE_FAILURE_OF_IMAG.xhtmlux5cux23id_686}{25}}.
Dion. Cart., Dialogion de fide cath., Opera, t. XVIII, p. 366.

\emph{\protect\hypertarget{23_NOTES.xhtmlux5cux23id_685}{\protect\hyperlink{17_Chapter_Ten__THE_FAILURE_OF_IMAG.xhtmlux5cux23id_684}{26}}}.
Dion. Cart., Dialogion de fide cath., t. XLI, p. 489.

\protect\hypertarget{23_NOTES.xhtmlux5cux23id_683}{\protect\hyperlink{17_Chapter_Ten__THE_FAILURE_OF_IMAG.xhtmlux5cux23id_682}{27}}.
Dion. Cart., De laudibus sanctae et individuae trinitatis, t. XXXV, p.
137; de laud glor. Virg. Mariae and passim. He borrowed the use of the
super-terms from Dionysius Areopagita.

\protect\hypertarget{23_NOTES.xhtmlux5cux23id_681}{\protect\hyperlink{17_Chapter_Ten__THE_FAILURE_OF_IMAG.xhtmlux5cux23id_680}{28}}.
James, Varieties of religious experience, p. 419.

\protect\hypertarget{23_NOTES.xhtmlux5cux23id_679}{\protect\hyperlink{17_Chapter_Ten__THE_FAILURE_OF_IMAG.xhtmlux5cux23id_678}{29}}.
Joannis Scoti, De divisione naturae, 1. III c. 19, Migne, Patr. latina,
t. CXXII, p. 681.

\protect\hypertarget{23_NOTES.xhtmlux5cux23id_677}{\protect\hyperlink{17_Chapter_Ten__THE_FAILURE_OF_IMAG.xhtmlux5cux23id_676}{30}}.
Angelus Silesius, Cherubinischer Wandersmann, I, 25, Halle a. S., M.
Niemeyer, 1895.

\protect\hypertarget{23_NOTES.xhtmlux5cux23id_675}{\protect\hyperlink{17_Chapter_Ten__THE_FAILURE_OF_IMAG.xhtmlux5cux23id_674}{31}}.
Opera, I, p. xliv.

\protect\hypertarget{23_NOTES.xhtmlux5cux23id_673}{\protect\hyperlink{17_Chapter_Ten__THE_FAILURE_OF_IMAG.xhtmlux5cux23id_672}{32}}.
Seuse, Leben, cap. 3, ed. K. Bihlmeyer, Deutsche Schriften, Stuttgard
1907, p. 14. See cap. 5, p. 21, 1. 3 and below.

\protect\hypertarget{23_NOTES.xhtmlux5cux23id_671}{\protect\hyperlink{17_Chapter_Ten__THE_FAILURE_OF_IMAG.xhtmlux5cux23id_670}{33}}.
Meister Eckhart, Predigten, nos. 60 and 76, ed. F. Pfeiffer, Deutsche
Mystiker des XIV. Jahrh., Leipzig 1857, II, p. 193, 11. 34ff; p. 242,
11. 2ff.

\protect\hypertarget{23_NOTES.xhtmlux5cux23page_428}{\protect\hyperlink{17_Chapter_Ten__THE_FAILURE_OF_IMAG.xhtmlux5cux23id_669}{34}}.
Tauler, Predigten, no. 28, ed. F. Vettor Deutsche Texte des
Mittelalters, XI), Berlin 1910, p. 117, 11. 3off.

\protect\hypertarget{23_NOTES.xhtmlux5cux23id_668}{\protect\hyperlink{17_Chapter_Ten__THE_FAILURE_OF_IMAG.xhtmlux5cux23id_667}{35}}.
Ruusbroec, Dat boec van seven sloten, cap. 19, Werken, ed. David, IV,
pp. 106--8.

\protect\hypertarget{23_NOTES.xhtmlux5cux23id_666}{\protect\hyperlink{17_Chapter_Ten__THE_FAILURE_OF_IMAG.xhtmlux5cux23id_665}{36}}.
Ruusbroec, Dat boec van den rike de ghelieven, cap. 43, ed. David, IV,
p. 264.

\protect\hypertarget{23_NOTES.xhtmlux5cux23id_664}{\protect\hyperlink{17_Chapter_Ten__THE_FAILURE_OF_IMAG.xhtmlux5cux23id_663}{37}}.
Ibid., cap. 35, p. 246.

\protect\hypertarget{23_NOTES.xhtmlux5cux23id_662}{\protect\hyperlink{17_Chapter_Ten__THE_FAILURE_OF_IMAG.xhtmlux5cux23id_661}{38}}.
Ruusbroec, Van seven trappen in den graet der gheesteliker minnen, cap.
14, ed. David, IV, p. 53. For \emph{ontfonken} I read \emph{ontsonken}.

\protect\hypertarget{23_NOTES.xhtmlux5cux23id_660}{\protect\hyperlink{17_Chapter_Ten__THE_FAILURE_OF_IMAG.xhtmlux5cux23id_659}{39}}.
Ruusbroec, Boec vna der hoechster waerheit, ed. David, p. 263; see
Spieghel der ewigher salicheit, cap. 25, p. 231.

\protect\hypertarget{23_NOTES.xhtmlux5cux23id_658}{\protect\hyperlink{17_Chapter_Ten__THE_FAILURE_OF_IMAG.xhtmlux5cux23id_657}{40}}.
Spieghel der ewigher salicheit, cap. 19, p. 144, cap. 23, p. 227.

\protect\hypertarget{23_NOTES.xhtmlux5cux23id_656}{\protect\hyperlink{17_Chapter_Ten__THE_FAILURE_OF_IMAG.xhtmlux5cux23id_655}{41}}.
II, Par. 6, 1: Dominus pollicitus est, ut habitaret in caligine. Ps 17,
13: Et posuit tenebras latibulum suum. {[}Trans. Huizinga's references
are to the Vulgate (Latin) Bible. In English Bibles this would be II
Chronicles 6:1: The Lord has said that he would dwell in thick darkness.
Psalm 18:11: He made darkness his secret place.{]}

\protect\hypertarget{23_NOTES.xhtmlux5cux23id_654}{\protect\hyperlink{17_Chapter_Ten__THE_FAILURE_OF_IMAG.xhtmlux5cux23id_653}{42}}.
Dion. Cart., De laudibus sanctae et individuae trinijtatis per modum
horarum, Opera, t. XXXV, pp. 137--38, id. XLI, p. 263 etc.; see De
passione dni salvatoris dialogus, t. XXXV, p. 274: ``ingrediendo
caliginem hoc est ad super-splendidissimae ac prorsus incomprehensibilis
Deitatis praefatam notitiam pertingendo per omnem negationem ab ea.''

\protect\hypertarget{23_NOTES.xhtmlux5cux23id_652}{\protect\hyperlink{17_Chapter_Ten__THE_FAILURE_OF_IMAG.xhtmlux5cux23id_651}{43}}.
Jostes, Meister Eckhart und seine Jünger, 1895, p. 95.

\protect\hypertarget{23_NOTES.xhtmlux5cux23id_650}{\protect\hyperlink{17_Chapter_Ten__THE_FAILURE_OF_IMAG.xhtmlux5cux23id_649}{44}}.
Dion. Cart., De contemplatione, lib. III, art. 5, Opera, t. XLI, p. 259.

\protect\hypertarget{23_NOTES.xhtmlux5cux23id_648}{\protect\hyperlink{17_Chapter_Ten__THE_FAILURE_OF_IMAG.xhtmlux5cux23id_647}{45}}.
Dion. Cart., De contemplatione, t. XLI, p. 269, after Dion. Areop.

\protect\hypertarget{23_NOTES.xhtmlux5cux23id_646}{\protect\hyperlink{17_Chapter_Ten__THE_FAILURE_OF_IMAG.xhtmlux5cux23id_645}{46}}.
Cankara ad Brahmasûtram, 3, 2, 17.

\protect\hypertarget{23_NOTES.xhtmlux5cux23id_644}{\protect\hyperlink{17_Chapter_Ten__THE_FAILURE_OF_IMAG.xhtmlux5cux23id_643}{47}}.
Chandogya-upanishad, 8.

\protect\hypertarget{23_NOTES.xhtmlux5cux23id_642}{\protect\hyperlink{17_Chapter_Ten__THE_FAILURE_OF_IMAG.xhtmlux5cux23id_641}{48}}.
Brhadâranyaka-upanishad, 4, 3, 21, 22.

\protect\hypertarget{23_NOTES.xhtmlux5cux23id_640}{\protect\hyperlink{17_Chapter_Ten__THE_FAILURE_OF_IMAG.xhtmlux5cux23id_639}{49}}.
Seuse, Leben, cap. 4, p. 14.

\protect\hypertarget{23_NOTES.xhtmlux5cux23id_638}{\protect\hyperlink{17_Chapter_Ten__THE_FAILURE_OF_IMAG.xhtmlux5cux23id_637}{50}}.
Eckhart, Predigten, no. 40, p. 136, par. 23.

\protect\hypertarget{23_NOTES.xhtmlux5cux23id_636}{\protect\hyperlink{17_Chapter_Ten__THE_FAILURE_OF_IMAG.xhtmlux5cux23id_635}{51}}.
Eckhart, Predigten, no. 9, pp. 47ff.

\protect\hypertarget{23_NOTES.xhtmlux5cux23id_634}{\protect\hyperlink{17_Chapter_Ten__THE_FAILURE_OF_IMAG.xhtmlux5cux23id_633}{52}}.
Thomas à Kempis, Soliloquium animae, Opera omnia, ed. M. J. Pohl,
Freiburg 1902--10, 7 vols., I, p. 230.

\protect\hypertarget{23_NOTES.xhtmlux5cux23id_632}{\protect\hyperlink{17_Chapter_Ten__THE_FAILURE_OF_IMAG.xhtmlux5cux23id_631}{53}}.
Thomas à Kempis, Soliloquium animae, p. 222.

\textbf{\emph{Chapter 11}}

\protect\hypertarget{23_NOTES.xhtmlux5cux23id_2334}{\protect\hyperlink{18_Chapter_Eleven__THE_FORMS_OF_THO.xhtmlux5cux23id_2333}{*\textsuperscript{15}}}
``and then I have heard it said by the ancients who knew .~.~. ''

\protect\hypertarget{23_NOTES.xhtmlux5cux23id_2336}{\protect\hyperlink{18_Chapter_Eleven__THE_FORMS_OF_THO.xhtmlux5cux23id_2335}{†\textsuperscript{16}}}
``at which the people mocked a good deal''

\protect\hypertarget{23_NOTES.xhtmlux5cux23id_2338}{\protect\hyperlink{18_Chapter_Eleven__THE_FORMS_OF_THO.xhtmlux5cux23id_2337}{‡\textsuperscript{17}}}
``But at present everyone does what he pleases: because of which we may
well be afraid that all will go badly.''

\protect\hypertarget{23_NOTES.xhtmlux5cux23id_2340}{\protect\hyperlink{18_Chapter_Eleven__THE_FORMS_OF_THO.xhtmlux5cux23id_2339}{§\textsuperscript{18}}}
``fruitmaster .~.~. wax department .~.~. so that this matter is very
well ordained thus.''

\protect\hypertarget{23_NOTES.xhtmlux5cux23id_2342}{\protect\hyperlink{18_Chapter_Eleven__THE_FORMS_OF_THO.xhtmlux5cux23id_2341}{**\textsuperscript{19}}}
``grand sergeanty''

\protect\hypertarget{23_NOTES.xhtmlux5cux23id_2607}{\protect\hyperlink{18_Chapter_Eleven__THE_FORMS_OF_THO.xhtmlux5cux23id_2608}{*\textsuperscript{20}}}
``Thus had good duke John drawn the moral inference of the case.''

\protect\hypertarget{23_NOTES.xhtmlux5cux23id_2610}{\protect\hyperlink{18_Chapter_Eleven__THE_FORMS_OF_THO.xhtmlux5cux23id_2609}{*\textsuperscript{21}}}
``I have a great deal of fear in my heart, truly fear so great that my
spirit and my memory have fled, and, that to make it worse, the little
understanding I had, has completely left me.''

\protect\hypertarget{23_NOTES.xhtmlux5cux23id_2612}{\protect\hyperlink{18_Chapter_Eleven__THE_FORMS_OF_THO.xhtmlux5cux23id_2611}{*\textsuperscript{22}}}
``I shall prove this truth by twelve reasons in honor of the twelve
apostles.''

\protect\hypertarget{23_NOTES.xhtmlux5cux23id_2344}{\protect\hyperlink{18_Chapter_Eleven__THE_FORMS_OF_THO.xhtmlux5cux23id_2343}{*\textsuperscript{23}}}
``I avouch you.''

\protect\hypertarget{23_NOTES.xhtmlux5cux23id_2346}{\protect\hyperlink{18_Chapter_Eleven__THE_FORMS_OF_THO.xhtmlux5cux23id_2345}{†\textsuperscript{24}}}
``The big fishes eat the smaller.'' ``The badly dressed are placed with
their back to the wind.'' ``None is chaste if it's not necessary .~.~.
`` ``Men are good so long as it saves their skin.'' ``At need we let the
devil help us.'' .~.~. ``There is no horse so well shod that he never
slips.''

\protect\hypertarget{23_NOTES.xhtmlux5cux23id_2614}{\protect\hyperlink{18_Chapter_Eleven__THE_FORMS_OF_THO.xhtmlux5cux23id_2613}{*\textsuperscript{25}}}
``He who is silent about all things is troubled by nothing.'' ``A
well-groomed head wears the helmet badly.'' ``You cut wide belts from
the skin of your neighbor.'' ``As the lord, so the servant.'' ``As the
judge, so the judgment.'' ``He who serves the common weal is paid by
none for his trouble.'' ``Those who have head sores should not take off
their hats.''

\protect\hypertarget{23_NOTES.xhtmlux5cux23id_2616}{\protect\hyperlink{18_Chapter_Eleven__THE_FORMS_OF_THO.xhtmlux5cux23id_2615}{†\textsuperscript{26}}}
``That's the way it is with fighting, sometimes you win, sometimes you
lose.'' ``Now, there is nothing which people won't eventually get tired
of.'' ``People say, and it's true, that there is nothing more certain
than death.''

\protect\hypertarget{23_NOTES.xhtmlux5cux23id_2618}{\protect\hyperlink{18_Chapter_Eleven__THE_FORMS_OF_THO.xhtmlux5cux23id_2617}{‡\textsuperscript{27}}}
``When will it be?'' ``Soon or late it may come.'' ``Onward.'' ``Better
next time.'' ``More sorrow than joy.''

\protect\hypertarget{23_NOTES.xhtmlux5cux23id_2620}{\protect\hyperlink{18_Chapter_Eleven__THE_FORMS_OF_THO.xhtmlux5cux23id_2619}{*\textsuperscript{28}}}
``I shall have no other.'' ``Your pleasure.'' ``Remember.'' ``More than
all.''

\protect\hypertarget{23_NOTES.xhtmlux5cux23id_2622}{\protect\hyperlink{18_Chapter_Eleven__THE_FORMS_OF_THO.xhtmlux5cux23id_2621}{†\textsuperscript{29}}}
``You have my heart.'' ``I desire it.'' ``Forever.'' ``All for you.''

\protect\hypertarget{23_NOTES.xhtmlux5cux23id_2624}{\protect\hyperlink{18_Chapter_Eleven__THE_FORMS_OF_THO.xhtmlux5cux23id_2623}{*\textsuperscript{30}}}
``and people did not know how one could bear the shame after the great
joy that had been displayed.''

\protect\hypertarget{23_NOTES.xhtmlux5cux23id_2626}{\protect\hyperlink{18_Chapter_Eleven__THE_FORMS_OF_THO.xhtmlux5cux23id_2625}{†\textsuperscript{31}}}
``banned for all his days.''

\emph{\protect\hypertarget{23_NOTES.xhtmlux5cux23id_2628}{\protect\hyperlink{18_Chapter_Eleven__THE_FORMS_OF_THO.xhtmlux5cux23id_2627}{\$\textsuperscript{32}}}}
``Good people, say your paternosters for the soul of the late Laurent
Guernier, in his life an inhabitant of Provins, who was lately found
dead under an oak tree.''

\protect\hypertarget{23_NOTES.xhtmlux5cux23id_2630}{\protect\hyperlink{18_Chapter_Eleven__THE_FORMS_OF_THO.xhtmlux5cux23id_2629}{*\textsuperscript{33}}}
``the children, the pretty scholars, like innocent lambs''

\protect\hypertarget{23_NOTES.xhtmlux5cux23id_2632}{\protect\hyperlink{18_Chapter_Eleven__THE_FORMS_OF_THO.xhtmlux5cux23id_2631}{*\textsuperscript{34}}}
I have seen an unknown thing:/A dead man revived,/And on his return/ Buy
for thousands. /The one says: He is alive, /The other: It is but wind.
/All good hearts, void of envy/regret his loss often.

\protect\hypertarget{23_NOTES.xhtmlux5cux23id_2348}{\protect\hyperlink{18_Chapter_Eleven__THE_FORMS_OF_THO.xhtmlux5cux23id_2347}{*\textsuperscript{35}}}
Free Will

\protect\hypertarget{23_NOTES.xhtmlux5cux23id_2634}{\protect\hyperlink{18_Chapter_Eleven__THE_FORMS_OF_THO.xhtmlux5cux23id_2633}{*\textsuperscript{36}}}
``agreeing with the names of common articles of clothing, instruments
and games of the present time such as Pantoufle, Courtaulx and
Mornifle.''

\protect\hypertarget{23_NOTES.xhtmlux5cux23id_2636}{\protect\hyperlink{18_Chapter_Eleven__THE_FORMS_OF_THO.xhtmlux5cux23id_2635}{*\textsuperscript{37}}}
``He could not voluntarily extirpate from his mind the aforesaid signs
and their effect against God .~.~. from this great folly, which is an
enemy to the Christian soul.''

\protect\hypertarget{23_NOTES.xhtmlux5cux23id_2638}{\protect\hyperlink{18_Chapter_Eleven__THE_FORMS_OF_THO.xhtmlux5cux23id_2637}{†\textsuperscript{38}}}
``wild Scotland.''

\protect\hypertarget{23_NOTES.xhtmlux5cux23id_2640}{\protect\hyperlink{18_Chapter_Eleven__THE_FORMS_OF_THO.xhtmlux5cux23id_2639}{*\textsuperscript{39}}}
``Do I not have before me the candle stubs, baptised by devilish means
and full of abominable mysteries against me and the others?''

\protect\hypertarget{23_NOTES.xhtmlux5cux23id_2642}{\protect\hyperlink{18_Chapter_Eleven__THE_FORMS_OF_THO.xhtmlux5cux23id_2641}{†\textsuperscript{40}}}
``because in all things he showed himself to be a man of correct and
complete trust in God, without having any need to know His secrets.''

\protect\hypertarget{23_NOTES.xhtmlux5cux23id_2644}{\protect\hyperlink{18_Chapter_Eleven__THE_FORMS_OF_THO.xhtmlux5cux23id_2643}{*\textsuperscript{41}}}
``And when someone argued with him, be it a cleric or a lay person, he
said that that one ought to be seized as suspected of the Waldensian
heresy.

\protect\hypertarget{23_NOTES.xhtmlux5cux23id_2646}{\protect\hyperlink{18_Chapter_Eleven__THE_FORMS_OF_THO.xhtmlux5cux23id_2645}{†\textsuperscript{42}}}
``such things were never before heard of happening in these countries.''

\protect\hypertarget{23_NOTES.xhtmlux5cux23id_2648}{\protect\hyperlink{18_Chapter_Eleven__THE_FORMS_OF_THO.xhtmlux5cux23id_2647}{‡\textsuperscript{43}}}
``a matter conjured by a few bad subjects.''

\protect\hypertarget{23_NOTES.xhtmlux5cux23id_2650}{\protect\hyperlink{18_Chapter_Eleven__THE_FORMS_OF_THO.xhtmlux5cux23id_2649}{*\textsuperscript{44}}}
There is no aged woman so stupid /Who has been guilty of committing the
least of these deeds,/But in order to have them burned or hanged,/The
enemy confuses human nature./He who knows how to set so many traps,/To
make the mind malicious./There are neither sticks nor rods/On which a
human could fly,/ But when the devil has confused/Their mind, they
believe that they fly/Some-where to indulge in pleasure/And to
accomplish their lust./Then they can be heard to speak of Rome,/Though
they have never been there./ .~.~. /The devils are all in hell, /Tied
up---says Franc-Vouloir---/And they will never get pliers or files/ To
get rid of their chains./How, then, should they be able to come/And play
so many tricks on the Christian Children/And to indulge in so many
lascivious adventures?/I can't understand your silliness.

\protect\hypertarget{23_NOTES.xhtmlux5cux23id_2652}{\protect\hyperlink{18_Chapter_Eleven__THE_FORMS_OF_THO.xhtmlux5cux23id_2651}{†\textsuperscript{45}}}
As long as I live, I shall not believe/That a woman can bodily /Travel
through the air like blackbird or thrush,/---Said the Champion
forthwith.---/Saint Augustine says plainly/That it is an illusion and
fantasy,/And others think it is nothing, /Also Gregory, Ambroise and
Jerome./When the poor woman lies in her bed, /In order to sleep and rest
there, /The enemy who never lies down to sleep/Comes and remains by her
side./Then to call up illusions/Before her he can so subtly,/That she
thinks she does or proposes to do/What she only dreams. / Perhaps the
gammer will dream/That on a cat or a dog/She will go to the meeting;/But
certainly nothing will happen;/And there is neither a stick nor a beam/
That could lift her a step.

\protect\hypertarget{23_NOTES.xhtmlux5cux23id_2654}{\protect\hyperlink{18_Chapter_Eleven__THE_FORMS_OF_THO.xhtmlux5cux23id_2653}{*\textsuperscript{46}}}
``which many ignorant people keep in a quiet place, and they have such
great faith in this manure, that they truly strongly believe that as
long as they keep it beautifully draped in silk or in linen, they will
never be poor.''

\protect\hypertarget{23_NOTES.xhtmlux5cux23id_630}{\protect\hyperlink{18_Chapter_Eleven__THE_FORMS_OF_THO.xhtmlux5cux23id_629}{1}}.
Alienor de Poitiers, Les honneurs de la cour, pp. 184, 189, 242, 266.

\protect\hypertarget{23_NOTES.xhtmlux5cux23id_628}{\protect\hyperlink{18_Chapter_Eleven__THE_FORMS_OF_THO.xhtmlux5cux23id_627}{2}}.
Olivier de la Marche, L'estat de la maison etc., t. IV, p. 56; see
similar questions above, p. 44.

\protect\hypertarget{23_NOTES.xhtmlux5cux23id_626}{\protect\hyperlink{18_Chapter_Eleven__THE_FORMS_OF_THO.xhtmlux5cux23id_625}{3}}.
J. H. Round, The king's Serjeants and officers of state with their
coronation services, London 1911, p. 41.

\protect\hypertarget{23_NOTES.xhtmlux5cux23id_624}{\protect\hyperlink{18_Chapter_Eleven__THE_FORMS_OF_THO.xhtmlux5cux23id_623}{4}}.
{[}Trans.{]} For instance, the cell of Maximillian in Bruges was called
the ``Broodhuis.'' See p. 299.

\protect\hypertarget{23_NOTES.xhtmlux5cux23id_622}{\protect\hyperlink{18_Chapter_Eleven__THE_FORMS_OF_THO.xhtmlux5cux23id_621}{5}}.
Le livre des trahisons, p. 27

\protect\hypertarget{23_NOTES.xhtmlux5cux23id_620}{\protect\hyperlink{18_Chapter_Eleven__THE_FORMS_OF_THO.xhtmlux5cux23id_619}{6}}.
Rel. de S. Denis, III, p. 464ff.; Juvenal des Ursins, p. 440; Noël
Valois, La France et le grand schisme d'occident, Paris, 1896--1902, 4
vols., III, p. 433.

\protect\hypertarget{23_NOTES.xhtmlux5cux23id_618}{\protect\hyperlink{18_Chapter_Eleven__THE_FORMS_OF_THO.xhtmlux5cux23id_617}{7}}.
Juvenal des Ursins, p. 342.

\protect\hypertarget{23_NOTES.xhtmlux5cux23page_429}{\protect\hyperlink{18_Chapter_Eleven__THE_FORMS_OF_THO.xhtmlux5cux23id_616}{8}}.
{[}Trans.{]} \emph{Athalia}: A Jewish figure who slew all the male heirs
to the throne so that she might gain the sucession. II Kings 22--23.

\protect\hypertarget{23_NOTES.xhtmlux5cux23id_615}{\protect\hyperlink{18_Chapter_Eleven__THE_FORMS_OF_THO.xhtmlux5cux23id_614}{9}}.
{[}Trans.{]} \emph{bal des ardents}: At a masquerade ball given by the
queen (Isabella of Bavaria) to celebrate the wedding of one of her
ladies, Charles VI (thirteen years old!) and a group of his playmates
dressed themselves as ``wood savages'' in costumes made of wax and hemp.
Knowing that these costumes were dangerously flammable, the king gave
orders that no flames were to be allowed in the ballroom, but Louis
d'Orléans entered with his retinue carrying torches. Louis himself held
the torch that set the revelers aflame. The king was saved by the
duchess de Berry (herself only fifteen), who put out his fire with her
dress. All save one of the king's companions died. See Tuchmann,
\emph{Mirror}.

\protect\hypertarget{23_NOTES.xhtmlux5cux23id_613}{\protect\hyperlink{18_Chapter_Eleven__THE_FORMS_OF_THO.xhtmlux5cux23id_612}{10}}.
{[}Trans.{]} Charles VI was a victim of frequent spells of insanity,
during which he relentlessly persecuted the queen.

\protect\hypertarget{23_NOTES.xhtmlux5cux23id_611}{\protect\hyperlink{18_Chapter_Eleven__THE_FORMS_OF_THO.xhtmlux5cux23id_610}{11}}.
Monstrelet, I, pp. 177--42; Coville, Le véritable texte de la
justification du duc de Bourgogne par Jean Petit (Bibl. de l'Ecole de
chartes), 1911, p. 57. For a draft of a second justification in which
Petit refutes the testimony that Abbot Thomas von Cerisi had given on
Sept 11, 1408; see O. Cartellieri, Beiträge zur Geschichte der Herzöge
von Burgund, V, Sitzungsbericht der Heidelberger Akademie der
Wissenschaften 1914, p. 6; further Wolfgang Seiferth, Der Tyrannenmord
von 1407, Leipziger Inaugural-Dissertation, 1922.

\protect\hypertarget{23_NOTES.xhtmlux5cux23id_609}{\protect\hyperlink{18_Chapter_Eleven__THE_FORMS_OF_THO.xhtmlux5cux23id_608}{12}}.
Leroux de Lincy, Le proverbe français, see E. E. Langois (Bibl. de
l'Ecole des chartes), LX, 1899, p. 569; J. Ulrich, Zeitschr. f. franz
Sprache u. Lit. XXIV, 1902, p. 191.

\protect\hypertarget{23_NOTES.xhtmlux5cux23id_607}{\protect\hyperlink{18_Chapter_Eleven__THE_FORMS_OF_THO.xhtmlux5cux23id_606}{13}}.
Les Grandes chroniques de France, ed. P. Paris, IV, p. 478.

\protect\hypertarget{23_NOTES.xhtmlux5cux23id_605}{\protect\hyperlink{18_Chapter_Eleven__THE_FORMS_OF_THO.xhtmlux5cux23id_604}{14}}.
Alain Chartier, ed. Duchesne, p. 717.

\protect\hypertarget{23_NOTES.xhtmlux5cux23id_603}{\protect\hyperlink{18_Chapter_Eleven__THE_FORMS_OF_THO.xhtmlux5cux23id_602}{15}}.
Jean Molinet, Faictz et Dictz, ed. Paris, 1537, fos. 80, 119, 152, 161,
170.

\protect\hypertarget{23_NOTES.xhtmlux5cux23id_601}{\protect\hyperlink{18_Chapter_Eleven__THE_FORMS_OF_THO.xhtmlux5cux23id_600}{16}}.
Coquillart, Oeuvres, I, p. 6.

\protect\hypertarget{23_NOTES.xhtmlux5cux23id_599}{\protect\hyperlink{18_Chapter_Eleven__THE_FORMS_OF_THO.xhtmlux5cux23id_598}{17}}.
Villon, ed. Longnon, p. 134.

\protect\hypertarget{23_NOTES.xhtmlux5cux23id_597}{\protect\hyperlink{18_Chapter_Eleven__THE_FORMS_OF_THO.xhtmlux5cux23id_596}{18}}.
Roberti Gaguini, Epistole et orationes., ed. Thuasne, II, p. 366.

\protect\hypertarget{23_NOTES.xhtmlux5cux23id_595}{\protect\hyperlink{18_Chapter_Eleven__THE_FORMS_OF_THO.xhtmlux5cux23id_594}{19}}.
Gerson, Opera, IV, p. 657; ibid. I, p. 936; Carnahan, The Ad Deum vadit
of Jean Gerson, pp. 61, 71; see Leroux de Lincy, Le proverbe français,
I, p. lii.

\protect\hypertarget{23_NOTES.xhtmlux5cux23id_593}{\protect\hyperlink{18_Chapter_Eleven__THE_FORMS_OF_THO.xhtmlux5cux23id_592}{20}}.
Geoffroi de Paris, ed. de Wailly et Delisle, Bouquet, Recueil des
Historiens des Gaules et de la France, XXII, p. 87, see index rerum et
personarum s.v. Proverbia, p. 926.

\protect\hypertarget{23_NOTES.xhtmlux5cux23id_591}{\protect\hyperlink{18_Chapter_Eleven__THE_FORMS_OF_THO.xhtmlux5cux23id_590}{21}}.
Froissart, ed. Luce, XI, p. 119; ed. Kervyn, XIII, p. 41, XIV, p. 33,
XV, p. 10; Lejouvencel, I, p. 60, 62, 63, 74, 78, 93.

\protect\hypertarget{23_NOTES.xhtmlux5cux23id_589}{\protect\hyperlink{18_Chapter_Eleven__THE_FORMS_OF_THO.xhtmlux5cux23id_588}{22}}.
``Je l'envie'' is a play on words with the meaning, I command you here,
I invite. ``Ic houd'' is the answer thereto: I accept. ``Cominus et
eminus'' is an allusion to the belief that the porcupine could also
shoot its quills.

\protect\hypertarget{23_NOTES.xhtmlux5cux23id_587}{\protect\hyperlink{18_Chapter_Eleven__THE_FORMS_OF_THO.xhtmlux5cux23id_586}{23}}.
See my ``Uit de voorgeschiedenis van ons nationaal besef,'' De Gids,
1912, I.

\protect\hypertarget{23_NOTES.xhtmlux5cux23id_585}{\protect\hyperlink{18_Chapter_Eleven__THE_FORMS_OF_THO.xhtmlux5cux23id_584}{24}}.
See above, p. 143.

\protect\hypertarget{23_NOTES.xhtmlux5cux23id_583}{\protect\hyperlink{18_Chapter_Eleven__THE_FORMS_OF_THO.xhtmlux5cux23id_582}{25}}.
A. Piaget, Le livre Messire Geoffroy de Charny, Romania XXVI, 1897, p.
396.

\protect\hypertarget{23_NOTES.xhtmlux5cux23id_581}{\protect\hyperlink{18_Chapter_Eleven__THE_FORMS_OF_THO.xhtmlux5cux23id_580}{26}}.
L'arbre des batailles, Paris, Michel le Noir 1515. See for Bonet,
Molinier, Sources de l'histoire de France, no. 3861.

\protect\hypertarget{23_NOTES.xhtmlux5cux23page_430}{\protect\hyperlink{18_Chapter_Eleven__THE_FORMS_OF_THO.xhtmlux5cux23id_579}{27}}.
Chap. 25, p. 85 bis (numbers 80--90 appear in the edition of 1515
twice), pp. 124--26.

\protect\hypertarget{23_NOTES.xhtmlux5cux23id_578}{\protect\hyperlink{18_Chapter_Eleven__THE_FORMS_OF_THO.xhtmlux5cux23id_577}{28}}.
Chaps. 56, 60, 84, 132.

\protect\hypertarget{23_NOTES.xhtmlux5cux23id_576}{\protect\hyperlink{18_Chapter_Eleven__THE_FORMS_OF_THO.xhtmlux5cux23id_575}{29}}.
Chaps. 82, 89, 80 bis and ff.

\protect\hypertarget{23_NOTES.xhtmlux5cux23id_574}{\protect\hyperlink{18_Chapter_Eleven__THE_FORMS_OF_THO.xhtmlux5cux23id_573}{30}}.
Lejouvencel, I, p. 222, II, p. 8, 93, 96, 133, 124.

\protect\hypertarget{23_NOTES.xhtmlux5cux23id_572}{\protect\hyperlink{18_Chapter_Eleven__THE_FORMS_OF_THO.xhtmlux5cux23id_571}{31}}.
Les vers de maitre Henri Baude, poete du XVe siecle, ed. Quicherat
(Trésor des pieces rares ou inédites), 1856, pp. 20--25.

\protect\hypertarget{23_NOTES.xhtmlux5cux23id_570}{\protect\hyperlink{18_Chapter_Eleven__THE_FORMS_OF_THO.xhtmlux5cux23id_569}{32}}.
Champion, Villon, II, p. 182.

\protect\hypertarget{23_NOTES.xhtmlux5cux23id_568}{\protect\hyperlink{18_Chapter_Eleven__THE_FORMS_OF_THO.xhtmlux5cux23id_567}{33}}.
Still stronger is the formalism of South American tribes who demand that
anyone who accidentally wounds himself must pay his clan blood money
because he has spilled the blood of the clan. L. Farrand, Basis of
American history, p. 198 (The American nation, A history, vol. II).

\protect\hypertarget{23_NOTES.xhtmlux5cux23id_566}{\protect\hyperlink{18_Chapter_Eleven__THE_FORMS_OF_THO.xhtmlux5cux23id_565}{34}}.
La Marche, II, p. 80.

\protect\hypertarget{23_NOTES.xhtmlux5cux23id_564}{\protect\hyperlink{18_Chapter_Eleven__THE_FORMS_OF_THO.xhtmlux5cux23id_563}{35}}.
La Marche, II, p. 168.

\protect\hypertarget{23_NOTES.xhtmlux5cux23id_562}{\protect\hyperlink{18_Chapter_Eleven__THE_FORMS_OF_THO.xhtmlux5cux23id_561}{36}}.
Chastellain, IV, p. 169.

\protect\hypertarget{23_NOTES.xhtmlux5cux23id_560}{\protect\hyperlink{18_Chapter_Eleven__THE_FORMS_OF_THO.xhtmlux5cux23id_559}{37}}.
Chron. scand., II, p. 83.

\protect\hypertarget{23_NOTES.xhtmlux5cux23id_558}{\protect\hyperlink{18_Chapter_Eleven__THE_FORMS_OF_THO.xhtmlux5cux23id_557}{38}}.
Petit-Dutaillis, Documents nouveaux sur les moeurs populaires etc.; see
Chastellain, V, p. 399 in Jacques du Clercq, passim.

\protect\hypertarget{23_NOTES.xhtmlux5cux23id_556}{\protect\hyperlink{18_Chapter_Eleven__THE_FORMS_OF_THO.xhtmlux5cux23id_555}{39}}.
Du Clercq, IV, p. 264; see III, pp. 180, 184, 206, 209.

\protect\hypertarget{23_NOTES.xhtmlux5cux23id_554}{\protect\hyperlink{18_Chapter_Eleven__THE_FORMS_OF_THO.xhtmlux5cux23id_553}{40}}.
Monstrelet, I, p. 342, V, p. 333; Chastellain, II, p. 389; La Marche,
II, pp. 284, 331; Le livre des trahisons, pp. 34, 226.

\protect\hypertarget{23_NOTES.xhtmlux5cux23id_552}{\protect\hyperlink{18_Chapter_Eleven__THE_FORMS_OF_THO.xhtmlux5cux23id_551}{41}}.
Quicherat, Th. Basin, I, p. xliv.

\protect\hypertarget{23_NOTES.xhtmlux5cux23id_550}{\protect\hyperlink{18_Chapter_Eleven__THE_FORMS_OF_THO.xhtmlux5cux23id_549}{42}}.
Chastellain, III, p. 106.

\protect\hypertarget{23_NOTES.xhtmlux5cux23id_548}{\protect\hyperlink{18_Chapter_Eleven__THE_FORMS_OF_THO.xhtmlux5cux23id_547}{43}}.
Sermo de nativ. domini, Gerson, Opera, III, p. 947.

\protect\hypertarget{23_NOTES.xhtmlux5cux23id_546}{\protect\hyperlink{18_Chapter_Eleven__THE_FORMS_OF_THO.xhtmlux5cux23id_545}{44}}.
Le pastoralet, vs. 2043.

\protect\hypertarget{23_NOTES.xhtmlux5cux23id_544}{\protect\hyperlink{18_Chapter_Eleven__THE_FORMS_OF_THO.xhtmlux5cux23id_543}{45}}.
Jean Jouffroy, Oratio, I, p. 188.

\protect\hypertarget{23_NOTES.xhtmlux5cux23id_542}{\protect\hyperlink{18_Chapter_Eleven__THE_FORMS_OF_THO.xhtmlux5cux23id_541}{46}}.
La Marche, I, p. 63.

\protect\hypertarget{23_NOTES.xhtmlux5cux23id_540}{\protect\hyperlink{18_Chapter_Eleven__THE_FORMS_OF_THO.xhtmlux5cux23id_539}{47}}.
Gerson, Querela nomine Universitatis etc., Opera, IV, p. 574; see Rel.
de S. Denis, III, p. 185.

\protect\hypertarget{23_NOTES.xhtmlux5cux23id_538}{\protect\hyperlink{18_Chapter_Eleven__THE_FORMS_OF_THO.xhtmlux5cux23id_537}{48}}.
Chastellain, II, p. 375, see 307.

\protect\hypertarget{23_NOTES.xhtmlux5cux23id_536}{\protect\hyperlink{18_Chapter_Eleven__THE_FORMS_OF_THO.xhtmlux5cux23id_535}{49}}.
Commines, I, p. 111, 363.

\protect\hypertarget{23_NOTES.xhtmlux5cux23id_534}{\protect\hyperlink{18_Chapter_Eleven__THE_FORMS_OF_THO.xhtmlux5cux23id_533}{50}}.
Monstrelet, IV, p. 388.

\protect\hypertarget{23_NOTES.xhtmlux5cux23id_532}{\protect\hyperlink{18_Chapter_Eleven__THE_FORMS_OF_THO.xhtmlux5cux23id_531}{51}}.
Basin, I, p. 66.

\protect\hypertarget{23_NOTES.xhtmlux5cux23id_530}{\protect\hyperlink{18_Chapter_Eleven__THE_FORMS_OF_THO.xhtmlux5cux23id_529}{52}}.
La Marche, I pp. 60, 63, 83, 88, 91, 94, 1341; III p. 101.

\protect\hypertarget{23_NOTES.xhtmlux5cux23id_528}{\protect\hyperlink{18_Chapter_Eleven__THE_FORMS_OF_THO.xhtmlux5cux23id_527}{53}}.
Commines, I, pp. 170, 262, 391, 413, 460.

\protect\hypertarget{23_NOTES.xhtmlux5cux23id_526}{\protect\hyperlink{18_Chapter_Eleven__THE_FORMS_OF_THO.xhtmlux5cux23id_525}{54}}.
Basin, II, pp. 417, 419; Molinet, Faictz et Dictz f. 205. In the third
line I read sa for la.

\protect\hypertarget{23_NOTES.xhtmlux5cux23id_524}{\protect\hyperlink{18_Chapter_Eleven__THE_FORMS_OF_THO.xhtmlux5cux23id_523}{55}}.
Deschamps, Oeuvres, t. IX.

\protect\hypertarget{23_NOTES.xhtmlux5cux23id_522}{\protect\hyperlink{18_Chapter_Eleven__THE_FORMS_OF_THO.xhtmlux5cux23id_521}{56}}.
Deschamps, Oeuvres, t. IX, pp. 219ff.

\protect\hypertarget{23_NOTES.xhtmlux5cux23id_520}{\protect\hyperlink{18_Chapter_Eleven__THE_FORMS_OF_THO.xhtmlux5cux23id_519}{57}}.
Deschamps, Oeuvres, t. IX, pp. 293ff.

\protect\hypertarget{23_NOTES.xhtmlux5cux23id_518}{\protect\hyperlink{18_Chapter_Eleven__THE_FORMS_OF_THO.xhtmlux5cux23id_517}{58}}.
See Marett, The threshold of religion, passim.

\protect\hypertarget{23_NOTES.xhtmlux5cux23id_516}{\protect\hyperlink{18_Chapter_Eleven__THE_FORMS_OF_THO.xhtmlux5cux23id_515}{59}}.
Monstrelet, IV, p. 93; Livre des trahisons, p. 157; Molinet, II, p. 129;
see du Clercq, IV, pp. 203, 273; Th. Pauli, p. 278.

\protect\hypertarget{23_NOTES.xhtmlux5cux23id_514}{\protect\hyperlink{18_Chapter_Eleven__THE_FORMS_OF_THO.xhtmlux5cux23id_513}{60}}.
Molinet, I, p. 65.

\protect\hypertarget{23_NOTES.xhtmlux5cux23id_512}{\protect\hyperlink{18_Chapter_Eleven__THE_FORMS_OF_THO.xhtmlux5cux23id_511}{61}}.
Molinet, IV, p. 417; Courtaulx is a musical instrument,'Mornifle is a
card game.

\protect\hypertarget{23_NOTES.xhtmlux5cux23id_510}{\protect\hyperlink{18_Chapter_Eleven__THE_FORMS_OF_THO.xhtmlux5cux23id_509}{62}}.
Gerson, Opera, I, p. 205.

\protect\hypertarget{23_NOTES.xhtmlux5cux23page_431}{\protect\hyperlink{18_Chapter_Eleven__THE_FORMS_OF_THO.xhtmlux5cux23id_508}{63}}.
Le songe du vieil pelerin, in Jorga, Phil. de Mézières, p. 691.

\protect\hypertarget{23_NOTES.xhtmlux5cux23id_507}{\protect\hyperlink{18_Chapter_Eleven__THE_FORMS_OF_THO.xhtmlux5cux23id_506}{64}}.
Juvenal des Ursins, p. 425.

\protect\hypertarget{23_NOTES.xhtmlux5cux23id_505}{\protect\hyperlink{18_Chapter_Eleven__THE_FORMS_OF_THO.xhtmlux5cux23id_504}{65}}.
Juvenal des Ursins, p. 415.

\emph{\protect\hypertarget{23_NOTES.xhtmlux5cux23id_503}{\protect\hyperlink{18_Chapter_Eleven__THE_FORMS_OF_THO.xhtmlux5cux23id_502}{66}}}.
Gerson, Opera, I, p. 206.

\protect\hypertarget{23_NOTES.xhtmlux5cux23id_501}{\protect\hyperlink{18_Chapter_Eleven__THE_FORMS_OF_THO.xhtmlux5cux23id_500}{67}}.
Gerson, Sermo coram rege Franciae, Opera, IV, p. 620; Juvenal des
Ursins, pp. 415, 423.

\protect\hypertarget{23_NOTES.xhtmlux5cux23id_499}{\protect\hyperlink{18_Chapter_Eleven__THE_FORMS_OF_THO.xhtmlux5cux23id_498}{68}}.
Gerson, Opera, I, p. 216.

\protect\hypertarget{23_NOTES.xhtmlux5cux23id_497}{\protect\hyperlink{18_Chapter_Eleven__THE_FORMS_OF_THO.xhtmlux5cux23id_496}{69}}.
Chastellain, IV, pp. 324, 323, 3141; see du Clercq, III, p. 236.

\protect\hypertarget{23_NOTES.xhtmlux5cux23id_495}{\protect\hyperlink{18_Chapter_Eleven__THE_FORMS_OF_THO.xhtmlux5cux23id_494}{70}}.
Chastellain, II, p. 376; III, pp. 446, 4471, 448; IV p. 213; V, p. 32.

\protect\hypertarget{23_NOTES.xhtmlux5cux23id_493}{\protect\hyperlink{18_Chapter_Eleven__THE_FORMS_OF_THO.xhtmlux5cux23id_492}{71}}.
Monstrelet, V, p. 425. {[}Trans.{]} \emph{Gilles de Rais}: See above,
chap. 3, note 10.

\protect\hypertarget{23_NOTES.xhtmlux5cux23id_491}{\protect\hyperlink{18_Chapter_Eleven__THE_FORMS_OF_THO.xhtmlux5cux23id_490}{72}}.
{[}Trans.{]} \emph{Malleus Maleficarum}: See above, chapter 8, note 59.

\protect\hypertarget{23_NOTES.xhtmlux5cux23id_489}{\protect\hyperlink{18_Chapter_Eleven__THE_FORMS_OF_THO.xhtmlux5cux23id_488}{73}}.
Chronique de Pierre le Prêtre, in Bourquelot, La vauderie d'Arras (Bibl.
de l'Ecole des chartes), 2 série, III, p. 109.

\protect\hypertarget{23_NOTES.xhtmlux5cux23id_487}{\protect\hyperlink{18_Chapter_Eleven__THE_FORMS_OF_THO.xhtmlux5cux23id_486}{74}}.
Jacques du Clercq, III, passim; Matthieu d'Escouchy, II, pp. 416ff.

\protect\hypertarget{23_NOTES.xhtmlux5cux23id_485}{\protect\hyperlink{18_Chapter_Eleven__THE_FORMS_OF_THO.xhtmlux5cux23id_484}{75}}.
Martin lefranc, Le champion des dames, in Bourquelot, La vauderie
d'Arras, p. 86; in Ro. Gaguini, ed. Thuasne, II, p. 474.

\protect\hypertarget{23_NOTES.xhtmlux5cux23id_483}{\protect\hyperlink{18_Chapter_Eleven__THE_FORMS_OF_THO.xhtmlux5cux23id_482}{76}}.
Froissart, ed. Kervyn, XI, p. 193.

\protect\hypertarget{23_NOTES.xhtmlux5cux23id_481}{\protect\hyperlink{18_Chapter_Eleven__THE_FORMS_OF_THO.xhtmlux5cux23id_480}{77}}.
Gerson, Contra superstitionem praesertim Innocentum, Op. I, p. 205; De
erroribus circa artem magicam, I, p. 211; De falsis prophetis I, p. 545;
De passioni-bus animae, III, p. 142.

\protect\hypertarget{23_NOTES.xhtmlux5cux23id_479}{\protect\hyperlink{18_Chapter_Eleven__THE_FORMS_OF_THO.xhtmlux5cux23id_478}{78}}.
Journal d'un bourgeois, p. 236.

\protect\hypertarget{23_NOTES.xhtmlux5cux23id_477}{\protect\hyperlink{18_Chapter_Eleven__THE_FORMS_OF_THO.xhtmlux5cux23id_476}{79}}.
Journal d'un bourgeois, p. 220.

\protect\hypertarget{23_NOTES.xhtmlux5cux23id_475}{\protect\hyperlink{18_Chapter_Eleven__THE_FORMS_OF_THO.xhtmlux5cux23id_474}{80}}.
Dion. Cart., Contra vitia superstitionum quibus circa cultum veri Dei
erratur, Opera, t. XXXVI, pp. 211ff.; see A. Franz, Die kirchlichen
Benediktionen im Mittelalter, Freiburg 1909, 2 vols.

\protect\hypertarget{23_NOTES.xhtmlux5cux23id_473}{\protect\hyperlink{18_Chapter_Eleven__THE_FORMS_OF_THO.xhtmlux5cux23id_472}{81}}.
For example, Jacques du Clercq, III, pp. 104--7.

\textbf{\emph{Chapter 12}}

\protect\hypertarget{23_NOTES.xhtmlux5cux23id_2656}{\protect\hyperlink{20_ILLUSTRATIONS_FOLLOW_PAGE.xhtmlux5cux23id_2655}{*\textsuperscript{1}}}
``representing the dead man when he was alive.''

\protect\hypertarget{23_NOTES.xhtmlux5cux23id_2658}{\protect\hyperlink{20_ILLUSTRATIONS_FOLLOW_PAGE.xhtmlux5cux23id_2657}{†\textsuperscript{2}}}
``Five sols to Blaise for representing the dead knight at the funeral.''

\protect\hypertarget{23_NOTES.xhtmlux5cux23id_2660}{\protect\hyperlink{20_ILLUSTRATIONS_FOLLOW_PAGE.xhtmlux5cux23id_2659}{*\textsuperscript{3}}}
``Breadhouse''

\protect\hypertarget{23_NOTES.xhtmlux5cux23id_2662}{\protect\hyperlink{20_ILLUSTRATIONS_FOLLOW_PAGE.xhtmlux5cux23id_2661}{*\textsuperscript{4}}}
``One could not even conceive a way to make the ship prettier which the
lord de la Trémoïlle did not have done. And all this paid for by the
poor people of France.''

\protect\hypertarget{23_NOTES.xhtmlux5cux23id_2664}{\protect\hyperlink{20_ILLUSTRATIONS_FOLLOW_PAGE.xhtmlux5cux23id_2663}{*\textsuperscript{5}}}
``Because great and honorable achievements deserve a lasting renown and
perpetual remembrance.''

\protect\hypertarget{23_NOTES.xhtmlux5cux23id_2666}{\protect\hyperlink{20_ILLUSTRATIONS_FOLLOW_PAGE.xhtmlux5cux23id_2665}{*\textsuperscript{6}}}
``And this certainly was a very fine entremet for there were more than
forty persons in it.''

\protect\hypertarget{23_NOTES.xhtmlux5cux23id_2668}{\protect\hyperlink{20_ILLUSTRATIONS_FOLLOW_PAGE.xhtmlux5cux23id_2667}{*\textsuperscript{7}}}
``chamberlain of the duke of Burgundy''

\protect\hypertarget{23_NOTES.xhtmlux5cux23id_2670}{\protect\hyperlink{20_ILLUSTRATIONS_FOLLOW_PAGE.xhtmlux5cux23id_2669}{*\textsuperscript{8}}}
``by an ingenious mechanism''

\protect\hypertarget{23_NOTES.xhtmlux5cux23id_2672}{\protect\hyperlink{20_ILLUSTRATIONS_FOLLOW_PAGE.xhtmlux5cux23id_2671}{†\textsuperscript{9}}}
``as if he had returned to heaven of his own accord.''

\protect\hypertarget{23_NOTES.xhtmlux5cux23id_2674}{\protect\hyperlink{20_ILLUSTRATIONS_FOLLOW_PAGE.xhtmlux5cux23id_2673}{‡\textsuperscript{10}}}
``on an artificial horse, sallying forth and caracoling in such a way
that it was a fine thing to see.''

\protect\hypertarget{23_NOTES.xhtmlux5cux23id_2350}{\protect\hyperlink{20_ILLUSTRATIONS_FOLLOW_PAGE.xhtmlux5cux23id_2349}{*\textsuperscript{11}}}
``a false book .~.~. of a block of white wood painted to look like a
book, in which there were no leaves and nothing was written.''

\protect\hypertarget{23_NOTES.xhtmlux5cux23id_2676}{\protect\hyperlink{20_ILLUSTRATIONS_FOLLOW_PAGE.xhtmlux5cux23id_2675}{*\textsuperscript{12}}}
``Dressed in gold cloth and royal ornaments, befitting her estate, and
appearing to be the most worldly of them all, giving an ear to those
empty words (as one does) and displaying the outward nature of the most
careless and conceited, she wore day after day the hair shirt on her
naked skin, fasted secretly on bread and water and, when her husband was
absent, slept in the straw of her bed.''

\protect\hypertarget{23_NOTES.xhtmlux5cux23id_2678}{\protect\hyperlink{20_ILLUSTRATIONS_FOLLOW_PAGE.xhtmlux5cux23id_2677}{*\textsuperscript{13}}}
``in order to live a beautiful and pious life.''

\protect\hypertarget{23_NOTES.xhtmlux5cux23id_2679}{\protect\hyperlink{20_ILLUSTRATIONS_FOLLOW_PAGE.xhtmlux5cux23id_2680}{†\textsuperscript{14}}}
``pomp and pride.''

\protect\hypertarget{23_NOTES.xhtmlux5cux23id_2682}{\protect\hyperlink{20_ILLUSTRATIONS_FOLLOW_PAGE.xhtmlux5cux23id_2681}{‡\textsuperscript{15}}}
``the outrageous excess and the great waste that was made for the
purpose of that banquet.''

\protect\hypertarget{23_NOTES.xhtmlux5cux23id_2684}{\protect\hyperlink{20_ILLUSTRATIONS_FOLLOW_PAGE.xhtmlux5cux23id_2683}{*\textsuperscript{16}}}
``came from the humble people,''

\protect\hypertarget{23_NOTES.xhtmlux5cux23id_2686}{\protect\hyperlink{20_ILLUSTRATIONS_FOLLOW_PAGE.xhtmlux5cux23id_2685}{†\textsuperscript{17}}}
``He used to govern everything quite alone and manage and bear the
burden of all business by himself, be it of war, be it of peace, be it
of matters of finance.''

\protect\hypertarget{23_NOTES.xhtmlux5cux23id_2688}{\protect\hyperlink{20_ILLUSTRATIONS_FOLLOW_PAGE.xhtmlux5cux23id_2687}{‡\textsuperscript{18}}}
``Said chancellor was reputed among the sages of the realm, to speak
temporally; for as to spiritual matters, I shall be silent.''

\protect\hypertarget{23_NOTES.xhtmlux5cux23id_2352}{\protect\hyperlink{20_ILLUSTRATIONS_FOLLOW_PAGE.xhtmlux5cux23id_2351}{*\textsuperscript{19}}}
``Three things are required for beauty .~.~. First, integrity or
completeness, because what is unfinished is repugnant. Next, proportion
or consonance is required. And finally, clarity, since we call beautiful
whatever has a pure color.''

\protect\hypertarget{23_NOTES.xhtmlux5cux23id_2689}{\protect\hyperlink{20_ILLUSTRATIONS_FOLLOW_PAGE.xhtmlux5cux23id_2690}{*\textsuperscript{20}}}
``Because music is the resonance of the heavens, the voice of the
angels, the joy of paradise, the hope of the air, the organ of the
Church, the song of the little birds, the recreation of all gloomy and
despairing hearts, the persecution and the driving away of the devils.''

\protect\hypertarget{23_NOTES.xhtmlux5cux23id_2692}{\protect\hyperlink{20_ILLUSTRATIONS_FOLLOW_PAGE.xhtmlux5cux23id_2691}{*\textsuperscript{21}}}
``The chattering of women,''

\protect\hypertarget{23_NOTES.xhtmlux5cux23id_2694}{\protect\hyperlink{20_ILLUSTRATIONS_FOLLOW_PAGE.xhtmlux5cux23id_2693}{*\textsuperscript{22}}}
Some dress themselves for her in green,/The other blue, another in
white,/ Another in vermillion like blood,/And he who desires her
most/Because of his great sorrow dresses in black.

\protect\hypertarget{23_NOTES.xhtmlux5cux23id_2696}{\protect\hyperlink{20_ILLUSTRATIONS_FOLLOW_PAGE.xhtmlux5cux23id_2695}{†\textsuperscript{23}}}
You will have to dress in green, /It is the livery of those in love.

\protect\hypertarget{23_NOTES.xhtmlux5cux23id_2698}{\protect\hyperlink{20_ILLUSTRATIONS_FOLLOW_PAGE.xhtmlux5cux23id_2697}{*\textsuperscript{24}}}
To wear mottoes of love for one's lady,/or to wear blue is no proof,/But
to serve her with a perfectly loyal heart/And no others, and to keep her
from blame/ .~.~. Love lies in that, not in wearing blue,/But it may be
that many think/ To cover the offense of falsehood under a tombstone/By
wearing blue. .~.~.

\protect\hypertarget{23_NOTES.xhtmlux5cux23id_2700}{\protect\hyperlink{20_ILLUSTRATIONS_FOLLOW_PAGE.xhtmlux5cux23id_2699}{†\textsuperscript{25}}}
That he, who dresses me in the blue coat/And causes people to point
their fingers at me, will be killed!

\protect\hypertarget{23_NOTES.xhtmlux5cux23id_2702}{\protect\hyperlink{20_ILLUSTRATIONS_FOLLOW_PAGE.xhtmlux5cux23id_2701}{‡\textsuperscript{26}}}
Of all colors I love brown best,/And because I love it, I dress in it,/I
have forgotten all other colors./Alas! What I love is not here.

\protect\hypertarget{23_NOTES.xhtmlux5cux23id_2704}{\protect\hyperlink{20_ILLUSTRATIONS_FOLLOW_PAGE.xhtmlux5cux23id_2703}{*\textsuperscript{27}}}
I may well wear gray and tan/For I have no longer any hope.

\protect\hypertarget{23_NOTES.xhtmlux5cux23id_2706}{\protect\hyperlink{20_ILLUSTRATIONS_FOLLOW_PAGE.xhtmlux5cux23id_2705}{†\textsuperscript{28}}}
``and the duke was informed that it was meant for him.''/{[}Note 15{]}
furthermore, she did not have a formal hairdo like other ladies who were
of royal standing.

\protect\hypertarget{23_NOTES.xhtmlux5cux23id_471}{\protect\hyperlink{19_Chapter_Twelve__ART_IN_LIFE.xhtmlux5cux23id_470}{1}}.
The major parts of chapters 12 and 13 are a restructuring and expansion
of the essay: De Kunst der Van Eyck's in het leven van hun tijd, De
Gids, 1916, nos. 6 and 7.

\protect\hypertarget{23_NOTES.xhtmlux5cux23id_469}{\protect\hyperlink{19_Chapter_Twelve__ART_IN_LIFE.xhtmlux5cux23id_468}{2}}.
{[}Trans.{]} Hugo's \emph{Notre Dame de Paris} is best known to most
Americans as \emph{The Hunchback of Notre Dame}.

\protect\hypertarget{23_NOTES.xhtmlux5cux23id_467}{\protect\hyperlink{19_Chapter_Twelve__ART_IN_LIFE.xhtmlux5cux23id_466}{3}}.
Rel. de S. Denis, II, p. 78.

\protect\hypertarget{23_NOTES.xhtmlux5cux23id_465}{\protect\hyperlink{20_ILLUSTRATIONS_FOLLOW_PAGE.xhtmlux5cux23id_464}{4}}.
Rel. de S. Denis, II, p. 413.

\protect\hypertarget{23_NOTES.xhtmlux5cux23id_463}{\protect\hyperlink{20_ILLUSTRATIONS_FOLLOW_PAGE.xhtmlux5cux23id_462}{5}}.
Rel. de S. Denis, I, p. 358.

\protect\hypertarget{23_NOTES.xhtmlux5cux23id_461}{\protect\hyperlink{20_ILLUSTRATIONS_FOLLOW_PAGE.xhtmlux5cux23id_460}{6}}.
Rel. de S. Denis, I, p. 600; Juvenal des Ursins, p. 379.

\protect\hypertarget{23_NOTES.xhtmlux5cux23id_459}{\protect\hyperlink{20_ILLUSTRATIONS_FOLLOW_PAGE.xhtmlux5cux23id_458}{7}}.
La Curne de Sainte Palaye, I, p. 388; see also Journal d'un bourgeois,
p. 67.

\protect\hypertarget{23_NOTES.xhtmlux5cux23id_457}{\protect\hyperlink{20_ILLUSTRATIONS_FOLLOW_PAGE.xhtmlux5cux23id_456}{8}}.
Journal d'un bourgeois, p. 179 (Charles VI); 309 (Isabella of Barvaria);
Chastellain, IV, p. 42 (Charles VII), I, p. 332 (Henry V); Lefèvre de S.
Remy, II, p. 65; M. d'Escouchy, II, pp. 424, 432; Chron. scand., I, p.
21; Jean Chartier, p. 319 (Charles VII); Quatrebarbes, Oeuvres du roi
René, I, p. 129; Gaguini compendium super Francorum gestis, ed. Paris,
1500, burial of Charles VIII, f. 164.

\protect\hypertarget{23_NOTES.xhtmlux5cux23id_455}{\protect\hyperlink{20_ILLUSTRATIONS_FOLLOW_PAGE.xhtmlux5cux23id_454}{9}}.
Martial d'Auvergne, Vigilles de Charles VII. Les poésies de Martial de
Paris, dit d'Auvergne, Paris, 1724, 2 vols., II, p. 170.

\protect\hypertarget{23_NOTES.xhtmlux5cux23page_432}{\protect\hyperlink{20_ILLUSTRATIONS_FOLLOW_PAGE.xhtmlux5cux23id_453}{10}}.
For example Froissart, ed. Luce, VIII, p. 43.

\protect\hypertarget{23_NOTES.xhtmlux5cux23id_452}{\protect\hyperlink{20_ILLUSTRATIONS_FOLLOW_PAGE.xhtmlux5cux23id_451}{11}}.
Froissart, ed. Kervyn, XI, p. 367. A variant of the text has
``proviseurs'' for ``peintres.'' The context makes the latter more
probable.

\protect\hypertarget{23_NOTES.xhtmlux5cux23id_450}{\protect\hyperlink{20_ILLUSTRATIONS_FOLLOW_PAGE.xhtmlux5cux23id_449}{12}}.
{[}Trans.{]} \emph{Plourants: pleurants}, mourners.

\protect\hypertarget{23_NOTES.xhtmlux5cux23id_448}{\protect\hyperlink{20_ILLUSTRATIONS_FOLLOW_PAGE.xhtmlux5cux23id_447}{13}}.
Betty Kurth, Die Blutezeit der Bildwirkerkunst zu Tournay und der
Burgundische Hof, Jahrbuch der Kunstsammlungen des Kaiserhauses, 34,
1917, 3.

\protect\hypertarget{23_NOTES.xhtmlux5cux23id_446}{\protect\hyperlink{20_ILLUSTRATIONS_FOLLOW_PAGE.xhtmlux5cux23id_445}{14}}.
{[}Trans.{]} \emph{Nicopolis}: September 25, 1396, a combined European
force putatively led by John of Nevers (later John the Fearless, duke of
Burgundy) met the Turkish forces of Sultan Bajazet. The Europeans fought
bravely, but the disarray caused by their quarrels over precedence in
the order of battle combined with the genius of Bajazet to give the
battle to the Turks. Hundreds of knights were captured, stripped naked,
and executed one by one in full view of their fellows. Only the most
prominent were spared to be held for ransom. John of Nevers personally
pled with Bajazet for the life of Boucicaut, who was allowed to live
once it was understood that he, too, was wealthy. See Tuchmann,
\emph{Mirror}.

\protect\hypertarget{23_NOTES.xhtmlux5cux23id_444}{\protect\hyperlink{20_ILLUSTRATIONS_FOLLOW_PAGE.xhtmlux5cux23id_443}{15}}.
Pierre de Fenin, p. 624 of Bonne d'Artois: ``et avec ce ne portoit point
d'estate sur son chief comment autres dames à elle
pareilles.''{[}Trans.: ``furthermore, she did not have a formal hairdo
like other ladies who were of royal standing.''{]}

\protect\hypertarget{23_NOTES.xhtmlux5cux23id_442}{\protect\hyperlink{20_ILLUSTRATIONS_FOLLOW_PAGE.xhtmlux5cux23id_441}{16}}.
Le livre des trahisons, p. 156.

\protect\hypertarget{23_NOTES.xhtmlux5cux23id_440}{\protect\hyperlink{20_ILLUSTRATIONS_FOLLOW_PAGE.xhtmlux5cux23id_439}{17}}.
Chastellain, III, p. 375; La Marche, II p. 340, III p. 165; d'Escouchy,
II, p. 116; Laborde, II; see Moliner, Les sources de l'hist. de France,
nos. 3645, 3661, 3663, 5030; Inv. des arch. du Nord, IV, p. 195.

\protect\hypertarget{23_NOTES.xhtmlux5cux23id_438}{\protect\hyperlink{20_ILLUSTRATIONS_FOLLOW_PAGE.xhtmlux5cux23id_437}{18}}.
La Marche, II, pp. 34off.

\protect\hypertarget{23_NOTES.xhtmlux5cux23id_436}{\protect\hyperlink{20_ILLUSTRATIONS_FOLLOW_PAGE.xhtmlux5cux23id_435}{19}}.
This is a type of merchant ship; the low German form is \emph{Kracke}.

\protect\hypertarget{23_NOTES.xhtmlux5cux23id_434}{\protect\hyperlink{20_ILLUSTRATIONS_FOLLOW_PAGE.xhtmlux5cux23id_433}{20}}.
Laborde, II, p. 326.

\protect\hypertarget{23_NOTES.xhtmlux5cux23id_432}{\protect\hyperlink{20_ILLUSTRATIONS_FOLLOW_PAGE.xhtmlux5cux23id_431}{21}}.
La Marche, III, p. 197.

\protect\hypertarget{23_NOTES.xhtmlux5cux23id_430}{\protect\hyperlink{20_ILLUSTRATIONS_FOLLOW_PAGE.xhtmlux5cux23id_429}{22}}.
Laborde, II, p. 375, no. 4880.

\protect\hypertarget{23_NOTES.xhtmlux5cux23id_428}{\protect\hyperlink{20_ILLUSTRATIONS_FOLLOW_PAGE.xhtmlux5cux23id_427}{23}}.
Laborde, II, pp. 322, 329.

\protect\hypertarget{23_NOTES.xhtmlux5cux23id_426}{\protect\hyperlink{20_ILLUSTRATIONS_FOLLOW_PAGE.xhtmlux5cux23id_425}{24}}.
Although on the primary references, the master's seal says ``Claus
Sluter,'' one can hardly think that the non-Dutch Claus could have been
the original form of his Christian name.

\protect\hypertarget{23_NOTES.xhtmlux5cux23id_424}{\protect\hyperlink{20_ILLUSTRATIONS_FOLLOW_PAGE.xhtmlux5cux23id_423}{25}}.
A. Kleinclausz, Un atelier de sculpture au XVe siecle, Gazette des beaux
arts, t. 29, 1903, I.

\emph{\protect\hypertarget{23_NOTES.xhtmlux5cux23id_422}{\protect\hyperlink{20_ILLUSTRATIONS_FOLLOW_PAGE.xhtmlux5cux23id_421}{26}}}.
Exod. 12:6: ``The whole assembly of the congregation of Israel shall
kill it in the evening.'' Ps. 21:18: ``They pierced my hands and my
feet. I may tell all my bones.'' Isaiah 53:7: ``He is brought as a lamb
to the slaughter, and as a sheep before her shearers is dumb, so he
openeth not his mouth.'' Lamentations 1:12: ``All ye who pass by, behold
and see if there is any sorrow like unto my sorrow.'' Daniel 9:26:
``After threescore and two weeks shall Messiah be cut off.'' Zechariah
11:12: ``They weighed for my price 30 pieces of silver.''

\protect\hypertarget{23_NOTES.xhtmlux5cux23id_420}{\protect\hyperlink{20_ILLUSTRATIONS_FOLLOW_PAGE.xhtmlux5cux23id_419}{27}}.
The now vanished colors are known through a report composed in 1832.

\protect\hypertarget{23_NOTES.xhtmlux5cux23id_418}{\protect\hyperlink{20_ILLUSTRATIONS_FOLLOW_PAGE.xhtmlux5cux23id_417}{28}}.
Kleinclausz, L'art funéraire de la Bourgogne au moyen âge, Gazette des
beaux arts, 1902, t. 27.

\protect\hypertarget{23_NOTES.xhtmlux5cux23id_416}{\protect\hyperlink{20_ILLUSTRATIONS_FOLLOW_PAGE.xhtmlux5cux23id_415}{29}}.
Chastellain, V, p. 262, Doutrepont, p. 156.

\protect\hypertarget{23_NOTES.xhtmlux5cux23id_414}{\protect\hyperlink{20_ILLUSTRATIONS_FOLLOW_PAGE.xhtmlux5cux23id_413}{30}}.
{[}Trans.{]} \emph{The stag with the crown}: This emblem was of special
significance to Charles VI, who, when he was told that such a stag had
been taken, wearing
\protect\hypertarget{23_NOTES.xhtmlux5cux23page_433}{}{}a crown around
its neck inscribed \emph{Caesar hoc mihi donavit}, ordered the emblem
placed on the royal crockery. See Tuchmann, \emph{Mirror}.

\protect\hypertarget{23_NOTES.xhtmlux5cux23id_412}{\protect\hyperlink{20_ILLUSTRATIONS_FOLLOW_PAGE.xhtmlux5cux23id_411}{31}}.
Juvenal des Ursins, p. 378.

\protect\hypertarget{23_NOTES.xhtmlux5cux23id_410}{\protect\hyperlink{20_ILLUSTRATIONS_FOLLOW_PAGE.xhtmlux5cux23id_409}{32}}.
Jacques du Clercq, II, p. 280.

\protect\hypertarget{23_NOTES.xhtmlux5cux23id_408}{\protect\hyperlink{20_ILLUSTRATIONS_FOLLOW_PAGE.xhtmlux5cux23id_407}{33}}.
Foulquart, in d'Hericault, Oeuvres de Coquillart, I, p. 23\emph{1}.

\protect\hypertarget{23_NOTES.xhtmlux5cux23id_406}{\protect\hyperlink{20_ILLUSTRATIONS_FOLLOW_PAGE.xhtmlux5cux23id_405}{34}}.
Lefèvre de S. Remy, II, p. 291.

\protect\hypertarget{23_NOTES.xhtmlux5cux23id_404}{\protect\hyperlink{20_ILLUSTRATIONS_FOLLOW_PAGE.xhtmlux5cux23id_403}{35}}.
London, National Gallery; Berlin, Kaiser-Friedrich-Museum.

\protect\hypertarget{23_NOTES.xhtmlux5cux23id_402}{\protect\hyperlink{20_ILLUSTRATIONS_FOLLOW_PAGE.xhtmlux5cux23id_401}{36}}.
W. H. J. Weale, Hubert and John van Eyck, Their life and work,
London--New York, 1908, p. 701.

\protect\hypertarget{23_NOTES.xhtmlux5cux23id_400}{\protect\hyperlink{20_ILLUSTRATIONS_FOLLOW_PAGE.xhtmlux5cux23id_399}{37}}.
Froissart, ed. Kervyn, XI, p. 197.

\protect\hypertarget{23_NOTES.xhtmlux5cux23id_398}{\protect\hyperlink{20_ILLUSTRATIONS_FOLLOW_PAGE.xhtmlux5cux23id_397}{38}}.
P. Durrieu, Les très riches heures de Jean de France, duc de Bery
(Heures de Chantilly), Paris, 1904, p. 81.

\protect\hypertarget{23_NOTES.xhtmlux5cux23id_396}{\protect\hyperlink{20_ILLUSTRATIONS_FOLLOW_PAGE.xhtmlux5cux23id_395}{39}}.
Moll, Kerkgesch. II\textsuperscript{3}, p. 313; see J. G. R. Acquoy, Het
klooster van Windesheim en zijn invloed, Utrecht, 1875--90, 3 vols., II,
p. 249.

\protect\hypertarget{23_NOTES.xhtmlux5cux23id_394}{\protect\hyperlink{20_ILLUSTRATIONS_FOLLOW_PAGE.xhtmlux5cux23id_393}{40}}.
Th. à Kempis, Sermones ad novitios no. 29, Opera, ed. Pohl, t. VI, p.
287.

\protect\hypertarget{23_NOTES.xhtmlux5cux23id_392}{\protect\hyperlink{20_ILLUSTRATIONS_FOLLOW_PAGE.xhtmlux5cux23id_391}{41}}.
Moll, Kerkgesch. II\textsuperscript{2}, p. 321; Acquoy, Het klooster van
Windesheim .~.~. , p. 222.

\protect\hypertarget{23_NOTES.xhtmlux5cux23id_390}{\protect\hyperlink{20_ILLUSTRATIONS_FOLLOW_PAGE.xhtmlux5cux23id_389}{42}}.
Chastellain, IV, p. 218.

\protect\hypertarget{23_NOTES.xhtmlux5cux23id_388}{\protect\hyperlink{20_ILLUSTRATIONS_FOLLOW_PAGE.xhtmlux5cux23id_387}{43}}.
La Marche, II, p. 398.

\protect\hypertarget{23_NOTES.xhtmlux5cux23id_386}{\protect\hyperlink{20_ILLUSTRATIONS_FOLLOW_PAGE.xhtmlux5cux23id_385}{44}}.
La Marche, II, 369.

\protect\hypertarget{23_NOTES.xhtmlux5cux23id_384}{\protect\hyperlink{20_ILLUSTRATIONS_FOLLOW_PAGE.xhtmlux5cux23id_383}{45}}.
Chastellain, IV pp. 136, 275, 359, 361, V p. 225; du Clercq, IV, p. 7.

\protect\hypertarget{23_NOTES.xhtmlux5cux23id_382}{\protect\hyperlink{20_ILLUSTRATIONS_FOLLOW_PAGE.xhtmlux5cux23id_381}{46}}.
Chastellain, III, p. 332; du Clercq, III, p. 56.

\protect\hypertarget{23_NOTES.xhtmlux5cux23id_380}{\protect\hyperlink{20_ILLUSTRATIONS_FOLLOW_PAGE.xhtmlux5cux23id_379}{47}}.
Chastellain, V p. 44, II p. 281; La Marche, II, p. 85; du Clercq, III,
p. 56.

\protect\hypertarget{23_NOTES.xhtmlux5cux23id_378}{\protect\hyperlink{20_ILLUSTRATIONS_FOLLOW_PAGE.xhtmlux5cux23id_377}{48}}.
Chastellain, III, p. 330.

\protect\hypertarget{23_NOTES.xhtmlux5cux23id_376}{\protect\hyperlink{20_ILLUSTRATIONS_FOLLOW_PAGE.xhtmlux5cux23id_375}{49}}.
du Clercq, III, p. 203.

\protect\hypertarget{23_NOTES.xhtmlux5cux23id_374}{\protect\hyperlink{20_ILLUSTRATIONS_FOLLOW_PAGE.xhtmlux5cux23id_373}{50}}.
See p. 206.

\protect\hypertarget{23_NOTES.xhtmlux5cux23id_372}{\protect\hyperlink{20_ILLUSTRATIONS_FOLLOW_PAGE.xhtmlux5cux23id_371}{51}}.
Bonaventura's editor in Quaracchi ascribes them to Johannes de Caulibus,
a Frenchman of San Gimignano who died in 1370.

\protect\hypertarget{23_NOTES.xhtmlux5cux23id_370}{\protect\hyperlink{20_ILLUSTRATIONS_FOLLOW_PAGE.xhtmlux5cux23id_369}{52}}.
Facius, Liber de viris illustribus, ed. L. Mehus, Florenz 1745, p. 46.

\protect\hypertarget{23_NOTES.xhtmlux5cux23id_368}{\protect\hyperlink{20_ILLUSTRATIONS_FOLLOW_PAGE.xhtmlux5cux23id_367}{53}}.
Dion. Cart., Opera, t. XXXIV, p. 223.

\protect\hypertarget{23_NOTES.xhtmlux5cux23id_366}{\protect\hyperlink{20_ILLUSTRATIONS_FOLLOW_PAGE.xhtmlux5cux23id_365}{54}}.
Dion. Cart., Opera, t. XXXIV, pp. 247, 230.

\protect\hypertarget{23_NOTES.xhtmlux5cux23id_364}{\protect\hyperlink{20_ILLUSTRATIONS_FOLLOW_PAGE.xhtmlux5cux23id_363}{55}}.
O. Zöckler, Dionys des Kartäusers Schrift de venustate mundi, Beitrag
zur Vorgeschichte der Ästhetik, Theol. Studien und Kritiken, 1881, p.
651; see E. Anitchkoff, L'esthétique au moyen âge XX, 1918, p. 221.

\protect\hypertarget{23_NOTES.xhtmlux5cux23id_362}{\protect\hyperlink{20_ILLUSTRATIONS_FOLLOW_PAGE.xhtmlux5cux23id_361}{56}}.
Summa theologiae, pars. ia, q. XXXIX, art. 8.

\protect\hypertarget{23_NOTES.xhtmlux5cux23id_360}{\protect\hyperlink{20_ILLUSTRATIONS_FOLLOW_PAGE.xhtmlux5cux23id_359}{57}}.
Dion. Cart., Opera, t. I, Vita, p. xxxvi.

\protect\hypertarget{23_NOTES.xhtmlux5cux23id_358}{\protect\hyperlink{20_ILLUSTRATIONS_FOLLOW_PAGE.xhtmlux5cux23id_357}{58}}.
Dion. Cart., De vita canonicorum, art. 20, Opera, t. XXXVII, p. 197: An
discantus in divino obsequio sit commendabilis; see Thomas Aquinas,
Summa theologiae, IIa, Ilae, q. 91, art. 2: Utrum cantus sint assumendi
ad laudem divinam.

\protect\hypertarget{23_NOTES.xhtmlux5cux23id_356}{\protect\hyperlink{20_ILLUSTRATIONS_FOLLOW_PAGE.xhtmlux5cux23id_355}{59}}.
{[}Trans.{]} \emph{Text painting}: The effort to make the music mirror
the meaning of the words so that, for instance, in a mass, the word
\emph{ascendit} would be accompanied by a rising melodic line,
\emph{descendit}, the opposite. ``Suffered, crucified and was buried''
would be set in an agitated texture. This is the musical equivalent of
symbolism and allegory.

\protect\hypertarget{23_NOTES.xhtmlux5cux23id_354}{\protect\hyperlink{20_ILLUSTRATIONS_FOLLOW_PAGE.xhtmlux5cux23id_353}{60}}.
Molinet, I, p. 73; see p. 67.

\protect\hypertarget{23_NOTES.xhtmlux5cux23id_352}{\protect\hyperlink{20_ILLUSTRATIONS_FOLLOW_PAGE.xhtmlux5cux23id_351}{61}}.
Petri Alliaci, De falsis prophetis, in Gerson, Opera, I, p. 538.

\protect\hypertarget{23_NOTES.xhtmlux5cux23page_434}{\protect\hyperlink{20_ILLUSTRATIONS_FOLLOW_PAGE.xhtmlux5cux23id_350}{62}}.
La Marche, II, p. 361.

\protect\hypertarget{23_NOTES.xhtmlux5cux23id_349}{\protect\hyperlink{20_ILLUSTRATIONS_FOLLOW_PAGE.xhtmlux5cux23id_348}{63}}.
De venustate etc., t. XXXIV, p. 242.

\protect\hypertarget{23_NOTES.xhtmlux5cux23id_347}{\protect\hyperlink{20_ILLUSTRATIONS_FOLLOW_PAGE.xhtmlux5cux23id_346}{64}}.
Froissart, ed. Luce, IV p. 90, VIII p. 43, 58, XI pp. 53, 129; ed.
Kervyn, XI pp. 340, 360, XIII p. 150, XIV pp. 157, 215.

\protect\hypertarget{23_NOTES.xhtmlux5cux23id_345}{\protect\hyperlink{20_ILLUSTRATIONS_FOLLOW_PAGE.xhtmlux5cux23id_344}{65}}.
Deschamps, I p. 155; II p. 211, II, no. 307, p. 208; La Marche, I, p.
274.

\emph{\protect\hypertarget{23_NOTES.xhtmlux5cux23id_343}{\protect\hyperlink{20_ILLUSTRATIONS_FOLLOW_PAGE.xhtmlux5cux23id_342}{66}}}.
Livre des trahisons, pp. 150, 156; La Marche, II pp. 12, 347, III pp.
127, 89; Chastellain, IV, p. 44; Chron. scand., I, pp. 26, 126.

\protect\hypertarget{23_NOTES.xhtmlux5cux23id_341}{\protect\hyperlink{20_ILLUSTRATIONS_FOLLOW_PAGE.xhtmlux5cux23id_340}{67}}.
Lefèvre de S. Remy, II, pp. 294, 296.

\protect\hypertarget{23_NOTES.xhtmlux5cux23id_339}{\protect\hyperlink{20_ILLUSTRATIONS_FOLLOW_PAGE.xhtmlux5cux23id_338}{68}}.
Couderc, Les comptes d'un grand couturier parisien au XVe siecle,
Bulletin de la soc. de l'hist. de Paris, XXXVIII, 1911, pp. 125ff.

\protect\hypertarget{23_NOTES.xhtmlux5cux23id_337}{\protect\hyperlink{20_ILLUSTRATIONS_FOLLOW_PAGE.xhtmlux5cux23id_336}{69}}.
For example Monstrelet, V, p. 2; du Clercq, I, p. 348.

\protect\hypertarget{23_NOTES.xhtmlux5cux23id_335}{\protect\hyperlink{20_ILLUSTRATIONS_FOLLOW_PAGE.xhtmlux5cux23id_334}{70}}.
{[}Trans.{]} \emph{Palfrey}: a knight generally had to have at least two
horses; the warhorse was a stallion and the palfrey a less high-spirited
animal suited for general purposes. Some palfreys were especially
trained to be suitable for women and priests.

\protect\hypertarget{23_NOTES.xhtmlux5cux23id_333}{\protect\hyperlink{20_ILLUSTRATIONS_FOLLOW_PAGE.xhtmlux5cux23id_332}{71}}.
La Marche, II, p. 343.

\protect\hypertarget{23_NOTES.xhtmlux5cux23id_331}{\protect\hyperlink{20_ILLUSTRATIONS_FOLLOW_PAGE.xhtmlux5cux23id_330}{72}}.
Chastellain, VII, p. 223; La Marche, I p. 276, II pp. n, 68, 345; du
Clercq, II, p. 197; Jean Germain, Liber de virtutibus, p. 11; Jouffroy,
Oratio, p. 173.

\protect\hypertarget{23_NOTES.xhtmlux5cux23id_329}{\protect\hyperlink{20_ILLUSTRATIONS_FOLLOW_PAGE.xhtmlux5cux23id_328}{73}}.
d'Escouchy, I, p. 234.

\protect\hypertarget{23_NOTES.xhtmlux5cux23id_327}{\protect\hyperlink{20_ILLUSTRATIONS_FOLLOW_PAGE.xhtmlux5cux23id_326}{74}}.
See p. 142.

\protect\hypertarget{23_NOTES.xhtmlux5cux23id_325}{\protect\hyperlink{20_ILLUSTRATIONS_FOLLOW_PAGE.xhtmlux5cux23id_324}{75}}.
Le miroir de mariage, XVII vs. 1650, Deschamps, Oeuvres, IX, p. 57.

\protect\hypertarget{23_NOTES.xhtmlux5cux23id_323}{\protect\hyperlink{20_ILLUSTRATIONS_FOLLOW_PAGE.xhtmlux5cux23id_322}{76}}.
Chansons françaises du quinzième siècle, ed. G. Paris (Soc. des anciens
textes français), 1875, no. XLX, p. 50; see Deschamps, no. 415, III, p.
217, no. 419, ib. p. 223, no. 423, ib. p. 227, no. 481, ib. p. 302, no.
728, IV, p. 199; L'amant rendu cordelier, sect. 62, p. 23; Molinet,
Faictz et Dictz, fol. 176.

\protect\hypertarget{23_NOTES.xhtmlux5cux23id_321}{\protect\hyperlink{20_ILLUSTRATIONS_FOLLOW_PAGE.xhtmlux5cux23id_320}{77}}.
Blason des couleurs of the herald Sicile (in La Curne de Sainte Palaye,
Mémoires sur l'ancienne chevalerie II, p. 56). Concerning color
symbolism in Italy, see Bertoni, L'Orlando furioso, pp. 221ff.

\protect\hypertarget{23_NOTES.xhtmlux5cux23id_319}{\protect\hyperlink{20_ILLUSTRATIONS_FOLLOW_PAGE.xhtmlux5cux23id_318}{78}}.
Cent balades d'amant et de dame, no. 92, Christine d'Pisan, Oeuvres
poétiques, III, p. 299. See Deschamps, X, no. 52; L'histoire et
plaisante chronicque du petit Jehan de Saintré, ed. G. Hellény, Paris,
1890, p. 415.

\protect\hypertarget{23_NOTES.xhtmlux5cux23id_317}{\protect\hyperlink{20_ILLUSTRATIONS_FOLLOW_PAGE.xhtmlux5cux23id_316}{79}}.
{[}Trans.\} \emph{Huik}: a hooded cloak. In Holland, the saying has it
that one ``hangs one's huik out to test the wind.''

\protect\hypertarget{23_NOTES.xhtmlux5cux23id_315}{\protect\hyperlink{20_ILLUSTRATIONS_FOLLOW_PAGE.xhtmlux5cux23id_314}{80}}.
Le pastoralet, vs. 2054, p. 636; see Les cent nouvelles, II, p. 118:
``craindroit tres fort estre du rang des bleuz vestuz qu'on appelle
communement noz amis.''

\protect\hypertarget{23_NOTES.xhtmlux5cux23id_313}{\protect\hyperlink{20_ILLUSTRATIONS_FOLLOW_PAGE.xhtmlux5cux23id_312}{81}}.
{[}Trans.{]} \emph{The blue boat}: As in the painting by Bosch in the
Louvre.

\protect\hypertarget{23_NOTES.xhtmlux5cux23id_311}{\protect\hyperlink{20_ILLUSTRATIONS_FOLLOW_PAGE.xhtmlux5cux23id_310}{82}}.
Chansons du XVe siecle, no. 5, p. 5; no. 87, p. 85.

\protect\hypertarget{23_NOTES.xhtmlux5cux23id_309}{\protect\hyperlink{20_ILLUSTRATIONS_FOLLOW_PAGE.xhtmlux5cux23id_308}{83}}.
La Marche, II, p. 207.

\textbf{\emph{Chapter 13}}

\protect\hypertarget{23_NOTES.xhtmlux5cux23id_2354}{\protect\hyperlink{21_Chapter_Thirteen__IMAGE_AND_WORD.xhtmlux5cux23id_2353}{*\textsuperscript{1}}}
``And as the day ends, one minute before the voice of the curator breaks
into your contemplation, the eye sees how the masterpiece transforms
itself in the softness of twilight; how the sky becomes still darker;
how the main scene, whose colors are faded, submerges in the eternal
mystery of harmony and unity .~.~. ''

\protect\hypertarget{23_NOTES.xhtmlux5cux23id_2708}{\protect\hyperlink{21_Chapter_Thirteen__IMAGE_AND_WORD.xhtmlux5cux23id_2707}{*\textsuperscript{2}}}
To forget melancholy, /And to cheer myself, /One sweet morning I went
into the fields,/On the first day on which love joins/Hearts in the
beautiful season .~.~.

\protect\hypertarget{23_NOTES.xhtmlux5cux23id_2710}{\protect\hyperlink{21_Chapter_Thirteen__IMAGE_AND_WORD.xhtmlux5cux23id_2709}{†\textsuperscript{3}}}
All around birds were flying,/And they sang so very sweetly,/That there
is no heart that would not be gladdened by it. /And while singing they
rose up in the air,/And then passed and repassed one another/Vying with
each other as to which should rise the highest./The weather was not
cloudy at all./The heavens were clad in blue,/And the beautiful sun was
shining brightly.

\protect\hypertarget{23_NOTES.xhtmlux5cux23id_2712}{\protect\hyperlink{21_Chapter_Thirteen__IMAGE_AND_WORD.xhtmlux5cux23id_2711}{*\textsuperscript{4}}}
I saw the trees blossom,/And hares and rabbits run./Everything rejoiced
at the Spring. /Amour seemed to hold sway there. /None could age or
die,/It seemed to me, so long as he was there. /From the grass rose a
sweet smell, /Which the clear air made sweeter still,/And purling
through the valley,/A little brook passed/ Moistening the lands/Of which
the water was not salty./There drank the little birds/After they had fed
upon crickets,/Little flies and butterflies./I saw there lanners, hawks,
and merlins,/And flies with a sting/Who made pavilions of fine honey/In
the trees by measure./In another part was the enclosure/Of a charming
meadow, where nature/Strewed flowers on the verdure/White, yellow, red,
and violet./It was encircled by blossoming trees/As white as if pure
snow/Covered them, it looked like a painting,/So many various colors
there were.

\protect\hypertarget{23_NOTES.xhtmlux5cux23id_2356}{\protect\hyperlink{21_Chapter_Thirteen__IMAGE_AND_WORD.xhtmlux5cux23id_2355}{*\textsuperscript{5}}}
``a loyal Frenchman .~.~. French by birth,''

\protect\hypertarget{23_NOTES.xhtmlux5cux23id_2358}{\protect\hyperlink{21_Chapter_Thirteen__IMAGE_AND_WORD.xhtmlux5cux23id_2357}{†\textsuperscript{6}}}
``Flemish born, though writing in French.''

\protect\hypertarget{23_NOTES.xhtmlux5cux23id_2360}{\protect\hyperlink{21_Chapter_Thirteen__IMAGE_AND_WORD.xhtmlux5cux23id_2359}{‡\textsuperscript{7}}}
``his coarse speech .~.~. a Flemish man, a man of the cattle-breeding
marshes, rude, ignorant, stammering of tongue, greasy of mouth and of
palate and quite bemired with other physical defects, proper to the
nature of the land.''

\protect\hypertarget{23_NOTES.xhtmlux5cux23id_2362}{\protect\hyperlink{21_Chapter_Thirteen__IMAGE_AND_WORD.xhtmlux5cux23id_2361}{§\textsuperscript{8}}}
``this fat bell with the loud sound.''

\protect\hypertarget{23_NOTES.xhtmlux5cux23id_2364}{\protect\hyperlink{21_Chapter_Thirteen__IMAGE_AND_WORD.xhtmlux5cux23id_2363}{*\textsuperscript{9}}}
``The duke then, on a Monday, which was Saint Anthony's day after mass,
being very desirous that his house should remain peaceful and without
dissensions between his servants and that his son, too, should do his
will and pleasure, after he had already said a great part of his hours,
and the chapel was empty of people, called his son to come to him and
said to him gently: `Charles, the quarrel that is going on between the
lords of Sempy and of Hémeries, about this office of chamberlain, I wish
that you put a stop to it, and that the lord of Sempy obtain the
vacancy. ' Then said the count: `Monseigneur, you once gave me your
orders in which the lord of Sempy is not mentioned, and monseigneur, if
you please, I pray you, that I may keep to them.'---'Déa,' this said the
duke then, `do not trouble yourself about orders, it belongs to me to
raise and to lower, I wish that the lord of Sempy be placed
there.'---'Hahan!' this said the count (for he always swore like that),
`Monseigneur, I beg you, forgive me, for I could not do it, I abide by
what you have ordered me. This was done by my lord of Croy, who played
me this trick, I can see that.'---'How,' this said the duke, `will you
disobey me? will you not do what I wish?'---'Monseigneur, I shall gladly
obey you, but I shall not do this.' And the duke, at these words,
choking with anger, replied: `Há! boy, will you disobey my will? Go out
of my sight, ' and the blood with these words rushing to his heart, he
turned pale and then all at once flushed and there came such a horrible
expression on his face, as I heard from the clerk of the chapel, who was
alone with him, that it was hideous to look at him .~.~. ''

\protect\hypertarget{23_NOTES.xhtmlux5cux23id_2714}{\protect\hyperlink{21_Chapter_Thirteen__IMAGE_AND_WORD.xhtmlux5cux23id_2713}{*\textsuperscript{10}}}
``Caron, open the door for us,''

\protect\hypertarget{23_NOTES.xhtmlux5cux23id_2716}{\protect\hyperlink{21_Chapter_Thirteen__IMAGE_AND_WORD.xhtmlux5cux23id_2715}{†\textsuperscript{11}}}
``Faith, madam, monseigneur has forbidden me to come into his sight and
is indignant at me, so that, after this prohibition, I shall not return
to him so soon, but under God's care, I shall go away, I do not know
where.''

\protect\hypertarget{23_NOTES.xhtmlux5cux23id_2718}{\protect\hyperlink{21_Chapter_Thirteen__IMAGE_AND_WORD.xhtmlux5cux23id_2717}{‡\textsuperscript{12}}}
``My friend, now, now, open the door for us so that we may leave, or we
are dead.''

\protect\hypertarget{23_NOTES.xhtmlux5cux23id_2720}{\protect\hyperlink{21_Chapter_Thirteen__IMAGE_AND_WORD.xhtmlux5cux23id_2719}{§\textsuperscript{13}}}
``The days were short at that time, and it was already evening when that
prince here mounted his horse, and asked nothing but to be alone out in
the fields. It so happened that on that day after a long and sharp frost
it had begun to thaw, and because of a lasting thick fog that had been
about all day, in the evening a fine but very penetrating rain began to
fall, which soaked the fields and broke the ice, as did the wind that
joined in.''

\protect\hypertarget{23_NOTES.xhtmlux5cux23id_2722}{\protect\hyperlink{21_Chapter_Thirteen__IMAGE_AND_WORD.xhtmlux5cux23id_2721}{*\textsuperscript{14}}}
``But the more he approached it, the more it seemed a hideous and
frightful thing, for fire came out of a mound in more than a thousand
places with thick smoke, and, at that hour, anyone would think that it
was the purgatory of some soul or some other illusion of the devil.''

\protect\hypertarget{23_NOTES.xhtmlux5cux23id_2724}{\protect\hyperlink{21_Chapter_Thirteen__IMAGE_AND_WORD.xhtmlux5cux23id_2723}{†\textsuperscript{15}}}
``a multitude of faces in rusty helmets, framing the grinning beards of
villains, biting their lips.''

\protect\hypertarget{23_NOTES.xhtmlux5cux23id_2726}{\protect\hyperlink{21_Chapter_Thirteen__IMAGE_AND_WORD.xhtmlux5cux23id_2725}{*\textsuperscript{16}}}
``Then he heard the news that their town was taken. `And by what
people?' he asks. Those with whom he was speaking answered, `They are
Bretons!' `Ha,' says he, `Bretons are bad people, they will pillage and
burn and afterwards depart.' `And what war-cry do they cry?' said the
knight. `Sure my lord, they cry La Trimouille!'\,''

\protect\hypertarget{23_NOTES.xhtmlux5cux23id_2728}{\protect\hyperlink{21_Chapter_Thirteen__IMAGE_AND_WORD.xhtmlux5cux23id_2727}{†\textsuperscript{17}}}
``My lord, Gaston is dead.'' ``Dead?'' said the count. ``Indeed, he is
dead in sooth, my lord.''

\protect\hypertarget{23_NOTES.xhtmlux5cux23id_2730}{\protect\hyperlink{21_Chapter_Thirteen__IMAGE_AND_WORD.xhtmlux5cux23id_2729}{‡\textsuperscript{18}}}
``So he asked for counsel in matters of love and lineage. The archbishop
answered, `Counsel, sure, good nephew, it is too late for that. You want
to shut the stable when the horse is lost.'\,''

\protect\hypertarget{23_NOTES.xhtmlux5cux23id_2732}{\protect\hyperlink{21_Chapter_Thirteen__IMAGE_AND_WORD.xhtmlux5cux23id_2731}{§\textsuperscript{19}}}
Death, I complain.---Of whom?---Of you./---What have I done to
you?---You have taken my lady./---That is true.---Tell me why?/---It
pleased me.---You mistook.

\protect\hypertarget{23_NOTES.xhtmlux5cux23id_2734}{\protect\hyperlink{21_Chapter_Thirteen__IMAGE_AND_WORD.xhtmlux5cux23id_2733}{*\textsuperscript{20}}}
Sire .~.~.---What do you want?---Listen .~.~.---To what---To my
case./---Speak out.---I am .~.~.---Who?---Devastated France!/---By
whom?---By you.---How?---In all estates./---You lie.---I do not.---Who
says so?---My sufferings./---What do you
suffer?---Misery.---Which?---The extremity of misery./---I do not
believe a word of it.---Evidently.---Do not say any more about
it!/---Alas! I must.---You waste time.---What a shame!/---What ill have
I done?---Against peace.---And how?/---By making war .~.~.---With
whom?---With your friends and kinsmen./---Speak more pleasingly.---I
cannot, in truth.

\protect\hypertarget{23_NOTES.xhtmlux5cux23id_2736}{\protect\hyperlink{21_Chapter_Thirteen__IMAGE_AND_WORD.xhtmlux5cux23id_2735}{*\textsuperscript{21}}}
And on the other side the peasants sing at their work/So loudly, truly
unceasing, rejoicing./Their oxen, stoutly plow/The fertile earth, which
brings forth good food;/And they call them by their names:/One,
``Fauveau,'' another, ``Grison, ``/``Brunet,'' ``Blanchet,'' Blondeau''
or Compaignon'';/They often poke them with their pointed stick,/To make
them go forward.

\protect\hypertarget{23_NOTES.xhtmlux5cux23id_2738}{\protect\hyperlink{21_Chapter_Thirteen__IMAGE_AND_WORD.xhtmlux5cux23id_2737}{*\textsuperscript{22}}}
His eldest son, the dauphin of Viennois, /Gave this spot the name of
Beauty. / And justly, for it is very delectable:/One hears the
nightingale sing there;/The river Marne surrounds it, the lofty pleasant
woods/Of the noble park may be seen swaying on the wind .~.~. /Meadows
are near, pleasure gardens,/The fine lawns, beautiful and clear
fountains,/Vineyards and arable lands,/Turning mills, plains beautiful
to view.

\protect\hypertarget{23_NOTES.xhtmlux5cux23id_2740}{\protect\hyperlink{21_Chapter_Thirteen__IMAGE_AND_WORD.xhtmlux5cux23id_2739}{*\textsuperscript{23}}}
Alas! it is said that I no longer make anything./I who formerly made
many new things;/The reason is that I have no subject matter/Of which to
make good or fine things.

\protect\hypertarget{23_NOTES.xhtmlux5cux23id_2742}{\protect\hyperlink{21_Chapter_Thirteen__IMAGE_AND_WORD.xhtmlux5cux23id_2741}{*\textsuperscript{24}}}
``wise, cool and imaginative, and farsighted in business.''

\protect\hypertarget{23_NOTES.xhtmlux5cux23id_2744}{\protect\hyperlink{21_Chapter_Thirteen__IMAGE_AND_WORD.xhtmlux5cux23id_2743}{†\textsuperscript{25}}}
``so Jean de Blois acquired the wife and the war which was to cost him
so much.''

\protect\hypertarget{23_NOTES.xhtmlux5cux23id_2746}{\protect\hyperlink{21_Chapter_Thirteen__IMAGE_AND_WORD.xhtmlux5cux23id_2745}{*\textsuperscript{26}}}
On parting from you I leave you my heart. /And I go away lamenting and
weeping/That it may serve you without ever being retracted./On parting
from you I leave you my heart/And by my soul, I shall indeed have
nothing good nor peace,/Till my return, being thus discomforted./On
parting from you I leave you my heart/And I go away lamenting and
weeping.

\protect\hypertarget{23_NOTES.xhtmlux5cux23id_2748}{\protect\hyperlink{21_Chapter_Thirteen__IMAGE_AND_WORD.xhtmlux5cux23id_2747}{†\textsuperscript{27}}}
Will you love me indeed,/Tell me, by your soul?/If I love you/More than
anything,/Will you love me indeed?/God put so much goodness/In you that
it is balm./Therefore I proclaim myself/Yours. But how much/Will you
love me indeed?

\protect\hypertarget{23_NOTES.xhtmlux5cux23id_2750}{\protect\hyperlink{21_Chapter_Thirteen__IMAGE_AND_WORD.xhtmlux5cux23id_2749}{*\textsuperscript{28}}}
You are most welcome, /My love; now embrace me and kiss me,/And how have
you been/Since your departure? Healthy and at ease/Have you always been?
Here come/Beside me, sit down and tell me/How you have been, ill or
well,/ For of this I want to have an account./---My lady, to whom I am
bound/More than any other, may it displease no one, /Know that desire so
seized me/That I never had such discomfort,/Nor did I take pleasure in
anything/Far from you. Amour, who tames hearts,/Said to me, ``Remain
faithful to me,/For of this I want to have an account.''/---So you kept
your oath to me./I thank you much for it by Saint Nicaise;/And as you
came back safe and sound,/We shall have joy enough; now be at ease/And
tell me if you know by how much/The grief you had from it exceeds/That
which my heart has suffered,/For of this I want to have an
account./---More grief than you, as I think I had,/But you, tell me
accurately/ How many kisses shall I have for it?/For of this I want to
have an account.

\protect\hypertarget{23_NOTES.xhtmlux5cux23id_2752}{\protect\hyperlink{21_Chapter_Thirteen__IMAGE_AND_WORD.xhtmlux5cux23id_2751}{†\textsuperscript{29}}}
It is a month today/Since my lover departed./My heart remains gloomy and
silent,/It is a month today./``Good-bye,'' he said, ``I am
going'';/Since then he has not spoken to me,/It is a month today.

\protect\hypertarget{23_NOTES.xhtmlux5cux23id_2754}{\protect\hyperlink{21_Chapter_Thirteen__IMAGE_AND_WORD.xhtmlux5cux23id_2753}{‡\textsuperscript{30}}}
Friend, weep no more;/For I am so touched with pity/That my heart gives
itself up/To your sweet friendship./Change your manner;/For God's sake,
be sad no longer,/And show me a cheerful face:/I desire whatever you
wish.

\protect\hypertarget{23_NOTES.xhtmlux5cux23id_2757}{\protect\hyperlink{21_Chapter_Thirteen__IMAGE_AND_WORD.xhtmlux5cux23id_2756}{*\textsuperscript{31}}}
When everybody comes back from the army/Why do you stay behind?/You know
that I pledged you/My loyal love to protect and keep.

\textsuperscript{32\protect\hypertarget{23_NOTES.xhtmlux5cux23id_2758}{\protect\hyperlink{21_Chapter_Thirteen__IMAGE_AND_WORD.xhtmlux5cux23id_2755}{†\textsuperscript{}}}}
Froissart came back from Scotland/On a horse which was gray./He led a
white greyhound on a leash./``Alas,'' said the greyhound, ``I am
tired,/Grisel, when shall we rest?/It is time we were feeding.''

\protect\hypertarget{23_NOTES.xhtmlux5cux23id_2760}{\protect\hyperlink{21_Chapter_Thirteen__IMAGE_AND_WORD.xhtmlux5cux23id_2759}{*\textsuperscript{33}}}
We are the bones of the poor dead,/Here heaped up in measured mounds,/
Broken, fractured, without rule or order .~.~.

\protect\hypertarget{23_NOTES.xhtmlux5cux23id_2762}{\protect\hyperlink{21_Chapter_Thirteen__IMAGE_AND_WORD.xhtmlux5cux23id_2761}{*\textsuperscript{34}}}
It is a strange melody/Which is not a great amusement/To people who are
ill./First the ravens let us know/For certain as soon as it is day:/They
cry aloud with all their might,/The fat and the thin bird, without
interruption./Even the sound of a drum would be better/Than the cries of
various birds,/Then come the cattle; cows, calves,/Bellowing, lowing,
all this is noxious,/When one has an empty brain,/The bells of the
church join in,/Which destroys the reason altogether/Of people who are
ill.

\protect\hypertarget{23_NOTES.xhtmlux5cux23id_2764}{\protect\hyperlink{21_Chapter_Thirteen__IMAGE_AND_WORD.xhtmlux5cux23id_2763}{*\textsuperscript{35}}}
It is a cold hostel and ill refuge/For people who are ill.

\protect\hypertarget{23_NOTES.xhtmlux5cux23id_2366}{\protect\hyperlink{21_Chapter_Thirteen__IMAGE_AND_WORD.xhtmlux5cux23id_2365}{*\textsuperscript{36}}}
``Forward, forward! turn there./I see a marvel, it seems to
me.''/---``And what, watchman, do you see there?''/``I see ten thousand
rats assembled/And a multitude of mice collecting/On the seashore .~.~.
''

\protect\hypertarget{23_NOTES.xhtmlux5cux23id_2368}{\protect\hyperlink{21_Chapter_Thirteen__IMAGE_AND_WORD.xhtmlux5cux23id_2367}{*\textsuperscript{37}}}
People ask me every day/What I think of the present times,/And I answer,
it is all honor, /Loyalty, truth and faith, /Liberality, heroism and
order, /Largesse and kindness that will advance/The common good; but by
my faith, /I do not say what I think.

\protect\hypertarget{23_NOTES.xhtmlux5cux23id_2370}{\protect\hyperlink{21_Chapter_Thirteen__IMAGE_AND_WORD.xhtmlux5cux23id_2369}{†\textsuperscript{38}}}
``Take all these points the other way about.''

\protect\hypertarget{23_NOTES.xhtmlux5cux23id_2372}{\protect\hyperlink{21_Chapter_Thirteen__IMAGE_AND_WORD.xhtmlux5cux23id_2371}{‡\textsuperscript{39}}}
``It is a great sin to reproach the world in this manner.''

\protect\hypertarget{23_NOTES.xhtmlux5cux23id_2374}{\protect\hyperlink{21_Chapter_Thirteen__IMAGE_AND_WORD.xhtmlux5cux23id_2373}{§\textsuperscript{40}}}
Prince, if it is generally everywhere/As I know, all virtue abounds;/But
many a man hearing me will say, ``He lies .~.~. ''

\protect\hypertarget{23_NOTES.xhtmlux5cux23id_2376}{\protect\hyperlink{21_Chapter_Thirteen__IMAGE_AND_WORD.xhtmlux5cux23id_2375}{**\textsuperscript{41}}}
``under a bad picture done in bad colors and by the most paltry painter
in the world, in an ironical manner by master Jean Robertet.''

\protect\hypertarget{23_NOTES.xhtmlux5cux23id_2870}{\protect\hyperlink{21_Chapter_Thirteen__IMAGE_AND_WORD.xhtmlux5cux23id_2869}{*\textsuperscript{42}}}
``out of the depths.''

\protect\hypertarget{23_NOTES.xhtmlux5cux23id_2868}{\protect\hyperlink{21_Chapter_Thirteen__IMAGE_AND_WORD.xhtmlux5cux23id_2867}{†\textsuperscript{43}}}
``lover who has been turned away''

\protect\hypertarget{23_NOTES.xhtmlux5cux23id_2866}{\protect\hyperlink{21_Chapter_Thirteen__IMAGE_AND_WORD.xhtmlux5cux23id_2865}{‡\textsuperscript{44}}}
``I laugh through tears''

\protect\hypertarget{23_NOTES.xhtmlux5cux23id_2864}{\protect\hyperlink{21_Chapter_Thirteen__IMAGE_AND_WORD.xhtmlux5cux23id_2863}{§\textsuperscript{45}}}
``Even in laughter the heart may be sorrowful; and the end of mirth is
heaviness.''

\protect\hypertarget{23_NOTES.xhtmlux5cux23id_2862}{\protect\hyperlink{21_Chapter_Thirteen__IMAGE_AND_WORD.xhtmlux5cux23id_2861}{**\textsuperscript{46}}}
I don't have a mouth which could laugh,/Without my eyes belying it:/For
the heart would deny it/By the tears issuing from the eyes.

\protect\hypertarget{23_NOTES.xhtmlux5cux23id_2860}{\protect\hyperlink{21_Chapter_Thirteen__IMAGE_AND_WORD.xhtmlux5cux23id_2859}{††\textsuperscript{47}}}
He constrained himself to be cheerful/And showed a feigned joy,/And
forced his heart to sing/Not from pleasure, but from fear,/For ever a
reminder of complaint/Entwined itself with the tone of his voice,/And he
returned to his suffering/As the Ousel returns to his song in the wood.

\protect\hypertarget{23_NOTES.xhtmlux5cux23id_2858}{\protect\hyperlink{21_Chapter_Thirteen__IMAGE_AND_WORD.xhtmlux5cux23id_2857}{*\textsuperscript{48}}}
This book meant to speak and to describe/To pass the time without vulgar
mood/A simple clerk called Alain/Who speaks of love by hearsay.

\protect\hypertarget{23_NOTES.xhtmlux5cux23id_2856}{\protect\hyperlink{21_Chapter_Thirteen__IMAGE_AND_WORD.xhtmlux5cux23id_2855}{†\textsuperscript{49}}}
So he told me smiling/That I should sleep/And be not at all afraid/That
I should die of this evil.

\protect\hypertarget{23_NOTES.xhtmlux5cux23id_2854}{\protect\hyperlink{21_Chapter_Thirteen__IMAGE_AND_WORD.xhtmlux5cux23id_2853}{‡\textsuperscript{50}}}
I am the one whose heart is draped in black .~.~.

\protect\hypertarget{23_NOTES.xhtmlux5cux23id_2852}{\protect\hyperlink{21_Chapter_Thirteen__IMAGE_AND_WORD.xhtmlux5cux23id_2851}{*\textsuperscript{51}}}
``One day I was talking with my heart/Which secretly spoke to me,/And in
talking I asked it/If it had saved/Any goods while serving Amour:/It
said quite willingly/It would tell me the truth about it,/As soon as it
had consulted its papers./Having told me this it went away/And from me
departed./Then I saw it enter/In an accounts office it had:/There it
rummaged here and there,/In looking for several old writing books,/For
it would show me the truth,/As soon as it had consulted its papers .~.~.
''

\protect\hypertarget{23_NOTES.xhtmlux5cux23id_2850}{\protect\hyperlink{21_Chapter_Thirteen__IMAGE_AND_WORD.xhtmlux5cux23id_2849}{†\textsuperscript{52}}}
Do not knock at the door of my mind anymore,/Anxiety and Care, do not
trouble yourselves;/For it sleeps and does not want to wake, /It has
passed all night in torment. /It will be in danger, if not well
nursed;/Stop, stop, let it slumber;/Do not knock at the door of my mind
anymore,/Anxiety and Care, do not trouble yourselves.

\protect\hypertarget{23_NOTES.xhtmlux5cux23id_2848}{\protect\hyperlink{21_Chapter_Thirteen__IMAGE_AND_WORD.xhtmlux5cux23id_2847}{*\textsuperscript{53}}}
``God protect you.''

\protect\hypertarget{23_NOTES.xhtmlux5cux23id_2846}{\protect\hyperlink{21_Chapter_Thirteen__IMAGE_AND_WORD.xhtmlux5cux23id_2845}{†\textsuperscript{54}}}
And then, when I heard the window/Of the house clatter,/Then it seemed
to me that my prayers /Had been heard by her.

\protect\hypertarget{23_NOTES.xhtmlux5cux23id_2844}{\protect\hyperlink{21_Chapter_Thirteen__IMAGE_AND_WORD.xhtmlux5cux23id_2843}{*\textsuperscript{55}}}
So help me God, I was so ravished,/That I was scarcely conscious,/For,
without being told, it seemed to me/that the wind moved her window/And
she could well have recognized me,/Perhaps saying softly: ``Good night,
then,''/And God knows I felt like a great master/After this all night.

\protect\hypertarget{23_NOTES.xhtmlux5cux23id_2842}{\protect\hyperlink{21_Chapter_Thirteen__IMAGE_AND_WORD.xhtmlux5cux23id_2841}{†\textsuperscript{56}}}
I felt so refreshed/That, without turning or tossing,/I enjoyed golden
slumber, /Without waking up all night,/And then, before dressing,/To
praise Amour for it,/I kissed my pillow thrice,/While laughing to myself
at the angels.

\protect\hypertarget{23_NOTES.xhtmlux5cux23id_2840}{\protect\hyperlink{21_Chapter_Thirteen__IMAGE_AND_WORD.xhtmlux5cux23id_2839}{‡\textsuperscript{57}}}
The others, to hide their affliction/Controlled their hearts by
force,/Passing the time closing and opening /The breviaries they held in
their hands,/Of which they turned the leaves/As a sign of devotion;/But
by their sorrow and tears they/ Clearly showed their emotion.

\protect\hypertarget{23_NOTES.xhtmlux5cux23id_2838}{\protect\hyperlink{21_Chapter_Thirteen__IMAGE_AND_WORD.xhtmlux5cux23id_2837}{*\textsuperscript{58}}}
Sweet eyes that move back and forth;/Sweet eyes enwarming the skin,/Of
those who fall in love .~.~. /Sweet eyes of pearly clearness,/That say:
I am ready when you please,/to those who feel those eyes powerfully
.~.~.

\protect\hypertarget{23_NOTES.xhtmlux5cux23id_2836}{\protect\hyperlink{21_Chapter_Thirteen__IMAGE_AND_WORD.xhtmlux5cux23id_2835}{*\textsuperscript{59}}}
``I am weary of hoping so long. Where is he who holds his heart open?''

\protect\hypertarget{23_NOTES.xhtmlux5cux23id_2834}{\protect\hyperlink{21_Chapter_Thirteen__IMAGE_AND_WORD.xhtmlux5cux23id_2833}{†\textsuperscript{60}}}
``And it can't be laid to greed that they left them only in their pants
since they were worth only four deniers---which was terribly gruesome
and one of the greatest Christian inhumanities against one's neighbor
that it is possible to imagine.''

\protect\hypertarget{23_NOTES.xhtmlux5cux23id_2832}{\protect\hyperlink{21_Chapter_Thirteen__IMAGE_AND_WORD.xhtmlux5cux23id_2831}{*\textsuperscript{61}}}
``completely nude and with disheveled hair as they are painted''

\protect\hypertarget{23_NOTES.xhtmlux5cux23id_2830}{\protect\hyperlink{21_Chapter_Thirteen__IMAGE_AND_WORD.xhtmlux5cux23id_2829}{†\textsuperscript{62}}}
``And there were also three very handsome girls, representing quite
naked sirens, and one saw their beautiful erected, separate, round and
hard breasts, which was a very pleasant sight, and they recited little
motets and bergerettes; and near them several deep-toned instruments
were playing fine melodies.''

\protect\hypertarget{23_NOTES.xhtmlux5cux23id_2828}{\protect\hyperlink{21_Chapter_Thirteen__IMAGE_AND_WORD.xhtmlux5cux23id_2827}{‡\textsuperscript{63}}}
``but the stand at which the people looked with the greatest pleasure
was the history of the three goddesses represented nude by living
women.''

\protect\hypertarget{23_NOTES.xhtmlux5cux23id_2826}{\protect\hyperlink{21_Chapter_Thirteen__IMAGE_AND_WORD.xhtmlux5cux23id_2825}{*\textsuperscript{64}}}
``This dame here is said to have acrid conditions and very tart and
biting reasons; she ground her teeth and bit her lips; often nodded her
head; and indicated by gesture that she was arguing, jumped on her feet
and turned to this side and to that; she proved to be impatient and
inclined to contradict; the right eye was closed and the other open; she
had a bag full of books before her, of which she put some into her
girdle, as if they were dear to her, the others she thew away
spitefully; she tore up papers and leaves; she threw writing books into
the fire furiously; she smiled on some and kissed them; she spat on
others out of meanness and trod them underfoot; she had a pen in her
hand, full of ink, with which she crossed out many important writings
.~.~. ; also with a sponge she blackened some pictures, she scratched
out others with her nails, and others she erased wholly and smoothed
them as if to have them forgotten; and showed herself a hard and fell
enemy to many respectable people, more arbitrarily than reasonably.''

\protect\hypertarget{23_NOTES.xhtmlux5cux23id_2824}{\protect\hyperlink{21_Chapter_Thirteen__IMAGE_AND_WORD.xhtmlux5cux23id_2823}{†\textsuperscript{65}}}
``Peace of Heart, Peace of Mouth, Seeming Peace, Peace of True Effect.''

\protect\hypertarget{23_NOTES.xhtmlux5cux23id_2822}{\protect\hyperlink{21_Chapter_Thirteen__IMAGE_AND_WORD.xhtmlux5cux23id_2821}{‡\textsuperscript{66}}}
``Importance of your Lands, Various qualities and conditions of your
several peoples, The Envy and Hatred of Frenchmen and of Neighboring
Nations''

\protect\hypertarget{23_NOTES.xhtmlux5cux23id_2820}{\protect\hyperlink{21_Chapter_Thirteen__IMAGE_AND_WORD.xhtmlux5cux23id_2819}{*\textsuperscript{67}}}
``the inventor and conjuror of this vision''

\protect\hypertarget{23_NOTES.xhtmlux5cux23id_2818}{\protect\hyperlink{21_Chapter_Thirteen__IMAGE_AND_WORD.xhtmlux5cux23id_2817}{†\textsuperscript{68}}}
Physician, what about Law?/---By my soul, he is poorly .~.~. /---How
does Reason? .~.~. /She is out of her mind,/She speaks but feebly,/And
Justice has become an idiot .~.~.

\protect\hypertarget{23_NOTES.xhtmlux5cux23id_2816}{\protect\hyperlink{21_Chapter_Thirteen__IMAGE_AND_WORD.xhtmlux5cux23id_2815}{*\textsuperscript{69}}}
We God of Love, creator, king of glory/All hail to all true lovers in
their humility!/As it is true that since the victory/Of our son on Mount
Calvary/ Several soldiers through lack of knowledge/Of our arms, make an
alliance with the devil .~.~.

\protect\hypertarget{23_NOTES.xhtmlux5cux23id_2814}{\protect\hyperlink{21_Chapter_Thirteen__IMAGE_AND_WORD.xhtmlux5cux23id_2813}{†\textsuperscript{70}}}
Who now in sobbing and in tears, /With a contrite heart and faithful
without deceit.

\protect\hypertarget{23_NOTES.xhtmlux5cux23id_2812}{\protect\hyperlink{21_Chapter_Thirteen__IMAGE_AND_WORD.xhtmlux5cux23id_2811}{‡\textsuperscript{71}}}
``And so Sluys remained in peace that was included with her, for war was
excluded from her, lonelier than a recluse.''

\protect\hypertarget{23_NOTES.xhtmlux5cux23id_2810}{\protect\hyperlink{21_Chapter_Thirteen__IMAGE_AND_WORD.xhtmlux5cux23id_2809}{*\textsuperscript{72}}}
``And lest I lose the wheat of my labor, and that the flour into which
it will be ground may have wholesome blossom, I intend, if God gives me
grace for it, to turn and convert under my rough millstones the vicious
into the virtuous, the corporeal into the spiritual, the worldly into
the divine, and, above all, to moralize it. And in this way we shall
gather honey from the hard stone and the vermeil rose from sharp thorns,
where we shall find grains and seed, fruit, flower and leaf, very sweet
fragrance, sweet-smelling verdure, verdant florescence, flourishing
nurture, nourishing fruit and fruitful pasture.''

\protect\hypertarget{23_NOTES.xhtmlux5cux23id_2808}{\protect\hyperlink{21_Chapter_Thirteen__IMAGE_AND_WORD.xhtmlux5cux23id_2807}{†\textsuperscript{73}}}
Then I am gripped with the fever of thought/And the catarrh of
displeasure, / A migraine of sorrow,/Colic of impatience,/Unbearable
toothache./My heart could no longer take it,/The regrets of my
fate/Through unaccustomed sorrow.

\protect\hypertarget{23_NOTES.xhtmlux5cux23id_2806}{\protect\hyperlink{21_Chapter_Thirteen__IMAGE_AND_WORD.xhtmlux5cux23id_2805}{‡\textsuperscript{74}}}
``to eat his fill.''

\protect\hypertarget{23_NOTES.xhtmlux5cux23id_307}{\protect\hyperlink{21_Chapter_Thirteen__IMAGE_AND_WORD.xhtmlux5cux23id_306}{1}}.
Concerning this problem see my Renaissancestudiën I: Het probleem, de
Gids, 1920, IV.

\protect\hypertarget{23_NOTES.xhtmlux5cux23id_305}{\protect\hyperlink{21_Chapter_Thirteen__IMAGE_AND_WORD.xhtmlux5cux23id_304}{2}}.
La Renaissance septentrionale et les premiers maîtres des Flandres,
Bruxelles 1905.

\protect\hypertarget{23_NOTES.xhtmlux5cux23id_303}{\protect\hyperlink{21_Chapter_Thirteen__IMAGE_AND_WORD.xhtmlux5cux23id_302}{3}}.
Erasmus, Ratio seu Methodus compendio perveniendi ad veram theologiam,
ed. Basel 1520, p. 146.

\protect\hypertarget{23_NOTES.xhtmlux5cux23id_301}{\protect\hyperlink{21_Chapter_Thirteen__IMAGE_AND_WORD.xhtmlux5cux23id_300}{4}}.
E. Durand-Gréville, Hubert et Jean van Eyck, Bruxelles, 1910, p. 119.

\protect\hypertarget{23_NOTES.xhtmlux5cux23page_435}{\protect\hyperlink{21_Chapter_Thirteen__IMAGE_AND_WORD.xhtmlux5cux23id_299}{5}}.
{[}Trans.{]} In the English translation done by F. Hopman under
Huizinga's supervision this paragraph is replaced by the following:

Are not unity and harmony lost in this aggregation of details as
Michelangelo affirmed of Flemish art in general? Having recently seen
the picture again, I can no longer deny it as I formerly did on the
strength of recollections many years old.

If this reflects a true change of position on the issue on the part of
Huizinga, it is strange that the alteration did not make it into any
subsequent Dutch editions.

\protect\hypertarget{23_NOTES.xhtmlux5cux23id_298}{\protect\hyperlink{21_Chapter_Thirteen__IMAGE_AND_WORD.xhtmlux5cux23id_297}{6}}.
{[}Trans.{]} \emph{Unbridled elaboration}: The Dutch is
\emph{ongebreidelde uitwerking}; German, \emph{zügellose Detaillierung}.

\protect\hypertarget{23_NOTES.xhtmlux5cux23id_296}{\protect\hyperlink{21_Chapter_Thirteen__IMAGE_AND_WORD.xhtmlux5cux23id_295}{7}}.
P. 251.

\protect\hypertarget{23_NOTES.xhtmlux5cux23id_294}{\protect\hyperlink{21_Chapter_Thirteen__IMAGE_AND_WORD.xhtmlux5cux23id_293}{8}}.
Alain Chartier, Oeuvres, ed. Duchesne, p. 594.

\protect\hypertarget{23_NOTES.xhtmlux5cux23id_292}{\protect\hyperlink{21_Chapter_Thirteen__IMAGE_AND_WORD.xhtmlux5cux23id_291}{9}}.
Chastellain, I pp. 11, 12, IV pp. 21, 393, VII pp. 160; La Marche, I, p.
14; Molinet, I, p. 23.

\protect\hypertarget{23_NOTES.xhtmlux5cux23id_290}{\protect\hyperlink{21_Chapter_Thirteen__IMAGE_AND_WORD.xhtmlux5cux23id_289}{10}}.
{[}Trans.{]} \emph{cothurnism}: see chap. 2, n. 65

\protect\hypertarget{23_NOTES.xhtmlux5cux23id_288}{\protect\hyperlink{21_Chapter_Thirteen__IMAGE_AND_WORD.xhtmlux5cux23id_287}{11}}.
Jean Robertet, in Chastellain, VII, p. 182.

\protect\hypertarget{23_NOTES.xhtmlux5cux23id_286}{\protect\hyperlink{21_Chapter_Thirteen__IMAGE_AND_WORD.xhtmlux5cux23id_285}{12}}.
Chastellain, VII, p. 219.

\protect\hypertarget{23_NOTES.xhtmlux5cux23id_284}{\protect\hyperlink{21_Chapter_Thirteen__IMAGE_AND_WORD.xhtmlux5cux23id_283}{13}}.
Chastellain, III, pp. 231ff.---Saint Anthony's day is January 17.

\protect\hypertarget{23_NOTES.xhtmlux5cux23id_282}{\protect\hyperlink{21_Chapter_Thirteen__IMAGE_AND_WORD.xhtmlux5cux23id_281}{14}}.
Oratory, a carpeted and secluded little corner of a chapel.

\protect\hypertarget{23_NOTES.xhtmlux5cux23id_280}{\protect\hyperlink{21_Chapter_Thirteen__IMAGE_AND_WORD.xhtmlux5cux23id_279}{15}}.
{[}Trans.{]} \emph{prie-Dieu}: Dutch \emph{bidstoel}; German,
\emph{Betstuhl}. Literally, ``prayer stool.'' The piece of furniture is
most likely similar to that on which the donor kneels in the
\emph{Madonna of Chancellor Rolin}.

\protect\hypertarget{23_NOTES.xhtmlux5cux23id_278}{\protect\hyperlink{21_Chapter_Thirteen__IMAGE_AND_WORD.xhtmlux5cux23id_277}{16}}.
{[}Trans.{]} \emph{Camille Lemonnier}. A Belgian novelist who was active
during the last decade of the nineteenth century. In his memoirs,
Huizinga says he was strongly influenced by modern literary movements.
See his \emph{My Path of History}, page 253, where he says:

{[}S{]}oon after we enrolled ten of us from the class of 1891 formed a
club. .~.~. We were all enthusiastic supporters of the
\emph{Tachtigers}, a literary movement round the journal \emph{De Nieuwe
Gids} (1885), and consequently rated literature far higher than science,
sought the meaning of life within our selves (which was a great
blessing) and completely ignored politics and allied topics (which was a
grave fault). Throughout my student years I never took a newspaper. We
looked up to Van Deijssel, Kloos, Gorter \emph{et al}. as to so many
demigods. In the comfortable reading room of \emph{Mutua Fides}, we not
only followed the Kloos crisis in \emph{De Nieuwe Gids} month by month,
and dutifully denounced Van Eeden, but also devoured the \emph{Mercure
de France}, watched Pierre Louys's star rise by side of Rémy de
Gourmont's, and finally hailed Alfred Jarry's \emph{succès de
scandal}---in short, we took a most one-sided view of what was happening
in literature, even though Edgar Allan Poe, Robert Louis Stevenson,
Dante Gabriel Rossetti and many other authors made a great impression on
us as well. In later years our enthusiasm for the \emph{Tachtigers} was
jolted by the appearance of P. L. Tak's \emph{De Kroniek}, in which our
own contemporaries, among them Jan Kalf and the talented and precocious
André Jolles, put forward their views. \emph{At the time, literary ideas
had begun to have a profound effect on me}. {[}Emphasis added.{]}

\protect\hypertarget{23_NOTES.xhtmlux5cux23page_436}{\protect\hyperlink{21_Chapter_Thirteen__IMAGE_AND_WORD.xhtmlux5cux23id_276}{17}}.
Chastellain, III, p. 46; see above p. 109; and see Chastellain, III p.
104, V p. 259.

\protect\hypertarget{23_NOTES.xhtmlux5cux23id_275}{\protect\hyperlink{21_Chapter_Thirteen__IMAGE_AND_WORD.xhtmlux5cux23id_274}{18}}.
Chastellain, V, pp. 273, 269, 271.

\protect\hypertarget{23_NOTES.xhtmlux5cux23id_273}{\protect\hyperlink{21_Chapter_Thirteen__IMAGE_AND_WORD.xhtmlux5cux23id_272}{19}}.
See the reproduction in E. Chmelarz, Jahrb. der Kunsthist. Samml. des
allerh. Kaiserhauses XI, 1890; and P. Durrieu, Les belles heures du duc
de Berry, Gazette des beaux arts, 1906, t. 35, p. 283.

\protect\hypertarget{23_NOTES.xhtmlux5cux23id_271}{\protect\hyperlink{21_Chapter_Thirteen__IMAGE_AND_WORD.xhtmlux5cux23id_270}{20}}.
Froissart, ed. Kervyn, XIII, p. 50, XI, p. 99, XIII, p. 4.

\protect\hypertarget{23_NOTES.xhtmlux5cux23id_269}{\protect\hyperlink{21_Chapter_Thirteen__IMAGE_AND_WORD.xhtmlux5cux23id_268}{21}}.
Unknown poet printed in Deschamps, Oeuvres, X, no. 18; see Le Débat du
cuer et du corps de Villon, and Charles d'Orléans, rondel 192.

\protect\hypertarget{23_NOTES.xhtmlux5cux23id_267}{\protect\hyperlink{21_Chapter_Thirteen__IMAGE_AND_WORD.xhtmlux5cux23id_266}{22}}.
Ed. de 1522, fol. 101, in A. de la Borderie, Jean Meschinot etc., Bibl.
de l'Ecole des chartes LVI, 1895, p. 301. See die ballads von Henri
Baude, ed. Quicherat (Trésor des pieces rares ou inédites), Paris, pp.
26, 37, 55, 79.

\protect\hypertarget{23_NOTES.xhtmlux5cux23id_265}{\protect\hyperlink{21_Chapter_Thirteen__IMAGE_AND_WORD.xhtmlux5cux23id_264}{23}}.
Froissart, ed. Luce, I pp. 56, 66, 71, XI p. 13, ed. Kervyn, XII pp. 2,
23; see also Deschamps, III, p. 42.

\protect\hypertarget{23_NOTES.xhtmlux5cux23id_263}{\protect\hyperlink{21_Chapter_Thirteen__IMAGE_AND_WORD.xhtmlux5cux23id_262}{24}}.
Froissart, ed. Kervyn, XI, p. 89.

\protect\hypertarget{23_NOTES.xhtmlux5cux23id_261}{\protect\hyperlink{21_Chapter_Thirteen__IMAGE_AND_WORD.xhtmlux5cux23id_260}{25}}.
Durrieu, Les très-riches heures de Jean de France duc de Berry, 1904,
pi. 38.

\protect\hypertarget{23_NOTES.xhtmlux5cux23id_259}{\protect\hyperlink{21_Chapter_Thirteen__IMAGE_AND_WORD.xhtmlux5cux23id_258}{26}}.
Oeuvres du roi René, ed. Quatrebarbes, II, p. 105.

\protect\hypertarget{23_NOTES.xhtmlux5cux23id_257}{\protect\hyperlink{21_Chapter_Thirteen__IMAGE_AND_WORD.xhtmlux5cux23id_256}{27}}.
Deschamps, I, nos. 61, 144; III, nos. 454, 483, 524; IV, nos. 617, 636.

\protect\hypertarget{23_NOTES.xhtmlux5cux23id_255}{\protect\hyperlink{21_Chapter_Thirteen__IMAGE_AND_WORD.xhtmlux5cux23id_254}{28}}.
Durrieu, Les très-riches heures de Jean de France duc de Berry, pls. 3,
9, 12.

\protect\hypertarget{23_NOTES.xhtmlux5cux23id_253}{\protect\hyperlink{21_Chapter_Thirteen__IMAGE_AND_WORD.xhtmlux5cux23id_252}{29}}.
Deschamps, VI, p. 191, no. 1204.

\protect\hypertarget{23_NOTES.xhtmlux5cux23id_251}{\protect\hyperlink{21_Chapter_Thirteen__IMAGE_AND_WORD.xhtmlux5cux23id_250}{30}}.
Froissart, ed. Luce, V p. 64, VIII pp. 5, 48, XI p. 110; ed. Kervyn,
XIII pp. 14, 21, 84, 102, 264.

\protect\hypertarget{23_NOTES.xhtmlux5cux23id_249}{\protect\hyperlink{21_Chapter_Thirteen__IMAGE_AND_WORD.xhtmlux5cux23id_248}{31}}.
Froissart, ed. Kervyn, XV pp. 54, 109, 184; XVI pp. 23, 52; ed. Luce, I
p. 394.

\protect\hypertarget{23_NOTES.xhtmlux5cux23id_247}{\protect\hyperlink{21_Chapter_Thirteen__IMAGE_AND_WORD.xhtmlux5cux23id_246}{32}}.
Froissart, XIII, p. 13.

\protect\hypertarget{23_NOTES.xhtmlux5cux23id_245}{\protect\hyperlink{21_Chapter_Thirteen__IMAGE_AND_WORD.xhtmlux5cux23id_244}{33}}.
G. de Machaut, Poésies lyriques, ed. V. Chichmaref (Zapiski ist. fil.
fakulteta imp. S. Peterb. universiteta XCII, 1909) no. 60, I, p. 74.

\protect\hypertarget{23_NOTES.xhtmlux5cux23id_243}{\protect\hyperlink{21_Chapter_Thirteen__IMAGE_AND_WORD.xhtmlux5cux23id_242}{34}}.
La Borderie, Jean Meschinot etc., p. 618.

\protect\hypertarget{23_NOTES.xhtmlux5cux23id_241}{\protect\hyperlink{21_Chapter_Thirteen__IMAGE_AND_WORD.xhtmlux5cux23id_240}{35}}.
Christine de Pisan, Oeuvres poétiques, I, p. 276.

\protect\hypertarget{23_NOTES.xhtmlux5cux23id_239}{\protect\hyperlink{21_Chapter_Thirteen__IMAGE_AND_WORD.xhtmlux5cux23id_238}{36}}.
Ibid., I, p. 164, no. 30.

\protect\hypertarget{23_NOTES.xhtmlux5cux23id_237}{\protect\hyperlink{21_Chapter_Thirteen__IMAGE_AND_WORD.xhtmlux5cux23id_236}{37}}.
Ibid., I, p. 275, no. 5.

\protect\hypertarget{23_NOTES.xhtmlux5cux23id_235}{\protect\hyperlink{23_NOTES.xhtmlux5cux23id_235}{38}}.
Froissart, Poésies, ed. Schéler, II, p. 216.

\protect\hypertarget{23_NOTES.xhtmlux5cux23id_234}{\protect\hyperlink{21_Chapter_Thirteen__IMAGE_AND_WORD.xhtmlux5cux23id_233}{39}}.
P. Michault, La dance aux aveugles etc., Lille, 1748.

\protect\hypertarget{23_NOTES.xhtmlux5cux23id_232}{\protect\hyperlink{21_Chapter_Thirteen__IMAGE_AND_WORD.xhtmlux5cux23id_231}{40}}.
Recueil de poésies françoises des XV\textsuperscript{e} et
XVI\textsuperscript{e} siècles, ed. de Montaiglon (Bibl. elzavirienne),
IX, p. 59.

\protect\hypertarget{23_NOTES.xhtmlux5cux23id_230}{\protect\hyperlink{21_Chapter_Thirteen__IMAGE_AND_WORD.xhtmlux5cux23id_229}{41}}.
Deschamps, VI, no. 1202, p. 188.

\protect\hypertarget{23_NOTES.xhtmlux5cux23id_228}{\protect\hyperlink{21_Chapter_Thirteen__IMAGE_AND_WORD.xhtmlux5cux23id_227}{42}}.
Froissart, Poésies, I, p. 91.

\protect\hypertarget{23_NOTES.xhtmlux5cux23id_226}{\protect\hyperlink{21_Chapter_Thirteen__IMAGE_AND_WORD.xhtmlux5cux23id_225}{43}}.
Froissart, ed. Kervyn, XIII, p. 22.

\protect\hypertarget{23_NOTES.xhtmlux5cux23id_224}{\protect\hyperlink{21_Chapter_Thirteen__IMAGE_AND_WORD.xhtmlux5cux23id_223}{44}}.
Deschamps, I, p. 196, no. 90; p. 192, no. 87; IV, p. 294, no. 788; V, p.
94 no. 903, p. 97 no. 905, p. 121 no. 919; VII, p. 220, no. 1375. See
II, p. 86, no. 247, no. 250.

\protect\hypertarget{23_NOTES.xhtmlux5cux23id_222}{\protect\hyperlink{21_Chapter_Thirteen__IMAGE_AND_WORD.xhtmlux5cux23id_221}{45}}.
Durrieu, Les très-riches heures, pls. 38, 39, 60, 27, 28.

\protect\hypertarget{23_NOTES.xhtmlux5cux23id_220}{\protect\hyperlink{21_Chapter_Thirteen__IMAGE_AND_WORD.xhtmlux5cux23id_219}{46}}.
Deschamps, V, p. 351, no. 1060; V, p. 15, no. 844.

\protect\hypertarget{23_NOTES.xhtmlux5cux23id_218}{\protect\hyperlink{21_Chapter_Thirteen__IMAGE_AND_WORD.xhtmlux5cux23id_217}{47}}.
Chastellain, III, pp. 256ff.

\protect\hypertarget{23_NOTES.xhtmlux5cux23id_216}{\protect\hyperlink{21_Chapter_Thirteen__IMAGE_AND_WORD.xhtmlux5cux23id_215}{48}}.
Journal d'un bourgeois, p. 3252.

\protect\hypertarget{23_NOTES.xhtmlux5cux23page_437}{\protect\hyperlink{21_Chapter_Thirteen__IMAGE_AND_WORD.xhtmlux5cux23id_214}{49}}.
Deschamps, nos. 1229, 1230, 1233, 1259, 1299, 1300, 1477, VI pp. 230,
232, 237, 279, VII pp. 52, 54, VIII p. 182; see Gaguin's De validorum
mendicantium astucia, Thuasne, II, pp. 169ff.

\protect\hypertarget{23_NOTES.xhtmlux5cux23id_213}{\protect\hyperlink{21_Chapter_Thirteen__IMAGE_AND_WORD.xhtmlux5cux23id_212}{50}}.
Deschamps, no. 219, II, p. 44, no. 2, p. 71.

\protect\hypertarget{23_NOTES.xhtmlux5cux23id_211}{\protect\hyperlink{21_Chapter_Thirteen__IMAGE_AND_WORD.xhtmlux5cux23id_210}{51}}.
Ibid. IV, p. 291, no. 786.

\protect\hypertarget{23_NOTES.xhtmlux5cux23id_209}{\protect\hyperlink{21_Chapter_Thirteen__IMAGE_AND_WORD.xhtmlux5cux23id_208}{52}}.
Bibliothèque de l'école des chartes, 2\textsuperscript{e} série III
1846, p. 70.

\protect\hypertarget{23_NOTES.xhtmlux5cux23id_207}{\protect\hyperlink{21_Chapter_Thirteen__IMAGE_AND_WORD.xhtmlux5cux23id_206}{53}}.
Proverbs 14:13.

\protect\hypertarget{23_NOTES.xhtmlux5cux23id_205}{\protect\hyperlink{21_Chapter_Thirteen__IMAGE_AND_WORD.xhtmlux5cux23id_204}{54}}.
{[}Trans.{]} The Hopman translation includes two fragments from Granson
at this point

\emph{Veillier ou lit et jeuner à table}

\emph{Rire plourant et en plaignant chanter}.

\emph{{[}Lying abed awake and fasting at the board, laughing in tears
and lamenting in song.{]}}

And:

\emph{Je prins congiè de ce tresdoulz enfant}

\emph{Les yeulx mouilliez et la bouche riant}.

\emph{{[}I took leave of this most sweet child With tearful eyes and a
laughing mouth.{]}}

\protect\hypertarget{23_NOTES.xhtmlux5cux23id_203}{\protect\hyperlink{21_Chapter_Thirteen__IMAGE_AND_WORD.xhtmlux5cux23id_202}{55}}.
Alain Chartier, La belle dame dans mercy, pp. 503, 505,; see Le débat du
reveille-matin, p. 498; Chansons du XV\textsuperscript{e} siècle, p. 71,
no. 73; L'amant rendu cordelier à l'observance d'amours, vs. 371;
Molinet, Faictz et dictz, ed. 1537, 172f.

\protect\hypertarget{23_NOTES.xhtmlux5cux23id_201}{\protect\hyperlink{21_Chapter_Thirteen__IMAGE_AND_WORD.xhtmlux5cux23id_200}{56}}.
Alain Chartier, Le débat des deux fortunes d'amours, p. 581.

\protect\hypertarget{23_NOTES.xhtmlux5cux23id_199}{\protect\hyperlink{21_Chapter_Thirteen__IMAGE_AND_WORD.xhtmlux5cux23id_198}{57}}.
Oeuvres du roi René, ed. Quatrebarbes, III, p. 194.

\protect\hypertarget{23_NOTES.xhtmlux5cux23id_197}{\protect\hyperlink{21_Chapter_Thirteen__IMAGE_AND_WORD.xhtmlux5cux23id_196}{58}}.
Charles d'Orléans, Poésies complètes, p. 68.

\protect\hypertarget{23_NOTES.xhtmlux5cux23id_195}{\protect\hyperlink{21_Chapter_Thirteen__IMAGE_AND_WORD.xhtmlux5cux23id_194}{59}}.
Charles d'Orléans, Poésies complètes, p. 88, ballade no. 19.

\protect\hypertarget{23_NOTES.xhtmlux5cux23id_193}{\protect\hyperlink{21_Chapter_Thirteen__IMAGE_AND_WORD.xhtmlux5cux23id_192}{60}}.
Charles d'Orléans, Poésies complètes, chanson no. 62.

\protect\hypertarget{23_NOTES.xhtmlux5cux23id_191}{\protect\hyperlink{21_Chapter_Thirteen__IMAGE_AND_WORD.xhtmlux5cux23id_190}{61}}.
{[}Trans.{]} \emph{Pierrot}: The use of this image here is perhaps an
indication of Huizinga's awareness of modern art and literature.
Pierrot, the clown figure from French pantomime, figures prominently in
early twentieth-century art, noticeably the paintings of Picasso and in
the setting of the \emph{Pierrot Lunaire} (``Moonstruck Pierrot'') poems
of Albert Giraud (whose work Huizinga would have known) by Arnold
Schönberg in 1912. In these poems Pierrot is touchingly crushed by love.

\protect\hypertarget{23_NOTES.xhtmlux5cux23id_189}{\protect\hyperlink{21_Chapter_Thirteen__IMAGE_AND_WORD.xhtmlux5cux23id_188}{62}}.
Compare Alain Chartier, p. 549: ``Ou se le vent une fenestre boute/Dont
il cuide que sa dame l'escoute/S'en va coucher joyeulx .~.~. ''

\protect\hypertarget{23_NOTES.xhtmlux5cux23id_187}{\protect\hyperlink{21_Chapter_Thirteen__IMAGE_AND_WORD.xhtmlux5cux23id_186}{63}}.
Huitains 51, 53, 57, 167, 188, 192, ed. de Montaiglon (Soc. des anc.
textes français), 1881.

\protect\hypertarget{23_NOTES.xhtmlux5cux23id_185}{\protect\hyperlink{21_Chapter_Thirteen__IMAGE_AND_WORD.xhtmlux5cux23id_184}{64}}.
Museum of Leipzig, no. 509.

\protect\hypertarget{23_NOTES.xhtmlux5cux23id_183}{\protect\hyperlink{21_Chapter_Thirteen__IMAGE_AND_WORD.xhtmlux5cux23id_182}{65}}.
Journal d'un bourgeois, p. 96. Prof. D. C. Hesseling has brought to my
attention that, in addition to modesty, another image is in play here;
Namely, that the dead may not appear without a shroud at the last
judgment, and he refers me to a Greek text of the seventh century
(Johannes Moschus c. 78, Migne Patrol. graecam t. LXXXVII, p. 2933 D.),
which might be a parallel to Western conceptions. On the other hand, one
should not forget that in the depictions of the resurrection of the dead
in miniatures and in paintings, the dead always come from the grave
naked.

\emph{\protect\hypertarget{23_NOTES.xhtmlux5cux23id_181}{\protect\hyperlink{21_Chapter_Thirteen__IMAGE_AND_WORD.xhtmlux5cux23id_180}{66}}}.
{[}Trans.{]} \emph{Bastard of Vauru}: In the winter of 1421--22, the
city of Meaux was
\protect\hypertarget{23_NOTES.xhtmlux5cux23page_438}{}{}besieged by
Henry V. The Bastard of Vauru was one of the city garrison who exploited
the populace, demanding ransoms and hanging those who could not pay on
``Vauru's tree.'' The woman in question was pregnant (according to the
Burgher of Paris) and hung and left to die from this tree. ``That cruel
and evil monster, the Bastard de Vauru, hearing her saying things that
annoyed him, had her beaten with sticks and then dragged off at a great
rate to his elm. He had her tied to it and bound and all of her clothes
cut off short so that she was naked as far as her navel, an inhuman
thing to do!'' \emph{A Parisian Journal 1405--1449}, trans. Janet
Shirley, Oxford: Clarendon Press, 1968.

\protect\hypertarget{23_NOTES.xhtmlux5cux23id_179}{\protect\hyperlink{21_Chapter_Thirteen__IMAGE_AND_WORD.xhtmlux5cux23id_178}{67}}.
Juvenal des Ursins, 1418, p. 541; Journal d'un bourgeois, pp. 92, 172.

\protect\hypertarget{23_NOTES.xhtmlux5cux23id_177}{\protect\hyperlink{21_Chapter_Thirteen__IMAGE_AND_WORD.xhtmlux5cux23id_176}{68}}.
J. Veth and S. Muller, A. Dürers Niederläandische Reise, Berlin-Utrecht,
1918, 2 Bde., I, p. 13.

\protect\hypertarget{23_NOTES.xhtmlux5cux23id_175}{\protect\hyperlink{21_Chapter_Thirteen__IMAGE_AND_WORD.xhtmlux5cux23id_174}{69}}.
Chastellain, III, p. 414.

\protect\hypertarget{23_NOTES.xhtmlux5cux23id_173}{\protect\hyperlink{21_Chapter_Thirteen__IMAGE_AND_WORD.xhtmlux5cux23id_172}{70}}.
Chron. scand., I, p. 27.

\protect\hypertarget{23_NOTES.xhtmlux5cux23id_171}{\protect\hyperlink{21_Chapter_Thirteen__IMAGE_AND_WORD.xhtmlux5cux23id_170}{71}}.
Molinet, V, p. 15.

\protect\hypertarget{23_NOTES.xhtmlux5cux23id_169}{\protect\hyperlink{21_Chapter_Thirteen__IMAGE_AND_WORD.xhtmlux5cux23id_168}{72}}.
Lefebvre, Theatre de Lille, p. 54, in Doutrepont, p. 354.

\protect\hypertarget{23_NOTES.xhtmlux5cux23id_167}{\protect\hyperlink{21_Chapter_Thirteen__IMAGE_AND_WORD.xhtmlux5cux23id_166}{73}}.
Th. Godefroy, Le ceremonial françois, 1649, p. 617.

\protect\hypertarget{23_NOTES.xhtmlux5cux23id_165}{\protect\hyperlink{21_Chapter_Thirteen__IMAGE_AND_WORD.xhtmlux5cux23id_164}{74}}.
J. B. Houwaert, Declaratie van die triumphante Incompst van den .~.~.
Prince van Oraingnien etc.; t'Antwerpen, Plantijn, 1579, p. 39.

\protect\hypertarget{23_NOTES.xhtmlux5cux23id_163}{\protect\hyperlink{21_Chapter_Thirteen__IMAGE_AND_WORD.xhtmlux5cux23id_162}{75}}.
The thesis of Emile Mâle concerning the influence of theatrical
representations on paintings may be left standing in this instance.

\protect\hypertarget{23_NOTES.xhtmlux5cux23id_161}{\protect\hyperlink{21_Chapter_Thirteen__IMAGE_AND_WORD.xhtmlux5cux23id_160}{76}}.
See P. Durrieu, Gazette des beaux arts, 1906, t. 35, p. 275.

\protect\hypertarget{23_NOTES.xhtmlux5cux23id_159}{\protect\hyperlink{21_Chapter_Thirteen__IMAGE_AND_WORD.xhtmlux5cux23id_158}{77}}.
Christine de Pisan, Epitre d'Othéa à Hector, Ms. 9392 de Jean Miélot,
ed. J. van den Gheyn, Bruxelles 1913.

\protect\hypertarget{23_NOTES.xhtmlux5cux23id_157}{\protect\hyperlink{21_Chapter_Thirteen__IMAGE_AND_WORD.xhtmlux5cux23id_156}{78}}.
Ibid., Pls. 5, 8, 26, 24, 25.

\protect\hypertarget{23_NOTES.xhtmlux5cux23id_155}{\protect\hyperlink{21_Chapter_Thirteen__IMAGE_AND_WORD.xhtmlux5cux23id_154}{79}}.
Christine de Pisan, Epitre d'Othéa, pls. 1 and 3; Michel, Histoire de
l'art, IV, 2, p. 603: Michel Colombe, Grabmonument aus der Kathedrale
von Nantes, p. 616: figure of Temperantia on the grave monument of the
Cardinal of Amboise in the Rouen Cathedral.

\protect\hypertarget{23_NOTES.xhtmlux5cux23id_153}{\protect\hyperlink{21_Chapter_Thirteen__IMAGE_AND_WORD.xhtmlux5cux23id_152}{80}}.
See my essay Uit de voorgeschiedenis van ons nationaal besef, De Gids,
1912, I.

\protect\hypertarget{23_NOTES.xhtmlux5cux23id_151}{\protect\hyperlink{21_Chapter_Thirteen__IMAGE_AND_WORD.xhtmlux5cux23id_150}{81}}.
Expositions sur vérité mal prise, Chastellain, VI, p. 249.

\protect\hypertarget{23_NOTES.xhtmlux5cux23id_149}{\protect\hyperlink{21_Chapter_Thirteen__IMAGE_AND_WORD.xhtmlux5cux23id_148}{82}}.
Le livre de paix, Castellain, VII, p. 375.

\protect\hypertarget{23_NOTES.xhtmlux5cux23id_147}{\protect\hyperlink{21_Chapter_Thirteen__IMAGE_AND_WORD.xhtmlux5cux23id_146}{83}}.
Advertissement au duc Charles, Chastellain, VII, pp. 304ff.

\protect\hypertarget{23_NOTES.xhtmlux5cux23id_145}{\protect\hyperlink{21_Chapter_Thirteen__IMAGE_AND_WORD.xhtmlux5cux23id_144}{84}}.
Chastellain, VII, pp. 237ff.

\protect\hypertarget{23_NOTES.xhtmlux5cux23id_143}{\protect\hyperlink{21_Chapter_Thirteen__IMAGE_AND_WORD.xhtmlux5cux23id_142}{85}}.
Molinet, Le miroir de la mort, fragment in Chastellain, VI, p. 460.

\protect\hypertarget{23_NOTES.xhtmlux5cux23id_141}{\protect\hyperlink{21_Chapter_Thirteen__IMAGE_AND_WORD.xhtmlux5cux23id_140}{86}}.
Chastellain, VII, p. 419.

\protect\hypertarget{23_NOTES.xhtmlux5cux23id_139}{\protect\hyperlink{21_Chapter_Thirteen__IMAGE_AND_WORD.xhtmlux5cux23id_138}{87}}.
Deschamps, I, p. 170.

\protect\hypertarget{23_NOTES.xhtmlux5cux23id_137}{\protect\hyperlink{21_Chapter_Thirteen__IMAGE_AND_WORD.xhtmlux5cux23id_136}{88}}.
Le pastoralet, vs. 501, 7240, 5768.

\protect\hypertarget{23_NOTES.xhtmlux5cux23id_135}{\protect\hyperlink{21_Chapter_Thirteen__IMAGE_AND_WORD.xhtmlux5cux23id_134}{89}}.
Compare for the mixture of pastoral and politics, Deschamps, III, p. 62,
no. 344, p. 93, no. 359.

\protect\hypertarget{23_NOTES.xhtmlux5cux23id_133}{\protect\hyperlink{21_Chapter_Thirteen__IMAGE_AND_WORD.xhtmlux5cux23id_132}{90}}.
Molinet, Faictz et dictz, f. 1.

\protect\hypertarget{23_NOTES.xhtmlux5cux23id_131}{\protect\hyperlink{21_Chapter_Thirteen__IMAGE_AND_WORD.xhtmlux5cux23id_130}{91}}.
Molinet, Chronique, IV, p. 307.

\protect\hypertarget{23_NOTES.xhtmlux5cux23id_129}{\protect\hyperlink{21_Chapter_Thirteen__IMAGE_AND_WORD.xhtmlux5cux23id_128}{92}}.
In E. Langlois, Le roman de la rose (Soc. des anc. textes), 1914, I, p.
33.

\protect\hypertarget{23_NOTES.xhtmlux5cux23id_127}{\protect\hyperlink{21_Chapter_Thirteen__IMAGE_AND_WORD.xhtmlux5cux23id_126}{93}}.
Recueil de chansons etc. (Soc. des bibliophiles belges), III, p. 31.

\protect\hypertarget{23_NOTES.xhtmlux5cux23id_125}{\protect\hyperlink{21_Chapter_Thirteen__IMAGE_AND_WORD.xhtmlux5cux23id_124}{94}}.
La Borderie, Jean Meschinot etc., pp. 603, 632.

\textbf{\emph{\protect\hypertarget{23_NOTES.xhtmlux5cux23page_439}{}{}Chapter
14}}

\protect\hypertarget{23_NOTES.xhtmlux5cux23id_2804}{\protect\hyperlink{22_Chapter_Fourteen__THE_COMING_OF.xhtmlux5cux23id_2803}{*\textsuperscript{1}}}
``the good king Scipio of Africa.''

\protect\hypertarget{23_NOTES.xhtmlux5cux23id_2802}{\protect\hyperlink{22_Chapter_Fourteen__THE_COMING_OF.xhtmlux5cux23id_2801}{†\textsuperscript{2}}}
``that is to say, guardian of the multitude.''

\protect\hypertarget{23_NOTES.xhtmlux5cux23id_2800}{\protect\hyperlink{22_Chapter_Fourteen__THE_COMING_OF.xhtmlux5cux23id_2799}{*\textsuperscript{3}}}
O Socrates full of philosophy, /Seneca in morals and Englishman in
practice, / Great Ovid in your poetry,/Brief of speech, well-versed in
rhetoric,/Exalted eagle who by your erudition/Has illuminated the reign
of Aeneas./The island of the giants, and that of Brut, and those who
have/Sown flowers and planted the eglantine,/For the ignorant of the
language, you will pour your self forth,/Great translator, noble
Geoffrey Chaucer!/ .~.~. /From you therefore out of the fountain of Hey
le/I ask to have an authentic draught,/Of which the conduit is entirely
in your power/To slake my ethical thirst,/I who in Gaul shall be
paralyzed/Till you shall give me to drink.

\protect\hypertarget{23_NOTES.xhtmlux5cux23id_2798}{\protect\hyperlink{22_Chapter_Fourteen__THE_COMING_OF.xhtmlux5cux23id_2797}{†\textsuperscript{4}}}
``your very humble and obedient slave and servant, the city of Ghent,''
``the intestinal inward sorrow and tribulation.''

\protect\hypertarget{23_NOTES.xhtmlux5cux23id_2796}{\protect\hyperlink{22_Chapter_Fourteen__THE_COMING_OF.xhtmlux5cux23id_2795}{‡\textsuperscript{5}}}
``our French-born locution and vernacular tongue.''

\protect\hypertarget{23_NOTES.xhtmlux5cux23id_2794}{\protect\hyperlink{22_Chapter_Fourteen__THE_COMING_OF.xhtmlux5cux23id_2793}{§\textsuperscript{6}}}
``having drunk from the sweet and mellifluous liquor proceeding from the
Hippocrene fountain,'' ``this virtuous scipionic duke,'' ``people of
womanly courage.''

\protect\hypertarget{23_NOTES.xhtmlux5cux23id_2792}{\protect\hyperlink{22_Chapter_Fourteen__THE_COMING_OF.xhtmlux5cux23id_2791}{*\textsuperscript{7}}}
``I have for some time rested in our house during a part of this foggy
coldness.''

\protect\hypertarget{23_NOTES.xhtmlux5cux23id_2790}{\protect\hyperlink{22_Chapter_Fourteen__THE_COMING_OF.xhtmlux5cux23id_2789}{†\textsuperscript{8}}}
Struck in the eye by a terrible brightness/Touched in the heart by
incredible eloquence, /Difficult for the human mind to produce,/Quite
obscured by incendiary light/Penetrating with almost unbearable rays/To
a dark body that can never shine,/Ravished, distraught, I find myself in
my contemplation, /My body in ecstasy lying on the ground,/My feeble
spirit is at a loss to go in quest of a path/ In order to find a place
and opportune exit/From the narrow pass where I am hemmed in,/Caught in
the toils which true love has netted.

\protect\hypertarget{23_NOTES.xhtmlux5cux23id_2378}{\protect\hyperlink{22_Chapter_Fourteen__THE_COMING_OF.xhtmlux5cux23id_2377}{*\textsuperscript{9}}}
``where is the eye that could see such a visible object, where is the
ear to hear the high silver tone and golden tintinnabulations?''

\protect\hypertarget{23_NOTES.xhtmlux5cux23id_2379}{\protect\hyperlink{22_Chapter_Fourteen__THE_COMING_OF.xhtmlux5cux23id_2380}{†\textsuperscript{10}}}
``friend of the immortal gods, beloved of men, high Ulyssean breast,
full of mellifluous eloquence .~.~. is this not splendor equal to the
car of Phoebus?''

\protect\hypertarget{23_NOTES.xhtmlux5cux23id_2382}{\protect\hyperlink{22_Chapter_Fourteen__THE_COMING_OF.xhtmlux5cux23id_2381}{‡\textsuperscript{11}}}
``the reed of Amphion, the Mercurial flute, which caused Argus to fall
sleep?''

\protect\hypertarget{23_NOTES.xhtmlux5cux23id_2384}{\protect\hyperlink{22_Chapter_Fourteen__THE_COMING_OF.xhtmlux5cux23id_2383}{§\textsuperscript{12}}}
``in Italy, on which the kind influences of heaven bring beautiful
speech, and towards which all elemental sweetness is drawn, there to
dissolve into harmony.''

\protect\hypertarget{23_NOTES.xhtmlux5cux23id_2788}{\protect\hyperlink{22_Chapter_Fourteen__THE_COMING_OF.xhtmlux5cux23id_2787}{*\textsuperscript{13}}}
``is an example of Ciceronian art and a kind of Terence-like subtlety
.~.~. he, who was favored to absorb from our breasts our innermost
substance; who, going beyond the grace granted by his own soil, to new
refreshment in the land of good taste (Italy) has gone, there, where
children in morning songs speak to their mothers, eager for school and
widely learned beyond their years.''

\protect\hypertarget{23_NOTES.xhtmlux5cux23id_2786}{\protect\hyperlink{22_Chapter_Fourteen__THE_COMING_OF.xhtmlux5cux23id_2785}{†\textsuperscript{14}}}
``Robertet has rained on me from his cloud, he, whose pearls gather in
this cloud like hail, has made my garment shine; but what of the dark
body underneath, if my dress deceives the clear sighted?''

\protect\hypertarget{23_NOTES.xhtmlux5cux23id_2784}{\protect\hyperlink{22_Chapter_Fourteen__THE_COMING_OF.xhtmlux5cux23id_2783}{*\textsuperscript{15}}}
The old life displeased, the new morals are fallen;/Mankind sees the
face, yet the heart is open to honest Jupiter.

\protect\hypertarget{23_NOTES.xhtmlux5cux23id_2782}{\protect\hyperlink{22_Chapter_Fourteen__THE_COMING_OF.xhtmlux5cux23id_2781}{†\textsuperscript{16}}}
``Jupiter come from Paradise,''

\protect\hypertarget{23_NOTES.xhtmlux5cux23id_2780}{\protect\hyperlink{22_Chapter_Fourteen__THE_COMING_OF.xhtmlux5cux23id_2779}{‡\textsuperscript{17}}}
``High Goddess.''

\protect\hypertarget{23_NOTES.xhtmlux5cux23id_2778}{\protect\hyperlink{22_Chapter_Fourteen__THE_COMING_OF.xhtmlux5cux23id_2777}{§\textsuperscript{18}}}
``the temple in the high woods where people pray to the gods.''

\protect\hypertarget{23_NOTES.xhtmlux5cux23id_2776}{\protect\hyperlink{22_Chapter_Fourteen__THE_COMING_OF.xhtmlux5cux23id_2775}{*\textsuperscript{19}}}
``If, to lend my Muse some strangeness, I speak of the pagan gods, the
shepherds and myself are Christians all the same.''

\protect\hypertarget{23_NOTES.xhtmlux5cux23id_2774}{\protect\hyperlink{22_Chapter_Fourteen__THE_COMING_OF.xhtmlux5cux23id_2773}{†\textsuperscript{20}}}
``Reason and Understanding'' .~.~. ``You should do it, not to instill
faith in gods and goddesses, but because our Lord alone inspires people
as it pleases Him and frequently by differing means.''

\protect\hypertarget{23_NOTES.xhtmlux5cux23id_2772}{\protect\hyperlink{22_Chapter_Fourteen__THE_COMING_OF.xhtmlux5cux23id_2771}{‡\textsuperscript{21}}}
Formerly the gentile nations of the gods,/Sought love by humble
sacrifices,/ Which, taken for granted that they were useless,/Were
nevertheless profitable and prolific,/Of much important fruit and high
benefits,/Which shows by facts that offices of love/And of humble
homage, rendered wherever they were,/Were sufficient to pierce heaven
and hell.

\protect\hypertarget{23_NOTES.xhtmlux5cux23id_2770}{\protect\hyperlink{22_Chapter_Fourteen__THE_COMING_OF.xhtmlux5cux23id_2769}{§\textsuperscript{22}}}
``Which I by no means approve.''

\protect\hypertarget{23_NOTES.xhtmlux5cux23id_2768}{\protect\hyperlink{22_Chapter_Fourteen__THE_COMING_OF.xhtmlux5cux23id_2767}{*\textsuperscript{23}}}
``I wish I could have satisfied all my desires and never had any other
good.''

\protect\hypertarget{23_NOTES.xhtmlux5cux23id_2766}{\protect\hyperlink{22_Chapter_Fourteen__THE_COMING_OF.xhtmlux5cux23id_2765}{†\textsuperscript{24}}}
So help me God who was crucified:/I much repent that I made man.

\protect\hypertarget{23_NOTES.xhtmlux5cux23id_123}{\protect\hyperlink{22_Chapter_Fourteen__THE_COMING_OF.xhtmlux5cux23id_122}{1}}.
Alma Le Duc, Gontier Col and the French Prerenaissance, 1919, was not
available to me.

\protect\hypertarget{23_NOTES.xhtmlux5cux23id_121}{\protect\hyperlink{22_Chapter_Fourteen__THE_COMING_OF.xhtmlux5cux23id_120}{2}}.
N. de démanges, Opera, ed. Lydius, Lugd. Bat., 1613; Joh. de
Monasteriolo, Epistolae, Martene et Durand, Amplissima Collectio, II,
col. 1310.

\protect\hypertarget{23_NOTES.xhtmlux5cux23id_119}{\protect\hyperlink{22_Chapter_Fourteen__THE_COMING_OF.xhtmlux5cux23id_118}{3}}.
Montreuil, Epistolae 69, c. 1447, ep. 15, c. 1338.

\protect\hypertarget{23_NOTES.xhtmlux5cux23id_117}{\protect\hyperlink{22_Chapter_Fourteen__THE_COMING_OF.xhtmlux5cux23id_116}{4}}.
Epistolae 59, c. 1426, ep. 58, c. 1423.

\protect\hypertarget{23_NOTES.xhtmlux5cux23id_115}{\protect\hyperlink{22_Chapter_Fourteen__THE_COMING_OF.xhtmlux5cux23id_114}{5}}.
Epistolae 40, cols. 1388, 1396.

\protect\hypertarget{23_NOTES.xhtmlux5cux23id_113}{\protect\hyperlink{22_Chapter_Fourteen__THE_COMING_OF.xhtmlux5cux23id_112}{6}}.
Epistolae 59, 67, cols. 1427, 1435.

\protect\hypertarget{23_NOTES.xhtmlux5cux23id_111}{\protect\hyperlink{22_Chapter_Fourteen__THE_COMING_OF.xhtmlux5cux23id_110}{7}}.
Le livre du voir-dit, p. xviii.

\protect\hypertarget{23_NOTES.xhtmlux5cux23id_109}{\protect\hyperlink{22_Chapter_Fourteen__THE_COMING_OF.xhtmlux5cux23id_108}{8}}.
See p. 76.

\protect\hypertarget{23_NOTES.xhtmlux5cux23id_107}{\protect\hyperlink{22_Chapter_Fourteen__THE_COMING_OF.xhtmlux5cux23id_106}{9}}.
See p. 226.

\protect\hypertarget{23_NOTES.xhtmlux5cux23id_105}{\protect\hyperlink{22_Chapter_Fourteen__THE_COMING_OF.xhtmlux5cux23id_104}{10}}.
Gerson, Opera, I, p. 922.

\protect\hypertarget{23_NOTES.xhtmlux5cux23id_103}{\protect\hyperlink{22_Chapter_Fourteen__THE_COMING_OF.xhtmlux5cux23id_102}{11}}.
Epistolae 38, col. 1385.

\protect\hypertarget{23_NOTES.xhtmlux5cux23id_101}{\protect\hyperlink{22_Chapter_Fourteen__THE_COMING_OF.xhtmlux5cux23id_100}{12}}.
Dion. Cart., t. XXXVII, p. 495.

\protect\hypertarget{23_NOTES.xhtmlux5cux23id_99}{\protect\hyperlink{22_Chapter_Fourteen__THE_COMING_OF.xhtmlux5cux23id_98}{13}}.
Petrarca, Opera, ed. Basel, 1581, p. 847; Clémanges, Opera, Ep. 5, p.
24; J. de Montr., Ep. 50, col. 1428.

\protect\hypertarget{23_NOTES.xhtmlux5cux23id_97}{\protect\hyperlink{22_Chapter_Fourteen__THE_COMING_OF.xhtmlux5cux23id_96}{14}}.
Chastellain, VII, pp. 75--143, see V, pp. 38--40, VI, p. 80; VIII, p.
358, Le livre des trahisons, p. 145.

\protect\hypertarget{23_NOTES.xhtmlux5cux23id_95}{\protect\hyperlink{22_Chapter_Fourteen__THE_COMING_OF.xhtmlux5cux23id_94}{15}}.
Machaut, Le voir-dit, p. 230; Chastellain, VI, p. 194; La Marche, III,
p. 166; Le pastoralet vs. 2806; Le Jouvencel, I, p. 16.

\protect\hypertarget{23_NOTES.xhtmlux5cux23id_93}{\protect\hyperlink{22_Chapter_Fourteen__THE_COMING_OF.xhtmlux5cux23id_92}{16}}.
Le pastoralet, vs. 541, 4612.

\protect\hypertarget{23_NOTES.xhtmlux5cux23id_91}{\protect\hyperlink{22_Chapter_Fourteen__THE_COMING_OF.xhtmlux5cux23id_90}{17}}.
Chastellain, III, pp. 173, 117, 359 etc.; Molinet, II, p. 207.

\protect\hypertarget{23_NOTES.xhtmlux5cux23id_89}{\protect\hyperlink{22_Chapter_Fourteen__THE_COMING_OF.xhtmlux5cux23id_88}{18}}.
J. Germain, Liber de virtutibus Philippe ducis Burgundiae (Chron. rel. à
l'hist. de Belg. sous la dom. des ducs de Bourg. III).

\protect\hypertarget{23_NOTES.xhtmlux5cux23id_87}{\protect\hyperlink{22_Chapter_Fourteen__THE_COMING_OF.xhtmlux5cux23id_86}{19}}.
Chron. scand., II, p. 42.

\protect\hypertarget{23_NOTES.xhtmlux5cux23id_85}{\protect\hyperlink{22_Chapter_Fourteen__THE_COMING_OF.xhtmlux5cux23id_84}{20}}.
Christine de Pisan, Oeuvres poétiques, I, no. 90, p. 90.

\protect\hypertarget{23_NOTES.xhtmlux5cux23id_83}{\protect\hyperlink{22_Chapter_Fourteen__THE_COMING_OF.xhtmlux5cux23id_82}{21}}.
Deschamps, no. 285, II, p. 138.

\protect\hypertarget{23_NOTES.xhtmlux5cux23id_81}{\protect\hyperlink{22_Chapter_Fourteen__THE_COMING_OF.xhtmlux5cux23id_80}{22}}.
Villon, ed. Lognon, p. 15, h. 36--38; Rabelais, Pantagruel, 1.2, chap.
6.

\protect\hypertarget{23_NOTES.xhtmlux5cux23id_79}{\protect\hyperlink{22_Chapter_Fourteen__THE_COMING_OF.xhtmlux5cux23id_78}{23}}.
Chastellain, V, pp. 292ff.; La Marche, Parament et triumphe des dames,
Prologue; Molinet, Faictz et dictz, Prologue, Molinet, Chronique, I, pp.
72, 10, 54.

\protect\hypertarget{23_NOTES.xhtmlux5cux23id_77}{\protect\hyperlink{22_Chapter_Fourteen__THE_COMING_OF.xhtmlux5cux23id_76}{24}}.
Summaries by Kervyn de Lettenhove, Oeuvres de Chastellain, VII, 1. pp.
45--186; see P. Durrieu, Un barbier de nom français à Bruges, Académie
des inscriptions et belles-lettres, Comptes rendus, 1917, pp. 542--58.

\protect\hypertarget{23_NOTES.xhtmlux5cux23id_75}{\protect\hyperlink{22_Chapter_Fourteen__THE_COMING_OF.xhtmlux5cux23id_74}{25}}.
Chastellain, VII, p. 146.

\protect\hypertarget{23_NOTES.xhtmlux5cux23id_73}{\protect\hyperlink{22_Chapter_Fourteen__THE_COMING_OF.xhtmlux5cux23id_72}{26}}.
Chastellain, VII, p. 180.

\protect\hypertarget{23_NOTES.xhtmlux5cux23id_71}{\protect\hyperlink{22_Chapter_Fourteen__THE_COMING_OF.xhtmlux5cux23id_70}{27}}.
La Marche, I, pp. 15, 184--86; Molinet, I p. 14, III p. 99; Chastellain,
VI: Exposition sur vérité mal prise, VII pp. 76, 29, 142, 422; Commines,
I p. 3; see Doutrepont, p. 24.

\protect\hypertarget{23_NOTES.xhtmlux5cux23id_69}{\protect\hyperlink{22_Chapter_Fourteen__THE_COMING_OF.xhtmlux5cux23id_68}{28}}.
Chastellain, VII, p. 159.

\protect\hypertarget{23_NOTES.xhtmlux5cux23id_67}{\protect\hyperlink{22_Chapter_Fourteen__THE_COMING_OF.xhtmlux5cux23id_66}{29}}.
Ibid.

\protect\hypertarget{23_NOTES.xhtmlux5cux23id_65}{\protect\hyperlink{22_Chapter_Fourteen__THE_COMING_OF.xhtmlux5cux23id_64}{30}}.
R. Gaguini, Ep. et Or., ed. Thuasne, I, p. 126; Allen, Erasmi Epistolae
no. 43 I, p. 145.

\protect\hypertarget{23_NOTES.xhtmlux5cux23id_63}{\protect\hyperlink{22_Chapter_Fourteen__THE_COMING_OF.xhtmlux5cux23id_62}{31}}.
R. Gaguini, ed. Thuasne, I, p. 20.

\protect\hypertarget{23_NOTES.xhtmlux5cux23id_61}{\protect\hyperlink{22_Chapter_Fourteen__THE_COMING_OF.xhtmlux5cux23id_60}{32}}.
R. Gaguini, ed. Thuasne, I, p. 178, II, p. 509.

\protect\hypertarget{23_NOTES.xhtmlux5cux23page_440}{\protect\hyperlink{22_Chapter_Fourteen__THE_COMING_OF.xhtmlux5cux23id_59}{33}}.
See F. von Bezold, Das Fortleben der antiker Götter im mittelalterlichen
Humanismus, Bonn und Leipzig, 1922.

\protect\hypertarget{23_NOTES.xhtmlux5cux23id_58}{\protect\hyperlink{22_Chapter_Fourteen__THE_COMING_OF.xhtmlux5cux23id_57}{34}}.
Deschamps, no. 63, I, p. 158.

\protect\hypertarget{23_NOTES.xhtmlux5cux23id_56}{\protect\hyperlink{22_Chapter_Fourteen__THE_COMING_OF.xhtmlux5cux23id_55}{35}}.
Villon, Testament, vs. 899, ed. Longnon, p. 58.

\protect\hypertarget{23_NOTES.xhtmlux5cux23id_54}{\protect\hyperlink{22_Chapter_Fourteen__THE_COMING_OF.xhtmlux5cux23id_53}{36}}.
Le pastoralet, vs. 2094.

\protect\hypertarget{23_NOTES.xhtmlux5cux23id_52}{\protect\hyperlink{22_Chapter_Fourteen__THE_COMING_OF.xhtmlux5cux23id_51}{37}}.
Ibid., vs. 30, p. 574.

\protect\hypertarget{23_NOTES.xhtmlux5cux23id_50}{\protect\hyperlink{22_Chapter_Fourteen__THE_COMING_OF.xhtmlux5cux23id_49}{38}}.
Molinet, V, p. 21.

\protect\hypertarget{23_NOTES.xhtmlux5cux23id_48}{\protect\hyperlink{22_Chapter_Fourteen__THE_COMING_OF.xhtmlux5cux23id_47}{39}}.
Chastellain, Le dit de Vérité, VI, p. 221, see Exposition sur vérité mal
prise, ibid., pp. 297, 310.

\protect\hypertarget{23_NOTES.xhtmlux5cux23id_46}{\protect\hyperlink{22_Chapter_Fourteen__THE_COMING_OF.xhtmlux5cux23id_45}{40}}.
La Marche, II, p. \emph{68}.

\protect\hypertarget{23_NOTES.xhtmlux5cux23id_44}{\protect\hyperlink{22_Chapter_Fourteen__THE_COMING_OF.xhtmlux5cux23id_43}{41}}.
Roman de la rose, vs. 20141.
