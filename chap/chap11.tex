\chapter{THE FORMS OF THOUGHT IN PRACTICE}

TO UNDERSTAND MEDIEVAL THOUGHT IN ITS TOTAL unity, it is necessary to
study the fixed forms of thought not only as they occur in the
conceptions of faith and in the realms of higher speculation, but also
as they are found in everyday wisdom and mundane practices, since
medieval thought was dominated by the same large patterns in both its
higher and more common expressions. While in matters of faith and
contemplation the question is always open as to how far the forms of
thought are the result or the echo of a long written tradition going far
back to Greek and Jewish, or even Egyptian and Babylonian, roots, common
forms of life as we encounter them in their naive and spontaneous
expressions are unencumbered by the weight of neo-Platonism and the
like.

In his daily life medieval man thought in the same forms as in his
theology. The foundation is furnished in both instances by that
architectural idealism which the scholastics call realism: the need to
separate each insight and conceive of it as an individual entity and
then to link the entities into hierarchical units and to continuously
erect temples and cathedrals with them, like a child playing with
building blocks.

Everything that had won for itself a secure place in life, that was
melded into the forms of life, was taken to be ordained by God's plan
for the world. This applied to the most ordinary customs and usages as
well as the highest of things. This is clearly evident, for example, in
the perception of the rules of court etiquette revealed in the
descriptions of the courts by Olivier de la Marche and Alienor de
Poitiers. Old Alienor regards those rules as wise laws that were at one
time implemented in the courts of kings by judicious choice and that are
to be observed for all times to come. She speaks of them as if they were
the wisdom of the centuries: ``et alors j'ouy
\protect\hypertarget{18_Chapter_Eleven__THE_FORMS_OF_THO.xhtmlux5cux23page_269}{}{}dire
aux anciens qui sçavoient .~.~.
''\protect\hypertarget{18_Chapter_Eleven__THE_FORMS_OF_THO.xhtmlux5cux23id_2333}{\protect\hyperlink{23_NOTES.xhtmlux5cux23id_2334}{*\textsuperscript{15}}}
She sees the times degenerating. For about ten years, a few ladies in
Flanders have placed the maternity bed before the fire, ``de quoy l'on
s'est bien
mocqué'';\protect\hypertarget{18_Chapter_Eleven__THE_FORMS_OF_THO.xhtmlux5cux23id_2335}{\protect\hyperlink{23_NOTES.xhtmlux5cux23id_2336}{†\textsuperscript{16}}}
this has never been done before; where will it lead? ``Mais un chacun
fait à cette heure à guise; par quoy est à doubter que tout ira
mal.''\textsuperscript{\protect\hypertarget{18_Chapter_Eleven__THE_FORMS_OF_THO.xhtmlux5cux23id_629}{\protect\hyperlink{23_NOTES.xhtmlux5cux23id_630}{1}}}\protect\hypertarget{18_Chapter_Eleven__THE_FORMS_OF_THO.xhtmlux5cux23id_2337}{\protect\hyperlink{23_NOTES.xhtmlux5cux23id_2338}{‡\textsuperscript{17}}}

La Marche asks himself and his readers important questions with respect
to the rational cause of all these ceremonial things: why does the
``fruitier'' have ``le mestier de la cire'' in his department? The
answer is that the wax is sucked by the bees from the same flowers from
which come the fruit: ``pourquoy on a ordonné trés bien ceste
chose.''\textsuperscript{\protect\hypertarget{18_Chapter_Eleven__THE_FORMS_OF_THO.xhtmlux5cux23id_627}{\protect\hyperlink{23_NOTES.xhtmlux5cux23id_628}{2}}}\protect\hypertarget{18_Chapter_Eleven__THE_FORMS_OF_THO.xhtmlux5cux23id_2339}{\protect\hyperlink{23_NOTES.xhtmlux5cux23id_2340}{§\textsuperscript{18}}}
This strong medieval tendency to create an organ for each function is
nothing but the result of that way of thinking which ascribed
independence to any quality, and which saw each one as a separate idea.
The King of England had an official under his \emph{magna sergenteria
\protect\hypertarget{18_Chapter_Eleven__THE_FORMS_OF_THO.xhtmlux5cux23id_2341}{\protect\hyperlink{23_NOTES.xhtmlux5cux23id_2342}{**\textsuperscript{19}}}}
whose office it was to hold the head of the king whenever he crossed the
channel and became seasick. In 1442 this position was held by a certain
John Baker, who passed it on to his two
daughters.\textsuperscript{\protect\hypertarget{18_Chapter_Eleven__THE_FORMS_OF_THO.xhtmlux5cux23id_625}{\protect\hyperlink{23_NOTES.xhtmlux5cux23id_626}{3}}}

The habit of giving names to all things, even those that are inanimate,
should be regarded in the same light. This is a faint whiff of primitive
anthropomorphism that occurs even today in military life, itself in many
respects a return to a primitive form of life, when cannons are given
names. That urge was much stronger during medieval times. Just as the
swords in the knightly novels, the bombards of the wars of the
fourteenth and fifteenth centuries had their names: ``le Chien
d'Orléans,'' ``la Gringade,'' ``la Bourgeoisie,'' ``de Dulle Griete.'' A
remnant of this practice remains in that individual diamonds still have
names. Several of the jewels of Charles the Bold were named: ``le
sancy,'' ``les trois frères,'' ``la hote,'' ``la balle de Flandres.''
That ships in our time have retained their names, but houses and church
bells have not, may be due to the fact that a ship changes its location
and has to be recognizable
\protect\hypertarget{18_Chapter_Eleven__THE_FORMS_OF_THO.xhtmlux5cux23page_270}{}{}at
any time, but also in part because a ship retains more of a personal
quality than a house; a feeling that is expressed by speakers of English
who use the pronoun ``she'' when referring to a ship. The personal
perception of inanimate objects was much more prominent during medieval
times; then everything was given a name: the cells in a prison just as
well as every house and every
clock.\textsuperscript{\protect\hypertarget{18_Chapter_Eleven__THE_FORMS_OF_THO.xhtmlux5cux23id_623}{\protect\hyperlink{23_NOTES.xhtmlux5cux23id_624}{4}}}

Medieval men sought, as they put it, the ``morality,'' the hidden lesson
in everything, the essential ethical significance. Every historical or
literary incident has the potential to crystalize into a parable, into a
moral example, into evidential proof; every statement could become a
dictum, a text, or a saying. Just like the holy symbolic links between
the Old and the New Testaments, moral links come into being that make it
possible to immediately hold up to any event the mirror of a model, an
exemplary type from the Bible, history, or literature. To prompt someone
to forgiveness, confront him with biblical cases of forgiveness. To warn
of the dangers of marriage, string together all instances of the
unfortunate marriages of antiquity. John the Fearless, in order to
excuse the murder of Orléans, compares himself to Joab and his victim to
Absalom, and claims to have been better than Joab because the king had
not expressedly prohibited murder. ``Ainssy a voit le bon duc Jehan
attrait ce fait à
moralité.''\textsuperscript{\protect\hypertarget{18_Chapter_Eleven__THE_FORMS_OF_THO.xhtmlux5cux23id_621}{\protect\hyperlink{23_NOTES.xhtmlux5cux23id_622}{5}}}\protect\hypertarget{18_Chapter_Eleven__THE_FORMS_OF_THO.xhtmlux5cux23id_2608}{\protect\hyperlink{23_NOTES.xhtmlux5cux23id_2607}{*\textsuperscript{20}}}
This is a broad and naive application of the principles of jurisprudence
that only now is beginning to be seen in modern legal practice as a
residue of an obsolete way of thinking.

Every serious attempt to arrive at proof will be grounded in a text as
the point of support and departure. The twelve propositions for and
against revoking obedience to the Avignon pope with which the matter of
the schism was debated during the national Council of Paris in 1406 were
all based on a scriptural
passage.\textsuperscript{\protect\hypertarget{18_Chapter_Eleven__THE_FORMS_OF_THO.xhtmlux5cux23id_619}{\protect\hyperlink{23_NOTES.xhtmlux5cux23id_620}{6}}}
A worldly orator will begin, just like a preacher, with his
text.\textsuperscript{\protect\hypertarget{18_Chapter_Eleven__THE_FORMS_OF_THO.xhtmlux5cux23id_617}{\protect\hyperlink{23_NOTES.xhtmlux5cux23id_618}{7}}}

There is no more striking example of all the features mentioned above
than the notorious plea by Master Jean Petit with which he attempted to
justify the assassination of Louis d'Orléans at the instigation of the
duke of Burgundy.

More than three months had passed since the brother of the king had been
cut down one evening by the hired assassins for whom
\protect\hypertarget{18_Chapter_Eleven__THE_FORMS_OF_THO.xhtmlux5cux23page_271}{}{}John
the Fearless had secured lodging in a house in the Rue Vieille du Temple
just prior to the event. The Burgundian initially expressed great sorrow
during the funeral service, but as soon as he saw that the investigation
would lead to the hôtel d'Artois where he had the murderers hidden, he
conferred with his uncle Berry and confessed to him that he had ordered
the murder to be carried out because he had succumbed to the
instigations of the Devil. He thereupon fled from Paris to Flanders. At
Ghent he proclaimed his first justification for his crime and returned
to Paris relying on the hatred directed by everyone towards Orléans and
on his own popularity with the Parisians, who, in fact, did gladly
welcome him back. In Amiens, the duke had consulted with two men who had
distinguished themselves as orators during the 1406 church assembly in
Paris: Master Jean Petit and Pierre aux Boeufs. They were given the task
of sprucing up the plea given at Ghent, which had been written by Simon
de Saulx, so that it could be presented at Paris before the princes and
nobles and ensure that they be dutifully impressed and the duke's
actions justified.

Therewith, Master Jean Petit, biblical scholar, preacher, and poet,
appears on March 8, 1408, in the Hôtel de Saint Pol in Paris before a
most exhalted audience, among whom the Dauphin, the King of Naples, the
dukes of Berry and Brittany are the front rank. He begins with
appropriate humility, I, wretch that I am, am neither theologian nor
jurist, ``une très grande paour me fiert au cuer, voire si grand, que
mon engin et ma mémoire s'en fuit, et ce peu de sens que je cuidoie
avoir, m'a jà du tout
laissé.''\protect\hypertarget{18_Chapter_Eleven__THE_FORMS_OF_THO.xhtmlux5cux23id_2609}{\protect\hyperlink{23_NOTES.xhtmlux5cux23id_2610}{*\textsuperscript{21}}}
He then proceeds to elaborate, in a highly restrained style, on a
masterpiece of dark political malice that his mind had erected on the
text ``Radix omnium malorum cupiditas.'' The entire plea is artfully
illustrated with scholastic distinctions and secondary texts; it is
illustrated by examples from scripture and history. From the colorful
details with which the defense describes the perfidy of the slain
Orléans, the plea acquires a devilish liveliness and romantic tension.
It opens with a list of the twelve obligations binding the duke of
Burgundy to favor, love, and avenge the King of France. Commending
himself to the aid of God, the Virgin, and St. John
\protect\hypertarget{18_Chapter_Eleven__THE_FORMS_OF_THO.xhtmlux5cux23page_272}{}{}the
Evangelist, Petit begins to detail his evidence for the defense, which
is divided into major and minor proofs and a conclusion. At the head of
them all he states his text: ``Radix omnium malorum cupiditas.'' Two
practical applications are derived from it: greed generates apostates,
it creates traitors. The evils of apostasy and treason are divided and
subdivided and then demonstrated by the use of three examples: Lucifer,
Absalom, and
Athalia\textsuperscript{\protect\hypertarget{18_Chapter_Eleven__THE_FORMS_OF_THO.xhtmlux5cux23id_616}{\protect\hyperlink{23_NOTES.xhtmlux5cux23page_429}{8}}}
are conjured up in the minds of the audience as the types of traitors.
This is followed by eight truths that justify the murder of tyrants: he
who conspires against the king deserves death and damnation, the more so
the higher his position; anyone is free to kill him. ``Je prouve ceste
verité par douze raison en `honneur des douze
apostres'':\protect\hypertarget{18_Chapter_Eleven__THE_FORMS_OF_THO.xhtmlux5cux23id_2611}{\protect\hyperlink{23_NOTES.xhtmlux5cux23id_2612}{*\textsuperscript{22}}}
three pronouncements from doctors of the church, three from
philosophers, three from jurists, and three from Holy Scripture. The
plea continues in this manner until the eight truths are covered. A
quotation from ``De casibus virorum illustrium'' by ``le philosophe
moral Boccace'' is given in proof that the tyrant may be killed from
ambush. The eight truths produce eight ``correlaria'' to which a ninth
is added that hints at all the dark events in which slander and
suspicion had assigned Orléans a gruesome part. All the old suspicions
that had followed the ambitious and reckless prince since the days of
his youth are again fanned to a state of red heat, how he, in 1392, had
been the deliberate instigator of the fateful ``bal des ardents'' during
which his brother, the young king, had barely escaped the fiery death of
his companions, who, disguised as ruffians, had come in contact with a
carelessly held
torch.\textsuperscript{\protect\hypertarget{18_Chapter_Eleven__THE_FORMS_OF_THO.xhtmlux5cux23id_614}{\protect\hyperlink{23_NOTES.xhtmlux5cux23id_615}{9}}}
Orleans's conversations with the ``magician'' Philippe de Mézières in
the monastery of the Celestines furnished material for all kind of
insinuations about plans for murder and poisonings. His generally known
predilection for magic gives rise to the most lively horror stories: for
example, Orléans was said to have gone, one Sunday morning, to La Tour
Montjay on the Marne in the company of an apostate monk, a page, and a
servant; the monk made two brown and green clad devils named Heremas and
Estramain appear. In a hellish ceremony they consecrated a sword, a
dagger, and a ring, whereupon the travelers took down a hanged man from
the gallows at Montfaucon, and so forth. Master Jean even manages to
extract
\protect\hypertarget{18_Chapter_Eleven__THE_FORMS_OF_THO.xhtmlux5cux23page_273}{}{}some
dark meanings from the meaningless ramblings of the mad
king.\textsuperscript{\protect\hypertarget{18_Chapter_Eleven__THE_FORMS_OF_THO.xhtmlux5cux23id_612}{\protect\hyperlink{23_NOTES.xhtmlux5cux23id_613}{10}}}

After things have been raised in this way to the general-ethical level
by putting them into the light of biblical patterns and moral dictums,
thus artfully feeding the fires of disgust and horror, there bursts
forth in the minor proofs, which step by step follow the structure of
the major proofs, the flood of direct accusations. All the passionate
party hatred at the disposal of a mind unleashed is used to attack the
memory of the murder victim.

Jean Petit held the floor for four hours. After he had finished, his
client, the duke of Burgundy, said, ``Je vous
avoue.''\protect\hypertarget{18_Chapter_Eleven__THE_FORMS_OF_THO.xhtmlux5cux23id_2343}{\protect\hyperlink{23_NOTES.xhtmlux5cux23id_2344}{*\textsuperscript{23}}}
The text of the justification was presented to the duke and his closest
relatives in four precious volumes, bound in pressed leather, decorated
with gold, and illustrated with miniatures. One copy is still preserved
in Vienna. A printed version of the tract could also be
purchased.\textsuperscript{\protect\hypertarget{18_Chapter_Eleven__THE_FORMS_OF_THO.xhtmlux5cux23id_610}{\protect\hyperlink{23_NOTES.xhtmlux5cux23id_611}{11}}}

The need to elevate any event of life into a moral model, to raise all
sentences to maxims, whereby they became something substantial and
unassailable, in short, that process of crystalizing thought, finds its
most general and natural expression in the proverb. The proverb had a
very lively function in medieval thinking. Hundreds were in general use,
almost all of them deft and hitting the mark. The wisdom shown in
proverbs is at times conventional, occasionally beneficial and profound;
the tone is frequently ironic, the mood mostly kind and always resigned.
The proverb never preaches opposition, always only surrender. With a
smile or a sigh, it allows the egoist to triumph, lets the hypocrite go
scot-free. ``Les grans poissons mangent les plus petis.'' ``Les mal
vestus assiet on dos ou vent.'' ``Nul n'est chaste si ne besongne.''
Many sound cynical. ``L'homme est bon tant qu'il craint sa peau.'' ``Au
besoing on s'aide du diable.'' But beneath them all resides a gentle
spirit that does not desire to be judgmental. ``Il n'est si ferré qui ne
glice.''\protect\hypertarget{18_Chapter_Eleven__THE_FORMS_OF_THO.xhtmlux5cux23id_2345}{\protect\hyperlink{23_NOTES.xhtmlux5cux23id_2346}{t\textsuperscript{24}}}
The lamentations of the moralists over human sinfulness and corruption
are confronted by the smiling understanding of folk
\protect\hypertarget{18_Chapter_Eleven__THE_FORMS_OF_THO.xhtmlux5cux23page_274}{}{}wisdom.
In the proverb, the wisdom and morality of all times and spheres are
condensed into a single image. On some occasions the tone of the
proverbs is nearly evangelical, but they also sometimes are almost
paganly naive. A people with many proverbs in living use leaves matters
of dispute, motivation, and argumentation to the theologians and
philosophers; the proverbs settle every argument by reference to a
judgment that hits the nail right on the head. They scorn weighty
arguments and avoid much confusion. The proverb cuts through knotty
problems; once a proverb is applied, the matter is settled. This ability
to crystalize ideas has significant advantages for culture.

It is surprising how many proverbs were familiar in late medieval
times.\textsuperscript{\protect\hypertarget{18_Chapter_Eleven__THE_FORMS_OF_THO.xhtmlux5cux23id_608}{\protect\hyperlink{23_NOTES.xhtmlux5cux23id_609}{12}}}
In their common utility they conform so much to the intellectual content
of literature that the authors of those days make generous use of them.
For example, poems in which every stanza ends with a proverb are very
popular. After a scandalous incident, a slanderous poem, written in this
form by an unknown author, is directed against the \emph{prévôt} of
Paris, Hugues
Aubriot.\textsuperscript{\protect\hypertarget{18_Chapter_Eleven__THE_FORMS_OF_THO.xhtmlux5cux23id_606}{\protect\hyperlink{23_NOTES.xhtmlux5cux23id_607}{13}}}
Other examples are Alain Chartier's ``Ballade de
Fougères,''\textsuperscript{\protect\hypertarget{18_Chapter_Eleven__THE_FORMS_OF_THO.xhtmlux5cux23id_604}{\protect\hyperlink{23_NOTES.xhtmlux5cux23id_605}{14}}}
Molinet in a number of different pieces from his ``Faictz et
Dictz,''\textsuperscript{\protect\hypertarget{18_Chapter_Eleven__THE_FORMS_OF_THO.xhtmlux5cux23id_602}{\protect\hyperlink{23_NOTES.xhtmlux5cux23id_603}{15}}}
Coquillart's ``Complaincte de
Eco,''\textsuperscript{\protect\hypertarget{18_Chapter_Eleven__THE_FORMS_OF_THO.xhtmlux5cux23id_600}{\protect\hyperlink{23_NOTES.xhtmlux5cux23id_601}{16}}}
and Villon's ballade that is entirely made up of
proverbs.\textsuperscript{\protect\hypertarget{18_Chapter_Eleven__THE_FORMS_OF_THO.xhtmlux5cux23id_598}{\protect\hyperlink{23_NOTES.xhtmlux5cux23id_599}{17}}}
Robert Gaguins's ``Le passe temps
d'oysiveté''\textsuperscript{\protect\hypertarget{18_Chapter_Eleven__THE_FORMS_OF_THO.xhtmlux5cux23id_596}{\protect\hyperlink{23_NOTES.xhtmlux5cux23id_597}{18}}}
belongs to this category. With few exceptions, all its 171 stanzas end
with a suitable proverb. Or, do these proverb-like moral dictums (of
which only a few can be found in collections of proverbs familiar to me)
spring from Gaguin's own poetic mind? If this should be the case, it
only would provide stronger proof of the vital function allotted to
proverbs in late medieval thought. In this instance we would be able to
prove that, linked to a poem, they arose consciously from the mind of an
individual poet to provide well-rounded, fixed, generally understandable
judgments.

Even sermons do not shun putting proverbs side by side with sacred
texts, and they both mingle together in debates during church or state
assemblies. Gerson, Jean de Varennes, Jean Petit, Guillaume Fillastre,
Olivier Maillard employ the most common proverbs in their sermons in
support of their arguments: ``Qui de tout se tait, de tout a paix,''
``Chef bien peigné porte mal bacinet.'' ``D'aultrui cuir large
courroye.'' ``Selon seigneur mesnie duite.'' ``De tel juge tel
jugement.'' ``Qui commun sert, nul ne l'en paye.''
\protect\hypertarget{18_Chapter_Eleven__THE_FORMS_OF_THO.xhtmlux5cux23page_275}{}{}``Qui
est tigneux, il ne doit pas oster son
chaperon.''\textsuperscript{\protect\hypertarget{18_Chapter_Eleven__THE_FORMS_OF_THO.xhtmlux5cux23id_594}{\protect\hyperlink{23_NOTES.xhtmlux5cux23id_595}{19}}}\protect\hypertarget{18_Chapter_Eleven__THE_FORMS_OF_THO.xhtmlux5cux23id_2613}{\protect\hyperlink{23_NOTES.xhtmlux5cux23id_2614}{*\textsuperscript{25}}}
There is even a link between the proverb and the \emph{Imitatio
Christi}, which, in form, is based on collections, or ``rapiaria,'' of
wisdom of various kinds and origins.

There are, in the waning Middle Ages, many authors whose powers of
judgment do not really rise above the level of the proverbs they employ
so consistently. A chronicler from the late fourteenth century, Geoffroi
de Paris, laces his rhymed chronicle with proverbs expressing the moral
lessons of the events he
records.\textsuperscript{\protect\hypertarget{18_Chapter_Eleven__THE_FORMS_OF_THO.xhtmlux5cux23id_592}{\protect\hyperlink{23_NOTES.xhtmlux5cux23id_593}{20}}}
In the use of this technique he is wiser than Froissart and Le
Jouvencel, whose homemade dictums frequently sound like half-baked
proverbs: ``Enssi aviennent li fait d'armes: on piert une fois et
l'autre fois gaagn'on.'' ``Or n'est-il riens dont on ne se tanne.'' ``On
dit, et vray est, que il n'est chose plus certaine que la
mort.''\textsuperscript{\protect\hypertarget{18_Chapter_Eleven__THE_FORMS_OF_THO.xhtmlux5cux23id_590}{\protect\hyperlink{23_NOTES.xhtmlux5cux23id_591}{21}}}\protect\hypertarget{18_Chapter_Eleven__THE_FORMS_OF_THO.xhtmlux5cux23id_2615}{\protect\hyperlink{23_NOTES.xhtmlux5cux23id_2616}{†\textsuperscript{26}}}

Another crystalized form of thought similar to the proverb is the motto,
a favorite object of careful cultivation during late medieval times.
Mottoes are not, as are proverbs, wisdom generally applied, but are
entirely personal adages. The motto was raised to the status of a kind
of insignia for the person who possessed it, attached with golden
letters to his own life to serve as a lesson that, by virtue of its
formal repetition, resulting from the fact that it was attached to all
pieces of clothing and personal objects, was expected to provide support
for himself and others and to suggest ideas to both. The sentiment of
the motto, in most instances, is one of surrender, just as in the case
of proverbs, or one of expectation, occasionally with the touch of an
unarticulated element to render the motto mysterious: ``Quand sera ce?''
``Tost ou tard vienne'' ``Va oultre'' ``Autre fois mieulx.'' ``Plus
deuil que
joye.''\protect\hypertarget{18_Chapter_Eleven__THE_FORMS_OF_THO.xhtmlux5cux23id_2617}{\protect\hyperlink{23_NOTES.xhtmlux5cux23id_2618}{‡\textsuperscript{27}}}
The great majority of them refer to love: ``Aultre naray.'' ``Vostre
plaisir.''
\protect\hypertarget{18_Chapter_Eleven__THE_FORMS_OF_THO.xhtmlux5cux23page_276}{}{}``Souvienne
vous'' ``Plus que toutes.''
\protect\hypertarget{18_Chapter_Eleven__THE_FORMS_OF_THO.xhtmlux5cux23id_2619}{\protect\hyperlink{23_NOTES.xhtmlux5cux23id_2620}{*\textsuperscript{28}}}
These are knightly mottoes displayed on saddle blankets and armor. On
rings we find some of a more intimate nature: ``Mon cuer avez.'' ``Je le
desire.'' ``Pour toujours'' ``Tout pour
vous.''\protect\hypertarget{18_Chapter_Eleven__THE_FORMS_OF_THO.xhtmlux5cux23id_2621}{\protect\hyperlink{23_NOTES.xhtmlux5cux23id_2622}{t\textsuperscript{29}}}

Emblems complement mottoes. They either illustrate the mottoes in
tangible form or are loosely connected with them, like the knotty staff
joined to the motto ``Je l'envie,'' and the porcupine with ``Cominus et
enimus'' of Louis d'Orléans, and the plane with ``Ic houd'' of his enemy
John the Fearless, or the flint and steel of Philip the
Good.\textsuperscript{\protect\hypertarget{18_Chapter_Eleven__THE_FORMS_OF_THO.xhtmlux5cux23id_588}{\protect\hyperlink{23_NOTES.xhtmlux5cux23id_589}{22}}}
Motto and emblem have their home in the heraldic sphere of thought. To
medieval man, the coat of arms is more than a genealogical hobby. A
man's arms assume a significance like that of a
totem.\textsuperscript{\protect\hypertarget{18_Chapter_Eleven__THE_FORMS_OF_THO.xhtmlux5cux23id_586}{\protect\hyperlink{23_NOTES.xhtmlux5cux23id_587}{23}}}
Lions, lilies, and crosses become symbols in which an entire complex of
pride and aspiration, fidelity and sense of community are expressed in
an independent, indivisible image.

The need to isolate every case as an independently existing entity, to
see it as an idea, expresses itself as the strong medieval inclination
towards casuistry: another result of the far-reaching idealism. To every
question, there is an ideal solution; this ideal solution is arrived at
as soon as one has recognized the correct relationship between the case
at hand and the eternal truths. This relationship is to be deduced by
the application of formal rules to the facts. Not only questions of
ethics and law are answered in this way; the casuistic view also
dominates a number of other spheres of life. Wherever style and form are
the main concern, wherever the element of play comes to the forefront in
a cultural form, casuistry is triumphant. This is true, first and
foremost, in matters of ceremony and etiquette. Here the casuistic view
has its proper place; here it is an adequate form of thought for the
questions raised because they involve only a sequence of cases that are
determined by honored precedence and formal rules. The same is true for
the similar ``games'' of the coat of arms and the hunt. As mentioned
earlier,\textsuperscript{\protect\hypertarget{18_Chapter_Eleven__THE_FORMS_OF_THO.xhtmlux5cux23id_584}{\protect\hyperlink{23_NOTES.xhtmlux5cux23id_585}{24}}}
the conception of love as a beautiful social game of stylish forms and
rules gave rise to the need for an elaborate casuistry.

Finally, all kinds of casuistry were attached to the customs of war. The
strong influence of the chivalric idea on the entire notion of war gave
to the latter an element of play. Issues such as legal
\protect\hypertarget{18_Chapter_Eleven__THE_FORMS_OF_THO.xhtmlux5cux23page_277}{}{}claims
to booty, the opening of hostilities, the adherence to a word of honor
were joined to that category of rules that governed the tournament and
the amusements of the hunt. The need to limit the application of force
by laws and rules arose not as much from an instinct for international
law, as from chivalric conceptions of honor and style of life. Only by a
conscientious casuistry and the formulation of strictly formal rules was
it possible to bring the conduct of war somewhat into harmony with the
honor of the knightly estate.

The beginnings of international law are therefore mixed with the rules
of the game involving the use of weapons. In 1352, Geoffroy de Charny
put a number of casuistic questions before King John II of France for
his decision in his capacity as Grand Master of the Order of the Star,
which he had just founded: twenty of the questions concern the
``jouste,'' twenty-one concern the tournament, and ninety-three,
war.\textsuperscript{\protect\hypertarget{18_Chapter_Eleven__THE_FORMS_OF_THO.xhtmlux5cux23id_582}{\protect\hyperlink{23_NOTES.xhtmlux5cux23id_583}{25}}}
Twenty-five years later, Honoré Bonet, prior of Selonart in the Provence
and doctor of canon law, dedicated to the young Charles VI his
\emph{Arbre des batailles}, a tract about martial law, which, according
to later editions, still had practical value in the latter part of the
sixteenth
century.\textsuperscript{\protect\hypertarget{18_Chapter_Eleven__THE_FORMS_OF_THO.xhtmlux5cux23id_580}{\protect\hyperlink{23_NOTES.xhtmlux5cux23id_581}{26}}}
The tract contains a mixture of questions, some of which are of greatest
significance for international law while others are of trifling value
and only concerned with the rules of the game. Is it permissible to wage
war against the infidel without compelling reason? Bonet answers most
emphatically: No, not even for the purpose of converting them. Is a
prince allowed to refuse passage through his territory to another
prince? Does the sacred protection (much violated) that the plowman and
his ox enjoy against the force of war extend to his asses and
servants?\textsuperscript{\protect\hypertarget{18_Chapter_Eleven__THE_FORMS_OF_THO.xhtmlux5cux23id_579}{\protect\hyperlink{23_NOTES.xhtmlux5cux23page_430}{27}}}
Is a clergyman obligated to help his father or his bishop? Is someone
who has lost his borrowed armor during battle obligated to return it? Is
it permissible to do battle on holy days? Which is better, to do battle
on an empty stomach or after a
meal?\textsuperscript{\protect\hypertarget{18_Chapter_Eleven__THE_FORMS_OF_THO.xhtmlux5cux23id_577}{\protect\hyperlink{23_NOTES.xhtmlux5cux23id_578}{28}}}
To all these questions the prior has answers based on biblical passages,
canonical law, and the commentaries.

The most important points of the rules of war were those involving the
taking of prisoners. The ransom of a noble prisoner was among the most
tempting prospects of battle for nobleman and mercenary alike. Here was
an unlimited field for casuistic rules. Here too, international law and
chivalric ``point d'honneur'' are
\protect\hypertarget{18_Chapter_Eleven__THE_FORMS_OF_THO.xhtmlux5cux23page_278}{}{}scrambled.
Are Frenchmen permitted to make, on English soil, captives of poor
merchants, farmers, and herdsmen because a state of war exists with
England? Under what circumstances is it permissible to escape from
captivity? What is the value of a safe
conduct?\textsuperscript{\protect\hypertarget{18_Chapter_Eleven__THE_FORMS_OF_THO.xhtmlux5cux23id_575}{\protect\hyperlink{23_NOTES.xhtmlux5cux23id_576}{29}}}
In the biographical novel \emph{Le Jouvencel} such cases are dealt with
in terms of practical experiences. A dispute between two captains over a
prisoner is brought before the commander: ``I first grabbed him,'' says
the one, ``first by the arm and his right hand and ripped off his
glove.'' ``But,'' says the other, ``he gave me his right hand and his
word first.'' Both actions established claims to the precious
possession, the prisoner, but the latter is recognized as having
precedent. To whom belongs a prisoner who escapes and is recaptured? The
solution is this: If the case happens in the area of the battle, the
prisoner belongs to the new captor, if outside the battlefield, he
remains the property of the original owner. Is a prisoner who has given
his word allowed to run away if his captor puts him in chains in spite
of his having given his word not to run away? What if the captor had
neglected to have him give his
word?\textsuperscript{\protect\hypertarget{18_Chapter_Eleven__THE_FORMS_OF_THO.xhtmlux5cux23id_573}{\protect\hyperlink{23_NOTES.xhtmlux5cux23id_574}{30}}}

The medieval inclination to overestimate the independent value of a
thing or a case results, aside from the casuistic way of thinking, in
another consequence. We are familiar with François Villon's grand
satiric poem ``The Testament,'' in which he bequeathed all his
possessions to friends and enemies. There are several such poetic
testaments; for example, that of ``Barbeau's Mule'' by Henri
Baude.\textsuperscript{\protect\hypertarget{18_Chapter_Eleven__THE_FORMS_OF_THO.xhtmlux5cux23id_571}{\protect\hyperlink{23_NOTES.xhtmlux5cux23id_572}{31}}}
The testament is a handy form, but it is only intelligible if we keep in
mind that medieval men were accustomed to disposing in a will of even
the most trivial of their possessions separately and in great detail. A
poor woman bequeaths her Sunday dress and her bonnet to her parish; her
bed to her godchild, a fur to her nurse, her everyday dress to a pauper,
and four pounds \emph{turnoise}, which constitutes her only wealth,
together with yet another dress and bonnet to the
Minorites.\textsuperscript{\protect\hypertarget{18_Chapter_Eleven__THE_FORMS_OF_THO.xhtmlux5cux23id_569}{\protect\hyperlink{23_NOTES.xhtmlux5cux23id_570}{32}}}
Is this not a very trivial example of the frame of mind that postulated
every case of virtue to be an eternal example and that saw in every
fashion a divine ordinance? The adhesion of the mind to the
particularity and value of each single thing is what dominates the mind
of the collector and the miser like a disease.

All the features listed above may be summarized under the term
formalism. The inherent conception of the transcendent reality of
\protect\hypertarget{18_Chapter_Eleven__THE_FORMS_OF_THO.xhtmlux5cux23page_279}{}{}things
means that every notion is defined by fixed borders, that it stands
isolated in a plastic form, and that this form is all important. Mortal
and venial sins can be distinguished according to fixed rules. The sense
of justice is unshakable; it need not sway for a moment: as the old
legal dictum has it, the deed judges the man. In judging a deed, the
formal content is always the main point. Long ago, in the primitive law
of ancient Germanic times, that formalism was so strong that the
dispensations of justice did not take into account whether intent or
negligence was involved: the deed was the deed and brought in its wake
its punishment. A deed left undone, or a crime merely attempted, went
unpunished.\textsuperscript{\protect\hypertarget{18_Chapter_Eleven__THE_FORMS_OF_THO.xhtmlux5cux23id_567}{\protect\hyperlink{23_NOTES.xhtmlux5cux23id_568}{33}}}
Even in much more recent times, an accidental lapse during the
recitation of the oath formula could still lead to the loss of legal
rights: an oath is an oath and is very sacred. Economic interests meant
the end of that formalism. The foreign merchant who had only an
imperfect command of the local language could not be exposed to this
risk without raising the possibility that commerce might be impeded. It
is only to be expected that in the laws of the cities, the ``Vare,'' the
danger of losing one's rights in this way was eliminated; initially as a
special privilege and ultimately as a general rule. However, the
vestiges of a far-reaching formalism in legal matters remain quite
numerous in later medieval times.

The extreme sensitivity to anything touching external honor is a
phenomenon rooted in formalistic thought. In 1445 a certain Jan van
Domburg had fled into a church in Middelburg in order to seek asylum
because of a charge of murder against him. As was the custom, the place
of refuge was surrounded on all sides. His sister, a nun, was observed
repeatedly urging him to be killed fighting rather than force his family
to endure the shame of having him fall into the hand of his
executioners. When this finally happened, the Domburg maiden claimed his
body so that at least he could be given a dignified
burial.\textsuperscript{\protect\hypertarget{18_Chapter_Eleven__THE_FORMS_OF_THO.xhtmlux5cux23id_565}{\protect\hyperlink{23_NOTES.xhtmlux5cux23id_566}{34}}}
For tournaments the saddle blanket of a nobleman's horse was customarily
decorated with his coat of arms. Olivier de la Marche finds this rather
improper because the horse, ``une beste irraisonnable,'' might stumble
and the arms be dragged in the dirt. The entire family would be
dishonored.\textsuperscript{\protect\hypertarget{18_Chapter_Eleven__THE_FORMS_OF_THO.xhtmlux5cux23id_563}{\protect\hyperlink{23_NOTES.xhtmlux5cux23id_564}{35}}}
Shortly after the duke of Burgundy had paid a visit to the Chastel en
Porcien, a nobleman in a fit of madness attempted suicide there. The
event causes indescribable horror, ``et n'en
savoit-\protect\hypertarget{18_Chapter_Eleven__THE_FORMS_OF_THO.xhtmlux5cux23page_280}{}{}on
comment porter la honte après se grant joye
demenée.''\protect\hypertarget{18_Chapter_Eleven__THE_FORMS_OF_THO.xhtmlux5cux23id_2623}{\protect\hyperlink{23_NOTES.xhtmlux5cux23id_2624}{*\textsuperscript{30}}}
Although it was well known that the act was caused by madness, the
unfortunate perpetrator was banned from the chateau after his recovery,
``et ahonty à tous
jours.''\textsuperscript{\protect\hypertarget{18_Chapter_Eleven__THE_FORMS_OF_THO.xhtmlux5cux23id_561}{\protect\hyperlink{23_NOTES.xhtmlux5cux23id_562}{36}}}\protect\hypertarget{18_Chapter_Eleven__THE_FORMS_OF_THO.xhtmlux5cux23id_2625}{\protect\hyperlink{23_NOTES.xhtmlux5cux23id_2626}{†\textsuperscript{31}}}

A telling example of the plastic way in which the need to rehabilitate
violated honor was met is provided by the following case. In 1478 a
certain Laurent Guernier was erroneously hanged in Paris. He had
received a pardon for his crime, but he was not informed in time. This
was discovered after a year, and his body was given an honorable funeral
at the request of his brother. The bier was preceded by the four town
criers replete with their rattles, the coat of arms of the dead man on
their chests. Surrounding the bier were four candle bearers and eight
torchbearers. All wore mourning clothes and displayed the same coat of
arms. The funeral party proceeded through Paris from the Porte Saint
Denis to the Porte Saint Antoine. From there, the body was transported
to its birthplace in Provins. One of the criers shouted repeatedly,
``Bonnes gens, dictes voz patenostres pour l'âme de feu Laurent
Guernier, en son vivant demourant à Provins qu'on a nouvellement trouvé
mort soubz ung
chesne.''\textsuperscript{\protect\hypertarget{18_Chapter_Eleven__THE_FORMS_OF_THO.xhtmlux5cux23id_559}{\protect\hyperlink{23_NOTES.xhtmlux5cux23id_560}{37}}}\protect\hypertarget{18_Chapter_Eleven__THE_FORMS_OF_THO.xhtmlux5cux23id_2627}{\protect\hyperlink{23_NOTES.xhtmlux5cux23id_2628}{†\textsuperscript{32}}}

The great strength of the principle of blood revenge, which particularly
thrived in such prosperous and highly cultured regions as northern
France and the southern Netherlands, is also linked to the formalistic
nature of
thought.\textsuperscript{\protect\hypertarget{18_Chapter_Eleven__THE_FORMS_OF_THO.xhtmlux5cux23id_557}{\protect\hyperlink{23_NOTES.xhtmlux5cux23id_558}{38}}}
The motivation in such cases of revenge is frequently not fiery rage or
relentless hatred, but rather that the honor of the offended family has
to be given its due by the shedding of blood. Sometimes, a decision is
made not to kill someone and instead the attempt is made to wound him in
the thighs, arms, or face. Special care is taken not to be burdened with
the responsibility of having had one's opponent die in a state of sin.
Du Clercq tells of a case where people who wanted to murder their
sister-in-law took pains to bring along a
priest.\textsuperscript{\protect\hypertarget{18_Chapter_Eleven__THE_FORMS_OF_THO.xhtmlux5cux23id_555}{\protect\hyperlink{23_NOTES.xhtmlux5cux23id_556}{39}}}

The formal character of atonement and revenge, in its turn,
cre\protect\hypertarget{18_Chapter_Eleven__THE_FORMS_OF_THO.xhtmlux5cux23page_281}{}{}ates
the situation wherein injustice is corrected by symbolic punishments or
exercises of penance. In all the great reconciliations of the fifteenth
century strong emphasis is placed on the symbolic element: the
demolition of the houses that are reminders of the transgression, the
donation of memorial crosses, the walling shut of gates, not to mention
public ceremonies of penance and the funding of masses for the souls of
the departed or chapels. This happened as part of the suit brought by
the house of Orléans against John the Fearless; the peace of Arras in
1435; the penance of rebellious Bruges in 1437; and the even more severe
penance given to Ghent in 1453, where the entire population dressed
entirely in black, without belts, with bare heads and feet and, led by
the main perpetrators, who wore nothing but their shirts in a heavy
downpour, marched in procession to plead with the duke in unison for
forgiveness.\textsuperscript{\protect\hypertarget{18_Chapter_Eleven__THE_FORMS_OF_THO.xhtmlux5cux23id_553}{\protect\hyperlink{23_NOTES.xhtmlux5cux23id_554}{40}}}
During the reconciliation with his brother in 1469, Louis XI first
demands the ring with which the bishop of Lisieux had installed the
prince as duke of Normandy and then, at Rouen, has the ring broken on an
anvil in the presence of
notables.\textsuperscript{\protect\hypertarget{18_Chapter_Eleven__THE_FORMS_OF_THO.xhtmlux5cux23id_551}{\protect\hyperlink{23_NOTES.xhtmlux5cux23id_552}{41}}}

The generally prevailing formalism is also at the base of that faith in
the effect of the spoken word which reveals itself in primitive cultures
in all its fullness and still survives in late medieval times in the
form of formulas of blessing, of magic, and of condemnation. A solemn
appeal still has something of the quality of a wish in a fairy tale.
When intense pleas fail to move Philip the Good to grant clemency to a
condemned man, the task is given to Isabella of Bourbon, Philip's
beloved daughter-in-law, with the hope that he will not be able to turn
her down, because, says she, ``I have never asked you for anything
important.''\textsuperscript{\protect\hypertarget{18_Chapter_Eleven__THE_FORMS_OF_THO.xhtmlux5cux23id_549}{\protect\hyperlink{23_NOTES.xhtmlux5cux23id_550}{42}}}
And the plan works.---The same spirit is also revealed in Gerson's
expressed surprise that morals have not improved in spite of all the
preaching: I don't know what I can say, sermons are given all the time,
but always in
vain.\textsuperscript{\protect\hypertarget{18_Chapter_Eleven__THE_FORMS_OF_THO.xhtmlux5cux23id_547}{\protect\hyperlink{23_NOTES.xhtmlux5cux23id_548}{43}}}

Those qualities that frequently give the mind of the later Middle Ages
its hollow and superficial character are directly spawned by general
formalism. First is the unusual simplification of motivation. Given the
hierarchical order of the system of classification, and taking as the
point of departure the plastic independence of any notion and the need
to explain any connection on the basis of a generally valid truth, the
causal mental function works like a telephone switchboard: all kinds of
connections may occur at any time,
\protect\hypertarget{18_Chapter_Eleven__THE_FORMS_OF_THO.xhtmlux5cux23page_282}{}{}but
always only of two numbers at a time. Only isolated features of any
condition or any connection are seen and these features are greatly
exaggerated and variously embellished; the picture of an experience
always shows the few and heavy lines of a primitive woodcut. One motif
always suffices as an explanation with a predilection for the most
general, the most direct, or the crudest. For the Burgundians, the
motive for the murder of the duke of Orléans can be only one thing: the
king has asked the duke of Burgundy to avenge the adulterous affair of
the queen with
Orléans.\textsuperscript{\protect\hypertarget{18_Chapter_Eleven__THE_FORMS_OF_THO.xhtmlux5cux23id_545}{\protect\hyperlink{23_NOTES.xhtmlux5cux23id_546}{44}}}
A pure question of the style of a formal letter is enough explanation in
the minds of contemporaries for the great uprising at
Ghent.\textsuperscript{\protect\hypertarget{18_Chapter_Eleven__THE_FORMS_OF_THO.xhtmlux5cux23id_543}{\protect\hyperlink{23_NOTES.xhtmlux5cux23id_544}{45}}}

The medieval mind loves to generalize a case. Olivier de la Marche
concludes from a single case of English impartiality in earlier times
that the English were virtuous in those days and that, for that very
reason, they would be able to conquer
France.\textsuperscript{\protect\hypertarget{18_Chapter_Eleven__THE_FORMS_OF_THO.xhtmlux5cux23id_541}{\protect\hyperlink{23_NOTES.xhtmlux5cux23id_542}{46}}}
The tremendous exaggeration that results directly from the fact that
cases are seen as too highly colored and too isolated is further
strengthened by virtue of there always being for every case ready
parallels from Holy Scripture that elevate the case into a sphere of
higher consequence. When, for example, in 1404 a procession of Parisian
students is disrupted and two of the students are injured while another
has his coat ripped to shreds, the outraged chancellor is placated by a
suggestive connection, ``les enfants, les jolis escoliers comme agneaux
innocens,''\protect\hypertarget{18_Chapter_Eleven__THE_FORMS_OF_THO.xhtmlux5cux23id_2629}{\protect\hyperlink{23_NOTES.xhtmlux5cux23id_2630}{*\textsuperscript{33}}}
whereupon he immediately compares the case to the slaughter of the
Innocents in
Bethlehem.\textsuperscript{\protect\hypertarget{18_Chapter_Eleven__THE_FORMS_OF_THO.xhtmlux5cux23id_539}{\protect\hyperlink{23_NOTES.xhtmlux5cux23id_540}{47}}}

Where for every case an explanation is so easily at hand and where this
explanation, once it has been accepted, is firmly believed, an unusual
potential for mistaken judgments prevails. If we have to assume, with
Nietzsche, that ``to do without mistaken judgments would make life
impossible,'' then the vigorous life that attracts our attention in
earlier times has to be credited in part to these very mistaken
judgments. Any age demanding an extraordinary mobilization of all its
strength demands that mistaken judgments come in a higher degree to the
assistance of the nerves. Medieval men may be said to have lived
continuously in such an intellectual crisis; they were unable, even for
a moment, to do without the crudest of mistaken judgments that, under
the influence of partisanship, reached an unparalleled degree of
viciousness. The
\protect\hypertarget{18_Chapter_Eleven__THE_FORMS_OF_THO.xhtmlux5cux23page_283}{}{}Burgundian
attitude towards the great enmity with Orléans demonstrates this. The
numerical proportions of the dead of both sides were distorted by the
victors to a ridiculous degree: Chastellain has five noblemen killed in
the battle of Gavere on the side of the princes as compared to twenty or
thirty thousand on the side of the Ghent
rebels.\textsuperscript{\protect\hypertarget{18_Chapter_Eleven__THE_FORMS_OF_THO.xhtmlux5cux23id_537}{\protect\hyperlink{23_NOTES.xhtmlux5cux23id_538}{48}}}
It is one of Commines's most modern characteristics that he does not
indulge in these
exaggerations.\textsuperscript{\protect\hypertarget{18_Chapter_Eleven__THE_FORMS_OF_THO.xhtmlux5cux23id_535}{\protect\hyperlink{23_NOTES.xhtmlux5cux23id_536}{49}}}

What are we to make of the peculiar rashness that is continuously
revealed in the superficiality, inexactness, and credulity of the waning
Middle Ages? It is almost as if they had not even the slightest need for
real thought, as if the passage of fleeting and dream-like images
provided sufficient nourishment for their minds. Purely outward
circumstances, superficially described: that is the hallmark of scribes
such as Froissart and Monstrelet. How did the endlessly indecisive
battles and sieges over which Froissart wasted his talent manage to
rivet his attention? Side by side with determined partisans, we find
among the chroniclers men whose political sympathies cannot be
determined at all, for example, Froissart and Pierre de Fenin, because
their talents are exhausted to such a large degree in narrating the
minutiae of external events. They do not distinguish the important from
the unimportant. Monstrelet was present at the conversation between the
duke of Burgundy and his captive Jeanne d'Arc, but does not recall what
they talked
about.\textsuperscript{\protect\hypertarget{18_Chapter_Eleven__THE_FORMS_OF_THO.xhtmlux5cux23id_533}{\protect\hyperlink{23_NOTES.xhtmlux5cux23id_534}{50}}}
This inexactness, even with respect to events that were important to
themselves, knows no bounds. Thomas Basin, who supervised the process of
rehabilitating Jeanne d'Arc, says in his chronicle that she was born in
Vaucouleurs. He has her brought by Baudricourt himself, whom he
identifies as the lord rather than the captain of the city of Tours, and
miscalculates the date of her first meeting with the Dauphin by three
months.\textsuperscript{\protect\hypertarget{18_Chapter_Eleven__THE_FORMS_OF_THO.xhtmlux5cux23id_531}{\protect\hyperlink{23_NOTES.xhtmlux5cux23id_532}{51}}}
Olivier de la Marche, the jewel among all the courtiers, errs
consistently in matters of descent and relationship in the ducal family
and even is mistaken about the date (in 1468) of the marriage of Charles
the Bold and Margaret of York. He had himself participated in the
festivities of the event, but dates them after the siege of Neuss in
1475.\textsuperscript{\protect\hypertarget{18_Chapter_Eleven__THE_FORMS_OF_THO.xhtmlux5cux23id_529}{\protect\hyperlink{23_NOTES.xhtmlux5cux23id_530}{52}}}
Even Commines is not free of such confusions: he repeatedly multiplies
any given span of years by two, and repeats his tale of the death of
Adolph of Gelder three
times.\textsuperscript{\protect\hypertarget{18_Chapter_Eleven__THE_FORMS_OF_THO.xhtmlux5cux23id_527}{\protect\hyperlink{23_NOTES.xhtmlux5cux23id_528}{53}}}

The lack of an ability to make critical distinctions and the degree of
credulity are so clearly manifest on each page of medieval
litera\protect\hypertarget{18_Chapter_Eleven__THE_FORMS_OF_THO.xhtmlux5cux23page_284}{}{}ture
that it is unnecessary to cite examples. Naturally there are large
gradations depending on the level of education of particular
individuals. Among the people of Burgundy that peculiar form of barbaric
credulity still dominated that never really believed in the death of an
imposing ruler; this credulity was alive with respect to Charles the
Bold so that as late as ten years after the battle of Nancy, people
would still lend money on the terms that it would be repaid when the
duke would return. Basin sees in this nothing but foolishness, as does
Molinet; he mentions it among his ``Mervilles du monde:''

\emph{J'ay veu chose incongneue}:

\emph{Ung mort ressusciter},

\emph{Et sur sa revenue}

\emph{Par milliers achapter}.

\emph{L'ung dit: il est en vie},

\emph{L'autre: ce n'est que vent}.

\emph{Tous bons cueurs sans envie}

\emph{Le regrettent
souvent.\textsuperscript{\protect\hypertarget{18_Chapter_Eleven__THE_FORMS_OF_THO.xhtmlux5cux23id_525}{\protect\hyperlink{23_NOTES.xhtmlux5cux23id_526}{54}}}\protect\hypertarget{18_Chapter_Eleven__THE_FORMS_OF_THO.xhtmlux5cux23id_2631}{\protect\hyperlink{23_NOTES.xhtmlux5cux23id_2632}{*\textsuperscript{34}}}}

However, given the influence of strong passion and the all too ready
power of imagination, belief in the reality of imagined facts easily
took root among the people. Given the disposition of the mind to think
in terms of strongly isolated conceptions, the mere presence of an idea
in the mind soon led to the assumption of its credibility. Once an idea
had begun to bounce about in the brain with a particular name or form,
it likely would be taken into the system of moral and religious images
and automatically come to share their high credibility.

While on the one hand, ideas, by virtue of their sharp definition, their
hierarchical connections, and their frequently anthropomorphic
character, are particularly fixed and immobile, there is, on the other
hand, the danger that in the vivid form of the idea its content would be
lost. Eustache Deschamps dedicated a long allegorical and satirical
didactic poem, ``Le Miroir de
Mariage,''\textsuperscript{\protect\hypertarget{18_Chapter_Eleven__THE_FORMS_OF_THO.xhtmlux5cux23id_523}{\protect\hyperlink{23_NOTES.xhtmlux5cux23id_524}{55}}}
to the
disad\protect\hypertarget{18_Chapter_Eleven__THE_FORMS_OF_THO.xhtmlux5cux23page_285}{}{}vantages
of marriage. One of its major characters is Franc
Vouloir,\protect\hypertarget{18_Chapter_Eleven__THE_FORMS_OF_THO.xhtmlux5cux23id_2347}{\protect\hyperlink{23_NOTES.xhtmlux5cux23id_2348}{*\textsuperscript{35}}}
spurred on by Folie and Désir to marriage, but prevented from doing so
by Repertoire de science.

What meaning does the poet intend to confer on the abstraction Franc
Vouloir? In one sense he is the gay freedom of the bachelor, but at
other times, free will in the philosophical sense. The imagination of
the poet is so strongly absorbed by the personification of Franc
Vouloir, in its own right, that he does not feel any need to clearly
define the idea of his figure, but allows him to move from one extreme
to another.

The same poem illustrates in another way how an idea, once elaborated,
becomes amorphous or evaporates entirely. The tone of the poem echoes
the familiar philistine and basically sensuous ridicule of the weakness
and virtue of women; an amusement throughout the Middle Ages. To our
sensibilities, the pious praises of a spiritual marriage and of the
contemplative life itself, which Repertoire de science serves up to his
friend Franc Vouloir in the latter part of the poem, are a crass
dissonance with that
tone.\textsuperscript{\protect\hypertarget{18_Chapter_Eleven__THE_FORMS_OF_THO.xhtmlux5cux23id_521}{\protect\hyperlink{23_NOTES.xhtmlux5cux23id_522}{56}}}
But it is equally strange to us that the poet occasionally puts high
truths in the mouths of Folie and Désir, truths that we would expect to
come from the other side of the
dispute.\textsuperscript{\protect\hypertarget{18_Chapter_Eleven__THE_FORMS_OF_THO.xhtmlux5cux23id_519}{\protect\hyperlink{23_NOTES.xhtmlux5cux23id_520}{57}}}

Here, as is frequently the case in the expressions of the Middle Ages,
we are faced with the question: Did the poet take what he praised
seriously? Just as we could have asked: Did Jean Petit and his
Burgundian patrons believe all the gruesome details with which they
soiled the memory of Orléans? Did the princes and noblemen really take
all the bizarre fantasies and comedies with which they embellished their
knightly schemes and vows seriously? It is extremely difficult, in
matters of medieval thought, to clearly separate seriousness from play,
the honest conviction from that mental disposition that the English call
``pretending,'' which is the disposition of a child at play that also
occupies such an important place in primitive
cultures,\textsuperscript{\protect\hypertarget{18_Chapter_Eleven__THE_FORMS_OF_THO.xhtmlux5cux23id_517}{\protect\hyperlink{23_NOTES.xhtmlux5cux23id_518}{58}}}
and that is expressed less through \emph{geveinsdheid}
(``make-believe'') than through \emph{aacnstellerij} (``act as if'').

This blending of seriousness and play is characteristic of several
areas. Above all, it is war into which people like to inject a comic
element. The ridicule directed by the besieged upon their enemies is
something they are sometimes made to pay for dearly. The
peo\protect\hypertarget{18_Chapter_Eleven__THE_FORMS_OF_THO.xhtmlux5cux23page_286}{}{}ple
of Meaux put an ass on their wall to torment Henry V of England; the
people of Condé declare that they are not yet able to surrender because
they were still baking their Easter cakes; in Montereau the burghers
standing on the walls dust off their helmets after the cannons of the
besiegers
fire.\textsuperscript{\protect\hypertarget{18_Chapter_Eleven__THE_FORMS_OF_THO.xhtmlux5cux23id_515}{\protect\hyperlink{23_NOTES.xhtmlux5cux23id_516}{59}}}
In the same vein, the camp of Charles the Bold at Neuss was set up like
a vast country fair; the noblemen have, ``par plaisance,'' their tents
built in the form of castles complete with galleries and gardens. All
kinds of amusements are
provided.\textsuperscript{\protect\hypertarget{18_Chapter_Eleven__THE_FORMS_OF_THO.xhtmlux5cux23id_513}{\protect\hyperlink{23_NOTES.xhtmlux5cux23id_514}{60}}}

There is one area where the addition of ridicule to the most serious
matters seems garish: the dark arena of the belief in devils and
witches. Although the fantasies about devils were directly rooted in the
deep fear that nourished this belief, the naive imagination nevertheless
rendered such figures so childishly colorful and so familiar to everyone
that they sometimes lost their terrifying aspect. It is not only in
literature that the Devil appears as a comic figure; even in the
gruesome seriousness of the witchcraft trials Satan's company is
frequently fashioned in the manner of Hieronymus Bosch and the hellish
smell of sulfur blends with the fluff of the farce. The devils who,
under their captains Tahu and Gorgias, threw a cloister of nuns into
disorder have names ``assez consonnans aux noms de mondains habits,
instruments et jeux du temps présent, comme Pantoufle, Courtaulx et
Mornifle.''\textsuperscript{\protect\hypertarget{18_Chapter_Eleven__THE_FORMS_OF_THO.xhtmlux5cux23id_511}{\protect\hyperlink{23_NOTES.xhtmlux5cux23id_512}{61}}}\protect\hypertarget{18_Chapter_Eleven__THE_FORMS_OF_THO.xhtmlux5cux23id_2633}{\protect\hyperlink{23_NOTES.xhtmlux5cux23id_2634}{*\textsuperscript{36}}}

The fifteenth century is more than any other the century of the
persecution of witches. At the very moment with which we customarily
conclude the Middle Ages and delight in the flourishing of humanism, the
systematic elaboration of the witch craze, that terrible outgrowth of
medieval thought, is revealed by the \emph{Malleus maleficarum} and the
\emph{Bulle summis desiderantes} (1487 and 1484). No humanism, no
Reformation prevents this madness. Does not the humanist Jean Bodin,
even after 1550, in his \emph{Demonomie} give the most learned and
substantial nourishment to this persecution mania? The new times and the
new knowledge did not immediately reject the cruelties of the
witch-hunts. Oddly, the more temperate pronouncements about witchcraft
that were proclaimed by the Gelder physician Johannes Wier were already
widely suggested during the fifteenth century.

\protect\hypertarget{18_Chapter_Eleven__THE_FORMS_OF_THO.xhtmlux5cux23page_287}{}{}The
attitude of the late medieval mind towards superstition, that is
witchcraft and magic, is quite vacillatory and fluid. The age is not
quite as helplessly given to all this witchcraft madness as one is
tempted to conclude given its general credulity and lack of critical
thinking. There are many expressions of doubt and of rational thought.
Time and again evil erupts from a new cauldron of demonic mania and
succeeds in surviving for a long time. Magic and witches were at home in
special regions, for the most part mountainous areas: Savoy,
Switzerland, Lorraine, and Scotland. But epidemic eruptions occur
outside of those areas. Around 1400 even the French court was a hotbed
of sorcery. A preacher warns the court nobility that care must be taken
lest the phrase ``vieilles sorcières'' gradually come to be ``nobles
sorciers.''\textsuperscript{\protect\hypertarget{18_Chapter_Eleven__THE_FORMS_OF_THO.xhtmlux5cux23id_509}{\protect\hyperlink{23_NOTES.xhtmlux5cux23id_510}{62}}}
Particularly the atmosphere of the circles around Louis d'Orléans was
charged with the devilish arts; the charges and suspicions raised by
Jean Petit did not, in this respect, lack all justification. Orléans's
friend and adviser, the aged Philippe de Mézières himself, suspected by
the Burgundians to be the mysterious instigator of all those misdeeds,
reports that he had learned magic from a Spaniard some time ago and how
much effort it had cost him to forget this evil knowledge. As late as
ten or twelve years after he had left Spain, ``à sa volenté ne povoit
pas bien extirper de son cuer les dessusdits signes et l'effect d'iceulx
contra
Dieu,''\protect\hypertarget{18_Chapter_Eleven__THE_FORMS_OF_THO.xhtmlux5cux23id_2635}{\protect\hyperlink{23_NOTES.xhtmlux5cux23id_2636}{*\textsuperscript{37}}}
until he was finally saved through confession and resistance with the
help of God's mercy, ``de ceste grand folie, qui est à l'âme crestienne
anemie.''\textsuperscript{\protect\hypertarget{18_Chapter_Eleven__THE_FORMS_OF_THO.xhtmlux5cux23id_508}{\protect\hyperlink{23_NOTES.xhtmlux5cux23page_431}{63}}}
Masters of witchcraft were preferably sought in remote regions; a man
desirous of conversing with the devil and unable to find anyone to teach
him this art is told to go to ``Ecosse la
sauvage.''\textsuperscript{\protect\hypertarget{18_Chapter_Eleven__THE_FORMS_OF_THO.xhtmlux5cux23id_506}{\protect\hyperlink{23_NOTES.xhtmlux5cux23id_507}{64}}}\protect\hypertarget{18_Chapter_Eleven__THE_FORMS_OF_THO.xhtmlux5cux23id_2637}{\protect\hyperlink{23_NOTES.xhtmlux5cux23id_2638}{†\textsuperscript{38}}}

Orléans had his own masters of witchcraft and necromancers. He had one
of them, whose skill did not satisfy him, burnt at the
stake.\textsuperscript{\protect\hypertarget{18_Chapter_Eleven__THE_FORMS_OF_THO.xhtmlux5cux23id_504}{\protect\hyperlink{23_NOTES.xhtmlux5cux23id_505}{65}}}
Admonished to check whether in the opinion of the scholars his
superstitious practices were permissible, he responded, ``Why should I
consult them? I am well aware that they would counsel against it and yet
I am absolutely determined to keep acting and believing as before, and I
will not give it
up.''\textsuperscript{\protect\hypertarget{18_Chapter_Eleven__THE_FORMS_OF_THO.xhtmlux5cux23id_502}{\protect\hyperlink{23_NOTES.xhtmlux5cux23id_503}{66}}}
Gerson links
\protect\hypertarget{18_Chapter_Eleven__THE_FORMS_OF_THO.xhtmlux5cux23page_288}{}{}these
stubbornly sinful practices to Orléans's sudden death; he disapproves of
the attempts to cure the mad king with the help of magic; one
practitioner had already died in the flames for his lack of
success.\textsuperscript{\protect\hypertarget{18_Chapter_Eleven__THE_FORMS_OF_THO.xhtmlux5cux23id_500}{\protect\hyperlink{23_NOTES.xhtmlux5cux23id_501}{67}}}

One special magical practice in particular is repeatedly mentioned as
having been current at princely courts; this practice, called
\emph{invultare} in Latin and \emph{envoûtement} in French, is the
attempt, known all over the world, to destroy one's enemies by having a
baptised wax figure, or another image, cursed in his name, or melted or
pierced. Philip VI of France is said to have had such an image of
himself, which had come into his possession, thrown into the fire with
the words, ``We will see who is the more powerful, the devil to ruin me
or God to save
me.''\textsuperscript{\protect\hypertarget{18_Chapter_Eleven__THE_FORMS_OF_THO.xhtmlux5cux23id_498}{\protect\hyperlink{23_NOTES.xhtmlux5cux23id_499}{68}}}---The
Burgundian dukes, too, were persecuted in this manner. ``N'ay-je devers
moy''---complains Charolais bitterly---``les bouts de cire baptisés
dyaboliquement et pleins d'abominables mystères contre moy et
autres?''\textsuperscript{\protect\hypertarget{18_Chapter_Eleven__THE_FORMS_OF_THO.xhtmlux5cux23id_496}{\protect\hyperlink{23_NOTES.xhtmlux5cux23id_497}{69}}}\protect\hypertarget{18_Chapter_Eleven__THE_FORMS_OF_THO.xhtmlux5cux23id_2639}{\protect\hyperlink{23_NOTES.xhtmlux5cux23id_2640}{*\textsuperscript{39}}}
Philip the Good, who, in contrast to his royal nephew, represents in
many ways a more conservative view of life, such as in his preference
for chivalry and splendor, in his crusade plans, in the old-fashioned
literary forms that he protected, seems to have been leaning to a more
enlightened opinion in manners of superstition than the French court,
particularly Louis XI himself. Philip puts no store in the inauspicious
day of the Innocent Children that repeats itself every week; he does not
seek information about the future from astrologers and fortune-tellers,
``car en toutes choses se monstra homme de lealle entière foy envers
Dieu, sans enquerir riens de ses
secrets,''\protect\hypertarget{18_Chapter_Eleven__THE_FORMS_OF_THO.xhtmlux5cux23id_2641}{\protect\hyperlink{23_NOTES.xhtmlux5cux23id_2642}{†\textsuperscript{40}}}
says Chastellain, who shares the same
position.\textsuperscript{\protect\hypertarget{18_Chapter_Eleven__THE_FORMS_OF_THO.xhtmlux5cux23id_494}{\protect\hyperlink{23_NOTES.xhtmlux5cux23id_495}{70}}}
Through the duke's intervention the terrible persecution of witches and
magicians in Arras in 1461, one of the great epidemics of the witch
craze, came to an end.

The terrible delusion of which the persecution of witches is in part the
result was contributed to by the fact that the concepts of magic and
heresy had become confused. In general, everything emotionally linked to
the disgust, fear, and hatred of intolerable
\protect\hypertarget{18_Chapter_Eleven__THE_FORMS_OF_THO.xhtmlux5cux23page_289}{}{}transgressions,
even such things outside of the direct realm of faith, was expressed by
the term heresy. Monstrelet, for example, calls the sadistic crimes of
Gilles de Rais simply
``hérésie.''\textsuperscript{\protect\hypertarget{18_Chapter_Eleven__THE_FORMS_OF_THO.xhtmlux5cux23id_492}{\protect\hyperlink{23_NOTES.xhtmlux5cux23id_493}{71}}}
The common word for magic in fifteenth-century France was
\emph{vauderie}, which had lost its particular link with the
Waldensians. In the \emph{``Vauderie d'Arras,''} we can trace both the
terrifyingly sick delusion that was shortly to hatch the \emph{Malleus
maleficarum}\textsuperscript{\protect\hypertarget{18_Chapter_Eleven__THE_FORMS_OF_THO.xhtmlux5cux23id_490}{\protect\hyperlink{23_NOTES.xhtmlux5cux23id_491}{72}}}
and the general doubt, among common people and nobles alike, as to the
reality of all the misdeeds that were uncovered. One of the inquisitors
claimed that one-third of Christianity was soiled by \emph{vauderie}.
His trust in God led him to the terrifying conclusion that anyone
accused of sorcery would of necessity be guilty. God would not allow
someone not a magician to be accused of such practices. ``Et quand on
arguoit contre lui, fuissent clercqs ou aultres, disoit qu'on debvroit
prendre iceulx comme suspects d'estre
vauldois.''\protect\hypertarget{18_Chapter_Eleven__THE_FORMS_OF_THO.xhtmlux5cux23id_2643}{\protect\hyperlink{23_NOTES.xhtmlux5cux23id_2644}{*\textsuperscript{41}}}
If someone insists that some of the apparitions are the products of
imagination, this inquisitor calls him suspicious. The inquisitor even
claimed that he could tell if someone was involved in \emph{vauderie}
merely by looking at him. The man went mad in his later years, but the
witches and magicians had been burnt at the stake in the meantime.

The city of Arras acquired such an evil reputation as a result of these
persecutions that its merchants were refused lodging or credit out of
fear that, on the next day, they might be accused of witchcraft and lose
all their possessions to confiscation. All this notwithstanding, says
Jacques du Clercq, outside of Arras not one in a thousand believed in
the truth of all this: ``onques on n'avoit veu es marches de par decha
tels cas
advenu,''\protect\hypertarget{18_Chapter_Eleven__THE_FORMS_OF_THO.xhtmlux5cux23id_2645}{\protect\hyperlink{23_NOTES.xhtmlux5cux23id_2646}{†\textsuperscript{42}}}
When the victims are forced to recant their evil deeds prior to their
execution, the people of Arras themselves have their doubts. A poem,
full of hatred for the prosecutors, accuses them of having started it
all out of greed; the bishop himself calls it a conspiracy, ``une chose
controuvée par aulcunes mauvaises
personnes.''\textsuperscript{\protect\hypertarget{18_Chapter_Eleven__THE_FORMS_OF_THO.xhtmlux5cux23id_488}{\protect\hyperlink{23_NOTES.xhtmlux5cux23id_489}{73:}}}\protect\hypertarget{18_Chapter_Eleven__THE_FORMS_OF_THO.xhtmlux5cux23id_2647}{\protect\hyperlink{23_NOTES.xhtmlux5cux23id_2648}{‡\textsuperscript{43}}}
The duke of Burgundy asks the faculty of Louvain for advice, and several
of its representatives declare that \emph{vauderie} is not real, that it
is a mere illusion. There
\protect\hypertarget{18_Chapter_Eleven__THE_FORMS_OF_THO.xhtmlux5cux23page_290}{}{}upon
Philip sends his king of arms, Toison d'or, to the city and thenceforth
there were no more victims and those still under investigation were
treated more gently.

Finally, the witch trials of Arras were entirely annulled. The city
celebrated the occasion with a joyful feast and edifying moral
plays.\textsuperscript{\protect\hypertarget{18_Chapter_Eleven__THE_FORMS_OF_THO.xhtmlux5cux23id_486}{\protect\hyperlink{23_NOTES.xhtmlux5cux23id_487}{74}}}

The view that the delusions of the witches themselves, their rides
through the sky and their sabbath orgies, were nothing but figments of
their own imagination is a position that had already been advanced
during the fifteenth century by several individuals. But this did not
mean that the role of the Devil has been dropped from the agenda,
because it is he who creates this fateful illusion in the first place;
it is an error, but one which originates with Satan. This is still the
position of Johannes Wier in the sixteenth century. Martin Lefranc,
prior of the Church of Lausanne, the poet of the great ``Le Champion des
dames,'' which he dedicated in 1440 to Philip the Good, expressed the
following enlightened position on witchcraft:

\emph{Il n'est vielle tant estou(r)dye},

\emph{Qui fist de ces choses la mendre}

\emph{Mais pour la faire ou ardre ou pendre},

\emph{L'ennemy de nature humaine},

\emph{Qui trop de faulx engins scet tendre},

\emph{Les sens faussement lui demaine}.

\emph{Il n'est ne baston ne bastonne}

\emph{Sur quoy puist personne voler},

\emph{Mais quant le diable leur estonne}

\emph{La teste, elles cuident aler}

\emph{En quelque place pour galer}

\emph{Et accomplir leur volonté}.

\emph{De Romme on les orra parler},

\emph{Et sy n'y auront jâ esté}.

. \emph{.~.~. . .~.~. .~.~}.

\emph{Les dyables sont tous en abisme},

---\emph{Dist Franc-Vouloir}---\emph{enchaienniez}

\emph{Et n'auront turquoise ni lime}

\emph{Dont soient jà desprisonnez}.

\emph{Comment dont aux cristiennez}

\emph{Viennent ilz faire tant de ruzes}

\emph{\protect\hypertarget{18_Chapter_Eleven__THE_FORMS_OF_THO.xhtmlux5cux23page_291}{}{}Et
tant de cas désordonnez?}

\emph{Entendre ne sçay tes
babuzes.\protect\hypertarget{18_Chapter_Eleven__THE_FORMS_OF_THO.xhtmlux5cux23id_2649}{\protect\hyperlink{23_NOTES.xhtmlux5cux23id_2650}{*\textsuperscript{44}}}}

And, at another place in the same poem:

\emph{Je ne croiray tant que je vive}

\emph{que femme corporellement}

\emph{Voit par l'air comme merle ou grive},

---\emph{Dit le Champion prestement}.---

\emph{Saint Augustine dit plainement}

\emph{C'est illusion et fantosme};

\emph{Et ne le croient aultrement}

\emph{Gregoire, Ambroise ne Jherosme}.

\emph{Quant la pourelle est en sa couche},

\emph{Pour y dormir et reposer},

\emph{L'ennemi qui point ne se couche}

\emph{Se vient encoste allé poser}.

\emph{Lors illusions composer}

\emph{Lui scet sy tres soubtillement},

\emph{Qu'elle croit faire ou proposer}

\emph{Ce qu'elle songe seulement}.

\emph{Force la vielle songera}

\emph{Que sur un chat ou sur un chien}

\emph{A l'assemblée s'en ira};

\emph{Mais certes il n'en sera rien}:

\emph{Et sy n'est baston ne mesrien}

\emph{Qui le peut ung pas
enlever}.\textsuperscript{\protect\hypertarget{18_Chapter_Eleven__THE_FORMS_OF_THO.xhtmlux5cux23id_484}{\protect\hyperlink{23_NOTES.xhtmlux5cux23id_485}{75}}}\protect\hypertarget{18_Chapter_Eleven__THE_FORMS_OF_THO.xhtmlux5cux23id_2651}{\protect\hyperlink{23_NOTES.xhtmlux5cux23id_2652}{†\textsuperscript{45}}}

\protect\hypertarget{18_Chapter_Eleven__THE_FORMS_OF_THO.xhtmlux5cux23page_292}{}{}Froissart
considers the case of the Gascon nobleman and his demon companion,
Horton, whom he describes so masterfully, to be an
``erreur.''\textsuperscript{\protect\hypertarget{18_Chapter_Eleven__THE_FORMS_OF_THO.xhtmlux5cux23id_482}{\protect\hyperlink{23_NOTES.xhtmlux5cux23id_483}{76}}}
Gerson displays a tendency to take his evaluation of devilish illusion
one step further and seek a natural explanation for all kinds of
superstitious phenomena. Many of them, he says, arise simply from the
human imagination and melancholy delusions, and these, in turn, are
based in most instances on some corruption of the power of imagination
that itself can be caused by damage to the brain. Such a view seems
enlightened enough; even as that which holds that pagan vestiges and
poetic inventions play a part in superstition. But even though Gerson
admits that many alleged devilish deeds can be attributed to natural
causes, he too ultimately gives credit to the Devil; that internal
damage to the brain is itself the result of devilish
illusion.\textsuperscript{\protect\hypertarget{18_Chapter_Eleven__THE_FORMS_OF_THO.xhtmlux5cux23id_480}{\protect\hyperlink{23_NOTES.xhtmlux5cux23id_481}{77}}}

Outside the frightful sphere of the persecution of witches, the church
countered superstition with effective and suitable means. The preacher
Brother Richard has ``Madagoires'' (mandrakes, mandragora, alraun)
brought to be burned, ``que maintes sotes gens gardoient en lieux repos,
et avoient si grant foy en celle ordure, que pour vray ilz creoient
fermement, que tant comme ilz l'avoient, mais qu'il fust bien nettement
en beaux drapeaulx de soie ou de lin enveloppé, que jamais jour de leur
vie ne seroient
pouvres.''\textsuperscript{\protect\hypertarget{18_Chapter_Eleven__THE_FORMS_OF_THO.xhtmlux5cux23id_478}{\protect\hyperlink{23_NOTES.xhtmlux5cux23id_479}{78}}}\protect\hypertarget{18_Chapter_Eleven__THE_FORMS_OF_THO.xhtmlux5cux23id_2653}{\protect\hyperlink{23_NOTES.xhtmlux5cux23id_2654}{*\textsuperscript{46}}}---Burghers
who had their palm read by a band of gypsies are excommunicated, and a
procession is held to ward off that evil which could come from
godlessness.\textsuperscript{\protect\hypertarget{18_Chapter_Eleven__THE_FORMS_OF_THO.xhtmlux5cux23id_476}{\protect\hyperlink{23_NOTES.xhtmlux5cux23id_477}{79}}}

A tract by Denis the Carthusian shows clearly where the border between
faith and superstition was drawn, on which basis church doctrine
attempted to reject some ideas and to purify others by imparting to them
a truly religious content. Amulets, acts of conjuration, blessings, and
so forth, says Denis, do not by themselves have the power to cause an
effect. In this, they differ from the
\protect\hypertarget{18_Chapter_Eleven__THE_FORMS_OF_THO.xhtmlux5cux23page_293}{}{}words
of the sacrament, which, when recited with the correct intentions, do
have an undeniable effect because God has, so to speak, tied his power
to these words. Benedictions, however, are only to be regarded as humble
requests, are only to be uttered with the appropriate pious formulation,
and are only based on faith in God. If they usually have an effect, this
is the case either because God imparts to them, if properly done, that
effect or because, in cases where they are done differently---if, for
instance, the Sign of the Cross is made improperly---their effect,
should they actually be effective in spite of everything, is a delusion
of the Devil. But the Devil's works are not miracles, because the devils
know the secret powers of nature; the effect is therefore a natural one
just as the premonitions of birds, etc., are based on natural
causes.---Denis concedes that in folk practices all those blessings,
amulets, and the like appear to have an evident worth, but he denies
their value and voices the opinion that clerics should rather prohibit
all such
things.\textsuperscript{\protect\hypertarget{18_Chapter_Eleven__THE_FORMS_OF_THO.xhtmlux5cux23id_474}{\protect\hyperlink{23_NOTES.xhtmlux5cux23id_475}{80}}}

Generally speaking, the attitude towards anything that appeared to be
supernatural may be characterized as vacillating between rational,
natural explanations, spontaneous pious affirmation, and the distrust of
the tricks of the Devil and devilish deceit. The dictum ``Omnia quae
visibiliter fiunt in hoc mundo, possunt fieri per daemones''
(``Everything that appears visible in this world could be caused by the
Devil''), affirmed by the authority of St. Augustine and St. Thomas
Aquinas, caused uncertainty among the pious. The cases in which a
miserable hysteria drove the burghers into a short-lived frenzy only to
be unmasked after it had run its course are not counted among the rarer
of
occasions.\textsuperscript{\protect\hypertarget{18_Chapter_Eleven__THE_FORMS_OF_THO.xhtmlux5cux23id_472}{\protect\hyperlink{23_NOTES.xhtmlux5cux23id_473}{81}}}
