\chapter{TRANSLATOR'S INTRODUCTION}

THE IDEA OF THIS TRANSLATION HAD ITS MOMENT of conception in Karl J.
Weintraub's class in History of Culture at the University of Chicago
(now more than twenty years ago) when Weintraub commented, with some
heat, on the deficiencies of the English translation of \emph{Herfsttij
der Middeleeuwen} that we students were using when it was compared to
the elegance of the Dutch edition he had on the lectern. The tiny
margins of my crumbling paperback are filled with all my efforts to get
down the corrections. When I began my own teaching of Huizinga's text,
which I had come to treasure, those illegible notes suggested that what
I was professing fell far short and an examination of the original
showed me that Weintraub's observations were justified. Yet, in spite of
the shortcomings of the translation, my students always responded well
to Huizinga. Later Professor Weintraub commented to me that it was an
indication of the power of its subject and style that Huizinga's book
commonly captivated readers in spite of the ``very inferior, crippled
version''\textsuperscript{\protect\hypertarget{05_TRANSLATOR_S_INTRODUCTION.xhtmlux5cux23id_2247}{\protect\hyperlink{23_NOTES.xhtmlux5cux23id_2248}{1}}}
in which it appeared in English.

Therefore when my colleague Ulrich Mammitzsch, now deceased, and I
agreed to attempt a new translation there was a certain feeling of being
the rescuers of something fine that had been corrupted and undervalued.
However, this feeling was somewhat challenged by the fact that Huizinga
not only authorized the English translation, but also apparently
collaborated with Fritz Hopman in producing it as a variant version of
the book. He specifically approved the results in the preface he wrote
for the translation.

This English edition is not a simple translation of the original Dutch
(second edition 1921, first 1919), but the result of a work of
adaptation, reduction and consolidation under the author's direction.
The references, here left out, may be found in full in the original.
.~.~.

\protect\hypertarget{05_TRANSLATOR_S_INTRODUCTION.xhtmlux5cux23page_x}{}{}The
author wishes to express his sincere thanks to .~.~. Mr. F. Hopman, of
Leiden, whose clear insight into the exigencies of translation rendered
the recasting possible, and whose endless patience with the wishes of an
exacting author made the difficult task a work of friendly
cooperation.\textsuperscript{\protect\hypertarget{05_TRANSLATOR_S_INTRODUCTION.xhtmlux5cux23id_2245}{\protect\hyperlink{23_NOTES.xhtmlux5cux23id_2246}{2}}}

Even given this endorsement by the author, I think that any studious
reader of both the Dutch (or the very accurate German translation) and
the English would conclude that the original is a much better book. The
original is nearly one-third longer and has many more citations of
original material. In the Hopman translation, blocks of text are
inexplicably moved around, and sometimes Hopman's usually good English
fails him as when he translates ``mystiek en détail'' as ``mysticism by
retail.'' It seems that Huizinga ultimately must have thought the
original better, as none of the ``adaptation, reduction and
consolidation'' found its way into subsequent Dutch printings or foreign
translations of the book with the exception of the revised arrangement
of chapters.

The route by which Huizinga arrived at the Hopman translation can be
traced in the \emph{Briefwisseling (Correspondence)}, if not, entirely,
his motivation for taking
it.\textsuperscript{\protect\hypertarget{05_TRANSLATOR_S_INTRODUCTION.xhtmlux5cux23id_2243}{\protect\hyperlink{23_NOTES.xhtmlux5cux23id_2244}{3}}}
Huizinga had begun negotiations with the French publisher Edouard
Champion of Paris, who preferred a shortened version of the book and
without the references. This project fell through, owing to
disagreements over the rights of publication of the French edition in
Holland in 1923 (letter 457), and Huizinga was left with the condensed,
but unpublished, French manuscript. (An accurate French edition was
eventually published by the firm of Payot in 1932, in a translation by
Julia Bastin {[}letter 559{]}.) In 1923, Huizinga was also negotiating
with Edward Arnold and Company about an English edition, and, owing to
the fact that Arnold had no one in their office who could read Dutch,
they reviewed it in the condensed French version. Sir Rennell Rodd, a
diplomat, poet, and historian, and Arnold's reviewer, thought the
original form of the book would sell only to scholars and preferred it
in its French form, which he thought might have a popular audience
(letter 462) and, although Huizinga protested, he did not do so very
strongly (letter 466). An abridgment on the lines of the French
manuscript was ultimately ageed upon (letters 472 and 477) and the
Hopman version, called \emph{The Waning of the Middle Ages}, is the
result.

\protect\hypertarget{05_TRANSLATOR_S_INTRODUCTION.xhtmlux5cux23page_xi}{}{}All
this was taking place while the final arrangements for the German
edition were being set. The German edition is precise in all
particulars, but the fourteen original Dutch chapters are broken up into
twenty-three, which are more even in length. This was Huizinga's own
idea, evidently incorporated in the unpublished French translation and
eventually carried forward in the English as well (letter 470).

Thus Huizinga clearly preferred a complete translation of
\emph{Herfsttij}, although he did think the chapter divisions could be
improved. His quarrel with Champion over distribution rights, however,
suggests that remuneration was an important issue, as he raised
practically no objections to the condensation ultimately produced by
Hopman for Arnold and Company. It is possible, too, given that the
prospect of a wide market for the book might have had something to do
with his thinking, that in obtaining an English edition Huizinga was
also looking forward to the American market. Huizinga wrote two books
about America, both gently
critical.\textsuperscript{\protect\hypertarget{05_TRANSLATOR_S_INTRODUCTION.xhtmlux5cux23id_2241}{\protect\hyperlink{23_NOTES.xhtmlux5cux23id_2242}{4}}}
Like his contemporary Freud, Huizinga thought American life suffered
from its lack of social forms; he considered Americans to be
materialistic and far, far too hasty in the pursuit of their affairs. He
invented a motto for America, ``This Here, and Soon,'' to characterize
this haste, which, he thought, all too often, led to superficiality.
Perhaps this perception caused him to believe that a simplified and less
allusive \emph{Autumn} might succeed best in the American market. The
fourteen uneven Dutch chapters became the twenty-three short chapters as
in the German edition, much more suitable for daily classroom
assignments and for a people with a short attention span. The work of
preparing the English translation was given to Fritz Hopman, a student
of English literature and journalist, who at one time was chairman of
the Maatschappij der Nederlandse Letterkunde (Dutch Literature Society).
He was in financial difficulties in 1924, and Huizinga was probably glad
to be able to provide him with
work.\textsuperscript{\protect\hypertarget{05_TRANSLATOR_S_INTRODUCTION.xhtmlux5cux23id_2239}{\protect\hyperlink{23_NOTES.xhtmlux5cux23id_2240}{5}}}

F. W. N. Hugenholtz's study of the history of the text, \emph{The Fame
of a
Masterwork},\textsuperscript{\protect\hypertarget{05_TRANSLATOR_S_INTRODUCTION.xhtmlux5cux23id_2237}{\protect\hyperlink{23_NOTES.xhtmlux5cux23id_2238}{6}}}
shows that the first recognition of the book's importance came, not from
Huizinga's Dutch colleagues, but in German reviews. The Dutch were
inclined to consider \emph{The Autumn of the Middle
Ages\textsuperscript{\protect\hypertarget{05_TRANSLATOR_S_INTRODUCTION.xhtmlux5cux23id_2235}{\protect\hyperlink{23_NOTES.xhtmlux5cux23id_2236}{7}}}}
far too literary for serious history and mistakenly thought its approach
to be old-fashioned rather than realizing that it was truly a
revolutionary innovation. \emph{Autumn} was Huizinga's first major work
published after he became professor of history at
\protect\hypertarget{05_TRANSLATOR_S_INTRODUCTION.xhtmlux5cux23page_xii}{}{}Leiden,
and Leiden was not at that time Holland's ``first'' university, nor was
Huizinga the most famous professor of history. Defensive, in the face of
native criticism of the work he might, indeed, have considered the
English translation a step to a further revision (the second Dutch
edition had appeared in 1921, the Hopman translation came out in 1924).
It seems to me, that much of what is left out of the Hopman version are
elements which contribute to the ``literary,'' that is to say aesthetic
character of the book and this might be a direct response to his Dutch
critics.

There is another possible reason for the truncated English version.
Probably anyone who reads \emph{Autumn} will notice that it reveals a
great deal of the private side of Huizinga himself. In it, the reader
sees not only Huizinga's opinions and strong convictions, but glimpses
his passions and, I think, his spiritual side as well. Perhaps he
realized this and the drawing back so apparent in the original English
is an instinctive reaction that he also exhibited in other
circumstances.

In his brief autobiography written at the very end of his
life\textsuperscript{\protect\hypertarget{05_TRANSLATOR_S_INTRODUCTION.xhtmlux5cux23id_2233}{\protect\hyperlink{23_NOTES.xhtmlux5cux23id_2234}{8}}}
Huizinga reveals that he consistently hid his true self even from his
colleagues and students. ``It is not false modesty when I say that,
though I have been known as an early riser since childhood, I never rose
quite as early as people believed.'' The relationship of his work to his
private self was frequently misjudged by others. Huizinga almost seems
pleased at their confusion.

Regarding my biography of Erasmus, many people have expressed the view
that here was a man after my own heart. As far as I can tell, nothing
could be farther from the truth for, much though I admire Erasmus, he
inspires me with little sympathy and, as soon as the work was done, I
did my best to put him out of my mind. I remember a conversation in
January 1932 with a German colleague who contended that \emph{Erasmus}
was much more my line of country than the \emph{Waning of the Middle
Ages} with which, he claimed, I must have struggled manfully. I thought
about the matter for a moment and then I had to smile. In fact, my
historical and literary studies never struck me as partaking of the
nature of struggle in any way, nor any of my work as a great challenge.
Indeed, the whole idea of having to overcome enormous obstacles was as
alien to me as having to compete in a
\protect\hypertarget{05_TRANSLATOR_S_INTRODUCTION.xhtmlux5cux23page_xiii}{}{}race,
as alien as the spirit of competition whose importance in cultural life
I myself have emphasized in my \emph{Homo Ludens}.

When he finds himself on the edge of a deep personal revelation,
Huizinga goes so far, and no further.

.~.~. In September 1899, I was granted two weeks' extra leave,
immediately after beginning of term, to attend the Congress of
Orientalists in Rome. I went there with J. P. Vogel, who intended to go
on to India, and with André Jolles with whom I had started a close
friendship in the autumn of 1896. This friendship was to play a large
part in my life for more than 35 years, until 9th October 1933 when it
was abruptly cut short---and not by me. I could write a whole book on my
relation with Jolles, so full is my mind of him and despite all that has
happened---my heart as well.

Huizinga's later works do not reveal the personality of the author as
much as \emph{Autumn} does. A prominent sense of the author only again
becomes apparent in his great moral essay of the thirties, \emph{In the
Shadow of
Tomorrow.\textsuperscript{\protect\hypertarget{05_TRANSLATOR_S_INTRODUCTION.xhtmlux5cux23id_2231}{\protect\hyperlink{23_NOTES.xhtmlux5cux23id_2232}{9}}}}

Given Huizinga's importance to historiography, the fact that the English
translation is a variant text has not been given enough attention. With
the single exception of Weintraub, no one, to my knowledge, has pointed
out the critical importance of that fact, even though the introduction
might have served as a warning to a professionally critical discipline.
Is it possible that English-speaking historians have been discussing
this book with their foreign colleagues without realizing that they were
reading a significantly different text? If this is so, it is a primary
justification for the present translation.

Hopman's work does have the virtue of being graceful. He did have an
excellent grasp of English vocabulary, and his rendition is sometimes
lovely, but it is not literal and sometimes something more than a
literal quality is missing. It is not proper for a translator in the
second place to judge too harshly the work of a predecessor, but a
reader deserves some indication why one translation should be preferred
over another. The most glaring changes in the Hopman from the Dutch
second edition are the many omissions of
\protect\hypertarget{05_TRANSLATOR_S_INTRODUCTION.xhtmlux5cux23page_xiv}{}{}examples
drawn from the (in most instances) medieval French sources that Huizinga
cites in the original language (although there are a few instances where
Hopman includes examples not in the Dutch edition). The present
translators felt that the original divisions of the text much more
clearly reflected the organization of Huizinga's argument in spite of
their rather uneven lengths and Huizinga's second thoughts about the
matter. Finally, the Hopman translation omits, as its introduction
points out, the documentation. These alterations are restored in this
translation.

Much more serious issues are those alterations by Hopman that tend to
distort Huizinga's meaning. Hopman is sometimes prone to pull Huizinga's
punches. For instance, one of the most significant elements in
\emph{Autumn} is its assertions about the proper use of sources, an
issue addressed several times. Here is a representative passage in this
translation:

Daily life offered unlimited range for acts of flaming passion and
childish imagination. Our medieval historians who prefer to rely as much
as possible on official documents because the chronicles are unreliable
fall thereby victim to an occasionally dangerous error. The documents
tell us little about the difference in tone that separates us from those
times; they let us forget the fervent pathos of medieval life. Of all
the passions permeating medieval life with their color, only two are
mentioned, as a rule by legal documents: greed and quarrelsomeness. Who
has not frequently wondered about the nearly incredible violence and
stubbornness with which greed, pugnacity, or vindictiveness rise to
prominence in the court documents of that period! It is only in the
general context of the passions which inflame every sphere of life that
these tensions become acceptable and intelligible to us. This is why the
authors of the chronicles, no matter how superficial they may be with
respect to the actual facts and no matter how often they may err in
reporting them, are indispensable if we want to understand that age
correctly.

And here is the same passage in Hopman:

A scientific historian of the Middle Ages, relying first and foremost on
official documents, which rarely refer to the
\protect\hypertarget{05_TRANSLATOR_S_INTRODUCTION.xhtmlux5cux23page_xv}{}{}passions,
except violence and cupidity, occasionally runs the risk of neglecting
the difference of tone between the life of the expiring Middle Ages and
that of our own days. Such documents would sometimes make us forget the
vehement pathos of medieval life, of which the chroniclers, however
defective as to material facts, always keep us in mind.

Not only has Hopman made a strong statement weak, his version misses the
nuance of just how passionate Huizinga was about the passions of the
Middle Ages.

Similar distortions frequently occur. Here is Hopman's translation of a
passage about the profane interest in such things as Mary's marital
relationship with Joseph:

This familiarity with sacred things is, on the one hand, a sign of deep
and ingenuous faith; on the other, it entails irreverence whenever
mental contact with the infinite fails. Curiosity, ingenuous though it
be, leads to profanation.

Here is this translation:

This fatuous familiarity with God in daily life has to be seen in two
ways. On the one hand it testifies to the absolute stability and
immediacy of faith, but where this familiarity becomes habitual, it
increases the danger that the godless (who are always with us), but also
the pious, in moments of insufficient religious tension, continuously
profane faith more or less consciously and intentionally.

For the student interested in historiography itself, perhaps the
omissions of theoretical statements are the most serious. In the famous
discussion of the three routes to the beautiful life in the second
chapter, Hopman omits this statement of serious interest to anyone
concerned with Huizinga's definitions of culture and civilization and
with the movement of his thinking towards the theoretical statement of
\emph{Homo
Ludens},\textsuperscript{\protect\hypertarget{05_TRANSLATOR_S_INTRODUCTION.xhtmlux5cux23id_2229}{\protect\hyperlink{23_NOTES.xhtmlux5cux23id_2230}{10}}}
which defines the role of play in culture.

The great divide in the perception of the beauty of life comes much more
between the Renaissance and the Modern Period than between the Middle
Ages and the Renaissance. The turnabout occurs at the point where art
and life begin to diverge. It is the point where art begins to be no
longer
\protect\hypertarget{05_TRANSLATOR_S_INTRODUCTION.xhtmlux5cux23page_xvi}{}{}in
the midst of life, as a noble part of the joy of life itself, but
outside of life as something to be highly venerated, as something to
turn to in moments of edification or rest. The old dualism separating
God and world has thus returned in another form, that of the separation
of art and life. Now a line has been drawn right through the enjoyments
offered by life. Henceforth they are separated into two halves---one
lower, one higher. For medieval man they were all sinful without
exception; now they are all considered permissible, but their ethical
evaluation differs according to their greater or lesser degree of
spirituality.

The things which can make life enjoyable remain the same. They are, now
as before, reading, music, fine arts, travel, the enjoyment of nature,
sports, fashion, social vanity (knightly orders, honorary offices,
gatherings) and the intoxication of the senses. For the majority, the
border between the higher and lower levels seems now to be located
between the enjoyment of nature and sports. But this border is not firm.
Most likely sport will sooner or later again be counted among the higher
enjoyments---at least insofar as it is the art of physical strength and
courage. For medieval man the border lay, in the best of cases, right
after reading; the enjoyment of reading could only be sanctified through
striving for virtue or wisdom. For music and the fine arts, it was their
service to faith alone which was recognized as being good. Enjoyment
\emph{per se} was sinful. The Renaissance had managed to free itself
from the rejection of all the joy of life as something sinful, but had
not yet found a new way of separating the higher and lower enjoyments of
life; the Renaissance wanted an unencumbered enjoyment of all of life.
The new distinction is the result of the compromise between the
Renaissance and Puritanism that is at the base of modern spiritual
attitudes. It amounted to a mutual capitulation in which the one side
insisted on saving beauty while the other insisted on the condemnation
of sin. Strict Puritanism, just as did the Middle Ages, still condemned
as basically sinful and worldly the entire sphere of the beautification
of life with an exception being made in cases where such efforts assumed
expressly religious forms and sanctified themselves through their use in
the service of faith. Only
\protect\hypertarget{05_TRANSLATOR_S_INTRODUCTION.xhtmlux5cux23page_xvii}{}{}after
the Puritan worldview lost its intensity did the Renaissance
receptiveness to all the joys of life gain ground again; perhaps even
more ground than before because, beginning with the eighteenth century
there is a tendency to regard the natural \emph{per se} an element of
the ethically good. Anyone attempting to draw the dividing line between
the higher and lower enjoyment of life according to the dictates of
ethical consciousness would no longer separate art from sensuous
enjoyment, the enjoyment of nature from the cult of the body, the
elevated from the natural, but would only separate egotism, lies, and
vanity from purity.

There are many such issues to which we could point, not in the spirit of
demeaning a translation that has served Huizinga well, but in the sense
that having done its work and brought the importance of the mind of
Huizinga to the attention of the English-speaking world, it is now
obsolete and a more critical and deeper look at Huizinga requires access
to a version of the work closer to that known by the rest of the world.

This translation was made from the second Dutch edition of 1921. Seen
from the vantage point of the second edition, the first has a tentative
character that Huizinga eliminated in his revision. Huizinga made
further minor revisions in later editions, but the second represents his
thinking at its most seminal stage. We compared our work carefully with
the German translation of 1924, which, Huizinga notes, follows the
second Dutch edition exactly. We have included not only the preface to
the Dutch edition, but also the preface that Huizinga wrote for the
German translation, for the insight it gives into the title and its
comment on the question of translation itself. We have restored the
documentation and added a few translators' notes to clarify Huizinga's
references to things that might be common knowledge or self-evident to a
Dutch reader but not necessarily so to others. This version also
includes translations of the citations that Huizinga makes in the
original languages. Such translations have become customary in later
editions, although they do not appear in the Dutch original we followed.
Our translations follow Hopman, but we have made several alterations
according to our own judgment.

*~~~*~~~*

\protect\hypertarget{05_TRANSLATOR_S_INTRODUCTION.xhtmlux5cux23page_xviii}{}{}Ulrich
Mammitzsch, my colleague and co-translator, was a noted specialist in
Buddhist art and literature, but his formidable erudition extended to
great works of all cultures and he was as pleased to discuss Schiller as
he was his beloved mandalas. He felt a special affinity for Huizinga,
who began his academic life as a student of Eastern culture, and who had
a love of literature much like Ulrich's. Mostly, however, Ulrich's
dedication to Huizinga was because they were alike in their
high-mindedness. As Ulrich Mammitzsch fled the East Zone, not because of
political theory, but because he found the Communists to be unethical,
so Johan Huizinga was brought to denounce the Nazis from the first
principles of civilized behavior. The two minds spoke to one another
directly and I will never forget Ulrich's excitement as we read
Huizinga's description of the tension in the life of medieval common
people, strung between the church and the nobility---a tension which,
Ulrich exclaimed, he had seen the last of as a child in rural Germany
before the war. He read the book from the inside, so to speak, and I
would like to attribute whatever virtues this translation has to his
insightful sensitivity.

\emph{RODNEY J. PAYTON}
