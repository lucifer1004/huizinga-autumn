\chapter{THE COMING OF THE NEW FORM}

THE RELATIONSHIP BETWEEN RISING HUMANISM AND the dying spirit of the
Middle Ages is much more complicated than we are inclined to imagine. To
us, who see the two cultural complexes very sharply separated, it
appears as if the receptiveness to the eternal youth of antiquity and
the denial of the entire worn-out apparatus of the medieval expression
of thought had come, like a sudden revelation, to everyone at once. As
if the spirit, mortally tired of allegory and the flamboyant style, had
suddenly understood: not this, but that! As if the golden harmony of
classical antiquity had suddenly stood before their eyes like a
long-awaited liberation, and as if they had embraced antiquity with the
joy of someone who had finally found his salvation.

But this was not the case. In the middle of the garden of medieval
thought, between the luxuriantly growing old seeds, classicism grew
gradually. At the beginning it is only a formal element of the
imagination. Only later does it become a great new inspiration of the
soul. And even then, the spirit and forms of expression that we are
accustomed to regard as the old, medieval ones do not die on the vine.

In order to recognize this more clearly, it would be useful to observe
the approach of the Renaissance in greater detail than can be done here.
This scrutiny should focus, not on Italy but on France, the country that
had provided the most fertile soil for everything that comprised the
splendid wealth of genuinely medieval culture. Viewing the Italian
Quattrocento in its glorious contrast to late medieval life anywhere
else, we gain an overall impression of balance, gaiety, and freedom,
pure and sonorous. Taken together, these qualities are regarded to be
the Renaissance, and perhaps taken for the signature of the new spirit.
In the meantime, thanks to the unavoidable one-sidedness without which
no historical judgment can be reached, it is forgotten that in the Italy
of the
\protect\hypertarget{22_Chapter_Fourteen__THE_COMING_OF.xhtmlux5cux23page_383}{}{}Quattrocento,
too, the firm foundation of cultural life still remained genuinely
medieval, that in the minds of the Renaissance itself medieval features
are much more deeply impressed than is generally realized. But in the
general perception it is the tone of the Renaissance that dominates.

However, a general look at the French-Burgundian world of the fifteenth
century gives the primary impression of a fundamentally somber mood, a
barbarian splendor, bizarre and overloaded forms, an imagination that
had become threadbare---all the signs of the medieval spirit in its last
gasps. In this instance it is easily forgotten that, here too, the
Renaissance was approaching from all sides; but here it had not yet
become dominant, and had not yet transformed the underlying groundtone.

The remarkable thing in all this is that the new arrives as form before
it really becomes a new spirit.

The new classical forms arise in the middle of the old notions and
relationships of life. Humanism got its start by nothing more dramatic
than that a learned circle took more care than usual to observe a pure
Latin and classical sentence structure. Such a circle flourished around
1400 in France; it was comprised of a few clerics and magistrates: Jean
de Montreuil, canon of Lille and royal secretary; Nicolas de Clémanges,
the famous literary leader; the reform-minded cleric Gontier Col;
Ambrosius Miliis, princely private secretary (as was the first
named).\textsuperscript{\protect\hypertarget{22_Chapter_Fourteen__THE_COMING_OF.xhtmlux5cux23id_122}{\protect\hyperlink{23_NOTES.xhtmlux5cux23id_123}{1}}}
They write beautiful and proud humanist letters to one another, which
are in no way inferior to later products of the genre, neither in the
hollow generality of the thought, in the deliberate importance, in the
forced structure of the sentences and the unclear expression, nor in the
delight in learned play. Jean de Montreuil gets excited over the
question whether ``orreolum'' and ``schedula'' were to be written with
or without an ``h'' and over the use of ``k'' in Latin words. ``If you
do not come to my assistance, worthy teacher and brother,'' he writes to
Clémanges,\textsuperscript{\protect\hypertarget{22_Chapter_Fourteen__THE_COMING_OF.xhtmlux5cux23id_120}{\protect\hyperlink{23_NOTES.xhtmlux5cux23id_121}{2}}}
``I will lose my good name and deserve death now that I have noticed
that in my last letter to my Lord and Father, the Bishop of Cambray,
overly hasty and casual as the pen is wont to be, I put in the place of
the comparative case `proprior,' the word `proximore.' Do correct it or
our critics will write denunciatory tracts about
it.''\textsuperscript{\protect\hypertarget{22_Chapter_Fourteen__THE_COMING_OF.xhtmlux5cux23id_118}{\protect\hyperlink{23_NOTES.xhtmlux5cux23id_119}{3}}}
It is clear that these letters are designed to be learned literary
exercises for the public. Also genuinely humanistic
\protect\hypertarget{22_Chapter_Fourteen__THE_COMING_OF.xhtmlux5cux23page_384}{}{}are
his attacks on his friend Ambrosius, who had charged Cicero with
self-contradiction and who had ranked Ovid above
Virgil.\textsuperscript{\protect\hypertarget{22_Chapter_Fourteen__THE_COMING_OF.xhtmlux5cux23id_116}{\protect\hyperlink{23_NOTES.xhtmlux5cux23id_117}{4}}}

In one of his letters Montreuil provides a leisurely description of the
cloister of Charlieu near Senlis. It is very noticeable how he suddenly
becomes more readable as soon as he simply narrates, in medieval style,
what can be seen there. How the sparrows in the refectory share the
meal, so that one may entertain doubts whether the king had instituted
the benefice for the monks or for the birds; how a little wren acts as
if he were the abbot, how the donkey of the gardener asks the letter
writer to keep him in mind in his epistle. All this sounds fresh and
attractive, but not specifically
humanist.\textsuperscript{\protect\hypertarget{22_Chapter_Fourteen__THE_COMING_OF.xhtmlux5cux23id_114}{\protect\hyperlink{23_NOTES.xhtmlux5cux23id_115}{5}}}
But let us not forget that we had already met the same Jean de Montreuil
and Gontier Col as passionate defenders of the \emph{Roman de la rose}
and as members of the \emph{Cours d'amours} of 1401. Does this not
demonstrate how external an element of life this early humanism was? It
is only a reinforced effect of medieval school erudition and is little
different from the revival of the classical Latin tradition that can be
observed in Alcuin and his fellow spirits during the time of Charlemagne
and later in the French schools of the twelfth century.

Though this first French humanism spends its force in the small circle
of men who had nourished it without finding a direct successor, it is
nonetheless already linked to the large international intellectual
movement. Jean de Montreuil and his kindred spirits already see in
Petrarch their shining model. They repeatedly mention Coluccio Salutati,
the Florentine Chancellor who in the middle of the fourteenth century
had introduced the new Latin rhetoric into the language of state
documents.\textsuperscript{\protect\hypertarget{22_Chapter_Fourteen__THE_COMING_OF.xhtmlux5cux23id_112}{\protect\hyperlink{23_NOTES.xhtmlux5cux23id_113}{6}}}
However, in France, Petrarch, if we may say so, is still embraced with
the medieval spirit. He had been a personal friend of some of the
leading minds of an earlier generation, the poet Philippe de Vitri; the
philosopher and politician Nicolas Oresme, who had educated the dauphin
(Charles V). Philippe de Mézières also seems to have known Petrarch.
These men are no humanists even though the ideas of Oresme contained
much that is new. If it is true, as Paulin
Paris\textsuperscript{\protect\hypertarget{22_Chapter_Fourteen__THE_COMING_OF.xhtmlux5cux23id_110}{\protect\hyperlink{23_NOTES.xhtmlux5cux23id_111}{7}}}
assumed, that Machaut's Peronne d'Armentières was influenced in her
desire for a poetic conduct of courtship not only by the example of
Heloise, but also by Laura, \emph{Le voir-dit} would constitute a
remarkable testimony to the fact that a work in which we are prone to
sense above all the arrival of modern ideas could, nonetheless, inspire
a medieval work.

\protect\hypertarget{22_Chapter_Fourteen__THE_COMING_OF.xhtmlux5cux23page_385}{}{}But
are we not, as a rule, already predisposed to see Petrarch and Boccaccio
too exclusively from their modern side? We regard them as the first
innovators and do so with justification. But it would be wrong to assume
that, being the first humanists, they would actually no longer properly
fit into the fourteenth century. Their entire work, no matter how much
of a new breath may permeate it, stands on the culture of their age.
Beyond that, it should be noted that Petrarch and Boccaccio were known
outside of Italy during the waning Middle Ages not because of their
vernacular writings, which were to make them immortal, but through their
Latin works. Petrarch was, to his contemporaries, primarily an Erasmus
\emph{avant la lettre}, a many-talented and tasteful author of treatises
about ethics and life, a great writer of letters, the romanticist of
antiquity with his ``Liber de viris illustribus'' and ``Rerum memorandum
libri IV.'' The subjects he dealt with, \emph{De contemptu mundiy De
otio religiosorum, De vita solitaria}, are completely in the tradition
of medieval thought. His glorification of the heroes of antiquity is
much closer to the veneration of the \emph{neuf
preux}\textsuperscript{\protect\hypertarget{22_Chapter_Fourteen__THE_COMING_OF.xhtmlux5cux23id_108}{\protect\hyperlink{23_NOTES.xhtmlux5cux23id_109}{8}}}
than one might assume. It is far from peculiar that there were contacts
between Petrarch and Geert Groote or that Jean de Varennes, the fanatic
of
Saint-Lié,\textsuperscript{\protect\hypertarget{22_Chapter_Fourteen__THE_COMING_OF.xhtmlux5cux23id_106}{\protect\hyperlink{23_NOTES.xhtmlux5cux23id_107}{9}}}
invokes Petrarch's authority to defend himself against suspicions of
heresy\textsuperscript{\protect\hypertarget{22_Chapter_Fourteen__THE_COMING_OF.xhtmlux5cux23id_104}{\protect\hyperlink{23_NOTES.xhtmlux5cux23id_105}{10}}}
and that he borrows from Petrarch the text for a new prayer: ``tota
caeca christianitas.'' How much Petrarch meant to his century is
expressed by Jean de Montreuil in the following words: ``devotissimus,
catholicus ac celeberrimus philosophus
moralis.''\textsuperscript{\protect\hypertarget{22_Chapter_Fourteen__THE_COMING_OF.xhtmlux5cux23id_102}{\protect\hyperlink{23_NOTES.xhtmlux5cux23id_103}{11}}}
Denis the Carthusian was still in a position to borrow from Petrarch a
lament over that truly medieval idea, the loss of the Holy Sepulchre;
``but since the style of Franciscus is rhetorical and difficult, I will
cite the meaning of his words rather than their
form.''\textsuperscript{\protect\hypertarget{22_Chapter_Fourteen__THE_COMING_OF.xhtmlux5cux23id_100}{\protect\hyperlink{23_NOTES.xhtmlux5cux23id_101}{12}}}

Petrarch had given particular impetus to those classic literary
expressions of the first French humanists by his derisive statement that
there were no rhetoricians and poets outside Italy. The \emph{bels
esprits} in France would not take this lying down. Nicolas de Clémanges
and Jean de Montreuil raise a lively protest against such a
claim.\textsuperscript{\protect\hypertarget{22_Chapter_Fourteen__THE_COMING_OF.xhtmlux5cux23id_98}{\protect\hyperlink{23_NOTES.xhtmlux5cux23id_99}{13}}}

Boccaccio's influence was similar to that of Petrarch albeit in a more
limited field. He was not venerated as the author of the
\emph{Decameron} but as ``le docteur de patience en adversité,'' the
author of \emph{Libri de casibus virorum illustrium} and \emph{De claris
mulieribus}. Boccaccio
\protect\hypertarget{22_Chapter_Fourteen__THE_COMING_OF.xhtmlux5cux23page_386}{}{}had
staked out for himself the role of a kind of impresario of Fortuna with
these strange collections of works about the vagrancies of human fate.
Chastellain sees him in this
light.\textsuperscript{\protect\hypertarget{22_Chapter_Fourteen__THE_COMING_OF.xhtmlux5cux23id_96}{\protect\hyperlink{23_NOTES.xhtmlux5cux23id_97}{14}}}
He entitles a rather bizarre treatise about all sorts of tragic
individual fates of his time \emph{Le Temple de Bocace}, in which the
spirit of the ``noble historien'' is evoked to console Margaret of
England over her misfortune. The claim that Boccaccio was insufficiently
or mistakenly seen by the still too medieval Burgundians is without
merit. They grasped his strongly medieval side, which we run the risk of
forgetting entirely.

It is not so much a difference in effort or mood that distinguishes
rising humanism in France from that in Italy, but rather a nuance in
taste and erudition. The imitation of antiquity does not come as easily
to the French as it does to those born under the skies of Tuscany or in
the shadow of the Colosseum even though the learned authors early
acquired a facile command of the classical-Latin style of letters. The
secular authors are, however, still inexperienced in the fine points of
mythology and history. Machaut, who was no scholar and has to be
regarded as secular poet in spite of all his spiritual dignity,
hopelessly confuses the names of the seven sages. Chastellain mistakes
Pelleus with Pelias; La Marche, Proteus with Pirithous. The poet of the
``Pastoralet'' speaks of ``le bon roy scypion
d'afrique.''\protect\hypertarget{22_Chapter_Fourteen__THE_COMING_OF.xhtmlux5cux23id_2803}{\protect\hyperlink{23_NOTES.xhtmlux5cux23id_2804}{*\textsuperscript{1}}}
The authors of \emph{Le Jouvencel} derive ``politique'' from πολυç and
an allegedly Greek ``icos, gardien,'' ``qui est à dire gardien de
pluralité.''\textsuperscript{\protect\hypertarget{22_Chapter_Fourteen__THE_COMING_OF.xhtmlux5cux23id_94}{\protect\hyperlink{23_NOTES.xhtmlux5cux23id_95}{15}}}\protect\hypertarget{22_Chapter_Fourteen__THE_COMING_OF.xhtmlux5cux23id_2801}{\protect\hyperlink{23_NOTES.xhtmlux5cux23id_2802}{†\textsuperscript{2}}}

Nonetheless, the classical vision does, time and again, break through
their medieval allegorical form. A poet such as that of the ragged
pastorale ``Le pastoralet'' suddenly bestows a hint of the splendor of
the Quattrocento on his description of the god Silvanus and in a prayer
to Pan only to return abruptly to the well traveled paths of the old
ways.\textsuperscript{\protect\hypertarget{22_Chapter_Fourteen__THE_COMING_OF.xhtmlux5cux23id_92}{\protect\hyperlink{23_NOTES.xhtmlux5cux23id_93}{16}}}
Just as Van Eyck sometimes presents the forms of classical architecture
within his purely medieval compositions, authors attempt to incorporate
still purely formal and ornamental classical features. The chroniclers
test their strength with speeches on matters of state and war,
\emph{contiones}, in the style of Livy, or they mention wondrous signs,
\emph{prodigia}, because Livy did the
same.\textsuperscript{\protect\hypertarget{22_Chapter_Fourteen__THE_COMING_OF.xhtmlux5cux23id_90}{\protect\hyperlink{23_NOTES.xhtmlux5cux23id_91}{17}}}
The clumsier this application of classical forms, the more
\protect\hypertarget{22_Chapter_Fourteen__THE_COMING_OF.xhtmlux5cux23page_387}{}{}instructive
the material for studying the transition from the Middle Ages to the
Renaissance. The bishop of Châlons, Jean Germain, attempts to describe
the peace conference at Arras in 1435 in the forceful marked style of
the Romans. His intent is to attain a Livian effect by using short
sentences and descriptions of vivid clarity. But the result is a
caricature of ancient prose, just as bloated as it is naive, drawn like
the little figures of a calendar page from a breviary, but failing in
style.\textsuperscript{\protect\hypertarget{22_Chapter_Fourteen__THE_COMING_OF.xhtmlux5cux23id_88}{\protect\hyperlink{23_NOTES.xhtmlux5cux23id_89}{18}}}
Antiquity is still seen in extraordinarily alien terms. On the occasion
of the funeral services for Charles the Bold in Nancy, the young duke of
Lorraine, who defeated him, appears dressed in mourning, ``à
l'antique,'' to pay his last respects to the body of his enemy; that is,
he wears a long golden beard stretching down to his belt. In this way he
represents one of the nine \emph{preux} and celebrates his own triumph.
In this masquerade he prays for a quarter of an
hour.\textsuperscript{\protect\hypertarget{22_Chapter_Fourteen__THE_COMING_OF.xhtmlux5cux23id_86}{\protect\hyperlink{23_NOTES.xhtmlux5cux23id_87}{19}}}

In the minds of the French around 1400, the terms \emph{rhétorique,
orateur, poésie} were congruous with the idea of antiquity. To them they
meant the enviable perfection of antiquity, above all a form artfully
elaborated. All these poets of the fifteenth century (a few even
earlier), whenever they follow their feelings and actually have
something to say, compose flowing, simple, frequently powerful, and
occasionally tender poems. But if they intend these poems to be
particularly beautiful, they draw on mythology, use pretentious
Latinizing expressions and call themselves ``Rhetoricien.'' Christine de
Pisan expressedly distinguishes a mythological poem from her usual work
as ``balade
pouétique.''\textsuperscript{\protect\hypertarget{22_Chapter_Fourteen__THE_COMING_OF.xhtmlux5cux23id_84}{\protect\hyperlink{23_NOTES.xhtmlux5cux23id_85}{20}}}
When Eustache Deschamps sends his works to his fellow artist and admirer
Chaucer, he resorts to the most unpalatable, quasi-classical mishmash:

O \emph{Socrates plains de philosophie},

\emph{Seneque en meurs et Anglux en pratique},

\emph{Ovides grans en ta poeterie},

\emph{Bries en parler, saiges en rethorique}

\emph{Aigles tres haulz, qui par ta théorique}

\emph{Enlumines le regne d'Eneas}.

\emph{L'Isle aux Geans, ceuls de Bruth, et qui as}

\emph{Semé les fleurs et panté le rosier},

\emph{Aux ignorans de la langue pandras},

\emph{Grant translateur, noble Geffroy Chaucier!}

. \emph{.~.~. . .~.~. . .~.~. . .~.~. . .~.~. . .~.~}.

\emph{\protect\hypertarget{22_Chapter_Fourteen__THE_COMING_OF.xhtmlux5cux23page_388}{}{}A
toy pour ce de la fontaine Helye}

\emph{Requier avoir un buvraige autentique},

\emph{Dont la doys est du tout en ta baillie}.

\emph{Pour rafrener d'elle ma soif ethique},

\emph{Qui en Gaule seray paralitique}

\emph{Jusques a ce que tu
m'abuveras}.\textsuperscript{\protect\hypertarget{22_Chapter_Fourteen__THE_COMING_OF.xhtmlux5cux23id_82}{\protect\hyperlink{23_NOTES.xhtmlux5cux23id_83}{21}}}\emph{\protect\hypertarget{22_Chapter_Fourteen__THE_COMING_OF.xhtmlux5cux23id_2799}{\protect\hyperlink{23_NOTES.xhtmlux5cux23id_2800}{*\textsuperscript{3}}}}

Here we have the beginning of that \emph{manière} that was soon to
evolve into the ridiculous Latinization of the noble French language on
which Villon and Rabelais were to heap their
scorn.\textsuperscript{\protect\hypertarget{22_Chapter_Fourteen__THE_COMING_OF.xhtmlux5cux23id_80}{\protect\hyperlink{23_NOTES.xhtmlux5cux23id_81}{22}}}
This style is found over and over in poetic correspondence, in
dedications and speeches, in short, whenever something is expected to be
particularly beautiful. Chastellain speaks of ``Vostre très-humble et
obéissante serve et ancelle, la ville de Gand,'' ``la viscéral intime
douleur et
tribulation'';\protect\hypertarget{22_Chapter_Fourteen__THE_COMING_OF.xhtmlux5cux23id_2797}{\protect\hyperlink{23_NOTES.xhtmlux5cux23id_2798}{†\textsuperscript{4}}}
La Marche of ``nostre francigéne locution et langue
vernacule'';\protect\hypertarget{22_Chapter_Fourteen__THE_COMING_OF.xhtmlux5cux23id_2795}{\protect\hyperlink{23_NOTES.xhtmlux5cux23id_2796}{‡\textsuperscript{5}}}
Molinet of ``abreuvé de la doulce et melliflue liqueur procedant de la
fontaine caballine,'' ``ce vertueux duc scipionique,'' ``gens de
muliébre
courage.''\textsuperscript{\protect\hypertarget{22_Chapter_Fourteen__THE_COMING_OF.xhtmlux5cux23id_78}{\protect\hyperlink{23_NOTES.xhtmlux5cux23id_79}{23}}}\protect\hypertarget{22_Chapter_Fourteen__THE_COMING_OF.xhtmlux5cux23id_2793}{\protect\hyperlink{23_NOTES.xhtmlux5cux23id_2794}{§\textsuperscript{6}}}

These ideals of a refined \emph{rhetorique} are not only the ideals of a
pure literary expression, they are at the same time the ideals of the
highest literary communication. All of humanism, just as the poetry of
the troubadours had been, is a social game, a kind of conversation, a
striving for a higher form of life. Even the conversation of the learned
men of the sixteenth and seventeenth centuries by no means denies this
fact. In this respect, France occupies the middle ground between Italy
and the Netherlands. In Italy, where
lan\protect\hypertarget{22_Chapter_Fourteen__THE_COMING_OF.xhtmlux5cux23page_389}{}{}guage
and thought still had the least distance between themselves and genuine,
authentic antiquity, humanistic forms could readily be accepted into the
highest life of the people. The Italian language can hardly be said to
have been violated by a somewhat stronger Latinized form of expression.
The humanist club spirit readily matched the customs of society. The
Italian humanists represented the gradual development of Italian folk
culture and, because of that, the first type of modern man. In the
Burgundian regions, however, the spirit and form of society were still
so medieval that the effort towards a renewed and purified expression
could initially be embodied in perfectly old-fashioned form in the
``Chambers of Rhetoricians.'' As cooperatives they are but a
continuation of the medieval brotherhood, and the spirit they emanate
is, for the time being, new only with respect to the entirely external
formal aspect. Modern culture is first inaugurated in them by the
biblical humanism of Erasmus.

France, with the exception of the northern provinces, does not know the
old-fashioned apparatus of the ``Chambers of Rhetoricians,'' but its
\emph{noble rhetoriciens} do not yet resemble the Italian humanists
either. They, too, still retain much of the spirit and form of the
medieval age. It may be claimed without exaggeration, that the French
authors and poets of the fifteenth century who best manage to steer
clear of classicism are closer to the modern development of literature
than those who pay homage to Latinism and oratorical form. The modern
authors, such as Villon, Coquillart, Henri Baude, as well as Charles
d'Orléans and the author of ``L'amant rendu cordelier,'' are those whose
minds are unencumbered by all this, even if still dressed in medieval
form. In poetry and prose, at least, the classicistic aspiration proves
to have a retarding influence. The pompous spokesmen of the heavily
draped Burgundian ideal, such as Chastellain, La Marche, Molinet, are
the old-fashioned minds of French literature. But even they, once they
manage to free themselves here and there from their artfully embellished
ideal and compose or write as it comes from the heart and goes straight
to the heart, without much ado, they become readable and, at the same
time, appear to be more modern.

A second-rate poet, Jean Robertet (1420--90), secretary to three dukes
of Bourbon and three French kings, regarded Georges Chastellain, the
Flemish Burgundian, as the peak of the most noble poetic art. This
admiration gave rise to a literary correspondence
\protect\hypertarget{22_Chapter_Fourteen__THE_COMING_OF.xhtmlux5cux23page_390}{}{}that
is offered here as an illustration of the above comments. To make the
acquaintance of Chastellain, Robertet relied on the services of a
certain Montferrant who lived in Bruges as governor of a young Bourbon
who had been raised at the court of his uncle the duke of Burgundy. He
sent to Montferrant, in addition to a bombastic hymn of praise for the
aging court chronicler and poet, two letters addressed to Chastellain:
one in French and one in Latin. When Chastellain did not immediately
accept the proposal for a literary correspondence, Montferrant fashioned
a laborious encouragement according to the old recipe: \emph{les Douze
Dames de Rhétorique} had appeared to him. They were called
\emph{Science, Eloquence, Gravité de Sens, Profondité}, etc. Chastellain
succumbed to this temptation. The letters of the three are arranged
around the \emph{Douze Dames de
Rhétorique.\textsuperscript{\protect\hypertarget{22_Chapter_Fourteen__THE_COMING_OF.xhtmlux5cux23id_76}{\protect\hyperlink{23_NOTES.xhtmlux5cux23id_77}{24}}}}
It was, however, not long before Chastellain had his fill and
discontinued the exchange of letters.

Robertet uses quasi-modern Latinism in its silliest form. He describes a
cold: ``J'ay esté en aucun temps en la case nostre en repos, durant une
partie de la brumale
froidure.''\textsuperscript{\protect\hypertarget{22_Chapter_Fourteen__THE_COMING_OF.xhtmlux5cux23id_74}{\protect\hyperlink{23_NOTES.xhtmlux5cux23id_75}{25}}}\protect\hypertarget{22_Chapter_Fourteen__THE_COMING_OF.xhtmlux5cux23id_2791}{\protect\hyperlink{23_NOTES.xhtmlux5cux23id_2792}{*\textsuperscript{7}}}
The hyperbolic expressions in which Robertet couches his admiration are
just as simpleminded. After he had finally received his poetic letter
from Chastellain (which, in fact, was much better than his own poetry),
he wrote to Montferrant:

\emph{Frappé en l'oeil d'une clarté terrible}

\emph{Attaint au coeur d'éloquence incrédible},

\emph{A humain sens difficile à produire},

\emph{Tout offusquié de lumière incendible}

\emph{Outre percant de ray presqu'impossible}

\emph{Sur obscur corps qui jamais ne peut luire},

\emph{Ravi, abstrait me trouve en mon déduire},

\emph{En extase corps gisant à la terre},

\emph{Foible esperit perplex à voye enquerre}

\emph{Pour trouver lieu et oportune yssue}

\emph{Du pas estroit où je suis mis en serre},

\emph{Pris à la rets qu'amour vraye a
tissue}.\protect\hypertarget{22_Chapter_Fourteen__THE_COMING_OF.xhtmlux5cux23id_2789}{\protect\hyperlink{23_NOTES.xhtmlux5cux23id_2790}{†\textsuperscript{8}}}

\protect\hypertarget{22_Chapter_Fourteen__THE_COMING_OF.xhtmlux5cux23page_391}{}{}And
continues in prose: ``Où est l'oeil capable de tel objet visible,
l'oreille pour ouyr le haut son argentin et tintinabule
d'or?''\protect\hypertarget{22_Chapter_Fourteen__THE_COMING_OF.xhtmlux5cux23id_2377}{\protect\hyperlink{23_NOTES.xhtmlux5cux23id_2378}{*\textsuperscript{9}}}
And what, he asks of Montferrant, ``amy des dieux immortels et chéri des
hommes, haut pis Ulixien, plein de melliflue faconde,'' about that?
``N'est-ce resplendeur équale au curre
Phoebus?''\protect\hypertarget{22_Chapter_Fourteen__THE_COMING_OF.xhtmlux5cux23id_2380}{\protect\hyperlink{23_NOTES.xhtmlux5cux23id_2379}{†\textsuperscript{10}}}
``Is it not more than Orpheus's lyre, la tube d'Amphion, la Mercuriale
fleute qui endormyt
Argus?''\protect\hypertarget{22_Chapter_Fourteen__THE_COMING_OF.xhtmlux5cux23id_2381}{\protect\hyperlink{23_NOTES.xhtmlux5cux23id_2382}{‡\textsuperscript{11}}}
etc.\textsuperscript{\protect\hypertarget{22_Chapter_Fourteen__THE_COMING_OF.xhtmlux5cux23id_72}{\protect\hyperlink{23_NOTES.xhtmlux5cux23id_73}{26}}}

With this extreme display of bloated authorial humility, these three
poets, adhering faithfully to the medieval prescription, keep step. But
not these three alone; all their contemporaries still honor this form.
La Marche hopes that there may be some use for his memoirs as modest
flowers in a wreath, and compares his work to the ruminations of a stag.
Molinet invites all ``orateurs'' to trim all that is superfluous from
his work. Even Commines expresses the hope that the archbishop of Vienna
to whom he sends his efforts may perhaps have occasion to include them
in a Latin
work.\textsuperscript{\protect\hypertarget{22_Chapter_Fourteen__THE_COMING_OF.xhtmlux5cux23id_70}{\protect\hyperlink{23_NOTES.xhtmlux5cux23id_71}{27}}}

The poetic correspondence between Robertet, Chastellain, and Montferrat
demonstrates that the golden luster of the new classicism is only
attached to a picture that is really medieval. And this Robertet, we
should remind ourselves, has spent some time in Italy, ``en Ytalie, sur
qui les respections du ciel influent aorné parler, et vers qui tyrent
toutes douceurs élémentaires pour la fondre
harmonie.''\textsuperscript{\protect\hypertarget{22_Chapter_Fourteen__THE_COMING_OF.xhtmlux5cux23id_68}{\protect\hyperlink{23_NOTES.xhtmlux5cux23id_69}{28}}}\protect\hypertarget{22_Chapter_Fourteen__THE_COMING_OF.xhtmlux5cux23id_2383}{\protect\hyperlink{23_NOTES.xhtmlux5cux23id_2384}{§\textsuperscript{12}}}
But he apparently did not bring home much of the harmony of the
Quattrocento. To his mind, the excellence of Italy consisted only in the
``aorné parler,'' in the purely external cultivation of an artificial
style.

The only thing that renders this impression of delicately embellished
antiquating dubious for a moment is a touch of irony that occasionally
surfaces unmistakably amid the affected outpourings
\protect\hypertarget{22_Chapter_Fourteen__THE_COMING_OF.xhtmlux5cux23page_392}{}{}of
the heart. Your Robertet, the ladies of rhetoric tell
Montferrant,\textsuperscript{\protect\hypertarget{22_Chapter_Fourteen__THE_COMING_OF.xhtmlux5cux23id_66}{\protect\hyperlink{23_NOTES.xhtmlux5cux23id_67}{29}}}---``il
est exemple de Tullian art, et forme de subtilité Térencienne .~.~. qui
succié a de nos seins notre plus intériore substance par faveur; qui,
outre la grâce donnée en propre terroir, se est allé rendre en pays
gourmant pour réfection nouvelle (that is, Italian), là ou enfans
parlent en aubes à leurs mères, frians d'escole en doctrine sur
permission de
eage.''\protect\hypertarget{22_Chapter_Fourteen__THE_COMING_OF.xhtmlux5cux23id_2787}{\protect\hyperlink{23_NOTES.xhtmlux5cux23id_2788}{*\textsuperscript{13}}}
Chastellain terminated the correspondence because he was tired of it;
the gate had long been open for \emph{Dame Vanité}, he was now locking
it. ``Robertet m'a surfondu de sa nuée, et dont les perles, qui en celle
se congréent comme grésil, me font resplendir mes vestements; mais qu'en
est mieux au corps obscur dessoubs, lorsque ma robe decoit les
voyans?''\protect\hypertarget{22_Chapter_Fourteen__THE_COMING_OF.xhtmlux5cux23id_2785}{\protect\hyperlink{23_NOTES.xhtmlux5cux23id_2786}{†\textsuperscript{14}}}
If Robertet were to continue in this manner, Chastellain would throw
them into the fire unread. In the event that he wants to speak
unaffectedly, as is proper among friends, George would not withdraw his
affection.

The fact that under the classical gown still dwells a medieval mind is
less obvious in cases where the humanist uses only Latin. In those cases
the imperfect notion of the spirit of antiquity does not betray itself
by a clumsy treatment; then the scholar can imitate without encumbrance
and imitate quite effectively. A humanist such as Robert Gaguin
(1433--1501) appears to us in his letters and speeches to be already
almost as modern as Erasmus, who owed his early fame to him; it was
Gaguin who published in his compendium of French history the first
academic work of history in France
(1495),\textsuperscript{\protect\hypertarget{22_Chapter_Fourteen__THE_COMING_OF.xhtmlux5cux23id_64}{\protect\hyperlink{23_NOTES.xhtmlux5cux23id_65}{30}}}
a letter by Erasmus who thus came to see some of his writing in print
for the first time. Even if Gaguin knew as little Greek as
Petrarch,\textsuperscript{\protect\hypertarget{22_Chapter_Fourteen__THE_COMING_OF.xhtmlux5cux23id_62}{\protect\hyperlink{23_NOTES.xhtmlux5cux23id_63}{31}}}
he is, for this reason, no less a genuine humanist. We see at the same
time, however, how the old spirit still lives in him. He still uses his
rhetorical skills in Latin for the old medieval subjects such as a
diatribe against
marriage\textsuperscript{\protect\hypertarget{22_Chapter_Fourteen__THE_COMING_OF.xhtmlux5cux23id_60}{\protect\hyperlink{23_NOTES.xhtmlux5cux23id_61}{32}}}
or disapproval of life at court, by retranslating Alain Chartier's
\emph{Curial} back into Latin.
\protect\hypertarget{22_Chapter_Fourteen__THE_COMING_OF.xhtmlux5cux23page_393}{}{}Or,
in this instance in a French poem, he treats the social value of the
estates in the frequently used form of a debate, ``Le debat du
laboureur, du prestre et du gendarme.'' In his French poems, Gaguin, who
had a perfect command of the Latin style, did not indulge in any of the
rhetorical embellishments; there were no Latinized forms, no hyperbolic
phrases, no mythology. As a French poet he stands squarely on the side
of those who preserve in their medieval form their naturalness and, with
it, their readability. For him, the humanist form is hardly anything
other than a gown that he can casually don; it may fit him well, but he
moves more freely without that splendiferousness. The Renaissance is
only loosely tied to the French spirit of the fifteenth century.

We are accustomed to regard the appearance of pagan-sounding expression
as an unmistakable criterion for the beginning of the Renaissance. But
every student of medieval literature knows that this literary paganism
was by no means limited to the sphere of the Renaissance. When humanists
call God ``princips superum'' and Mary ``genetrix tonantis,'' they do
not commit a sacrilege. This purely external transposing of the persons
of the Christian faith into the names of pagan mythology is very
old\textsuperscript{\protect\hypertarget{22_Chapter_Fourteen__THE_COMING_OF.xhtmlux5cux23id_59}{\protect\hyperlink{23_NOTES.xhtmlux5cux23page_440}{33}}}
and means little or nothing for the content of religious sentiment. The
arch-poet of the twelfth century unconcernably rhymes in his confession:

\emph{Vita vetus displicet, mores placent novi};

\emph{Homo videt faciem, sed cor patet
Iovi.\protect\hypertarget{22_Chapter_Fourteen__THE_COMING_OF.xhtmlux5cux23id_2783}{\protect\hyperlink{23_NOTES.xhtmlux5cux23id_2784}{*\textsuperscript{15}}}}

When Deschamps speaks of ``Jupiter venu de Paradis,
``\textsuperscript{\protect\hypertarget{22_Chapter_Fourteen__THE_COMING_OF.xhtmlux5cux23id_57}{\protect\hyperlink{23_NOTES.xhtmlux5cux23id_58}{34}}}\protect\hypertarget{22_Chapter_Fourteen__THE_COMING_OF.xhtmlux5cux23id_2781}{\protect\hyperlink{23_NOTES.xhtmlux5cux23id_2782}{†\textsuperscript{16}}}
he does not intend any godlessness; as little as does Villon in the
touching ballade he made for his mother when, in order to pray, he calls
Our Dear Lady ``haulte
Déesse.''\textsuperscript{\protect\hypertarget{22_Chapter_Fourteen__THE_COMING_OF.xhtmlux5cux23id_55}{\protect\hyperlink{23_NOTES.xhtmlux5cux23id_56}{35}}}\protect\hypertarget{22_Chapter_Fourteen__THE_COMING_OF.xhtmlux5cux23id_2779}{\protect\hyperlink{23_NOTES.xhtmlux5cux23id_2780}{‡\textsuperscript{17}}}

A certain heathen coloration also belongs to the pastorale; there it was
safe to have the pagan deities appear. In ``Le pastoralet,'' the
Celestine monastery in Paris is called ``temple au haulz bois pour le
diex
prier.''\textsuperscript{\protect\hypertarget{22_Chapter_Fourteen__THE_COMING_OF.xhtmlux5cux23id_53}{\protect\hyperlink{23_NOTES.xhtmlux5cux23id_54}{36}}}\protect\hypertarget{22_Chapter_Fourteen__THE_COMING_OF.xhtmlux5cux23id_2777}{\protect\hyperlink{23_NOTES.xhtmlux5cux23id_2778}{§\textsuperscript{18}}}
But nobody was led astray by such innocent
pa\protect\hypertarget{22_Chapter_Fourteen__THE_COMING_OF.xhtmlux5cux23page_394}{}{}ganism.
On top of all this, the poet also declared ``se pour estrangier ma muse
je parle des dieux des païens, sy sont les pastours crestiens et
moy.''\textsuperscript{\protect\hypertarget{22_Chapter_Fourteen__THE_COMING_OF.xhtmlux5cux23id_51}{\protect\hyperlink{23_NOTES.xhtmlux5cux23id_52}{37}}}\protect\hypertarget{22_Chapter_Fourteen__THE_COMING_OF.xhtmlux5cux23id_2775}{\protect\hyperlink{23_NOTES.xhtmlux5cux23id_2776}{*\textsuperscript{19}}}
Molinet, too, shifts responsibility for having Mars and Minerva appear
in a dream poem to ``Raison et entendement,'' who say to him: ``Tu le
dois faire non pas pour adjouter foy aux dieux et déesses, mais pour ce
que Nostre Seigneur seul inspire les gens ainsi qu'il lui plaist, et
souventes fois par divers
inspirations.''\textsuperscript{\protect\hypertarget{22_Chapter_Fourteen__THE_COMING_OF.xhtmlux5cux23id_49}{\protect\hyperlink{23_NOTES.xhtmlux5cux23id_50}{38}}}\protect\hypertarget{22_Chapter_Fourteen__THE_COMING_OF.xhtmlux5cux23id_2773}{\protect\hyperlink{23_NOTES.xhtmlux5cux23id_2774}{†\textsuperscript{20}}}

Much of the literary paganism of the fully developed Renaissance should
not be taken more seriously than these expressions. However, if a feel
for recognizing pagan faith as such, particularly pagan sacrifices,
announces itself, it is of deeper significance for the advance of the
new spirit. This sentiment may surface even among those as deeply rooted
in the Middle Ages as Chastellain:

\emph{Des dieux jadis les nations gentiles}

\emph{Quirent l'amour par humbles sacrifices},

\emph{Lesquels, posé que ne fussent utiles}.

\emph{Furent nientmoins rendables et fertiles}

\emph{De maint grant fruit et de haulx bénéfices},

\emph{Monstrans par fait que d'amour les offices}

\emph{Et d'honneur humble, impartis où qu'ils soient}

\emph{Pour percer ciel et enfer
suffisoient}.\textsuperscript{\protect\hypertarget{22_Chapter_Fourteen__THE_COMING_OF.xhtmlux5cux23id_47}{\protect\hyperlink{23_NOTES.xhtmlux5cux23id_48}{39}}}\protect\hypertarget{22_Chapter_Fourteen__THE_COMING_OF.xhtmlux5cux23id_2771}{\protect\hyperlink{23_NOTES.xhtmlux5cux23id_2772}{‡\textsuperscript{21}}}

The sound of the Renaissance may suddenly ring out in the midst of
medieval life. During a \emph{pas d'armes} in Arras in 1446, Philippe de
Ternant, contrary to the then prevailing custom, appears without a
``bannerole de devocion,'' a banner with a pious saying or figure.
``Laquelle chose je ne prise
point,''\protect\hypertarget{22_Chapter_Fourteen__THE_COMING_OF.xhtmlux5cux23id_2769}{\protect\hyperlink{23_NOTES.xhtmlux5cux23id_2770}{§\textsuperscript{22}}}
comments La Marche on this infamy. But still more infamous is the motto
worn by Ternant:
\protect\hypertarget{22_Chapter_Fourteen__THE_COMING_OF.xhtmlux5cux23page_395}{}{}``Je
souhaite que avoir puisse de mes desirs assouvissance et jamais aultre
bien
n'eusse.''\textsuperscript{\protect\hypertarget{22_Chapter_Fourteen__THE_COMING_OF.xhtmlux5cux23id_45}{\protect\hyperlink{23_NOTES.xhtmlux5cux23id_46}{40}}}\protect\hypertarget{22_Chapter_Fourteen__THE_COMING_OF.xhtmlux5cux23id_2767}{\protect\hyperlink{23_NOTES.xhtmlux5cux23id_2768}{*\textsuperscript{23}}}
This could well have been the motto of the most free-thinking libertines
of the sixteenth century.

Individuals did not have to go to classic literature for a source for
this real paganism. They could find it in their own medieval treasury,
the \emph{Roman de la rose}. The paganism was found in the erotic
cultural forms. Here Venus and the God of Love had, for a long time,
their hiding place where they received something more than purely
rhetorical veneration. Jean de Meun embodies the great pagan. For
innumerable readers since the thirteenth century, the school of paganism
had not been his merging of the names of the gods of antiquity with
those of Jesus and Mary, but the fact that he offered in a most daring
fashion earthly lust permeated with the Christian notion of bliss. It is
difficult to imagine a greater blasphemy than the words from Genesis:
``Then the Lord regretted that he had made man on the earth,'' put, with
a reversed meaning, into the mouth of Mother Nature, who, in his poem,
functions perfectly as a demiurge: Nature regretted having created human
beings because they do not pay attention to the commandment to
procreate:

\emph{Si m'aïst Diex li crucefis},

\emph{Moult me repens dont homme
fis}\textsuperscript{\protect\hypertarget{22_Chapter_Fourteen__THE_COMING_OF.xhtmlux5cux23id_43}{\protect\hyperlink{23_NOTES.xhtmlux5cux23id_44}{41}}}\protect\hypertarget{22_Chapter_Fourteen__THE_COMING_OF.xhtmlux5cux23id_2765}{\protect\hyperlink{23_NOTES.xhtmlux5cux23id_2766}{†\textsuperscript{24}}}

It is amazing that the church that was overzealously on guard against
small dogmatic deviations of a speculative nature, and reacted to them
with such vehemence, allowed the teaching of this breviary for the
aristocracy to continue to grow luxuriously in the mind without putting
any impediments in its way.

The new form and the new spirit do not correspond to each other. Just as
the thoughts of the coming age were expressed in a medieval garb, the
most medieval ideas were presented in sapphic meters with a whole train
of mythological figures. Classicism and the modern spirit are two
entirely different entities. Literary classicism is a child born aged.
Antiquity hardly held more significance for the renewal of \emph{la
belle littérature} than the arrows of Philotectes. The case is entirely
different in the fine arts and scientific thinking:
\protect\hypertarget{22_Chapter_Fourteen__THE_COMING_OF.xhtmlux5cux23page_396}{}{}for
both, antique purity of presentation and expression, antique
multifaceted interests, antique control of one's own life and insight
into man, meant much more than a mere crutch on which to lean. In the
fine arts, the overcoming of superfluity, exaggeration, twistedness, of
the grimace and of the flamboyantly curved, was all the work of
antiquity. In the domain of thought, it was still more indispensable and
fertile. But in the literary domain classicism was more an impediment
than a prerequsite to an unfolding simplicity and harmony.

Those few in the France of the fifteenth century who adopt humanistic
forms do not yet ring in the Renaissance because their sentiments and
orientation are still medieval. The Renaissance only arrives when the
``tone of life'' is changing, when the ebb tide of the deadly denial of
life has given way to a new flood and a stiff, fresh breeze is blowing;
it arrives only when the joyful insight (or was it an illusion?) has
ripened that all the glories of the ancient world, of which for so long
men had seen themselves the reflection, could be reclaimed.
