\chapter{RELIGIOUS EXCITATION AND RELIGIOUS FANTASY}

FROM THE TIME IN THE TWELFTH CENTURY WHEN the lyrical mysticism of
Bernard of Clairvaux had begun the fugue of flowering emotion over the
passion of Christ, the medieval mind came in ever increasing measure to
be filled with devout empathy with the story of his passion;
consciousness was entirely permeated and saturated with Christ and the
Cross. In earliest childhood the image of the Crucifixion was planted in
tender minds in such large and somber dimensions that it overshadowed
all other emotion with its dark mood. When still a child, Jean de Gerson
was told by his father, who was standing in front of a wall with his
arms outspread, ``Look, my boy, here is how your God, who created and
saved you, was crucified and died.'' The image remained etched in his
memory even in old age, and he blessed his pious father for it on the
day of his father's death, which fell on the feast of the Exaltation of
the
Cross.\textsuperscript{\protect\hypertarget{15_Chapter_Eight__RELIGIOUS_EXCITAT.xhtmlux5cux23id_905}{\protect\hyperlink{23_NOTES.xhtmlux5cux23id_906}{1}}}
Colette, when a child four years of age, heard her mother weep and sigh
during her daily prayers as she vicariously suffered the mockery, the
beatings, and the martyrdom of the Passion. The memory of these prayers
became fixed in her oversensitive mind with such intensity that she
experienced daily throughout her whole life a most intense tightening
and pain in her heart at the hour of the Crucifixion. She suffered more
than any woman in childbirth during the times she was reading about the
sufferings of the
Lord.\textsuperscript{\protect\hypertarget{15_Chapter_Eight__RELIGIOUS_EXCITAT.xhtmlux5cux23id_903}{\protect\hyperlink{23_NOTES.xhtmlux5cux23id_904}{2}}}
A preacher might occasionally stand silently in front of his
congregation, in the position of the crucified Lord, for a quarter of an
hour.\textsuperscript{\protect\hypertarget{15_Chapter_Eight__RELIGIOUS_EXCITAT.xhtmlux5cux23id_901}{\protect\hyperlink{23_NOTES.xhtmlux5cux23id_902}{3}}}

The mind was filled with Christ to such a degree that the Christological
note immediately began to sound whenever any act or thought showed even
the slightest congruence with the life or suffering of the Lord. A poor
nun who carries firewood to the kitchen imagines that she is carrying
the Cross. The notion of carrying wood by itself is enough to bathe the
activity in the bright glow
\protect\hypertarget{15_Chapter_Eight__RELIGIOUS_EXCITAT.xhtmlux5cux23page_221}{}{}of
the highest act of love. A blind woman, washing her laundry, imagines
the washtub and washroom to be the manger and the
stable.\textsuperscript{\protect\hypertarget{15_Chapter_Eight__RELIGIOUS_EXCITAT.xhtmlux5cux23id_899}{\protect\hyperlink{23_NOTES.xhtmlux5cux23id_900}{4}}}
The profaning overflow of princely homage into religious ideas, such as
the comparison of Louis XI with Jesus or of the Emperor and his son and
grandson with the
Trinity,\textsuperscript{\protect\hypertarget{15_Chapter_Eight__RELIGIOUS_EXCITAT.xhtmlux5cux23id_897}{\protect\hyperlink{23_NOTES.xhtmlux5cux23id_898}{5}}}
was equally the result of that overabundance of devotional content.

The fifteenth century displays this strong religious emotion in a dual
form. On the one side it reveals itself in those vehement moments when
an itinerant preacher periodically seizes a whole crowd with his words,
igniting all that spiritual fuel like dry tinder. This is the spasmodic
expression of that Christological emotion: passionate, intense, but
highly transitory. The other aspect is shown by a few individuals who
lead their sensitivity into a path of eternal quietude and normalize it
into a new life form, that of introspectiveness. This is the pietistic
circle of those who, fully conscious of being innovators, call
themselves the Modern Devotees, that is contemporary people of piety. As
a formal movement the \emph{devotio moderna} is restricted to the
northern Netherlands and lower Germany, but the spirit that gave rise to
it also existed in France.

Little remains to us of the forceful impact of the sermon on spiritual
culture. We know what tremendous influence preachers
had,\textsuperscript{\protect\hypertarget{15_Chapter_Eight__RELIGIOUS_EXCITAT.xhtmlux5cux23id_895}{\protect\hyperlink{23_NOTES.xhtmlux5cux23id_896}{6}}}
but it is not granted to us to relive the actual excitement they
generated. The written versions of sermons do not touch our hearts; how
could they? The people of the time had already become indifferent to the
written versions. Many who heard Vincent Ferrer and then read his
sermons, says his biographer, assure us that they are barely a shadow of
what they experienced when they heard them out of his own
mouth.\textsuperscript{\protect\hypertarget{15_Chapter_Eight__RELIGIOUS_EXCITAT.xhtmlux5cux23id_893}{\protect\hyperlink{23_NOTES.xhtmlux5cux23id_894}{7}}}
No wonder. What we have in the printed sermons of Vincent Ferrer or
Olivier
Maillard\textsuperscript{\protect\hypertarget{15_Chapter_Eight__RELIGIOUS_EXCITAT.xhtmlux5cux23id_891}{\protect\hyperlink{23_NOTES.xhtmlux5cux23id_892}{8}}}
is barely more than the basic material used by their eloquence, stripped
of all oratorical heat and apparently formal in their divisions into
sections, first, seventh, etc. We know that the people were always moved
by the gripping account of the terror of hell, by the thundering threats
of the punishment of sin, and by all the lyrical outpourings about the
Passion story and God's love. We know the devices employed by the
preachers; no effect was too crude, no change from laughter to tears too
abrupt, no intemperate raising of the voice too
crass.\textsuperscript{\protect\hypertarget{15_Chapter_Eight__RELIGIOUS_EXCITAT.xhtmlux5cux23id_889}{\protect\hyperlink{23_NOTES.xhtmlux5cux23id_890}{9}}}
But we really have to guess about the kinds of excitement they generated
on the basis of always identical reports about quarrels between cities
over who would have the favor of the
\protect\hypertarget{15_Chapter_Eight__RELIGIOUS_EXCITAT.xhtmlux5cux23page_222}{}{}next
sermon and over the ostentation, usually reserved for princes, lavished
by the officials and the people on the procession welcoming the preacher
into the city; and about how the preacher would occasionally have to
interrupt his sermon because of the weeping of the crowd. During a
sermon by Vincent Ferrer, a man and a woman under sentence of death were
led past the site. Vincent asked that the execution be delayed, hid the
pair beneath his pulpit for the duration, and preached about their sins.
After the sermon, nothing was found beneath the pulpit but a few bones.
The people believed nothing but that the words of the saintly man had
burnt the sinners and at the same time saved
them.\textsuperscript{\protect\hypertarget{15_Chapter_Eight__RELIGIOUS_EXCITAT.xhtmlux5cux23id_887}{\protect\hyperlink{23_NOTES.xhtmlux5cux23id_888}{10}}}

The tense emotions of the masses as they listened to the words of the
preacher always evaporated without a chance of becoming a part of the
written tradition, but the ``introspectiveness'' of the modern devotion
we know much better. In every pietist circle, religion not only supplies
the form of life, but also the forms of its socialization: the cozy
spiritual intercourse in quiet intimacy between simple men and women
whose vast heaven covered a tiny world over which sweeps the mighty
rustle of eternity. The friends of Thomas à Kempis admire his ignorance
of mundane things; a prior of Windesheim is given the complimentary
nickname Jan I-Don't-Know. They have no use for the world unless it is
simplified; they purify it by excluding
evil.\textsuperscript{\protect\hypertarget{15_Chapter_Eight__RELIGIOUS_EXCITAT.xhtmlux5cux23id_885}{\protect\hyperlink{23_NOTES.xhtmlux5cux23id_886}{11}}}
Within this narrow sphere, they live in the joy of a sensitive mutual
fondness. They keep one another in constant view so that all the signs
of blessedness can be detected; visits are their
delight,\textsuperscript{\protect\hypertarget{15_Chapter_Eight__RELIGIOUS_EXCITAT.xhtmlux5cux23id_883}{\protect\hyperlink{23_NOTES.xhtmlux5cux23id_884}{12}}}
hence their special inclination towards biographical description to
which we owe our detailed knowledge of their spiritual state.

In the regulated form in the Netherlands, the \emph{devotio moderna}
created a strong conventional form for the pious life. Devotees were
recognized by their calm, measured motions, their stooped way of
walking, some by the broad smile on their face or their new clothes that
were intentionally patched. Not least of all, there were their copious
tears: ``Devotio est quaedam cordis teneritudo, qua quis in pias
faciliter resolvitur lacrimas.''---Devotion is a certain tenderness of
the heart that allows an individual to easily dissolve in tears. One has
to ask God for the ``daily baptism of tears,'' they are the wings of
prayer or, in the words of St. Bernard, the wine of angels. One should
surrender to the bliss of praiseworthy tears, should be ready for them,
and encourage them, throughout
\protect\hypertarget{15_Chapter_Eight__RELIGIOUS_EXCITAT.xhtmlux5cux23page_223}{}{}the
year, but especially during Lent, so that one will be able to say with
the Psalmist, ``Fuerunt mihi lacrimae meae panes die ac nocte.''
Sometimes tears come so readily that we pray with sighs and wailing
(``ita ut suspiriose ac cum rugitu oremus''), but if they do not come of
themselves, one should not squeeze them out too hard, but be satisfied
with the tears of the heart. In the presence of others, the signs of an
unusual spiritual devotion should be
avoided.\textsuperscript{\protect\hypertarget{15_Chapter_Eight__RELIGIOUS_EXCITAT.xhtmlux5cux23id_881}{\protect\hyperlink{23_NOTES.xhtmlux5cux23id_882}{13}}}

Vincent Ferrer shed so many tears every time he celebrated mass that
nearly all those present wept with him and occasionally produced a
wailing like that over a death. Weeping was such a joy to him that he
was reluctant to withhold his
tears.\textsuperscript{\protect\hypertarget{15_Chapter_Eight__RELIGIOUS_EXCITAT.xhtmlux5cux23id_879}{\protect\hyperlink{23_NOTES.xhtmlux5cux23id_880}{14}}}

In France the new pietism did not experience such a strange forcing into
a particular form such as that of the Dutch Fraterhouses or Windesheim
Congregations. Related spirits in France either remained in the secular
world or joined existing orders where the new pietism did then gradually
lead to the implementation of a stricter observance. In France, the
phenomenon is not widely known among bourgeois circles. This may have
accounted for the fact that French piety had a more passionate,
spasmodic character than its Dutch counterpart. It resorted more readily
to exaggerated forms but also evaporated more easily. Towards the end of
the Middle Ages, visitors from more southern regions to the northern
Netherlands notice on more than one occasion the serious and general
piety that they observe to be a special characteristic of the people of
this
country.\textsuperscript{\protect\hypertarget{15_Chapter_Eight__RELIGIOUS_EXCITAT.xhtmlux5cux23id_877}{\protect\hyperlink{23_NOTES.xhtmlux5cux23id_878}{15}}}

Dutch devotees abandoned the contact with intensive mysticism that had
been characteristic of the initial stages of their life form and along
with that abandonment they managed to keep dangerous and fantastic
heretical deviations in check. Dutch modern devotion was obedient and
orthodox, practical, decent and occasionally even sensible. In contrast,
French devotion seems to have oscillated much more widely, touching time
and again on the more extravagant phenomena of faith.

When the Groningen Dominican Mattheus Grabow went to Constance to
present to the council all the complaints of the mendicant orders
against the new Brothers of the Common Life and, if possible, to get
them condemned, it was in the great leader of conservative church
politics, Jean de Gerson himself, that the disciples of Gerard Groote
found their defender. Gerson was completely
\protect\hypertarget{15_Chapter_Eight__RELIGIOUS_EXCITAT.xhtmlux5cux23page_224}{}{}competent
to judge whether this was an expression of genuine piety and a permitted
form of organization, since the distinction between genuine piety and
exaggerated expressions of faith was one of those points to which he was
constantly attentive. Gerson had a cautious, conscientious, academic
mind, honest, pure, and of good will. He had a cautious concern for good
form that frequently betrayed the fact that his fine mind had risen from
its humble origin to its truly aristocratic reputation. Moreover, he was
a psychologist and had a sense of style. As we know, sense of style and
orthodoxy are most intimately related. Small wonder that the
contemporary expressions of the life of faith repeatedly aroused his
suspicion and concern. It is strange that the types of piety of which he
disapproved as exaggerated and dangerous strongly remind us of the
modern devotion that he defended. But this is understandable. His French
sheep lacked a secure sheepfold, a discipline and organization that
would hold all those zealous believers within the borders of what the
church could tolerate.

Gerson saw the dangers of popular devotion everywhere. He considered it
a mistake to take mysticism into the
street.\textsuperscript{\protect\hypertarget{15_Chapter_Eight__RELIGIOUS_EXCITAT.xhtmlux5cux23id_875}{\protect\hyperlink{23_NOTES.xhtmlux5cux23id_876}{16}}}
The world, he says, is in the last period shortly before the end and,
like a demented old man, is victimized by all kinds of fantasies,
visions, and illusions, which make many individuals stray from the
truth.\textsuperscript{\protect\hypertarget{15_Chapter_Eight__RELIGIOUS_EXCITAT.xhtmlux5cux23id_873}{\protect\hyperlink{23_NOTES.xhtmlux5cux23id_874}{17}}}
Lacking proper guidance, many succumb to all too strict fasts, all too
extended vigils, and superfluous tears with which they cloud their
brains. They turn a deaf ear to admonitions for moderation. Let them be
on guard, because they can easily fall victim to the Devil's delusions.
A short time ago, he had visited a wife and mother in Arras who, against
the wishes of her husband, engaged in total fasts that lasted two to
four days, for which she was greatly admired. Gerson had talked with
her, had thoroughly tested her, and found that her abstinence was
arrogant and stubborn because she ate with insatiable voracity when such
fasts ended. As a reason for her self-inflicted austerities, she stated
nothing other than that she was unworthy to eat bread. Her external
appearance already betrayed her approaching
insanity.\textsuperscript{\protect\hypertarget{15_Chapter_Eight__RELIGIOUS_EXCITAT.xhtmlux5cux23id_871}{\protect\hyperlink{23_NOTES.xhtmlux5cux23id_872}{18}}}
Another woman, an epileptic, whose corns twinged whenever a soul went to
hell, who read sins from foreheads, and who claimed that she saved three
souls a day, confessed under threat of torture that she only behaved
this way because it was her means of making a
living.\textsuperscript{\protect\hypertarget{15_Chapter_Eight__RELIGIOUS_EXCITAT.xhtmlux5cux23id_869}{\protect\hyperlink{23_NOTES.xhtmlux5cux23id_870}{19}}}

Gerson did not value very highly the visions and revelations of
\protect\hypertarget{15_Chapter_Eight__RELIGIOUS_EXCITAT.xhtmlux5cux23page_225}{}{}recent
times that were available to the reading public everywhere. He even
rejected those of famous saints such as Bridget of Sweden and Catherine
of
Siena.\textsuperscript{\protect\hypertarget{15_Chapter_Eight__RELIGIOUS_EXCITAT.xhtmlux5cux23id_867}{\protect\hyperlink{23_NOTES.xhtmlux5cux23id_868}{20}}}
He had heard so many revelations that he had been robbed of his trust in
them. Many declared that it had been revealed to them that they would
become pope; a learned man had even described it with his own hand and
had supported it with various proofs. Yet another had been convinced, in
succession, that he would become pope, still later, that he would become
the Antichrist, or at least his forerunner, and had for this reason
entertained the idea of committing suicide so that Christendom would be
spared such an
evil.\textsuperscript{\protect\hypertarget{15_Chapter_Eight__RELIGIOUS_EXCITAT.xhtmlux5cux23id_865}{\protect\hyperlink{23_NOTES.xhtmlux5cux23id_866}{21}}}
Nothing is so dangerous, says Gerson, as ignorant devotion. If the pious
poor hear that Mary's spirit exulted in God, they try to exult too and,
sometimes in love, sometimes in fear, imagine all kinds of things to
happen. They see all kinds of visions that they cannot distinguish from
the truth and that they take for miracles and proof of their excellent
devotion.\textsuperscript{\protect\hypertarget{15_Chapter_Eight__RELIGIOUS_EXCITAT.xhtmlux5cux23id_863}{\protect\hyperlink{23_NOTES.xhtmlux5cux23id_864}{22}}}
This, however, is exactly what the modern devotion prescribed: ``He who
is intent upon making himself, by this path, with all his heart and by
all his efforts, equal in spirit to the sufferings of the Lord, should
strive to make himself humble and fearful. And if he is in danger, he
should join this danger to the danger of Christ and be ready to share it
with
him.''\textsuperscript{\protect\hypertarget{15_Chapter_Eight__RELIGIOUS_EXCITAT.xhtmlux5cux23id_861}{\protect\hyperlink{23_NOTES.xhtmlux5cux23id_862}{23}}}

The contemplative life is fraught with great dangers, says Gerson; many
have become melancholy or mentally ill because of
it.\textsuperscript{\protect\hypertarget{15_Chapter_Eight__RELIGIOUS_EXCITAT.xhtmlux5cux23id_859}{\protect\hyperlink{23_NOTES.xhtmlux5cux23id_860}{24}}}
He knows how easily overlong fasts can lead to madness or
hallucinations; he also knows of the role of fasts in
magic.\textsuperscript{\protect\hypertarget{15_Chapter_Eight__RELIGIOUS_EXCITAT.xhtmlux5cux23id_857}{\protect\hyperlink{23_NOTES.xhtmlux5cux23id_858}{25}}}
Now, where is the man with such a keen eye for the psychological element
in the expressions of faith to draw the boundary between what is holy
and permissible and what should be rejected? He himself sensed that-the
question of orthodoxy alone did not suffice; as a trained theologian it
was easy enough to pronounce judgment whenever there was evident
departure from dogma. There remained, however, those cases that were
outside the pale of dogma, where an ethical evaluation of the
expressions of faith had to be his guide and where his sense of
appropriateness and good taste had to prompt his judgment. There is no
virtue more lost from sight in these miserable times of the schism than
that of ``discretio,'' says
Gerson.\textsuperscript{\protect\hypertarget{15_Chapter_Eight__RELIGIOUS_EXCITAT.xhtmlux5cux23id_855}{\protect\hyperlink{23_NOTES.xhtmlux5cux23id_856}{26}}}

While to Jean de Gerson the criterion of dogma was no longer the only
one to tip the scale in distinguishing between false and true piety, so
much the more are we today inclined to judge types
\protect\hypertarget{15_Chapter_Eight__RELIGIOUS_EXCITAT.xhtmlux5cux23page_226}{}{}of
religious intensity not only by the yardstick of their orthodoxy or
heresy, but according to their psychological nature. Dogmatic
distinctions were not seen by the people of that time themselves. They
received from the heretical Brother Thomas just as much edification as
they did from the saintly Vincent Ferrer; they denounced St. Colette and
her successor as swindlers and
hypocrites.\textsuperscript{\protect\hypertarget{15_Chapter_Eight__RELIGIOUS_EXCITAT.xhtmlux5cux23id_853}{\protect\hyperlink{23_NOTES.xhtmlux5cux23id_854}{27}}}
Colette displays all those characteristics that James called the
theopathic
state,\textsuperscript{\protect\hypertarget{15_Chapter_Eight__RELIGIOUS_EXCITAT.xhtmlux5cux23id_851}{\protect\hyperlink{23_NOTES.xhtmlux5cux23id_852}{28}}}
which is rooted in the soil of a most painful supersensitivity. She
cannot stand the sight of fire or tolerate its heat. Candles are the
only exception. She has an exaggerated fear of flies, slugs, ants,
stench, or impurity. She feels the same sickening disgust for sexuality
that was in later times displayed by St. Aloysius Gonzaga. She prefers
to have only virgins in her congregation, loathes married saints, and
regrets that her mother took her father in a second
marriage.\textsuperscript{\protect\hypertarget{15_Chapter_Eight__RELIGIOUS_EXCITAT.xhtmlux5cux23id_849}{\protect\hyperlink{23_NOTES.xhtmlux5cux23id_850}{29}}}
This passion for purest virginity was praised by the church as uplifting
and worthy of emulation. Virginity was harmless as long as it proclaimed
itself in the form of a personal disgust for anything sexual. But this
same feeling became dangerous, both to the church and to the individual
proclaiming his adherence to it, in another form: whenever virginity no
longer, like a snail, pulled in its feelers so as to lock itself
securely into a sphere of purity of its own, but rather desired this
preoccupation with chastity to be applied to the church and the social
lives of others. The church had to deny this striving for purity time
and again whenever it assumed revolutionary forms and expressed itself
in vehement attacks on the impurity of priests or the debauchery of
monks, because the medieval church knew that it was not in her power to
turn away these evils. Jean de Varennes paid for his insistence in the
miserable prison in which he was locked by the bishop of Reims. This
Jean de Varennes was a learned theologian and a famous preacher who
seemed to be in line for a bishopric or even a cardinal's hat while he
was serving at the papal court of Avignon as chaplain to the young
cardinal of Luxembourg. But all this had come to an end when he abruptly
renounced all his benefices, with the exception of a sinecure at Notre
Dame at Reims, renounced his rank and returned from Avignon to his
native region where he began to lead a saintly life and to preach in
Saint Lié. ``Et avoit moult grant hantise e poeuple qui le venoient vier
de tous pays pour la simple vie très-noble et moult honneste qui il
\protect\hypertarget{15_Chapter_Eight__RELIGIOUS_EXCITAT.xhtmlux5cux23page_227}{}{}menoit.''\protect\hypertarget{15_Chapter_Eight__RELIGIOUS_EXCITAT.xhtmlux5cux23id_2563}{\protect\hyperlink{23_NOTES.xhtmlux5cux23id_2564}{*\textsuperscript{1}}}
People thought that he would become pope; he was called ``le saint homme
de S.
Lié.''\protect\hypertarget{15_Chapter_Eight__RELIGIOUS_EXCITAT.xhtmlux5cux23id_2565}{\protect\hyperlink{23_NOTES.xhtmlux5cux23id_2566}{†\textsuperscript{2}}}
Many tried to kiss his hand and touch his gown because of his miraculous
powers. Some considered him to be a messenger of God or even a divine
being; all of France spoke of nothing
else.\textsuperscript{\protect\hypertarget{15_Chapter_Eight__RELIGIOUS_EXCITAT.xhtmlux5cux23id_847}{\protect\hyperlink{23_NOTES.xhtmlux5cux23id_848}{30}}}

But not everyone believed in the sincerity of his intentions; there were
those who spoke of the ``fou de S.
Lié''\protect\hypertarget{15_Chapter_Eight__RELIGIOUS_EXCITAT.xhtmlux5cux23id_2567}{\protect\hyperlink{23_NOTES.xhtmlux5cux23id_2568}{‡\textsuperscript{3}}}
or suspected him of using these sensational means for the purpose of
gaining the high clerical offices that had escaped him so far. Jean de
Varennes, as many earlier individuals, allows us to see how a passion
for sexual purity is transposed into revolutionary ways of thought. It
is as if he reduces all the complaints about the degeneration of the
church to one and the same evil: unchastity. He preaches, red-hot in his
outrage, protests, and complaints, against the church authorities, most
of all against the archbishop of Reims. ``Au loup, au
loup,''\protect\hypertarget{15_Chapter_Eight__RELIGIOUS_EXCITAT.xhtmlux5cux23id_2569}{\protect\hyperlink{23_NOTES.xhtmlux5cux23id_2570}{§\textsuperscript{4}}}
he shouts to the masses, and they understand only too well who is meant
by the ``wolf'' and eagerly shout back, ``Hahay, aus leus, mes bones
gens, aus leus.'' But it seems that Jean de Varennes did not have all
the courage of his convictions. In his defense from prison he claims
that he had never said that he meant the archbishop; he was only quoting
the proverb ``qui est tigneus, il ne doit pas oster son chaperon'' (``he
who has head sores should not take off his
cap'').\textsuperscript{\protect\hypertarget{15_Chapter_Eight__RELIGIOUS_EXCITAT.xhtmlux5cux23id_845}{\protect\hyperlink{23_NOTES.xhtmlux5cux23id_846}{31}}}
No matter how far he may have actually gone, what his audience had heard
in the sermon was the old teaching that had so often threatened to
disrupt the life of the church: the sacraments of a priest living an
unchaste life are invalid. The host he celebrates is nothing but bread,
baptism and absolution are worthless if done by him. In the case of Jean
de Varennes this was only part of a more extremist program of chastity:
priests should not be allowed to live even with a nun or an old woman;
twenty-two or twenty-three sins are connected with marriage; adulterers
should be punished according to the old covenant; Christ himself would
have stoned the woman taken in adultery if he had been
\protect\hypertarget{15_Chapter_Eight__RELIGIOUS_EXCITAT.xhtmlux5cux23page_228}{}{}certain
of her guilt; there was no chaste woman in all of France; no bastard
could do good or be
blessed.\textsuperscript{\protect\hypertarget{15_Chapter_Eight__RELIGIOUS_EXCITAT.xhtmlux5cux23id_843}{\protect\hyperlink{23_NOTES.xhtmlux5cux23id_844}{32}}}

The church always had to defend itself out of its sense of
self-preservation against this insistent form of disgust over
unchastity. Once doubts were to be raised about the efficacy of the
sacraments dispensed by unworthy priests, the entire life of the church
would be shaken to its foundations. Gerson put Jean de Varennes on the
same level as Jan Hus, someone with originally good intentions led
astray by
zealousness.\textsuperscript{\protect\hypertarget{15_Chapter_Eight__RELIGIOUS_EXCITAT.xhtmlux5cux23id_841}{\protect\hyperlink{23_NOTES.xhtmlux5cux23id_842}{33}}}

On the other hand, the church was usually very indulgent in another
area, tolerating highly sensuous depictions of the love of God. But the
conscientious chancellor of the University of Paris saw danger here,
too, and warned against it.

He knew this danger from his great psychological experience and in
various aspects, dogmatic as well as ethical. ``One day would not be
sufficient for me,'' he said, ``if I had to count all the many crazed
acts of those in love, of those who have lost their senses: amantium,
immo et
amentium.''\textsuperscript{\protect\hypertarget{15_Chapter_Eight__RELIGIOUS_EXCITAT.xhtmlux5cux23id_839}{\protect\hyperlink{23_NOTES.xhtmlux5cux23id_840}{34}}}
Indeed, he knew from experience, ``Amor spiriritualis facile labitur in
nudum carnalum amorem'' (``Spirtual love easily turns into purely carnal
love'').\textsuperscript{\protect\hypertarget{15_Chapter_Eight__RELIGIOUS_EXCITAT.xhtmlux5cux23id_837}{\protect\hyperlink{23_NOTES.xhtmlux5cux23id_838}{35}}}
Who else but himself could Gerson have meant when he speaks of a man
known to him once who had, out of praiseworthy devotion, cultivated a
trusting friendship in the Lord with a spiritual sister: ``In the
beginning there was no fire of a carnal nature, but gradually a love
grew from the regular meetings that was no longer rooted in God, so that
he could no longer resist visiting her or thinking of her in her
absence. Still, he suspected nothing sinful, no devilish deceit, until a
more prolonged absence caused him to gain an insight into the danger and
God turned him away from it just in
time.''\textsuperscript{\protect\hypertarget{15_Chapter_Eight__RELIGIOUS_EXCITAT.xhtmlux5cux23id_835}{\protect\hyperlink{23_NOTES.xhtmlux5cux23id_836}{36}}}
From then on, he was ``un homme averti'' and benefited from it. His
entire work, ``De diversis diaboli
tentationibus,''\textsuperscript{\protect\hypertarget{15_Chapter_Eight__RELIGIOUS_EXCITAT.xhtmlux5cux23id_833}{\protect\hyperlink{23_NOTES.xhtmlux5cux23id_834}{37}}}
is a keen analysis of a state of mind comparable to that of the Dutch
modern devotees. Above all, it is the ``dulcedo Dei,'' the ``sweetness''
of the Windesheimers that Gerson distrusted. The Devil, he says,
sometimes offers man an immeasurable and wondrous sweetness
\emph{(dulcedo)} of a kind like devotion and resembling it, so that one
makes that enjoyment and sweetness \emph{(suavitas)} his sole goal and
only loves and follows God in order to attain that
enjoyment.\textsuperscript{\protect\hypertarget{15_Chapter_Eight__RELIGIOUS_EXCITAT.xhtmlux5cux23id_831}{\protect\hyperlink{23_NOTES.xhtmlux5cux23id_832}{38}}}
Elsewhere\textsuperscript{\protect\hypertarget{15_Chapter_Eight__RELIGIOUS_EXCITAT.xhtmlux5cux23id_829}{\protect\hyperlink{23_NOTES.xhtmlux5cux23id_830}{39}}}
he says of the same \emph{dulcedo dei}: Many have been defeated by the
all too intense cultivation of such feelings; they have
\protect\hypertarget{15_Chapter_Eight__RELIGIOUS_EXCITAT.xhtmlux5cux23page_229}{}{}turned
to the ravings of their heart as if they were embracing God, and have
been miserably mistaken. This excess leads to all kinds of useless
effort. Some try to attain that state of complete insensitivity and
passivity in which all their acts are the result of the will of God, or
that mystic realization of and union with God wherein He is no longer
conceived of as a being or truth or goodness.---This was also the basis
of Gerson's critique of
Ruusbroec,\textsuperscript{\protect\hypertarget{15_Chapter_Eight__RELIGIOUS_EXCITAT.xhtmlux5cux23id_827}{\protect\hyperlink{23_NOTES.xhtmlux5cux23id_828}{40}}}
in whose naiveté he did not believe. He criticizes Ruusbroec's notion of
the ``Ornament of the Spiritual Wedding,'' which implies that the
perfect soul viewing God does not only view Him by means of the clarity
that is the divine essence, but by means of the fact that this soul is
God,
Himself.\textsuperscript{\protect\hypertarget{15_Chapter_Eight__RELIGIOUS_EXCITAT.xhtmlux5cux23id_825}{\protect\hyperlink{23_NOTES.xhtmlux5cux23id_826}{41}}}

The sense of the destruction of individuality that the mystics of all
times have enjoyed could not be admitted by a defender of the moderate,
old-fashioned Bernardinian mysticism such as Gerson. A visionary had
told him that her spirit had been destroyed in a real act of destruction
while viewing God, and had then been created anew. ``How do you know
that?'' he asked her. She herself had felt it, was her answer. The
logical absurdity of this explanation is triumphant proof, to the
intellectual chancellor, of how reprehensible such a feeling
is.\textsuperscript{\protect\hypertarget{15_Chapter_Eight__RELIGIOUS_EXCITAT.xhtmlux5cux23id_824}{\protect\hyperlink{23_NOTES.xhtmlux5cux23page_424}{42}}}
It was dangerous to express such emotions intellectually; the church
could only tolerate them in the form of images such as the one that had
the heart of Catherine of Siena transformed into the heart of Christ.
But Marguerite Porete of Hennegouw, a member of the Brothers of the Free
Spirit, who had also felt her soul to be destroyed in God, was burnt at
the stake in Paris in
1310.\textsuperscript{\protect\hypertarget{15_Chapter_Eight__RELIGIOUS_EXCITAT.xhtmlux5cux23id_822}{\protect\hyperlink{23_NOTES.xhtmlux5cux23id_823}{43}}}

The great danger posed by the feeling of annihilation of self was the
conclusion reached by Indian as well as Christian mystics that the
perfect, viewing and loving soul is no longer capable of sinning.
Immersed in God, it no longer has a will of its own; what remains is
only the divine will and if there should exist any carnal inclinations
there can be no sin in
them.\textsuperscript{\protect\hypertarget{15_Chapter_Eight__RELIGIOUS_EXCITAT.xhtmlux5cux23id_820}{\protect\hyperlink{23_NOTES.xhtmlux5cux23id_821}{44}}}
Innumerable poor and ignorant persons have been misled by such teachings
into a life of the most terrible excesses, as illustrated, for example,
by the sects of the Bégards, the Brothers of the Free Spirit, and the
Turlupins. Whenever Gerson spoke about the dangers of the mad love of
God,\textsuperscript{\protect\hypertarget{15_Chapter_Eight__RELIGIOUS_EXCITAT.xhtmlux5cux23id_818}{\protect\hyperlink{23_NOTES.xhtmlux5cux23id_819}{45}}}
he remembers the warning examples of those
sects.\textsuperscript{\protect\hypertarget{15_Chapter_Eight__RELIGIOUS_EXCITAT.xhtmlux5cux23id_816}{\protect\hyperlink{23_NOTES.xhtmlux5cux23id_817}{46}}}
But nearly identical emotions are found among the devotees. The
Windesheimer Hendrik van Herp accuses his own spiritual relatives of
spiritual
\protect\hypertarget{15_Chapter_Eight__RELIGIOUS_EXCITAT.xhtmlux5cux23page_230}{}{}adultery.\textsuperscript{\protect\hypertarget{15_Chapter_Eight__RELIGIOUS_EXCITAT.xhtmlux5cux23id_814}{\protect\hyperlink{23_NOTES.xhtmlux5cux23id_815}{47}}}
There were in this sphere of thought devilish traps, producing the most
perverse godlessness. Gerson tells of a respected man who had confessed
to a Carthusian that he would not be barred from the love of God by a
mortal sin, and he specifically named unchastity, that rather it
inflamed him to praise and to desire the divine sweetness the more
intensely.\textsuperscript{\protect\hypertarget{15_Chapter_Eight__RELIGIOUS_EXCITAT.xhtmlux5cux23id_812}{\protect\hyperlink{23_NOTES.xhtmlux5cux23id_813}{48}}}

The church was on guard as soon as the stirrings of mysticism were
transformed into clearly formulated convictions or found social
applications. As long as the results were mere passionate fantasies of a
symbolic nature, the church permitted even the most exuberant of them.
Johannes Brugman was allowed to apply all the characteristics of a
drunkard to the Incarnation of Christ, a drunkard who forgets himself,
sees no danger, does not get angry when mocked, gives everything away.
``O, and was he not drunk when love forced him to come from the highest
heaven into this lowest valley of the world?'' He walks around heaven
and serves the prophets from well-filled jugs, ``and they drank until
they burst and then David with his harp jumped before the table as if he
were the fool of my
Lord.''\textsuperscript{\protect\hypertarget{15_Chapter_Eight__RELIGIOUS_EXCITAT.xhtmlux5cux23id_810}{\protect\hyperlink{23_NOTES.xhtmlux5cux23id_811}{49}}}

Not only the grotesque Brugman, but also the pure Ruusbroec liked to
picture the love of God in the guise of drunkenness. Next to drunkenness
there is the image of hunger. They may both be an allusion to the
biblical ``que edunt me, adhuc esurient, et qui bibunt me, adhuc
sitient,''\textsuperscript{\protect\hypertarget{15_Chapter_Eight__RELIGIOUS_EXCITAT.xhtmlux5cux23id_808}{\protect\hyperlink{23_NOTES.xhtmlux5cux23id_809}{50}}}
which, uttered by Sapientia, was taken to be the word of the Lord. The
metaphor of the human spirit constantly visited by an eternal hunger for
God was put in this manner: ``An eternal hunger begins here which is
never satisfied. This is an internal longing and desire of the loving
power and of the created spirit for an uncreated good. .~.~. These are
the poorest people alive because they are greedy and voracious and they
are possessed by greed. No matter how much they eat and drink, they
never become satiated by it, because this kind of hunger is eternal.
.~.~. And if God were to grant to these unfortunates, all the gifts of
the saints save the gift of Himself, the gaping greed of the spirit
would yet remain hungry and unsatisfied.'' But just as the guise of
drunkenness, that of hunger is subject to a reversal.

Christ's hunger is great beyond measure; he devours us all to the
ground, because he is a greedy indulger and his hunger is insatiable. He
devours the marrow from our
\protect\hypertarget{15_Chapter_Eight__RELIGIOUS_EXCITAT.xhtmlux5cux23page_231}{}{}bones.
Yet we do not begrudge it and we will begrudge it less the better we
taste him. No matter what he eats of us he cannot become satiated for he
is greedy and his hunger is beyond measure. Although we are poor, he
pays no mind to it and has no wish to leave anything to us. First he
prepares his food and burns all our sins and infirmities in love; when
we are cleansed and roasted in love he yawns greedily to swallow all
this. .~.~. If we could see the greedy lust which Christ has for our
bliss, we would not be able to stop flying into his mouth. And if Jesus
devours us entirely, he gives us, in turn, himself; and he gives us the
spiritual hunger and thirst to partake of Him with eternal lust. He
gives us spiritual hunger and to our heartfelt love his own body as
food. And if we eat this body and, within us, enjoy it with deep
devotion, from it will flow his glorious hot blood into our nature and
into all our veins. .~.~. Look, thus we will always eat and be eaten,
and rise and fall with love, and this is our life in
eternity.\textsuperscript{\protect\hypertarget{15_Chapter_Eight__RELIGIOUS_EXCITAT.xhtmlux5cux23id_806}{\protect\hyperlink{23_NOTES.xhtmlux5cux23id_807}{51}}}

One small step more and we have gone once again from the highest
mysticism to a flat symbolism. ``You will eat him,'' says Jean
Berthelemy speaking of Communion in \emph{Le livre de crainte
amoureuse}, ``roasted in fire, well cooked, but not burnt. Just as the
Easter lamb was well cooked and roasted between two fires of wood or
coal, so was sweet Jesus tied on Good Friday on the spit of the worthy
cross and, between the two fires of very painful death and suffering and
that of all consuming love and \emph{Minne} which he bore for our souls
and for our bliss, he was roasted and slowly cooked in order to save
us.''\textsuperscript{\protect\hypertarget{15_Chapter_Eight__RELIGIOUS_EXCITAT.xhtmlux5cux23id_804}{\protect\hyperlink{23_NOTES.xhtmlux5cux23id_805}{52}}}

The metaphors of drunkenness and hunger by themselves contradict the
view that the feeling of religious bliss had to be symbolized
erotically.\textsuperscript{\protect\hypertarget{15_Chapter_Eight__RELIGIOUS_EXCITAT.xhtmlux5cux23id_802}{\protect\hyperlink{23_NOTES.xhtmlux5cux23id_803}{53}}}
The influx of the divine influence was felt just like drinking or
becoming satiated. A female devotee feels flooded by the blood of Jesus
Christ and
faints.\textsuperscript{\protect\hypertarget{15_Chapter_Eight__RELIGIOUS_EXCITAT.xhtmlux5cux23id_800}{\protect\hyperlink{23_NOTES.xhtmlux5cux23id_801}{54}}}
The blood fantasy that was continually invigorated by the belief in
transubstantiation expresses itself in the most intoxicating extremes of
red-hot emotion. The wounds of Jesus, says Bonaventura, are the blood
red flowers of our sweet and blooming paradise, above which the soul,
like a butterfly, has to fly, drinking first from one flower and then
from another. Through the wound in his side, the soul has to penetrate
\protect\hypertarget{15_Chapter_Eight__RELIGIOUS_EXCITAT.xhtmlux5cux23page_232}{}{}to
His heart itself. At the same time, His blood flows in the brooks of
paradise. All the warm, red blood has flowed through Suso's mouth into
his heart and
soul.\textsuperscript{\protect\hypertarget{15_Chapter_Eight__RELIGIOUS_EXCITAT.xhtmlux5cux23id_798}{\protect\hyperlink{23_NOTES.xhtmlux5cux23id_799}{55}}}
Catherine of Siena is one of the saints who have drunk from the blood
flowing from the wound in Christ's side, just as it was granted to
others to taste the milk from Mary's breast, as, for instance, St.
Bernard, Henry Suso, and Alain de la Roche.

Alain de la Roche, in Latin Alanus de Rupe, called Van der Klip by his
Dutch friends, may be regarded as one of the most noticeable types of
the French, more extreme devotion, and of the ultra-concrete fantasies
of faith of late medieval times. Born around 1428 in Brittany, he was
active as a Dominican primarily in the north of France and in the
Netherlands. He died in 1475 in Zwolle among the Brethren of the Common
Life, with whom he maintained lively relations. His main task was
agitation for the use of the rosary; for this purpose, he founded a
worldwide prayer brotherhood for which he prescribed the fixed system of
Hail Marys alternating with Our Fathers. In the printed works of this
visionary, mainly sermons and descriptions of his
visions,\textsuperscript{\protect\hypertarget{15_Chapter_Eight__RELIGIOUS_EXCITAT.xhtmlux5cux23id_796}{\protect\hyperlink{23_NOTES.xhtmlux5cux23id_797}{56}}}
we notice the strongly sexual element of his fantasy, but at the same
time the absence of any note of a glowing passion which could justify
his sexual emotions. The sensual expression of the all dissolving love
of God has here become mere \emph{procédé}. It contains nothing of the
overflowing fervor which elevates the fantasies about hunger, thirst,
blood, and love of the great mystics. His meditations about every part
of Mary's body, which he recommends, his exact description of how he had
repeatedly been refreshed by Mary's milk, in his systematic symbolism in
which he identifies each word of the Lord's Prayer as the bridal bed of
one of the virtues, all this reveals a spirit on the decline, the decay
of the strongly colored piety of the late medieval period into the form
of a flower past its prime.

The sexual element also has a place in the satanic fantasies. Alain de
la Roche sees the monsters of sin with disgusting genitals from which a
fiery and sulphur-like cloud is emitted that darkens the earth with its
smoke. He sees the \emph{meretrix
apostasiae\protect\hypertarget{15_Chapter_Eight__RELIGIOUS_EXCITAT.xhtmlux5cux23id_2571}{\protect\hyperlink{23_NOTES.xhtmlux5cux23id_2572}{*\textsuperscript{5}}}}
who devours the apostates, vomits them up and devours them again, kisses
and
\protect\hypertarget{15_Chapter_Eight__RELIGIOUS_EXCITAT.xhtmlux5cux23page_233}{}{}cuddles
them like a mother, and from her womb gives birth to them over and over
again.\textsuperscript{\protect\hypertarget{15_Chapter_Eight__RELIGIOUS_EXCITAT.xhtmlux5cux23id_794}{\protect\hyperlink{23_NOTES.xhtmlux5cux23id_795}{57}}}

This is the dark side of the ``sweetness'' of the devotees. As an
inevitable complement to the sweet heavenly fantasy, the mind harbored a
black cesspool of hellish notions that were expressed in the fiery
language of earthly sensuality. It is not so strange that there are
connections between the sedate circles of the Windesheimers and the
darkest product of the Middle Ages in their final years: the
witch-hunting madness that had by then grown into that fatally
concluding system of theological zeal and judicial severity. Alanus de
Rupe is a link in the chain. He was the teacher of his fellow Dominican,
Jacob Sprenger, who not only coauthored the \emph{Hammer of
Witches\textsuperscript{\protect\hypertarget{15_Chapter_Eight__RELIGIOUS_EXCITAT.xhtmlux5cux23id_792}{\protect\hyperlink{23_NOTES.xhtmlux5cux23id_793}{58}}}}
with Heinrich Institoris, but was also the promoter in Germany of
Alanus's Brotherhood of the Rosary.
