\chapter{THE VISION OF DEATH}

NO OTHER AGE HAS SO FORCEFULLY AND CONTINUously impressed the idea of
death on the whole population as did the fifteenth century, in which the
call of the memento mori
\protect\hypertarget{12_Chapter_Five__THE_VISION_OF_DEAT.xhtmlux5cux23id_2952}{\protect\hyperlink{23_NOTES.xhtmlux5cux23id_2951}{*\textsuperscript{1}}}
echoes throughout the whole of life. Denis the Carthusian, in the book
he wrote for the guidance of the nobleman, makes the exhortation that
``when he goes to bed, he should imagine not that he is putting himself
to bed, but that others are laying him in his
grave.''\textsuperscript{\protect\hypertarget{12_Chapter_Five__THE_VISION_OF_DEAT.xhtmlux5cux23id_1349}{\protect\hyperlink{23_NOTES.xhtmlux5cux23id_1350}{1}}}
In earlier times, too, religion had been very serious about reinforcing
the constant preoccupation with death, but the pious tracts of the early
medieval period had only reached those who had already taken the path
that put the world behind them. It was only after the rise of the
popular preachers of the mendicant orders that the admonitions rose to a
threatening chorus that echoed through the world with the force of a
fugue. Towards the end of the medieval period, the voice of the
preachers was joined by a new kind of pictorial representation that,
mostly in the form of woodcuts, reached all levels of society. These two
forceful means of expression, the sermon and the picture, could only
express the concept of death in very simple, direct, and lively images,
abrupt and sharp. The contents of earlier monastic meditations about
death were now condensed into a superficial, primitive, popular, and
lapidary image and in this form held up to the multitudes in sermons and
representations. This image of death was able to contain only one of the
large number of conceptions related to death, and that was
perishability. It seems as if the late medieval mind could see no other
aspect of death than that of decay.

There were three themes that furnished the melody for the never ending
lament about the end of all earthly glory. First there was the
\protect\hypertarget{12_Chapter_Five__THE_VISION_OF_DEAT.xhtmlux5cux23page_157}{}{}motif
that asked, where have all those gone who once filled the earth with
their glory? Then there was the motif of the horrifying sight of the
decomposition of all that had once constituted earthly beauty. The last
was the motif of the \emph{danse macabre} or \emph{Totentanz}, the dance
of death, which whirls away people of any age or profession.

Compared to the two final motifs in their oppressive dreadfulness, the
first, where has all the former splendor gone?, is only a soft elegiac
sigh. It is of ancient vintage and is known throughout the world of
Christianity and Islam. It originated in Greek paganism, the church
fathers knew it, and Byron perpetuates
it.\textsuperscript{\protect\hypertarget{12_Chapter_Five__THE_VISION_OF_DEAT.xhtmlux5cux23id_1347}{\protect\hyperlink{23_NOTES.xhtmlux5cux23id_1348}{2}}}
In the later Middle Ages it enjoyed a period of unusual popularity. It
can be found in the heavily rhymed hexameters of the Cluniac monk
Bernard of Morlay around 1140:

\emph{Est ubi gloria nunc Babylonia? nunc ubi dims}

\emph{Nabugodonosor}, \emph{et Darii vigor, illeque Cyrus?}

\emph{Qualiter orbita viribus inscita (?) praeterierunt},

\emph{Fama relinquiter, iliaque figitur, hi putruerunt}.

\emph{Nunc ubi curia, pompaque Julia? Caesar abisti!}

\emph{Te truculentior, orbe potentior ipse fuisti}.

. \emph{.~.~. . .~.~. . .~.~. . .~.~. . .~.~. . .~.~}.

\emph{Nunc ubi Marius atque fabricius inscius auri?}

\emph{Mors ubi nobilis et memorabilis actio Pauli?}

\emph{Diva philippica vox ubi coelica nunc Ciceronis?}

\emph{Pax ubi civibus atque rebellibus ira Catonis?}

\emph{Nunc ubi Regulus? aut ubi Romulus, aut ubi Remus?}

\emph{stat rosa pristina nomine, nomina nuda tenemus?}
\textsuperscript{\protect\hypertarget{12_Chapter_Five__THE_VISION_OF_DEAT.xhtmlux5cux23id_1345}{\protect\hyperlink{23_NOTES.xhtmlux5cux23id_1346}{3}}
\emph{\protect\hypertarget{12_Chapter_Five__THE_VISION_OF_DEAT.xhtmlux5cux23id_2955}{\protect\hyperlink{23_NOTES.xhtmlux5cux23id_2956}{*\textsuperscript{2}}}}}

It sounds again, this time less pedantically, in verses that retain the
sound of the rhymed hexameters in spite of their shorter structure,
\protect\hypertarget{12_Chapter_Five__THE_VISION_OF_DEAT.xhtmlux5cux23page_158}{}{}in
the Franciscan verses of the thirteenth century. Jacopone of Todi, the
jester of the Lord, is most likely the poet of the verses that appeared
under the title ``Cur mundis militat sub vana gloria.'' They include the
lines:

\emph{Die ubi Salomon, olim tant nobilis}

\emph{Vel Sampson ubi est, dux invincibilis}

\emph{Et pulcher Absalon, vultu mirabilis},

\emph{Aut dulcis Jonathas, multum amabilis?}

\emph{Quo Cesar abiit, celsus imperio?}

\emph{Quo Dives splendidus totus in prandio?}

\emph{Dic ubi Tullius, clarus eloquio}

\emph{Vel Aristoteles, summus
ingenio?}\textsuperscript{\protect\hypertarget{12_Chapter_Five__THE_VISION_OF_DEAT.xhtmlux5cux23id_1343}{\protect\hyperlink{23_NOTES.xhtmlux5cux23id_1344}{4}}}\protect\hypertarget{12_Chapter_Five__THE_VISION_OF_DEAT.xhtmlux5cux23id_2953}{\protect\hyperlink{23_NOTES.xhtmlux5cux23id_2954}{*\textsuperscript{3}}}

Deschamps sets the same theme in verse several times, Gerson uses it in
a sermon; Denis the Carthusian treats it in his tract about the ``Four
Last Things.'' Chastellain turns it into a long poem, \emph{Le pas de la
mort}, not to mention his other efforts in the same
vein.\textsuperscript{\protect\hypertarget{12_Chapter_Five__THE_VISION_OF_DEAT.xhtmlux5cux23id_1341}{\protect\hyperlink{23_NOTES.xhtmlux5cux23id_1342}{5}}}
Villon manages to add a new touch, that of gentle sorrow, in the
``Ballade des dames du temps jadis'' with the refrain:

\emph{Mais où sont les neiges
d'antan}?\textsuperscript{\protect\hypertarget{12_Chapter_Five__THE_VISION_OF_DEAT.xhtmlux5cux23id_1339}{\protect\hyperlink{23_NOTES.xhtmlux5cux23id_1340}{6}}}\protect\hypertarget{12_Chapter_Five__THE_VISION_OF_DEAT.xhtmlux5cux23id_2957}{\protect\hyperlink{23_NOTES.xhtmlux5cux23id_2958}{†\textsuperscript{4}}}

And soon he garnishes it with irony in the ballade about noblemen where,
while thinking about the kings, poets, and princes of his time, it
occurs to him:

\emph{Hélas! et le bon roy d'Espaigne}

\emph{Duquel je ne sçray pas le
nom?}\textsuperscript{\protect\hypertarget{12_Chapter_Five__THE_VISION_OF_DEAT.xhtmlux5cux23id_1337}{\protect\hyperlink{23_NOTES.xhtmlux5cux23id_1338}{7}}}\protect\hypertarget{12_Chapter_Five__THE_VISION_OF_DEAT.xhtmlux5cux23id_2959}{\protect\hyperlink{23_NOTES.xhtmlux5cux23id_2960}{‡\textsuperscript{5}}}

The brave courtier Olivier de la Marche would not have dared to make
such a joke in his ``Parement et triumphe des dames,'' in
\protect\hypertarget{12_Chapter_Five__THE_VISION_OF_DEAT.xhtmlux5cux23page_159}{}{}which
he thinks about all the dead princesses of his own time in the context
of this same theme.

What is left of all human glory and splendor? Memories, a name. But the
sadness of this thought was not satisfying enough given the need for a
sharp shudder in the face of death. Consequently, the age looks in the
mirror of visible terror, and finds there, in the image of the rotting
corpse, perishability condensed into a shorter frame of time.

The mind of world-denying medieval man had always liked to dwell amidst
dust and worms; in the ecclesiastical tracts about the decay of the
world, all the horrifying ideas about decomposition had already been
evoked. But the elaboration of details only comes later; it is only
towards the end of the fourteenth century that the visual arts take up
the
motif.\textsuperscript{\protect\hypertarget{12_Chapter_Five__THE_VISION_OF_DEAT.xhtmlux5cux23id_1335}{\protect\hyperlink{23_NOTES.xhtmlux5cux23id_1336}{8}}}
A certain degree of skill in realistic expression is required for
dealing properly with this motif in sculpture or painting. This power
was attained around 1400. At the same time, the motif spread from
ecclesiastical to popular literature. Until late in the sixteenth
century, gravestones depict the disgustingly varied notion of the naked
corpse, with cramped hands and feet, gaping mouth, with worms writhing
in the intestines. The mind is over and again invited to dwell on this
frightful image. Is it not strange that they dare not take the further
step of seeing that decay itself will perish and turn into earth and
flowers?

Is it truly pious thinking that entangles itself in this loathing of the
purely earthly side of death? Or is it the reaction of an all too
intense sensuality that can only awaken itself from its intoxication
with life in this manner? Or is it the dread of life that so strongly
permeates the age the mood of disappointment and discouragement of one
who has fought and won and now would prefer a complete surrender to that
which is transcendent, but somehow is still too close to earthly passion
to be able to make that surrender? All these elements of feeling are
united in these expressions of the concept of death.

The fear of life: the denial of beauty and joy because suffering and
pain are bound up with them. There is an astonishing similarity between
the ancient Indian, that is the Buddhist, and the medieval Christian
expressions of this sentiment. There, too, is found the incessant
preoccupation with disgusting age, sickness, and death, there too, the
exaggerated depiction of putrefaction. The naive Indian ascetics even
had their own poetic genre, bîbhatsa-rasa, or the
\protect\hypertarget{12_Chapter_Five__THE_VISION_OF_DEAT.xhtmlux5cux23page_160}{}{}sentiment
of abhorrence, which was divided into three subdivisions depending on
whether the sentiment was caused by disgust, terror, or
lust.\textsuperscript{\protect\hypertarget{12_Chapter_Five__THE_VISION_OF_DEAT.xhtmlux5cux23id_1333}{\protect\hyperlink{23_NOTES.xhtmlux5cux23id_1334}{9}}}
The Christian monk thought he had put it so well when he pointed to the
superficiality of physical beauty. ``The beauty of the body is that of
skin alone. If people could see what is underneath the skin, as it is
said in Boethia that the lynx can do, they would find the sight of woman
abhorrent. Her charm consists of slime and blood, of wetness and gall.
If anyone considers what is hidden in the nostrils and in the throat and
in the belly, he will always think of filth. And if we cannot bring
ourselves to touch slime and filth with our fingertips, how can we bring
ourselves to embrace the dirt bag
itself?\textsuperscript{\protect\hypertarget{12_Chapter_Five__THE_VISION_OF_DEAT.xhtmlux5cux23id_1331}{\protect\hyperlink{23_NOTES.xhtmlux5cux23id_1332}{10}}}

The discouraged refrain of contempt for the world was codified for the
later Middle Ages in many tracts, but above all in that of Pope Innocent
III, \emph{De contemptu mundi}. It is strange that this powerful
statesman, favored by good fortune, holder of the throne of St. Peter,
concerned about so many earthly things and interests and actively
involved in them, could in his earlier years have been the author of
such a scornful view of life. ``Concipit mulier cum immunditia et
fetore, park cum tristitia et dolore, nutrit cum angustia et labore,
custodit cum instantia et
timore.''\textsuperscript{\protect\hypertarget{12_Chapter_Five__THE_VISION_OF_DEAT.xhtmlux5cux23id_1329}{\protect\hyperlink{23_NOTES.xhtmlux5cux23id_1330}{11}}}
(``Woman conceives in impurity and stench. She gives birth in sorrow and
pain. She suckles with strain and effort. She wakes full of dread and
fear.'') O, those laughing joys of motherhood!---``quis unquam vel
unicam diem totam duxit in sua delectatione jucundam .~.~. quern denique
visus vel auditus vel aliquis ictus non offenderit?'' (``Who has ever
spent a single day totally immersed in pleasure .~.~. without being hurt
by the sight of something, the sound of something or the impact of
something?'')\textsuperscript{\protect\hypertarget{12_Chapter_Five__THE_VISION_OF_DEAT.xhtmlux5cux23id_1327}{\protect\hyperlink{23_NOTES.xhtmlux5cux23id_1328}{12}}}
Is this Christian wisdom or the pouting of a spoiled child?

There is, undoubtedly, in all of this a spirit of tremendous materialism
that could not bear the thought of the passing of beauty without
despairing of beauty itself. And one should note how (especially in
literature, less in the fine arts) female beauty in particular was
deplored. Here there is hardly any difference between the religious
admonition to think on death and the fleeting nature of earthly things
and the regret of an aging courtesan over the decay of beauty that she
can no longer offer.

We have first an example in which the edifying admonition is
\protect\hypertarget{12_Chapter_Five__THE_VISION_OF_DEAT.xhtmlux5cux23page_161}{}{}still
in the foreground. In the Celestine monastery in Avignon there existed,
before the Revolution, a wall painting that tradition ascribed to the
artistic founder of the cloister, King René himself. It showed a female
corpse, standing upright, wearing an elegant headdress, wrapped in her
shroud; worms were devouring her body. The first lines of the
inscription read:

\emph{Une fois sur toute femme belle}

\emph{Mais par la mort suis devenue telle}.

\emph{Ma chair estoit très belle, fraische et tendre}

\emph{Or est-elle toute tournée en cendre}.

\emph{Mon corps estoit très plaisant et très gent},

\emph{Je me souloye souvent vestrir de soye},

\emph{Or en droict fault que toute nue je soye}.

\emph{Forrée estoit de gris et de menu vair},

\emph{En grand palais me logeois à mon vueil},

\emph{Or suis logiée en ce petit cercueil}.

\emph{Ma chambre estoit de beaux tapis ornée}

\emph{Or est d'aragnes ma fosse
environée}.\textsuperscript{\protect\hypertarget{12_Chapter_Five__THE_VISION_OF_DEAT.xhtmlux5cux23id_1326}{\protect\hyperlink{23_NOTES.xhtmlux5cux23page_415}{13}}}\emph{\protect\hypertarget{12_Chapter_Five__THE_VISION_OF_DEAT.xhtmlux5cux23id_2961}{\protect\hyperlink{23_NOTES.xhtmlux5cux23id_2962}{*\textsuperscript{6}}}}

That these admonitions had their desired effect is proven by the legend
that arose later that the royal artist himself, the lover of life and
beauty \emph{par excellence}, had looked at his beloved three days after
her burial and had then painted her.

The sentiment moves slightly in the direction of sensuality once the
warning about perishability is not illustrated by the gruesome corpse of
someone else, but when the issue is the bodies of the living, now still
beautiful but soon food for the worms. Olivier de la Marche ends his
didactic allegorical poem about female clothing, ``Le parement et
triumphe des
dames,''\protect\hypertarget{12_Chapter_Five__THE_VISION_OF_DEAT.xhtmlux5cux23id_2963}{\protect\hyperlink{23_NOTES.xhtmlux5cux23id_2964}{†\textsuperscript{7}}}
with death who holds the mirror up to all beauty and conceit:

\emph{\protect\hypertarget{12_Chapter_Five__THE_VISION_OF_DEAT.xhtmlux5cux23page_162}{}{}Ces
doulx regards, ces yeulz faiz pour plaisance},

\emph{Pensez y bien, ilz perdront leur clarté}

\emph{Nez et sourcilz, la bouche d'eloquence}

\emph{Se pourrione} .~.~.
\textsuperscript{\protect\hypertarget{12_Chapter_Five__THE_VISION_OF_DEAT.xhtmlux5cux23id_1324}{\protect\hyperlink{23_NOTES.xhtmlux5cux23id_1325}{14}}}\protect\hypertarget{12_Chapter_Five__THE_VISION_OF_DEAT.xhtmlux5cux23id_2965}{\protect\hyperlink{23_NOTES.xhtmlux5cux23id_2966}{*\textsuperscript{8}}}

So far this is still an honest memento mori, but it edges imperceptibly
into a dispirited, worldly and self-seeking complaint about the
disadvantages of old age:

Se \emph{vous vivez le droit cours de nature}

\emph{Dont LX ans est pour ung bien grant nombre},

\emph{Vostre beaulté changera en laydure},

\emph{Vostre santé en maladie obscure},

\emph{Et ne ferez en ce monde que encombre}.

\emph{Se fille avez, vous luy serez ung umbre},

\emph{Celle sera requise et demandée}

\emph{Et la chascun la mère
habandonnée}.\textsuperscript{\protect\hypertarget{12_Chapter_Five__THE_VISION_OF_DEAT.xhtmlux5cux23id_1322}{\protect\hyperlink{23_NOTES.xhtmlux5cux23id_1323}{15}}}\protect\hypertarget{12_Chapter_Five__THE_VISION_OF_DEAT.xhtmlux5cux23id_2967}{\protect\hyperlink{23_NOTES.xhtmlux5cux23id_2968}{†\textsuperscript{9}}}

Any pious elevated meaning is very remote when Villon composes his
ballades in which ``la belle heaulmière,'' once a famous Parisian
courtesan, compares her formerly irresistible charms with the sad decay
of her aging body:

\emph{Qu'est devenu ce front poly},

\emph{Ces cheveulx blons sourcils voultiz},

\emph{Grant entroeil le regard joly},

\emph{Dont prenoie les plus soubtilz};

\emph{Ce beau nez droit, grant ne petiz},

\emph{Ces petites joinctes oreilles},

\emph{Menton fourchu, cler viz traictiz}

\emph{Et ces belles levres vermeilles?}

. \emph{.~.~. . .~.~. . .~.~. . .~.~. .} .

\emph{\protect\hypertarget{12_Chapter_Five__THE_VISION_OF_DEAT.xhtmlux5cux23page_163}{}{}Le
front ridé, les cheveux gris}

\emph{Les sourcils cheuz, les yeux estains} . .
.\textsuperscript{\protect\hypertarget{12_Chapter_Five__THE_VISION_OF_DEAT.xhtmlux5cux23id_1320}{\protect\hyperlink{23_NOTES.xhtmlux5cux23id_1321}{16}}}\protect\hypertarget{12_Chapter_Five__THE_VISION_OF_DEAT.xhtmlux5cux23id_2969}{\protect\hyperlink{23_NOTES.xhtmlux5cux23id_2970}{*\textsuperscript{10}}}

In one of the poetic books of the southern Buddhists there is a song of
an old pious nun, Ambapâlî, who has the same past as ``la belle
heaulmiére.'' She, too, compares her earlier beauty with disgusting old
age, but she is full of gratitude for the demise of useless
beauty.\textsuperscript{\protect\hypertarget{12_Chapter_Five__THE_VISION_OF_DEAT.xhtmlux5cux23id_1318}{\protect\hyperlink{23_NOTES.xhtmlux5cux23id_1319}{17}}}
But is the distance between this feeling and the preceding as great as
it might seem to us?

The vehement disgust over the decomposition of the body explains the
great significance that people put on the bodies of some saints, such as
that of St. Rosa of Viterbo, which did not decompose. It is one of the
most precious glories of Mary that her body was spared earthly
decomposition by virtue of its Ascension to
heaven.\textsuperscript{\protect\hypertarget{12_Chapter_Five__THE_VISION_OF_DEAT.xhtmlux5cux23id_1316}{\protect\hyperlink{23_NOTES.xhtmlux5cux23id_1317}{18}}}
What is speaking in all this is basically a materialistic spirit that
cannot shake its preoccupation with the body. The same spirit
occasionally reveals itself in the special care with which some bodies
were handled. There was a custom of painting the facial features on the
corpse of a prominent person immediately after death so that no changes
would be noticeable prior to the
funeral.\textsuperscript{\protect\hypertarget{12_Chapter_Five__THE_VISION_OF_DEAT.xhtmlux5cux23id_1314}{\protect\hyperlink{23_NOTES.xhtmlux5cux23id_1315}{19}}}
The body of a preacher of the apostate sect of the Turlupins who had
died in prison prior to the announcement of the verdict upon him, was
kept for fourteen days sealed in chalk so that it could be burned along
with a living
apostate.\textsuperscript{\protect\hypertarget{12_Chapter_Five__THE_VISION_OF_DEAT.xhtmlux5cux23id_1312}{\protect\hyperlink{23_NOTES.xhtmlux5cux23id_1313}{20}}}
The practice of taking the bodies of prominent persons, cutting them up,
and boiling them until the bones separated from the flesh was
widespread. The bones were cleaned and then sent off in a casket for
final burial while the flesh and intestines were buried on the spot. In
the twelfth and thirteenth centuries, this was quite customary in the
case of bishops as well as with a number of
kings.\textsuperscript{\protect\hypertarget{12_Chapter_Five__THE_VISION_OF_DEAT.xhtmlux5cux23id_1310}{\protect\hyperlink{23_NOTES.xhtmlux5cux23id_1311}{21}}}
In 1299 and again in 1300 the practice was most strictly forbidden by
Pope Boniface VIII as a ``detestandae feritatis abusus, quern ex quodam
more horribili nonnuli fideles improvide
prosequuntur.''\protect\hypertarget{12_Chapter_Five__THE_VISION_OF_DEAT.xhtmlux5cux23id_2971}{\protect\hyperlink{23_NOTES.xhtmlux5cux23id_2972}{†\textsuperscript{11}}}
Nevertheless, in the
four\protect\hypertarget{12_Chapter_Five__THE_VISION_OF_DEAT.xhtmlux5cux23page_164}{}{}teenth
century there were many papal dispensations that lifted the prohibition
and in the fifteenth century the custom was still prized by the English
in France. The bodies of Edward of York and Michael de la Pole, count of
Suffolk, the most famous Englishmen to die at Agincourt, were handled in
this
manner.\textsuperscript{\protect\hypertarget{12_Chapter_Five__THE_VISION_OF_DEAT.xhtmlux5cux23id_1308}{\protect\hyperlink{23_NOTES.xhtmlux5cux23id_1309}{22}}}
It happened to Henry V himself, and to William Glasdale, who drowned
during Joan of Arc's liberation of Orléans, and to a nephew of Sir John
Falstaff who fell in the siege of St. Denis in
1435.\textsuperscript{\protect\hypertarget{12_Chapter_Five__THE_VISION_OF_DEAT.xhtmlux5cux23id_1306}{\protect\hyperlink{23_NOTES.xhtmlux5cux23id_1307}{23}}}

In the fourteenth century, the strange word ``macabre'' appeared, or, as
it was originally spelled, ``Macabré.'' ``Je fis Macabré la
dance,''\protect\hypertarget{12_Chapter_Five__THE_VISION_OF_DEAT.xhtmlux5cux23id_2973}{\protect\hyperlink{23_NOTES.xhtmlux5cux23id_2974}{*\textsuperscript{12}}}
says the poet Jean Le Fèvre in 1376. It is a personal name and this
might be the much disputed etymology of the
word.\textsuperscript{\protect\hypertarget{12_Chapter_Five__THE_VISION_OF_DEAT.xhtmlux5cux23id_1304}{\protect\hyperlink{23_NOTES.xhtmlux5cux23id_1305}{24}}}
It is only much later that the adjective is abstracted from \emph{``la
danse macabre''} that has acquired for us such a crisp and particular
nuance of meaning that with it we can label the entire late medieval
vision of death. The motif of death in the form of the ``macabre'' is
primarily found in our times in village cemeteries where one can still
sense its echo in verses and figures. By the end of the Middle Ages,
this notion had become an important cultural conception. There entered
into the realm surrounding the idea of death a new, grippingly fantastic
element, a shiver that arose from the gruesomely conscious realm of
ghostly fear and cold terror. The all-encompassing religious mechanism
immediately turned it into morality by linking it back to the memento
mori, but also made use of the entirely gruesome suggestion that the
ghostly character of the image brought with it.

Around the \emph{danse macabre} are grouped some related images, which,
along with death, are very well suited to frighten and to warn. The
depiction of the three dead men and the three living precedes the image
of the \emph{danse
macabre}.\textsuperscript{\protect\hypertarget{12_Chapter_Five__THE_VISION_OF_DEAT.xhtmlux5cux23id_1302}{\protect\hyperlink{23_NOTES.xhtmlux5cux23id_1303}{25}}}
It had already appeared in French literature in the thirteenth century.
Three young noblemen suddenly meet three ghastly dead men who point to
their own former earthly glory and to the imminent end that awaits the
living. The touching figures in the Campo Santo in Pisa are the earliest
representation of this theme in formal art; the sculptures on the portal
of the Church of the Innocents in Paris where the Duke of Berry had the
topic depicted in 1408 are lost. But miniatures and
\protect\hypertarget{12_Chapter_Five__THE_VISION_OF_DEAT.xhtmlux5cux23page_165}{}{}woodcuts
make this subject a common possession during the fifteenth century and
it is also widespread as wall paintings.

The depiction of the three dead men and the three living provides the
connection between the repugnant image of decay and the thought, made
into an image in the \emph{danse macabre}, that all are equal in death.
The development of this subject in the history of art can only be
mentioned here in passing. France appears to be the country where the
\emph{danse macabre} originates, but how did it come about? Was it
actually acted out or was it an image? It is known that the thesis of
Emile Mâle that the motifs of the pictorial art of the fifteenth century
have their origin in dramatic performances, as a general principle,
cannot withstand its critics. But in respect to the \emph{danse
macabre}, there might be an exception to the rejection of the thesis,
that here the depiction was actually preceded by a performance. In any
case, be it earlier or later, the \emph{danse macabre} was actually
performed as well as painted and depicted in woodcuts. The duke of
Burgundy had it performed in 1449 at his residence in
Bruges.\textsuperscript{\protect\hypertarget{12_Chapter_Five__THE_VISION_OF_DEAT.xhtmlux5cux23id_1300}{\protect\hyperlink{23_NOTES.xhtmlux5cux23id_1301}{26}}}
If we had any idea of the nature of such a performance, of the colors,
the movements, the play of light and shade over the dancers, we would be
much better able to understand the strong sense of shock in the minds of
the onlookers than we are from the woodcuts of Guyot Marchant and
Holbein.

The woodcuts
(\protect\hyperlink{20_ILLUSTRATIONS_FOLLOW_PAGE.xhtmlux5cux23id_4}{plate
3}) with which the Parisian printer Guyot Marchant illustrated the first
edition of the \emph{Danse Macabre} in 1485 were almost certainly
borrowed from the most famous of the \emph{danses macabres}, the 1424
wall painting done in the Hall of Columns in the Cemetery of the
Innocents in Paris. The inscriptions beneath the paintings that are
preserved in the 1485 edition may perhaps be traced back to the lost
poem of Jean Le Fèvre that in turn may have followed a Latin original.
Be that as it may, the \emph{danse macabre} in the Cemetery of the
Innocents disappeared during the seventeenth century when the hall was
torn down. It was the most popular depiction of decay known to the
Middle Ages. Day by day, thousands viewed the simple figures at the
Cemetery of the Innocents, which served as a strange and macabre meeting
place, and read the easily comprehended verses. Each strophe concluded
with a well-known proverb. The people found solace in the equality of
all in decay and shivered at the prospect of their own end. Nowhere else
was ape-like death so much in his own place. Grimly, with the gait of an
old, stiff dancing master, he led the pope, the
em\protect\hypertarget{12_Chapter_Five__THE_VISION_OF_DEAT.xhtmlux5cux23page_166}{}{}peror,
the nobleman, the day laborer, the monk, the small child, the fool, and
all the other professions and estates away. Do the woodcuts of 1485 come
anywhere close to the impact of the famous wall paintings? Probably not;
the dress of the figures shows that they are not true copies of the
painting of 1424. To get a true impression of the \emph{danse macabre}
of the Cemetery of the Innocents, one should see those from the church
of La
Chaise-Dieu,\textsuperscript{\protect\hypertarget{12_Chapter_Five__THE_VISION_OF_DEAT.xhtmlux5cux23id_1298}{\protect\hyperlink{23_NOTES.xhtmlux5cux23id_1299}{27}}}
where the ghostly element is further enhanced by the half-finished state
of the painting.

The corpse, who recurs forty times leading away the living, is actually
not Death, but rather a dead man. The verses call the figure \emph{Le
Mort} (in the \emph{danse macabre} of women, \emph{La Morte)}; it is a
dance of the dead, not of
Death.\textsuperscript{\protect\hypertarget{12_Chapter_Five__THE_VISION_OF_DEAT.xhtmlux5cux23id_1296}{\protect\hyperlink{23_NOTES.xhtmlux5cux23id_1297}{28}}}
Furthermore, there is no skeleton, but a body not yet entirely stripped
of its flesh, with its abdomen slit open. Only around 1500 does the
figure of the great dancer become the skeleton we know from Holbein. In
the meantime, the notion evolves of an unknown deadly
\emph{dubbelganger}\textsuperscript{\protect\hypertarget{12_Chapter_Five__THE_VISION_OF_DEAT.xhtmlux5cux23id_1294}{\protect\hyperlink{23_NOTES.xhtmlux5cux23id_1295}{29}}}
who personally ends life. ``Yo so la Muerte cierta a todas criaturas,''
\protect\hypertarget{12_Chapter_Five__THE_VISION_OF_DEAT.xhtmlux5cux23id_2975}{\protect\hyperlink{23_NOTES.xhtmlux5cux23id_2976}{*\textsuperscript{13}}}
begins an impressive Spanish \emph{danse macabre} from the end of the
fifteenth
century.\textsuperscript{\protect\hypertarget{12_Chapter_Five__THE_VISION_OF_DEAT.xhtmlux5cux23id_1292}{\protect\hyperlink{23_NOTES.xhtmlux5cux23id_1293}{30}}}
In the older \emph{danse macabre} the untiring dancer is still the
living person himself as he will be in the near future, a frightening
duplication of his own person, the image that he sees in the mirror.
Not, as some would have it, an earlier person of the same rank and
status who had died. Here is the point: you, yourself, are in the
\emph{danse macabre}, and this is what bestows on it its gruesome
powers.

On the fresco that graced the vault of the tomb monument of King René
and his Queen Isabella in the Cathedral of Angers, it was actually the
king himself who was depicted. One could see a skeleton there (had this
earlier been a corpse too?) in a long coat, sitting on a golden throne
and kicking away with his feet mitre, crown, orb, and books. The head
was resting on a shriveled hand that tried to support a sagging
crown.\textsuperscript{\protect\hypertarget{12_Chapter_Five__THE_VISION_OF_DEAT.xhtmlux5cux23id_1290}{\protect\hyperlink{23_NOTES.xhtmlux5cux23id_1291}{31}}}

The original \emph{danse macabre} depicted only men. The intent to tie
the admonition of the perishability and conceit of earthly matters to
the lesson of social equality naturally moves men, as the holders of
social professions and dignities, to the forefront. The \emph{danse
macabre} was not only a pious admonition, it was also social satire and
\protect\hypertarget{12_Chapter_Five__THE_VISION_OF_DEAT.xhtmlux5cux23page_167}{}{}the
verses that accompanied it have a faint irony. The same Guyot Marchant
published as a continuation of his earlier edition a \emph{danse
macabre} of women with verses by Martial d'Auvergne. The unknown
engraver of the woodcuts was not up to the standard of the earlier
edition; his only contribution was the gruesome figure of a skeleton
around whose head a few sparse strands of woman's hair still flutter. In
this female version, the sensual theme of beauty that turns into
corruption is immediately struck. How could it be otherwise? There were
not forty occupations and estates for women. Along with the most noble
estates, such as queen, noblewoman, and so forth, there are a few of the
spiritual functions or estates such as abbess and nun, and with a few
professions such as merchant, baker, etc., the list exhausts itself.
Otherwise the list has to view women in the temporary stages of their
womanly lives as maiden, beloved, bride, newlywed, and expectant. And so
here again, the theme of past or never-achieved joy or beauty sounds yet
more shrilly.

One picture was still lacking in the terrifying depiction of the act of
dying, and that was of the hour of death itself. The horror of this hour
could not be brought to the mind in a more dreadful image than that of
the raising of Lazarus. After his resurrection he had known nothing but
a sorrowful dread of the death that he had suffered once before. If the
righteous must feel such fear, what of the
sinner?\textsuperscript{\protect\hypertarget{12_Chapter_Five__THE_VISION_OF_DEAT.xhtmlux5cux23id_1288}{\protect\hyperlink{23_NOTES.xhtmlux5cux23id_1289}{32}}}
The vision of the death struggle was the first of the Four Last Things,
the ``quattour hominum novissima,'' upon which it behooved man to think:
death, final judgment, hell, and heaven. They were, as such, also part
of the vision of the beyond. In this instance, in a preliminary way,
only the issue of the death of the body itself is raised. Closely
related to the theme of the Four Last Things was the \emph{Ars
moriendi}, a creation of the fifteenth century that also gained a wide
circulation as part of pious thought through printing and the woodcut.
It dealt with the five temptations with which the devil snared the
dying: doubt of faith, desperation over one's sins, attachment to
earthly goods, desperation about one's own sufferings, and finally
conceit about one's own virtue. Always an angel appears to fend off
Satan's snares with his consolation. The description of the death
struggle is an old subject of spiritual literature. One sees the same
images recur in it over and over
again.\textsuperscript{\protect\hypertarget{12_Chapter_Five__THE_VISION_OF_DEAT.xhtmlux5cux23id_1286}{\protect\hyperlink{23_NOTES.xhtmlux5cux23id_1287}{33}}}

In a detailed poem entitled ``Le pas de la
Mort,''\textsuperscript{\protect\hypertarget{12_Chapter_Five__THE_VISION_OF_DEAT.xhtmlux5cux23id_1284}{\protect\hyperlink{23_NOTES.xhtmlux5cux23id_1285}{34}}}
Chastellain
\protect\hypertarget{12_Chapter_Five__THE_VISION_OF_DEAT.xhtmlux5cux23page_168}{}{}has
brought all of these motifs together. He begins with a moving narrative
that, even given the solemn verbosity characteristic of this author,
does not fail to have its effect. His dying beloved calls him to herself
and in a broken voice says:

\emph{Mon amy, regardez ma face}.

\emph{Voyez que fait dolante mort}

\emph{Et ne l'oubliez désormais};

\emph{C'est celle qu'aimiez si fort};

\emph{Et ce corps vostre, vil et ort},

\emph{Vous perderez pour un jamais};

\emph{Ce sera puant entremais}

\emph{A la terre et à la vermine};

\emph{Dure mort toute beauté fine.
\protect\hypertarget{12_Chapter_Five__THE_VISION_OF_DEAT.xhtmlux5cux23id_2977}{\protect\hyperlink{23_NOTES.xhtmlux5cux23id_2978}{*\textsuperscript{14}}}}

This induces the poet to compose a Mirror of Death. First, he works out
the theme ``Where are all the great ones of the earth now?'' at far too
great a length, his style a little schoolmasterish without any of the
easy sadness of Villon. This is followed by something like a first
attempt at a \emph{danse macabre}, but without vigor or imagination. At
the end he puts in the \emph{Ars moriendi} in verse form. Here is his
description of the death struggle:

\emph{Il n'a membre ne facture}

\emph{Qui ne sente sa pourreture}.

\emph{Avant que l'esperit soit hors},

\emph{Le coeur qui veult crevier au corps}

\emph{Haulce et souliève la poitrine}

\emph{Qui se veult joindre à son eschine}.

---\emph{La face est tainte et apalie},

\emph{Et les yeux treilliés en la teste}.

\emph{La parolle luy est faillie},

\emph{Car la langue au palais se lie}.

\emph{Le poulx tressault et sy halette}.

. \emph{.~.~. . .~.~. . .~.~. . .~.~. .} .

\emph{\protect\hypertarget{12_Chapter_Five__THE_VISION_OF_DEAT.xhtmlux5cux23page_169}{}{}Les
os desjoindent à tous lez};

\emph{Il n'a nerf qu'au rompre ne
tende.\textsuperscript{\protect\hypertarget{12_Chapter_Five__THE_VISION_OF_DEAT.xhtmlux5cux23id_1282}{\protect\hyperlink{23_NOTES.xhtmlux5cux23id_1283}{35}}}\protect\hypertarget{12_Chapter_Five__THE_VISION_OF_DEAT.xhtmlux5cux23id_2979}{\protect\hyperlink{23_NOTES.xhtmlux5cux23id_2980}{*\textsuperscript{15}}}}

Villon puts all that in half a verse that is much more moving, but one
recognizes the common
model.\textsuperscript{\protect\hypertarget{12_Chapter_Five__THE_VISION_OF_DEAT.xhtmlux5cux23id_1280}{\protect\hyperlink{23_NOTES.xhtmlux5cux23id_1281}{36}}}

\emph{La mort le fait fremir, pallir},

\emph{Le nez courber, les vaines tendre}.

\emph{Le col enfler, la chair mollir},

\emph{Joinctes et nerfs croistre et
estendre}.\protect\hypertarget{12_Chapter_Five__THE_VISION_OF_DEAT.xhtmlux5cux23id_2981}{\protect\hyperlink{23_NOTES.xhtmlux5cux23id_2982}{†\textsuperscript{16}}}

And then again the sensual element that runs through all these
terrifying notions:

\emph{Corps femenin, qui tant est tendre},

\emph{Poly, souef, si precieux},

\emph{Te fauldra il ces maulx attendre?}

\emph{Oy, ou tout vif aller es
cieulx}.\protect\hypertarget{12_Chapter_Five__THE_VISION_OF_DEAT.xhtmlux5cux23id_2983}{\protect\hyperlink{23_NOTES.xhtmlux5cux23id_2984}{‡\textsuperscript{17}}}

Nowhere else was everything concerning death more completely brought
together before the eyes than in the Cemetery of the Innocents in Paris.
There one experienced the macabre to the fullest; everything worked
together to provide the somber holiness and colorful forms that the late
Middle Ages craved so much. The saints to whom the church and churchyard
were dedicated, the innocent children who were butchered in place of
Christ, evoked with their lamentable martyrdom, the bloody pity in which
the age indulged. It is precisely in this century that the veneration of
the Holy Children became very popular. But there was more than one
\emph{relique} of the boys of Bethlehem there. Louis XI had given to the
church
\protect\hypertarget{12_Chapter_Five__THE_VISION_OF_DEAT.xhtmlux5cux23page_170}{}{}that
he had dedicated ``un Innocent
entier''\textsuperscript{\protect\hypertarget{12_Chapter_Five__THE_VISION_OF_DEAT.xhtmlux5cux23id_1278}{\protect\hyperlink{23_NOTES.xhtmlux5cux23id_1279}{37}}}
in a great crystal shrine. People were fond of coming to the churchyard
to take their ease. A bishop of Paris had some earth from the Churchyard
of the Innocents placed in his grave when it happened that he could not
be buried
there.\textsuperscript{\protect\hypertarget{12_Chapter_Five__THE_VISION_OF_DEAT.xhtmlux5cux23id_1276}{\protect\hyperlink{23_NOTES.xhtmlux5cux23id_1277}{38}}}
The rich and the poor rested there side by side, but not for long, as
the burial ground, which twenty churches had the right to use, was in so
much demand that after a few years the bodies were exhumed and the
tombstones sold. It was said that there a corpse would decompose down to
the bare bones in about nine
days.\textsuperscript{\protect\hypertarget{12_Chapter_Five__THE_VISION_OF_DEAT.xhtmlux5cux23id_1274}{\protect\hyperlink{23_NOTES.xhtmlux5cux23id_1275}{39}}}
The skulls and bones were then piled up in the bone chambers above the
Hall of Columns that surrounded the cemetery on three sides. They lay
there in their thousands, open and exposed, preaching the lesson of the
equality of all. Beneath the arcade, the same lesson could be seen and
read in the paintings and verses of the \emph{danse macabre}. For the
construction of the ``beaux charniers''
\protect\hypertarget{12_Chapter_Five__THE_VISION_OF_DEAT.xhtmlux5cux23id_2985}{\protect\hyperlink{23_NOTES.xhtmlux5cux23id_2986}{*\textsuperscript{18}}}
the noble Boucicaut, among others, had made
contributions.\textsuperscript{\protect\hypertarget{12_Chapter_Five__THE_VISION_OF_DEAT.xhtmlux5cux23id_41}{\protect\hyperlink{23_NOTES.xhtmlux5cux23id_42}{40}}}
On the portal of the church, the duke of Berry, who wished to be buried
there, had the figures of the three living and the three dead men
sculpted. During the sixteenth century the large statue of Death was
still standing in the cemetery. In the Louvre now, it is the sole
surviving remnant of all that was assembled there
(\protect\hyperlink{20_ILLUSTRATIONS_FOLLOW_PAGE.xhtmlux5cux23id_2297}{plate
4}).

For the people of the fifteenth century, this place was what the
melancholy palais royal was to the people of 1789. There amid the
continuous burials and exhumations was a promenade and a meeting place.
Small shops were found near the bare bones and easy women under the
arcades. There was even an aged female recluse who lived in the side of
the church. Sometimes a mendicant monk preached in that place that was
itself a sermon in the medieval style. Many times processions of
children assembled there; 12,500 says the Burgher of Paris, all with
candles. They marched from the Innocents to Notre Dame and back. Even
festivities were held
there.\textsuperscript{\protect\hypertarget{12_Chapter_Five__THE_VISION_OF_DEAT.xhtmlux5cux23id_1273}{\protect\hyperlink{23_NOTES.xhtmlux5cux23page_416}{41}}}
So much had the dreadful become the familiar.

In the drive to create an unmitigated depiction of death, in which
everything intangible had to be abandoned, only the coarser aspects of
death made it into consciousness. The macabre vision of death lacked
everything elegiac as well as everything tender. And at root, it is a
very earthly, self-preoccupied attitude towards death. It does
\protect\hypertarget{12_Chapter_Five__THE_VISION_OF_DEAT.xhtmlux5cux23page_171}{}{}not
deal with sadness over the loss of those beloved, but rather with regret
about one's own approaching death, which can be seen only as misfortune
and terror. There is no thought given to death as consolation, to the
end of suffering, eternal rest, the task completed or broken off, no
tender memories, no surrender. Nothing of the ``divine depth of
sorrow.'' Only once can there be heard a softer sound. In the
\emph{danse macabre}, Death addresses the day laborer as follows:

\emph{Laboureur qui en soing et painne}

\emph{Avez vescu tout vostre temps},

\emph{Morir fault, e'est chose certainne},

\emph{Reculler n'y vault ne contens}

\emph{De mort devez estre contens}

\emph{Car de grant soussy vous delivre} .~.~.
\protect\hypertarget{12_Chapter_Five__THE_VISION_OF_DEAT.xhtmlux5cux23id_2987}{\protect\hyperlink{23_NOTES.xhtmlux5cux23id_2988}{*\textsuperscript{19}}}

But the laborer mourns the life that he had often wished would come to
an end.

Martial d'Auvergne in his \emph{danse macabre} of women has the little
girl call out to her mother, take care of my doll, my dice, and my
beautiful dress! The touching accents of childhood are extraordinarily
rare in the literature of the late Middle Ages. There was no room for
them in the weighty rigidity of the grand style. Neither churchly or
worldly literature really knew the child. When Antoine de la Salle in
``Le
Reconfort''\textsuperscript{\protect\hypertarget{12_Chapter_Five__THE_VISION_OF_DEAT.xhtmlux5cux23id_1271}{\protect\hyperlink{23_NOTES.xhtmlux5cux23id_1272}{42}}}
seeks to comfort a noblewoman over the loss of her little son, he knows
no better way to do so than to tell the story of a boy who lost his
young life in an even more cruel way; he died as a hostage. He has
nothing to offer her to allay her pain other than the lesson of not
attaching oneself to anything earthly, but then continues with that
story that we know as the fairy tale of the death shroud. The tale of
the dead child who comes to its mother and begs her not to cry anymore
in order that its shroud might dry. And here is suddenly a much more
tender single note than is heard in the memento mori that is sung with a
thousand notes. Is it possible that folktale and folk song during all
\protect\hypertarget{12_Chapter_Five__THE_VISION_OF_DEAT.xhtmlux5cux23page_172}{}{}those
centuries knew all kinds of emotions well that literature hardly knew at
all?

Ecclesiastical thought of the late Middle Ages knew only the two
extremes: the lament over perishability, over the end of power, glory,
and joy, over the decay of beauty, and, on the other hand, jubilation
over the saved soul in its state of bliss. Everything in between was
unexpressed. In the fixed representation of the \emph{danse macabre} and
the gruesome skeleton, the living emotions are ossified.
