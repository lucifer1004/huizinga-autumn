\chapter{THE PASSIONATE INTENSITY OF LIFE}

WHEN THE WORLD WAS HALF A THOUSAND YEARS younger all events had much
sharper outlines than now. The distance between sadness and joy, between
good and bad fortune, seemed to be much greater than for us; every
experience had that degree of directness and absoluteness that joy and
sadness still have in the mind of a child. Every event, every deed was
defined in given and expressive forms and was in accord with the
solemnity of a tight, invariable life style. The great events of human
life---birth, marriage, death---by virtue of the sacraments, basked in
the radiance of the divine mystery. But even the lesser events---a
journey, labor, a visit---were accompanied by a multitude of blessings,
ceremonies, sayings, and conventions.

There was less relief available for misfortune and for sickness; they
came in a more fearful and more painful way. Sickness contrasted more
strongly with health. The cutting cold and the dreaded darkness of
winter were more concrete evils. Honor and wealth were enjoyed more
fervently and greedily because they contrasted still more than now with
lamentable poverty. A fur-lined robe of office, a bright fire in the
oven, drink and jest, and a soft bed still possessed that high value for
enjoyment that perhaps the English novel, in describing the joy of life,
has affirmed over the longest period of time. In short, all things in
life had about them something glitteringly and cruelly public. The
lepers, shaking their rattles and holding processions, put their
deformities openly on display. Every estate, order, and craft could be
recognized by its dress. The notables, never appearing without the
ostentatious display of their weapons and liveried servants, inspired
awe and envy. The administration of justice, the sales of goods,
weddings and funerals---all announced themselves through processions,
shouts, lamentations and music. The lover carried the emblem of his
lady, the member
\protect\hypertarget{08_Chapter_One__THE_PASSIONATE_INTE.xhtmlux5cux23page_2}{}{}the
insignia of his fraternity, the party the colors and coat of arms of its
lord.

In their external appearance, too, town and countryside displayed the
same contrast and color. The city did not dissipate, as do our cities,
into carelessly fashioned, ugly factories and monotonous country homes,
but, enclosed by its walls, presented a completely rounded picture that
included its innumerable protruding towers. No matter how high and
weighty the stone houses of the noblemen or merchants may have been,
churches with their proudly rising masses of stone, dominated the city
silhouettes.

Just as the contrast between summer and winter was stronger then than in
our present lives, so was the difference between light and dark, quiet
and noise. The modern city hardly knows pure darkness or true silence
anymore, nor does it know the effect of a single small light or that of
a lonely distant shout.

From the continuing contrast, from the colorful forms with which every
phenomenon forced itself on the mind, daily life received the kind of
impulses and passionate suggestions that is revealed in the vacillating
moods of unrefined exuberance, sudden cruelty, and tender emotions
between which the life of the medieval city was suspended.

But one sound always rose above the clamor of busy life and, no matter
how much of a tintinnabulation, was never confused with other noises,
and, for a moment, lifted everything into an ordered sphere: that of the
bells. The bells acted in daily life like concerned good spirits who,
with their familiar voices, proclaimed sadness or joy, calm or unrest,
assembly or exhortation. People knew them by familiar names: Fat
Jacqueline, Bell Roelant; everyone knew their individual tones and
instantly recognized their meaning. People never became indifferent to
these sounds, no matter how overused they were. During the notorious
duel between two burghers of Valenciennes in 1455 that kept the city and
the entire court of Burgundy in extraordinary suspense, the great bell
sounded as long as the fight lasted, ``laquelle fait hideux a
oyr''\protect\hypertarget{08_Chapter_One__THE_PASSIONATE_INTE.xhtmlux5cux23id_2303}{\protect\hyperlink{23_NOTES.xhtmlux5cux23id_2304}{*\textsuperscript{11}}}
says
Chastellain,\textsuperscript{\protect\hypertarget{08_Chapter_One__THE_PASSIONATE_INTE.xhtmlux5cux23id_2226}{\protect\hyperlink{23_NOTES.xhtmlux5cux23page_398}{2}}}
``Sonner l'effroy,'' ``faire l'effroy ``was what the ringing of the
alarm bell was
called.\textsuperscript{\protect\hypertarget{08_Chapter_One__THE_PASSIONATE_INTE.xhtmlux5cux23id_2224}{\protect\hyperlink{23_NOTES.xhtmlux5cux23id_2225}{3}}}
How deafening the sound must have been when the bells of all the
churches and cloisters of Paris pealed all day, or even all night,
because a pope had been
\protect\hypertarget{08_Chapter_One__THE_PASSIONATE_INTE.xhtmlux5cux23page_3}{}{}elected
who was to end the schism or because peace had been arranged between
Burgundy and
Armagnac.\textsuperscript{\protect\hypertarget{08_Chapter_One__THE_PASSIONATE_INTE.xhtmlux5cux23id_2222}{\protect\hyperlink{23_NOTES.xhtmlux5cux23id_2223}{4}}}

Processions must have also been deeply moving. During sad times---and
these came often---they could occasionally take place day after day even
for weeks on end. In 1412, when the fatal conflict between the houses of
Orléans and Burgundy had finally led to open civil war, King Charles VI
seized the oriflamme so that he and John the Fearless could fight
against the Armagnacs, who, by virtue of their alliance with England,
had become traitors to their country. Daily processions were ordered to
be held in Paris as long as the king was on foreign soil. They continued
from the end of May into July and involved ever different groups, orders
or guilds, ever different routes and ever different relics: ``les plus
piteuses processions qui oncques eussent été veues de aage de
homme.''\protect\hypertarget{08_Chapter_One__THE_PASSIONATE_INTE.xhtmlux5cux23id_2252}{\protect\hyperlink{23_NOTES.xhtmlux5cux23id_2251}{*\textsuperscript{2}}}
All were barefoot with empty stomachs, members of parliament and poor
burghers alike; every one who was able carried a candle or a torch.
There were always many small children with them. Even the poor country
folk from the villages around Paris came running on bare feet.
Processions were joined or watched, ``en grant pleur, en grant lermes,
en grant
devocion.''\protect\hypertarget{08_Chapter_One__THE_PASSIONATE_INTE.xhtmlux5cux23id_2250}{\protect\hyperlink{23_NOTES.xhtmlux5cux23id_2249}{†\textsuperscript{3}}}
And heavy rain fell almost constantly during the entire
period.\textsuperscript{\protect\hypertarget{08_Chapter_One__THE_PASSIONATE_INTE.xhtmlux5cux23id_2220}{\protect\hyperlink{23_NOTES.xhtmlux5cux23id_2221}{5}}}

Then there were the princely entry processions prepared with all the
varied formal skills at the disposal of the main actors. And, with
uninterrupted frequency, there were executions. The gruesome fascination
and coarse compassion stirred at the place of execution became an
important element in the spiritual nourishment of the people. For
dealing with vicious robbers and murderers the courts invented terrible
punishments: in Brussels a young arsonist and murderer was tied with a
chain so that he could move in a circle about a stake surrounded by
burning bundles of fagots. He introduced himself to the people in moving
words as a warning example: ``et tellement fit attendrir les coeurs que
tout le monde fondoit en larmes de compassion.'' ``Et fut sa fin
reccommandée la plus belle que l'on avait oncques
vue.''\textsuperscript{\protect\hypertarget{08_Chapter_One__THE_PASSIONATE_INTE.xhtmlux5cux23id_2218}{\protect\hyperlink{23_NOTES.xhtmlux5cux23id_2219}{6}}}\protect\hypertarget{08_Chapter_One__THE_PASSIONATE_INTE.xhtmlux5cux23id_2256}{\protect\hyperlink{23_NOTES.xhtmlux5cux23id_2255}{‡\textsuperscript{4}}}
During the Burgundian reign of terror in Paris, Messire Nansart du Bois,
an Armagnac,
\protect\hypertarget{08_Chapter_One__THE_PASSIONATE_INTE.xhtmlux5cux23page_4}{}{}was
beheaded. Not only did he grant forgiveness to the executioner, who, as
was customary, requested it, but he even asked to be kissed by him.
``Foison de peuple y avoit, qui quasi tous ploroient à chaudes
larmes.''\textsuperscript{\protect\hypertarget{08_Chapter_One__THE_PASSIONATE_INTE.xhtmlux5cux23id_2216}{\protect\hyperlink{23_NOTES.xhtmlux5cux23id_2217}{7}}}\protect\hypertarget{08_Chapter_One__THE_PASSIONATE_INTE.xhtmlux5cux23id_2254}{\protect\hyperlink{23_NOTES.xhtmlux5cux23id_2253}{*\textsuperscript{5}}}
Frequently the sacrificial victims were great lords; in those cases the
people had the even greater satisfaction of witnessing stern justice and
a more forceful warning about the insecurity of high position than would
be conveyed by a painting or a \emph{danse
macabre}.\textsuperscript{\protect\hypertarget{08_Chapter_One__THE_PASSIONATE_INTE.xhtmlux5cux23id_2214}{\protect\hyperlink{23_NOTES.xhtmlux5cux23id_2215}{8}}}
The authorities took pains that nothing was lacking in the impression
the spectacle made. The nobles took their last walk bedecked in the
symbols of their greatness. Jean de Montaigu, grand maitre d'hotel of
the king and a victim of the hatred of John the Fearless, travels to the
gallows seated high on top of a cart. Two trumpeters precede him. He is
dressed in his robes of state, cap, vest, and pants---half white, half
red---with golden spurs on his feet. The beheaded body was left hanging
on the gallows still wearing those golden spurs. The wealthy canon
Nicholas d'Orgemont---who fell victim to the vendetta of the Armagnacs
in 1416---was carried through Paris on a garbage cart, clad in a wide
purple cloak and cap of the same color to witness the execution of two
of his comrades before he was led away to lifelong captivity: ``au pain
de doleur et à eaue
d'angoisse.''\protect\hypertarget{08_Chapter_One__THE_PASSIONATE_INTE.xhtmlux5cux23id_2257}{\protect\hyperlink{23_NOTES.xhtmlux5cux23id_2258}{†\textsuperscript{6}}}
The head of Maître Oudart de Bussy, who had turned down a place in
parliament, was exhumed by special order of Louis XI and, dressed with a
crimson, fur-lined hood, ``selon la mode des conseillers de
parlement,''\protect\hypertarget{08_Chapter_One__THE_PASSIONATE_INTE.xhtmlux5cux23id_2259}{\protect\hyperlink{23_NOTES.xhtmlux5cux23id_2261}{‡\textsuperscript{7}}}
was put on display with an attached explanatory poem in the town square
of Hesdin. The king himself writes about this case with grim
humor.\textsuperscript{\protect\hypertarget{08_Chapter_One__THE_PASSIONATE_INTE.xhtmlux5cux23id_2212}{\protect\hyperlink{23_NOTES.xhtmlux5cux23id_2213}{9}}}

Rarer than the processions and executions were the sermons given by
itinerant preachers who came, from time to time, to stir the people with
their words. We, readers of newspapers, can hardly imagine anymore the
tremendous impact of the spoken word on naive and ignorant minds. The
popular preacher Brother Richard, who may have served Jeanne d'Arc as
father confessor, preached in Paris in 1429 for ten days running. He
spoke from five until ten
\protect\hypertarget{08_Chapter_One__THE_PASSIONATE_INTE.xhtmlux5cux23page_5}{}{}or
eleven o'clock in the morning in the Cemetery of the Innocents---where
the famous \emph{danse macabre} had been painted---with his back to the
bone chambers where skulls were piled up above the vaulted walkways to
be viewed by the visitors. When he informed his audience after his tenth
sermon that it would have to be his last since he had not received
permission for any more, ``les gens grans et petiz plouroient si
piteusement et si fondement, comme s'ilz veissent porter en terre leurs
meilleurs amis, et lui
aussi.''\protect\hypertarget{08_Chapter_One__THE_PASSIONATE_INTE.xhtmlux5cux23id_2260}{\protect\hyperlink{23_NOTES.xhtmlux5cux23id_2263}{*\textsuperscript{8}}}
When he finally leaves Paris, the people believe that the next Sunday he
will still preach at St. Denis; a large number, perhaps as many as six
thousand, according to the Burgher of Paris, leave the city on Saturday
evening and spend the night out in the fields in order to secure good
places.\textsuperscript{\protect\hypertarget{08_Chapter_One__THE_PASSIONATE_INTE.xhtmlux5cux23id_2210}{\protect\hyperlink{23_NOTES.xhtmlux5cux23id_2211}{10}}}

Antoine Fradin, a Franciscan, was also prohibited from preaching in
Paris, because he railed against evil government. But this is precisely
what made him so beloved by the people. They guarded him day and night
in the monastery of the Cordeliers; the women stood watch with their
ammunition of ashes and stones ready. People laughed at the proclamation
prohibiting the watch: the king knows nothing about it! When Fradin is
finally banned and has to leave the city, the people give him an escort,
``crians et soupirans moult fort son
departement.''\textsuperscript{\protect\hypertarget{08_Chapter_One__THE_PASSIONATE_INTE.xhtmlux5cux23id_2208}{\protect\hyperlink{23_NOTES.xhtmlux5cux23id_2209}{11}}}\protect\hypertarget{08_Chapter_One__THE_PASSIONATE_INTE.xhtmlux5cux23id_2262}{\protect\hyperlink{23_NOTES.xhtmlux5cux23id_2264}{†\textsuperscript{9}}}

In all cities where the saintly Dominican Vincent Ferrer comes to
preach, the people, the magistrates, the clergy---including bishops and
prelates---go out to welcome him, singing his praises. He travels with a
large numbers of supporters, who, every evening after sunset, go on
processions with flagellations and songs. In every town he is joined by
new followers. He has carefully arranged for the food and lodging of all
his companions by employing men of spotless reputation as his
quartermasters. Numerous priests from different orders travel with him
so that they can assist him in taking confessions and celebrating mass.
A few notaries accompany him to record the legal reconciliations that
the holy preacher manages to arrange wherever he goes. When he preaches,
a wooden frame has to protect him and his entourage against the
\protect\hypertarget{08_Chapter_One__THE_PASSIONATE_INTE.xhtmlux5cux23page_6}{}{}throngs
who want to kiss his hand or his gown. Work comes to a standstill as
long as he speaks. It was a rare occasion when he failed to move his
audience to tears, and when he spoke of Judgment Day and the pains of
hell or of the sufferings of the Lord, he, just as his audience, broke
into such great tears that he had to remain silent, for a time, until
the weeping had stopped. The penitents fell to their knees before all
the onlookers to tearfully confess their great
sins.\textsuperscript{\protect\hypertarget{08_Chapter_One__THE_PASSIONATE_INTE.xhtmlux5cux23id_2206}{\protect\hyperlink{23_NOTES.xhtmlux5cux23id_2207}{12}}}
When the famous Olivier Maillard gave the Lenten sermon at Orléans in
1485, so many people climbed on the roofs of the houses that the roofers
submitted claims for sixty-four days of repair
work.\textsuperscript{\protect\hypertarget{08_Chapter_One__THE_PASSIONATE_INTE.xhtmlux5cux23id_2204}{\protect\hyperlink{23_NOTES.xhtmlux5cux23id_2205}{13}}}

All this has the atmosphere of the English-American revivals or of the
Salvation Army, but boundlessly extended and much more publicly exposed.
There is no reason to suspect that the descriptions of Ferrer's impact
are pious exaggerations by his biographers. The sober and dry Monstrelet
describes in almost the same manner the impact of the sermons of a
certain Brother Thomas---claiming to be a Carmelite, but later found to
be an imposter---in northern France and Flanders in 1498. He, too, was
escorted into the city by the magistrate while nobles held the reins of
his mules; and for his sake many, among them notables whom Monstrelet
identifies by name, left home and servants to follow him wherever he
went. The prominent burghers erected high pulpits for him and draped
them with the most expensive tapestries they could find.

Next to the popular preacher's accounts of the Passion and the Last
Things, his attacks on luxury and vanity deeply moved his listeners. The
people, Monstrelet writes, were particularly grateful to and fond of
Brother Thomas because he attacked ostentation and displays of vanity
and especially because he heaped criticism on nobility and clergy. He
liked to set small boys (with the promise of indulgences, claims
Monstrelet) on those noble ladies who ventured among the congregation
wearing their high coiffures, crying ``au hennin! au
hennin!''\textsuperscript{\protect\hypertarget{08_Chapter_One__THE_PASSIONATE_INTE.xhtmlux5cux23id_2202}{\protect\hyperlink{23_NOTES.xhtmlux5cux23id_2203}{14}}}
so that women during the entire period no longer dared to wear hennins
and began to wear hoods like the
Beguines.\textsuperscript{\protect\hypertarget{08_Chapter_One__THE_PASSIONATE_INTE.xhtmlux5cux23id_2200}{\protect\hyperlink{23_NOTES.xhtmlux5cux23id_2201}{15}}}
``Mais à l'exemple du lymeçon,'' says the faithful chronicler, ``lequel
quand on passe près de luy retrait ses cornes par dedens et quand il ne
ot plus riens les reboute dehors, ainsy firent ycelles. Car en assez
brief terme après que ledit prescheur se fust départy du pays, elles
mesmes recommencèrent comme devant et
\protect\hypertarget{08_Chapter_One__THE_PASSIONATE_INTE.xhtmlux5cux23page_7}{}{}oublièrent
sa doctrine, et reprinrent petit à petit leur viel estat, tel ou plus
grant qu'elles avoient accoustumé de
porter.''\textsuperscript{\protect\hypertarget{08_Chapter_One__THE_PASSIONATE_INTE.xhtmlux5cux23id_2198}{\protect\hyperlink{23_NOTES.xhtmlux5cux23id_2199}{16}}}\protect\hypertarget{08_Chapter_One__THE_PASSIONATE_INTE.xhtmlux5cux23id_2267}{\protect\hyperlink{23_NOTES.xhtmlux5cux23id_2266}{*\textsuperscript{10}}}

Brother Richard, as well as Brother Thomas, lit funeral pyres of the
vanities, just as Florence was to do in 1497 to such an unprecedented
extent, and with such irreplaceable losses for art, at the will of
Savonarola. In Paris and Artois, in 1428 and 1429, such actions remained
confined to the destruction of playing cards, game boards, dice, hair
ornaments, and various baubles that were willingly handed over by men
and women. In fifteenth-century France and Italy, these funeral pyres
were a frequently repeated expression of the deep piety aroused by the
preachers.\textsuperscript{\protect\hypertarget{08_Chapter_One__THE_PASSIONATE_INTE.xhtmlux5cux23id_2196}{\protect\hyperlink{23_NOTES.xhtmlux5cux23id_2197}{17}}}
The turning away from vanity and lust on the part of the remorseful had
become embodied in ceremonial form; passionate piety was stylized into
solemn communal acts, just as those times tended to turn everything into
stylized forms.

We have to transpose ourselves into this impressionability of mind, into
this sensitivity to tears and spiritual repentance, into this
susceptibility, before we can judge how colorful and intensive life was
then.

Scenes of public mourning appeared to be responses to genuine
calamities. During the funeral of Charles VII, the people lost their
composure when the funeral procession came into view: all court
officials ``vestus de dueil angoisseux, lesquelz il faisoit moult piteux
veoir; et de la grant tristesse et courroux qu'on leur veoit porter pour
la mort de leur dit maistre, furent grant pleurs et lamentacions faictes
parmy tout ladicte
ville.''\protect\hypertarget{08_Chapter_One__THE_PASSIONATE_INTE.xhtmlux5cux23id_2269}{\protect\hyperlink{23_NOTES.xhtmlux5cux23id_2268}{†\textsuperscript{11}}}
There were six page boys of the king riding six horses draped entirely
in black velvet: ``Et Dieu scet le doloreux et piteux dueil qu'ilz
faisoient pour leur dit maistre.'' One of the lads was so saddened that
he did not eat nor drink for four days, said the people with great
emotion.\textsuperscript{\protect\hypertarget{08_Chapter_One__THE_PASSIONATE_INTE.xhtmlux5cux23id_2194}{\protect\hyperlink{23_NOTES.xhtmlux5cux23id_2195}{18}}}\protect\hypertarget{08_Chapter_One__THE_PASSIONATE_INTE.xhtmlux5cux23id_2271}{\protect\hyperlink{23_NOTES.xhtmlux5cux23id_2274}{‡\textsuperscript{12}}}

\protect\hypertarget{08_Chapter_One__THE_PASSIONATE_INTE.xhtmlux5cux23page_8}{}{}But
a surplus of tears came not only from great mourning, a vigorous sermon,
or the mysteries of faith. Each secular festival also unleashed a flood
of tears. An envoy from the King of France to Philip the Good repeatedly
breaks into tears during his address. When young John of Coimbra is
given his farewell at the Burgundian court, everyone weeps loudly, just
as happened on the occasion when the Dauphin was welcomed or during the
meeting of the Kings of England and France at Ardres. King Louis XI was
observed to shed tears while making his entry into Arras; during his
time as Crown Prince at the court of Burgundy, he is described by
Chastellain as sobbing or crying on several
occasions.\textsuperscript{\protect\hypertarget{08_Chapter_One__THE_PASSIONATE_INTE.xhtmlux5cux23id_2192}{\protect\hyperlink{23_NOTES.xhtmlux5cux23id_2193}{19}}}
Understandably, these accounts are exaggerated: compare them to the
``there wasn't a dry eye in the house'' of a newspaper report. In his
description of the peace congress at Arras in 1435, Jean Germain makes
the audience fall to the ground filled with emotions, speechless,
sighing, sobbing and crying during the moving addresses by the
delegates.\textsuperscript{\protect\hypertarget{08_Chapter_One__THE_PASSIONATE_INTE.xhtmlux5cux23id_2190}{\protect\hyperlink{23_NOTES.xhtmlux5cux23id_2191}{20}}}
This, most likely, did not happen in this manner, but the bishop of
Chalons found that it had to be that way. In the exaggeration, one can
detect the underlying truth. The same holds true for the floods of tears
ascribed to the sensitive minds of the eighteenth century; weeping was
both edifying and beautiful. Furthermore, who does not know, even today,
the strong emotions, even goose flesh and tears, solemn entry
processions can arouse even if the prince who is at the center of all
this pomp leaves us indifferent? During those times, such an unmediated
emotional state was filled with a half-religious veneration of pomp and
greatness and vented itself in genuine tears.

Those who do not comprehend this difference in susceptibility between
the fifteenth century and our time may be able to come to appreciate it
through a small example from a sphere divorced from that of tears; that
is, the sphere of sudden rage. To us, there is hardly a game more
peaceful and quiet than chess. La Marche says that during chess games
fights break out ``et que le plus saige y pert
patience.''\textsuperscript{\protect\hypertarget{08_Chapter_One__THE_PASSIONATE_INTE.xhtmlux5cux23id_2188}{\protect\hyperlink{23_NOTES.xhtmlux5cux23id_2189}{21}}}\protect\hypertarget{08_Chapter_One__THE_PASSIONATE_INTE.xhtmlux5cux23id_2305}{\protect\hyperlink{23_NOTES.xhtmlux5cux23id_2306}{*\textsuperscript{13}}}
A conflict between royal princes over a chessboard was still as
plausible as a motive in the fifteenth century as in Carolingian
romance.

Daily life offered unlimited range for acts of flaming passion and
childish imagination. Our medieval historians who prefer to rely
\protect\hypertarget{08_Chapter_One__THE_PASSIONATE_INTE.xhtmlux5cux23page_9}{}{}as
much as possible on official documents because the chronicles are
unreliable fall thereby victim to an occasionally dangerous error. The
documents tell us little about the difference in tone that separates us
from those times; they let us forget the fervent pathos of medieval
life. Of all the passions permeating medieval life with their color,
only two are mentioned, as a rule by legal documents: greed and
quarrelsomeness. Who has not frequently wondered about the nearly
incredible violence and stubbornness with which greed, pugnacity, or
vindictiveness rise to prominence in the court documents of that period!
It is only in the general context of the passions that inflame every
sphere of life that these tensions become acceptable and intelligible to
us. This is why the authors of the chronicles, no matter how superficial
they may be with respect to the actual facts and no matter how often
they may err in reporting them, are indispensable if we want to
understand that age correctly.

In many respects life still wore the color of fairy tales. If the court
chroniclers, learned and respected men who knew their princes
intimately, were unable to see and describe these distinguished persons
other than in terms of archaic and hieratic figures, how great the magic
splendor of royalty must have been in the naive imagination of the
people. Here is an example of that fairy-tale quality from the
historical writings of Chastellain: The young Charles the Bold, still
the count of Charolais, has arrived from Sluis of Gorkum, and learns
there that his father, the duke, has canceled his pension and all of his
benefices. Chastellain now proceeds to describe how the count assembles
all his retainers, down to the kitchen boys, and informs them of his
misfortunes in a moving address in which he proclaims his respect for
his father, his concern for the wellbeing of his people, and his love
for them all. Those who have means of their own he asks to await his
fate along with him; those who are poor he sets free to go and, if they
should happen to learn that the count's fortune had taken a turn for the
better, ``return then and you shall find your positions waiting, and you
shall be welcomed by me, and I shall reward the patience you have shown
for my sake.'' ``Lors oyt-l'on voix lever et larmes espandre et clameur
ruer par commun accord: Nous tous, nous tous, monseigneur, vivrons
avecques vous et
mourrons.''\protect\hypertarget{08_Chapter_One__THE_PASSIONATE_INTE.xhtmlux5cux23id_2273}{\protect\hyperlink{23_NOTES.xhtmlux5cux23id_2272}{*\textsuperscript{14}}}
Deeply moved,
\protect\hypertarget{08_Chapter_One__THE_PASSIONATE_INTE.xhtmlux5cux23page_10}{}{}Charles
accepts their offer of fidelity: ``Or vivez doncques et souffrez; et moy
je souffreray pour vous, premier que vous ayez
faute.''\protect\hypertarget{08_Chapter_One__THE_PASSIONATE_INTE.xhtmlux5cux23id_2277}{\protect\hyperlink{23_NOTES.xhtmlux5cux23id_2276}{*\textsuperscript{15}}}
Thereupon the noblemen approach and offer him all their possessions,
``disant l'un: j'ay mille, l'autre: dix mille, l'autre: j'ay cecy, j'ay
cela pour mettre pour vous et pour attendre tout vostre
advenir.''\protect\hypertarget{08_Chapter_One__THE_PASSIONATE_INTE.xhtmlux5cux23id_2275}{\protect\hyperlink{23_NOTES.xhtmlux5cux23id_2280}{†\textsuperscript{16}}}
And everything went on as usual and there was not a single chicken
lacking in the kitchen because of all
this.\textsuperscript{\protect\hypertarget{08_Chapter_One__THE_PASSIONATE_INTE.xhtmlux5cux23id_2186}{\protect\hyperlink{23_NOTES.xhtmlux5cux23id_2187}{22}}}

The embellishments of this picture are, of course, Chastellain's. We do
not know how far his report stylized what had actually happened. But
what really matters is that he sees the prince in the simple forms of
the folk ballads. To him, the entire situation is totally dominated by
the most primitive emotions of mutual loyalty, which express themselves
with epic simplicity.

While the mechanism of the administration of the state and the state
budget had in reality already assumed complicated forms, politics were
embodied in the minds of the people in particular, invariable, simple
figures. The political references with which the people live are those
of the folk song and chivalric romances. Similarly, the kings of the
period are reduced to a few types, each of which more or less correspond
to a motif from song or adventure story: the noble, just prince, the
prince betrayed by evil counselors, the prince as avenger of his
family's honor, the prince supported by his followers during reverses in
his fortune. The subjects of a late medieval state, carrying a heavy
burden and being without any voice in the administration of the taxes,
lived in constant apprehension that their pennies would be wasted,
suspecting that they were not actually spent for the benefit and welfare
of the country. This suspicion directed towards the administration of
the state was transposed into the simplified notion that the king is
surrounded by greedy, tricky advisers or that the ostentation and
wastefulness of the royal court was to blame for the poor state of the
country. Thus political questions were reduced, in the popular mind, to
the typical events of a fairy tale. Philip the Good understood what sort
of language would be intelligible to the people. During his
festivi\protect\hypertarget{08_Chapter_One__THE_PASSIONATE_INTE.xhtmlux5cux23page_11}{}{}ties
in The Hague in 1456 he had displayed in a room adjacent to the Knight's
Hall precious utensils worth thirty thousand marks in order to impress
the Dutch and Frisians who believed that he lacked the funds to take
over the Bishopric of Utrecht. Everyone could come there to see the
display. Moreover, two boxes containing one hundred thousand golden
lions each had been brought from Lille. People were allowed to try to
lift them, but tried in
vain.\textsuperscript{\protect\hypertarget{08_Chapter_One__THE_PASSIONATE_INTE.xhtmlux5cux23id_2185}{\protect\hyperlink{23_NOTES.xhtmlux5cux23page_399}{23}}}
Can anyone imagine a more pedagogically skillful mixture of state credit
and county-fair amusement?

The lives and deeds of the princes occasionally display a fantastic
element that is reminiscent of the Caliph of \emph{Thousand and One
Nights}. In the midst of coolly calculated political undertakings, the
heroes may occasionally display a daring bravado, or even risk their
lives and personal achievements on a whim. Edward III gambled with his
own life, that of the Prince of Wales, and the fate of his country by
attacking a fleet of Spanish merchant vessels in order to exact
vengeance for some acts of
piracy.\textsuperscript{\protect\hypertarget{08_Chapter_One__THE_PASSIONATE_INTE.xhtmlux5cux23id_2183}{\protect\hyperlink{23_NOTES.xhtmlux5cux23id_2184}{24}}}
Philip the Good had taken it into his head to marry one of his archers
to the daughter of a rich brewer in Lille. When the father resisted and
involved the parliament of Paris in the affair, the enraged duke
suddenly broke off the important affairs of state that had kept him in
Holland and, even though it was the holy season preceding Easter,
undertook a dangerous sea voyage from Rotterdam to Sluis to have his.
own
way.\textsuperscript{\protect\hypertarget{08_Chapter_One__THE_PASSIONATE_INTE.xhtmlux5cux23id_2181}{\protect\hyperlink{23_NOTES.xhtmlux5cux23id_2182}{25}}}
Another time in a blinding rage over a quarrel with his son, he ran away
from Brussels and lost his way in the forest like a truant schoolboy.
When he finally returns, the delicate task of getting him back to his
normal routine falls to the knight Phillipe Pot. This adroit courtier
finds the right words: ``Bonjour monseigneur, bonjour qu'est cecy?
Faites-vous du roy Artus maintenant ou de messire
Lancelot?''\textsuperscript{\protect\hypertarget{08_Chapter_One__THE_PASSIONATE_INTE.xhtmlux5cux23id_2179}{\protect\hyperlink{23_NOTES.xhtmlux5cux23id_2180}{26}}}\protect\hypertarget{08_Chapter_One__THE_PASSIONATE_INTE.xhtmlux5cux23id_2279}{\protect\hyperlink{23_NOTES.xhtmlux5cux23id_2278}{*\textsuperscript{17}}}

How caliph-like it seems to us when the same duke, being told by his
physician to have his head shaved, issues an order that all noblemen are
to follow his example and orders Peter von Hagenbach to strip the hair
from any who fail to
comply.\textsuperscript{\protect\hypertarget{08_Chapter_One__THE_PASSIONATE_INTE.xhtmlux5cux23id_2177}{\protect\hyperlink{23_NOTES.xhtmlux5cux23id_2178}{27}}}
Or when the young King Charles VI of France, riding on one horse with a
friend in order to witness the entry procession of his own bride,
Isabella of Bavaria, was, in the press of the crowd, thrashed by the
guards.\textsuperscript{\protect\hypertarget{08_Chapter_One__THE_PASSIONATE_INTE.xhtmlux5cux23id_2175}{\protect\hyperlink{23_NOTES.xhtmlux5cux23id_2176}{28}}}
\protect\hypertarget{08_Chapter_One__THE_PASSIONATE_INTE.xhtmlux5cux23page_12}{}{}A
poet complains that princes promote their jesters or musicians to the
position of councilor or minister as indeed happened to Coquinet the
Fool of
Burgundy.\textsuperscript{\protect\hypertarget{08_Chapter_One__THE_PASSIONATE_INTE.xhtmlux5cux23id_2173}{\protect\hyperlink{23_NOTES.xhtmlux5cux23id_2174}{29}}}

Politics are not yet completely in the grip of bureaucracy and protocol;
at any moment the prince may abandon them and look elsewhere for
guidelines for his administration. Fifteenth-century princes repeatedly
consulted visionary ascetics and renowned popular preachers on matters
of state. Denis the Carthusian and Vincent Ferrer served as political
advisers; the noisy popular preacher Olivier Maillard was privy to the
most secret negotiations between princely
courts.\textsuperscript{\protect\hypertarget{08_Chapter_One__THE_PASSIONATE_INTE.xhtmlux5cux23id_2171}{\protect\hyperlink{23_NOTES.xhtmlux5cux23id_2172}{30}}}
Because of this, an element of religious
tension\textsuperscript{\protect\hypertarget{08_Chapter_One__THE_PASSIONATE_INTE.xhtmlux5cux23id_2169}{\protect\hyperlink{23_NOTES.xhtmlux5cux23id_2170}{31}}}
exists in the highest realms of politics.

At the end of the fourteenth and beginning of the fifteenth centuries,
the people, observing the higher realms of princely life and fate, must
have, more than ever, thought of it as a bloody romantic sphere filled
with dramas of unmitigated tragedy, and the most moving falls from
majesty and glory. During the same September month of 1399 when the
English Parliament, meeting in Westminster, learned that King Richard II
had been defeated and imprisoned by his cousin Lancaster and had
resigned the throne, the German electors were gathered in Mainz to
depose their king, Wenzel of Luxemburg. The latter was just as
vacillating in spirit, incapable of ruling and as moody as his cousin in
England, but did not come to as tragic an end as Richard. Wenzel
remained for many years King of Bohemia, while Richard's deposition was
followed by his mysterious death in prison, which recalled the murder of
his grand-father, Edward II, also in prison, seventy years before. Was
not the crown a tragic possession, fraught with danger? In a third large
kingdom of Christendom a madman, Charles VI, occupied the throne and the
country was soon to be ruined by unrestrained factionalism. The jealousy
between the houses of Orléans and Burgundy erupted into open hostilities
in 1407: Louis of Orléans, the brother of the king, fell victim to vile
murderers hired by his cousin the duke of Burgundy, John the Fearless.
Twelve years later, vengeance: John the Fearless was treacherously
murdered during the solemn meeting on the bridge of Montereau. These two
princely murders with their never ending trail of revenge and strife
left an undertone of dark hatred in the history of France for a whole
century. The popular mind views the misfortunes such as befell France
\protect\hypertarget{08_Chapter_One__THE_PASSIONATE_INTE.xhtmlux5cux23page_13}{}{}in
the light of the great dramatic motifs; it cannot comprehend causes
other than personalities and passions.

The Turks appear in the midst of all this and threaten more ominously
than before. A few years earlier, 1396, they had destroyed the splendid
French army of knights that had recklessly ventured to face them under
the same John the Fearless, then still count of Nevers, near Nicopolis.
And Christendom was torn apart by the Great Schism, which by now had
lasted a quarter of a century. Two individuals called themselves pope,
neither one recognized in heartfelt conviction by a number of Western
countries. As soon as the Council of Pisa of 1409 had ignominiously
failed in its attempt to restore the unity of the church, there would be
three who would compete for the papal title. The stubborn Aragonese,
Peter von Luna, who hung on in Avignon as Benedict XIII, was known in
popular parlance as ``The Pope of the Moon.'' Did this title have the
ring of near insanity for simple folks?

In these centuries a good many dethroned kings made the rounds of the
princely courts---usually short of money and rich in plans, bathed in
the splendor of the mysterious East from which they came: Armenia,
Cyprus, and even Constantinople; every one of them a figure from the
picture of the Wheel of Fortune
(\protect\hyperlink{20_ILLUSTRATIONS_FOLLOW_PAGE.xhtmlux5cux23id_2}{plate
1}) from which kings with scepters and crowns came tumbling down. René
of Anjou was not one of this number. Though a king without a crown, he
lived very well on his wealthy estates in Anjou in Provence. But nobody
embodied more clearly the vagaries of princely fortune than this prince
from the House of France who had missed the best opportunities time and
again, who had reached for the crowns of Hungary, Sicily, and Jerusalem
and suffered nothing but defeats, narrow escapes, and long periods of
imprisonment. This poet-king without a throne, who delighted in poems of
hunting and the art of miniatures, must have been of deep frivolity of
mind or he would have been cured by his fate. He had seen almost all of
his children die and the daughter who was left to him suffered a fate
that in its dark sadness was worse than his own. Margaret of Anjou, full
of intelligence, honor, and passion, had, at the age of sixteen married
King Henry VI of England, who was weak-minded. The English court was a
hell of hatred. Nowhere else had suspicions of royal relatives, charges
against powerful servants of the crown, and secretive and judicial
murders for the sake
\protect\hypertarget{08_Chapter_One__THE_PASSIONATE_INTE.xhtmlux5cux23page_14}{}{}of
security and partisanship so permeated the political scene as in
England. Margaret lived for many years in this atmosphere of persecution
and fear before the great family feud between the Lancasters, the house
of her husband, and the Yorks, that of her numerous and active cousins,
broke out into open, bloody strife. Margaret lost crown and possessions.
The changing fortunes of the War of the Roses meant most terrifying
dangers and bitter poverty for her. Finally, secure in asylum at the
Burgundian court, she gave in her own words to Chastellain, the court
chronicler, the moving report of her misfortunes and her aimless
wanderings: how she and her young son had been at the mercy of
highwaymen, how she had had to beg a Scottish archer for a penny as
offering during a mass, ``qui demy à dur et à regret luy tira un gros
d'Escosse de sa bourse et le luy
presta.''\protect\hypertarget{08_Chapter_One__THE_PASSIONATE_INTE.xhtmlux5cux23id_2283}{\protect\hyperlink{23_NOTES.xhtmlux5cux23id_2282}{*\textsuperscript{18}}}
The good chronicler, moved by so much suffering, dedicated for her
consolation a tract, the \emph{Temple of
Bocace}\textsuperscript{\protect\hypertarget{08_Chapter_One__THE_PASSIONATE_INTE.xhtmlux5cux23id_2167}{\protect\hyperlink{23_NOTES.xhtmlux5cux23id_2168}{32}}}---``Alcun
petit traité de fortune, prenant pied sur son inconstance et déceveuse
nature.''\protect\hypertarget{08_Chapter_One__THE_PASSIONATE_INTE.xhtmlux5cux23id_2281}{\protect\hyperlink{23_NOTES.xhtmlux5cux23id_2285}{†\textsuperscript{19}}}
He believed, in accordance with the standard recipe of those days, that
he could not comfort the troubled princess better than with this gloomy
gallery of princely misfortunes. Neither of them could know that the
worst was yet to come. In 1471 near Tewkesbury, the Lancasters were
decisively beaten, Margaret's only son was killed in the battle or
murdered shortly thereafter, her husband was secretly killed; she
herself spent five years in the Tower, only to be sold by Edward IV to
Louis XI, to whom she had to cede the legacy of her father, King René,
as a show of gratitude for her liberation.

Hearing of genuine royal children suffering such fates, how could the
Burgher of Paris not believe the stories of lost crowns and banishment
that vagabonds occasionally told to evoke sympathy and compassion? In
1427 a band of Gypsies appeared in Paris and represented themselves as
penitents, ``ung duc et ung conte et dix hommes tous à
cheval,''\protect\hypertarget{08_Chapter_One__THE_PASSIONATE_INTE.xhtmlux5cux23id_2284}{\protect\hyperlink{23_NOTES.xhtmlux5cux23id_2289}{‡\textsuperscript{20}}}
The rest, 120 people, had to remain outside the city. They claimed to
have come from Egypt and said that the Pope had made them do penitence
for having left the
Chris\protect\hypertarget{08_Chapter_One__THE_PASSIONATE_INTE.xhtmlux5cux23page_15}{}{}tian
faith. As punishment they had to spend seven years wandering without
ever sleeping in a bed. They said that they had originally numbered
about 1,200, but that their king and queen and all the others had died
on the road. As the only mitigation, they claimed, the Pope had ordered
that each bishop and abbot should give them ten pounds tournois. The
inhabitants of Paris came in huge throngs to see the strange little band
and to have the Gypsy women read their palms. These managed to move the
money from the purses of the people to their own, ``par art magicque ou
autrement.''\textsuperscript{\protect\hypertarget{08_Chapter_One__THE_PASSIONATE_INTE.xhtmlux5cux23id_2165}{\protect\hyperlink{23_NOTES.xhtmlux5cux23id_2166}{33}}}\protect\hypertarget{08_Chapter_One__THE_PASSIONATE_INTE.xhtmlux5cux23id_2288}{\protect\hyperlink{23_NOTES.xhtmlux5cux23id_2287}{*\textsuperscript{21}}}

An aura of adventure and passion surrounded the life of princes, but it
was not only the popular imagination that saw it that way. Modern man
has, as a rule, no idea of the unrestrained extravagance and
inflammability of the medieval heart. Those who only consult official
documents, which are correctly held to contain the most reliable
information for our understanding of history, could fashion for
themselves from this piece of medieval history a picture that would not
be substantially different from a description of ministerial and
ambassadorial politics of the eighteenth century. But such a picture
would lack an important element: the crass colors of the tremendous
passions that inspired the people as well as the princes. There is, no
doubt, a passionate element remaining in contemporary politics, but,
with the exception of days of turmoil and civil war, it encounters more
checks and obstacles. It is led in hundreds of ways into fixed channels
by the complicated mechanisms of communal life. During the fifteenth
century the immediate emotional affect is still directly expressed in
ways that frequently break through the veneer of utility and
calculation. If emotions go hand in hand with a sense of power, as in
the case of princes, the effect is doubled. Chastellain, in his stilted
way, expresses this quite bluntly: Small wonder, he says, that princes
are frequently locked in hostilities with one another, ``puisque les
princes sont hommes, et leurs affaires sont haulx et agus, et leurs
natures sont subgettes à passions maintes comme à haine et envie, et
sont leurs coeurs vray habitacle d'icelles des passions à cause de leur
gloire en
régner.''\textsuperscript{\protect\hypertarget{08_Chapter_One__THE_PASSIONATE_INTE.xhtmlux5cux23id_2163}{\protect\hyperlink{23_NOTES.xhtmlux5cux23id_2164}{34}}}\protect\hypertarget{08_Chapter_One__THE_PASSIONATE_INTE.xhtmlux5cux23id_2286}{\protect\hyperlink{23_NOTES.xhtmlux5cux23id_2291}{†\textsuperscript{22}}}
Does this not corne close to what Burckhardt called ``the pathos of
rule?''

\protect\hypertarget{08_Chapter_One__THE_PASSIONATE_INTE.xhtmlux5cux23page_16}{}{}Whoever
would write a history of the House of Burgundy would have to let the
motif of revenge sound through their narrative like a pedal point, as
black as a catafalque, advising each one at every turn and in battle
giving to each heart its bitter thirst and the taste of broken pride.
Certainly, it would be very naive to return to the all too uncomplicated
view of its history that the fifteenth century itself had. It will not
do, of course, to trace the power struggle from which arose the
centuries-long quarrel between France and the Hapsburgs to the blood
feud between Orléans and Burgundy, the two branches of the House of
Valois. But we should be aware, more than is generally the rule in
researching general political and economic causes, that for
contemporaries, be they observers or participants in the great legal
battles, blood revenge was the essential element that dominated the
actions and fates of princes and countries. For them Philip the Good is
the foremost of the avengers, ``celluy qui pour vengier l'outraige fait
sur la personne du duc Jehan soustint la gherre seize
ans.''\textsuperscript{\protect\hypertarget{08_Chapter_One__THE_PASSIONATE_INTE.xhtmlux5cux23id_2161}{\protect\hyperlink{23_NOTES.xhtmlux5cux23id_2162}{35}}}\protect\hypertarget{08_Chapter_One__THE_PASSIONATE_INTE.xhtmlux5cux23id_2290}{\protect\hyperlink{23_NOTES.xhtmlux5cux23id_2292}{*\textsuperscript{23}}}
Philip took it upon himself as a sacred duty, ``en toute criminelle et
mortelle aigreur, il tireroit à la vengeance du mort, si avant que Dieu
luy vouldroit permettre; et y mettroit corps et âme, substance et pays
tout en l'adventure et en la disposition de fortune, plus réputant
oeuvre salutaire et agréable à Dieu de y entendre que de la
laisser.''\protect\hypertarget{08_Chapter_One__THE_PASSIONATE_INTE.xhtmlux5cux23id_2296}{\protect\hyperlink{23_NOTES.xhtmlux5cux23id_2295}{†\textsuperscript{24}}}
The Dominican who preached the funeral service for the murdered duke
caused considerable outrage because he dared to point out the Christian
duty of not taking
revenge.\textsuperscript{\protect\hypertarget{08_Chapter_One__THE_PASSIONATE_INTE.xhtmlux5cux23id_2159}{\protect\hyperlink{23_NOTES.xhtmlux5cux23id_2160}{36}}}
La Marche spoke as if honor and revenge were both political desires of
the lands ruled by the duke: all estates of his lands joined his cry for
revenge, he
said.\textsuperscript{\protect\hypertarget{08_Chapter_One__THE_PASSIONATE_INTE.xhtmlux5cux23id_2157}{\protect\hyperlink{23_NOTES.xhtmlux5cux23id_2158}{37}}}

The treaty of Arras in 1435, which was supposed to bring peace between
France and Burgundy, begins with penance for the murder at Montereau: a
chapel should be dedicated in the church of Noreau where John had first
been buried, a requiem should be sung there everyday until the end of
time, there should be in the same city a
\protect\hypertarget{08_Chapter_One__THE_PASSIONATE_INTE.xhtmlux5cux23page_17}{}{}Carthusian
monastery, a cross on the bridge itself where the murder happened, and a
mass should be held in the Carthusian church at Dijon where the
Burgundian dukes are
buried.\textsuperscript{\protect\hypertarget{08_Chapter_One__THE_PASSIONATE_INTE.xhtmlux5cux23id_2155}{\protect\hyperlink{23_NOTES.xhtmlux5cux23id_2156}{38}}}
But these were only a part of all the public penances and debasements
demanded by Chancellor Rolin in the name of the duke: churches with
chapters not only at Montereau but also at Rome, Ghent, Paris, Santiago
de Compostella, and Jerusalem must carve the narrative in
stone.\textsuperscript{\protect\hypertarget{08_Chapter_One__THE_PASSIONATE_INTE.xhtmlux5cux23id_2153}{\protect\hyperlink{23_NOTES.xhtmlux5cux23id_2154}{39}}}

A thirst for revenge dressed in such belabored forms must have dominated
the intellect. And what could the people better comprehend of the
politics of their princes than these simple, primitive motives of hatred
and revenge? The attachment to the prince was childish-impulsive in
character; it was a direct feeling of fidelity and community. It was an
extension of the strong old emotion that bound the oath-taker to the
bailiff and the vassals to their lord. This same emotion blazed into
reckless passions during feuds and strife. It was the feeling of party,
not of statehood. The later medieval period was the time of the great
party conflicts. In Italy, these parties consolidated as early as the
thirteenth century, in France and in the Netherlands they popped up
everywhere during the fourteenth century. Anyone who studies the history
of that period will at times be shocked at the inadequacy of the efforts
of modern historians to explain these parties in terms of
economic-political causes. Opposing economic interests, held to be
basic, are purely mechanical constructions. No one, even with the best
of intentions, can find them by reading the sources. This is not an
attempt to deny the presence of economic causes in the formation of
these party groups, but, dissatisfied with the efforts made to explain
them to date, one might well be justified in asking whether a
political-psychological view could not offer greater advantages than the
economic-political for an explanation of late medieval party conflicts.

What the sources reveal about the rise of the parties is approximately
this: in purely feudal times, separate and isolated feuds can be seen
everywhere, in which one cannot find any other economic motive than envy
by one side of the wealth and possessions of the other. But in addition
to the question of material wealth, there is not less importantly that
of honor. Family pride and the thirst for vengeance or the passionate
loyalty on the part of supporters are, in such cases, primary
motivations. To the degree that the power
\protect\hypertarget{08_Chapter_One__THE_PASSIONATE_INTE.xhtmlux5cux23page_18}{}{}of
the state is consolidating and spreading, all these family feuds are
polarizing themselves, so to speak, along the lines of regional power
and are coagulating into parties that perceive even the cause of their
divisions in no other terms than those based on a foundation of
solidarity and shared honor. Do we see any more deeply into these causes
if we postulate economic conflicts? When an acute contemporary observer
declares that no one could discover valid reasons for the hatred between
Hoecken and Kabeljauen in
Holland,\textsuperscript{\protect\hypertarget{08_Chapter_One__THE_PASSIONATE_INTE.xhtmlux5cux23id_2151}{\protect\hyperlink{23_NOTES.xhtmlux5cux23id_2152}{40}}}
we should not shrug our shoulders in contempt and pretend to be smarter
than he is. There is, in fact, no single satisfactory explanation why
the Edmonds were Kabeljauisch and the Wassenaers, Hoeckish. The economic
contrasts that typify these families are only the products of their
position vis-à-vis the prince as followers of this or that
party.\textsuperscript{\protect\hypertarget{08_Chapter_One__THE_PASSIONATE_INTE.xhtmlux5cux23id_2149}{\protect\hyperlink{23_NOTES.xhtmlux5cux23id_2150}{41}}}

How violent the emotions caused by the attachment to the prince could
become can be read on any page of medieval history. The author of the
miracle play \emph{Little Mary of Nymwegen} shows us how Little Mary's
evil aunt, after she and the neighbor ladies work themselves up to the
point of exhaustion over the conflict between Arnold and Adolf of
Geldern,\textsuperscript{\protect\hypertarget{08_Chapter_One__THE_PASSIONATE_INTE.xhtmlux5cux23id_2147}{\protect\hyperlink{23_NOTES.xhtmlux5cux23id_2148}{42}}}
finally hangs herself because she is upset that the old duke has been
freed from captivity. The intent of the author is to warn of the dangers
of such partisanship; for that reason he picks an extreme example, a
suicide out of partisanship---doubtlessly overdone, but evidence for the
party feeling about which the sensitive poet spoke.

There are, however, more comforting examples. The Sheriffs of Abbeville
had the bells rung in the middle of the night because a messenger had
come from Charles of Charolais with the request to pray for the recovery
of his father. The frightened citizens crowded the church, lit hundreds
of candles, knelt or lay in tears throughout the night while the bells
kept on
ringing.\textsuperscript{\protect\hypertarget{08_Chapter_One__THE_PASSIONATE_INTE.xhtmlux5cux23id_2145}{\protect\hyperlink{23_NOTES.xhtmlux5cux23id_2146}{43}}}

When the people of Paris---in 1429 still favoring the English-Burgundian
side\textsuperscript{\protect\hypertarget{08_Chapter_One__THE_PASSIONATE_INTE.xhtmlux5cux23id_2143}{\protect\hyperlink{23_NOTES.xhtmlux5cux23id_2144}{44}}}---learned
that Brother Richard, who had just a short time before moved them with
his sermons, was an Armagnac who surreptitiously won over the towns he
visited, they cursed him in the name of God and all the saints; and in
place of the tin penny bearing the name of Jesus that he had given them,
they took up the cross of St. Andrew, the sign of the Burgundian party.
People resumed the practice of playing dice against which Brother
\protect\hypertarget{08_Chapter_One__THE_PASSIONATE_INTE.xhtmlux5cux23page_19}{}{}Richard
had railed so much, ``en despit de
luy,''\protect\hypertarget{08_Chapter_One__THE_PASSIONATE_INTE.xhtmlux5cux23id_2294}{\protect\hyperlink{23_NOTES.xhtmlux5cux23id_2293}{*\textsuperscript{25}}}
comments the Burgher de
Paris.\textsuperscript{\protect\hypertarget{08_Chapter_One__THE_PASSIONATE_INTE.xhtmlux5cux23id_2141}{\protect\hyperlink{23_NOTES.xhtmlux5cux23id_2142}{45}}}

It would be natural to assume that the schism between Avignon and Rome,
since it had no basis in dogma, could not arouse the passions of faith:
in any case, not in places far from the centers of those events, where
both popes were only known by name, and which were not directly affected
by the split. But here too, the schism immediately evoked keen and
violent partisanship even to the point of confrontations between
believers and nonbelievers. When Bruges changes from the Roman pope to
that of Avignon, numerous people leave home and city, profession or
benefice, so that they may live in Liege or in another area in
conformity to the obedience owed to Urban by their
party.\textsuperscript{\protect\hypertarget{08_Chapter_One__THE_PASSIONATE_INTE.xhtmlux5cux23id_2139}{\protect\hyperlink{23_NOTES.xhtmlux5cux23id_2140}{46}}}
Before the battle of Rosebeke in 1382, the leaders of the French troops
are in doubt whether the oriflamme, the sacred royal flag only to be
used in holy war, can be unfurled in a battle against the Flemish
rebels. The decision to do so is made because the Flemish are Urbanites
and thus
infidels.\textsuperscript{\protect\hypertarget{08_Chapter_One__THE_PASSIONATE_INTE.xhtmlux5cux23id_2137}{\protect\hyperlink{23_NOTES.xhtmlux5cux23id_2138}{47}}}
The French political agent and writer Pierre Salmon, on the occasion of
his visit to Utrecht, is unable to find a priest who will let him
celebrate Easter, ``pour ce qu'ils disoient que je estoie scismatique et
que je créoie en Benedic
l'antipape,''\protect\hypertarget{08_Chapter_One__THE_PASSIONATE_INTE.xhtmlux5cux23id_2385}{\protect\hyperlink{23_NOTES.xhtmlux5cux23id_2386}{†\textsuperscript{26}}}
so that he, alone in a chapel, has to offer confession as if he were
before a priest and heard mass in a Carthusian
monastery.\textsuperscript{\protect\hypertarget{08_Chapter_One__THE_PASSIONATE_INTE.xhtmlux5cux23id_2135}{\protect\hyperlink{23_NOTES.xhtmlux5cux23id_2136}{48}}}

The highly emotional character of partisanship and princely allegiance
was still further enhanced by the powerfully suggestive effect of all
the party signs, colors, emblems, devices, mottoes, which many times
alternated in colorful succession, usually pregnant with murder and
mayhem, but occasionally also with humor. In 1380 as many as two
thousand persons came out to welcome the young Charles VI to Paris, all
dressed alike, half green, half white. Three times between 1411 and
1413, all of Paris suddenly displayed different insignia, purple caps
with the cross of St. Andrew, white caps, and then purple again. Even
priests and women and children wore them. During the Burgundian reign of
terror in Paris in 1411, the Armagnacs were excommunicated every Sunday
to the sound of
\protect\hypertarget{08_Chapter_One__THE_PASSIONATE_INTE.xhtmlux5cux23page_20}{}{}the
church bells. The figures of saints were crowned with the cross of St.
Andrew; it was even claimed that a few priests did not want to make the
sign of the cross in the straight way the Lord was crucified, but made a
slanted
version.\textsuperscript{\protect\hypertarget{08_Chapter_One__THE_PASSIONATE_INTE.xhtmlux5cux23id_2133}{\protect\hyperlink{23_NOTES.xhtmlux5cux23id_2134}{49}}}

The blind passion with which a man supported his party and his lord and,
at the same time, pursued his own interests was, in part, an expression
of an unmistakable, stone-hard sense of right that medieval man thought
proper. It demonstrated an unshakable certainty that every deed
justified ultimate retribution. The sense of justice was still three
quarters heathen and dominated by a need for vengeance. Though the
church sought to soften judicial usage, by pressing for meekness, peace
and reconciliation, it failed to change the actual sense of justice. On
the contrary, that sense was rendered sterner still by adding to the
need for retribution the hatred of sin. All too often sin was, for these
agitated minds, whatever their enemy did. The sense of justice had
gradually escalated to an extreme tension between the two poles of a
barbaric notion of an eye for an eye, a tooth for a tooth and that of a
religious abhorrence of sin, while the role of the state, to punish
severely, came to be considered more and more an urgent necessity. The
sense of insecurity, which in any crisis looks to the power of the state
to implement a reign of terror, became chronic in the later Middle Ages.
The conception of atonement by transgressors gradually faded into an
almost idyllic vestige of an ancient naiveté while the notion that
transgressions were both threats to the community and attacks on the
majesty of God gained ground. The end of the Middle Ages was an
intoxicating time when painful justice and judicial cruelty were in full
bloom. People did not doubt for an instant that the criminal deserved
his punishment. Intense satisfaction was derived from exemplary deeds of
justice performed by the princes themselves. From time to time the
authorities waged campaigns of stern justice, sometimes against robbers
and petty thieves, sometimes against witches and magicians, sometimes
against sodomy.

What strikes us about the judicial cruelty of the later Middle Ages is
not the perverse sickness of it, but the dull, animal-like enjoyment,
the country fair--like amusement, it provided for the people. The people
of Mons paid far too high a price for a robber chief, merely for the
pleasure of quartering him, ``dont
\protect\hypertarget{08_Chapter_One__THE_PASSIONATE_INTE.xhtmlux5cux23page_21}{}{}le
peuple fust plus joyeulx que si un nouveau corps sainct estoit
ressuscité.''\textsuperscript{\protect\hypertarget{08_Chapter_One__THE_PASSIONATE_INTE.xhtmlux5cux23id_2131}{\protect\hyperlink{23_NOTES.xhtmlux5cux23id_2132}{50}}}\protect\hypertarget{08_Chapter_One__THE_PASSIONATE_INTE.xhtmlux5cux23id_2387}{\protect\hyperlink{23_NOTES.xhtmlux5cux23id_2388}{*\textsuperscript{27}}}
During the imprisonment of Maximilian at Bruges in 1488, the rack stands
on a high platform in sight of the imprisoned king. The people cannot
get enough of the spectacle of magistrates, suspected of treason,
undergoing repeated torture. The people delay executions, which the
victims themselves request, for the enjoyment of seeing them subjected
to even more
sufferings.\textsuperscript{\protect\hypertarget{08_Chapter_One__THE_PASSIONATE_INTE.xhtmlux5cux23id_2129}{\protect\hyperlink{23_NOTES.xhtmlux5cux23id_2130}{51}}}

The unchristian extreme to which this mixture of faith and thirst for
revenge led is shown by the prevailing custom in England and France of
refusing individuals under the sentence of death not only extreme
unction, but also confession. There was no intent to save souls; rather,
the intent was to intensify the fear of death by the certainty of the
punishments of hell. In vain, Pope Clement V ordered, in 1311, that
prisoners condemned to death at least be given the sacrament of penance.
The political idealist Philippe de Mézières lobbied repeatedly that this
be done, first with Charles V of France, then with Charles VI. But the
Chancellor Pierre d'Orgemone, whose ``forte cervelle,'' says Mézières,
was more difficult to move than a millstone, resisted, and the wise,
peace-loving Charles V declared that the custom was not to be changed in
his lifetime. Only after the voice of Jean de Gerson had joined that of
Mézières in five considerations against this abuse did a royal edict of
February 12, 1397, order that the condemned be granted confession.
Pierre de Craon, to whose efforts the decision has to be credited, had a
stone cross erected at the gallows in Paris so that the Minorites could
assist the condemned
there.\textsuperscript{\protect\hypertarget{08_Chapter_One__THE_PASSIONATE_INTE.xhtmlux5cux23id_2127}{\protect\hyperlink{23_NOTES.xhtmlux5cux23id_2128}{52}}}
However, even then the old custom did not disappear from popular usage;
as late as shortly after 1500, the bishop of Paris, Etienne Ponchier,
found it necessary to reissue the edict of Clement V. In 1427 a robber
baron was hanged in Paris; during the execution a respected official,
grand treasurer in the service of the regent, vents his hatred of the
condemned by preventing the confession that the prisoner had requested.
Using abusive language, he follows the condemned up the ladder, hits him
with a stick, and attacks the executioner because he has admonished the
victim to think of the bliss of his soul. The
\protect\hypertarget{08_Chapter_One__THE_PASSIONATE_INTE.xhtmlux5cux23page_22}{}{}hangman,
terrified, hurries his task; the rope breaks, the poor victim falls to
the ground, breaks his legs and ribs and must move up the ladder once
more.\textsuperscript{\protect\hypertarget{08_Chapter_One__THE_PASSIONATE_INTE.xhtmlux5cux23id_2125}{\protect\hyperlink{23_NOTES.xhtmlux5cux23id_2126}{53}}}

During medieval times, all those emotions were missing that have made us
cautious and tentative in matters of justice: the insight into
diminished capacity, the concept of judicial fallibility, the awareness
that society has to share in the blame for the guilt of individuals, the
question whether an individual ought not be rehabilitated rather than
made to suffer. Or, perhaps, better stated: a vague sense of all this is
not lacking, but rather concentrates itself, unverbalized, in instant
impulses of charity and forgiveness (unconcerned with the issue of
guilt) which could suddenly break through the cruel satisfaction over
the administration of justice. While we administer a hesitant,
toned-down justice, partially filled with a guilty conscience, the
Middle Ages knew only two extremes: the full measure of cruel punishment
or mercy. In granting mercy the question whether the guilty person
deserved mercy for any particular reason was asked much less frequently
than now: for any transgression, even the most blatant, full pardon
could be granted at any time. In practice, it was not only pure mercy
that tipped the scale in favor of acquittal. It is surprising with what
equanimity contemporaries report how intervention by respected relatives
had secured for a convict ``lettres de rémission.'' Yet most of these
letters do not apply to prominent lawbreakers, but to poor common folk
who did not have highly placed
advocates.\textsuperscript{\protect\hypertarget{08_Chapter_One__THE_PASSIONATE_INTE.xhtmlux5cux23id_2123}{\protect\hyperlink{23_NOTES.xhtmlux5cux23id_2124}{54}}}

The direct juxtaposition of hard-heartedness and mercy characterizes
customs outside the administration of justice. On the one side,
frightful harshness towards the wretched and handicapped; on the other,
unlimited compassion and the most intimate empathy with the poor, sick,
and irrational, which we, in conjunction with cruelty, still know from
Russian literature. Satisfaction with an execution was accompanied, and,
at least to a certain degree justified, by a strong sense of right. The
incredible harshness, the lack of tender sentiment, the cruel mockery,
the secret joy behind the pleasure of watching others suffer lacked even
this element of justice satisfied. The chronicler Pierre de Fenin closes
his report on the end of a band of robbers with the words, ``et
faisoit-on grant risée, pour ce que c'estoient tous gens de povre
estat.''\textsuperscript{\protect\hypertarget{08_Chapter_One__THE_PASSIONATE_INTE.xhtmlux5cux23id_2121}{\protect\hyperlink{23_NOTES.xhtmlux5cux23id_2122}{55}}}\protect\hypertarget{08_Chapter_One__THE_PASSIONATE_INTE.xhtmlux5cux23id_2389}{\protect\hyperlink{23_NOTES.xhtmlux5cux23id_2390}{*\textsuperscript{28}}}

\protect\hypertarget{08_Chapter_One__THE_PASSIONATE_INTE.xhtmlux5cux23page_23}{}{}In
Paris in 1425 an ``esbatement'' was held in which four armored blind men
were made to fight for a pig. In the days before they were seen in their
battle dress throughout the city, a bagpiper and a man with a huge
banner on which the pig is depicted, preceded
them.\textsuperscript{\protect\hypertarget{08_Chapter_One__THE_PASSIONATE_INTE.xhtmlux5cux23id_2119}{\protect\hyperlink{23_NOTES.xhtmlux5cux23id_2120}{56}}}

Velázquez has shown us the touching facial expressions of the female
dwarfs who as fools occupied positions of honor at the Spanish court of
his time
(\protect\hyperlink{20_ILLUSTRATIONS_FOLLOW_PAGE.xhtmlux5cux23id_3}{plate
2}). They were prized diversions at the princely courts of the fifteenth
century. During the artful
\emph{entremets}\textsuperscript{\protect\hypertarget{08_Chapter_One__THE_PASSIONATE_INTE.xhtmlux5cux23id_2117}{\protect\hyperlink{23_NOTES.xhtmlux5cux23id_2118}{57}}}
of the great courts they displayed their skills and their deformities.
Madame d'Or, the golden blonde female dwarf of Philip of Burgundy, was
well known. She was made to wrestle with the acrobat
Hans.\textsuperscript{\protect\hypertarget{08_Chapter_One__THE_PASSIONATE_INTE.xhtmlux5cux23id_2115}{\protect\hyperlink{23_NOTES.xhtmlux5cux23id_2116}{58}}}
To the wedding of Charles the Bold and Margaret of York in 1468 came
Madame de Beaugrant, ``la naine de Mademoiselle de
Bourgogne,''\protect\hypertarget{08_Chapter_One__THE_PASSIONATE_INTE.xhtmlux5cux23id_2391}{\protect\hyperlink{23_NOTES.xhtmlux5cux23id_2392}{*\textsuperscript{29}}}
dressed as a shepherdess, riding around on a golden lion larger than a
horse. The Lion could open and close his mouth and sang a song of
welcome. The little shepherd girl is given to the young duchess as a
gift and is sat on the
table.\textsuperscript{\protect\hypertarget{08_Chapter_One__THE_PASSIONATE_INTE.xhtmlux5cux23id_2113}{\protect\hyperlink{23_NOTES.xhtmlux5cux23id_2114}{59}}}
We know of no laments over the lot of these little women, but we do have
items from expense accounts that tell us more about them. These accounts
report how a duchess had one such little dwarf removed from the house of
her parents, how the father or mother came to deliver her, and how they
came now and then for a visit and were given a gratuity: ``au pere de
Belon la folle, qui estoit venu veoir sa fille .~.~. ``
\protect\hypertarget{08_Chapter_One__THE_PASSIONATE_INTE.xhtmlux5cux23id_2393}{\protect\hyperlink{23_NOTES.xhtmlux5cux23id_2394}{†\textsuperscript{30}}}
Did the father go home well pleased and highly honored by the court
position of his daughter? During the same year a locksmith of Blois
delivered two iron necklaces, one ``pour attacher Belon la folle et
l'autre por mettre au col de la cingesse de Madame la
Duchesse.''\textsuperscript{\protect\hypertarget{08_Chapter_One__THE_PASSIONATE_INTE.xhtmlux5cux23id_2111}{\protect\hyperlink{23_NOTES.xhtmlux5cux23id_2112}{60}}}\protect\hypertarget{08_Chapter_One__THE_PASSIONATE_INTE.xhtmlux5cux23id_2395}{\protect\hyperlink{23_NOTES.xhtmlux5cux23id_2396}{‡\textsuperscript{31}}}

How the mentally ill were treated can be ascertained from a report about
the provisions made for Charles VI, who, as king, enjoyed treatment that
contrasted favorably with that afforded all others. To bring a wretched
mental case to his senses, no better method was conceived than to have
him frightened by twelve blackened individuals as if devils had come to
take him
away.\textsuperscript{\protect\hypertarget{08_Chapter_One__THE_PASSIONATE_INTE.xhtmlux5cux23id_2109}{\protect\hyperlink{23_NOTES.xhtmlux5cux23id_2110}{61}}}

\protect\hypertarget{08_Chapter_One__THE_PASSIONATE_INTE.xhtmlux5cux23page_24}{}{}There
is a degree of naiveté in the hard-heartedness of the time that makes
our condemnation die on our lips. In the middle of an outbreak of plague
that afflicted Paris, the dukes of Burgundy and Orléans called for the
installation of a ``cour d'amour'' to divert the
people.\textsuperscript{\protect\hypertarget{08_Chapter_One__THE_PASSIONATE_INTE.xhtmlux5cux23id_2107}{\protect\hyperlink{23_NOTES.xhtmlux5cux23id_2108}{62}}}
During a break in the cruel slaughter of the Armagnacs in 1418, the
people of Paris founded the Brotherhood of St. Andrew in the Church of
St. Eustatius; every priest and layman carried a wreath of red roses:
the church is full of them and smells, ``comme s'il fust lavé d'eau
rose.''\textsuperscript{\protect\hypertarget{08_Chapter_One__THE_PASSIONATE_INTE.xhtmlux5cux23id_2105}{\protect\hyperlink{23_NOTES.xhtmlux5cux23id_2106}{63}}}\protect\hypertarget{08_Chapter_One__THE_PASSIONATE_INTE.xhtmlux5cux23id_2397}{\protect\hyperlink{23_NOTES.xhtmlux5cux23id_2398}{*\textsuperscript{32}}}
When the witch trials that had descended upon Arras in 1461 like a
hellish plague were finally canceled, the burghers celebrated the
victory of law with a competition of performances of ``folies
moralisées''; first prize was a silver fleur-de-lis, fourth prize, two
capons: the martyred victims were by this time long
dead.\textsuperscript{\protect\hypertarget{08_Chapter_One__THE_PASSIONATE_INTE.xhtmlux5cux23id_2103}{\protect\hyperlink{23_NOTES.xhtmlux5cux23id_2104}{64}}}

So intense and colorful was life that it could stand the mingling of the
smell of blood and roses. Between hellish fears and the most childish
jokes, between cruel harshness and sentimental sympathy the people
stagger---like a giant with the head of a child, hither and thither.
Between the absolute denial of all worldly joys and a frantic yearning
for wealth and pleasure, between dark hatred and merry conviviality,
they live in extremes.

From the brighter half of their lives little has come down to us: it
seems as if the gay mildness and serenity of soul of the fifteenth
century have been swallowed into paintings and crystalized in the
transparent purity of their lofty music. The laughter of that generation
is dead, their untroubled joy and natural zest for life lives only in
folk song and farce. This is enough to add to our nostalgia for the lost
beauty of other times, a longing for the sunlight of the century of the
Van Eycks. But those who really delve into that time must frequently try
very hard in order to capture its brighter aspects since, outside the
sphere of art, darkness rules. In the dire warnings of the preachers, in
the tired sighs of the greatest literature, in the monotonous reports of
the chronicles and sources, we hear only the cries of motley sins and
the lamentations of misery.

Post-Reformation times no longer saw the cardinal sins of pride, anger,
and greed in the purple full-bloodedness and shameless assertiveness
with which they walked among the humanity of the fifteenth century. The
unlimited arrogance of Burgundy! The
\protect\hypertarget{08_Chapter_One__THE_PASSIONATE_INTE.xhtmlux5cux23page_25}{}{}whole
history of that family, from the deeds of knightly bravado, in which the
fast-rising fortunes of the first Philip take root, to the bitter
jealousy of John the Fearless and the black lust for revenge in the
years after his death, through the long summer of that other magnifico,
Philip the Good, to the deranged stubbornness with which the ambitious
Charles the Bold met his ruin---is this not a poem of heroic pride?
Their lands were the scene of the most intensive lives of the West:
Burgundy, as dark with power as with wine, ``la colérique
Picardie,''\protect\hypertarget{08_Chapter_One__THE_PASSIONATE_INTE.xhtmlux5cux23id_2399}{\protect\hyperlink{23_NOTES.xhtmlux5cux23id_2400}{*\textsuperscript{33}}}
greedy, rich Flanders. These are the same lands in which the splendor of
painting, sculpture, and music flower, and where the most violent code
of revenge ruled and the most brutal barbarism spread among the
aristocracy and
burghers.\textsuperscript{\protect\hypertarget{08_Chapter_One__THE_PASSIONATE_INTE.xhtmlux5cux23id_2102}{\protect\hyperlink{23_NOTES.xhtmlux5cux23page_401}{65}}}

That age is more conscious of greed than of any other evil. Pride and
greed can be placed beside one another as the sins of the old and the
new times. Pride is the sin of the feudal and hierarchic period during
which possessions and wealth circulate very little. A sense of power is
not primarily tied to wealth, it is rather more personal, and power, in
order to make itself known, has to manifest itself through imposing
displays: a numerous following of faithful retainers, precious
adornments, and the impressive appearance of the powerful. The feeling
of being more than other men is constantly nourished by feudal and
hierarchic thought with living forms: through kneeling obeisance and
allegiance, solemn respect and majestic splendor, which, all taken
together, make superiority appear as something substantial and
sanctioned.

Pride is a symbolic and theological sin; it is rooted deeply in the soil
of every conception of life and the world. \emph{Superbia} was the root
of all evil: Lucifer's pride was the beginning and cause of all ruin. So
Augustine saw it, and it remained so in the minds of those who came
after: pride is the source of all sins, they come forth from it as if
from their root and
stem.\textsuperscript{\protect\hypertarget{08_Chapter_One__THE_PASSIONATE_INTE.xhtmlux5cux23id_2100}{\protect\hyperlink{23_NOTES.xhtmlux5cux23id_2101}{66}}}

But next to the scripture from which this notion comes, \emph{A superbia
initium sumpsit omnis
perdito},\textsuperscript{\protect\hypertarget{08_Chapter_One__THE_PASSIONATE_INTE.xhtmlux5cux23id_2098}{\protect\hyperlink{23_NOTES.xhtmlux5cux23id_2099}{67}}}\protect\hypertarget{08_Chapter_One__THE_PASSIONATE_INTE.xhtmlux5cux23id_2401}{\protect\hyperlink{23_NOTES.xhtmlux5cux23id_2402}{†\textsuperscript{34}}}
there is another, \emph{Radix omnium malorum est
cupiditas}.\textsuperscript{\protect\hypertarget{08_Chapter_One__THE_PASSIONATE_INTE.xhtmlux5cux23id_2096}{\protect\hyperlink{23_NOTES.xhtmlux5cux23id_2097}{68}}}\protect\hypertarget{08_Chapter_One__THE_PASSIONATE_INTE.xhtmlux5cux23id_2403}{\protect\hyperlink{23_NOTES.xhtmlux5cux23id_2404}{‡\textsuperscript{35}}}
Following this, one could regard greed as the root of all evil. Because
of this, \emph{cupiditas}, which, as such, has
\protect\hypertarget{08_Chapter_One__THE_PASSIONATE_INTE.xhtmlux5cux23page_26}{}{}no
place in the list of deadly sins, was understood as \emph{avaritia}, as
it in fact appears in another reading of the
text.\textsuperscript{\protect\hypertarget{08_Chapter_One__THE_PASSIONATE_INTE.xhtmlux5cux23id_2094}{\protect\hyperlink{23_NOTES.xhtmlux5cux23id_2095}{69}}}
And it appears that since about the twelfth century, the conviction had
gained credence that it was unrestrained greed that ruined the world and
thus replaced pride in the minds of the people as the first and most
fatal of sins. The old primacy theology assigns to superbia is drowned
out by the steadily rising chorus that blames all the misery of the
times to ever-increasing greed. How Dante had cursed it: \emph{la cieca
cupidigia}!

But greed lacks the symbolic and theological character of pride; it is
the natural and material sin, the purely earthly passion. It is the sin
of that period of time in which the circulation of money has changed and
loosened the conditions for the deployment of power. Judging human worth
becomes an arithmetical process. Now there is much greater leeway for
the satisfaction of unrestrained desires and for the accumulation of
treasures. And these treasures have not yet that ghostly intangibility
that modern credit procedures have bestowed on capital; it is still
yellow gold itself that is in the forefront of fantasy. And the
utilization of wealth does not yet have that automatic and mechanical
character of the routine investment of money: satisfaction still lies in
the most drastic extremes of avidity and prodigality. In this
extravagance greed enters into marriage with the older pride. Pride was
still strong and alive: hierarchic, feudal thought had lost none of its
bloom, the lust for pomp and splendor, finery and pageantry was still
crimson.

It is precisely this affinity with a primitive pride that bestows on the
avidity or greed of the later medieval period its direct, passionate,
desperate quality that later times seem to have entirely lost.
Protestantism and the Renaissance have given greed an ethical value;
they have legalized it as useful to promote welfare. Its stigma has
given way to the degree that the denial of all earthly goods are praised
with less conviction. In late medieval times, by contrast, the mind was
still able to positively grasp the distinction, not yet lost, between
sinful greed versus charity or freely willed poverty.

Throughout the literature and chronicles of the time, from proverb to
pious tract, there echoes the bitter hatred of the rich, the complaint
over the greed of the great. Sometimes it sounds like a dark
anticipation of class struggle, expressed through moral out
\protect\hypertarget{08_Chapter_One__THE_PASSIONATE_INTE.xhtmlux5cux23page_27}{}{}rage.
In this area, we can get a sense of the rich tone of life of this time
equally well from documents or narrative sources, but it is the legal
documents that reveal the most unabashed greed.

It was possible, in 1436, for the services in one of the best-attended
churches in Paris to be suspended for twenty-two days because the bishop
refused to reconsecrate the church until he had received a certain
number of pennies from two beggars, who had desecrated the church with a
bloody stain during a scuffle, and who, being poor, did not have the
money. The bishop, Jacques du Chatelier, was considered, ``ung homme
très pompeux, convoicteux, plus mondain que son estat ne
requeroit.''\protect\hypertarget{08_Chapter_One__THE_PASSIONATE_INTE.xhtmlux5cux23id_2405}{\protect\hyperlink{23_NOTES.xhtmlux5cux23id_2406}{*\textsuperscript{36}}}
However, in 1441, under his successor, Denys des Moulins, it happened
again. This time, for four months, no funerals or processions could be
held at the Cemetery of the Innocents, the most famous and sought-after
in Paris, because the bishop demanded more for these services than the
church could raise. The bishop was called, ``homme très pou piteux à
quelque personne, s'il ne recevoit argent ou aucun don qui le vaulsist,
et pour vray on disoit qu'il avait plus de cinquante procès en
Parlement, car de lui n'avoit on rien sans
procès.''\textsuperscript{\protect\hypertarget{08_Chapter_One__THE_PASSIONATE_INTE.xhtmlux5cux23id_2092}{\protect\hyperlink{23_NOTES.xhtmlux5cux23id_2093}{70}}}\protect\hypertarget{08_Chapter_One__THE_PASSIONATE_INTE.xhtmlux5cux23id_2407}{\protect\hyperlink{23_NOTES.xhtmlux5cux23id_2408}{†\textsuperscript{37}}}
One would only have to trace in detail the history of one of the
``nouveaux riches'' of that time, the d'Orgemont family, for example, in
all its base stinginess and legal wrangling, in order to understand the
tremendous hatred of the people and the scorn that the preachers and
poets alike were constantly pouring out against the
rich.\textsuperscript{\protect\hypertarget{08_Chapter_One__THE_PASSIONATE_INTE.xhtmlux5cux23id_2090}{\protect\hyperlink{23_NOTES.xhtmlux5cux23id_2091}{71}}}

The people could not perceive their own fates and the events of their
time other than as a continuous succession of economic mishandling,
exploitation, war and robbery, inflation, want, and pestilence. The
chronic form that war tended to take, the constant threats to the town
and the country from all kinds of dangerous riffraff, the eternal threat
from a harsh and unreliable administration of justice, and on top of all
this, the pressure of the fear of hell and the anxiety about devils and
witches, nourished a feeling of general insecurity that tended to paint
life's background in dark colors. It
\protect\hypertarget{08_Chapter_One__THE_PASSIONATE_INTE.xhtmlux5cux23page_28}{}{}was
not only the life of the poor and small that was insecure. In the lives
of the nobility and magistrates too, dramatic turns of fate and constant
dangers became almost the rule. Mathieu d'Escouchy, a Picard, is one of
those chroniclers of which there were so many in the fifteenth century;
his chronicle, simple, exact and impartial, filled with the conventional
veneration for the knightly ideal and with the traditional moralizing
tendency, lets us assume himself to be an honorable writer who dedicated
his talent to accurate historical work. But what a picture of the life
of the author of this historical work is shown us by the editor of the
original
sources!\textsuperscript{\protect\hypertarget{08_Chapter_One__THE_PASSIONATE_INTE.xhtmlux5cux23id_2088}{\protect\hyperlink{23_NOTES.xhtmlux5cux23id_2089}{72}}}
Mathieu d'Escouchy began his professional career as counselor, alderman,
juror, and bailiff {[}\emph{prévôt}{]} of the city of Péronne between
1440 and 1450. From the beginning, we find him in a kind of feud with
the family of the city attorney, Jean Froment, a feud that is carried
out in the courts. Soon the attorney prosecutes d'Escouchy on charges of
forgery and murder, then for ``excès et attemptaz.'' The bailiff, on his
side, threatens the widow of his enemy with an investigation into the
witchcraft of which she is suspected. But the widow succeeds in getting
an injunction that forces d'Escouchy to put the investigation in the
hands of the court. The matter comes before the Parliament of Paris and
d'Escouchy ends up in prison for the first time. Six more times we find
him accused and under arrest, and once a prisoner of war. In every
instance these were serious criminal cases, and more than once he was
kept in heavy chains. The battle of mutual accusations between the
families of Froment and d'Escouchy is interrupted by a violent clash
during which the son of Froment injures d'Escouchy. Both hire assassins
to take their opponent's life. After this long drawn out feud drops out
of our historical horizon, attacks from elsewhere appear. This time the
bailiff is wounded by a monk. New complaints, then, in 1461:
d'Escouchy's move to Nesle apparently under suspicion of wrongdoing. But
this does not hinder him from advancing his career. He becomes bailiff,
alderman, of Ribemont, procurator of the king in Saint Quentin, and is
elevated to the nobility. After new attacks, incarcerations and
penances, we find him again serving in a war. In 1465 he fights at
Montlhéry for the king against Charles the Bold and is taken prisoner.
From a later campaign he returns a cripple. He marries, but that does
not mean the beginning of a quiet life for him. We find him charged with
forging seals, being
\protect\hypertarget{08_Chapter_One__THE_PASSIONATE_INTE.xhtmlux5cux23page_29}{}{}taken
as a prisoner to Paris, ``comme larron et
murdrier,''\protect\hypertarget{08_Chapter_One__THE_PASSIONATE_INTE.xhtmlux5cux23id_2409}{\protect\hyperlink{23_NOTES.xhtmlux5cux23id_2410}{*\textsuperscript{38}}}
again in a new feud with a magistrate of Compiégne, made under torture
to confess his guilt, prevented from appealing, sentenced,
rehabilitated, sentenced anew, until the traces of this life of hatred
and persecution finally disappear from the documents.

Such biographies, full of sudden turns, are found whenever we study the
lives of individuals identified in the sources of that period. One
reads, for instance, the examples collected by Pierre Champion of all
those whom Villon considered or named in his
will,\textsuperscript{\protect\hypertarget{08_Chapter_One__THE_PASSIONATE_INTE.xhtmlux5cux23id_2086}{\protect\hyperlink{23_NOTES.xhtmlux5cux23id_2087}{73}}}
or in the notes by Tutetey on the diary of the Burgher of Paris. It is
always litigations, crimes, conflicts, and persecutions without end that
we meet. And we are dealing here with the lives of people randomly
brought to light by court, church, or other documents. Chronicles, like
that of Jacques du Clercq, which are just a collection of misdeeds may
paint too dark a picture of those times. Even the ``lettres de
rémission,'' which put daily life before our eyes in such lively
precision, point only to the dark side of life, because they deal with
nothing but crime. Yet any other probe into randomly chosen material,
only confirms our dark vision.

It is an evil world. The fires of hatred and violence burn fiercely.
Evil is powerful, the devil covers a darkened earth with his black
wings. And soon the end of the world is expected. But mankind does not
repent, the church struggles, and the preachers and poets warn and
lament in vain.