\chapter{THE DECLINE OF SYMBOLISM}

RELIGIOUS EMOTIONS ALWAYS TENDED TO TRANSFORM themselves into lively
images. The mentality of the time believed it had come to understand a
mystery once it had placed it before its eyes. Therefore, this need to
worship the inexpressible through visible signs resulted in the constant
creation of new images. In the fourteenth century, the image of the
cross and the lamb were not any longer sufficient to contain the
overflowing love of Jesus; added to them was the veneration of the name
of Jesus itself, and, for some, the new image threatened to become
dominant. Henry Suso had the name of Jesus tattooed over his heart and
compared it to the picture of the beloved that a lover has sewn into his
clothing. He sent handkerchiefs with the sweet name embroidered on them
to his spiritual
children.\textsuperscript{\protect\hypertarget{16_Chapter_Nine__THE_DECLINE_OF_SYM.xhtmlux5cux23id_790}{\protect\hyperlink{23_NOTES.xhtmlux5cux23id_791}{1}}}
Bernardino of Siena, concluding his powerful sermon, lit two candles and
displayed a tablet a yard square on which the name of Jesus, in blue on
gold and surrounded with an aurora, could be seen. ``The people who
filled the church, fell to their knees sobbing and crying over the love
of
Jesus.''\textsuperscript{\protect\hypertarget{16_Chapter_Nine__THE_DECLINE_OF_SYM.xhtmlux5cux23id_788}{\protect\hyperlink{23_NOTES.xhtmlux5cux23id_789}{2}}}
Many other Franciscans and preachers of other orders imitated the
practice. Denis the Carthusian was depicted holding such a tablet raised
in his hands. The sun-like rays around the crest on the arms of Geneva
are derived from this form of
veneration.\textsuperscript{\protect\hypertarget{16_Chapter_Nine__THE_DECLINE_OF_SYM.xhtmlux5cux23id_786}{\protect\hyperlink{23_NOTES.xhtmlux5cux23id_787}{3}}}
The practice appeared suspect to the church authorities; there was talk
of superstition and idolatry, and riots for and against the custom
occurred. Bernardino was invited to appear before the Curia, and Pope
Martin V prohibited the
practice.\textsuperscript{\protect\hypertarget{16_Chapter_Nine__THE_DECLINE_OF_SYM.xhtmlux5cux23id_785}{\protect\hyperlink{23_NOTES.xhtmlux5cux23page_425}{4}}}
However, the urge to worship the Lord in a visible sign soon found a
different and sanctioned form: the
monstrance,\textsuperscript{\protect\hypertarget{16_Chapter_Nine__THE_DECLINE_OF_SYM.xhtmlux5cux23id_783}{\protect\hyperlink{23_NOTES.xhtmlux5cux23id_784}{5}}}
which displaced the Host itself as an object of veneration. Replacing
the form of a tower, which it had at the time of its first appearance in
the fourteenth century, the monstrance took the shape of the radiant
sun, the symbol of divine love. Again, the church had reservations; at
first the use of the
\protect\hypertarget{16_Chapter_Nine__THE_DECLINE_OF_SYM.xhtmlux5cux23page_235}{}{}monstrance
was restricted to the week of the Festival of the Sacraments.

The excess of elements into which the declining Middle Ages dissolved
almost everything would have resulted in nothing but a wild
phantasmagoria had it not been for the fact that almost every image
could find a place in the huge, all encompassing mental system of
symbolism.

There was no great truth of which the medieval mind was more certain
than those words from the Corinthians, ``Videmus nunc per speculum in
aenigmate, tunc autem facie ad faciem'' (``For now we see through a
glass darkly; but then face to face''). They never forgot that
everything would be absurd if it exhausted its meaning in its immediate
function and form of manifestation, and that all things extend in an
important way into the world beyond. That insight is still familiar to
us as an inarticulate feeling in those moments when the sound of rain on
leaves or the light of a lamp on a table penetrates momentarily into a
deeper level of perception than that serving practical thought and
action. It may surface in the form of a sickening obsession to the
effect that all things seem to be pregnant with a threatening personal
intent or with an enigma that we must solve but cannot. It may also,
more frequently, fill us with that calm and strengthening certainty that
our own life shares in the mysterious meaning of the world. The more
that feeling condenses into awe of the One from which all things flow,
the more readily it will move from the clear certainty of isolated
moments to a lasting, ever present feeling or even to an articulated
conviction. ``By cultivating the continuous sense of our connection with
the power that made things as they are, we are tempered more towardly
for their reception. The outward face of nature need not alter, but the
expressions of meaning in it alter. It was dead and is alive again. It
is like the difference between looking on a person without love, or upon
the same person with love. .~.~. When we see all things in God, and
refer all things to him, we read in common matters superior expressions
of
meaning.''\textsuperscript{\protect\hypertarget{16_Chapter_Nine__THE_DECLINE_OF_SYM.xhtmlux5cux23id_781}{\protect\hyperlink{23_NOTES.xhtmlux5cux23id_782}{6}}}

This is the emotional foundation from which symbolism arises. In God,
nothing empty or meaningless exists. ``Nihil vacuum neque sine signo
apud
Deum.''\textsuperscript{\protect\hypertarget{16_Chapter_Nine__THE_DECLINE_OF_SYM.xhtmlux5cux23id_779}{\protect\hyperlink{23_NOTES.xhtmlux5cux23id_780}{7}}}
As soon as the idea of God was conceptualized, everything originating in
Him and finding meaning in Him also crystalized into thoughts
articulated in words. And thus comes into being that noble and lofty
idea of the world as a
\protect\hypertarget{16_Chapter_Nine__THE_DECLINE_OF_SYM.xhtmlux5cux23page_236}{}{}great
symbolic nexus---a cathedral of ideas, the highest rhythmic and
polyphonic expression of all that can be thought.

The symbolic mode of thought is independent of and of equal value to the
genetic mode. The latter, perceiving the world as development, was not
as alien to the medieval mind as is often depicted. But the arising of
one thing from another was only seen in the naive figure of direct
procreation or in a branching off and, by logical deduction, applied to
the things of the mind. One could see it in the structure of
genealogies, of the branches of trees: an ``arbor e origine iuris et
legnum'' ordered everything, as far as the law was concerned, into the
image of a tree and its widely spreading branches. In this deductive
application, evolutionary thought retained a somewhat formalized,
arbitrary, and barren quality.

Viewed from the standpoint of causal thinking, symbolism represents an
intellectual shortcut. Thought attempts to find the connection between
things, not by tracing the hidden turns of their causal ties, but rather
by suddenly jumping over these causal connections. The connection is not
a link between cause and effect, but one of meaning and purpose. The
conviction that such a link exists may come into existence whenever two
things share an essential quality that relates to something of general
value. Or, in other words, any association on the basis of any identity
may be directly transformed into an awareness of an essential and mystic
connection. From a psychological vantage point this may appear to us as
a very meager intellectual function. From an ethnological viewpoint we
can see that it is also very primitive. Primitiveness of thought reveals
itself in its weak ability to perceive the boundaries between things; it
attempts to incorporate into the idea of a particular thing all that
which constitutes by its very presence any kind of connection based on
similarity or membership in a particular category. The symbolizing
function is most intimately related to this.

But symbolism loses any semblance of arbitrariness and immaturity as
soon as we realize that it is inseparably linked to that worldview that
was known as realism during medieval times and that we, somewhat less
fittingly, call Platonic idealism.

The symbolic postulation of identity on the basis of shared
characteristics is only meaningful if the qualities shared by the symbol
and the thing symbolized are regarded as being truly essential. White
and red roses bloom among thorns. The medieval mind
\protect\hypertarget{16_Chapter_Nine__THE_DECLINE_OF_SYM.xhtmlux5cux23page_237}{}{}immediately
sees in this fact symbolic significance: virgins and martyrs shine in
glory among those who persecute them. How is this postulate of identity
achieved? By virtue of the identity of the qualities: beauty,
tenderness, purity. The blood red tint of the roses is also that of the
virgin and the martyr. But this connectedness is only truly meaningful
and full of mystic significance if the linkage, the quality, the essence
between the two constituents of the particular symbolism are shared by
each of them. In other words, where red and white are regarded not as
mere labels for physical differences on a quantitative basis, but as
real entities, as realities themselves. Our intellect is still capable
of seeing things in this way at any
time\textsuperscript{\protect\hypertarget{16_Chapter_Nine__THE_DECLINE_OF_SYM.xhtmlux5cux23id_777}{\protect\hyperlink{23_NOTES.xhtmlux5cux23id_778}{8}}}
if we can momentarily capture the wisdom of primitive man, the child,
the poet, or the mystic. For all these, the natural essence of things is
locked up in their general quality. This characteristic is their being,
the nucleus of their essence. Beauty, tenderness, whiteness by being
essences are identities: everything white is beautiful, tender, and
everything that is white has to be connected, has to have the same basis
to its existence, has to have the same importance before God.---This is
why, in medieval thought, an inseparable link exists between symbolism
and realism (in the medieval sense of the word).

We should not be too concerned, here, with the quarrel over
``universals.'' To be sure, the realism proclaimed by the
\emph{universalia ante res}, which ascribed essence and preexistence to
general terms, did not dominate medieval thought. There were also
nominalists: the \emph{universalia post rem} had its defenders. However,
the thesis is not too daring that radical nominalism has never been
something else other than a countercurrent, a reaction, an opposition,
and that the younger, more moderate nominalism only accommodated certain
philosophical reservations about an extreme realism, but placed no
obstacle in the path of the inherent-realistic thought of medieval
intellectual culture in
general.\textsuperscript{\protect\hypertarget{16_Chapter_Nine__THE_DECLINE_OF_SYM.xhtmlux5cux23id_775}{\protect\hyperlink{23_NOTES.xhtmlux5cux23id_776}{9}}}

Inherent to the entire culture. Because what matters is not primarily
that dispute among keen-minded theologians, but the ideas that
completely dominate the life of fantasy and thought as it is expressed
in art, ethics, and daily life. They are all extremely realistic, not
because high theology had been educated in a long tradition of
neo-Platonism, but because realism, independent of philosophy, is the
primitive mode of thought. To the primitive mind, everything that is
capable of being named immediately assumes an
es\protect\hypertarget{16_Chapter_Nine__THE_DECLINE_OF_SYM.xhtmlux5cux23page_238}{}{}sence,
be it a quality, a form, or something else. They project themselves
automatically on the heavens. Their essence may almost always (but not
necessarily always) be personified; the dance of anthropomorphic terms
may begin at any moment.

All realism in the medieval sense is ultimately anthropomorphism. If the
thought that ascribes an independent essence to an idea wishes to make
it visible, there is no other way except through personification. Here
is the locus where symbolism and realism turn into allegory. An allegory
is symbolism projected on a superficial power of imagination; it is the
intentional expression and, with it, also the exhausting of a symbol;
the transposition of a passionate cry into a grammatically correct
sentence. Goethe describes the contrast as follows: ``Allegory changes
manifestation into a term, the term into an image, but does so in such a
way that the term can always be kept linked to the image and preserved
in it. The term is completely captured in the image and expressed by it.
Symbolism changes the manifestation into an idea, the idea into an image
and does so in such a manner that the idea remains forever effective and
unreachable and, though spoken of in all languages,
inexpressible.''\textsuperscript{\protect\hypertarget{16_Chapter_Nine__THE_DECLINE_OF_SYM.xhtmlux5cux23id_773}{\protect\hyperlink{23_NOTES.xhtmlux5cux23id_774}{10}}}

Allegory has the potential of being reduced to a pedantic commonplace
and, at the same time, of reducing an idea to an image. The manner by
which allegory entered medieval thought, namely as a literary product of
late antiquity, in the allegorical productions of Martianus Capella and
Prudentius, increased its pedantic and senile character. However, it
would be wrong to believe that medieval allegory and personification
lack authenticity and vitality. If it lacked these, why did medieval
culture cultivate allegory so consistently and with such dedication?

Taken together, these three ways of thought---realism, symbolism, and
personification---illuminated the medieval mind like a flood of light.
Psychology is prone to deal with symbolism in its entirety in terms of
the association of ideas. Historians of culture, however, have to view
that form of thought with greater reverence. The value for life of a
symbolic interpretation of all of existence was incalculable. Symbolism
created an image of the world more strictly unified by stronger
connections than causal-scientific thought is capable of. It embraces in
its strong arms all of nature and all of history. In both, it creates an
indissoluble order of rank, an architectural structure, a hierarchical
subordination. Since in any
\protect\hypertarget{16_Chapter_Nine__THE_DECLINE_OF_SYM.xhtmlux5cux23page_239}{}{}symbolic
context one thing has to be higher and another lower, things of equal
value cannot be symbols of each other, but, together, can only point to
a third that is higher than they. There is ample room in symbolic
thought for an immeasurable variety of relationships among things, since
anything with its individual qualities can be the symbol of yet other
things, and may, with one and the same quality, signify quite various
other things, since the highest things are symbolized by thousands of
lower things. Nothing is too low to signify the highest of things and to
point to it for the purpose of its glorification. The walnut signifies
Christ; the seed kernel is the divine nature, the outer shell His human
nature, and the woody membrane in between is the cross. All things offer
support and stability to the mind as it climbs to the eternal; all
things lift each other to the heights. Symbolic thought causes the
continuous transfusion of the feeling for God's majesty and for eternity
into everything that can be perceived and thought. It never allows the
fire of the mystic life to be extinguished. It permeates the idea of
anything with heightened aesthetic and ethical value. Just try to
imagine the enjoyment of seeing every jewel sparkle with the splendor of
its symbolic value, of the moment when the identity of roses with
virginity is more than just poetic Sunday dress, the time when
identification points to the essence of both. It is a true polyphony of
thought. In a completely thought-out symbolism, each element
reverberates in a harmonious musical chord of symbols. Symbolic thinking
yields to that intoxication of thought, leads to that pre-intellectual
obscuring of the definition of things, that muting of rational thought,
which lifts the intensity of the feeling for life to its very peak.

All realms of thought are joined in this harmonious connectedness. The
facts of the Old Testament have meaning, they prefigure those of the New
Testament. Profane history reflects them both. In any thinking, just as
in a kaleidoscope, a beautiful symmetric figure takes shape from the
chaotic mass of particles. Every symbol, by virtue of the fact that all
of them are ultimately aligned around the central miracle of the
Eucharist, attains a super-value, a much stronger degree of reality, and
at this level signification is no longer symbolic, it is identity; The
Host is Christ. And the priest who eats it becomes the tomb of the Lord
because of his act. The derived symbol partakes in the reality of the
highest mystery, every act of signification becomes a mystic
one-being.\textsuperscript{\protect\hypertarget{16_Chapter_Nine__THE_DECLINE_OF_SYM.xhtmlux5cux23id_771}{\protect\hyperlink{23_NOTES.xhtmlux5cux23id_772}{11}}}

\protect\hypertarget{16_Chapter_Nine__THE_DECLINE_OF_SYM.xhtmlux5cux23page_240}{}{}Through
symbolism it becomes possible both to honor and enjoy the world, which,
by itself, is damnable, and to ennoble the earthly enterprise, since
every profession has its relationship to the highest and holiest. The
labor of the craftsman is the eternal generation and incarnation of the
word and the alliance between God and the
soul.\textsuperscript{\protect\hypertarget{16_Chapter_Nine__THE_DECLINE_OF_SYM.xhtmlux5cux23id_769}{\protect\hyperlink{23_NOTES.xhtmlux5cux23id_770}{12}}}
Even between earthly and divine love the threads of symbolic contact run
to and fro. The strong religious individualism, that is the cultivation
of one's own soul to attain virtue and bliss, found its wholesome
counterweight in realism and symbolism that separated one's own
suffering and one's own virtue from the particular character of the
individual personality and elevated both into the sphere of universals.

The ethical value of symbolic thought is inseparable from its formative
value. Symbolic formulation is like music added to the text of logically
formulated doctrinal statements that would sound stiff and insufficient
without the music. ``En ce temps où la speculation est encour toute
scolaire, les concepts définis sont facilement en désaccord avec les
intuitions
profondes.''\textsuperscript{\protect\hypertarget{16_Chapter_Nine__THE_DECLINE_OF_SYM.xhtmlux5cux23id_767}{\protect\hyperlink{23_NOTES.xhtmlux5cux23id_768}{13}}}\protect\hypertarget{16_Chapter_Nine__THE_DECLINE_OF_SYM.xhtmlux5cux23id_2573}{\protect\hyperlink{23_NOTES.xhtmlux5cux23id_2574}{*\textsuperscript{1}}}
Symbolism opened the entire wealth of religious notions to art, which
could express them with rich sound and in full color and, at the same
time, bestow on them both an obscure and a soaring quality that allowed
art to become the vehicle for the most profound intuitions on their way
to the understanding of the inexpressible.

The
waning\textsuperscript{\protect\hypertarget{16_Chapter_Nine__THE_DECLINE_OF_SYM.xhtmlux5cux23id_765}{\protect\hyperlink{23_NOTES.xhtmlux5cux23id_766}{14}}}
Middle Ages display this entire world of thought in its last
flourishing. The world was perfectly pictured through that all
encompassing symbolism, and the individual symbols turned into petrified
flowers. From the time of antiquity, symbolism had a tendency to become
purely mechanical. Once established as a principle of thought, symbolism
arises not only from poetic imagination and enthusiasm, but attaches
itself to the intellectual function like a parasitic plant and
degenerates into pure habit and a disease of thought. Whole perspectives
of symbolic contact arise, particularly when the symbolic contact comes
from a mere correspondence in number. Symbolizing becomes simply the use
of arithmetical tables. The twelve months are supposed to signify the
twelve apostles, the four seasons the Evangelists, and the entire year
is then bound to mean
Christ.\textsuperscript{\protect\hypertarget{16_Chapter_Nine__THE_DECLINE_OF_SYM.xhtmlux5cux23id_763}{\protect\hyperlink{23_NOTES.xhtmlux5cux23id_764}{15}}}

\protect\hypertarget{16_Chapter_Nine__THE_DECLINE_OF_SYM.xhtmlux5cux23page_241}{}{}Conglomerates
of systems based on the number seven take shape. The Seven Cardinal
Virtues correspond to the seven requests of the Lord's Prayer, the Seven
Gifts of the Holy Spirit, The Seven Praises of Bliss, and the Seven
Penitential Psalms. These, in turn are related to the Seven Moments of
the Passion and the Seven Sacraments. Every individual unit of the
sevens corresponds again as contrast or cure for the Seven Cardinal
Sins, which are represented by seven animals that are followed by seven
diseases.\textsuperscript{\protect\hypertarget{16_Chapter_Nine__THE_DECLINE_OF_SYM.xhtmlux5cux23id_761}{\protect\hyperlink{23_NOTES.xhtmlux5cux23id_762}{16}}}
For a true healer of souls and moralist such as Gerson, from whom the
above examples are taken, the practical ethical value of the symbolic
relations predominates. For a visionary such as Alain de la Roche, it is
the aesthetic element in the relationship that is most
important.\textsuperscript{\protect\hypertarget{16_Chapter_Nine__THE_DECLINE_OF_SYM.xhtmlux5cux23id_759}{\protect\hyperlink{23_NOTES.xhtmlux5cux23id_760}{17}}}
He has to establish a system depending on the numbers ten and fifteen
because the prayer cycle of the Brotherhood of the Rosary, which
commanded his zealous support, comprises 150 aves interrupted by fifteen
paters. The fifteen paters are the fifteen moments of the Passion, the
150 aves are the Psalms. But they mean much more. Multiplying the eleven
heavenly spheres plus the four elements by the ten categories:
\emph{substantia, qualitas, quantitas}, etc. yields the 150
\emph{habitudines naturales}; the same \emph{habitudines naturales} one
obtains by multiplying the Ten Commandments by the fifteen virtues. The
three theological, the four cardinal, the seven capital virtues amount
to fourteen, ``restant duae: religio et poenitentia,'' which means that
there is one too many, but Temperantia, the Cardinal Virtue, corresponds
to
Abstinentia,\textsuperscript{\protect\hypertarget{16_Chapter_Nine__THE_DECLINE_OF_SYM.xhtmlux5cux23id_757}{\protect\hyperlink{23_NOTES.xhtmlux5cux23id_758}{18}}}
the Capital Virtue, which means that fifteen are left. Each of these
virtues is a queen who has her bridal bed in one of the segments of the
Lord's Prayer. Each of the words of the ave means one of the Fifteen
Perfections of Mary and at the same time a precious stone on the
\emph{rupis angelica}, which is Mary herself; every word drives away a
sin or the animal symbolizing it. They are also branches of a tree laden
with fruit in which all the saints are sitting, and the steps of a
stair. For example, the word ave signifies Mary's innocence and a
diamond. It drives away pride, which, in turn, is symbolized by a lion.
The word Mary means her wisdom and a carbuncle; it drives away envy,
symbolized by a black dog. In his vision, Alain sees the disgusting
figures of the sin-symbolizing animals and the shining colors of the
precious stones. The stones' miraculous powers, long famous, give rise,
in turn, to new symbolic associations. The sardonyx is black, red, and
white just as Mary was black in her humility, red in her
\protect\hypertarget{16_Chapter_Nine__THE_DECLINE_OF_SYM.xhtmlux5cux23page_242}{}{}pain,
and white in her glory and mercy. Used as a seal, wax will not stick to
this stone. This signifies the virtue of honorability, it drives away
unchastity and causes people to be honorable and chaste. The pearl is
the word \emph{gratia} and also Mary's own mercy. It is generated inside
a seashell from a heavenly dew ``sine admixtione cuiuscunque seminis
propagationis.'' Mary herself is this shell; in this instance the
symbolism is slightly shifted because one would expect that Mary would
be the pearl if one were to follow the pattern of the other precious
stones. The kaleidoscopic nature of symbolism is also strikingly
expressed here: the words ``created from heavenly dew'' also call to
mind, albeit not made explicit, the other trope of the virgin birth, the
fleece on which Gideon prayed that the holy sign might descend.

The symbolizing mode of thought had been almost entirely spent. Finding
symbols and allegories had become mere play, a superficial fantasizing
on a simple association of ideas. A symbol retains an emotional value
only by virtue of the holiness of the thing it symbolizes; as soon as
symbolizing shifts from the purely religious realm to one exclusively
moral, its hopeless degeneration is exposed. Froissart, in an elaborate
poem ``Le orloge amoureus,'' manages to connect all the qualities of
love to the different parts of a
clockwork.\textsuperscript{\protect\hypertarget{16_Chapter_Nine__THE_DECLINE_OF_SYM.xhtmlux5cux23id_756}{\protect\hyperlink{23_NOTES.xhtmlux5cux23page_426}{19}}}
Chastellain and Molinet compete in political symbolism. In the three
estates the characteristic qualities of Mary are represented; the seven
Electors of the Holy Roman Empire, three spiritual and four secular,
represent the three Theological and the four Cardinal Virtues; the five
cities, St. Omer, Aire, Lille, Douai, and Valenciennes, that remained
loyal to Burgundy in 1477 become the Five Wise
Virgins.\textsuperscript{\protect\hypertarget{16_Chapter_Nine__THE_DECLINE_OF_SYM.xhtmlux5cux23id_754}{\protect\hyperlink{23_NOTES.xhtmlux5cux23id_755}{20}}}
Actually, this is a reverse symbolism; the lower does not point to the
higher, but rather the higher to the lower, since, in the mind of the
inventor, the earthly things that he intends to glorify with some
heavenly ornamentation are fore-most. The \emph{Donatus moralisatus seu
per allegoriam traductus}, occasionally ascribed to Gerson, blended
Latin grammar with theological symbolism: the noun is man, pronouns show
that he is a sinner. At the lowest level of signification is a poem such
as ``Le parement et triumphe des dames,'' by Olivier de la Marche, in
which the entire female toilette is compared to virtues and outstanding
qualities. The old courtier's worthy sermon is punctuated by an
occasional facetious wink of the eye. The slipper signifies humility:

\emph{\protect\hypertarget{16_Chapter_Nine__THE_DECLINE_OF_SYM.xhtmlux5cux23page_243}{}{}De
la pantouffle ne vous vient que santé}

\emph{Et tout prouffil sans griefve maladie},

\emph{Pour luy donner tiltre d'auctorité}

\emph{Je luy donne le nom
d'humilité.\protect\hypertarget{16_Chapter_Nine__THE_DECLINE_OF_SYM.xhtmlux5cux23id_2575}{\protect\hyperlink{23_NOTES.xhtmlux5cux23id_2576}{*\textsuperscript{2}}}}

In this way shoes become caution and industry; stockings endurance; the
garter resolution; the shirt honorability; and the corset
chastity.\textsuperscript{\protect\hypertarget{16_Chapter_Nine__THE_DECLINE_OF_SYM.xhtmlux5cux23id_752}{\protect\hyperlink{23_NOTES.xhtmlux5cux23id_753}{21}}}

But even in their most listless expressions, symbolism and allegory
retained for the medieval mind much more lively emotional value than we
might realize. The function of symbolic equations and personalized
figures was so fully developed that any thought would almost
automatically be transposed into a ``personage,'' that is into a
character. Any idea was regarded as an entity, any quality as substance,
and as entity it was immediately personified by the intelligence that
conceived it. Denis the Carthusian saw the church in his visions in just
as personal a shape as it had when it was presented onstage at the court
festivity at Lille. In one of his revelations, he envisions the future
\emph{Reformatio} that the church sought brought about by the fathers of
the Council and by Denis's brother-in-spirit, Nicholas of Cusa: the
church to come in its future purity. He envisions the spiritual beauty
of that purified church as a marvelous and very precious garment, of
indescribable physical beauty in its artistic blend of colors and
figures. In another instance, he sees the church oppressed: ugly, mangy,
bloodless, poor, weak, and downtrodden. The Lord says to Denis, Hark to
your mother, my bride, the Holy Church, and Denis hears an inner voice
as if it emanates from the figure of the church, ``quasi ex persona
ecclesiae.''\textsuperscript{\protect\hypertarget{16_Chapter_Nine__THE_DECLINE_OF_SYM.xhtmlux5cux23id_750}{\protect\hyperlink{23_NOTES.xhtmlux5cux23id_751}{22}}}
In this, the idea is so bound up with the image that it is hardly felt
to be necessary to trace back from the image to the idea, or that the
allegory be explained in all its details. Only the theme need be given,
no matter how imperfectly. The colorful garment is completely adequate
for conveying the ideal of spiritual perfection; this is the dissolution
of a concept into an image; a phenomena that is familiar to us from
moments when ideas dissolve into music.

\protect\hypertarget{16_Chapter_Nine__THE_DECLINE_OF_SYM.xhtmlux5cux23page_244}{}{}We
should remind ourselves at this point of the allegorical figures from
the \emph{Roman de la rose}. When we encounter the names Bel-Accueil,
Doulce Mercy, Humble Requeste, it is only with difficulty that we think
of something tangible. But for the people of the time they were
realities clothed in living form and imbued with passion. They are
perfectly comparable to Roman divinities that were also derived from
abstractions, such as Pavor, Pallor, and Concordia, etc. What Usener
says about them is almost perfectly applicable to medieval allegorical
figures: ``The conception confronted the soul with sensual force and
exercised such power that the word that designated it could be
considered a divine individual, in spite of all the adjectival mobility
that it had left at its
disposal.''\textsuperscript{\protect\hypertarget{16_Chapter_Nine__THE_DECLINE_OF_SYM.xhtmlux5cux23id_748}{\protect\hyperlink{23_NOTES.xhtmlux5cux23id_749}{23}}}
Otherwise the \emph{Roman de la rose} would have been unreadable.
Doux-Penser, Honte, Souvenirs, and the others lived in the heads of the
declining Middle Ages as semi-divine beings. In the case of one of the
\emph{Rose} figures an even further concretization took place: Danger,
originally the menace threatening the suitor during the courtship,
became, in the jargon of love, the betrayed husband.

Repeatedly allegories are employed to express ideas particularly
important to an argument. The bishop of Chalons, intent upon issuing a
very serious warning over political activities to Philip the Good,
presents the remonstrance which he gave on St. Andrew's Day in the
castle of Hesdin, to the duke, the duchess, and the entourage, in the
form of an allegory. He has \emph{Haultesse de Signourie}, who first
resided in the Empire, later at the French and finally at the Burgundian
court, sitting and wailing inconsolably about the fact that she was
threatened in Burgundy by ``Uncaring of Princes,'' ``Weakness of
Councils,'' ``Envy of Servants,'' ``Extortion of Subjects.'' He has
these confronted by other personalities such as ``Alertness of
Princes,'' and so forth, who have to expel the unfaithful servants of
the
court.\textsuperscript{\protect\hypertarget{16_Chapter_Nine__THE_DECLINE_OF_SYM.xhtmlux5cux23id_746}{\protect\hyperlink{23_NOTES.xhtmlux5cux23id_747}{24}}}
Every quality has been rendered independent and personified. This was
obviously the way to make an impression, something that we will find
understandable if we realize that allegory still served a very vital
function in the thought of those times.

The Burgher of Paris is a conventional fellow rarely given to the
enjoyment of cleverly turned phrases or mental games. Yet, when
approaching the most terrible events he has to describe, the Burgundian
murders that permeated Paris in June 1418 with the same stench of blood
smelled in September 1792, he resorts to
allegory:\textsuperscript{\protect\hypertarget{16_Chapter_Nine__THE_DECLINE_OF_SYM.xhtmlux5cux23id_744}{\protect\hyperlink{23_NOTES.xhtmlux5cux23id_745}{25}}}\protect\hypertarget{16_Chapter_Nine__THE_DECLINE_OF_SYM.xhtmlux5cux23page_245}{}{}``Lors
se leva la deesse de Discorde, qui estoit en la tour de Mau-conseil, et
esveilla Ire la forcenée et Convoitise et Enragerie et Vengence, et
prindrent armes de toutes manières et bouterent hors d'avec eulx Raison,
Justice, Memoire de Dieu et Atrempance moult honteusement.''
\protect\hypertarget{16_Chapter_Nine__THE_DECLINE_OF_SYM.xhtmlux5cux23id_2577}{\protect\hyperlink{23_NOTES.xhtmlux5cux23id_2578}{*\textsuperscript{3}}}
This continues in the same vein, alternating with direct descriptions of
the atrocities. ``Et en mains que on yroit cent pas de terre depuis que
mors estoient, ne leur demouroit que leurs brayes, et estoient en tas
comme porcs ou millieu de la boe .~.~.
``;\protect\hypertarget{16_Chapter_Nine__THE_DECLINE_OF_SYM.xhtmlux5cux23id_2579}{\protect\hyperlink{23_NOTES.xhtmlux5cux23id_2580}{†\textsuperscript{4}}}
torrents of rain wash the wounds clean. Why are allegories employed at
this juncture? Because the author wants to rise to a higher intellectual
level than that of the description of everyday events in the other parts
of his diary. He has a need to see these terrible events rising out of
something other than mere personal intentions, and allegory serves him
as a means to express the tragic sentiment.

How much of the function of personifying and allegorizing was still
alive in late medieval times is demonstrated exactly in those places
where it irritates us the most. In the \emph{tableau-vivant}, where
conventional figures are draped in nonessential garb, thus telling
people that this is only play, we, too, are still somewhat able to enjoy
allegory. But during the fifteenth century holy, as well as allegorical,
figures went about in everyday garb, and new personifications could be
added at any moment to serve any ideas one wanted to express. When
Charles de Rochefort in ``L'abuzé en court'' wants to tell about the
\emph{Moralité} of the careless youth who had strayed from the path
because of his experiences at court, he pulls out of his sleeve a number
of new allegories in the style of the \emph{Roman de la rose}. All of
them, beginning with Fol cuidier and Fol bombance, are completely
lacking in lifelike qualities to our taste. Towards the end, when
Pauvreté and Maladie drag the youth to the hospital, they appear, in the
miniatures that illustrate the poem, as noblemen of the time; even Le
Temps requires no beard or scythe but appears dressed in regular vest
and pants. The personifications appear too primitive to us because of
the naive rigidity
\protect\hypertarget{16_Chapter_Nine__THE_DECLINE_OF_SYM.xhtmlux5cux23page_246}{}{}of
the illustrations; everything tender and moving that is felt by the age
itself in the conception of the figures is thereby lost to us. But in
the commonplace nature of the piece is the sign of its vitality. Olivier
de la Marche was not embarrassed at all by the fact that the twelve
Virtues, performing an \emph{estrement} during a court festival in Lille
in 1454, begin to dance, ``en guise de mommerie et à faire bonne chiere,
pour la feste plus joyeusement
parfournier,''\protect\hypertarget{16_Chapter_Nine__THE_DECLINE_OF_SYM.xhtmlux5cux23id_2581}{\protect\hyperlink{23_NOTES.xhtmlux5cux23id_2582}{*\textsuperscript{5}}}
after their poem had been
read.\textsuperscript{\protect\hypertarget{16_Chapter_Nine__THE_DECLINE_OF_SYM.xhtmlux5cux23id_742}{\protect\hyperlink{23_NOTES.xhtmlux5cux23id_743}{26}}}---In
our understanding, human characteristics can still be somewhat linked,
albeit unintentionally, to virtues and emotions, but the medieval mind
does not hesitate to turn ideas into persons, even in cases where we
fail to see anthropomorphic links. Lent, as a personified figure, taking
to the field against the army of Carnival is not a creation of
Brueghel's mad imagination; the poem ``Bataille de Karesme et de
charnage,'' in which the cheese fights against the roach, the sausage
against the eel, originated as early as the end of the thirteenth
century and was already imitated by 1330 by the Spanish poet Juan
Ruiz.\textsuperscript{\protect\hypertarget{16_Chapter_Nine__THE_DECLINE_OF_SYM.xhtmlux5cux23id_740}{\protect\hyperlink{23_NOTES.xhtmlux5cux23id_741}{27}}}
Lent appears in proverbs, too: ``Quaresme fait ses flans la nuit de
pasques'' (``During Easter week, Lent bakes his cakes''). Elsewhere the
formative process goes still farther; in some northern German churches a
doll was suspended in the choir of the church and called ``Lent.''
Wednesday before Easter these \emph{hungerdocks} were cut down during
mass.\textsuperscript{\protect\hypertarget{16_Chapter_Nine__THE_DECLINE_OF_SYM.xhtmlux5cux23id_738}{\protect\hyperlink{23_NOTES.xhtmlux5cux23id_739}{28}}}

Was there any difference between the reality of the holy figures and the
purely symbolic? The former were confirmed by the church, had a
historical character of their own, and had been shaped into images of
wood and stone. On the other hand, the latter had points of contact with
the life of one's own soul and with free fantasy. One may in all
seriousness consider that Fortune and Faux-Semblant were just as alive
as St. Barbara and St. Christopher. Let us not forget that one figure
rose from free fantasy outside of any dogmatic sanction and acquired a
greater reality than any saint and survived them all: Death.

There is actually no essential contrast between the allegory of the
Middle Ages and the mythology of the Renaissance. In the first place,
the figures of mythology are companions to free allegories
\protect\hypertarget{16_Chapter_Nine__THE_DECLINE_OF_SYM.xhtmlux5cux23page_247}{}{}during
a good part of the Middle Ages; Venus plays her part in poems that are
purely medieval. Second, free allegory is still in full bloom until well
into the sixteenth century and beyond. During the fourteenth century, a
virtual contest between allegory and mythology took place. In the poems
of Froissart next to Doux-Semblant, Jonece, Plaisance, Refus, Danger,
Escondit, Franchise there appears a strange collection of mythologems
sometimes disfigured beyond recognition: Atropos, Clothos, Lachesis,
Telephus, Idrophus, Neptisphoras! As far as the wealth of their forms is
concerned, the gods and goddesses still come out on the short end if
compared to the personages of the \emph{Roman de la rose}; they remain
hollow and shadowlike. Or, in cases where they have the scene all to
themselves, they become extremely baroque and unclassical, as in the
``Epistre d'Othéa'' of Christine de Pisan. This relationship is reversed
with the arrival of the Renaissance. Gradually the Olympians and the
nymphs come to replace the Rose and the symbols in importance. From the
treasures of antiquity, the classical figures obtained a wealth of style
and sentiment, a poetic beauty, and, above all, a sense of unity with
nature, in the face of which once lively allegory faded and wasted away.

Symbolism, with its handmaid allegory, had become a mere mental game;
the meaningful had became meaningless. Symbolic thought prevented the
development of causal-genetic thinking. This is not to say that
symbolism precluded it; the natural-genetic connection of things has its
place alongside the symbolic connection, but it remained unimportant as
long as interest had not shifted away from symbolism and turned towards
the natural development of things. One clarifying example: for the
relationship between spiritual and worldly authority, the medieval world
had settled on two symbolic comparisons: one was that of the two
heavenly bodies, the one that God had placed above the other at the time
of the creation; the other was that of the two swords that the disciples
had with them when Christ was arrested. To the medieval mind, these
symbols were by no means merely clever comparisons; they established the
basis of the relationships between authorities that were not allowed to
shed this mystic linkage. These images have the same conceptual value
Peter has as the rock of the church. The force of the symbol gets in the
way of examining the historical development of both powers. When Dante
recognizes secular
au\protect\hypertarget{16_Chapter_Nine__THE_DECLINE_OF_SYM.xhtmlux5cux23page_248}{}{}thority
to be necessary and decisive in \emph{De monarchia}, he first must
destroy the power of the symbol by questioning its applicability in
order to clear the path for the historical investigation.

A comment by Luther attacks the evil of arbitrary, haphazard allegory in
theology. He is speaking of the great masters of medieval theology, of
Denis the Carthusian, of Guilielmus Durandus, the author of the
``Rationale divinorum officiorum,'' of Bonaventura and Gerson, when he
exclaims, ``Allegorical studies are the work of idle people. Or do you
think it would be difficult to spin an allegory about any given matter?
Who is so poor in mind that he could not try his hand at
allegory?''\textsuperscript{\protect\hypertarget{16_Chapter_Nine__THE_DECLINE_OF_SYM.xhtmlux5cux23id_736}{\protect\hyperlink{23_NOTES.xhtmlux5cux23id_737}{29}}}

Symbolism was a poor means of expressing those connections that we know
to be essential at times when they rise to consciousness as we listen to
music---``Videmus nunc per speculum in aenigmate.'' There was an
awareness of looking at an enigma yet here were attempts to distinguish
the images in the mirror, to explain images through images, and to hold
up mirrors to mirrors. The whole world was capsulated in independent
figures; it was a time of overripeness and the falling of blossoms.
Thought had become too dependent on figures; the visual tendency, so
very characteristic of the waning Middle Ages, was now overpowering.
Everything that could be thought had become plastic and pictorial. The
conception of the world had reached the quietude of a cathedral in the
moonlight in which thought was allowed to rest.
