\chapter{IMAGE AND WORD}

EACH ATTEMPT MADE TO DATE TO CLEARLY SEPARATE the Middle Ages and the
Renaissance has resulted in an apparent pushing of the boundaries ever
further back. People saw in the distant Middle Ages forms and movements
that already appeared to bear the stamp of the age to come, and the term
``Renaissance,'' to make it include all these phenomena, has been
stretched to the point that it has lost all its dynamic
powers.\textsuperscript{\protect\hypertarget{21_Chapter_Thirteen__IMAGE_AND_WORD.xhtmlux5cux23id_306}{\protect\hyperlink{23_NOTES.xhtmlux5cux23id_307}{1}}}
All this holds true on the opposite side. Those who take in the spirit
of the Renaissance without preconceived notions find more ``medieval''
elements in it than theory would seem to permit. Ariosto, Rabelais,
Margarete of Navarre, Castiglione, just as are all of the fine arts with
respect to form and content, are full of medieval elements. And yet to
us, the contrast continues to exist: Middle Ages and Renaissance are
expressions in which we sense the basic differences in the nature of an
age just as clearly as the difference between apple and strawberry,
while at the same time it remains virtually impossible to describe this
difference in greater detail.

But it is necessary to retrace the term Renaissance (which, in contrast
to the term Middle Ages, does not, \emph{a priori}, have a restricted
reference to a period of time) as much as possible to its original
meaning. It is clearly objectionable to count, as Fierens
Gevaert\textsuperscript{\protect\hypertarget{21_Chapter_Thirteen__IMAGE_AND_WORD.xhtmlux5cux23id_304}{\protect\hyperlink{23_NOTES.xhtmlux5cux23id_305}{2}}}
and others do, Sluter and the Van Eycks among Renaissance painters.
These artists have an entirely medieval taste about them. They are also
medieval in matters of form and content. In content, because their art
has shed nothing of the old and embraced nothing of the new as far as
subject matter, ideas, and meaning are concerned. In form, because their
conscientious realism and their desire to depict everything as
corporeally as possible is, above all, the perfect product of a
genuinely medieval spirit. Then, this is how we observed this spirit
working in religious thought and creativity, in the thought forms of
everyday life and everywhere else.
\protect\hypertarget{21_Chapter_Thirteen__IMAGE_AND_WORD.xhtmlux5cux23page_330}{}{}The
tendency towards this elaborate realism is abandoned by the Renaissance
during the period of its full development in the Cinquecento, while the
Quattrocento still shares it with the North.

For all practical purposes, this new spirit does not find its expression
in the fine arts and in the literature of the fifteenth century in
France and Burgundy, whatever else of the new beauty may appear there.
Art and literature still serve a spirit that was on the verge of losing
its bloom; they belong to the system of medieval thought and its
ultimate perfection. They have no other task than that of providing
perfect depictions and embellishments of concepts that have been long
thought through. The mind seems to be exhausted, the spirit awaits new
inspiration.

Now, in periods where the depictions of beauty are limited to nothing
other than exact descriptions and the pure expression of intellectual
material that is already clarified and worked through, the pictorial
arts are of more profound value than literature. This, of course, is not
the judgment of contemporaries. To them, the idea, even if no longer
flourishing, still retains so much of its convincing and significant
qualities that it is loved and admired particularly in its embellished
literary form. All those poems echoing the melody of the fifteenth
century, which appear to our sensitivities to be so hopelessly
monotonous and superficial, were praised much more enthusiastically than
any painting was ever praised. The profound emotional value of the
pictorial arts had not yet dawned on the contemporaries, or at least not
to the degree that they were able to express it.

The fact that to us by far the largest part of that literature has lost
any fragrance and luster while we are moved, possibly more profoundly
than were contemporaries, by the fine arts, may be explained by the
profound difference in the effect of art and word. But it would be all
too convenient and, at the same time, all too incomprehensible if we
were to look for the difference in the quality of the talents and to
assume that the poets, excepting Villon and Charles d'Orléans, had been
nothing but conventional empty heads, in contrast to the painters, who
were all geniuses.

Where two do the same thing, it is not necessarily the same. If the
painter limits himself to the simple reproduction in line and color of
an external reality, there is always found behind all that purely formal
imitation an ultimate remainder that is left unsaid and that cannot be
spoken of. However, if the poet aspires to
noth\protect\hypertarget{21_Chapter_Thirteen__IMAGE_AND_WORD.xhtmlux5cux23page_331}{}{}ing
higher than a mere linguistic expression of a visible or already
comprehended reality, he exhausts with his words the treasure of the
unspoken. Granted, there is the possibility of adding a new unexpressed
beauty by virtue of rhythm and sound, but if these elements, too, are
weak, the effect of the poem only lasts as long as the idea itself
captivates the listener. Contemporaries still react to an idea with a
number of living associations because the idea is interwoven with their
lives and they regard it as new and blooming in the splendor of the new
words found for it.

But if the idea does not any longer captivate us for its own sake, the
poem can remain effective only through its form. The form is
incomparably important and may even be so new and alive that questions
concerning the content of the idea are rarely raised. In the literature
of the fifteenth century a new beauty of forms is already beginning to
blossom. But by far the most poetry is still in old forms. Rhythm and
sound are of weak quality. Under these circumstances literature, devoid
of new ideas and new forms, remains an endless series of postludes on
worn-out themes. Those poets have no future.

For the painter of such a period of intellectual history, the day comes
later, because he lives on the treasure of the unspoken and it is the
wealth of this treasure that determines the most profound and most
lasting effect of all art. Look at the portraits by Van Eyck. Here we
have the sharply cut, distant face of his wife
(\protect\hyperlink{20_ILLUSTRATIONS_FOLLOW_PAGE.xhtmlux5cux23id_23}{plate
24}). There the rigid, morose, aristocratic head of Baudouin de Lannoy
(\protect\hyperlink{20_ILLUSTRATIONS_FOLLOW_PAGE.xhtmlux5cux23id_2299}{plate
25}). And there, again, the frightening, mysterious facial expression of
the Canon van de Paele, the sickly, relaxed pose of the Berlin Arnolfini
(\protect\hyperlink{20_ILLUSTRATIONS_FOLLOW_PAGE.xhtmlux5cux23id_25}{plate
27}), the Egyptian-mysterious quality of \emph{Leal Souvenier}
(\protect\hyperlink{20_ILLUSTRATIONS_FOLLOW_PAGE.xhtmlux5cux23id_2300}{plate
28}). Hidden deep in all these is the miracle of the personality
explored to its innermost reaches. In this we encounter the most
profound characterization possible: we are allowed to see it, but it
cannot be put into words. Even if Van Eyck had simultaneously also been
the greatest poet of his century, the secret that reveals itself in the
pictures would not have opened itself to him in words.

The lack of congruence in the attitude and spirit of art and literature
of the fifteenth century rests most profoundly on this fact. But once
the difference is correctly understood, a comparison of literary and
pictorial expression in certain examples and in particular details
nonetheless reveals again much greater congruence than one
ini\protect\hypertarget{21_Chapter_Thirteen__IMAGE_AND_WORD.xhtmlux5cux23page_332}{}{}tially
may have assumed. If the work of the Van Eycks and their successors is
selected as the most representative expression in art, what literary
works would have to be juxtaposed with them for a proper comparison?
Those dealing with the same subject matter do not come first, but rather
those that rise from the same sources, are the products of the same
sphere of life. This sphere, as we have earlier indicated, is that of
the extravagant princely court and the wealthy, ostentatious
bourgeoisie. The literature that is at the same level as the art of Jan
Van Eyck is courtly, or at least aristocratic, is written in French, and
is read and admired by the same circles that place their orders with the
great painters.

On the surface it appears as if we are faced with a great contrast that
makes any comparison nearly meaningless: the subject matter of painting
is overwhelmingly religious, that of French-Burgundian literature
overwhelmingly secular. But our view does not extend far enough in both
directions: in the fine arts, the secular element at one time occupied a
much larger place than the remnants lead us to assume; and in literature
worldly genres tend to attract much too much of our attention. The forms
of expression of major concern in literary history are the
\emph{Minnelied}, the sequels of the \emph{Roman de la rose}, the later
versions of the chivalric novel, the rising novella, satire, and
historiography. In painting, the profound seriousness of the altar
pictures and the portrait come first to mind. In literature we are first
reminded of the lustful leer of the erotic satires and the monotonous
terrors of the chronicles. It almost seems as if the fifteenth century
painted its virtues but described its sins. But even on the literary
side such a view is too limited: not only did pious books still occupy
the larger amount of space in the well-stocked libraries of the
Burgundian dukes, but the pious, edifying, and moralizing element
continued to make its claims even in secular literature and even among
the displays of the greatest frivolity.

Let us return once more to the premise that the effects produced by the
art and literature of the fifteenth century are strongly mismatched. The
literature, with the exception of only a few poets, fatigues and bores
us. It is all endlessly elaborated allegories in which not a single
figure offers anything new or individually its own and which, in
content, present nothing other than the long-established ethical thought
of past centuries, which has often gone stale. Over and again the same
themes: the sleeping hero in an orchard has a vision of a symbolic lady,
the morning stroll early
\protect\hypertarget{21_Chapter_Thirteen__IMAGE_AND_WORD.xhtmlux5cux23page_333}{}{}in
the month of May, the quarrel between the lady and her lover or between
two female friends or any other combination about a point of the
casuistry of love. Hopeless superficiality, a style ornamented with
fool's gold, sugar-sweet romanticism, worn-out fantasy, sober
moralizing: we sigh and ask ourselves over and over again, Are these
really the contemporaries of Jan van Eyck? Could he really have admired
all this?---Most likely, yes. This is not any stranger than seeing Bach
making do with the work of a petit bourgeoisie rhyme-smith inspired by a
rheumatic parochial dogmatism.

Contemporaries who see the works of art being born accept them without
distinction into their life dream. They do not appreciate them on the
basis of objective aesthetic perfection, but on the basis of the
resounding reverberation within them of the sacredness or passionate
vitality of their subject matter. Only when the old life dream is
dreamed out with the passing of time, and sacredness and passion have
vanished like the scent of a rose, only then, by virtue of its means of
expression, that is, its style, structure, and its harmony, does the
purely artistic effect of a work of art begin. These elements may
actually be the same in both the fine arts and literature, but they may,
nonetheless, generate an entirely different artistic evaluation.

Literature and art of the fifteenth century possess both parts of that
general characteristic that we have already spoken of as being essential
for the medieval mind: the full elaboration of all details, the tendency
not to leave any thought unexpressed, no matter what idea urges itself
on the mind, so that eventually everything could be turned into images
as distinctly visible and conceptualized as possible. Erasmus tells us
that he had once listened, in Paris, to a clergyman preach about the
prodigal son for forty days running and had, in this way, filled the
entire Lenten period. The preacher described the son's journey away from
home and back again, how he had once in a lodge eaten tongue pâté for
lunch, at another time had passed a water mill, had gambled and had
stopped at a vegetarian kitchen. The preacher labored to squeeze the
words of the prophets and evangelists for all they were worth to make
them fit all his freely invented chatter. ``Because of this, he seemed
God-like to the inexperienced crowd and the important fat notables
alike.''\textsuperscript{\protect\hypertarget{21_Chapter_Thirteen__IMAGE_AND_WORD.xhtmlux5cux23id_302}{\protect\hyperlink{23_NOTES.xhtmlux5cux23id_303}{3}}}

We propose now to demonstrate this characteristically unrestrained
elaboration by an analysis of two paintings by Jan van
\protect\hypertarget{21_Chapter_Thirteen__IMAGE_AND_WORD.xhtmlux5cux23page_334}{}{}Eyck.
First, \emph{The Madonna of Chancellor Rolin}, now in the Louvre (plates
15, 16).

The scrupulous exactness with which the material of the dresses, the
marble of the floor tiles and of the columns, the sparkle of the
windowpane and the mass book of the chancellor are treated would be
called pedantic in any artist other than Van Eyck. In one detail, the
exaggerated execution has a truly irritating effect, that is, in the
adornment of the capitals on which, in a corner, as if put into
brackets, are depicted the expulsion from paradise, the sacrifice of
Cain and Abel, the exit from the Ark of Noah, and Ham's sin. But this
ardor for elaborating detail only reaches its climax outside of the open
hall that encloses the main figures. Here we find opening, as a wide
vista through the colonnades, the most marvelous perspective Van Eyck
ever painted. We quote Durand-Gréville's
description:\textsuperscript{\protect\hypertarget{21_Chapter_Thirteen__IMAGE_AND_WORD.xhtmlux5cux23id_300}{\protect\hyperlink{23_NOTES.xhtmlux5cux23id_301}{4}}}

If one, tempted by curiosity, is careless enough to get too close, one
is lost; one is captured for the entire time the effort of a sustained
attention may last; as in a dream one sees ornament after ornament, the
crown of the Virgin, the art of the goldsmith; one sees, figure by
figure, the groups that, without rendering them overweight, fill the
capitals of the columns; blossom by blossom, leaf by leaf in the
profusion of the ground; the surprised eye discovers, between the head
of the divine child and the Virgin, a city replete with gables and
beautiful church steeples, a large church with numerous buttresses, a
spacious square cut into two parts in its whole width by a staircase,
and on the square come, walk, run, innumerable brush stokes that signify
an equal number of living figures; our eye is attracted to a bridge
formed like the back of a donkey (dropping off on both ends) that is
crowded with groups of peoples thronging and crossing each other's
paths; our eye follows the bends of a river where microscopically small
barks travel; in the middle of the river is an island, smaller than the
finger nail of a child, on which rises a stately castle complete with
numerous bell towers and surrounded by trees; to the left, our eye scans
a river embankment lined by trees and crowded with people on a stroll;
moving still farther, it transcends one by one the round peaks of green
hills, comes to rest for a
\protect\hypertarget{21_Chapter_Thirteen__IMAGE_AND_WORD.xhtmlux5cux23page_335}{}{}moment
on the distant line of snow-capped mountains and then loses itself in
the infinity of a faintly blue sky where surging clouds are fading into
oblivion.

And O, Wonder!: in all this, contrary to the claims of Michelangelo's
pupil, unity and harmony are not lost. ``Et quand le jour tombe, une
minute avant que la voix des gardiens ne vienne mettre fin a votre
contemplation, voyez comme le chef d'oeuvre se transfigure dans la
douceur du crépuscule; comme son ciel devient encore plus profond; comme
la scène principale, dont les couleurs se sont évanouies, se plonge dans
l'infini mystère de l'Harmonie et de l'Unité .~.~.
''\textsuperscript{\protect\hypertarget{21_Chapter_Thirteen__IMAGE_AND_WORD.xhtmlux5cux23id_299}{\protect\hyperlink{23_NOTES.xhtmlux5cux23page_435}{5}}}\protect\hypertarget{21_Chapter_Thirteen__IMAGE_AND_WORD.xhtmlux5cux23id_2353}{\protect\hyperlink{23_NOTES.xhtmlux5cux23id_2354}{*\textsuperscript{1}}}

Another painting particularly suited for the study of the technique of
infinite detailing is the \emph{Annunciation}, now in the Hermitage in
St. Petersburg
(\protect\hyperlink{20_ILLUSTRATIONS_FOLLOW_PAGE.xhtmlux5cux23id_26}{plate
29}). At the time the triptych, of which this work constitutes the right
wing, actually existed as a whole, what a miraculous creation it must
have been. It seems as if Van Eyck intended to demonstrate the complete
virtuosity, shrinking away from nothing, of a master who can do
anything, and dares everything. None of his works are simultaneously
more primitive, more hieratic, and more contrived. The angel does not
enter with his message into the intimacy of a dwelling chamber (the
scene that the entire genre of domestic painting took as its point of
departure), but, as was prescribed by the code of forms of the older
art, into a church. Both figures lack in pose and facial expression the
gentle sensitivity displayed in the depiction of the Annunciation on the
outer side of the altar in Ghent. The angel greets Mary with a formal
nod, not, as in Ghent, with a lily; he does not wear a small diadem, but
is depicted with scepter and a splendid crown; and he has a rigid
Aegean-smile on his face. In the glowing splendor of the colors of his
garments, the luster of the pearls, the gold and precious stones, he
excels all the other angelic figures painted by Van Eyck. The dress is
green and gold, the brocade coat dark red and gold, and his wings are
decked with peacock feathers. Mary's book, the pillow on the chair,
everything is again detailed with the
\protect\hypertarget{21_Chapter_Thirteen__IMAGE_AND_WORD.xhtmlux5cux23page_336}{}{}greatest
of care. In the church building the details are fitted with anecdotal
elaborations. The tiles show the signs of the zodiac, of which five are
visible, and in addition three scenes from the story of Samson and one
from the life of David. The back of the church, between its vaults, is
decorated with images of Isaac and Jacob in the form of medallions;
Christ on a globe accompanied by two Seraphim can be seen in the
uppermost part of a glass window; and, next to it, in wall paintings,
are the scenes of the finding of Moses and the reception of the tablets
of the law. All is explained by clearly readable inscriptions. Only in
the compartments of the ceiling do the decorations, still hinted at even
there, become unclear to the eye.

And again the miracle that in such an amassing of elaborate details,
just as in the case of the \emph{Madonna} of Rolin, the unity of key and
mood is not lost! In that case there is the gaiety of bright daylight
that pulls the eye across the main scene into the great distance; in
this case the most mysterious darkness of the high vaulted church veils
the entire scene in such a mist of sobriety and \emph{mysterium} that it
is difficult for the eye to detect all the anecdotal details.

This is the effect of the ``unbridled
elaboration''\textsuperscript{\protect\hypertarget{21_Chapter_Thirteen__IMAGE_AND_WORD.xhtmlux5cux23id_297}{\protect\hyperlink{23_NOTES.xhtmlux5cux23id_298}{6}}}
in painting! The painter, this painter, was able to vent his most
unbounded lust for detail (or, should we say, to meet the annoying
orders of an ignorant, but pious donor?) in an area less than half a
square meter in size without tiring us any more than a glance at the
lively throngs of reality would do. Because one glance is all we are
given; the dimension alone already exercises a limiting force and
entering into the beauty and the distinct qualities of everything
depicted takes place without the expense of much intellectual effort;
many perfections are not even noticed, or if they are, immediately
vanish from consciousness and are totally immersed in the effect of
color or perspective.

If we postulate that the literature of the fifteenth century (that is,
of \emph{belle littérature}, since folk art does not enter into this
context) shared the general quality of the ``endless expression of
detail,'' this happens in an entirely different sense. Not in the sense
of a minutely detailed, spider web--like realism that delights in the
surface appearance of things. This is not yet the way this quality
expresses itself in literature. Descriptions of nature and persons still
rely on the simple means of medieval poetry; the individual objects
participat\protect\hypertarget{21_Chapter_Thirteen__IMAGE_AND_WORD.xhtmlux5cux23page_337}{}{}ing
in generating the mood of the poet are mentioned but not described: the
substantive dominates the adjective. Only the main qualities of objects,
as, for example, their colors (their sounds), are given. The
unrestrained elaboration of details in the literary imagination is more
quantitative than qualitative in nature; it consists in piling up very
many individual objects rather than in analyzing their qualities in
detail. The poet does not understand the art of omission. He does not
know the empty spot; he lacks the sense for appreciating the effect of
that which is left unexpressed. This applies to the thoughts expressed
by him as well as to the images he conjures up. Even the thoughts evoked
by the object are linked as completely as possible. All of poetry is
just as overcrowded with details as is painting. But why is it that in
literature such overabundance of details leaves a so much less
harmonious impression?

This may be explained up to a point by the fact that the relationship
between primary and secondary concerns is exactly the reverse in poetry
of what it is in painting. In painting, the difference between the main
concern, that is, the adequate expression of the object, and secondary
concerns is small. In painting everything is essential. To us the
perfect harmony of painting can lie in a single detail.

In the paintings of the fifteenth century is it, above all, the profound
piety and thus the competent expression of the subject matter that we
admire first? Take the example of the Ghent Altarpiece! (Plates 8, 9.)
How little is our attention drawn to the large figures of God, Mary, and
John the Baptist! In the main scene, our eye shifts time and again away
from the lamb, the center of the picture, to the throng of worshipers on
the sides and the nature scene in the background. The eye continues to
be drawn to the margin: to Adam and Eve and to the portraits of the
donors. And if, at least in the Annunciation scene, the most moving
magic charm rests in the figure of the angel and the Virgin, that is, in
the expressive pious element, we delight, even in this instance, almost
more intensively in the copper kettle and the view of the sunny street.
In these details, which were only a secondary concern for the artist,
the mystery of everyday things blossoms in its quiet glow. Here we sense
the direct emotional stirring about the miraculous quality of all
things. There is---other than that we approach \emph{The Lamb of God}
with a preconceived religious conception---no difference between the
art-emotion in viewing the sacred depiction of the worship of
\protect\hypertarget{21_Chapter_Thirteen__IMAGE_AND_WORD.xhtmlux5cux23page_338}{}{}the
Eucharist and the emotion we feel seeing the \emph{Fishmonger's Stall}
(\protect\hyperlink{20_ILLUSTRATIONS_FOLLOW_PAGE.xhtmlux5cux23id_27}{plate
30}) by Emmanuel de Witte in the Rotterdam Museum.

It is exactly in the details that the artist has complete freedom.
Strict conventions are imposed in matters of the depiction of the main
concern of the painting, the depiction of the sacred subject matter.
Every church painting has its iconographic code from which no deviation
is tolerated. But the artist has an unlimited field left to him where he
can freely unfold his creative urges. In the garments, the props, and
the background he is able to do, without encumbrances and outside
guidance, what is the essential task of a painter: to paint. Here he can
reproduce, unrestrained by any convention, what he sees and how he sees
it. The solid, rigid edifice of the holy picture carries the wealth of
his details like a shining treasure, just like a woman with a flower on
her dress.

In the poetry of the fifteenth century, this relationship is, in a
certain sense, precisely reversed. The poet has a free hand with respect
to the main issue: he may, if he is able, find a new idea, but detail
and background are to the highest degree subject to the force of
convention. There exists for nearly every detail a normed form of
expression, a stencil, which they were reluctant to abandon. Everything,
flowers, the enjoyment of nature, sadness and joy, has its fixed form of
expression that the poet can somewhat polish and color, without creating
it anew.

He polishes and colors his subject matter endlessly because the
wholesome restriction imposed on the painter by the surface he has to
fill is lacking; the surface confronting the poet is always infinite.
The limitation of subject matter is unknown to him. Because of this very
freedom, he has to be a greater mind than the painter if he wants to
accomplish something exceptional. Even the average painter will give joy
to later generations; the average poet, however, sinks into oblivion.

To demonstrate the effect of ``unbridled elaboration'' using a work of
the fifteenth century, it would be necessary to go straight to such a
work in its entirety (and they are long!). But since this is not
possible, a few samples will have to do.

Alain Chartier was regarded as the greatest poet of his time. He was
compared to Petrarch, and Clément Marot still counts him among the best.
The brief anecdote I mentioned earlier may be taken as proof of his
popularity.\textsuperscript{\protect\hypertarget{21_Chapter_Thirteen__IMAGE_AND_WORD.xhtmlux5cux23id_295}{\protect\hyperlink{23_NOTES.xhtmlux5cux23id_296}{7}}}
By the standards of his time he could be put at the level of one of the
greatest painters. The
begin\protect\hypertarget{21_Chapter_Thirteen__IMAGE_AND_WORD.xhtmlux5cux23page_339}{}{}ning
of his poem \emph{Le livre des quatre dames}, a conversation among four
noble women whose lovers had fought at Agincourt, provides for us, as
the rules required, the landscape that is the background of the
picture.\textsuperscript{\protect\hypertarget{21_Chapter_Thirteen__IMAGE_AND_WORD.xhtmlux5cux23id_293}{\protect\hyperlink{23_NOTES.xhtmlux5cux23id_294}{8}}}
This landscape should be compared to the well-known landscape of the
Ghent Altarpiece: the wondrous flowery meadow with scrupulously executed
vegetation, with the church steeples rising behind shady hilltops, an
example of the most unbridled elaboration.

The poet ventures out into the spring morning to dispel his prolonged
melancholy.

\emph{Pour oblier melencolie},

\emph{Et pour faire chiere plus lie},

\emph{Ung doulx matin aux champs issy},

\emph{Au premier jour qu'amours ralie}

\emph{Les cueurs en la saison jolie} .~.~.
\protect\hypertarget{21_Chapter_Thirteen__IMAGE_AND_WORD.xhtmlux5cux23id_2707}{\protect\hyperlink{23_NOTES.xhtmlux5cux23id_2708}{*\textsuperscript{2}}}

All this is purely conventional and no rhythmic or formal beauty lifts
it above the most ordinary mediocrity. Now follows the description of
the spring morning:

\emph{Tout autour oiseaulx voletoient},

\emph{Et si très-doulcement chantoient},

\emph{Qu'il n'est cueur qui n'en fust joyeulx}.

\emph{Et en chantant en l'air montoient},

\emph{Et puis l'un l'autre surmontoient}

\emph{A l'estrivée à qui mieulx mieulx}.

\emph{Le temps n'estoit mie nueux},

\emph{De bleu estoient vestuz les cieux},

\emph{Et le beau soleil cler
luisoit}.\protect\hypertarget{21_Chapter_Thirteen__IMAGE_AND_WORD.xhtmlux5cux23id_2709}{\protect\hyperlink{23_NOTES.xhtmlux5cux23id_2710}{†\textsuperscript{3}}}

\protect\hypertarget{21_Chapter_Thirteen__IMAGE_AND_WORD.xhtmlux5cux23page_340}{}{}The
simple acknowledgment of the glories of the season and the location
would have had a very good effect, if only the poet had known how to
restrain himself. There is a real charm in the very simplicity of this
nature poem, but it lacks any strong form. The narrative continues in
its measured clip; after a closer description of the songs of the birds
there follows:

\emph{Les arbres regarday flourir},

\emph{Et lièvres et connins courir}.

\emph{Du printempts tout s'esjouyssoit}.

\emph{Là sembloit amour seignourir}.

\emph{Nul n'y peult vieillir ne mourir},

\emph{Ce me semble, tant qu'il y soit}.

\emph{Des erbes ung flair doulx issoit},

\emph{Que l'air sery adoulcissoit},

\emph{Et en bruiant par la valee}

\emph{Ung petit ruisselet passoit},

\emph{Qui les pays amoitissoit},

\emph{Dont l'eau ne e'estoit pas salee}.

\emph{Là buvoient les oysillons},

\emph{Après ce que des grisillons},

\emph{Des mouschettes et papillons}

\emph{Ilz avoient pris leur pasture}.

\emph{Lasniers, aoutours, esmerillons}

\emph{Vy, et mouches, aux aguillons},

\emph{Qui de beau miel paveillons}

\emph{Firent aux arbres par mesure}.

\emph{De l'autre part fut la closture}

\emph{D'ung pré gracieux, où nature}

\emph{Sema les fleurs sur la verdure},

\emph{blanches, jaunes, rouges et perses}.

\emph{D'arbres flouriz fut la ceinture},

\emph{Aussi blancs que se neige pure}

\emph{Les couvroit ce sembloit paincture},

\emph{Tant y eut de couleurs
diverses.\protect\hypertarget{21_Chapter_Thirteen__IMAGE_AND_WORD.xhtmlux5cux23id_2711}{\protect\hyperlink{23_NOTES.xhtmlux5cux23id_2712}{*\textsuperscript{4}}}}

\protect\hypertarget{21_Chapter_Thirteen__IMAGE_AND_WORD.xhtmlux5cux23page_341}{}{}A
brook rushes over pebbles; fish are swimming in it; a small forest
spreads its branches like green curtains above the banks. There is
another list of birds: ducks, doves, egrets, and pheasants are nesting
yonder.

What, compared to the painting, constitutes in the poem the different
effect of the detailed elaboration of the natural scenery? What, in
other words, is the effect of one and the same inspiration merely
expressed by different means? It is that the painter is compelled by the
character of his art to adhere to simple faithfulness to nature while
the poem loses itself in formless superficiality and the listing of
conventional motifs.

In this instance, poetry is not as close to painting as is prose. The
latter is less tied to particular motifs. It intends, at times, to put
greater emphasis on the conscientious depiction of perceived reality and
executes this with the help of freer means. In this way, perhaps, prose,
better than poetry, demonstrates the more profound relationships between
literature and the fine arts.

The basic characteristic of the late medieval mind is its predominantly
visual nature. This characteristic is closely related to the atrophy of
the mind. Thought takes place exclusively through visual conceptions.
Everything that is expressed is couched in visual terms. The absolute
lack of intellectual content in the allegorical recitations and poems
was bearable because satisfaction was attained through the visual
realization alone. The tendency to express directly the external aspects
of things found a stronger and more perfect means of expression through
pictorial rather than literary means. In the same way, it was able to
express itself more forcefully in prose than in poetry. This is the
reason why the prose of the fifteenth century constitutes in several
respects the middle term between painting and poetry. All three have the
unrestrained elaboration of details in common, but in painting and prose
this leads to a direct realism unknown in poetry, which is left without
anything better at its disposal to replace it.

\protect\hypertarget{21_Chapter_Thirteen__IMAGE_AND_WORD.xhtmlux5cux23page_342}{}{}There
is one author above all in whose work we notice the same crystal-clear
view of the external manifestations as in Van Eyck: Georges Chastellain.
He was Flemish, from the Aalst region. Though he calls himself ``léal
François,'' ``François de
naissance,''\protect\hypertarget{21_Chapter_Thirteen__IMAGE_AND_WORD.xhtmlux5cux23id_2355}{\protect\hyperlink{23_NOTES.xhtmlux5cux23id_2356}{*\textsuperscript{5}}}
it appears that Flemish was his mother tongue. La Marche calls him,
``natif flameng, toutesfois mettant par escript en langaige
franchois.''\protect\hypertarget{21_Chapter_Thirteen__IMAGE_AND_WORD.xhtmlux5cux23id_2357}{\protect\hyperlink{23_NOTES.xhtmlux5cux23id_2358}{†\textsuperscript{6}}}
He himself pointed with modest pride to his Flemish characteristics of
unrefined rusticity; he speaks of ``sa brute langue,'' calls himself,
``homme flandrin, homme de palus bestiaux, ygnorant, bloisant de langue,
gras de bouche et de palat et tout enfangié d'autres povretés
corporelles à la nature de la
terre.''\textsuperscript{\protect\hypertarget{21_Chapter_Thirteen__IMAGE_AND_WORD.xhtmlux5cux23id_291}{\protect\hyperlink{23_NOTES.xhtmlux5cux23id_292}{9}}}\protect\hypertarget{21_Chapter_Thirteen__IMAGE_AND_WORD.xhtmlux5cux23id_2359}{\protect\hyperlink{23_NOTES.xhtmlux5cux23id_2360}{‡\textsuperscript{7}}}
He owes the all too heavy
cothurnism\textsuperscript{\protect\hypertarget{21_Chapter_Thirteen__IMAGE_AND_WORD.xhtmlux5cux23id_289}{\protect\hyperlink{23_NOTES.xhtmlux5cux23id_290}{10}}}
of his stilted prose to that people's manner as well as the grave
``grandiloquence'' that makes him more or less unpalatable to French
readers. His ornate style has something of an elephantine clumsiness
about it; a contemporary rightly calls him ``cette grosse cloche si
hault
sonnant.''\textsuperscript{\protect\hypertarget{21_Chapter_Thirteen__IMAGE_AND_WORD.xhtmlux5cux23id_287}{\protect\hyperlink{23_NOTES.xhtmlux5cux23id_288}{11}}}\protect\hypertarget{21_Chapter_Thirteen__IMAGE_AND_WORD.xhtmlux5cux23id_2361}{\protect\hyperlink{23_NOTES.xhtmlux5cux23id_2362}{§\textsuperscript{8}}}
But we may perhaps credit his Flemish nature for the keen observations
of his style and his vivid colorfulness. Both remind us repeatedly of
modern Belgian authors.

An unmistakable kinship exists between Chastellain and Jan van
Eyck---and at the same time a difference of artistic level. The less
valued qualities of Van Eyck correspond, under the most favorable
circumstances, to the best in Chastellain and it means a lot to be the
equal of Jan van Eyck at his least. I am thinking, for example, of the
singing angels on the Ghent Altarpiece. The heavy garments, all dark red
and gold with sparkling gems, the overly emphasized, distorted face, the
somewhat pedantic ornaments of the music stand all are the painterly
equivalent of the dazzling bombast of the literary Burgundian court
style. But while in painting this rhetorical element occupies a
subordinate position, it becomes the major concern in Chastellain's
prose. His keen observations and his lively realism drown, in most
instances, in a flood of overly beautifully elaborated phrases and in a
clamor of decorative words.

\protect\hypertarget{21_Chapter_Thirteen__IMAGE_AND_WORD.xhtmlux5cux23page_343}{}{}But
whenever Chastellain describes an event that particularly captivates his
Flemish spirit, an entirely direct, plastic earthiness enters into his
narrative, all ceremonial elements notwithstanding, which makes it
extraordinarily suitable for its task. His repertoire of ideas is no
larger than that of his contemporaries; the counterfeit coin of
religious, ethical, and knightly convictions that had been passed around
for some time function with him for ideas. The conception is entirely
superficial, but its depiction is crisp and lively.

His portrayal of Philip the Good nearly approaches the directness of a
Van
Eyck.\textsuperscript{\protect\hypertarget{21_Chapter_Thirteen__IMAGE_AND_WORD.xhtmlux5cux23id_285}{\protect\hyperlink{23_NOTES.xhtmlux5cux23id_286}{12}}}
With the deliberateness of a chronicler who is a novelist at heart, he
tells in particular detail of a quarrel between the duke and his son
Charles early in the year 1457. Nowhere else does his strongly visual
perception of things come so sharply into focus. All the external
circumstances surrounding the event are presented with perfect clarity.
It is mandatory that a few lengthy passages of this narrative be
presented now.

At issue is a position at the court of the young count of Charolais. The
duke, in spite of an earlier promise, intended to give the position to
one of the Croys who enjoyed great favor with him; Charles, who disliked
seeing these favors go their way, opposed them.

Le duc donques par un lundy qui estoit le jour
Saint-Anthoine,\textsuperscript{\protect\hypertarget{21_Chapter_Thirteen__IMAGE_AND_WORD.xhtmlux5cux23id_283}{\protect\hyperlink{23_NOTES.xhtmlux5cux23id_284}{13}}}
après sa messe, aiant bien désir que sa maison demorast paisible et sans
discention entre ses serviteurs, et que son fils aussi fist par son
conseil et plaisir, après que jà avoit dit une grant part de ses heurs
et que la cappelle estoit vuide de gens, il appela son fils à venir vers
luy et lui dist doucement: ``Charles, de l'estrif qui est entre les
sires de Sempy et de Hémeries pour le lieu de chambrelen, je vueil que
vous y mettez cès et que le sire de Sempy obtiengne le lieu vacant.''
Adont dist le conte: ``Monseigneur, vous m'avez baillié une fois vostre
ordonnance en laquelle le sire de Sempy n'est point, et monseigneur,
s'il vous plaist, je vous prie que ceste-là je la puisse
garder.''---``Déa, ce dit le duc lors, ne vous chailliez des
ordonnances, c'est à moy à croistre et à diminuer, je vueil que le sire
de Sempy y soit mis.''---``Hahan! ce dist le conte (car ainsi jurait
tousjours), monseigneur, je vous prie, pardonnez-moy, car je ne le
\protect\hypertarget{21_Chapter_Thirteen__IMAGE_AND_WORD.xhtmlux5cux23page_344}{}{}pourroye
faire, je me tiens a ce que vous m'avez ordonné. Ce a fait le seigneur
de Croy qui m'a brassé cecy, je le vois bien.''---``Comment, ce dist le
duc, me désobéyrez-vous? ne ferez-vous pas ce que je
vueil?''---``Monseigneur, je vous obéyray volentiers, mais je ne feray
point cela.'' Et le duc, à ces mots, enfelly de ire, respondit: Hà!
garsson, désobéyras-tu à ma volenté? va hoys de mes yeux,'' et le sang,
avecques les paroles, lui tira à coeur, et devint pâle et puis à coup
enflambé et si espoentable en son vis, comme je Toys recorder au clerc
de la chappelle qui seul estoit emprès luy, qui hideur estoit à le
regarder .~.~.
\protect\hypertarget{21_Chapter_Thirteen__IMAGE_AND_WORD.xhtmlux5cux23id_2363}{\protect\hyperlink{23_NOTES.xhtmlux5cux23id_2364}{*\textsuperscript{9}}}

Is this not filled with vigor? The soft opening phrases, the rising rage
during the brief exchange of words, the hesitating speech of the son in
which one can already hear the speech of the Charles the Bold he is to
be.

The way the duke looks at his son so terrifies the duchess (whose
presence has not been mentioned until this point) that she, pushing her
son in front of her, hastily tries to flee the wrath of her husband,
silently making her way from the
oratory\textsuperscript{\protect\hypertarget{21_Chapter_Thirteen__IMAGE_AND_WORD.xhtmlux5cux23id_281}{\protect\hyperlink{23_NOTES.xhtmlux5cux23id_282}{14}}}
through the chapel. But she has to turn various corners before she
reaches the door and
\protect\hypertarget{21_Chapter_Thirteen__IMAGE_AND_WORD.xhtmlux5cux23page_345}{}{}the
scribe has the key: ``Caron,
ouvrenous,''\protect\hypertarget{21_Chapter_Thirteen__IMAGE_AND_WORD.xhtmlux5cux23id_2713}{\protect\hyperlink{23_NOTES.xhtmlux5cux23id_2714}{*\textsuperscript{10}}}
says the duchess, but the scribe falls down at her feet and pleads that
her son first ask for forgiveness before they leave the chapel. She
turns to Charles to plead in all earnesty, but he answers arrogantly and
loudly: ``Déa, madame, monseigneur m'a deffendu ses yeux et est indigné
sur moy, par quoy, après avoir eu celle deffense, je ne m'y retourneray
point si tost ains m'en yray a la garde de Dieu, je ne sçray
où.''\protect\hypertarget{21_Chapter_Thirteen__IMAGE_AND_WORD.xhtmlux5cux23id_2715}{\protect\hyperlink{23_NOTES.xhtmlux5cux23id_2716}{†\textsuperscript{11}}}
Suddenly the voice of the duke, who had remained seated on his
\emph{prie-Dieu},\textsuperscript{\protect\hypertarget{21_Chapter_Thirteen__IMAGE_AND_WORD.xhtmlux5cux23id_279}{\protect\hyperlink{23_NOTES.xhtmlux5cux23id_280}{15}}}
exhausted with rage, was heard .~.~. and the duchess cries out in mortal
fear to the scribe: ``Mon amy, tost tost ouvreznous, il nous convient
partir ou nous sommes
morts,''\protect\hypertarget{21_Chapter_Thirteen__IMAGE_AND_WORD.xhtmlux5cux23id_2717}{\protect\hyperlink{23_NOTES.xhtmlux5cux23id_2718}{‡\textsuperscript{12}}}

Philip is now under the spell of the hot blood of the Valois: having
returned to his chambers, the old duke falls into a kind of youthful
frenzy. Towards evening he secretly rides out of Brussels, alone and
insufficiently protected. ``Les jours pour celle heurre d'alors estoient
courts, et estoit jà basse vesprée quant ce prince droit-cy monta à
cheval, et ne demandoit riens autre fors estre emmy les champs seul et à
par luy. Sy porta ainsy l'anventure que ce propre jour-là, après un long
et âpre gel, il faisoit un releng, et par une longue épaisse bruyne, qui
a voit couru tout ce jour là, vesprée tourna en pluie bien menue, mais
très-mouillant et laquelle destrempoit les terres et romoit glasces
avecques vent qui s'y
entrebouta.''\protect\hypertarget{21_Chapter_Thirteen__IMAGE_AND_WORD.xhtmlux5cux23id_2719}{\protect\hyperlink{23_NOTES.xhtmlux5cux23id_2720}{§\textsuperscript{13}}}
Doesn't this sound like a Camille
Lemonnier?\textsuperscript{\protect\hypertarget{21_Chapter_Thirteen__IMAGE_AND_WORD.xhtmlux5cux23id_277}{\protect\hyperlink{23_NOTES.xhtmlux5cux23id_278}{16}}}

Then follows the description of the nocturnal wanderings through fields
and forests in which the most lively naturalism and a moralizing
rhetoric filled with a strange sense of its own importance enter into a
peculiar mixture. The duke wanders about tired and hungry. His cries are
unanswered. He is lured by a river that
\protect\hypertarget{21_Chapter_Thirteen__IMAGE_AND_WORD.xhtmlux5cux23page_346}{}{}looks
to him like a path, but his horse shys away just in time. He falls off
his horse and injures himself. He listens in vain for the crowing of a
rooster or the barking of a dog that could lead him back to human
habitations. Finally he sees a light shining and tries to get near it;
he loses sight of it, finds it again, and finally is able to reach it.
``Mais plus l'approchoit, plus sambloit hideuse chose et espoentable,
car feu partoit d'une mote d'en plus de mille lieux, avecques grosse
fumière, dont nul ne pensast à celle heure fors que ce fust ou
purgatoire d'aucune âme ou autre illusion de l'ennemy .~.~. ``
\protect\hypertarget{21_Chapter_Thirteen__IMAGE_AND_WORD.xhtmlux5cux23id_2721}{\protect\hyperlink{23_NOTES.xhtmlux5cux23id_2722}{*\textsuperscript{14}}}
He abruptly halts his horse. But then he recalls that charcoal makers
burn their coal deep in the woods. This was indeed such a charcoal fire.
But there was no house or cottage anywhere near. Only after having
wandered for a while more is he led by the barking of a dog to the hut
of a poor man where he finds rest and food.

Similarly characteristic passages from the work of Chastellain are the
descriptions of a duel between two burghers at Valenciennes, of the
nocturnal fight between the Frisian delegation in Haag and the
Burgundian noblemen whom they disturb in their nightly rest by playing
catch in an upper room in their wooden shoes, of the riot in 1467 in
Ghent when Charles's first visit as duke coincides with the fair in
Houthen from which the people return with the shrine of St.
Lieven.\textsuperscript{\protect\hypertarget{21_Chapter_Thirteen__IMAGE_AND_WORD.xhtmlux5cux23id_276}{\protect\hyperlink{23_NOTES.xhtmlux5cux23page_436}{17}}}

Time and again we see, by virtue of unintended trifling details, how
clearly the author really perceives all these external things. The duke,
confronted with the riot, faces ``multitude de faces en bacinets
enrouillés et dont les dedans estoient grignans barbes de vilain,
mordans
lèvres.''\protect\hypertarget{21_Chapter_Thirteen__IMAGE_AND_WORD.xhtmlux5cux23id_2723}{\protect\hyperlink{23_NOTES.xhtmlux5cux23id_2724}{†\textsuperscript{15}}}
The rogue who forces his way to the side of the duke at the window wears
an iron glove with a black finish. He bangs it on the windowsill to
compel
silence.\textsuperscript{\protect\hypertarget{21_Chapter_Thirteen__IMAGE_AND_WORD.xhtmlux5cux23id_274}{\protect\hyperlink{23_NOTES.xhtmlux5cux23id_275}{18}}}

This ability to narrate in crisp, simple words that which is perceived,
precisely and directly, corresponds in literature to that which in
painting is accomplished, with a perfect power of
expres\protect\hypertarget{21_Chapter_Thirteen__IMAGE_AND_WORD.xhtmlux5cux23page_347}{}{}sion,
by the tremendous visual sharpness of a Van Eyck. In literature that
realism is usually interfered with by conventional forms. It is retarded
in its expression and remains an exception in the midst of a mountain of
dry rhetoric while shining in painting like the blossoms on an apple
tree.

In this regard, painting is far ahead of literature in its means of
expression. In reproducing the effects of light painting already has an
astonishing virtuosity. Above all, the miniaturist strove to capture the
glow of a moment. In painting, this talent is seen to have first come to
its full development in the \emph{Nativity}
(\protect\hyperlink{20_ILLUSTRATIONS_FOLLOW_PAGE.xhtmlux5cux23id_28}{plate
31}) by Geertgen tot Sint Jans. The illuminators had already tried to
capture the play of the light of the torches on the armor of the
soldiers during the capture of Christ. The strange master who
illuminated King Rene's \emph{Cuer d'amours espris} succeeded in
depicting a radiant sunrise and the most mysterious effects of dusk. The
master of the \emph{Heures d' Ailly} already dares to try his hand on
the sun breaking through the clouds after a
storm.\textsuperscript{\protect\hypertarget{21_Chapter_Thirteen__IMAGE_AND_WORD.xhtmlux5cux23id_272}{\protect\hyperlink{23_NOTES.xhtmlux5cux23id_273}{19}}}

Literature had only primitive means at its disposal for the specific
reproduction of light effects. There is, to be sure, a high sensitivity
to the play of bright light, as mentioned above. There is even an
awareness of beauty as being first of all, a matter of shining elegance.
All the writers and poets of the fifteenth century like to mention the
glow of sunlight, candles, and weapons. But they do not go beyond simple
acknowledgment; there is as yet no literary procedure for the
description of such things.

We have to look elsewhere if we desire to find a literary equivalent for
the effect of light in painting. In literature, the momentary expression
is primarily achieved by a lively application of direct speech. There is
hardly another literature that so consistently reproduces speech
directly. This practice leads to tiresome abuse. Froissart and his
fellow spirits even dress explanations of political conditions in the
form of questions and answers. The endless dialogues, in their
ceremonial key and with their hollow sound, occasionally heighten rather
than interrupt monotony. But the writers do frequently succeed in
creating the illusion of directness and spontaneity completely
convincingly by the use of this technique. Froissart, above all, is a
master of lively dialogue.

``Lors il entendi les nouvelles que leur ville estoit prise.'' (The
whole speech is shouted.) `{}``Et de quel gens?' demande-il.
Respondirent ceulx qui à luy parloient: `Ce sont Bretons!'---'Ha,'
\protect\hypertarget{21_Chapter_Thirteen__IMAGE_AND_WORD.xhtmlux5cux23page_348}{}{}dist-il,
`Bretons sont mal gent, ils pilleront et ardront la ville et puis
partiront.' {[}The shouting continues{]} `Et quel cry crient-ils?' dist
le chevalier.---'Certes, sire, ils crient La
Trimouille!'\,''\protect\hypertarget{21_Chapter_Thirteen__IMAGE_AND_WORD.xhtmlux5cux23id_2725}{\protect\hyperlink{23_NOTES.xhtmlux5cux23id_2726}{*\textsuperscript{16}}}

Froissart employs the device of always having the partner in the
dialogue repeat in amazement the last word of the speaker so that a
certain element of haste is created.---`{}``Monseigneur, Gaston est
mort.'---'Mort?' dist le conte.---'Certes, mort est-il pour vray
monseigneur.'\,''\protect\hypertarget{21_Chapter_Thirteen__IMAGE_AND_WORD.xhtmlux5cux23id_2727}{\protect\hyperlink{23_NOTES.xhtmlux5cux23id_2728}{†\textsuperscript{17}}}
And elsewhere: ``Si luy demanda, en cause d'amours et de lignaige,
conseil.---'Conseil, respondi l'archevesque, `certes, beaux nieps, c'est
trop tard. Vous voulés clore l'estable quant le cheval est
perdu.'\,''\textsuperscript{\protect\hypertarget{21_Chapter_Thirteen__IMAGE_AND_WORD.xhtmlux5cux23id_270}{\protect\hyperlink{23_NOTES.xhtmlux5cux23id_271}{20}}}\protect\hypertarget{21_Chapter_Thirteen__IMAGE_AND_WORD.xhtmlux5cux23id_2729}{\protect\hyperlink{23_NOTES.xhtmlux5cux23id_2730}{‡:\textsuperscript{18}}}

Poetry too, makes generous use of this stylistic device. In a short
rhyme sequence the question and answer may alternate twice:

\emph{Mort, je me plaing}.---\emph{De qui?}---\emph{De toy}.

---\emph{Que t'ay je fait?}---\emph{Ma dame as pris}.

---\emph{C'est vérité}.---\emph{Dy moy pour quoy}.

\emph{Il me plaisoit}.---\emph{Tu as
mespris}.\textsuperscript{\protect\hypertarget{21_Chapter_Thirteen__IMAGE_AND_WORD.xhtmlux5cux23id_268}{\protect\hyperlink{23_NOTES.xhtmlux5cux23id_269}{21}}}\protect\hypertarget{21_Chapter_Thirteen__IMAGE_AND_WORD.xhtmlux5cux23id_2731}{\protect\hyperlink{23_NOTES.xhtmlux5cux23id_2732}{§\textsuperscript{19}}}

In this example the technique of repeated breaks in the dialogue is no
longer a means, but rather an end: it is a virtuosity. The poet Jean
Meschinot knew how to take this artistic device to its extreme. In a
ballade in which poor France remonstrates with her king (Louis XI) about
his guilt, the speaker changes in each of the thirty lines three or four
times. We have to admit that the effect of the poem as political satire
does not suffer from this peculiar form. The first segment reads as
follows:

\emph{\protect\hypertarget{21_Chapter_Thirteen__IMAGE_AND_WORD.xhtmlux5cux23page_349}{}{}Sire}
. . .---\emph{Que veux?}---\emph{Entendez} . .
.---\emph{Quoy?}---\emph{Mon cas}.

---\emph{Or dy}.---\emph{Je suys} . . .---\emph{Qui?}---\emph{La
destruicte France!}

---\emph{Par qui?}---\emph{Par vous}.---\emph{Comment?}---\emph{En tous
estats}.

---\emph{Tu mens}.---\emph{Non fais}.---\emph{Que le dit?}---\emph{Ma
souffrance}.

---\emph{Que souffres tu?}---\emph{Meschief}---\emph{Quel?}---\emph{A
oultrance}.

---\emph{Je n'en croy rien}.---\emph{Bien y pert}.---\emph{N'en dy
plus!}

---\emph{Las! siferay}.---\emph{Tu perds temps}.---\emph{Quelz abus!}

---\emph{Qu'ay-je mal fait?}---\emph{Contre paix}.---\emph{Et comment?}

---\emph{Guerroyant .~.~}.---\emph{Qui?}---\emph{Vos amys et congnus}.

---\emph{Parle plus beau}.---\emph{Je ne puis,
bonnement}.\textsuperscript{\protect\hypertarget{21_Chapter_Thirteen__IMAGE_AND_WORD.xhtmlux5cux23id_266}{\protect\hyperlink{23_NOTES.xhtmlux5cux23id_267}{22}}}\emph{\protect\hypertarget{21_Chapter_Thirteen__IMAGE_AND_WORD.xhtmlux5cux23id_2733}{\protect\hyperlink{23_NOTES.xhtmlux5cux23id_2734}{*\textsuperscript{20}}}}

There is another example of superficial naturalism in the literature of
the time. Though Froissart is concerned with the description of heroic
knightly deeds, what he describes almost against his will, one is
tempted to say, is the prosaic reality of war. Just as Cornmines, who
had his fill of chivalry, Froissart well describes the atmosphere of
fatigue, the futile pursuits, the random movements, and the restlessness
of a camp at night. He is a master at describing hesitation and
waiting.\textsuperscript{\protect\hypertarget{21_Chapter_Thirteen__IMAGE_AND_WORD.xhtmlux5cux23id_264}{\protect\hyperlink{23_NOTES.xhtmlux5cux23id_265}{23}}}

In his simple and precise reproduction of the external conditions of an
event, he does on occasion even attain an almost tragic power, as, for
example, in the report of the death of young Gaston Phébus, who had been
stabbed by his father in a
rage.\textsuperscript{\protect\hypertarget{21_Chapter_Thirteen__IMAGE_AND_WORD.xhtmlux5cux23id_262}{\protect\hyperlink{23_NOTES.xhtmlux5cux23id_263}{24}}}---The
work is so photographically exact that in his words the quality of the
narrator to whom he owed his endless \emph{faits divers} can be
detected. Everything he tells us about his traveling companion, the
knight Espaing, for example, is told very well, indeed.

Whenever literature is at work, simply observing and without the
encumbrance of convention, it is comparable to painting, even if it does
not attain its level.

We should not look for the literary descriptions that come closest
\protect\hypertarget{21_Chapter_Thirteen__IMAGE_AND_WORD.xhtmlux5cux23page_350}{}{}to
painting among the descriptions of nature precisely because we are
concerned with the unself-conscious observation of an individual event
about which we are told. Nature descriptions in the fifteenth century
are not as yet based on direct unself-conscious observation. Events are
related because they appear to be important. Their external
circumstances are reported just as a film sensitive to light makes a
record. A conscious literary procedure does not yet exist. A description
of nature, however, which in painting is merely a secondary appendage,
that is, presents itself totally unself-consciously, appears in
literature to be a conscious artistic device. Being of a purely
secondary character, descriptions of nature in painting could, by virtue
of this fact, retain their purity and simplicity. Since the background
was not important to the subject matter of the painting itself and
played no part in its hieratic style, the painters of the fifteenth
century were able to put into their landscapes a degree of harmonious
naturalness that was still prohibited in the strict disposition of the
subject matter constituting the main concern. An exact parallel to this
phenomenon is offered by Egyptian art: it abandons the code of forms
when modeling the miniature figure of a slave because the figure of a
slave is of no significance. The formal code usually requires that the
human figure be distorted, but figures created outside the code of forms
may, therefore, on occasion share in the simple natural faithfulness of
animal figures.

The weaker the relationship between the landscape and the main subject
matter to be depicted, the stronger the harmonious and natural qualities
of the painting as a whole. Behind the reckless, bizarre, and pompous
veneration of the kings in the \emph{Très-riches heures de
Chantilly}\textsuperscript{\protect\hypertarget{21_Chapter_Thirteen__IMAGE_AND_WORD.xhtmlux5cux23id_260}{\protect\hyperlink{23_NOTES.xhtmlux5cux23id_261}{25}}}
appears the view of Bourges in all the atmospheric and rhythmic
perfection of its dreamlike softness.

In literature, nature descriptions are still entirely dressed in the
garb of the pastorale. We have already drawn attention to the argument
at court over the pros and cons of the simple rustic life. Just as in
those days when Rousseau had his way, it was in good taste to declare
that one was tired of the vanity of courtly life and to affect a wise
flight from court replete with dark bread and the carefree love of Robin
and Marion. This was a sentimental reaction to the full-blooded splendor
and proud egotism of reality, not totally lacking in genuine sentiment,
yet in its major components merely a literary attitude.

\protect\hypertarget{21_Chapter_Thirteen__IMAGE_AND_WORD.xhtmlux5cux23page_351}{}{}The
love of nature belonged to this attitude. Its poetic expression is
conventional. Nature was a necessary element in the grand social game of
courtly-erotic culture. The terms for the beauty of flowers and the
songs of birds were intentionally cultivated in the customary forms that
every player understood. The description of nature in literature is thus
at an entirely different level than that in painting.

Disregarding for a moment pastorales and the opening stanzas of poems
with their obligatory motif of spring mornings, one rarely senses a
desire for descriptions of nature. Though occasionally a few such
descriptions may appear in literature, as for example, in the work of
Chastellain when he describes the beginning of spring thaw (and it is
precisely this sort of unintentional description that is by far the most
suggestive), it is pastorale poetry that remains the most likely place
to locate the rising literary feeling for nature. Next to the pages from
Alain Chartier, which we quoted above, to illustrate the effects of
elaborate details in general, we could place the poem ``Regnault et
Jehanneton,'' in which the kingly shepherd René dresses his love for
Jeanne de Laval. Here too, we find no coherent vision of a piece of
nature, no unity such as the painter could bestow on his landscape
through color and light, but only an unhurried enumeration of details.
First the singing of birds, one after the other, the insects, the frogs,
followed by the ploughing peasants:

\emph{Et d'autre part, les paisans au labour}

\emph{Si chantent hault, voire sans nul séjour, Resjoyssant}

\emph{Leurs beufs, lesquelx vont tout-bel charmant}

\emph{La terre grasse, qui le bon froment rent};

\emph{Et en ce point ilz les vont rescriant, Selon leur nom}:

\emph{A l'un fauveau et l'autre Grison},

\emph{Brunet, Blanchet, Blondeau ou Compaignon};

\emph{Puis les touchent tel foiz de l'aiguillon pour
avancer}.\textsuperscript{\protect\hypertarget{21_Chapter_Thirteen__IMAGE_AND_WORD.xhtmlux5cux23id_258}{\protect\hyperlink{23_NOTES.xhtmlux5cux23id_259}{26}}}\emph{\protect\hypertarget{21_Chapter_Thirteen__IMAGE_AND_WORD.xhtmlux5cux23id_2735}{\protect\hyperlink{23_NOTES.xhtmlux5cux23id_2736}{*\textsuperscript{21}}}}

\protect\hypertarget{21_Chapter_Thirteen__IMAGE_AND_WORD.xhtmlux5cux23page_352}{}{}Admittedly,
there is a certain freshness in all this and a happy tone, but it should
be compared with the calendar depictions of the breviaries. King René
presents us with all the ingredients for a good description of nature, a
palette of colors, so to speak, but nothing else. Moreover, in
describing the coming of dusk, his effort to express a certain mood is
unmistakable. The other birds are silent, but the quail still cries,
partridges scurry to their nests, deer and rabbits emerge. The sun just
a moment ago was still brightening the top of a tower, then the air
turns cold, owls and bats begin to make their fluttering sounds, and the
bell of the chapel sounds the Ave.

The calendar leaves of the \emph{Très-riches heures} provide an
opportunity to compare the same motif in the fine arts and in
literature. The splendid castles that fill in the background in the
works of the Limburg brothers are well known. The poetic works of
Eustace Deschamps may be cited as their literary counterparts. In a
number of seven short poems he sings the praises of different northern
French castles: Beauté, which was later to provide shelter for Agnes
Sorel, Bièvre, Cachan, Clermont, Nieppe, Noroy, and
Coucy.\textsuperscript{\protect\hypertarget{21_Chapter_Thirteen__IMAGE_AND_WORD.xhtmlux5cux23id_256}{\protect\hyperlink{23_NOTES.xhtmlux5cux23id_257}{27}}}
Deschamps would have to be a poet with much more powerful wings if he
were to achieve the same effect as the Limburg brothers managed to
convey in these most tender and delicate expressions of the art of
miniatures. On the September leaf
(\protect\hyperlink{20_ILLUSTRATIONS_FOLLOW_PAGE.xhtmlux5cux23id_29}{plate
32}), the castle of Saumur rises behind the grape harvesting scene as in
a dream: the tops of towers with their high wind vanes, the finials, the
lily ornaments of the spires, the twenty slender chimneys, all that
blossoms like a bed of wild white flowers in the dark blue air. Or take
the majestic broad somberness of the princely Lusignan on the March leaf
(\protect\hyperlink{20_ILLUSTRATIONS_FOLLOW_PAGE.xhtmlux5cux23id_2301}{plate
33}), the gloomy towers of Vincennes rising threateningly above the
dried foliage of the December forest
(\protect\hyperlink{20_ILLUSTRATIONS_FOLLOW_PAGE.xhtmlux5cux23id_30}{plate
34}).\textsuperscript{\protect\hypertarget{21_Chapter_Thirteen__IMAGE_AND_WORD.xhtmlux5cux23id_254}{\protect\hyperlink{23_NOTES.xhtmlux5cux23id_255}{28}}}

Did the poet, or at least this poet, possess equivalent means of
expression for evoking such images? Of course not. The description of
the architectural forms of a castle, such as in the poem on Bièvre
castle could not have any effect. As a matter of fact, all he has to
offer is a listing of the enjoyments offered by the castle. Naturally,
the painter, being outside the castle, looks at it, while the poet,
being inside, looks out:

\emph{\protect\hypertarget{21_Chapter_Thirteen__IMAGE_AND_WORD.xhtmlux5cux23page_353}{}{}Son
filz ainsné, daulphin de Viennois},

\emph{Donna le nom à ce lieu de Beauté}.

\emph{Et c'est bien drois, car moult est delectables}:

\emph{L'en y oit bien le rossignol chanter};

\emph{Marne l'ensaint, les haulz bois profitables}

\emph{Du noble parc puet l'en veoir branler} .~.~.

\emph{Les prez sont pres, les jardins deduisables},

\emph{Les beaus preaulx, fontenis bel et cler},

\emph{Vignes aussi et les terres arables}.

\emph{Moulins tournans, beaus plains à
regarder\protect\hypertarget{21_Chapter_Thirteen__IMAGE_AND_WORD.xhtmlux5cux23id_2737}{\protect\hyperlink{23_NOTES.xhtmlux5cux23id_2738}{*\textsuperscript{22}}}}

How different this effect from that of the miniatures! Yet in spite of
everything, painting and poem share both procedure and subject matter:
they list what is visible (and in the poem what is audible). The view of
the painter, however, is firmly focused on a particular and limited
complex: in his listing he has to present unity, limitation, and
coherence. Paul van Limburg may put all the details of winter in his
February picture
(\protect\hyperlink{20_ILLUSTRATIONS_FOLLOW_PAGE.xhtmlux5cux23id_2302}{plate
35}): the peasants warming themselves over a fire in the foreground, the
laundry hung for drying, the crows on a snowy ground, the sheepfold, the
beehives, the barrel, and the cart; all this and the entire country
background with the quiet village and the lonely house on the hill. Yet
the calm unity of the painting remains perfect. But the poet's view
keeps moving aimlessly; it finds no point of rest. He does not know how
to limit himself and does not convey a unified vision.

The form is overrun by the content. In literature, form and content are
both old; in painting, however, content is old while the form is new. In
painting, there is much more expression in form than in content. The
painter is able to put his entire unarticulated wisdom into the form:
the idea, the mood, the psychology can be reproduced without the trouble
of putting all this into words. The period is predominantly visually
oriented. This explains why the pictorial expression is so superior to
the literary: a literature whose perception is primarily visual fails.

\protect\hypertarget{21_Chapter_Thirteen__IMAGE_AND_WORD.xhtmlux5cux23page_354}{}{}The
poetry of the fifteenth century seems to live on almost no new ideas. A
general impotence to invent new forms prevails; all that is left is to
rework or modernize the old subject matter. There is a pause in all
thought; the mind, having completed the medieval edifice, is tired and
hesitates. Emptiness and barrenness everywhere. One despairs of the
world; everything regresses; a strong depression of the soul alone
predominates. Deschamps sighs:

\emph{Helas! on dit que je ne fais mès rien},

\emph{Qui jadis fis mainte chose nouvelle};

\emph{La raison est que je n'ay pas merrien}

\emph{Dont je fisse chose bonne ne
belle.\textsuperscript{\protect\hypertarget{21_Chapter_Thirteen__IMAGE_AND_WORD.xhtmlux5cux23id_252}{\protect\hyperlink{23_NOTES.xhtmlux5cux23id_253}{29}}}\protect\hypertarget{21_Chapter_Thirteen__IMAGE_AND_WORD.xhtmlux5cux23id_2739}{\protect\hyperlink{23_NOTES.xhtmlux5cux23id_2740}{*\textsuperscript{23}}}}

To us, nothing seems to provide stronger proof of stagnation and decay
than the fact that the old rhymed chivalric novels and other poems were
rendered in overly long equivalent prose. Yet in spite of everything,
this ``de-rhyming'' of the fifteenth century augurs a transition to a
new spirit. As late as the thirteenth century everything could be put
into rhymes, including matters concerning medicine, and natural history,
just as ancient Indian literature applied the verse form to all academic
pursuits. The fixed form signified that the oral presentation is the
intended form of communication. This is not a personal, emotional,
expressive presentation, but a mechanical recitation since in more
primitive literary epochs verses are virtually sung to a fixed and
monotonous melody. The new need for prose reveals a drive for
expression, an ascendancy of the more modern practice of reading over
the old form of oral presentation. This is also linked to the division
of the subject matter into smaller chapters with summarizing titles that
becomes generally accepted during the fifteenth century while earlier
books were less structured. Prose was confronted with relatively higher
demands than poetry; in the old rhymed forms everything is still
accepted as before; prose, in contrast, is the art form.

But the higher quality of prose in general is found in its formal
elements. It is just as little filled with new ideas as poetry.
Froissart
\protect\hypertarget{21_Chapter_Thirteen__IMAGE_AND_WORD.xhtmlux5cux23page_355}{}{}is
the perfect type of a mind that does not think in words but simply
depicts. He rarely has ideas, but only images of facts. He knows only a
few ethical motifs and emotions: fidelity, honor, greed, courage, and
all these only in their simplest form. He applies no theology, no
allegory, no mythology; if hard-pressed, some morality; he only
narrates, correctly, effortlessly, totally matter-of-factly, but is
lacking in content and he never grips our emotions except with the
mechanical superficiality of the way reality is reproduced in the
cinema. His contemplations are of an unparalleled banality; everything
is boring, nothing is more certain than death, sometimes, though, one
may win or lose. Particular notions are accompanied with automatic
certainty by the same set judgments: for example, whenever he speaks of
the Germans he maintains that they treat their prisoners badly and that
they are particularly
greedy.\textsuperscript{\protect\hypertarget{21_Chapter_Thirteen__IMAGE_AND_WORD.xhtmlux5cux23id_250}{\protect\hyperlink{23_NOTES.xhtmlux5cux23id_251}{30}}}

Even Froissart's frequently cited clever \emph{bons mots} lose, if read
in context, much of their impact. For example, it is frequently
considered to be an astute characterization of the first duke of
Burgundy, calculating and persistent Philip the Bold, when Froissart
calls him ``sage, froid et imaginatif, et qui sur ses besognes veoit au
loin.''\protect\hypertarget{21_Chapter_Thirteen__IMAGE_AND_WORD.xhtmlux5cux23id_2741}{\protect\hyperlink{23_NOTES.xhtmlux5cux23id_2742}{*\textsuperscript{24}}}
But Froissart applies this label to
everyone!\textsuperscript{\protect\hypertarget{21_Chapter_Thirteen__IMAGE_AND_WORD.xhtmlux5cux23id_248}{\protect\hyperlink{23_NOTES.xhtmlux5cux23id_249}{31}}}
Even the well-known ``Ainsi ot messire Jehan de Blois femme et guerre
qui trop luy
cousta,''\textsuperscript{\protect\hypertarget{21_Chapter_Thirteen__IMAGE_AND_WORD.xhtmlux5cux23id_246}{\protect\hyperlink{23_NOTES.xhtmlux5cux23id_247}{32}}}\protect\hypertarget{21_Chapter_Thirteen__IMAGE_AND_WORD.xhtmlux5cux23id_2743}{\protect\hyperlink{23_NOTES.xhtmlux5cux23id_2744}{†\textsuperscript{25}}}
if taken in context, does not actually make the point one reads into it.

One element is totally missing in Froissart: rhetoric. And, it is
precisely rhetoric that hid from his contemporaries the lack of new
ideas. They may be said to have reveled in the splendor of an artfully
embellished style: Ideas are regarded as new because of their stately
appearance. All terms wear brocaded garments. Terms of honor and duty
wear the colorful costume of the chivalric illusion. The sense of nature
is clothed in the costume of the pastorale, and love is mostly
restricted by the allegory of the \emph{Roman de la rose}. Not a single
thought is allowed to stand naked and free. Thoughts are rarely allowed
to move other than in the measured steps of endless processions.

\protect\hypertarget{21_Chapter_Thirteen__IMAGE_AND_WORD.xhtmlux5cux23page_356}{}{}The
rhetorical-ornamental element is also not lacking in the fine arts.
There are innumerable parts in particular paintings that can be called
painted rhetoric. As, for example, St. George on Van Eyck's
\emph{Madonna of the Canon Van de Paele}, who recommends the donor to
the Virgin. How clearly the artist tried to make the golden armor and
splendid helmet of St. George antique. How weakly rhetorical is his
gesture. The work of Paul van Limburg also displays this consciously
rhetorical element in the overloaded bizarre splendor, an unmistakable
effort at an exotic, theatrical expression, with which the three kings
make their appearance.

The poetry of the fifteenth century puts forth its most advantageous
side as long as it does not attempt to express profound ideas and is
freed also from the task of doing this beautifully. It is at its best
when it evokes, only for a moment, an image, a mood. Its effect depends
on its formal elements: the image, the tone, the rhythm. This is why
this poetry fails in large-scale, long-winded artful works where
rhythmic and tonal qualities are subordinate. However, this poetry is
fresh in those genres where form is the main concern: the rondeau, the
ballade, which build on a simple light idea and derive their power from
image, tone, and rhythm. These are the simple and directly creative
qualities of the folk song; whenever the art song gets closest to the
folk song it exudes its greatest charm.

During the fourteenth century a reversal occurs in the relationship
between lyrical poetry and music. During the older periods poems, even
nonlyrical ones, were inseparably tied to musical presentations. It is
even assumed that the \emph{chansons de geste}, those sequences often or
twelve syllables (just as the Indian sloka), were also sung in the same
manner. The normal type of the medieval poet is the one who has written
the poem and composed its accompanying music as well. This holds true in
the fourteenth century for a figure such as Guillaume de Machaut. It is
he who establishes both the most common lyrical forms of his times,
ballade, rondel, etc., and the form of the \emph{débat}. Machaut's
rondels and ballades are characterized by great simplicity, little
color, and light intellectual content. These are advantageous features
because in this instance the poem comprises only half of the work of
art. The song with music is the better the less colorful and expressive
it is, as, for example the simple rondel:

\emph{\protect\hypertarget{21_Chapter_Thirteen__IMAGE_AND_WORD.xhtmlux5cux23page_357}{}{}Au
departir de vous mon cuer vous lais}

\emph{Et je m'en vois dolans et esplourés}.

\emph{Pour vous servir, sans retraire jamais},

\emph{Au departir de vous mon cuer vous lais}.

\emph{Et par m'ame, je n'arai bien ne pais}

\emph{Jusqu'au retour, einsi desconfortés}.

\emph{Au departir de vous mon cuer vous lais}.

\emph{Et je m'en vois dolans et
esplourés}.\textsuperscript{\protect\hypertarget{21_Chapter_Thirteen__IMAGE_AND_WORD.xhtmlux5cux23id_244}{\protect\hyperlink{23_NOTES.xhtmlux5cux23id_245}{33}}}\protect\hypertarget{21_Chapter_Thirteen__IMAGE_AND_WORD.xhtmlux5cux23id_2745}{\protect\hyperlink{23_NOTES.xhtmlux5cux23id_2746}{*\textsuperscript{26}}}

Deschamps is no longer the composer of the music for his ballades. He is
therefore much more colorful and restless than Machaut, and for the same
reason frequently more interesting though of lesser poetic style. It
goes without saying that the fleeting, light poem that almost lacks any
content and is meant to be accompanied by a certain tune does not die
out even though the poets are no longer the composers of the melodies.
The rondel retains its style, as is shown by the following by Jean
Meschinot:

\emph{M'aimerez-vous bien},

\emph{Dictes, par vostre ame?}

\emph{Mais que je vous ame}

\emph{Plus que nulle rien},

\emph{M'aimerez-vous bien?}

\emph{Dieu mit tant de bien}

\emph{En vous, que c'est basme};

\emph{Pour ce je me clame}

\emph{Vostre. Mais combien}

\emph{M'aimerez-vous
bien?}\textsuperscript{\protect\hypertarget{21_Chapter_Thirteen__IMAGE_AND_WORD.xhtmlux5cux23id_242}{\protect\hyperlink{23_NOTES.xhtmlux5cux23id_243}{34}}}\protect\hypertarget{21_Chapter_Thirteen__IMAGE_AND_WORD.xhtmlux5cux23id_2747}{\protect\hyperlink{23_NOTES.xhtmlux5cux23id_2748}{†\textsuperscript{27}}}

\protect\hypertarget{21_Chapter_Thirteen__IMAGE_AND_WORD.xhtmlux5cux23page_358}{}{}The
clean, simple talent of Christine de Pisan is particularly suited for
these fleeting effects. She had the same facility in composing verses as
all her contemporaries: with little variation in form and idea, smooth
and colorless, calm and quiet, accompanied by a soft jesting melancholy.
These are truly literary poems. They are entirely courtly in thought and
tone. They remind us of the ivory plaques of the fourteenth century that
repeat, over and over, in purely conventional depictions the same
motifs: a hunting scene, an event from \emph{Tristan and Isolde} or from
the \emph{Roman de la rose}, graceful, cool, and charming. Where
Christine in her gentle refinement finds also the tone of the folk song,
the result is sometimes something totally pure. A reunion, for example:

\emph{Tu soies le très bien venu},

\emph{M'amour, or m'embrace et me baise}

\emph{Et comment t'es tu maintenu}

\emph{Puis ton depart? Sain et bien aise}

\emph{As tu esté tousjours? Ça vien}

\emph{Costé moy, te sié et me conte}

\emph{Comment t'a esté, mal ou bien},

\emph{Car de ce vueil savoir le compte}.

---\emph{Ma dame, a qui je suis tenu}

\emph{Plus que aultre, a nul n'en desplaise},

\emph{Sachés que desir m'a tenu}

\emph{Si court qu'oncques n'oz tel mesaise},

\emph{Ne plaisir ne prenoie en rien}

\emph{Loings de vous. Amours, qui cuers dompte},

\emph{Me disoit: ``Loyauté me tien},

\emph{Car de ce vueil savoir le compte.''}

---\emph{Dont m'as tu ton serment tenu},

\emph{Bon gré t'en sçray par saint Nicaise};

\emph{Et puis que sain es revenu}

\emph{Joye arons assez; or t'apaise}

\emph{Et me dis se scez de combien}

\emph{Le mal qu'en as eu a plus monte}

\emph{Que cil qu'a souffert le cuer mien},

\emph{Car de ce vueil savoir le compte}.

---\emph{Plus mal que vous, si com retien},

\emph{Ay eu, mais dites sanz mesconte},

\emph{\protect\hypertarget{21_Chapter_Thirteen__IMAGE_AND_WORD.xhtmlux5cux23page_359}{}{}Quarts
baisiers en aray je bien?}

\emph{Car de ce vueil savoir le
compte.\textsuperscript{\protect\hypertarget{21_Chapter_Thirteen__IMAGE_AND_WORD.xhtmlux5cux23id_240}{\protect\hyperlink{23_NOTES.xhtmlux5cux23id_241}{35}}}\protect\hypertarget{21_Chapter_Thirteen__IMAGE_AND_WORD.xhtmlux5cux23id_2749}{\protect\hyperlink{23_NOTES.xhtmlux5cux23id_2750}{*\textsuperscript{28}}}}

Or a lover's longing:

\emph{Il a au jour d'ui un mois}

\emph{Que mon ami s'en ala}.

\emph{Mon cuer remaint morne et cois},

\emph{Il a au jour d'ui un mois}.

\emph{``A Dieu, me dit, je m'en vois''};

\emph{Ne puis a moy ne parla},

\emph{Il a au jour d'ui un
mois}.\textsuperscript{\protect\hypertarget{21_Chapter_Thirteen__IMAGE_AND_WORD.xhtmlux5cux23id_238}{\protect\hyperlink{23_NOTES.xhtmlux5cux23id_239}{36}}}\protect\hypertarget{21_Chapter_Thirteen__IMAGE_AND_WORD.xhtmlux5cux23id_2751}{\protect\hyperlink{23_NOTES.xhtmlux5cux23id_2752}{†\textsuperscript{29}}}

A surrender:

\emph{``Mon ami, ne plourez plus};

\emph{Car tant me faittes pitié}

\emph{Que mon cuer se rent conclus}

\emph{A vostre doulce amistié}.

\emph{Reprenez autre maniere};

\emph{Pour Dieu, plus ne vous douiez},

\emph{Et me faittes bonne chiere}:

\emph{Je vueil quanque vous
voulez}.''\protect\hypertarget{21_Chapter_Thirteen__IMAGE_AND_WORD.xhtmlux5cux23id_2753}{\protect\hyperlink{23_NOTES.xhtmlux5cux23id_2754}{‡\textsuperscript{30}}}

\protect\hypertarget{21_Chapter_Thirteen__IMAGE_AND_WORD.xhtmlux5cux23page_360}{}{}The
tender, spontaneous femininity of these songs, lacking the
masculine-weighty fantastic reflections and the colorful dress of the
figures from the \emph{Roman de la rose}, makes them palatable to us.
All they offer to us is a single, just felt emotion; the theme has just
touched the heart and is then immediately turned into an image without
requiring any help from an idea to accomplish this. This poetry has that
quality that is characteristic of music and poetry of all periods in
which the inspiration is based exclusively on the simple vision of a
moment: the theme is pure and strong, the song opens with clear and firm
notes, like the song of a blackbird, but the poet or composer has
already spent himself after the stanza; the mood vanishes, the execution
is trapped in the quagmire of weak rhetoric. We meet with the same
disappointment that almost all poets of the fifteenth century have for
us.

Here is an example from Christine's ballades:

\emph{Quant chacun s'en revient de l'ost}

\emph{Pou quoy demeures tu derriere?}

\emph{Et si scez que m'amour entiere}

\emph{T'ay baillée en garde et
depost}.\textsuperscript{\protect\hypertarget{21_Chapter_Thirteen__IMAGE_AND_WORD.xhtmlux5cux23id_236}{\protect\hyperlink{23_NOTES.xhtmlux5cux23id_237}{37}}}\emph{\protect\hypertarget{21_Chapter_Thirteen__IMAGE_AND_WORD.xhtmlux5cux23id_2756}{\protect\hyperlink{23_NOTES.xhtmlux5cux23id_2757}{*\textsuperscript{31}}}}

We would now expect an accomplished medieval-French Leonore ballade. But
the poetess has nothing else to say but this opening. Another two short
unimportant stanzas put an end to the matter.

But how fresh is the opening of Froissart's ``Le debat dou cheval et dou
levrier'':

\emph{Froissart d'Escoce revenoit}

\emph{Sus un cheval cui gris estoit},

\emph{Un blanc levrier menoit en lasse}.

\emph{``Las,'' dist le levrier, ``Je me lasse},

\emph{Grisel, quant nous reposerons?}

\emph{Il est heure que nous
mongons.''\textsuperscript{38}\protect\hypertarget{21_Chapter_Thirteen__IMAGE_AND_WORD.xhtmlux5cux23id_2755}{\protect\hyperlink{23_NOTES.xhtmlux5cux23id_2758}{†\textsuperscript{}}}}

\protect\hypertarget{21_Chapter_Thirteen__IMAGE_AND_WORD.xhtmlux5cux23page_361}{}{}This
tone is, however, not carried through; the poem collapses immediately.
The theme is only sensed, not thought. The themes are, however, at times
splendidly suggestive. In Pierre Michault's ``Danse aux aveugles'' we
see mankind engaged in the eternal dance around the throne of love, good
fortune, and
death.\textsuperscript{\protect\hypertarget{21_Chapter_Thirteen__IMAGE_AND_WORD.xhtmlux5cux23id_233}{\protect\hyperlink{23_NOTES.xhtmlux5cux23id_234}{39}}}
But the execution is below standard from the very beginning. An
``Exclamacion des os Sainct Innocent,'' by an unknown poet, opens with
the shout of the bones in the bone houses of the famous cemetery:

\emph{Les os sommes de povres trepassez}.

\emph{Cy amassez par monceaulx compassez}.

\emph{Rompus, cassez, sans reigle ne compas} .~.~.
\textsuperscript{\protect\hypertarget{21_Chapter_Thirteen__IMAGE_AND_WORD.xhtmlux5cux23id_231}{\protect\hyperlink{23_NOTES.xhtmlux5cux23id_232}{40}}}\protect\hypertarget{21_Chapter_Thirteen__IMAGE_AND_WORD.xhtmlux5cux23id_2759}{\protect\hyperlink{23_NOTES.xhtmlux5cux23id_2760}{*\textsuperscript{33}}}

As the opening of the most somber lament of death, these lines are well
suited; but all of this leads to nothing other than a \emph{memento
mori} of the most ordinary kind.

These are all preliminary sketches suitable for pictorial works. For the
painter, such a single vision already contains the subject matter for a
fully elaborated picture, but for the poet, it remains insufficient.

Does all this mean that the power of painting in the fifteenth century
excels that of literature in every respect? No. There are always areas
in which literature has richer and more immediate means of expression at
its disposal than do the fine arts. Ridicule, above all, is one such
area. The fine arts, whenever they lower themselves to the level of
caricature, are able to express the comic sentiment only to a small
degree. Visually expressed, the comic element tends to become serious.
Only in cases where the admixture of the comic element in the complexity
of life is very small, where it is only seasoning and not the dominant
taste, are works of fine art able to keep pace with the spoken or
written word. Genre painting contains the comic element at its weakest.

Here the fine arts are still completely on their own ground. The
unbridled elaboration of detail that we already ascribed to the painting
of the fifteenth century shades imperceptibly into the leisurely
narration of trivia until it becomes genre. With the Master of Flémalle
detail becomes ``genre.'' His carpenter Joseph sits and makes
\protect\hypertarget{21_Chapter_Thirteen__IMAGE_AND_WORD.xhtmlux5cux23page_362}{}{}mousetraps
(\protect\hyperlink{20_ILLUSTRATIONS_FOLLOW_PAGE.xhtmlux5cux23id_31}{plate
36}); the character of genre is present in all the details. The step
from the purely painterly vision to that of genre is taken when Van Eyck
leaves a window shade open or paints a sideboard or a fireplace in the
manner of the Master of Flémalle.

But even in the case of genre, words have a dimension in which they
surpass depiction; they are capable of explicitly expressing states of
mind. In our discussion of Deschamps's descriptions of the beauty of
castles, we stated that they had actually failed and were infinitely far
behind that of which miniature art was capable. But we should compare
the ballade in which Deschamps describes, as in a genre picture, how he
lay ill in his shabby castle of
Fismes.\textsuperscript{\protect\hypertarget{21_Chapter_Thirteen__IMAGE_AND_WORD.xhtmlux5cux23id_229}{\protect\hyperlink{23_NOTES.xhtmlux5cux23id_230}{41}}}
The owls, crows, starlings, and sparrows who nest in the tower keep him
awake:

\emph{C'est une estrange melodie}

\emph{Qui ne semble pas grant deduit}

\emph{A gens qui sont en maladie}.

\emph{Premiers les corbes font sçavoir}

\emph{Pour certain si tost qu'il est jour}:

\emph{De fort crier font loeur pouoir}

\emph{Le gros, le gresle, sanz sejour};

\emph{Mieulx vauldroit le son d'un tabour}

\emph{Que telz cris de divers oyseaulx},

\emph{Puis vient la proie; vaches, veaulx},

\emph{Crians, muyans, et trop vuit},

\emph{Joint du moustier la sonnerie}.

\emph{Qui tout l'entendement destruit}

\emph{A gens qui sont en
maladie.\protect\hypertarget{21_Chapter_Thirteen__IMAGE_AND_WORD.xhtmlux5cux23id_2761}{\protect\hyperlink{23_NOTES.xhtmlux5cux23id_2762}{*\textsuperscript{34}}}}

Towards evening the owls come and scare the patient with their lamenting
cries that make him think of death:

\emph{\protect\hypertarget{21_Chapter_Thirteen__IMAGE_AND_WORD.xhtmlux5cux23page_363}{}{}C'est
froit hostel et mal reduit}

\emph{A gens qui sont en
maladie.\protect\hypertarget{21_Chapter_Thirteen__IMAGE_AND_WORD.xhtmlux5cux23id_2763}{\protect\hyperlink{23_NOTES.xhtmlux5cux23id_2764}{*\textsuperscript{35}}}}

As soon as even a glimmer of a comic element or even only a more
leisurely way of narrating begins to appear, the method of stringing
lists of things together is no longer so tiring. Lively lists of
bourgeoisie customs, long, leisurely descriptions of the female toilet
break the monotony. In his long allegorical poem \emph{Le espinette
amoureuse}, Froissart suddenly enchants us with a listing of about sixty
children's games that he used to play as a little boy in
Valenciennes.\textsuperscript{\protect\hypertarget{21_Chapter_Thirteen__IMAGE_AND_WORD.xhtmlux5cux23id_227}{\protect\hyperlink{23_NOTES.xhtmlux5cux23id_228}{42}}}
The literary service of the devil of gluttony has already begun. The
abundant feasts of Zola, Huysmans, and Anatole France have their
prototypes in medieval times. How appetizingly Froissart describes the
\emph{bon vivants} from Brussels who crowd around fat Duke Wenzel at the
battle of Basweiler; they have their servants with them, each with a
large wine flask tied to the saddle, bread and cheese, smoked salmon,
trouts and eel paste, all neatly wrapped in small napkins; they
considerably confuse the order of
battle.\textsuperscript{\protect\hypertarget{21_Chapter_Thirteen__IMAGE_AND_WORD.xhtmlux5cux23id_225}{\protect\hyperlink{23_NOTES.xhtmlux5cux23id_226}{43}}}

As a result of its proclivity for genre-like qualities, the literature
of that time is capable of turning even the most sober subject into
verse. Deschamps is able to plead for money in verse without lowering
his accustomed poetic standards; in a series of ballades he begs for an
official robe that had been promised to him, for firewood, a horse, and
back pay that is due
him.\textsuperscript{\protect\hypertarget{21_Chapter_Thirteen__IMAGE_AND_WORD.xhtmlux5cux23id_223}{\protect\hyperlink{23_NOTES.xhtmlux5cux23id_224}{44}}}

From this it is only a small step from genre types to the bizarre and
burlesque or, if you want, to the art of Breughel. In this form of the
comic, painting is still the equal of literature. The Breughel-like
element is already completely present in the art of the fourteenth
century. It is there in Melchior Broedelam's \emph{Flight into Egypt}
(\protect\hyperlink{20_ILLUSTRATIONS_FOLLOW_PAGE.xhtmlux5cux23id_5}{plate
5}), in Dijon; in the three sleeping soldiers in the \emph{Marys at the
Sepulchre} that is ascribed to Hubert van Eyck
(\protect\hyperlink{20_ILLUSTRATIONS_FOLLOW_PAGE.xhtmlux5cux23id_32}{plate
37}). No one is as forceful with intentionally bizarre elements as is
Paul van Limburg. A spectator in \emph{The Purification of the Virgin}
wears a crooked magician's hat a meter high with fathom-long sleeves
(\protect\hyperlink{20_ILLUSTRATIONS_FOLLOW_PAGE.xhtmlux5cux23id_33}{plate
38}). There is the burlesque in the baptismal fount, which is decorated
with three grotesque masks with their tongues out. In the background of
the \emph{Visitation} a hero in a tower does battle with a snail
\protect\hypertarget{21_Chapter_Thirteen__IMAGE_AND_WORD.xhtmlux5cux23page_364}{}{}and
another man pushes a wheelbarrow holding a pig playing a bagpipe
(\protect\hyperlink{20_ILLUSTRATIONS_FOLLOW_PAGE.xhtmlux5cux23id_34}{plate
39}).\textsuperscript{\protect\hypertarget{21_Chapter_Thirteen__IMAGE_AND_WORD.xhtmlux5cux23id_221}{\protect\hyperlink{23_NOTES.xhtmlux5cux23id_222}{45}}}

The literature of the fifteenth century is bizarre on almost every one
of its pages; its artificial style and the strangely fantastic costumes
of its allegories testify to this fact. Motifs, through which Breughel's
unleashed fantasy was later to vent its fury, as for example the quarrel
between Lent and Carnival, the struggle between meat and fish, were
already popular in the literature of the fifteenth century
(\protect\hyperlink{20_ILLUSTRATIONS_FOLLOW_PAGE.xhtmlux5cux23id_35}{plate
40}). Breughelish to a high degree is Deschamps's keen vision in which
he has the troops, gathering in Sluis for a campaign against England,
appear to the guard as an army of rats and mice.

\emph{``Avant, avant! tirez-vous ça}.

\emph{Je voy merveille, ce me semble.''}

---\emph{``Et quoy, guette, que vois-tu là?''}

\emph{``Je voy dix mille rats ensemble}

\emph{Et mainte souris qui s'assemble}

\emph{Dessus la rive de la mer} . . .
''\emph{\protect\hypertarget{21_Chapter_Thirteen__IMAGE_AND_WORD.xhtmlux5cux23id_2365}{\protect\hyperlink{23_NOTES.xhtmlux5cux23id_2366}{*\textsuperscript{36}}}}

In another instance he is sitting at a table at court, mournful and
unfocused, when suddenly he notices how the courtiers are eating: one
chews like a pig, that one nibbles like a mouse, this uses his teeth
like a saw, that one distorts his face, the beard of the other whips up
and down. ``While they ate, they looked like
devils.''\textsuperscript{\protect\hypertarget{21_Chapter_Thirteen__IMAGE_AND_WORD.xhtmlux5cux23id_219}{\protect\hyperlink{23_NOTES.xhtmlux5cux23id_220}{46}}}

Whenever literature describes the life of ordinary people, it
automatically resorts to that deft realism mixed with humor that was so
to blossom in the fine arts. Chastellain's description of the poor
peasant who gives shelter to the lost duke of Burgundy is like a work by
Breughel.\textsuperscript{\protect\hypertarget{21_Chapter_Thirteen__IMAGE_AND_WORD.xhtmlux5cux23id_217}{\protect\hyperlink{23_NOTES.xhtmlux5cux23id_218}{47}}}
Pastorales, in their description of eating, dancing, and wooing
shepherds, are time and again drawn from their basic sentimental and
romantic theme towards a fresh naturalism of slightly comic effect. We
count among this an interest in worn-out clothing that had already begun
to stir in both the literature and art of the fifteenth century. The
calendar miniatures emphasize with great enjoyment the threadbare knees
of the mowers in the wheat
\protect\hypertarget{21_Chapter_Thirteen__IMAGE_AND_WORD.xhtmlux5cux23page_365}{}{}or
paint the rags of beggars receiving alms. In all this we have the point
of origin of that line that, via Rembrandt's sketches
(\protect\hyperlink{20_ILLUSTRATIONS_FOLLOW_PAGE.xhtmlux5cux23id_36}{plate
41}) and Murillo's begging youths
(\protect\hyperlink{20_ILLUSTRATIONS_FOLLOW_PAGE.xhtmlux5cux23id_37}{plate
42}), leads to the street people of Steinlen
(\protect\hyperlink{20_ILLUSTRATIONS_FOLLOW_PAGE.xhtmlux5cux23id_38}{plate
43}).

But at the same time, one is struck by the great difference between the
pictorial and the literary. While the fine arts already perceive the
picturesque qualities of a beggar, that is, are sensitive to the magic
of form, literature, for the time being, is only concerned with the
beggar's significance, whether it laments, praises, or condemns him. In
the condemnations, in particular, are the archetypes of the realistic
literary depictions of poverty. Beggars had become terribly troublesome
towards the end of the medieval period. Their pitiful hordes took
shelter in the churches and disrupted church services with their cries
and noisy carryings on. Among them were to be found many evil people,
\emph{validi mendicantes}. In 1428 the cathedral chapter of Notre Dame
in Paris attempted in vain to restrict them to the church doors; only
later were they at least pushed from the choir into the nave of the
church.\textsuperscript{\protect\hypertarget{21_Chapter_Thirteen__IMAGE_AND_WORD.xhtmlux5cux23id_215}{\protect\hyperlink{23_NOTES.xhtmlux5cux23id_216}{48}}}
Deschamps never tires of making his hatred of these miserable people
known; he regards them all as hypocrites and cheaters. Beat and drive
them from the churches, he shouts, hang or burn
them!\textsuperscript{\protect\hypertarget{21_Chapter_Thirteen__IMAGE_AND_WORD.xhtmlux5cux23id_214}{\protect\hyperlink{23_NOTES.xhtmlux5cux23page_437}{49}}}
The road traveled from here to the modern literary description of misery
seems to be much longer than that which the fine arts had to traverse.
In painting, a new element entered all on its own; in literature, in
contrast, a newly matured social sensitivity had first to create
entirely new forms of expression.

Wherever the comic element, be it either weaker or stronger, coarser or
more subtle, was already provided by the appearance of the subject
matter itself, as in the case of genre or burlesque, the fine arts were
able to keep pace with the word. But there were spheres of humor that
were quite inaccessible to pictorial expression, where neither color nor
line were able to express anything. In all places where the comic
element is intended to provoke healthy laughter: in the comedy, farce,
burlesque, the joke, in short, in all the forms of the crudely comic,
literature rules unchallenged. A very particular spirit is heard in that
rich treasure of late medieval culture.

Even where ridicule sounds its most exquisite notes and waxes about the
most serious things in life, about love and one's own suffering, in the
realm of the faint smile literature is master. The
\protect\hypertarget{21_Chapter_Thirteen__IMAGE_AND_WORD.xhtmlux5cux23page_366}{}{}affected,
smoothed-over, and worn forms of eroticism undergo refinement and
purification by the admixture of irony.

Outside eroticism, irony is still awkward and naive. French authors
around 1400 occasionally were careful to warn their readers when they
spoke ironically. Deschamps praises the good times; everything is going
perfectly, everywhere peace and justice reign:

\emph{L'en me demande chascun jour}

\emph{Qu'il me semble du temps que voy},

\emph{Et je respons: c'est tout honour},

\emph{Loyauté, verité et Joy},

\emph{Largesce, prouesce et arroy},

\emph{Charité et biens qui s'advance}

\emph{Pour le commun; mais, par ma loy},

\emph{Je ne di pas quanque je
pense.\protect\hypertarget{21_Chapter_Thirteen__IMAGE_AND_WORD.xhtmlux5cux23id_2367}{\protect\hyperlink{23_NOTES.xhtmlux5cux23id_2368}{*\textsuperscript{37}}}}

In another place at the end of a ballade with the same tendency he says:
``Tous ces poins a rebours
retien.''\textsuperscript{\protect\hypertarget{21_Chapter_Thirteen__IMAGE_AND_WORD.xhtmlux5cux23id_212}{\protect\hyperlink{23_NOTES.xhtmlux5cux23id_213}{50}}}\protect\hypertarget{21_Chapter_Thirteen__IMAGE_AND_WORD.xhtmlux5cux23id_2369}{\protect\hyperlink{23_NOTES.xhtmlux5cux23id_2370}{†\textsuperscript{38}}}
Yet another has the refrain: ``C'est grant pechiez d'anisy blasmer le
monde'':\protect\hypertarget{21_Chapter_Thirteen__IMAGE_AND_WORD.xhtmlux5cux23id_2371}{\protect\hyperlink{23_NOTES.xhtmlux5cux23id_2372}{‡\textsuperscript{39}}}

\emph{Prince, s'il est par tout generalment}

\emph{comme je say, toute vertu habonde};

\emph{Mais tel m'orroit qui diroit: ``Il se ment''} .~.~.
\textsuperscript{\protect\hypertarget{21_Chapter_Thirteen__IMAGE_AND_WORD.xhtmlux5cux23id_210}{\protect\hyperlink{23_NOTES.xhtmlux5cux23id_211}{51}}}\protect\hypertarget{21_Chapter_Thirteen__IMAGE_AND_WORD.xhtmlux5cux23id_2373}{\protect\hyperlink{23_NOTES.xhtmlux5cux23id_2374}{§\textsuperscript{40}}}

A \emph{bel esprit} from the second half of the fifteenth century
entitles his epigram: ``Soubz une meschante paincture faicte de
mauvaises couleurs et du plus meschant peinctre du monde, par manière
d'yronnie par maître Jehan
Robertet.''\textsuperscript{\protect\hypertarget{21_Chapter_Thirteen__IMAGE_AND_WORD.xhtmlux5cux23id_208}{\protect\hyperlink{23_NOTES.xhtmlux5cux23id_209}{52}}}\protect\hypertarget{21_Chapter_Thirteen__IMAGE_AND_WORD.xhtmlux5cux23id_2375}{\protect\hyperlink{23_NOTES.xhtmlux5cux23id_2376}{**\textsuperscript{41}}}

But how subtle irony becomes as soon as it deals with love. In these
instances irony blends with gentle melancholy, with the
sub\protect\hypertarget{21_Chapter_Thirteen__IMAGE_AND_WORD.xhtmlux5cux23page_367}{}{}dued
tenderness with which the eroticism of the fifteenth century puts
something new into the old forms. The hard heart melts with a sob. A
note sounds that had not before been heard in earthly love: \emph{de
profundis.\protect\hypertarget{21_Chapter_Thirteen__IMAGE_AND_WORD.xhtmlux5cux23id_2869}{\protect\hyperlink{23_NOTES.xhtmlux5cux23id_2870}{*\textsuperscript{42}}}}

This is self-mockery, the figure of the ``amant remis et
renié''\protect\hypertarget{21_Chapter_Thirteen__IMAGE_AND_WORD.xhtmlux5cux23id_2867}{\protect\hyperlink{23_NOTES.xhtmlux5cux23id_2868}{†\textsuperscript{43}}}
that Villon embraces; these are the muted small songs of disillusionment
sung by Charles d'Orléans, the smile through tears: ``Je riz en
pleurs,''\protect\hypertarget{21_Chapter_Thirteen__IMAGE_AND_WORD.xhtmlux5cux23id_2865}{\protect\hyperlink{23_NOTES.xhtmlux5cux23id_2866}{‡\textsuperscript{44}}}
which was not only of Villon's invention. An old biblical adage, ``risus
dolore miscebitur et extrema gaudii luctus
occupat,''\textsuperscript{\protect\hypertarget{21_Chapter_Thirteen__IMAGE_AND_WORD.xhtmlux5cux23id_206}{\protect\hyperlink{23_NOTES.xhtmlux5cux23id_207}{53}}}\protect\hypertarget{21_Chapter_Thirteen__IMAGE_AND_WORD.xhtmlux5cux23id_2863}{\protect\hyperlink{23_NOTES.xhtmlux5cux23id_2864}{§\textsuperscript{45}}}
revived in a new application, acquired a bitter and refined emotional
meaning. Alain Chartier, the slick court poet, knows this motif as well
as Villon, the vagabond. Earlier than both, it is already found in Othe
de
Granson.\textsuperscript{\protect\hypertarget{21_Chapter_Thirteen__IMAGE_AND_WORD.xhtmlux5cux23id_204}{\protect\hyperlink{23_NOTES.xhtmlux5cux23id_205}{54}}}
The following examples are from Alain Chartier.

\emph{Je n'ay bouche qui puisse rire},

\emph{Que les yeulx ne la desmentissent}:

\emph{Car le cueur l'en vouldroit desdire}

\emph{Par les lermes qui des yeulx
issent.\protect\hypertarget{21_Chapter_Thirteen__IMAGE_AND_WORD.xhtmlux5cux23id_2861}{\protect\hyperlink{23_NOTES.xhtmlux5cux23id_2862}{**\textsuperscript{46}}}}

Or of a disconsolate lover:

\emph{De faire chiere s'efforcoit}

\emph{Et menoit une joye fainte},

\emph{Et à chanter con cueur forçoit}

\emph{Non pas pour plaisir, mais pour crainte},

\emph{Car tousjours ung relaiz de plainte}

\emph{S'enlassoit au ton de sa voix},

\emph{Et revenoit à son attainte}

\emph{Comme l'oysel au chant du
bois}.\textsuperscript{\protect\hypertarget{21_Chapter_Thirteen__IMAGE_AND_WORD.xhtmlux5cux23id_202}{\protect\hyperlink{23_NOTES.xhtmlux5cux23id_203}{55}}}\protect\hypertarget{21_Chapter_Thirteen__IMAGE_AND_WORD.xhtmlux5cux23id_2859}{\protect\hyperlink{23_NOTES.xhtmlux5cux23id_2860}{††\textsuperscript{47}}}

\protect\hypertarget{21_Chapter_Thirteen__IMAGE_AND_WORD.xhtmlux5cux23page_368}{}{}At
the end of a poem, the poet, in the style of a vagabond song, denies his
suffering:

\emph{C'est livret voult dicter et faire escripre}

\emph{Pour passer temps sans courage villain}

\emph{Ung simple clerc que l'en appelle Alain},

\emph{Qui parle ainsi d'amours pour oyr
dire}.\textsuperscript{\protect\hypertarget{21_Chapter_Thirteen__IMAGE_AND_WORD.xhtmlux5cux23id_200}{\protect\hyperlink{23_NOTES.xhtmlux5cux23id_201}{56}}}\emph{\protect\hypertarget{21_Chapter_Thirteen__IMAGE_AND_WORD.xhtmlux5cux23id_2857}{\protect\hyperlink{23_NOTES.xhtmlux5cux23id_2858}{*\textsuperscript{48}}}}

Or, in a detailed imaginative scene at the end of King Rene's endless
\emph{Cuer d'amours espris}, the chamberlain, holding a candle, checks
to see if the king has lost his heart, but cannot discover any hole in
his side:

\emph{Sy me dist tout en soubzriant}

\emph{Que je dormisse seulement}

\emph{Et que n'avoye nullement}

\emph{Pour ce mal garde de
morir}.\textsuperscript{\protect\hypertarget{21_Chapter_Thirteen__IMAGE_AND_WORD.xhtmlux5cux23id_198}{\protect\hyperlink{23_NOTES.xhtmlux5cux23id_199}{57}}}\protect\hypertarget{21_Chapter_Thirteen__IMAGE_AND_WORD.xhtmlux5cux23id_2855}{\protect\hyperlink{23_NOTES.xhtmlux5cux23id_2856}{†\textsuperscript{49}}}

The old conventional forms had acquired a new freshness by the new
sentiment. No one has taken the customary personification of sentiments
as far as Charles d'Orléans. He views his own heart as a separate being:

\emph{Je suys celluy au cueur vestu de noir} .~.~.
\textsuperscript{\protect\hypertarget{21_Chapter_Thirteen__IMAGE_AND_WORD.xhtmlux5cux23id_196}{\protect\hyperlink{23_NOTES.xhtmlux5cux23id_197}{58}}}\protect\hypertarget{21_Chapter_Thirteen__IMAGE_AND_WORD.xhtmlux5cux23id_2853}{\protect\hyperlink{23_NOTES.xhtmlux5cux23id_2854}{‡\textsuperscript{50}}}

The older lyric, even the \emph{dolce stil nuova}, had taken
personification with sacred seriousness, but in the poems of Charles
d'Orléans the line between seriousness and mockery can no longer be
drawn; he exaggerates personification without losing the subtle feeling
in the process:

\emph{Un jour à mon cueur devisoye}

\emph{Qui en secret à moy parloit},

\emph{\protect\hypertarget{21_Chapter_Thirteen__IMAGE_AND_WORD.xhtmlux5cux23page_369}{}{}Et
en parlant lui demandoye}

\emph{Se point d'espargne fait avoit}

\emph{D'aucuns biens quant Amours servoit}:

\emph{Il me dist que très voulentiers}

\emph{La vérité m'en compteroit},

\emph{Mais qu'eust visité ses papiers}.

\emph{Quant ce m'eut dit, It print sa voye}

\emph{Et d'avecques moy se partoit}.

\emph{Après entrer je le véoye}

\emph{En ung comptouer qu'il avoit}:

\emph{Là, de ça et de là queroit},

\emph{En cherchant plusieurs vieulx caïers}

\emph{Car le vray monstrer me vouloit},

\emph{Mais qu'eust visitez ses papiers} . .
.\textsuperscript{\protect\hypertarget{21_Chapter_Thirteen__IMAGE_AND_WORD.xhtmlux5cux23id_194}{\protect\hyperlink{23_NOTES.xhtmlux5cux23id_195}{59}}}\protect\hypertarget{21_Chapter_Thirteen__IMAGE_AND_WORD.xhtmlux5cux23id_2851}{\protect\hyperlink{23_NOTES.xhtmlux5cux23id_2852}{*\textsuperscript{51}}}

In the above passage, the comic element predominates, in that which
follows, it is seriousness:

\emph{Ne hurtez plus à l'uis de ma pensée},

\emph{Soing et Soucy, sans tant vous travailler};

\emph{Car elle dort et ne veult s'esveiller},

\emph{Toute la nuit en peine a despensée}.

\emph{En dangier est, se s'elle n'est bien pansée};

\emph{Cessez, cessez, laissez la sommeiller};

\emph{Ne hurtez plus à l'uis de ma pensée},

\emph{Soing et Soucy, sans tant vous travailler .} .
.\textsuperscript{\protect\hypertarget{21_Chapter_Thirteen__IMAGE_AND_WORD.xhtmlux5cux23id_192}{\protect\hyperlink{23_NOTES.xhtmlux5cux23id_193}{60}}}\protect\hypertarget{21_Chapter_Thirteen__IMAGE_AND_WORD.xhtmlux5cux23id_2849}{\protect\hyperlink{23_NOTES.xhtmlux5cux23id_2850}{†\textsuperscript{52}}}

\protect\hypertarget{21_Chapter_Thirteen__IMAGE_AND_WORD.xhtmlux5cux23page_370}{}{}Disguising
the lovers in churchly forms not only serves obscenely graphic language
and crude irreverence as in the \emph{Cent nouvelles nouvelles}, it also
provides the most tender, nearly elegiac, love poem produced by the
fifteenth century with its form: ``L'amant rendu cordelier à
l'observance d'amours.'' By the mixture with the seasoning of blasphemy,
so favored by the mind of the fifteenth century, softly sad eroticism
acquired an even more pronounced taste.

The motif of the lovers as members of a spiritual order had already
given rise, in the circle of Charles d'Orléans, to a poetic brotherhood
that called itself \emph{les amoureux de l'observance}. But was it
really Martial d'Auvergne who elaborated this motif into this moving
poem that towers so much above his other work?

The wretched and disappointed lover renounces the world in the strange
monastery where only distressed lovers, \emph{les amoureux martyrs}, are
accepted. In a calm dialogue with the prior he tells the gentle story of
his unrequited love and is admonished to forget it. Beneath the
medieval-satirical dress here is fully formed the mood of a Watteau and
of the Pierrot cult, only without
moonlight.\textsuperscript{\protect\hypertarget{21_Chapter_Thirteen__IMAGE_AND_WORD.xhtmlux5cux23id_190}{\protect\hyperlink{23_NOTES.xhtmlux5cux23id_191}{61}}}
Did she not have the habit, asks the prior, of casting a loving glance
in your direction or saying in passing a ``Dieu
gart?''\protect\hypertarget{21_Chapter_Thirteen__IMAGE_AND_WORD.xhtmlux5cux23id_2847}{\protect\hyperlink{23_NOTES.xhtmlux5cux23id_2848}{*\textsuperscript{53}}}
It never went that far, answers the lover; but at night I stood for
three hours at her door and looked up at the eaves:

\emph{Et puis, quant je oyoye les verrières}

\emph{De la maison qui cliquetoient},

\emph{Lors me sembloit que mes priéres}

\emph{Exaussées d'elle sy
estoient}.\protect\hypertarget{21_Chapter_Thirteen__IMAGE_AND_WORD.xhtmlux5cux23id_2845}{\protect\hyperlink{23_NOTES.xhtmlux5cux23id_2846}{†\textsuperscript{54}}}

``Were you sure that she noticed you?'' asks the prior.

Se \emph{m'aist Dieu, j'estoye tant ravis},

\emph{Que ne savoye mon sens ne estre},

\emph{Car, sans parler, m'estoit advis}

\emph{Que le vent ventoit sa fenestre}

\emph{Et que m'avoit bien peu congnoistre},

\emph{En disant bas: ``Doint bonne nuyt,''}

\emph{\protect\hypertarget{21_Chapter_Thirteen__IMAGE_AND_WORD.xhtmlux5cux23page_371}{}{}Et
Dieu scet se j'estoye grant maistre}

\emph{Après cela toute la
nuyt}.\textsuperscript{\protect\hypertarget{21_Chapter_Thirteen__IMAGE_AND_WORD.xhtmlux5cux23id_188}{\protect\hyperlink{23_NOTES.xhtmlux5cux23id_189}{62}}}\emph{\protect\hypertarget{21_Chapter_Thirteen__IMAGE_AND_WORD.xhtmlux5cux23id_2843}{\protect\hyperlink{23_NOTES.xhtmlux5cux23id_2844}{*\textsuperscript{55}}}}

He slept wonderfully in this bliss:

\emph{Tellement estoie restauré}

\emph{Que, sans tourner ne travailler},

\emph{Ja faisoie un somme doré},

\emph{Sans point la nuyt me resveiller},

\emph{Et puis, avant que m'abilier},

\emph{Pour en rendre à Amours louanges},

\emph{Baisoie troys fois mon orillier},

\emph{En riant à par moy aux
anges}.\protect\hypertarget{21_Chapter_Thirteen__IMAGE_AND_WORD.xhtmlux5cux23id_2841}{\protect\hyperlink{23_NOTES.xhtmlux5cux23id_2842}{†\textsuperscript{56}}}

During his solemn acceptance into the order, his lady, who had scorned
him, faints, and a small golden heart enameled with tears, which he had
given her as a gift, falls out of her dress:

\emph{Les aultres, pour leur mal couvrir}

\emph{A force leurs cueurs retenoient},

\emph{Passans temps a clone et rouvrir}

\emph{Les heures qu'en leurs mains tenoient},

\emph{Dont souvent les feuilles tournoient}

\emph{En signe de devocion};

\emph{Mais les deulz et pleurs que menoient}

\emph{Monstroient bien leur
affection}.\protect\hypertarget{21_Chapter_Thirteen__IMAGE_AND_WORD.xhtmlux5cux23id_2839}{\protect\hyperlink{23_NOTES.xhtmlux5cux23id_2840}{‡\textsuperscript{57}}}

When the prior finally gets around to enumerating his new duties and
warns him never to listen to the nightingale, never to slumber
\protect\hypertarget{21_Chapter_Thirteen__IMAGE_AND_WORD.xhtmlux5cux23page_372}{}{}under
``eglantiers et aubespines,'' and, above all, never to look into the
eyes of women, then the poem laments on the theme of \emph{doux yeux} in
an endless melody with ever varying stanzas:

\emph{Doux yeulx qui tousjours vont et viennent};

\emph{Doulx yeulx eschauffans le plisson},

\emph{De ceulx qui amoureux deviennent .~.~}.

\emph{Doux yeulx a cler esperlissans},

\emph{Qui dient: C'est fait quant tu vouldras},

\emph{A ceulx qu'ils sentent bien puissans} .~.~.
\textsuperscript{\protect\hypertarget{21_Chapter_Thirteen__IMAGE_AND_WORD.xhtmlux5cux23id_186}{\protect\hyperlink{23_NOTES.xhtmlux5cux23id_187}{63}}}\protect\hypertarget{21_Chapter_Thirteen__IMAGE_AND_WORD.xhtmlux5cux23id_2837}{\protect\hyperlink{23_NOTES.xhtmlux5cux23id_2838}{*\textsuperscript{58}}}

During the fifteenth century all the conventional forms of eroticism
are, imperceptibly, permeated with this gentle, subdued note of relaxed
melancholy. The old satire of cynical derision of women is thus suddenly
pierced by an entirely different mood: in the \emph{Quinze joyes de
manage} the earlier imbecile reviling of women is tempered by a note of
quiet disillusionment and depression. This imparts to it the sadness of
a modern novel about marriage; the ideas are shallow and hastily
expressed; the conversations, in their tenderness, do not reflect
malicious intent.

In matters of the means of expression for love, literature had the
schooling of centuries behind it. Its masters were such diverse spirits
as Plato and Ovid, the troubadours and the minstrels, Dante and Jean de
Meun. The fine arts, in contrast, were still unusually primitive in this
arena and remained so for a long time. Only during the eighteenth
century does the artistic representation of love catch up with the
literary description in matters of refinement and wealth of expression.
The painting of the fifteenth century is still incapable of being
frivolous and sentimental. It is still denied the expression of the
roguish element. The picture of the maiden Lysbet van Durenvoode, by an
unknown master before 1430
(\protect\hyperlink{20_ILLUSTRATIONS_FOLLOW_PAGE.xhtmlux5cux23id_39}{plate
44}), shows a figure of such strict dignity that she was once described
as the figure of the donor of a devotional picture, but the text on the
banderole she holds in her hand reads: ``Mi verdriet lange te hopen,
\protect\hypertarget{21_Chapter_Thirteen__IMAGE_AND_WORD.xhtmlux5cux23page_373}{}{}Wie
is hi die syn hert hout
open?''\protect\hypertarget{21_Chapter_Thirteen__IMAGE_AND_WORD.xhtmlux5cux23id_2835}{\protect\hyperlink{23_NOTES.xhtmlux5cux23id_2836}{*\textsuperscript{59}}}
This art knows both chaste and obscene elements. It does not yet possess
the means to express the intermediate stages. It says little about the
life of love and does so in naive and innocent forms. We do have, of
course, to remind ourselves anew that most art of this sort that existed
has been lost. It would be of extraordinary interest to us if we could
compare the sort of nudity painted by Van Eyck in his \emph{Bath of
Women} or that by Rogier where two young men peep laughingly through a
chink (both pictures are described by Fazio) with that of Van Eyck's
Adam and Eve in the Ghent Altarpiece. Incidentally, the erotic element
is not altogether lacking in Adam and Eve; the artist undoubtedly
followed the conventional code of female beauty with respect to the
small breasts that are placed too high, the long slender arms, the
protruding belly. But how naively all this is done; he has neither the
ability nor the slightest desire to titillate the senses.---Charm,
however, is said to be the essential element of the \emph{Little Love
Magic} that is labeled ``from the school of Jan van
Eyck,''\textsuperscript{\protect\hypertarget{21_Chapter_Thirteen__IMAGE_AND_WORD.xhtmlux5cux23id_184}{\protect\hyperlink{23_NOTES.xhtmlux5cux23id_185}{64}}}
a room in which a girl, naked as is proper for magic, attempts to force
the appearance of her beloved by sorcery
(\protect\hyperlink{20_ILLUSTRATIONS_FOLLOW_PAGE.xhtmlux5cux23id_40}{plate
45}). Nudity is presented in this instance in that same unpretentious
concupiscence that we encounter in the nude pictures by Cranach.

It was not prudishness that so limited the role of depiction in
eroticism. The late Middle Ages display a peculiar contrast between a
strongly developed sense of modesty and a surprising lack of restraint.
For the latter we need not cite any examples; it shows itself
everywhere. The sense of modesty, on the other hand, can be seen, for
example, in the fact that the victims in the worst scenes of murder or
pillage are shown in their shirts or underwear. The Burgher of Paris is
never so disgusted as when this rule is violated: ``Et ne volut pas
convoitise que on leur laissast neis leurs brayes, pour tant qu'ilz
vaulsissent 4 deniers, qui estoit un des plus grans cruaultés et
inhumanité chrestienne à aultre de quoy on peut
parler.''\textsuperscript{\protect\hypertarget{21_Chapter_Thirteen__IMAGE_AND_WORD.xhtmlux5cux23id_182}{\protect\hyperlink{23_NOTES.xhtmlux5cux23id_183}{65}}}\protect\hypertarget{21_Chapter_Thirteen__IMAGE_AND_WORD.xhtmlux5cux23id_2833}{\protect\hyperlink{23_NOTES.xhtmlux5cux23id_2834}{†\textsuperscript{60}}}
In the report of the cruelty of the Bastard of
Vauru\textsuperscript{\protect\hypertarget{21_Chapter_Thirteen__IMAGE_AND_WORD.xhtmlux5cux23id_180}{\protect\hyperlink{23_NOTES.xhtmlux5cux23id_181}{66}}}
to a poor woman he is disgusted that the knavish villain cut her dress
\protect\hypertarget{21_Chapter_Thirteen__IMAGE_AND_WORD.xhtmlux5cux23page_374}{}{}off
slightly below the waist far more than he is about the cruelties
inflicted upon the other
victims.\textsuperscript{\protect\hypertarget{21_Chapter_Thirteen__IMAGE_AND_WORD.xhtmlux5cux23id_178}{\protect\hyperlink{23_NOTES.xhtmlux5cux23id_179}{67}}}---Given
the prevailing sense of modesty, it is even more remarkable that the
female nude, still little used in art, was given such free reign in the
\emph{tableau vivant}. No entry procession lacked the presentation,
\emph{personnages}, of naked goddesses or nymphs, as those Dürer saw
during Charles V's entry into Antwerp in
1520\textsuperscript{\protect\hypertarget{21_Chapter_Thirteen__IMAGE_AND_WORD.xhtmlux5cux23id_176}{\protect\hyperlink{23_NOTES.xhtmlux5cux23id_177}{68}}}
and who prompted Hans Makart's erroneous assumption that the women had
been part of the procession. The presentations were performed on small
stages at certain locations, sometimes even in water, as for example,
the sirens who swam by the bridge over the river Leie ``toutes nues et
échevelées ainsi comme on les
peint,''\protect\hypertarget{21_Chapter_Thirteen__IMAGE_AND_WORD.xhtmlux5cux23id_2831}{\protect\hyperlink{23_NOTES.xhtmlux5cux23id_2832}{*\textsuperscript{61}}}
during the entry of Philip the Good into Ghent in
1457.\textsuperscript{\protect\hypertarget{21_Chapter_Thirteen__IMAGE_AND_WORD.xhtmlux5cux23id_174}{\protect\hyperlink{23_NOTES.xhtmlux5cux23id_175}{69}}}
The Judgment of Paris was the most popular subject of such
performances.---They should be understood as nothing more than
manifestations of a naive popular sensuality rather than of a Greek
sense of beauty or of a trivial lack of modesty. Jean de Roye describes
the sirens, placed not too far from the figure of the crucified one
between the thieves, with the following words: ``Et sy avoit encores
trois bien belles filles, faisans personnages de seraines toutes nues,
et leur veoit on le beau tetin droit, separé, rond et dur, qui estoit
chose bien plaisant, et disoient de petiz motetz et bergeretes; et près
d'eulx jouoient plusieurs bas instrumens qui rendoient de grandes
melodies.''\textsuperscript{\protect\hypertarget{21_Chapter_Thirteen__IMAGE_AND_WORD.xhtmlux5cux23id_172}{\protect\hyperlink{23_NOTES.xhtmlux5cux23id_173}{70}}}\protect\hypertarget{21_Chapter_Thirteen__IMAGE_AND_WORD.xhtmlux5cux23id_2829}{\protect\hyperlink{23_NOTES.xhtmlux5cux23id_2830}{†\textsuperscript{62}}}
Molinet reports with what delight the people viewed the \emph{Judgement
of Paris} during the 1494 entry of Philip the Beautiful into Antwerp:
``mais le hourd ou les gens donnoient le plus affectueux regard fut sur
l'histoire des trois déesses, qui l'on véoit au nud et de femmes
vives.''\textsuperscript{\protect\hypertarget{21_Chapter_Thirteen__IMAGE_AND_WORD.xhtmlux5cux23id_170}{\protect\hyperlink{23_NOTES.xhtmlux5cux23id_171}{71}}}\protect\hypertarget{21_Chapter_Thirteen__IMAGE_AND_WORD.xhtmlux5cux23id_2827}{\protect\hyperlink{23_NOTES.xhtmlux5cux23id_2828}{†\textsuperscript{63}}}
Just consider how great is the distance from any pure sense of beauty
that is shown in Lille in 1468 when the performance of that scene during
the entry of Charles the Bold was parodied by a fat Venus, an emaciated
Juno, a hunchbacked Minerva, all wearing golden crowns on their
heads.\textsuperscript{\protect\hypertarget{21_Chapter_Thirteen__IMAGE_AND_WORD.xhtmlux5cux23id_168}{\protect\hyperlink{23_NOTES.xhtmlux5cux23id_169}{72}}}
The presentation of nudity remained the fashion until late in the
sixteenth century. During the entry of the
\protect\hypertarget{21_Chapter_Thirteen__IMAGE_AND_WORD.xhtmlux5cux23page_375}{}{}duke
of Brittany into Reims in 1532, a naked Ceres with a
Bacchus\textsuperscript{\protect\hypertarget{21_Chapter_Thirteen__IMAGE_AND_WORD.xhtmlux5cux23id_166}{\protect\hyperlink{23_NOTES.xhtmlux5cux23id_167}{73}}}
could be seen, and even William of Orange, on his entry into Brussels on
September 18, 1578, was still treated to the sight of an Andromeda, ``a
maiden in chains, as naked as she was born from her Mother's womb; she
seemed a marble statue.'' So comments Johan Baptista Houwaert, who had
arranged the
tableaux.\textsuperscript{\protect\hypertarget{21_Chapter_Thirteen__IMAGE_AND_WORD.xhtmlux5cux23id_164}{\protect\hyperlink{23_NOTES.xhtmlux5cux23id_165}{74}}}

The backwardness of pictorial expression compared to literature is,
incidentally, not limited to the comic, the sentimental, and the erotic.
Pictorial expression reaches a limit whenever it is no longer supported
by the predominantly visual orientation that we had been inclined to
regard as the reason for the superiority of painting over literature in
general. Whenever something more than a directly clear image of the
natural world was needed, painting fails step by step and it becomes
suddenly very evident how well founded Michelangelo's charge was: this
art seeks to depict many things simultaneously in perfection; while a
single one of them would be important enough to devote all energies to
it.

Let us return to a painting by Jan van Eyck. His art is unsurpassed as
long as it works close-up, microscopically, so to say: in the facial
features, for example, or the material of the garments and the jewels.
The absolutely keen observation suffices in these cases. But as soon as
the perceived reality has to be made part of a different equation, so to
speak, as is the case in the presentation of buildings and landscapes,
one becomes aware of some weaknesses, in spite of the intrinsic charm
exuded by the early perspective. For example, there can be a certain
lack of cohesion, a somewhat deficient disposition. The more the
depiction is conditioned by intentional composition, and the more a
pictorial form has to be created in order to do justice to the
particular subject matter of a painting, the more evident the failure
becomes.

No one will deny the claim that in the illustrated breviaries, the
calendar pages are superior to those with depictions of stories from the
Holy Scriptures. In the case of the former, direct perception and its
narrative reproduction sufficed. But for the composition of an important
action, or of a presentation with movement and many persons, a feeling
for rhythmic construction and cohesion is required above all. Giotto had
once possessed it and Michelangelo was to correctly handle it again. But
the characteristic of the art of the fifteenth century was its
many-faceted quality. Only where this many-faceted quality itself
becomes cohesiveness is that effect of a
\protect\hypertarget{21_Chapter_Thirteen__IMAGE_AND_WORD.xhtmlux5cux23page_376}{}{}high
degree of harmony attained as in the \emph{Adoration of the Lamb}. There
we actually find rhythm, an incomparably strong rhythm, the triumphant
rhythm of all those groups converging on the center. But this rhythm is,
so to speak, derived from purely mathematical coordination, from the
multifacetedness itself. Van Eyck avoids the difficulties of composition
by depicting only scenes of strict quietude; he achieves a static but
not a dynamic harmony.

This, above all, marks the great distance between Rogier van der Weyden
and Van Eyck. Rogier limits himself so that he may find rhythm; he does
not always attain it, but he is always aspiring to it.

There was an old, strict tradition of depiction with respect to the most
important themes of the Holy Scripture. The painter was no longer
required to find for himself the arrangement of his
painting.\textsuperscript{\protect\hypertarget{21_Chapter_Thirteen__IMAGE_AND_WORD.xhtmlux5cux23id_162}{\protect\hyperlink{23_NOTES.xhtmlux5cux23id_163}{75}}}
Some of these \emph{sujets} came close to a rhythmic structure of their
own. In scenes like that of a pietà, a deposition, the adoration of the
shepherds, rhythm comes naturally. Just recall such works as the
\emph{Pietà} by Rogier van der Weyden in Madrid, those of the Avignon
school in the Louvre and in Brussels, those of Petrus Christus, Geertgen
tot Sint Jans, the \emph{Belles heures
d'Ailly}.\textsuperscript{\protect\hypertarget{21_Chapter_Thirteen__IMAGE_AND_WORD.xhtmlux5cux23id_160}{\protect\hyperlink{23_NOTES.xhtmlux5cux23id_161}{76}}}

But wherever the scene becomes livelier, such as the mockery of Christ,
the carrying of the cross, the adoration of the Kings, difficulties of
composition mount and a certain unrest, an insufficient cohesion of
visual conception, results. In cases where the iconographic norms of the
church leave the artist to his own devices, he finds himself in a rather
helpless position. The judicial scenes to which Dirk Bouts and Gerard
David still gave a certain ceremonial arrangement were already rather
weak as far as composition is concerned. Composition becomes awkward and
clumsy in the \emph{Martyrdom of St. Erasmus} in Louvan, and in that of
\emph{St. Hippolytus}, who is quartered by horses, in Bruges, the flawed
structure is positively repugnant.

Whenever never before seen fantasies are to be depicted, the art of the
fifteenth century veers into the ridiculous. Great painting was
protected against this by its strict \emph{sujets}; but the art of book
illustration did not have the luxury of avoiding the depiction of all
the mythological and allegorical fantasies made available by literature.
The illustration of the \emph{Epitre d'Othéa à
Hector},\textsuperscript{\protect\hypertarget{21_Chapter_Thirteen__IMAGE_AND_WORD.xhtmlux5cux23id_158}{\protect\hyperlink{23_NOTES.xhtmlux5cux23id_159}{77}}}
a detailed mythological fantasy probably by Christine de Pisan, provides
a good example. Here we have the most awkward cases. The Greek gods have
large wings attached to the backs of their ermine coats or Burgundian
\protect\hypertarget{21_Chapter_Thirteen__IMAGE_AND_WORD.xhtmlux5cux23page_377}{}{}robes
of state; the entire design as expression misses the mark: Minos;
Saturn, who devours his children; Midas, who distributes the prizes, all
are fashioned equally naively, yet whenever the illustrator was allowed
to delight in the background with a small shepherd and his sheep, or a
little hill with a gallows and a wheel, he displays his usual
skill.\textsuperscript{\protect\hypertarget{21_Chapter_Thirteen__IMAGE_AND_WORD.xhtmlux5cux23id_156}{\protect\hyperlink{23_NOTES.xhtmlux5cux23id_157}{78}}}
But this is where the positive power of these artists has its limits. In
the final analysis, they are just about as limited as the poets in their
freely creative formative work.

The imagination had been led into a dead-end street by allegorical
presentation. An image cannot be freely fashioned because it has to
completely comprise the thought, and the thought is restrained in its
flight by the image. Imagination had become accustomed to transposing
thought as soberly as possible, and without a sense of style, to the
picture. \emph{Temperantia} wears a clockwork on her head to indicate
her nature. The illustrator of the \emph{Epitre d'Othéa} simply uses a
small wall clock for this purpose, which he also places on the wall of
Philip the
Good.\textsuperscript{\protect\hypertarget{21_Chapter_Thirteen__IMAGE_AND_WORD.xhtmlux5cux23id_154}{\protect\hyperlink{23_NOTES.xhtmlux5cux23id_155}{79}}}
If a keen and naturally observant mind like Chastellain paints
allegorical figures from his own experience they turn out to be
extraordinarily affected. For example, he envisions four ladies who
accuse him in his \emph{Exposition sur vérité mal prise}, which he wrote
to justify himself in the wake of his daring political poem \emph{Le dit
de
Vérité}.\textsuperscript{\protect\hypertarget{21_Chapter_Thirteen__IMAGE_AND_WORD.xhtmlux5cux23id_152}{\protect\hyperlink{23_NOTES.xhtmlux5cux23id_153}{80}}}
These ladies are called Indignation, Reprobation, Accusation, and
Vindication. We cite his description of the
second:\textsuperscript{\protect\hypertarget{21_Chapter_Thirteen__IMAGE_AND_WORD.xhtmlux5cux23id_150}{\protect\hyperlink{23_NOTES.xhtmlux5cux23id_151}{81}}}

Ceste dame droit-cy se monstroit avoir les conditions seures, raisons
moult aguës et mordantes; grignoit les dens et mâchoit ses lèvres:
niquoit de la teste souvent; et monstrant signe d'estre arguëresse,
sauteloit sur ses pieds et tournoit l'un costé puis çà l'autre costé
puis là; portoit manière d'impatience et de contradiction: le droit oeil
avoit clos et l'autre ouvert; avoit un sacq plein de livres devant lui,
dont les uns mit en son escours comme chéris, les autres jetta au loin
par despit; deschira papiers et feuilles; quayers jetta au loin par
despit; deschira papiers et feuilles; quayers jetta au feu félonnement;
rioit sur les uns et les baisoit, sur les autres cracha par vilennie et
les foula des pieds; avoit une plume en sa main, pleine d'encre, de
laquelle roioit maintes ecritures notables .~.~. d'une esponge aussy
noircissoit aucunes ymages, autres esgratinoit aux ongles .~.~. et les
tierces
\protect\hypertarget{21_Chapter_Thirteen__IMAGE_AND_WORD.xhtmlux5cux23page_378}{}{}rasoit
toutes au net et les planoit comme pour les mettres hors de mémoire; et
se monstroit dure et felle ennemie à beaucoup de gens de bien, plus
volontairement que par
raison.\protect\hypertarget{21_Chapter_Thirteen__IMAGE_AND_WORD.xhtmlux5cux23id_2825}{\protect\hyperlink{23_NOTES.xhtmlux5cux23id_2826}{*\textsuperscript{64}}}

But in another passage he observes how Lady \emph{Paix} spreads her
coat, raises it into the air, and how the coat then divides into four
other ladies: \emph{Paix de coeur, Paix de bouche, Paix de semblant,
Paix de vrax
effet}.\textsuperscript{\protect\hypertarget{21_Chapter_Thirteen__IMAGE_AND_WORD.xhtmlux5cux23id_148}{\protect\hyperlink{23_NOTES.xhtmlux5cux23id_149}{82}}}\protect\hypertarget{21_Chapter_Thirteen__IMAGE_AND_WORD.xhtmlux5cux23id_2823}{\protect\hyperlink{23_NOTES.xhtmlux5cux23id_2824}{†\textsuperscript{65}}}
In yet another of his allegories, female figures appear called
``Pesanteur de tes Pays, Diverse condition et qualité de tes divers
peuples, L'envie et haine des François et des voisines
nations,''\protect\hypertarget{21_Chapter_Thirteen__IMAGE_AND_WORD.xhtmlux5cux23id_2821}{\protect\hyperlink{23_NOTES.xhtmlux5cux23id_2822}{‡\textsuperscript{66}}}
as if political editorials could be
allegorized.\textsuperscript{\protect\hypertarget{21_Chapter_Thirteen__IMAGE_AND_WORD.xhtmlux5cux23id_146}{\protect\hyperlink{23_NOTES.xhtmlux5cux23id_147}{83}}}
That all these figures were not envisioned, but invented, is
demonstrated, on top of all this, by the fact that they display their
names on banderoles; he does not fashion these figures directly from his
living imagination, but presents them as in painting or a performance.

In \emph{La mort du duc Philippe, mystère par manière de lamentation} he
sees his duke as a flask filled with precious ointment that is suspended
on a thread from the sky; the earth has nourished the flask on its
breast.\textsuperscript{\protect\hypertarget{21_Chapter_Thirteen__IMAGE_AND_WORD.xhtmlux5cux23id_144}{\protect\hyperlink{23_NOTES.xhtmlux5cux23id_145}{84}}}
Molinet sees how Christ as the pelican (a customary image) not only
feeds his young with his blood but also washes the mirror of death with
it.\textsuperscript{\protect\hypertarget{21_Chapter_Thirteen__IMAGE_AND_WORD.xhtmlux5cux23id_142}{\protect\hyperlink{23_NOTES.xhtmlux5cux23id_143}{85}}}

The inspiration of beauty is lost here: a playful and false joke, an
exhausted spirit awaits new fertilization. In the dream motif,
consistently used as the framework of an action, we rarely sense
\protect\hypertarget{21_Chapter_Thirteen__IMAGE_AND_WORD.xhtmlux5cux23page_379}{}{}genuine
dream elements such as occur so movingly in Dante and Shakespeare. Not
even the illusion that the poet has really experienced his conception as
a vision is always maintained: Chastellain calls himself ``l'inventeur
ou le fantasieur de ceste
vision.''\textsuperscript{\protect\hypertarget{21_Chapter_Thirteen__IMAGE_AND_WORD.xhtmlux5cux23id_140}{\protect\hyperlink{23_NOTES.xhtmlux5cux23id_141}{86}}}\protect\hypertarget{21_Chapter_Thirteen__IMAGE_AND_WORD.xhtmlux5cux23id_2819}{\protect\hyperlink{23_NOTES.xhtmlux5cux23id_2820}{*\textsuperscript{67}}}

On the barren field of allegorical depiction only mockery can grow new
blossoms. As soon as an allegory is seasoned with humor, it still
manages to produce a certain effect. Deschamps asks the physician how
Virtues and Law are faring:

\emph{Phisicien, comment fait Droit?}

---\emph{Sur m'ame, il est en petit point} .~.~.

---\emph{Que fait Raison?} .~.~.

\emph{Perdu a son entendement},

\emph{Elle parle mais faiblement},

\emph{Et Justice est toute ydiote} .~.~.
\textsuperscript{\protect\hypertarget{21_Chapter_Thirteen__IMAGE_AND_WORD.xhtmlux5cux23id_138}{\protect\hyperlink{23_NOTES.xhtmlux5cux23id_139}{87}}}\protect\hypertarget{21_Chapter_Thirteen__IMAGE_AND_WORD.xhtmlux5cux23id_2817}{\protect\hyperlink{23_NOTES.xhtmlux5cux23id_2818}{†\textsuperscript{68}}}

The different types of fantasy are scrambled together without any sense
of style. There is no more bizarre product than a political pamphlet in
the garb of the pastorale. The unknown poet, who calls himself Burarius,
has in ``Le pastoralet'' described all the slander heaped by the
Burgundians on the party of Orléans and has done it in the tone of a
pastorale. He makes Orléans, John the Fearless, and their entire proud
and grim entourage into gentle shepherds. The coats of the shepherds
have either fleur-de-lis or lions rampant on them. There are ``bergiers
à long jupel.'' These are the
clergy.\textsuperscript{\protect\hypertarget{21_Chapter_Thirteen__IMAGE_AND_WORD.xhtmlux5cux23id_136}{\protect\hyperlink{23_NOTES.xhtmlux5cux23id_137}{88}}}
The shepherd, Tristifer-Orléans, takes bread and cheese away from the
others, also their apples, nuts, and flutes; he takes the bells from the
sheep. He threatens those who resist with his big shepherd's staff until
he himself is slain with one. Occasionally the poet almost forgets his
somber theme and indulges in the sweetest pastorale only to interrupt
this fantasy in a strange way with bitter political
slander.\textsuperscript{\protect\hypertarget{21_Chapter_Thirteen__IMAGE_AND_WORD.xhtmlux5cux23id_134}{\protect\hyperlink{23_NOTES.xhtmlux5cux23id_135}{89}}}

Molinet mixes all the motifs of faith, war, coats of arms, and love in a
proclamation from the Creator to all true lovers:

\emph{\protect\hypertarget{21_Chapter_Thirteen__IMAGE_AND_WORD.xhtmlux5cux23page_380}{}{}Nous
Dieu d'amours, créateur, roy de gloire}

\emph{Salut à tous vrays amans d'humble affaire}

\emph{Comme il soit vray depuis la victoire}

\emph{De nostre filz sur le mont de Calvaire}

\emph{Plusieurs souldars par peu de congnoissance}

\emph{De noz armes, font au dyable allyance} .~.~.
\protect\hypertarget{21_Chapter_Thirteen__IMAGE_AND_WORD.xhtmlux5cux23id_2815}{\protect\hyperlink{23_NOTES.xhtmlux5cux23id_2816}{*\textsuperscript{69}}}

Then the proper coat of arms is described for them: a shield of silver,
the upper part of gold with five wounds; the Church Militant is granted
the right to recruit and take into service all those who want to rally
to the arms:

\emph{Mais qu'en pleurs et en larmes},

\emph{De cueur contrict etfoy sans
abuser}.\textsuperscript{\protect\hypertarget{21_Chapter_Thirteen__IMAGE_AND_WORD.xhtmlux5cux23id_132}{\protect\hyperlink{23_NOTES.xhtmlux5cux23id_133}{90}}}\protect\hypertarget{21_Chapter_Thirteen__IMAGE_AND_WORD.xhtmlux5cux23id_2813}{\protect\hyperlink{23_NOTES.xhtmlux5cux23id_2814}{†\textsuperscript{70}}}

The devices employed by Molinet to gain the praise of his contemporaries
as an inspired rhetorician and poet appear to us like the last
degenerative stage of a form of expression shortly before its demise. He
engages in the most tasteless wordplay: ``Et ainsi demoura l'escluse en
paix qui lui fut incluse, car la guerre fut d'elle excluse plus
solitaire que rencluse.
''\textsuperscript{\protect\hypertarget{21_Chapter_Thirteen__IMAGE_AND_WORD.xhtmlux5cux23id_130}{\protect\hyperlink{23_NOTES.xhtmlux5cux23id_131}{91}}}\protect\hypertarget{21_Chapter_Thirteen__IMAGE_AND_WORD.xhtmlux5cux23id_2811}{\protect\hyperlink{23_NOTES.xhtmlux5cux23id_2812}{‡\textsuperscript{71}}}
In the introduction to his moralized prose version of the \emph{Roman de
la rose}, he plays with his name Molinet: ``Et affin que je ne perde le
froment de ma labeur, et que la farine que en sera molue puisse avoir
fleur salutaire, j'ay intencion, se Dieu m'en donne la grace, de tourner
et convertir soubz mes rudes meulles le vicieux au vertueux, le corporel
en l'espirituel, la mondanité en divinité, et souverainement de la
moraliser. Et par ainsi nous tirerons le miel hors de la dure pierre, et
las rose vermeille hors des poignans espines, où nous trouverons grain
et graine, fruict, fleur et feuille, très souefve odeur, odorant
verdure, verdoyant fioriture, florissant norriture, nourrissant fruict
et
fruc\protect\hypertarget{21_Chapter_Thirteen__IMAGE_AND_WORD.xhtmlux5cux23page_381}{}{}tifiant
pasture.''\textsuperscript{\protect\hypertarget{21_Chapter_Thirteen__IMAGE_AND_WORD.xhtmlux5cux23id_128}{\protect\hyperlink{23_NOTES.xhtmlux5cux23id_129}{92}}}\protect\hypertarget{21_Chapter_Thirteen__IMAGE_AND_WORD.xhtmlux5cux23id_2809}{\protect\hyperlink{23_NOTES.xhtmlux5cux23id_2810}{*\textsuperscript{72}}}
How much this looks like the end of an age---how threadbare and spent!
But this is precisely what the contemporary admired as something new;
medieval poetry actually didn't know the play on words, it played more
with images, as does Olivier de la Marche, who was Molinet's kindred
spirit and admirer:

\emph{Là prins fièvre de souvenance}

\emph{Et catherre de desplaisir},

\emph{Une migraine de souffrance},

\emph{Colicque d'une impascience},

\emph{Mal de dens non a soustenir},

\emph{Mon cueur ne porroit plus souffrir}

\emph{Les regretz de ma destinee}

\emph{Par douleur non
accoustumée}.\textsuperscript{\protect\hypertarget{21_Chapter_Thirteen__IMAGE_AND_WORD.xhtmlux5cux23id_126}{\protect\hyperlink{23_NOTES.xhtmlux5cux23id_127}{93}}}\protect\hypertarget{21_Chapter_Thirteen__IMAGE_AND_WORD.xhtmlux5cux23id_2807}{\protect\hyperlink{23_NOTES.xhtmlux5cux23id_2808}{†\textsuperscript{73}}}

Meschinot is just as much a slave to the feeble allegory as La Marche;
the eyeglasses of his ``Lunettes des princes'' are Prudence and Justice;
Force is the frame, Temperance the nail that holds everything together.
Raison hands the poet the pair of glasses together with directions for
their use. Sent by heaven, Raison enters his mind in order to hold a
feast there, but finds everything spoiled by Despair, which leaves
nothing there ``pour disner
bonne-ment.''\textsuperscript{\protect\hypertarget{21_Chapter_Thirteen__IMAGE_AND_WORD.xhtmlux5cux23id_124}{\protect\hyperlink{23_NOTES.xhtmlux5cux23id_125}{94}}}\protect\hypertarget{21_Chapter_Thirteen__IMAGE_AND_WORD.xhtmlux5cux23id_2805}{\protect\hyperlink{23_NOTES.xhtmlux5cux23id_2806}{‡\textsuperscript{74}}}

Everything seems degenerated and decayed. And yet, we have already
entered an age when the new spirit of the Renaissance is at large in the
land. Its great new inspiration, the pure new form---where do we find
it?
