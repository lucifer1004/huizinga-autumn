\chapter{THE DEPICTION OF THE SACRED}

THE DEPICTION OF DEATH MAY SERVE AS AN EXAMple of late medieval thought
in general, which frequently moves living thought from the abstract in
the direction of the pictorial as if the whole of intellectual life
sought concrete expression, as if the notion of gold were immediately
minted into coin. There is an unlimited desire to bestow form on
everything that is sacred, to give any religious idea a material shape
so that it exists in the mind like a crisply printed picture. This
tendency towards pictorial expression is constantly in jeopardy of
becoming petrified.

The development of external folk piety in late medieval times cannot be
put more succinctly than it is by Jacob Burckhardt in his
\emph{Weltgeschichtliche Betrachtungen}.

A powerful religion permeates all the affairs of life and lends color to
every movement of the spirit, to every element of culture.

In time, of course, those things come to react upon religion, and indeed
its living core may be stifled by the ideas and images it once took into
its sphere. The ``sanctification of all the concerns of life'' has its
fateful aspect.

And further:

Now, no religion has ever been quite independent of the culture of its
people and its time. It is just when religion exercises sovereign sway
through the agency of literally written scriptures, when all life seems
to revolve round that centre, ``when it is interwoven with life as a
whole,'' that life will most infallibly react upon it. Later, these
intimate connections with culture are no longer useful to it, but simply
a source of danger; nevertheless, a religion will always act in this way
as long as it is
alive.\textsuperscript{\protect\hypertarget{13_Chapter_Six__THE_DEPICTION_OF_TH.xhtmlux5cux23id_1269}{\protect\hyperlink{23_NOTES.xhtmlux5cux23id_1270}{1}}}

\protect\hypertarget{13_Chapter_Six__THE_DEPICTION_OF_TH.xhtmlux5cux23page_174}{}{}The
life of medieval Christendom is permeated in all aspects by religious
images. There is nothing and no action that is not put in its
relationship to Christ and faith. Indeed, everything is tuned to a
religious understanding of all things in a tremendous outpouring of
faith. But in this supernaturalized atmosphere, the religious
tension\textsuperscript{\protect\hypertarget{13_Chapter_Six__THE_DEPICTION_OF_TH.xhtmlux5cux23id_1267}{\protect\hyperlink{23_NOTES.xhtmlux5cux23id_1268}{2}}}
of true transcendence, the stepping away from the material, cannot
always occur. If this tension is missing, then everything intended to
awaken a consciousness of God rigidifies into terrible banality, being
an astonishing this-worldliness in other-worldly
terms.\textsuperscript{\protect\hypertarget{13_Chapter_Six__THE_DEPICTION_OF_TH.xhtmlux5cux23id_1265}{\protect\hyperlink{23_NOTES.xhtmlux5cux23id_1266}{3}}}
Even in a true saint such as Henry
Suso,\textsuperscript{\protect\hypertarget{13_Chapter_Six__THE_DEPICTION_OF_TH.xhtmlux5cux23id_1263}{\protect\hyperlink{23_NOTES.xhtmlux5cux23id_1264}{4}}}
in whom transcendence was probably never absent for a moment, the
distance from the sublime to the ridiculous is very short to our no
longer medieval sensibilities. He is sublime when, as the knight
Boucicaut honored all women for the sake of his earthly mistress, Suso
does so for the sake of Mary, or steps aside into the mud for a poor
woman. He follows the customs of chivalry and celebrates his bride,
Wisdom, on festivals with a wreath and a song. When he hears a
\emph{Minnesong} he immediately allegorizes it in terms of Wisdom. But
what are we to make of the following? At table Suso cuts his apple into
four parts: three parts he eats in the name of the Trinity, the fourth
he eats in memory of ``the love with which the Heavenly Mother gave the
infant Jesus a little apple to eat.'' And for this reason, he eats that
fourth part with its peel, since little boys do not like their apples
peeled. A few days after Christmas---at a time when the infant was too
young to eat apples---he does not eat the fourth part, but offers it to
Mary so that she will give it to her son. Whatever he drinks he takes in
five swallows for the sake of the five wounds of the Lord, but since
blood and water flowed from Christ's side, he takes the second swallow
twice.\textsuperscript{\protect\hypertarget{13_Chapter_Six__THE_DEPICTION_OF_TH.xhtmlux5cux23id_1261}{\protect\hyperlink{23_NOTES.xhtmlux5cux23id_1262}{5}}}
Here is the ``sanctification of all aspects of life'' in the most
extreme form.

Disregarding for the moment the degree of devotion, and speaking of the
liturgical forms within which medieval piety existed, we can see them as
examples of the excesses of religious life, provided that this is not
done from a dogmatic Protestant position. There had evolved within the
church a growth in the number of usages, concepts, and observances that,
leaving aside the quality of the ideas that motivated them, terrified
the serious theologians. The reforming spirit of the fifteenth century
did not turn against these new practices so much because they were
unholy or superstitious, but because they overloaded belief itself. The
signs of God's
ever-\protect\hypertarget{13_Chapter_Six__THE_DEPICTION_OF_TH.xhtmlux5cux23page_175}{}{}ready
mercy had steadily increased in number: next to the sacraments could be
found benedictions; relics had become charms; the power of prayer was
formalized in the rosary. The
colorful\textsuperscript{\protect\hypertarget{13_Chapter_Six__THE_DEPICTION_OF_TH.xhtmlux5cux23id_1259}{\protect\hyperlink{23_NOTES.xhtmlux5cux23id_1260}{6}}}
gallery of saints had acquired even more color and life. And even though
theology clamored for a precise distinction between sacrament and
sacramentalia, what means were there to keep the masses from basing
their faith and hope on the magical and the gaudy?
Gerson\textsuperscript{\protect\hypertarget{13_Chapter_Six__THE_DEPICTION_OF_TH.xhtmlux5cux23id_1257}{\protect\hyperlink{23_NOTES.xhtmlux5cux23id_1258}{7}}}
met someone in Auxerre who claimed that All Fool's Day, on which the
winter months were commemorated in churches and monasteries, was just as
sacred as the Feast of Mary's
Conception.\textsuperscript{\protect\hypertarget{13_Chapter_Six__THE_DEPICTION_OF_TH.xhtmlux5cux23id_1255}{\protect\hyperlink{23_NOTES.xhtmlux5cux23id_1256}{8}}}
Nicholas de Clémanges wrote a treatise against establishing any new
festivals. He declared that many of the new ones were of an entirely
apocryphal nature, and he approved of the action of the Bishop of
Auxerre which had abolished most of
them.\textsuperscript{\protect\hypertarget{13_Chapter_Six__THE_DEPICTION_OF_TH.xhtmlux5cux23id_1253}{\protect\hyperlink{23_NOTES.xhtmlux5cux23id_1254}{9}}}
Pierre d'Ailly in \emph{De
Reformatione}\textsuperscript{\protect\hypertarget{13_Chapter_Six__THE_DEPICTION_OF_TH.xhtmlux5cux23id_1251}{\protect\hyperlink{23_NOTES.xhtmlux5cux23id_1252}{10}}}
opposes the increase in numbers of churches, festivals, saints, and days
of rest. He deplores the plethora of pictures and painted objects and
the overly tedious minutiae of the liturgies. He objects to the
inclusion of apocryphal writings in the liturgy of the festivals, the
introduction of new hymns and prayers or other arbitrary innovations and
to the all too rigid intensification of vigils, prayers, fasts, and
abstentions. There was a tendency to link every detail of the veneration
of the Holy Mother to a special service. There were special masses,
later abolished by the church, of Mary's piety, of her seven pains, of
all festivals of Mary together, of her sisters Mary Jacoby and Mary
Salome, of the Angel Gabriel, and of all the saints who formed the
family tree of the Lord
Jesus.\textsuperscript{\protect\hypertarget{13_Chapter_Six__THE_DEPICTION_OF_TH.xhtmlux5cux23id_1249}{\protect\hyperlink{23_NOTES.xhtmlux5cux23id_1250}{11}}}
Furthermore, there are too many monastic orders, says d'Ailly, and this
leads to differences of custom, to divisiveness and to arrogance, to the
prideful elevation of one spiritual order above another. First of all he
wants to restrict the mendicant orders. Their existence is detrimental
to the homes for lepers and to hospitals and to the miserable poor and
truly needy who have the right and are entitled to
beg.\textsuperscript{\protect\hypertarget{13_Chapter_Six__THE_DEPICTION_OF_TH.xhtmlux5cux23id_1247}{\protect\hyperlink{23_NOTES.xhtmlux5cux23id_1248}{12}}}
He wants to ban the indulgence preachers from the church who soil it
with their lies and expose it to
ridicule.\textsuperscript{\protect\hypertarget{13_Chapter_Six__THE_DEPICTION_OF_TH.xhtmlux5cux23id_1245}{\protect\hyperlink{23_NOTES.xhtmlux5cux23id_1246}{13}}}
Where will the continual founding of new convents without sufficient
funds for their maintenance lead?

It is obvious that Pierre d'Ailly campaigned more against quantitative
than against any qualitative evil. In his sermons he does not
specifically question the piety and sanctity of all these things,
excepting his criticism of the indulgence sellers, but is more worried
\protect\hypertarget{13_Chapter_Six__THE_DEPICTION_OF_TH.xhtmlux5cux23page_176}{}{}about
their unrestrained growth as such. He sees the church suffocating under
the burden of trivial details. When Adamus de Ruper propagated his new
Rosicrucian Brotherhood, it, too, met resistance more because of its
novelty than its content. His opponents warned that the people, trusting
in the efficacy of such a grand society given to prayer, would neglect
their prescribed penances and the clergy their breviaries. The parish
churches would become empty if the brotherhood met only in the churches
of the Franciscans and the Dominicans, and these meetings would lead to
partisanship and conspiracies. Finally, the accusation was made that
what the brotherhood offered as grand and miraculous revelations were
mere phantasmagoria, a conglomeration of imagination and old wives'
tales.\textsuperscript{\protect\hypertarget{13_Chapter_Six__THE_DEPICTION_OF_TH.xhtmlux5cux23id_1243}{\protect\hyperlink{23_NOTES.xhtmlux5cux23id_1244}{14}}}

A characteristic example of the mechanical way in which sacred
observances tended to multiply in the absence of intervention by strict
authority was the weeklong veneration of the Innocent Children. During
the remembrances of the slaying of the children in Bethlehem on December
28, various semi-pagan solstice practices merged with mushy
sentimentality. The day was thought to be unlucky. There were many who
regarded the day of the week on which the last Day of Innocents fell to
be inauspicious throughout the year. No work should begin on that day,
nor a journey started. That day of the week was simply called Innocents'
Day like the festival itself. Louis XI observed this custom
conscientiously. The coronation of Edward IV was repeated because it was
found to have taken place the first time on that inauspicious day and
René of Lorraine had to forgo a battle because his mercenaries refused
to fight on the week-anniversary of the day of the
children.\textsuperscript{\protect\hypertarget{13_Chapter_Six__THE_DEPICTION_OF_TH.xhtmlux5cux23id_1241}{\protect\hyperlink{23_NOTES.xhtmlux5cux23id_1242}{15}}}

Jean de Gerson was prompted by the practice to author a treatise against
superstition in general and this one in
particular.\textsuperscript{\protect\hypertarget{13_Chapter_Six__THE_DEPICTION_OF_TH.xhtmlux5cux23id_1239}{\protect\hyperlink{23_NOTES.xhtmlux5cux23id_1240}{16}}}
He was one of those who clearly saw the danger to the church posed by
this wild growth of religious ideas. With his keen and somewhat sober
mind he realizes also something of the psychological ground from which
all these things arose. They arise ``ex sola hominum phantasiatione et
melancholica imaginatione'';
\protect\hypertarget{13_Chapter_Six__THE_DEPICTION_OF_TH.xhtmlux5cux23id_2989}{\protect\hyperlink{23_NOTES.xhtmlux5cux23id_2990}{*\textsuperscript{1}}}
their corrupt imagination results from damage to the brain that can be
traced to the devil's deceit. Thus the devil comes in for his share of
the blame.

\protect\hypertarget{13_Chapter_Six__THE_DEPICTION_OF_TH.xhtmlux5cux23page_177}{}{}The
process is one of ongoing reduction of the infinite to the finite; the
miracle is reduced to atoms. To every holy mystery, there attaches
itself like a barnacle to a ship, a growth of external elements of faith
that desecrate it. The miracle of the Eucharist is permeated with the
most sober and material superstitions: that one cannot go blind or
suffer a stroke on the day one hears a mass or that one does not age
during the time one spends at the
service.\textsuperscript{\protect\hypertarget{13_Chapter_Six__THE_DEPICTION_OF_TH.xhtmlux5cux23id_1237}{\protect\hyperlink{23_NOTES.xhtmlux5cux23id_1238}{17}}}
The church has to be constantly on guard so that God is not brought all
too close to earth. It declares heretical the claim that Peter, John,
and James had seen the Heavenly Being during the transfiguration just as
clearly as they do now in
heaven.\textsuperscript{\protect\hypertarget{13_Chapter_Six__THE_DEPICTION_OF_TH.xhtmlux5cux23id_1235}{\protect\hyperlink{23_NOTES.xhtmlux5cux23id_1236}{18}}}
It is blasphemy for one of the successors of Jeanne d'Arc to claim to
have seen God dressed in a long robe and a red
overcoat.\textsuperscript{\protect\hypertarget{13_Chapter_Six__THE_DEPICTION_OF_TH.xhtmlux5cux23id_1233}{\protect\hyperlink{23_NOTES.xhtmlux5cux23id_1234}{19}}}
But can the people be blamed for not being able to make the fine
distinctions of theology when the church offers so many colorful images?

Gerson himself did not stay completely away from the evil he fought. He
raised his voice against conceited curiosity; that is, that spirit of
inquiry that wants to penetrate nature to its last mystery, but he
himself dug with immodest zeal into the smallest details of sacred
matters. His particular veneration of St. Joseph, for whose festival he
assiduously labored, made him eager to know everything about the saint.
He dwelled on every detail of the marriage to Mary, their life together,
Joseph's abstention, how he came to know about her pregnancy, how old he
was. Gerson wanted no part of the caricature that art had made of
Joseph: of the old overworked figure depicted by Deschamps and painted
by Broederlam
(\protect\hyperlink{20_ILLUSTRATIONS_FOLLOW_PAGE.xhtmlux5cux23id_5}{plate
5}). He said that Joseph was not yet fifty years
old.\textsuperscript{\protect\hypertarget{13_Chapter_Six__THE_DEPICTION_OF_TH.xhtmlux5cux23id_1231}{\protect\hyperlink{23_NOTES.xhtmlux5cux23id_1232}{20}}}
Elsewhere he permitted himself an observation about the physical
constitution of John the Baptist: ``semen igiture materiale ex quo
corpus compaginandum erat nec durum nimis nec rursus fluidum abundantius
fuit.''\textsuperscript{\protect\hypertarget{13_Chapter_Six__THE_DEPICTION_OF_TH.xhtmlux5cux23id_1229}{\protect\hyperlink{23_NOTES.xhtmlux5cux23id_1230}{21}}}\protect\hypertarget{13_Chapter_Six__THE_DEPICTION_OF_TH.xhtmlux5cux23id_2991}{\protect\hyperlink{23_NOTES.xhtmlux5cux23id_2992}{*\textsuperscript{2}}}
The famous popular preacher Olivier Maillard used to present to his
audience, after his initial remarks, as ``une belle question
théologale''
\protect\hypertarget{13_Chapter_Six__THE_DEPICTION_OF_TH.xhtmlux5cux23id_2993}{\protect\hyperlink{23_NOTES.xhtmlux5cux23id_2994}{†\textsuperscript{3}}}
inquiries such as whether or not the Virgin must have actively
participated in the conception of Christ in order to be called truly the
Mother of God or whether Christ's body would have turned to ashes had
the Resurrection not
\protect\hypertarget{13_Chapter_Six__THE_DEPICTION_OF_TH.xhtmlux5cux23page_178}{}{}interfered.\textsuperscript{\protect\hypertarget{13_Chapter_Six__THE_DEPICTION_OF_TH.xhtmlux5cux23id_1227}{\protect\hyperlink{23_NOTES.xhtmlux5cux23id_1228}{22}}}
The controversy about Mary's Immaculate Conception was met by the
Dominicans, contrary to the growing popular view that felt the need of
absolving the Virgin from the beginning from original sin, by a mixture
of biological and embryonic speculations that, today, seem little
edifying. Yet, the most zealous theologians were so stubbornly convinced
of the importance of their arguments that they stooped so low as to take
the controversy before the larger public in their
sermons.\textsuperscript{\protect\hypertarget{13_Chapter_Six__THE_DEPICTION_OF_TH.xhtmlux5cux23id_1225}{\protect\hyperlink{23_NOTES.xhtmlux5cux23id_1226}{23}}}
If this was the direction of the highest churchmen, how could it be
other than that everything holy would dissolve into the mundane and the
detailed from which one could only occasionally rise to a consciousness
of the miraculous?

This fatuous familiarity with God in daily life has to be seen in two
ways. On the one hand, it testifies to the absolute stability and
immediacy of faith, but where this familiarity becomes habitual it
increases the danger that the godless (who are always with us), but also
the pious, in moments of insufficient religious tension, will
continuously profane faith more or less consciously and intentionally.
In particular, the most tender of all mysteries, the Eucharist, is
threatened in this way. There is certainly no stronger and more fervent
emotion of the Catholic faith than the belief in the direct and
essential presence of God in the consecrated host. It is an essential
element of the religion both in medieval times and now, but in medieval
times, given the naive unself-consciousness of unrestrained speech, it
brought about a use of language that, on occasion, seems profane. A
traveler dismounts for a moment and enters a church ``pour veoir Dieu en
passant.''
\protect\hypertarget{13_Chapter_Six__THE_DEPICTION_OF_TH.xhtmlux5cux23id_2995}{\protect\hyperlink{23_NOTES.xhtmlux5cux23id_2996}{*\textsuperscript{4}}}
Of a priest on a donkey proceeding on his way with a host it is said,
``un Dieu sur un
asne.''\textsuperscript{\protect\hypertarget{13_Chapter_Six__THE_DEPICTION_OF_TH.xhtmlux5cux23id_1223}{\protect\hyperlink{23_NOTES.xhtmlux5cux23id_1224}{24}}}\protect\hypertarget{13_Chapter_Six__THE_DEPICTION_OF_TH.xhtmlux5cux23id_2997}{\protect\hyperlink{23_NOTES.xhtmlux5cux23id_2998}{†\textsuperscript{5}}}
Of a woman on her sickbed it is said, ``Sy cuidoit transir de la mort et
se fist apporter beau sire
Dieux.''\textsuperscript{\protect\hypertarget{13_Chapter_Six__THE_DEPICTION_OF_TH.xhtmlux5cux23id_1221}{\protect\hyperlink{23_NOTES.xhtmlux5cux23id_1222}{25}}}\protect\hypertarget{13_Chapter_Six__THE_DEPICTION_OF_TH.xhtmlux5cux23id_2999}{\protect\hyperlink{23_NOTES.xhtmlux5cux23id_3000}{‡\textsuperscript{6}}}
\emph{Veoir Dieu} was the common expression if one saw the Host
elevated.\textsuperscript{\protect\hypertarget{13_Chapter_Six__THE_DEPICTION_OF_TH.xhtmlux5cux23id_1219}{\protect\hyperlink{23_NOTES.xhtmlux5cux23id_1220}{26}}}
In all these cases it is not the use of language itself that is profane,
but it becomes profane if the mind is impious or if words are uttered
thoughtlessly. In such cases what desecration such customary language
brings in its wake! From common usage it is only a small step to
mindless familiarities
\protect\hypertarget{13_Chapter_Six__THE_DEPICTION_OF_TH.xhtmlux5cux23page_179}{}{}such
as the saying ``Laissez faire à Dieu, qui est homme d'aage.
\protect\hypertarget{13_Chapter_Six__THE_DEPICTION_OF_TH.xhtmlux5cux23id_3001}{\protect\hyperlink{23_NOTES.xhtmlux5cux23id_3002}{*\textsuperscript{7}}}
``\textsuperscript{\protect\hypertarget{13_Chapter_Six__THE_DEPICTION_OF_TH.xhtmlux5cux23id_1217}{\protect\hyperlink{23_NOTES.xhtmlux5cux23id_1218}{27}}}
Or Froissart's ``et li prie à mains jointes, pour si hault homme que
Dieu
est.''\textsuperscript{\protect\hypertarget{13_Chapter_Six__THE_DEPICTION_OF_TH.xhtmlux5cux23id_1215}{\protect\hyperlink{23_NOTES.xhtmlux5cux23id_1216}{28}}}\protect\hypertarget{13_Chapter_Six__THE_DEPICTION_OF_TH.xhtmlux5cux23id_3003}{\protect\hyperlink{23_NOTES.xhtmlux5cux23id_3004}{†\textsuperscript{8}}}
A case that clearly shows how the word ``Dieu'' used for the host could
contaminate the belief in God itself is the following. The bishop of
Coutances celebrates a mass in the church of St. Denis. When he elevates
the body of the Lord, Hugues Aubriot, the provost of Paris who is
walking around the chapel where the mass is being held, is admonished to
pray. But Hugues, known as an \emph{esprit
fort}\protect\hypertarget{13_Chapter_Six__THE_DEPICTION_OF_TH.xhtmlux5cux23id_3005}{\protect\hyperlink{23_NOTES.xhtmlux5cux23id_3006}{‡\textsuperscript{9}}}
answers with a curse that he does not believe in the God of a bishop who
lives at the
court.\textsuperscript{\protect\hypertarget{13_Chapter_Six__THE_DEPICTION_OF_TH.xhtmlux5cux23id_1213}{\protect\hyperlink{23_NOTES.xhtmlux5cux23id_1214}{29}}}

There was not the least intention to mock the sacred in this
familiarity, yet the addiction to turning everything holy into pictorial
images seems shameless to us. People owned miniatures of Mary similar to
the sets of cups called ``Hansje in den
Kelder.''\textsuperscript{\protect\hypertarget{13_Chapter_Six__THE_DEPICTION_OF_TH.xhtmlux5cux23id_1211}{\protect\hyperlink{23_NOTES.xhtmlux5cux23id_1212}{30}}}
They were small golden figures, highly decorated with precious stones,
whose belly could be opened to reveal the Trinity. Such miniatures could
be found in the treasury of the dukes of
Burgundy.\textsuperscript{\protect\hypertarget{13_Chapter_Six__THE_DEPICTION_OF_TH.xhtmlux5cux23id_1209}{\protect\hyperlink{23_NOTES.xhtmlux5cux23id_1210}{31}}}
Gerson saw one in the monastery of the Carmelites in Paris. He
disapproved, not because of the lack of piety shown by such a crude
depiction of the miracle, but because of the heresy of depicting the
entire Trinity as the fruit of Mary's
womb.\textsuperscript{\protect\hypertarget{13_Chapter_Six__THE_DEPICTION_OF_TH.xhtmlux5cux23id_1207}{\protect\hyperlink{23_NOTES.xhtmlux5cux23id_1208}{32}}}

Life was permeated by religion to the degree that the distance between
the earthly and the spiritual was in danger of being obliterated at any
moment. While on the one hand all of ordinary life was raised to the
sphere of the divine, on the other the divine was bound to the mundane
in an indissoluble mixture with daily life. Earlier we spoke of the
Cemetery of the Innocents in Paris where bones were piled up and
exhibited all around the yard. Can one imagine anything more terrible
than the life of the nuns walled in the back of the churchyard in this
place of horrors? But let us read what contemporaries said about it:
``The hermits lived there in a cute little house, walled in to the
accompanyment of a beautiful sermon. They received from the king an
annual salary of eight pounds in eight
installments.''\textsuperscript{\protect\hypertarget{13_Chapter_Six__THE_DEPICTION_OF_TH.xhtmlux5cux23id_1205}{\protect\hyperlink{23_NOTES.xhtmlux5cux23id_1206}{33}}}
This as if we were dealing with ordinary nuns! Where is the religious
pathos? Where is it when an
\protect\hypertarget{13_Chapter_Six__THE_DEPICTION_OF_TH.xhtmlux5cux23page_180}{}{}indulgence
is granted for the most ordinary domestic chores such as firing the
oven, milking a cow, or scrubbing a
pot?\textsuperscript{\protect\hypertarget{13_Chapter_Six__THE_DEPICTION_OF_TH.xhtmlux5cux23id_1203}{\protect\hyperlink{23_NOTES.xhtmlux5cux23id_1204}{34}}}
At a raffle in Bergen op Zoom in 1518, either ``precious prizes'' or
indulgences could be
won.\textsuperscript{\protect\hypertarget{13_Chapter_Six__THE_DEPICTION_OF_TH.xhtmlux5cux23id_1201}{\protect\hyperlink{23_NOTES.xhtmlux5cux23id_1202}{35}}}
During princely processions into the cities, the precious reliquary
shrines, placed on altars, served by prelates and offered to the princes
to be kissed in veneration, competed at street corners with sensuous
performances, frequently performed in pagan
nudity.\textsuperscript{\protect\hypertarget{13_Chapter_Six__THE_DEPICTION_OF_TH.xhtmlux5cux23id_1199}{\protect\hyperlink{23_NOTES.xhtmlux5cux23id_1200}{36}}}

The apparent lack of distinction between the religious and worldly
spheres is most vividly expressed in the well-known fact that secular
melodies may be used---always unchanged---for sacred songs and vice
versa. Guillaume Dufay composed his masses to the themes of popular
songs such as ``Tant je me déduis,'' ``Se la face ay pale,'' ``L'omme
armé.''
\protect\hypertarget{13_Chapter_Six__THE_DEPICTION_OF_TH.xhtmlux5cux23id_3007}{\protect\hyperlink{23_NOTES.xhtmlux5cux23id_3008}{*\textsuperscript{10}}}
There is a constant interchange between religious and secular
terminology. Without objection, expressions for earthly things are
borrowed from liturgy and the other way around, too. Above the door to
the auditor office in Lille was a verse that reminded everyone that he
would eventually have to give account of his gifts to God:

\emph{Lors ouvrira, au son de buysine}

\emph{Sa générale et grant chambre des
comptes}.\textsuperscript{\protect\hypertarget{13_Chapter_Six__THE_DEPICTION_OF_TH.xhtmlux5cux23id_1197}{\protect\hyperlink{23_NOTES.xhtmlux5cux23id_1198}{37}}}\protect\hypertarget{13_Chapter_Six__THE_DEPICTION_OF_TH.xhtmlux5cux23id_3009}{\protect\hyperlink{23_NOTES.xhtmlux5cux23id_3010}{†\textsuperscript{11}}}

On the other hand, the solemn announcement of a tournament sounds as if
it were a festival where indulgences were sold:

\emph{Oez, oez, l'oneur et la louenge}

\emph{Et des armes grantdisime
pardon}.\textsuperscript{\protect\hypertarget{13_Chapter_Six__THE_DEPICTION_OF_TH.xhtmlux5cux23id_1195}{\protect\hyperlink{23_NOTES.xhtmlux5cux23id_1196}{38}}}\protect\hypertarget{13_Chapter_Six__THE_DEPICTION_OF_TH.xhtmlux5cux23id_3011}{\protect\hyperlink{23_NOTES.xhtmlux5cux23id_3012}{‡\textsuperscript{12}}}

By coincidence in the word ``mistère'' the concepts contained in both
``mysterium'' and ``ministerium'' were blended. This weakened the idea
of mystery in everyday language in which everything
\protect\hypertarget{13_Chapter_Six__THE_DEPICTION_OF_TH.xhtmlux5cux23page_181}{}{}was
called ``mistère'': the unicorn, the shields, and the doll used in the
``Pas d'ames de la Fontaine des
pleurs.''\textsuperscript{\protect\hypertarget{13_Chapter_Six__THE_DEPICTION_OF_TH.xhtmlux5cux23id_1193}{\protect\hyperlink{23_NOTES.xhtmlux5cux23id_1194}{39}}}

As a counterpart to religious symbolism, that is, the interpretation of
all earthly things and earthly events as symbols and prefigurations of
the divine, the praise of princes is transposed into liturgical
metaphor. Whenever the awe of worldly authority seizes medieval man, the
language of piety serves as the means of expressing his feelings. The
liegemen of the princes of the fifteenth century did not stop short of
any profanation. In the court case about the murder of Louis of Orléans
the advocate has the ghost of the murdered prince speak to his son, Look
at my wounds of which five are particularly cruel and
mortal.\textsuperscript{\protect\hypertarget{13_Chapter_Six__THE_DEPICTION_OF_TH.xhtmlux5cux23id_1191}{\protect\hyperlink{23_NOTES.xhtmlux5cux23id_1192}{40}}}
That is, he makes Christ the image of the murder victim. The bishop of
Chalons, in turn, does not shy away from comparing John the Fearless,
who was the victim of the avenger of the prince of Orléans, to the lamb
of
God.\textsuperscript{\protect\hypertarget{13_Chapter_Six__THE_DEPICTION_OF_TH.xhtmlux5cux23id_1189}{\protect\hyperlink{23_NOTES.xhtmlux5cux23id_1190}{41}}}
Molinet compares Emperor Frederick III, who sent his son Maximillian to
marry Mary of Burgundy, to God the Father who had sent his Son to earth,
and he spares no pious words to embellish the event. Later when
Frederick and Maximillian enter the city of Brussels with the young
Philip le Beau, Molinet has the burghers cry with tears in their eyes,
``Veez-ci figure de la Trinité, le Père, le Fils et Sainct Esprit.''
\protect\hypertarget{13_Chapter_Six__THE_DEPICTION_OF_TH.xhtmlux5cux23id_3013}{\protect\hyperlink{23_NOTES.xhtmlux5cux23id_3014}{*\textsuperscript{13}}}
He offers a wreath of flowers to Mary of Burgundy, a worthy image of our
beloved Lady, ``except for her
virginity.''\textsuperscript{\protect\hypertarget{13_Chapter_Six__THE_DEPICTION_OF_TH.xhtmlux5cux23id_1187}{\protect\hyperlink{23_NOTES.xhtmlux5cux23id_1188}{42}}}

``Not that I want to deify
princes,''\textsuperscript{\protect\hypertarget{13_Chapter_Six__THE_DEPICTION_OF_TH.xhtmlux5cux23id_1185}{\protect\hyperlink{23_NOTES.xhtmlux5cux23id_1186}{43}}}
says this creature of the courts. Perhaps these are merely hollow
phrases rather than deeply felt devotion, but they prove nevertheless
the devaluation of holy things by everyday use. How can we blame a poet
hired by the court when Gerson himself grants to the princely auditors
of his sermons special guardian angels of higher hierarchy and office
than those of other
men?\textsuperscript{\protect\hypertarget{13_Chapter_Six__THE_DEPICTION_OF_TH.xhtmlux5cux23id_1183}{\protect\hyperlink{23_NOTES.xhtmlux5cux23id_1184}{44}}}

In the transfer of religious expressions to the erotic, which we have
already mentioned, we are dealing, of course, with something entirely
different. In these cases there is an element of deliberate impiety and
genuine mockery that is absent in the examples just described. They are
related only in that they both arise from fatuous familiarity with the
sacred. The authors of the \emph{Cent nouvelles nou
\protect\hypertarget{13_Chapter_Six__THE_DEPICTION_OF_TH.xhtmlux5cux23page_182}{}{}velles}
engage in endless plays on words such as ``saint'' and ``seins,'' and
use ``dévotion, confesser, bénir''
\protect\hypertarget{13_Chapter_Six__THE_DEPICTION_OF_TH.xhtmlux5cux23id_3015}{\protect\hyperlink{23_NOTES.xhtmlux5cux23id_3016}{*\textsuperscript{14}}}
with obscene meanings. The author of the \emph{Quinze joyes de mariage}
chose his title in reference to the joys of
Mary.\textsuperscript{\protect\hypertarget{13_Chapter_Six__THE_DEPICTION_OF_TH.xhtmlux5cux23id_1181}{\protect\hyperlink{23_NOTES.xhtmlux5cux23id_1182}{45}}}
There has already been mention of the idea of love as a pious
observance. It is more serious when the defender of the \emph{Roman de
la rose} uses sacred terms to refer to ``partes corporis inhonesta et
pecatta immunda atque
turpia.''\textsuperscript{\protect\hypertarget{13_Chapter_Six__THE_DEPICTION_OF_TH.xhtmlux5cux23id_1179}{\protect\hyperlink{23_NOTES.xhtmlux5cux23id_1180}{46}}}\protect\hypertarget{13_Chapter_Six__THE_DEPICTION_OF_TH.xhtmlux5cux23id_3017}{\protect\hyperlink{23_NOTES.xhtmlux5cux23id_3018}{†\textsuperscript{15}}}
Here is well demonstrated something of the dangerous contact between the
religious and the erotic that the church, with good reason, so feared.
There is no more striking example of that contact than the Melun Madonna
(\protect\hyperlink{20_ILLUSTRATIONS_FOLLOW_PAGE.xhtmlux5cux23id_6}{plate
6}), ascribed to Foucquet, which used to be a diptych and was united
with the panel, now in Berlin, which shows the donor, Etienne Chevalier,
with St. Stephen
(\protect\hyperlink{20_ILLUSTRATIONS_FOLLOW_PAGE.xhtmlux5cux23id_7}{plate
7}). Earlier the united work hung in the choir of the Church of Our Lady
in Melun. An old tradition, noted in the seventeenth century by Denis
Godefroy, a man knowledgeable about medieval times, has it that the
features of the Madonna are those of Agnes
Sorel,\textsuperscript{\protect\hypertarget{13_Chapter_Six__THE_DEPICTION_OF_TH.xhtmlux5cux23id_1177}{\protect\hyperlink{23_NOTES.xhtmlux5cux23id_1178}{47}}}
the King's mistress. Chevalier did not hide his passion for her. Even
considering all the great qualities of the painting, it is a fashionable
doll that we encounter here with a rounded, clean-shaven forehead,
widely separated spherical breasts, a high narrow waist, a bizarre and
inscrutable facial expression, and surrounded by stiff red and blue
angels. All this bestows on the panel a touch of decadent godlessness
that is in marked contrast to the vigorous and simple depiction of the
donor and his saint on the other side panel. Godefroy saw, on the large
blue velvet frame, a series of E's in pearls joined by love knots of
gold and silver
threads.\textsuperscript{\protect\hypertarget{13_Chapter_Six__THE_DEPICTION_OF_TH.xhtmlux5cux23id_1175}{\protect\hyperlink{23_NOTES.xhtmlux5cux23id_1176}{48}}}
Does this not reveal a blasphemous nonchalance towards the sacred that
could not be outdone by any Renaissance spirit?

The profanation of daily religious practice was almost without bounds.
It is said that choristers sang the profane words of the song to whose
melodies the service had been set: such songs as ``Baisez moi''
\protect\hypertarget{13_Chapter_Six__THE_DEPICTION_OF_TH.xhtmlux5cux23id_3019}{\protect\hyperlink{23_NOTES.xhtmlux5cux23id_3020}{‡\textsuperscript{16}}}
and ``Rouge
nez.''\textsuperscript{\protect\hypertarget{13_Chapter_Six__THE_DEPICTION_OF_TH.xhtmlux5cux23id_1173}{\protect\hyperlink{23_NOTES.xhtmlux5cux23id_1174}{49}}}\protect\hypertarget{13_Chapter_Six__THE_DEPICTION_OF_TH.xhtmlux5cux23id_3021}{\protect\hyperlink{23_NOTES.xhtmlux5cux23id_3022}{§\textsuperscript{17}}}
David of Burgundy, the bastard of Philip the Good, made his entry as
bishop of Utrecht in the
com\protect\hypertarget{13_Chapter_Six__THE_DEPICTION_OF_TH.xhtmlux5cux23page_183}{}{}pany
of a war party consisting only of noblemen along with his brother, the
Bastard of Burgundy, who had accompanied him from Amersfoort. The new
bishop was clad in armor ``comme seroit un conquéeur de païs, prince
séculier,''
\protect\hypertarget{13_Chapter_Six__THE_DEPICTION_OF_TH.xhtmlux5cux23id_3023}{\protect\hyperlink{23_NOTES.xhtmlux5cux23id_3024}{*\textsuperscript{18}}}
says Chastellain with obvious disapproval. In this way he rode to the
cathedral and entered it with a procession complete with flags and
crosses to pray before the high
altar.\textsuperscript{\protect\hypertarget{13_Chapter_Six__THE_DEPICTION_OF_TH.xhtmlux5cux23id_1171}{\protect\hyperlink{23_NOTES.xhtmlux5cux23id_1172}{50}}}
Let us put this Burgundian arrogance beside the gentle frivolity of
Rudolf Agricola's father, the pastor of Baflo, who, on the day he was
selected Abbot of Selwert, received the news that his concubine had born
him a son, and said, ``Today I have twice become a father. May God's
blessing be upon
it.''\textsuperscript{\protect\hypertarget{13_Chapter_Six__THE_DEPICTION_OF_TH.xhtmlux5cux23id_1169}{\protect\hyperlink{23_NOTES.xhtmlux5cux23id_1170}{51}}}

Contemporary people regarded the growing disrespect for the church as a
recent evil:

\emph{On souloit estre ou temps passé}

\emph{En L'église béneignement}

\emph{A genoux en humilité}

\emph{Delez l'autel moult closement},

\emph{Tout nu le chief piteusement},

\emph{Maiz au jour d'uy, si comme beste},

\emph{On vient à l'autel bien souvent}

\emph{Chaperon et chapel en
teste}.\textsuperscript{\protect\hypertarget{13_Chapter_Six__THE_DEPICTION_OF_TH.xhtmlux5cux23id_1167}{\protect\hyperlink{23_NOTES.xhtmlux5cux23id_1168}{52}}}\protect\hypertarget{13_Chapter_Six__THE_DEPICTION_OF_TH.xhtmlux5cux23id_3025}{\protect\hyperlink{23_NOTES.xhtmlux5cux23id_3026}{†\textsuperscript{19}}}

On festive days, laments Nicolas of Clémanges, only a few attend mass.
They don't stay until it is over and are satisfied with touching the
holy water, saluting our Lady by bending their knees once, or kissing
the image of a saint. If they happen to see the host elevated, they take
pride in this act as if they had done a great deed for Christ. Matins
and vespers are read by the priest and his assistants
alone.\textsuperscript{\protect\hypertarget{13_Chapter_Six__THE_DEPICTION_OF_TH.xhtmlux5cux23id_1165}{\protect\hyperlink{23_NOTES.xhtmlux5cux23id_1166}{53}}}
The squire of the village makes the priest wait with his mass until he
and his wife have gotten up and
dressed.\textsuperscript{\protect\hypertarget{13_Chapter_Six__THE_DEPICTION_OF_TH.xhtmlux5cux23id_1163}{\protect\hyperlink{23_NOTES.xhtmlux5cux23id_1164}{54}}}

The most sacred festivals, even Christmas Eve itself, are spent in
debauchery with card games, cursing, and blasphemy; if the people are
admonished, they point to the example of the nobility and the higher and
lower priesthood who behave with
impunity.\textsuperscript{\protect\hypertarget{13_Chapter_Six__THE_DEPICTION_OF_TH.xhtmlux5cux23id_1161}{\protect\hyperlink{23_NOTES.xhtmlux5cux23id_1162}{55}}}
\protect\hypertarget{13_Chapter_Six__THE_DEPICTION_OF_TH.xhtmlux5cux23page_184}{}{}During
vigils there is dancing in the churches themselves to the accompaniment
of lascivious songs. Priests set the example by dicing and cursing
during their nightly
wakes.\textsuperscript{\protect\hypertarget{13_Chapter_Six__THE_DEPICTION_OF_TH.xhtmlux5cux23id_1159}{\protect\hyperlink{23_NOTES.xhtmlux5cux23id_1160}{56}}}
These are the practices documented by the moralists, who are perhaps
always given to taking the darkest view, but the sources more than once
confirm this dark image. The city council of Strasbourg every year
dispensed eleven hundred liters of wine to those who spent the night of
St. Adolf in the cathedral ``holding a wake and in
prayer.''\textsuperscript{\protect\hypertarget{13_Chapter_Six__THE_DEPICTION_OF_TH.xhtmlux5cux23id_1157}{\protect\hyperlink{23_NOTES.xhtmlux5cux23id_1158}{57}}}
A city councillor complained to Denis the Carthusian that the annual
procession of the holy relics provided the occasion for drinking and
numerous improprieties. How could this be stopped? The magistrate
himself would not be easily persuaded because the procession made money
for the city; it attracted people who needed lodging, food, and drink.
Above all, it was customary. Denis knew the problem. He knew how
shamelessly people acted during processions, by gossiping, laughing,
flirting, drinking, and indulging in other uncouth
pleasures.\textsuperscript{\protect\hypertarget{13_Chapter_Six__THE_DEPICTION_OF_TH.xhtmlux5cux23id_1155}{\protect\hyperlink{23_NOTES.xhtmlux5cux23id_1156}{58}}}
His melancholy sigh perfectly fits the procession of the delegation from
Ghent carrying the shrine of St. Liéin to the fair at Houthen. In the
old days, says Chastellain, the notables were in the habit of carrying
the holy body ``en grande et haute solempnité et
révérence,''\protect\hypertarget{13_Chapter_Six__THE_DEPICTION_OF_TH.xhtmlux5cux23id_3027}{\protect\hyperlink{23_NOTES.xhtmlux5cux23id_3028}{*\textsuperscript{20}}}
but now it is ``une multitude de respaille et de garconnaille
mauvaise,''\protect\hypertarget{13_Chapter_Six__THE_DEPICTION_OF_TH.xhtmlux5cux23id_3029}{\protect\hyperlink{23_NOTES.xhtmlux5cux23id_3030}{†\textsuperscript{21}}}
They carry it screaming and howling, singing and dancing, mocking
everything in sight, and they are all drunk. Moreover, they are armed
and indulge themselves in whatever they wish. Everything is at their
mercy given the excuse of their holy
burden.\textsuperscript{\protect\hypertarget{13_Chapter_Six__THE_DEPICTION_OF_TH.xhtmlux5cux23id_1153}{\protect\hyperlink{23_NOTES.xhtmlux5cux23id_1154}{59}}}

Going to church was an important element of social life. People went
there to enjoy dressing up, to show off their rank and prominence and to
compete in courtly manners and deportment. As already
mentioned,\textsuperscript{\protect\hypertarget{13_Chapter_Six__THE_DEPICTION_OF_TH.xhtmlux5cux23id_1151}{\protect\hyperlink{23_NOTES.xhtmlux5cux23id_1152}{60}}}
the paten, the ``paix,'' was a constant source of the most irritating
competitive courtesy. If a young nobleman enters, the gracious lady
stands up and kisses him on the mouth even while the priest elevates the
host and the people are on their knees
praying.\textsuperscript{\protect\hypertarget{13_Chapter_Six__THE_DEPICTION_OF_TH.xhtmlux5cux23id_1149}{\protect\hyperlink{23_NOTES.xhtmlux5cux23id_1150}{61}}}
Walking about and talking during mass must have been quite
customary.\textsuperscript{\protect\hypertarget{13_Chapter_Six__THE_DEPICTION_OF_TH.xhtmlux5cux23id_1147}{\protect\hyperlink{23_NOTES.xhtmlux5cux23id_1148}{62}}}
The use of the church as trysting place for young lads and girls was so
common that only the moralists were still upset about it. Young men
rarely come to church, complains
Nico\protect\hypertarget{13_Chapter_Six__THE_DEPICTION_OF_TH.xhtmlux5cux23page_185}{}{}las
of Clémanges, other than to watch the women who put their elaborate
hairstyles and generous décolletés on display there. The virtuous
Christine de Pisan rhymes in all innocence:

\emph{Se souvent vais ou moustier},

\emph{C'est tout pour veoir la belle}

\emph{Fresche comme rose
nouvelle}.\textsuperscript{\protect\hypertarget{13_Chapter_Six__THE_DEPICTION_OF_TH.xhtmlux5cux23id_1145}{\protect\hyperlink{23_NOTES.xhtmlux5cux23id_1146}{63}}}\protect\hypertarget{13_Chapter_Six__THE_DEPICTION_OF_TH.xhtmlux5cux23id_3031}{\protect\hyperlink{23_NOTES.xhtmlux5cux23id_3032}{``*\textsuperscript{22}}}

It was not only small trysts for which church afforded the occasion, not
merely for handing the beloved the consecrated water, giving her the
``paix,'' lighting a candle for her and kneeling beside her; it did not
stop at a few signs and furtive
glances.\textsuperscript{\protect\hypertarget{13_Chapter_Six__THE_DEPICTION_OF_TH.xhtmlux5cux23id_1143}{\protect\hyperlink{23_NOTES.xhtmlux5cux23id_1144}{64}}}
In the churches themselves, even on holy days, prostitutes looked for
customers\textsuperscript{\protect\hypertarget{13_Chapter_Six__THE_DEPICTION_OF_TH.xhtmlux5cux23id_1141}{\protect\hyperlink{23_NOTES.xhtmlux5cux23id_1142}{65}}}
and immoral pictures that corrupted the youth could be bought. No
preaching was effective against such
evils.\textsuperscript{\protect\hypertarget{13_Chapter_Six__THE_DEPICTION_OF_TH.xhtmlux5cux23id_1139}{\protect\hyperlink{23_NOTES.xhtmlux5cux23id_1140}{66}}}
Time and again the church and the altar were desecrated by immoral
acts.\textsuperscript{\protect\hypertarget{13_Chapter_Six__THE_DEPICTION_OF_TH.xhtmlux5cux23id_1137}{\protect\hyperlink{23_NOTES.xhtmlux5cux23id_1138}{67}}}

Pilgrimages, like church services, provided the occasion for pleasure,
especially of an amorous nature. Pilgrimages are frequently spoken of as
pleasure trips. The Knight de la Tour Landry, who took seriously the
instruction of his daughters in good manners, speaks of ladies of
leisure who liked to go to tournaments and on pilgrimages. As a warning
he cites the example of a woman who went on pilgrimage as an excuse to
meet her paramour. ``Et pour ce a cy bon exemple comment l'on ne doit
pas aler aux sains voiaiges pour nulle folle
plaisance.''\textsuperscript{\protect\hypertarget{13_Chapter_Six__THE_DEPICTION_OF_TH.xhtmlux5cux23id_1135}{\protect\hyperlink{23_NOTES.xhtmlux5cux23id_1136}{68}}}\protect\hypertarget{13_Chapter_Six__THE_DEPICTION_OF_TH.xhtmlux5cux23id_3033}{\protect\hyperlink{23_NOTES.xhtmlux5cux23id_3034}{†\textsuperscript{23}}}
Nicolas de Clémanges agrees. People go on pilgrimages to distant shrines
not so much to fulfill a vow, but to find freedom for straying from the
straight and narrow. Pilgrimages are the occasion for all sorts of
transgressions. Procuresses are always there to seduce young
girls.\textsuperscript{\protect\hypertarget{13_Chapter_Six__THE_DEPICTION_OF_TH.xhtmlux5cux23id_1133}{\protect\hyperlink{23_NOTES.xhtmlux5cux23id_1134}{69}}}
A common incident in the \emph{Quinze joyes de manage}: the young wife
wishes a little diversion and convinces her husband that the child is
ill because she has not completed the pilgrimage she vowed to take
during her
confinement.\textsuperscript{\protect\hypertarget{13_Chapter_Six__THE_DEPICTION_OF_TH.xhtmlux5cux23id_1131}{\protect\hyperlink{23_NOTES.xhtmlux5cux23id_1132}{70}}}
The preparations for the wedding of Charles VI with Isabella of Bavaria
are launched with a
pilgrimage.\textsuperscript{\protect\hypertarget{13_Chapter_Six__THE_DEPICTION_OF_TH.xhtmlux5cux23id_1129}{\protect\hyperlink{23_NOTES.xhtmlux5cux23id_1130}{71}}}
Small wonder that the serious men of the \emph{devotio moderna} have
little use
\protect\hypertarget{13_Chapter_Six__THE_DEPICTION_OF_TH.xhtmlux5cux23page_186}{}{}for
pilgrimages. Those who go on many pilgrimages rarely become saints, says
Thomas à Kempis, and Frederick van Heilo wrote a special work about the
matter, the \emph{Contra
peregrinantes}.\textsuperscript{\protect\hypertarget{13_Chapter_Six__THE_DEPICTION_OF_TH.xhtmlux5cux23id_1127}{\protect\hyperlink{23_NOTES.xhtmlux5cux23id_1128}{72}}}

In all these sacrileges of the holy through the unabashed intermingling
with sinful life there is more naive familiarity with liturgy than open
godlessness. Only a culture that is thoroughly permeated with
religiosity and that takes faith for granted knows these excesses and
degenerations. These people, following the sloppy course of a religious
practice half gone to seed, were the same who could suddenly rise to
extremes of religious fervor when prodded by the ardent words of one of
the mendicant preachers.

Even such a stupid sin as blasphemy only arises from strong faith.
Originally a conscious invocation, blasphemy is only the sign of an
immediate consciousness, extending to the most trivial things, of the
omnipresence of the divine. Only the feeling of truly challenging heaven
gives blasphemy its sinful attraction. Only when the oath becomes
mechanical and any fear of the fulfillment of the curse has gone does
blasphemy slide into the monotonous crudeness of later times. In the
late Middle Ages it still had that attraction of daring and arrogance
that made it the sport of the nobility. ``What?'' says the nobleman to
the peasant, ``You give your soul to the devil and deny God, yet you are
not a
nobleman?''\textsuperscript{\protect\hypertarget{13_Chapter_Six__THE_DEPICTION_OF_TH.xhtmlux5cux23id_1125}{\protect\hyperlink{23_NOTES.xhtmlux5cux23id_1126}{73}}}
Deschamps reports that the habit of swearing had descended to people of
low estate.

\emph{Si chétif n'y a qui ne die}:

\emph{Je renie Dieu et sa
mère}.\textsuperscript{\protect\hypertarget{13_Chapter_Six__THE_DEPICTION_OF_TH.xhtmlux5cux23id_1123}{\protect\hyperlink{23_NOTES.xhtmlux5cux23id_1124}{74}}}\emph{\protect\hypertarget{13_Chapter_Six__THE_DEPICTION_OF_TH.xhtmlux5cux23id_3035}{\protect\hyperlink{23_NOTES.xhtmlux5cux23id_3036}{*\textsuperscript{24}}}}

People compete in the composition of new and drastic oaths; the most
profane man is honored as a
master.\textsuperscript{\protect\hypertarget{13_Chapter_Six__THE_DEPICTION_OF_TH.xhtmlux5cux23id_1121}{\protect\hyperlink{23_NOTES.xhtmlux5cux23id_1122}{75}}}
Deschamps says that originally people everywhere in France swore in
Gascon and English and later in Breton and now in Burgundian. He
composed two ballads by stringing the most popular curses together, but
he gave them a pious meaning in the end. The Burgundian oath was the
worst of all, ``Je renie
Dieu,''\textsuperscript{\protect\hypertarget{13_Chapter_Six__THE_DEPICTION_OF_TH.xhtmlux5cux23id_1120}{\protect\hyperlink{23_NOTES.xhtmlux5cux23page_419}{76}}}\protect\hypertarget{13_Chapter_Six__THE_DEPICTION_OF_TH.xhtmlux5cux23id_3037}{\protect\hyperlink{23_NOTES.xhtmlux5cux23id_3038}{†\textsuperscript{25}}}
but it was toned down to ``Je
\protect\hypertarget{13_Chapter_Six__THE_DEPICTION_OF_TH.xhtmlux5cux23page_187}{}{}renie
de bottes.''
\protect\hypertarget{13_Chapter_Six__THE_DEPICTION_OF_TH.xhtmlux5cux23id_3039}{\protect\hyperlink{23_NOTES.xhtmlux5cux23id_3040}{*\textsuperscript{26}}}
Burgundians had the reputation of being master swearers. For the rest,
said Gerson, all of France, in spite of her Christianity, suffered more
than other countries from this disgusting sin, the cause of pestilence,
war, and
famine.\textsuperscript{\protect\hypertarget{13_Chapter_Six__THE_DEPICTION_OF_TH.xhtmlux5cux23id_1118}{\protect\hyperlink{23_NOTES.xhtmlux5cux23id_1119}{77}}}
Even monks
swore.\textsuperscript{\protect\hypertarget{13_Chapter_Six__THE_DEPICTION_OF_TH.xhtmlux5cux23id_1116}{\protect\hyperlink{23_NOTES.xhtmlux5cux23id_1117}{78}}}
Gerson wanted the authorities and estates to help eradicate the evil
through strict laws, but light penalties, which could then be rigorously
carried out. In 1397 a royal order was actually issued that renewed the
old ordinances of 1269 and 1347 against swearing, however not with light
and practical penalties, but with such time-honored threats as splitting
the lips or cutting out the tongue, penalties that expressed a holy
horror of blasphemy. In the register containing the ordinance, there is
a note on the margin, ``All these oaths are today, 1411, in common use
throughout the kingdom without any
penalty.''\textsuperscript{\protect\hypertarget{13_Chapter_Six__THE_DEPICTION_OF_TH.xhtmlux5cux23id_1114}{\protect\hyperlink{23_NOTES.xhtmlux5cux23id_1115}{79}}}
Pierre d'Ailly strongly urged the Council of
Constance\textsuperscript{\protect\hypertarget{13_Chapter_Six__THE_DEPICTION_OF_TH.xhtmlux5cux23id_1112}{\protect\hyperlink{23_NOTES.xhtmlux5cux23id_1113}{80}}}
to forcefully combat this evil.

Gerson knows the two extremes between which the sin of blasphemy
fluctuates. He has learned from his experience as a confessor that
uncorrupted young people, simple and chaste, are tortured by the sharp
temptation to deny God and blaspheme. He recommends that they avoid
overarduous contemplation of God, since they are not strong enough for
that.\textsuperscript{\protect\hypertarget{13_Chapter_Six__THE_DEPICTION_OF_TH.xhtmlux5cux23id_1110}{\protect\hyperlink{23_NOTES.xhtmlux5cux23id_1111}{81}}}
On the other hand, there are habitual blasphemers like the Burgundians
whose deed, abhorrent as it is, does not include perjury since they do
not have the intent of taking an
oath.\textsuperscript{\protect\hypertarget{13_Chapter_Six__THE_DEPICTION_OF_TH.xhtmlux5cux23id_1108}{\protect\hyperlink{23_NOTES.xhtmlux5cux23id_1109}{82}}}

The point where the habit of treating matters of faith lightly becomes
irreligiousness cannot be determined with exactitude. Certainly there
was in late medieval times a strong tendency to mock piety and the
pious. Some prefer to be \emph{esprits forts} and speak jokingly against
faith.\textsuperscript{\protect\hypertarget{13_Chapter_Six__THE_DEPICTION_OF_TH.xhtmlux5cux23id_1106}{\protect\hyperlink{23_NOTES.xhtmlux5cux23id_1107}{83}}}
The popular writers are frivolous and indifferent, as in the story from
the \emph{Cent nouvelles nouvelles} where the priest buries his dog in
consecrated ground and addresses him as ``mon bon chien, a qui Dieu
pardoint.''\protect\hypertarget{13_Chapter_Six__THE_DEPICTION_OF_TH.xhtmlux5cux23id_3041}{\protect\hyperlink{23_NOTES.xhtmlux5cux23id_3042}{†\textsuperscript{27}}}
The dog, thereupon, goes ``tout droit au paradis des
chiens.''\textsuperscript{\protect\hypertarget{13_Chapter_Six__THE_DEPICTION_OF_TH.xhtmlux5cux23id_1104}{\protect\hyperlink{23_NOTES.xhtmlux5cux23id_1105}{84}}}\protect\hypertarget{13_Chapter_Six__THE_DEPICTION_OF_TH.xhtmlux5cux23id_3043}{\protect\hyperlink{23_NOTES.xhtmlux5cux23id_3044}{‡\textsuperscript{28}}}
There is a great resentment of false or mocking piety. Every other word
is
``papelard.''\protect\hypertarget{13_Chapter_Six__THE_DEPICTION_OF_TH.xhtmlux5cux23id_3045}{\protect\hyperlink{23_NOTES.xhtmlux5cux23id_3046}{§\textsuperscript{29}}}
The
\protect\hypertarget{13_Chapter_Six__THE_DEPICTION_OF_TH.xhtmlux5cux23page_188}{}{}frequently
invoked saying ``De jeune angelot vieux diable'' or, in solemn Latin
meter, ``Angelicus juvenis senibus sathanizat in
annis''\protect\hypertarget{13_Chapter_Six__THE_DEPICTION_OF_TH.xhtmlux5cux23id_3047}{\protect\hyperlink{23_NOTES.xhtmlux5cux23id_3048}{*\textsuperscript{30}}}
is for Gerson a thorn in his side. Thus youth is corrupted, he says. A
brazen face, scurrilous language and curses, immodest looks and gestures
are praised in children. So, he says, What is to be expected of children
who play the devil when they get
old?\textsuperscript{\protect\hypertarget{13_Chapter_Six__THE_DEPICTION_OF_TH.xhtmlux5cux23id_1102}{\protect\hyperlink{23_NOTES.xhtmlux5cux23id_1103}{85}}}

As to the clerics and theologians themselves, Gerson distinguishes
between types. One group is composed of ignorant troublemakers, to whom
any serious discussion is a burden and religion a fairy tale. Everything
they are told about appearances and revelations they reject with loud
laughter and great disgust. Another group goes to the opposite extreme
and accepts every product of the imagination of deranged people, their
dreams and strange ideas as
revelation.\textsuperscript{\protect\hypertarget{13_Chapter_Six__THE_DEPICTION_OF_TH.xhtmlux5cux23id_1100}{\protect\hyperlink{23_NOTES.xhtmlux5cux23id_1101}{86}}}
The populace does not know how to maintain a middle position between two
such extremes. They believe prophecies by seers and soothsayers, but if
a genuine divine who has frequently had true revelations makes a single
error, then the worldly people scorn all those who belong to the clergy,
call them imposters and ``papelards'' and henceforth will no longer
listen to any clergyman, considering them all to be malevolent
hypocrites.\textsuperscript{\protect\hypertarget{13_Chapter_Six__THE_DEPICTION_OF_TH.xhtmlux5cux23id_1098}{\protect\hyperlink{23_NOTES.xhtmlux5cux23id_1099}{87}}}

In most instances of the loudly bemoaned lack of piety we are dealing
with the sudden ending of religious tension in a mental life
oversaturated with liturgical content and forms. Throughout the entire
Middle Ages there are numerous instances of spontaneous
unbelief\textsuperscript{\protect\hypertarget{13_Chapter_Six__THE_DEPICTION_OF_TH.xhtmlux5cux23id_1096}{\protect\hyperlink{23_NOTES.xhtmlux5cux23id_1097}{88}}}
that are not deviations from church teaching based on theological
reflection, but merely direct reactions against it. Even though it does
not mean much that poets and chroniclers---encountering the enormous
sinfulness of their time---exclaimed that no one any longer believed in
heaven or
hell,\textsuperscript{\protect\hypertarget{13_Chapter_Six__THE_DEPICTION_OF_TH.xhtmlux5cux23id_1094}{\protect\hyperlink{23_NOTES.xhtmlux5cux23id_1095}{89}}}
in more than one case the latent lack of faith had become conscious and
had hardened; it had hardened to the degree that this fact was well
known by all and admitted by the unbelievers themselves. ``Beaux
seigneurs,'' says Captain Bétisac to his
comrades,\textsuperscript{\protect\hypertarget{13_Chapter_Six__THE_DEPICTION_OF_TH.xhtmlux5cux23id_1092}{\protect\hyperlink{23_NOTES.xhtmlux5cux23id_1093}{90}}}
``je ay regardé à mes besongnes et en ma conscience je tiens grandement
Dieu avoir courrouchié, car jà de long temps j'ay erré contre la foy, et
ne puis croire qu'il soit riens de la Trinité, ne que le Fils de Dieu se
daignast tant abassier que il venist des chieux descendre en corps
humain de femme, et croy et dy que, quant nous morons, que il
\protect\hypertarget{13_Chapter_Six__THE_DEPICTION_OF_TH.xhtmlux5cux23page_189}{}{}n'est
riens de âme .~.~. J'ay tenu celle oppinion depuis que j'eus
congnoissance, et la tenray jusques à la fin.''
\protect\hypertarget{13_Chapter_Six__THE_DEPICTION_OF_TH.xhtmlux5cux23id_3049}{\protect\hyperlink{23_NOTES.xhtmlux5cux23id_3050}{*\textsuperscript{31}}}
Hugues Aubriot, provost of Paris, is a fiery enemy of the clergy. He
does not believe in the Eucharist and mocks it. He does not celebrate
Easter nor go to
confession.\textsuperscript{\protect\hypertarget{13_Chapter_Six__THE_DEPICTION_OF_TH.xhtmlux5cux23id_1090}{\protect\hyperlink{23_NOTES.xhtmlux5cux23id_1091}{91}}}
Jacques du Clercq tells of several noblemen who, in full possession of
their senses, refused Extreme
Unction.\textsuperscript{\protect\hypertarget{13_Chapter_Six__THE_DEPICTION_OF_TH.xhtmlux5cux23id_1088}{\protect\hyperlink{23_NOTES.xhtmlux5cux23id_1089}{92}}}
Jean de Montreuil, provost of Lille, writes to one of his learned
friends, more in the easy style of an enlightened humanist than as one
of the truly pious: ``You know our friend Ambrosius de Miliis; you have
frequently heard what he thought about religion, faith, Holy Scripture,
and all ecclesiastical prescriptions, so that Epicurus would have to be
called Catholic in comparison. Well, he is now completely converted.
Prior to his conversion he was nonetheless tolerated in the circles of
early humanists who were of a fully pious
disposition.''\textsuperscript{\protect\hypertarget{13_Chapter_Six__THE_DEPICTION_OF_TH.xhtmlux5cux23id_1086}{\protect\hyperlink{23_NOTES.xhtmlux5cux23id_1087}{93}}}

On one side of these spontaneous instances of unbelief is the literary
paganism of the Renaissance and the educated, a cautious form of
Epicureanism, named after Averroës, which flourished in such wide
circles as early as the thirteenth century. On the other side are the
passionate negations of the ignorant heretics who, whether they be
called Turlupins or Brothers of the Free Spirit, crossed the line
separating mysticism and pantheism. But these phenomena will be dealt
with in a different context later on. For the time being, we have to
remain in the sphere of external images of faith and external forms and
customs.

For the daily understanding of the mass of people, the existence of a
visible image made intellectual proof of faith entirely superfluous.
There was no room between what was depicted, and which one met in color
and form---that is, depictions of the Trinity, the flames of hell, the
catalog of saints---and faith in all this. There was no room for the
question, Is this true? All these representations went directly from
picture to belief. They existed in the mind fully
\protect\hypertarget{13_Chapter_Six__THE_DEPICTION_OF_TH.xhtmlux5cux23page_190}{}{}defined
and garbed in all the reality that the church could demand of faith and
then some.

But where faith rests on tangible images, it is hardly possible to make
qualitative distinctions between the nature and degree of sanctity of
the different elements of faith. One picture is as real and commands as
much awe as another. That God is to be worshiped and the saints merely
venerated is not taught by the picture itself and the difference is lost
unless the church constantly warns about the necessary distinction.
Nowhere else were pious notions so seriously threatened by the
overgrowth of colorful images than in the field of the veneration of
saints.

The strict position of the church was simple and elevated enough. Given
the belief in the continued existence of personalities after death, the
veneration of saints was natural and unquestioned. It was permissible to
honor them ``per imitationem et reductionem ad
Deum.''\protect\hypertarget{13_Chapter_Six__THE_DEPICTION_OF_TH.xhtmlux5cux23id_3051}{\protect\hyperlink{23_NOTES.xhtmlux5cux23id_3052}{*\textsuperscript{32}}}
In the same sense, it was permissible to venerate pictures, relics, holy
places, and consecrated objects since, ultimately, all this led to the
worship of God
himself.\textsuperscript{\protect\hypertarget{13_Chapter_Six__THE_DEPICTION_OF_TH.xhtmlux5cux23id_1084}{\protect\hyperlink{23_NOTES.xhtmlux5cux23id_1085}{94}}}
The technical distinction between the saints and ordinary people who
achieved salvation was established by official canonization. This
distinction, although a troublesome formalization, contained nothing
that contradicted the spirit of Christianity. The church remained aware
that sanctity and bliss were originally of equal value, just as it was
aware that canonization was somehow flawed. ``It may be assumed,'' said
Gerson, ``that infinitely more saints have died, and die everyday, than
have been
canonized.''\textsuperscript{\protect\hypertarget{13_Chapter_Six__THE_DEPICTION_OF_TH.xhtmlux5cux23id_1082}{\protect\hyperlink{23_NOTES.xhtmlux5cux23id_1083}{95}}}
That pictures were permitted, even though their existence violated the
prohibition of such in the second commandment, was justified by the
teaching that this prohibition had been necessary prior to the
Incarnation because God had been only spirit at that time, but that
Christ's coming had canceled that old commandment. Yet, the church
desired unconditionally to obey the rest of the second commandment,
``Non adorabis ea neque
coles,''\protect\hypertarget{13_Chapter_Six__THE_DEPICTION_OF_TH.xhtmlux5cux23id_3053}{\protect\hyperlink{23_NOTES.xhtmlux5cux23id_3054}{†\textsuperscript{33}}}
``We do not adore the images, but honor and adore he who is depicted,
God or His saint in whose image it
is.''\textsuperscript{\protect\hypertarget{13_Chapter_Six__THE_DEPICTION_OF_TH.xhtmlux5cux23id_1080}{\protect\hyperlink{23_NOTES.xhtmlux5cux23id_1081}{96}}}
Images were only intended to show the simpleminded, who did not know the
scriptures, what to believe
in.\textsuperscript{\protect\hypertarget{13_Chapter_Six__THE_DEPICTION_OF_TH.xhtmlux5cux23id_1078}{\protect\hyperlink{23_NOTES.xhtmlux5cux23id_1079}{97}}}

\protect\hypertarget{13_Chapter_Six__THE_DEPICTION_OF_TH.xhtmlux5cux23page_191}{}{}They
were the books of the
simpleminded,\textsuperscript{\protect\hypertarget{13_Chapter_Six__THE_DEPICTION_OF_TH.xhtmlux5cux23id_1076}{\protect\hyperlink{23_NOTES.xhtmlux5cux23id_1077}{98}}}
as we see from this prayer that Villon composed for his mother:

\emph{Femme je suis pourette et ancienne},

\emph{Qui riens ne sçai; oncques lettre ne leuz};

\emph{Au moustier voy dont suis paroissienne}

\emph{Paradis paint, ou sont harpes et luz},

\emph{Et ung enfer où dampnez sont boulluz}:

\emph{Uung me fait paour, I'autre joye et liesse .} .
.\textsuperscript{\protect\hypertarget{13_Chapter_Six__THE_DEPICTION_OF_TH.xhtmlux5cux23id_1074}{\protect\hyperlink{23_NOTES.xhtmlux5cux23id_1075}{99}}}\protect\hypertarget{13_Chapter_Six__THE_DEPICTION_OF_TH.xhtmlux5cux23id_3055}{\protect\hyperlink{23_NOTES.xhtmlux5cux23id_3056}{*\textsuperscript{34}}}

The church was mindful of the fact that to the simple mind just as much
opportunity to stray was offered by colorful pictures as by any personal
interpretation of scripture. It always treated those gently who lapsed
into the worship of images out of ignorance or simplemindedness. ``It is
enough,'' says Gerson, ``if they intend to do as the church
requires.''\textsuperscript{\protect\hypertarget{13_Chapter_Six__THE_DEPICTION_OF_TH.xhtmlux5cux23id_1072}{\protect\hyperlink{23_NOTES.xhtmlux5cux23id_1073}{100}}}

The question, purely one of history of dogma, whether the church always
managed to keep its injunction against direct veneration or even worship
of saints, not as intercessors but as the granters of requests, we can
leave as it is. We are dealing here, as a question of cultural history,
with how far it succeeded in keeping the people from error; that is,
what reality, what value for understanding did the saints have in the
popular consciousness of the late Middle Ages? To this question only one
answer is possible: the saints were such essential, material, and
familiar figures of the everyday life of faith that all the common and
sensual religious impulses were tied up with them. While the most
fervent emotions turned towards Christ and Mary, an entire store of
naive and everyday religious feeling crystalized around the veneration
of saints. All this helped to keep popular saints in the middle of
ordinary life. Popular imagination took hold of them: their figures are
as familiar as their attributes. Their gruesome tortures are known as
well as their astonishing miracles. They are dressed and endowed like
the people themselves. One could meet, everyday, ``Messires'' St. Roch
or St.
\protect\hypertarget{13_Chapter_Six__THE_DEPICTION_OF_TH.xhtmlux5cux23page_192}{}{}James
in the persons of living plague victims or pilgrims. It would be
interesting to study how long the dress of saints accorded with the
fashions of the day; certainly for the entire fifteenth century. But
where is the point at which church art removed them from living popular
imagination by dressing them in rhetorical robes? This was not just a
case of Renaissance sensitivity to historical costume; as an added
element, the popular imagination itself began to abandon them so that
they were no longer able to hold their own in popular church art. During
the Counter-Reformation, the saints, quite in line with the intent of
the church, climbed up several steps and moved out of touch with popular
life.

The physical presence that the saints possessed by virtue of their
depictions was unusually intensified by the fact that the church
permitted and even favored the veneration of their relics. It could not
be other than that this clinging to the material had a materializing
effect on faith that occasionally led to astonishing extremes. The
vigorous faith of the Middle Ages, whenever directed towards relics, was
not deterred by fear of secularization or desecration. The people of the
mountains of Umbria around the year 1000 tried to kill St. Romuald in
order to secure his bones. The monks of Fossanuova where Thomas Aquinas
died were so fearful of losing the precious relic that they did not
shrink from decapitating, boiling, and preserving the
corpse.\textsuperscript{\protect\hypertarget{13_Chapter_Six__THE_DEPICTION_OF_TH.xhtmlux5cux23id_1070}{\protect\hyperlink{23_NOTES.xhtmlux5cux23id_1071}{101}}}
Before St. Elizabeth of Thuringia was buried, a crowd of devotees cut or
tore strips from the winding-sheets of her face and cut off her hair and
nails, pieces of her ears and even her
nipples.\textsuperscript{\protect\hypertarget{13_Chapter_Six__THE_DEPICTION_OF_TH.xhtmlux5cux23id_1068}{\protect\hyperlink{23_NOTES.xhtmlux5cux23id_1069}{102}}}
At a solemn feast, Charles VI gave ribs of his ancestor St. Louis to
Pierre d'Ailly and to his uncles Berry and Burgundy. He gave a leg to
the prelates, who divided it after the
meal.\textsuperscript{\protect\hypertarget{13_Chapter_Six__THE_DEPICTION_OF_TH.xhtmlux5cux23id_1066}{\protect\hyperlink{23_NOTES.xhtmlux5cux23id_1067}{103}}}

No matter how real and alive the saints seemed to be, only relatively
few appear in supernatural experiences. The entire realm of visions,
appearances, signs, and ghosts remains separate from the popular
imagination about saints, but there are, of course, exceptions. The
figures of St. Michael, St. Catherine, and St. Margaret, who appeared to
Joan of Arc, come to mind. We could also cite a number of examples from
the visionary literature, but as a rule the examples we encounter in
these stories were embellished and interpreted, so to speak. When the
fourteen holy martyrs, who were so clearly identified by
iconography,\textsuperscript{\protect\hypertarget{13_Chapter_Six__THE_DEPICTION_OF_TH.xhtmlux5cux23id_1064}{\protect\hyperlink{23_NOTES.xhtmlux5cux23id_1065}{104}}}
appear to a young shepherd of Frankenthal near Bamberg in 1446, he does
not see
\protect\hypertarget{13_Chapter_Six__THE_DEPICTION_OF_TH.xhtmlux5cux23page_193}{}{}them
with their proper attributes, but as fourteen identical cherubim. They
\emph{tell} him they are the holy martyrs. The popular phantasmagoria is
filled with angels and devils, ghosts and white-clad women, but not with
saints. Only in exceptional cases do saints play a role in genuine, that
is not literary or theologically embellished, superstition. St. Bertulph
does at Ghent. Anytime something important is about to happen, he knocks
on his coffin in the abby of St. Peter ``moult dru et moult
fort.''\protect\hypertarget{13_Chapter_Six__THE_DEPICTION_OF_TH.xhtmlux5cux23id_3057}{\protect\hyperlink{23_NOTES.xhtmlux5cux23id_3058}{*\textsuperscript{35}}}
Sometimes the knocking is accompanied by a light earthquake so that the
frightened city tries to ward off the unknown danger by large
processions.\textsuperscript{\protect\hypertarget{13_Chapter_Six__THE_DEPICTION_OF_TH.xhtmlux5cux23id_1062}{\protect\hyperlink{23_NOTES.xhtmlux5cux23id_1063}{105}}}
But generally, cold fear attaches itself to vaguely imagined figures
rather than to the sharply chiseled images in the church. Just like
ghosts, the imagined move about aimlessly, show an indeterminate
expression of the horrible in a nebulous gown, or rising from the remote
recesses of the brain, show themselves in pure heavenly radiances or in
terrifying illusionary forms.

We should not be surprised by all this. It is precisely because the
saints had assumed such definite forms and material character that they
lacked horror and mystery. Supernatural fear results from unbridled
imagination, from the possibility that something new and dreadful could
suddenly appear. As soon as the image becomes clearly drawn and defined
it arouses a feeling of security and familiarity. The well-known figures
of the saints had the reassuring quality of the sight of a policeman in
a foreign city. Their veneration, and particularly their depiction,
created a neutral zone of comfortably calm faith between the ecstasy of
the vision of God and the sweet shudder of the love of Christ on the one
hand and the terrifying phantasmagoria, born of the fear of the devil
and the frenzy of witchcraft, on the other.

One could even posit that the veneration of saints was a very healthy
tempering of the exuberance of the medieval mind, since it was able to
deflect many visions of bliss and many fears and reduce them to familiar
notions.

By virtue of its perfectly pictorial quality, the veneration of saints
belongs to the outward manifestations of religion. It moves along with
the stream of everyday thought and occasionally loses its dignity in
this stream. The medieval veneration of Joseph is a case in point. It
might be looked upon as both a consequence of the
pas\protect\hypertarget{13_Chapter_Six__THE_DEPICTION_OF_TH.xhtmlux5cux23page_194}{}{}sionate
veneration of Mary and a backlash against it. This disrespectful
interest in the stepfather is the other side of the coin to all the love
and glorification showered on the Virgin Mother. The higher Mary rose,
the more Joseph became a mere caricature. Fine art had already given him
a form dangerously close to that of an uncouth peasant; thus is he
depicted on Melchoir Broederlam's diptych at Dijon. But in the fine arts
the most profane aspects remain unexpressed. Rather than hold that no
mortal could be more highly favored than Joseph, privileged to serve the
mother of God and raise her Son, Eustace Deschamps prefers, with naive
sobriety, but not godless mockery, to see him as the type of drudging
pitiful husband.

\emph{Vous qui servez à femme et à enfans},

\emph{Aiez Joseph toudis en remembrance};

\emph{Femme servit toujours tristes, dolans},

\emph{Et Jhesu Crist garda en son enfance};

\emph{A piè trotoit, son fardel sur sa lance};

\emph{En plusieurs lieux est figuré ainsi},

\emph{Lez un mulet, pour leur faire plaisance},

\emph{Et si n'ot oncq feste en ce monde
ci}.\textsuperscript{\protect\hypertarget{13_Chapter_Six__THE_DEPICTION_OF_TH.xhtmlux5cux23id_1060}{\protect\hyperlink{23_NOTES.xhtmlux5cux23id_1061}{106}}}\protect\hypertarget{13_Chapter_Six__THE_DEPICTION_OF_TH.xhtmlux5cux23id_3059}{\protect\hyperlink{23_NOTES.xhtmlux5cux23id_3060}{*\textsuperscript{36}}}

We could accept this if it were intended to console troubled husbands by
holding up for them a noble example, though the presentation is lacking
dignity. But Deschamps uses Joseph as a virtual warning against taking
up the burdens of a family.

\emph{Qu'ot Joseph de povreté}

\emph{De dureté}

\emph{De maleurté}

\emph{Quant Dieux nasquil!}

\emph{Maintefois Va comporté}

\emph{Et monté}

\emph{Par bonté}

\emph{\protect\hypertarget{13_Chapter_Six__THE_DEPICTION_OF_TH.xhtmlux5cux23page_195}{}{}Avec
sa mère autressi},

\emph{Sur sa mule les ravi}:

\emph{je le vi}

\emph{paint ainsi};

\emph{en Egipte en est alé}.

\emph{Le bonhomme est painturé}

\emph{tout lassé}

\emph{Et troussé}

\emph{D'un cote et d'un barry}.

\emph{Un baston au coul posé}

\emph{Vieil usé}

\emph{Et rusé}.

\emph{Feste n'a en ce monde cy},

\emph{Mais de lui}

\emph{Va le cri}

\emph{C'est Joseph le
rassoté}.\textsuperscript{\protect\hypertarget{13_Chapter_Six__THE_DEPICTION_OF_TH.xhtmlux5cux23id_1058}{\protect\hyperlink{23_NOTES.xhtmlux5cux23id_1059}{107}}}\emph{\protect\hypertarget{13_Chapter_Six__THE_DEPICTION_OF_TH.xhtmlux5cux23id_3061}{\protect\hyperlink{23_NOTES.xhtmlux5cux23id_3062}{*\textsuperscript{37}}}}

This shows how from the familiar image arose an all too familiar
conception that threatened any sense of sanctity. Joseph remained a
semi-comic figure. Dr. Johannes Eck still had to insist that he not
appear in Christmas plays if not in a proper depiction or at least that
he not be made to cook the porridge ``ne ecclesia Dei
irredeatur.''\textsuperscript{\protect\hypertarget{13_Chapter_Six__THE_DEPICTION_OF_TH.xhtmlux5cux23id_1057}{\protect\hyperlink{23_NOTES.xhtmlux5cux23page_420}{108}}}
Gerson's effort for a proper veneration of Joseph that eventually led to
the saint's inclusion in the liturgy in preference to all others was
motivated by these undignified
excesses.\textsuperscript{\protect\hypertarget{13_Chapter_Six__THE_DEPICTION_OF_TH.xhtmlux5cux23id_1055}{\protect\hyperlink{23_NOTES.xhtmlux5cux23id_1056}{109}}}
We have already seen, however, how Gerson's seriousmindedness did not
keep him from immodest curiosity about things that seem to be inevitably
linked to Joseph's marriage. Sober minds (and Gerson, despite his
predilection for mysticism, was in many respects a sober mind) were
often led by contemplations of Mary's marriage to considerations of an
earthly sort. The Knight de la Tour Landry, also a typically sober and
correct fellow, sees it in this light: ``Dieux voulst que elle espousast
le saint homme Joseph, qui estoit vieulx
\protect\hypertarget{13_Chapter_Six__THE_DEPICTION_OF_TH.xhtmlux5cux23page_196}{}{}et
preudomme; car Dieu voust naistre soubz umbre de mariage pour obéir à la
loy qui lors couroit, \emph{pour exchever les paroles du
monde}.''\textsuperscript{\protect\hypertarget{13_Chapter_Six__THE_DEPICTION_OF_TH.xhtmlux5cux23id_1053}{\protect\hyperlink{23_NOTES.xhtmlux5cux23id_1054}{110}}}\emph{\protect\hypertarget{13_Chapter_Six__THE_DEPICTION_OF_TH.xhtmlux5cux23id_3165}{\protect\hyperlink{23_NOTES.xhtmlux5cux23id_3166}{*\textsuperscript{38}}}}
An unpublished work of the fifteenth century presents the mystic
marriage of the soul with the heavenly bridegroom in the customary terms
of a bourgeois courtship. Jesus, the Bridegroom, tells God the Father:
``S'il te plaist, jeme mariray et auray grant foueson d'enfants et de
famille,''\protect\hypertarget{13_Chapter_Six__THE_DEPICTION_OF_TH.xhtmlux5cux23id_3167}{\protect\hyperlink{23_NOTES.xhtmlux5cux23id_3168}{†\textsuperscript{39}}}
The Father objects to his son's choice, a black Ethiopian. Here the
passage from the Song of Songs is echoed: ``Nigra sum sed
formosa.''\protect\hypertarget{13_Chapter_Six__THE_DEPICTION_OF_TH.xhtmlux5cux23id_3169}{\protect\hyperlink{23_NOTES.xhtmlux5cux23id_3170}{‡\textsuperscript{40}}}
Such a union would be a misalliance and dishonor the family. The angel
serving as intermediary puts in a good word for the bride: ``Combien que
ceste fille soit noire, neanmoins elle est gracieuse, et a belle
composicion de corps et de membres et est bien habille pour porter
fouezon
d'enfans.''\protect\hypertarget{13_Chapter_Six__THE_DEPICTION_OF_TH.xhtmlux5cux23id_3171}{\protect\hyperlink{23_NOTES.xhtmlux5cux23id_3172}{§\textsuperscript{41}}}
The father responds: ``Mon cher fils m'a dit qu'elle est noir et
brunete. Certes je vueil que son espouse soit jeune, courtoise, joyle,
gracieuse et belle et qu'elle ait beaux
membres.''\protect\hypertarget{13_Chapter_Six__THE_DEPICTION_OF_TH.xhtmlux5cux23id_3173}{\protect\hyperlink{23_NOTES.xhtmlux5cux23id_3174}{**\textsuperscript{42}}}
The angel then praises her face and all her limbs, which are the virtues
of her soul. The father declares himself bested and tells his son:

\emph{Prens la, car elle est plaisant}

\emph{Pour bien amer son doulx amant};

\emph{Or prens de nois biens largement}

\emph{Et luy en donne
habondamment}.\textsuperscript{\protect\hypertarget{13_Chapter_Six__THE_DEPICTION_OF_TH.xhtmlux5cux23id_1051}{\protect\hyperlink{23_NOTES.xhtmlux5cux23id_1052}{111}}}\protect\hypertarget{13_Chapter_Six__THE_DEPICTION_OF_TH.xhtmlux5cux23id_3175}{\protect\hyperlink{23_NOTES.xhtmlux5cux23id_3176}{††\textsuperscript{43}}}

There is no doubt of the seriously devout intent of this work. It is
only one example of how unbridled imagination leads to triviality.

Every saint, by the possession of a distinct and vivid outward
\protect\hypertarget{13_Chapter_Six__THE_DEPICTION_OF_TH.xhtmlux5cux23page_197}{}{}shape,
had his own marked
personality,\textsuperscript{\protect\hypertarget{13_Chapter_Six__THE_DEPICTION_OF_TH.xhtmlux5cux23id_1049}{\protect\hyperlink{23_NOTES.xhtmlux5cux23id_1050}{112}}}
quite different from the angels, who, with the exception of the three
great archangels, were never given personalized images. The personality
of each saint was strongly accentuated by the special function that each
one had. People turned to one saint for a certain emergency and to
another for recovery from a certain disease. Frequently a detail of the
saint's legend or an attribute of a depiction was the source of the
specialization, as in the case of St. Apollonia, who had her teeth
pulled during her martyrdom and was thus appealed to in case of
toothache. Once the functions of saints became so specialized, it was
inevitable that their veneration became somewhat mechanical. When the
cure of plague was attributed to St. Roch, it was inevitable that too
much stress was laid on his part in the healing and that the chain of
thought required by sound doctrine, namely that the saint worked his
healing by interceding with God, was in danger of being left out
altogether. This was notably the case in regard to the fourteen holy
martyrs (sometimes five, eight, ten, or fifteen) whose veneration was
especially important towards the end of the medieval period. St. Barbara
and St. Christopher are the most frequently depicted of this group.
According to popular tradition, God had granted to the fourteen the
power of warding off any imminent danger through the mere invocation of
their name.

\emph{Ilz sont cinq sains, en la genealogie},

\emph{Et cinq sainctes, à qui Dieu octria}

\emph{Benignement a la fin de leur vie},

\emph{Que quiconques de cuer les requerra},

\emph{En tous perilz, que Dieu essaucera}

\emph{Leur prieres, pour quelconque mesaise}.

\emph{Saiges est doc qui ces cinq servira},

\emph{Jorges, Denis, Christofle, Giles et
Blaise}.\textsuperscript{\protect\hypertarget{13_Chapter_Six__THE_DEPICTION_OF_TH.xhtmlux5cux23id_1047}{\protect\hyperlink{23_NOTES.xhtmlux5cux23id_1048}{113}}}\emph{\protect\hypertarget{13_Chapter_Six__THE_DEPICTION_OF_TH.xhtmlux5cux23id_3063}{\protect\hyperlink{23_NOTES.xhtmlux5cux23id_3064}{*\textsuperscript{44}}}}

In the popular imagination, any notion of the purely interceding
function was bound to be entirely lost by virtue of this delegation of
\protect\hypertarget{13_Chapter_Six__THE_DEPICTION_OF_TH.xhtmlux5cux23page_198}{}{}omnipotent
and spontaneous effect. The holy martyrs had become prefects of the
Deity. Various missals of the late medieval period that contain the
office of the fourteen holy martyrs clearly express the binding
character of their intercession: ``Deus qui electos sanctos tuos
Georgium etc. etc. specialibus privilegiis prae cunctis aliis decorasti,
ut omnes, qui in necessitatibus suis eorum implorant auxilium, secundum
promissionem tuae gratiae petitionis suae salutarem consequantur
effectum.''\textsuperscript{\protect\hypertarget{13_Chapter_Six__THE_DEPICTION_OF_TH.xhtmlux5cux23id_1045}{\protect\hyperlink{23_NOTES.xhtmlux5cux23id_1046}{114}}}\protect\hypertarget{13_Chapter_Six__THE_DEPICTION_OF_TH.xhtmlux5cux23id_3065}{\protect\hyperlink{23_NOTES.xhtmlux5cux23id_3066}{*\textsuperscript{45}}}
After the Council of Trent, the church abolished the mass of the Holy
Martyrs because of the danger that faith would attach itself to them as
to a talisman. In fact, it was already the case that a daily viewing of
the image of St. Christopher was considered sufficient for protection
against any
fatality.\textsuperscript{\protect\hypertarget{13_Chapter_Six__THE_DEPICTION_OF_TH.xhtmlux5cux23id_1043}{\protect\hyperlink{23_NOTES.xhtmlux5cux23id_1044}{115}}}

As to the reason that these fourteen were turned into a welfare company,
it should be noted that their depictions all had sensational attributes
that stimulated the imagination. St. Achatius had a crown of thorns, St.
Giles was accompanied by a hind, St. George by a dragon, St. Blaise was
in a den with wild beasts, St. Christopher was a giant, St. Cyriac had
the devil in chains. St. Denis was carrying his own head under his arm,
St. Erasmus was in his gruesome torture being disemboweled on the rack,
St. Eustace was with a stag carrying a cross between his antlers, St.
Pantaleon was depicted as a physician with a lion, St. Vitus in a
cauldron, St. Barbara in her tower, St. Catherine with her wheel and
sword, St. Margaret with a
dragon.\textsuperscript{\protect\hypertarget{13_Chapter_Six__THE_DEPICTION_OF_TH.xhtmlux5cux23id_1041}{\protect\hyperlink{23_NOTES.xhtmlux5cux23id_1042}{116}}}
It cannot be ruled out that the special attention given these fourteen
arose from the characteristics of their images.

A number of different saints were linked with specific diseases, such as
St. Anthony with various festering skin diseases, St. Maur with gout,
St. Sebastian, St. Roch, St. Giles, St. Christopher, St. Valentine, St.
Adrian with plague. Here we find yet another cause of the degeneration
of popular religion: the disease was named after the saint, St.
Anthony's fire, ``mal de St. Maur,'' and many others. The saint was
therefore from the very beginning in the forefront of the mind of those
who thought about the disease. Those
\protect\hypertarget{13_Chapter_Six__THE_DEPICTION_OF_TH.xhtmlux5cux23page_199}{}{}thoughts
were charged with violent swings of emotion, with fear and disgust. This
is particularly true with respect to the plague. The saints linked to
the plague were most eagerly venerated during the fifteenth century:
with services in the churches, through processions, brotherhoods, as
virtual spiritual health insurance. How easily the strong awareness of
God's wrath, rekindled by each epidemic, could be deflected against the
saint who took over as cause. The disease was not caused by God's
unfathomable justice, but by the wrath of the saint who sent the illness
and demanded propitiation. If he cured the disease, why should he have
not caused it in the first place? This constituted a heathen
transposition of faith from the religious-ethical to the magical sphere.
The church could have been held responsible for this only to the extent
that it did not take sufficiently into account that its pure teaching
would become clouded in ignorant minds.

The testimony for the presence of this notion among the people is so
large that it rules out any doubt that among the circles of the ignorant
the saints were occasionally really regarded as having caused the
disease. ``Que Saint Antoine me
arde''\protect\hypertarget{13_Chapter_Six__THE_DEPICTION_OF_TH.xhtmlux5cux23id_3067}{\protect\hyperlink{23_NOTES.xhtmlux5cux23id_3068}{*\textsuperscript{46}}}
is a common curse. ``Saint Antoine arde le tripot,'' ``Saint Antoine
arde la
monture!''\textsuperscript{\protect\hypertarget{13_Chapter_Six__THE_DEPICTION_OF_TH.xhtmlux5cux23id_1039}{\protect\hyperlink{23_NOTES.xhtmlux5cux23id_1040}{117}}}\protect\hypertarget{13_Chapter_Six__THE_DEPICTION_OF_TH.xhtmlux5cux23id_3069}{\protect\hyperlink{23_NOTES.xhtmlux5cux23id_3070}{†\textsuperscript{47}}}
are curses in which the saint functions entirely as an evil fire-demon.

\emph{Saint Anthoine me vent trop chier Son mal},

\emph{le feu ou corps me
boute}.\protect\hypertarget{13_Chapter_Six__THE_DEPICTION_OF_TH.xhtmlux5cux23id_3071}{\protect\hyperlink{23_NOTES.xhtmlux5cux23id_3072}{‡\textsuperscript{48}}}

Deschamps has a beggar say about his skin disease. And he barks at a
sufferer from gout: if you cannot walk, you at least save the road fee.

\emph{Saint Mor ne te fera
fremir}.\textsuperscript{\protect\hypertarget{13_Chapter_Six__THE_DEPICTION_OF_TH.xhtmlux5cux23id_1037}{\protect\hyperlink{23_NOTES.xhtmlux5cux23id_1038}{118}}}\protect\hypertarget{13_Chapter_Six__THE_DEPICTION_OF_TH.xhtmlux5cux23id_3073}{\protect\hyperlink{23_NOTES.xhtmlux5cux23id_3074}{§\textsuperscript{49}}}

Robert Gaguin, who did not attack the veneration of saints \emph{per se}
in a poem of ridicule, ``De validorum per Francium
mendi\protect\hypertarget{13_Chapter_Six__THE_DEPICTION_OF_TH.xhtmlux5cux23page_200}{}{}cantium
varia astucia,'' describes beggars as follows: ``This one falls to earth
while spitting stinking saliva and he rants that this is a miracle
worked by St. John. Others are visited with pustules by St. Fiacrius.
And you, O Damianius, keep me from passing water. St. Anthony makes
their joints burn with miserable fire. St. Pius turns them into cripples
and paralyses their
limbs.''\textsuperscript{\protect\hypertarget{13_Chapter_Six__THE_DEPICTION_OF_TH.xhtmlux5cux23id_1035}{\protect\hyperlink{23_NOTES.xhtmlux5cux23id_1036}{119}}}

Erasmus ridicules the same popular belief when he has Theotimus respond
to the question by Philecous whether saints are worse in heaven or on
earth: ``Yes, the saints who reign in heaven should not be insulted.
When they were alive who was more gentle than Cornelius, who more good
natured than Anthony, more patient than John the Baptist? But what
terrible diseases they send now if they are not, as you have heard,
venerated
properly.''\textsuperscript{\protect\hypertarget{13_Chapter_Six__THE_DEPICTION_OF_TH.xhtmlux5cux23id_1033}{\protect\hyperlink{23_NOTES.xhtmlux5cux23id_1034}{120}}}
Rabelais claims that popular preachers themselves told the congregations
that St. Sebastian was the originator of the plague and St. Eutropius of
dropsy (because of the phonetic similarity with
\emph{ydeopique}).\textsuperscript{\protect\hypertarget{13_Chapter_Six__THE_DEPICTION_OF_TH.xhtmlux5cux23id_1031}{\protect\hyperlink{23_NOTES.xhtmlux5cux23id_1032}{121}}}
Henry Estienne also mentions such
belief.\textsuperscript{\protect\hypertarget{13_Chapter_Six__THE_DEPICTION_OF_TH.xhtmlux5cux23id_1029}{\protect\hyperlink{23_NOTES.xhtmlux5cux23id_1030}{122}}}

The emotional and intellectual content of the veneration of saints had
been defined to such a large extent by the colors and forms of the
images that the direct aesthetic perception continuously threatened to
cancel out the religious notion. Between the sight of the radiance of
the gold, the scrupulously faithful description of the material of their
clothing, the pious look of the eyes, and the living reality of the
saints in the popular consciousness, there was hardly any room left for
considering what the church permitted and what it prohibited as
offerings of veneration and devotion to these splendid beings. The
saints lived in the minds of the people as gods. It is not surprising
that this danger to popular piety was feared in the concerned circles of
the Windesheimers, who were seeking to maintain a proper faith; but it
is also noticeable when the same idea strikes the mind of a superficial
and banal court poet such as Eustace Deschamps, since he, in all his
limitations, is such an excellent mirror of the intellectual life of his
times.

\emph{Ne faictes pas les dieux d'argent}

\emph{D'or, de fust, de pierre ou d'arain}.

\emph{Qui font ydolatrer la gent} . ..

\emph{Car l'ouvrage est forme plaisant};

\emph{Leur painture dont je me plain}

\emph{La beauté de l'or reluisant}.

\emph{\protect\hypertarget{13_Chapter_Six__THE_DEPICTION_OF_TH.xhtmlux5cux23page_201}{}{}Font
croire à maint peuple incertain}

\emph{que ce soient dieu pour certain},

\emph{Et servent par pensées foies}

\emph{Telz ymages qui font caroles}

\emph{Es moustiers où trop en mettons};

\emph{C'est tres mal fait: a brief paroles},

\emph{Telz simulacres n'aourons}.

. \emph{.~.~. . .~.~. . .~.~. . .~.~. . .~.~. .} .

\emph{Prince, un Dieu croions seulement}

\emph{Et aourons parfaictement}

\emph{Aux champs, partout, car c'est raisons},

\emph{Non pas fautz dieux, fer n'ayment},

\emph{Pierres qui n'ont entendement}:

\emph{Telz simulacres
n'aourons}.\textsuperscript{\protect\hypertarget{13_Chapter_Six__THE_DEPICTION_OF_TH.xhtmlux5cux23id_1027}{\protect\hyperlink{23_NOTES.xhtmlux5cux23id_1028}{123}}}\emph{\protect\hypertarget{13_Chapter_Six__THE_DEPICTION_OF_TH.xhtmlux5cux23id_3075}{\protect\hyperlink{23_NOTES.xhtmlux5cux23id_3076}{*\textsuperscript{50}}}}

Should we not regard the clamor for the veneration of angels as a
conscious reaction against the veneration of saints? Living faith had
crystalized too firmly in the veneration of saints; a need arose for a
more fluid understanding of veneration and ideas about protection. These
could attach themselves to the barely envisioned images of angels and
could thus again become unmediated religious experience. It is again
Gerson---this conscientious zealot for purity of faith---who repeatedly
recommends the veneration of guardian
angels.\textsuperscript{\protect\hypertarget{13_Chapter_Six__THE_DEPICTION_OF_TH.xhtmlux5cux23id_1025}{\protect\hyperlink{23_NOTES.xhtmlux5cux23id_1026}{124}}}
But there arises again the dangerous preoccupation with details that
could only damage the pious substance of this veneration. The
``studiositas theologorum,'' says Gerson, raises a number of questions
with respect to angels: whether they ever leave us, whether they know in
advance if we will be elected or damned, whether Christ or Mary had a
guardian angel, and whether the Antichrist will have one. Whether our
good angel can speak to our soul without the images of the imagination,
whether they spur us
\protect\hypertarget{13_Chapter_Six__THE_DEPICTION_OF_TH.xhtmlux5cux23page_202}{}{}to
do good just as the devil spurs us to do evil, whether they can see our
thoughts. How numerous the questions are. These \emph{studiositas},
Gerson concludes, belong to the theologians, but \emph{curiositas}
should be far removed from all those who should attend more to devotion
than to subtle
speculation.\textsuperscript{\protect\hypertarget{13_Chapter_Six__THE_DEPICTION_OF_TH.xhtmlux5cux23id_1023}{\protect\hyperlink{23_NOTES.xhtmlux5cux23id_1024}{125}}}

The Reformation, a century later, found the veneration of saints nearly
defenseless at a time when it did not attack belief in witches and
devils as such. It did not even attempt to do so because it itself was
still caught up in that belief. Was not this caused by the fact that the
veneration of saints had become \emph{caput mortuum}, that everything in
the veneration of saints had been expressed so completely in image,
legend, and prayer that it was no longer sustained by gripping awe? The
veneration of saints no longer had any roots in something unformed or
unexpressed---roots in which demonic thought was strongly anchored. And
when the Counter-Reformation cultivated anew a purified veneration of
saints, it had to work on the mind with the gardener's knife of a more
strict discipline so as to prune the all too luxuriant growth of the
popular imagination.
