\chapter{BIBLIOGRAPHY}

Achéry, Luc d', \emph{Spicilegium}, nova ed., Paris, 1723, III, p. 730:
\emph{Statuts de l'ordre de l'Etoile}.

Acquoy, J. G. R., \emph{Het klooster van Windesheim en zijn invloed}, 3
vols., Utrecht, 1875--80.

\emph{Acta Sanctorum}, see Colette, François de Paule, Pierre de
Luxembourg, Pierre Thomas, Vincent Ferrer.

Ailly Pierre d', \emph{De falsis prophetis}, in Gerson, \emph{Opera}, I,
p. 538; \emph{De Reformatione}, ibid., II, p. 911; \emph{Tractatus I
adversus cancellarium Parisiensem}, ibid., I, p. 723.

Alain de la Roche = Alanus de Rupe, Beatus Alanus redivivus, ed. J. A.
Coppenstein, Naples, 1642.

\emph{Amant rendu cordelier à l'observance d'amours, L'}, poème attribué
à Martial d'Auvergne, published by A. de Montaiglon (Société des anciens
textes français), 1881.

Anitchkoff, E., \emph{L'esthétique au moyen âge}, Le Moyen Age, vol. XX
(1918), p. 221.

Baisieux, Jacques de, \emph{Des trois chevaliers et del chainse},
Scheler, Trouvères belges, vol. I, 1876.

Basin, Thomas, \emph{De rebus gestis Caroli VII et Ludovici XI
historiarum libri XII}, ed. Quicherat, Société de l'histoire de France,
4 vols., 1855--59.

\emph{Baude, Les vers de maître Henri}, ed. Quicherat, Trésors des
pièces rares ou inédites, 1856.

Beatis, Antonio de, \emph{Die Reise des Kardinals Luigi d'Aragona}, ed.
L. von Pastor, Freiburg, 1905.

Becker, C. H., \emph{Ubi sunt qui ante nos in mundo fuere, Islamstudien
I}, 1924, p. 501.

Bertoni, G., \emph{L'Orlando furioso e la rinacenza a Ferrara}, Modena,
1919.

\emph{Blois, Extraict de l'enqueste faite pour la canonization de
Charles de}, in André du Chesne, \emph{Histoire de la maison de
Chastillon sur Marne}, Paris, 1621, Preuves, p. 223.

Bonaventura, Saint, \emph{Opera}, Paris, 1871.

Bonet, Honoré, \emph{L'arbre des batailles}, Paris, Michel le Noir,
1515.

\emph{Boucicaut, Le livre des faicts du mareschal de}, ed. Petitot,
Collection de mémoires, VI.

Bourquelot, F., \emph{Les Vaudois du quinzième siècle}, Bibliothèque de
l'Ecole des chartes, 2nd series, III, p. 109.

Burckhardt, J, \emph{Die Kultur der Renaissance in Italien}, 10th ed.,
Leipzig, 1908.

{~~~~~~~~}, \emph{Weltgeschichtliche Betrachtungen}, Berlin-Stuttgart,
1905.

Byvanck, W. G. C., \emph{Spécimen d'un essai critique sur les oeuvres de
Villon}, Leyde, 1882.

\protect\hypertarget{24_BIBLIOGRAPHY.xhtmlux5cux23page_442}{}{}{~~~~~~~~},
\emph{Un poète inconnu de la société de François Villon}, Paris, 1891.

Carnahan, D. H., \emph{The ``Ad Deum vadit'' of Jean Gerson}, University
of Illinois studies in language and literature, 1917, III, no. 1.

\emph{Caroli ducis Burgundice, De laudibus, De Morte, etc., Chroniques
relatives à l'histoire de la Belgique sous la domination des ducs de
Bourgogne}, ed. Kervyn de Lettenhove, vol. III, Brussels, 1873.

Cartellieri, O, \emph{Geschichte der Herzöge von Burgund, I Philipp der
Kuhne}, Leipzig, 1910.

{~~~~~~~~}, \emph{Beiträge zur Geschichte der Herzöge von Burgund},
Sitzungsbericht der Heidelberger Akademie der Wissenschaften, 1911, etc.

\emph{Cent ballades, Le livre des}, ed. G. Raynaud, Société des anciens
textes français, 1905.

\emph{Cent nouvelles nouvelles, Les}, ed. Th. Wright, Bibliothèque
elzévirienne, 2 vols., Paris, 1857--58.

Champion, P, \emph{Vie de Charles d'Orléans, 1394--1465}, Paris, 1911.

{~~~~~~~~}, \emph{François Villon, sa vie et son temps}, Bibliothèque du
XVe siècle, 2 vols., Paris, 1912.

\emph{Chansons françaises du quinzième siècle}, ed. G. Paris, Société
des anciens textes français, 1875.

Charney, Geoffroy de, see Piaget.

\emph{Chartier, Les oeuvres de maistre Alain}, ed. A. Du Chesne,
Tourangeau, Paris, 1617.

Chartier, Jean, \emph{Histoire de Charles VII}, ed. D. Godefroy, Paris,
1661.

\emph{Chastellain, Oeuvres de Georges}, ed. Kervyn de Lettenhove, 8
vols., Brussels, 1863--66. Especially Chronique, vols. I--V; Le miroir
des nobles hommes en France, Le dit de vérité, Exposition sur vérité mal
prise, La mort du roy Charles VII, vol. VI; L'entré du roy Loys en
nouveau règne, Advertissement au duc Charles, Le livre de la paix,
Recollection des merveilles, La temple de Bocace, Le douze Dames de
rhétorique, Le lyon rampant, Les hauts faits du duc de Bourgogne, La
mort du duc Philippe, vol. VII.

Chesne, André du, \emph{Histoire de la maison de Chastillon sur Marne},
Paris, 1621.

Chmelarz, E., \emph{König René der Gute und die Handschrift seines
Romanes ``Cuer d'amours espris'' in der K. K. Hojbibliothek}, Jahrbuch
de Kunsthist. Sammlungen des allerh Kaiserhauses, XI, Vienna, 1890.

Chopinel, Jean. See \emph{Roman de la rose}.

\emph{Chronique de Berne}, ed. H. Moranvillé, Société de l'histoire de
France, 3 vols. 1891--97.

\emph{Chronique scandaleuse}. See Roye.

Clémanges, Nicolas de, \emph{Opera}, ed. Lydius, Leyden, 1613.

Clercq, Jacques du, \emph{Mémoires (1448--1467)}, ed. de Reiffenberg, 4
vols. Brussels, 1823.

Clopinel, Jean. See \emph{Roman de la rose}.

Colette, Sainte, \emph{Acta Sanctorum Martii}, vol. I, 532--623.

Commines, Philippe de, \emph{Mémoires}, ed. B. de Mandrot, Collection de
textes pour servir à enseignement de l'histoire, 2 vols., 1901--3.

\emph{Complainte du povre commun et des povres laboureurs de France,
La}, in Monstrelet, \emph{Chronique}, vol. VI, p. 176.

\protect\hypertarget{24_BIBLIOGRAPHY.xhtmlux5cux23page_443}{}{}Coopland,
G. W., \emph{The Tree of Battles and Some of Its Sources}, Revue
d'histoire du droit, V, 173, Haarlem, 1923.

Coquillart, G., \emph{Oeuvres}, ed. Ch. d'Héricault, Bibliothèque
elzévirienne, 2 vols., 1857.

Couderc, C, \emph{Les comptes d'un grand couturier parisien au XVe
siècle}, Bulletin de la société de l'histoire de Paris, vol. XXXVIII
(1911), p. 118.

Coville, A, \emph{Les premiers Valois et la guerre de cent ans,
1328--1422}, in Lavisse, \emph{Histoire de France}, vol. IV, 1.

{~~~~~~~~}, \emph{Le véritable texte de la justification du duc de
Bourgogne par Jean Petit}, Bibliothèque de l'Ecole des chartes, 1911, p.
57.

\emph{Débat des hérauts d'armes de France et d'Angleterre, Le}, ed. L.
Pannier and P. Meyer, Société des anciens textes français, 1887.

Denifle, H, \emph{La désolation des églises, etc. en France, 2} vols,
Paris, 1897--99.

{~~~~~~~~}, and Chatelain, Aemilio, \emph{Chartularium universitatis
Parisiensis}, 4 + 2 vols., Paris, 1889--97.

Déprez, E, \emph{La Bataille de Najera, 3 avril 1367}, Revue historique,
vol. CXXXVI (1921), p. 37.

Deschamps, Eustache, Oeuvres complètes, ed. De Queux de Saint Hilaire et
G. Raynaud, Société des anciens textes français, 11 vols., 1878--1903.

Dionysius Cartusianus (or of Ryckel), \emph{Opera omnia, cura et labore
monachorum sacr. ord. Cart.}, 41 vols. Montreuil and Tournay,
1896--1913. Especially Dialogion de fide catholica, vol. 18; De
quotidiano baptismate lacrimarum, vol. 29; De munificentia et beneficiis
Dei, vol. 34; De laudibus sanctae et individuae trinitatis, de passione
domini salvatoris dialogus, vol. 35; De mutua cognitione, De modo agendi
processiones, Contra vitia superstitionum quibus circa cultum veri Dei
erratur, vol. 36; De vita et regimine episcoporum, nobilium, etc., etc.,
vol. 37ff.; Inter Jesum et puerum dialogus, vol. 38; Directorium vitae
nobilium, vol. 37; De vitiis et virtutibus, vol. 39; De contemplatione,
De quattuor hominum novissimis, vol. 41.

Dixmude, Jan van, \emph{Chronike}, ed. J. J. Lambin, Ypres, 1839.

Douet, d'Arcq, \emph{Choix de pièces inédites relatives au règne de
Charles VI}, Société de l'histoire de France, 2 vols., 1863.

Doutrepont, G., \emph{La littérature française à la cour des ducs de
Bourgogne}, Bibliothèque du XVe siècle, Paris, 1909.

Durand-Gréville, E., \emph{Hubert et Jean Van Eyck}, Bruxelles, 1910.

Durrieu, P., \emph{Les très-riches heures de Jean de France, duc de
Berry}, Paris, 1904.

{~~~~~~~~}, \emph{Les belles heures du duc de Berry}, Gazette des beaux
arts, 1906, vol. XXXV, p. 283.

{~~~~~~~~}, \emph{Un barbier de nom français à Bruges}, Comptes rendus
de l'Académie des inscriptions et belles--lettres, 1917, p. 542.

{~~~~~~~~}, \emph{La miniature flamande au temps de la cour de
Bourgogne} (1450--1530), Brussels, 1921.

Eckhart, Meister, \emph{Predigten}, ed. F. Pfeiffer, in \emph{Deutsche
Mystiker des XIV Jahrhunderts, 2} vols., Leipzig, 1857.

\emph{Elisabeth, Saint of Hungary, Report on an Autopsy of the body of,
by bishop Konrad of Hildesheim and abbot Hermann of Georgenthal},
Historisches Jahrbuch der Görresgesellschaft, vol. XXVIII, p. 887.

\protect\hypertarget{24_BIBLIOGRAPHY.xhtmlux5cux23page_444}{}{}Erasmus,
Desiderius, \emph{Opera omnia}, ed. J. Clericus, 10 vols., Leyden,
1703--6.

{~~~~~~~~}, \emph{Ratio seu methodus compendio perveniendi ad veram
theologiam}, ed. Basileae, 1520.

{~~~~~~~~}, \emph{Opus epistolarum} .~.~. \emph{denuo recognitum et
auctum}, P. S. a n d H . M . A l l e n, 5 vols., Oxford 1906--24
(--1524).

{~~~~~~~~}, \emph{Colloquia}, ed. Elzevier, 1636.

Escouchy, Mathieu d', \emph{Chronique}, ed. G. du Fresne de Beaucourt,
Société de l'histoire de France, 3 vols., 1863--64.

Estienne, Henri, \emph{Apologie pour Hérodote}, ed. Ristelhuber, 2
vols., 1879.

Facius, Bartolomaeus, \emph{De Viris illustribus liber}, ed. L. Mehus,
Florence, 1745.

Fenin, Pierre de, \emph{Mémoires}, Petitot, Collection de mémoires, VII.

Ferrer, see Vincent.

Fierens, Gevaert, \emph{La renaissance septentrionale et les premiers
maîtres des Flandres}, Brussels, 1905.

Fillastre, Guillaume, \emph{Le premier et le second volume de la toison
d'or}, Paris, Franc. Regnault, 1515--16.

François de Paule, Saint, \emph{Acta sanctorum Aprilis}, vol. I, pp.
103--234.

Fredericq, P., \emph{Codex documentorum sacratissimarum Indulgentiarum
Neerlandicarum}, Rijks geschiedkundige Publicatiën (small series), no.
21, The Hague, 1922.

Fresne de Beaucourt, G. du, \emph{Histoire de Charles VII, 6} vols.,
Paris, 1881--91.

Froissart, Jean, \emph{Chroniques}, ed. S. Luce et G. Raynaud, Société
de l'histoire de France, 11 vols., 1869--99 (--1385).

{~~~~~~~~}, \emph{Chroniques}, ed. Kervyn de Lettenhove, 29 vols.,
Brussels, 1867--77.

{~~~~~~~~}, \emph{Poésies}, ed. A. Scheler, Académie royale de Belgique,
3 vols., 1870--72.

{~~~~~~~~}, \emph{Meliador}, ed. A Longnon, Société des anciens textes
français, 3 vols., 1895--99.

Gaguin, Robert, \emph{Epistolae et orationes}, ed. L. Thuasne,
Bibliothèque littéraire de la Renaissance, 2 vols., Paris, 1903.

{~~~~~~~~}, \emph{Compendium super Francorum gestis}, Paris, 1500.

\emph{Gartia Dei, Oratio Antonii}, ed. Kervyn de Lettenhove,
\emph{Chron. rel. à l'hist. de la Belgique sous la dom. des ducs de
Bourgogne}, vol. III.

Geoffroi, de Paris, \emph{Chronique}, ed. De Wailly et Delisle, Bouquet,
Recueil des historiens, vol. XXII.

Germain, Jean, \emph{Liber de virtutibus Philippi ducis Burgundiae}, ed.
Kevyn de Lettenhove, \emph{Chron. rel. à l'hist. de la Belgique sous la
dom. des ducs de Bourgogne}, vol. II.

Gerson, Jean, \emph{Opera omnia}, ed. L. Ellies du Pin, 2nd ed., Hagae
Comitis, 1728, 5 vols. Especially vol. I, De examinatione doctrinarum,
De probatione spirituum, De distinctione vera visionum a falsis,
Epistola super librum Joh. Ruysbroeck, etc., Ep. contra libellum Joh. de
Schonhavia, id. contra defensionem Joh. de Schonhavia, Contra vanam
curiositatem, De libris caute legendis, De consolatione theologiae,
Contra superstitionem praesertim Innocentum, De erroribus circa artem
magicam, Compendium theologiae, De decern praeceptis, De praeceptis
decaolgi, De susceptione humanitatis Christi, De falsis prophetis; vol.
II, De nuptis Christi et ecclesiae, Expostulatio adv. eos qui publice
volunt dogmatizare, etc., Contra impugnantes ordinem Carthusiensium;
vol. III, Liber de vita spirituali animae, Regulae morales, De
passionibus animae,
Centi\protect\hypertarget{24_BIBLIOGRAPHY.xhtmlux5cux23page_445}{}{}logium
de impulsibus, Contra foedam tentationem blasphemiae, de parvulus ad
Christum trahendis, Expostulatio adversus corruptionem juventutis per
lascivas imagines, Discours de l'excellence de virginité, Oratio ad
bonum angelum suum, De monte contemplationis, De vita imitativa,
Considérations sur Saint Joseph, De triplici theologia, Considérations
sur le péché de blasphème, Contra gulam sermo, Sermo contra luxuriem,
Sermo de nativitate Domini, Sermo de natalitate b. Mariae Virginis,
Sermones in die S. Ludovici, Sermo de Angelis, Sermones de defunctis,
Sermo de S. Nicolao; vol. IV, Meditatio super VII\textsuperscript{mo}
psalmo poenitentiali, Tractatus super Magnificat, Querela nomine
Universitatis, Sermo coram rege Franciae, Oratio ad regem Franciae,
Josephina.

Godefroy, Th., \emph{Le cérémonial françois, 2} vols., Paris, 1649.

\emph{Grandes chroniques de France, Les}, ed. Paulin Paris, 6 vols.,
Paris, 1836--38

Hanotaux, \emph{G., Jeanne d'Arc}, Paris, 1911.

Hefele, K., \emph{Der heilige Bernhardin von Siena und die
franziskanische Wanderpredigt in Italien}, Freiburg, 1912.

Hintzen, J. D., \emph{De kruistochtplannen van Philips den Goede},
Rotterdam, 1918.

\emph{Histoire littéraire de la France, XlVe siècle}, vol. XXIV, 1862.

Hoepffner, E., \emph{Frage- und Antwortspiele in der französischen
Literatur des 14 Jahrhunderts}, Zeitschrift für romanische Philologie,
vol. XXXIII, 1909.

Hospinianus, R., \emph{De templis, hoc est de origine, progressu, usu et
abusu templorum, etc.}, 2nd ed., Zürich, 1603.

Houwaert, J. B., \emph{Declaratie van die triumphante incompst van den
Prince van Oraingnien, etc.}, Antwerp, Plantijn, 1579.

Huet, G., \emph{Notes d'histoire littéraire III}, in Le Moyen Age, vol.
XX, 1918.

Huizinga, J., \emph{Uit de voorgeschiedenis van ons nationaal besef}, De
Gids, 1912, vol. III. \emph{{~~~~~~~~}}, \emph{Renaissancestudiën I: Het
probleem}, De Gids, 1920, vol. IV.

James, W., \emph{The Varieties of Religious Experience}, London, 1903.

Jorga, N., \emph{Philippe de Mézières et la croisade au XlVe siècle},
Bibliothèque de l'Ecole des hautes études, Fasc. CX, 1896.

Jouffroy, Jean, \emph{De Philippo duce oratio}, ed. Kervyn de
Lettenhove, \emph{Chron. rel. à l'hist. de la Belgique sous la dom. des
ducs de Bourgogne}, vol. III.

\emph{Journal d'un bourgeois de Paris, 1405--1449}, ed. A. Tuetey,
publications de la Société de l'histoire de Paris, doc. no. III, 1881.

\emph{Jouvencel, Le}, ed. C. Favre et L. Lecestre, Société de l'histoire
de France, 2 vols., 1887--92.

Juvenal des Ursins, Jean, \emph{Chronique}, ed Michaud et Poujoulat,
Nouvelle collection des mémoires, II.

Kempis, Thomas à, \emph{Opera omnia}, ed. M. J. Pohl, 7 vols., Freiburg,
1902--10.

Kleinclauz, A., \emph{Histoire de Bourgogne}, Paris, 1909.

\emph{{~~~~~~~~}}, \emph{L'art funéraire de la Bourgogne à moyen âge},
Gazette des beaux arts, vol. XXVII, 1902.

\emph{{~~~~~~~~}}, \emph{Un atelier de sculpture au XVe siécle}, Gazette
des beaux arts, vol. XXIX, 1903.

Krogh-Tonning, K., \emph{Der letzte scholastiker, Eine Apologie},
Freiburg, 1904.

Kurth, Betty, \emph{Die Blütezeit der Bildwirkerkunst zu Tournay und der
burgundische Hof}, Jahrbuch der Kunstsammlungen des Kaiserhauses, XXXIV,
1917.

\protect\hypertarget{24_BIBLIOGRAPHY.xhtmlux5cux23page_446}{}{}Laborde,
L. de, \emph{Les ducs de Bourgogne, Etudes sur les lettres, les arts et
l'industrie pendant le XVe siècle}, 3 vols., Paris, 1849--53.

La Curne de Sainte Palaye, J. B., \emph{Mémoires sur l'ancienne
chevalerie}, 1781.

\emph{Lalaing, Le livre des faits du bon chevalier messire Jacques de},
ed. Kervyn de Lettenhove, \emph{Oeuvres de Chastellain}, vol. VIII.

La Marche, Olivier de, \emph{Mémoires}, ed. Beaune et d'Arbaumont,
Société de l'histoire de France, 4 vols., 1883--88.

\emph{{~~~~~~~~}}, \emph{Estat de la maison de duc Charles de
Bourgogne}, ibid., vol. IV.

\emph{{~~~~~~~~}}, \emph{Rationarium aulae et imperii Caroli Audacis
ducis Burgundiae}, ed. A. Matthaeus Analecta, I, pp. 357--494 (Middle
Dutch translation of the preceding work).

\emph{{~~~~~~~~}}, \emph{Le parement et triumphe des dames}, Paris,
Michel le Noir, 1520.

Langlois, E., \emph{Anciens proverbes français}, Bibliothèque de l'Ecole
des chartes, vol. LX (1899), p. 569.

\emph{{~~~~~~~~}}, \emph{Recueil d'arts de seconde rhétorique},
Documents inédites sur l'histoire de France, Paris, 1902.

Lannoy, Ghillebert de, Oeuvres, ed. Ch. Potvin, Louvain, 1878.

La Roche, see Alain.

La Salle, Antoine de la, \emph{La Salade}, Paris, Michel le Noir, 1521.

\emph{{~~~~~~~~}}, \emph{L'histoire et plaisante cronicque de Jehan de
Saintree}, ed. G. Helleny, Paris, 1890.

\emph{{~~~~~~~~}}, \emph{Le reconfort de Madame du Fresne}, ed. J. Nève,
Paris, 1903.

\emph{La Tour Landry, Le livre du chevalier de}, ed. A. de Montaiglon,
Bibliothèque elzévirienne, Paris, 1854.

Lefèvre de Saint Remy, Jean, \emph{Chronique}, ed. F. Morand, Société de
l'histoire de France, 2 vols., 1876.

Leroux, de Lincy, A., \emph{Le livre des proverbes français}, 2nd ed., 2
vols., Paris, 1859.

\emph{Liber Karoleidos}, ed. Kervyn de Lettenhove, \emph{Chron. rel. à
l'hist. de la Belgique sous la dom. des ducs de Bourgogne,}, vol. III.

\emph{Livre des trahisons, Le}, ed. id., ibid., vol. II.

Loër, Theodericus, \emph{Vita Dionysii Cartusiani}, in Dionysii, Opera,
I, p. xlii.

Lorris, Guillaume de, see \emph{Roman de la rose}.

\emph{Louis XI, lettres de}, ed. Vaesen, Charavay, de Mandrot, Société
de l'histoire de France, 11 vols., 1883--1909.

Luce, S., \emph{La France pendant la guerre de cent ans}, Paris, 1890.

Luther, Martin, \emph{De captivate babylonica ecclesiae praeludium},
Werke, Weimar edition, vol. VI.

Luxembourg, see Pierre.

Machaut, Guillaume de, \emph{Le livre du voir-dit}, ed. Paulin Paris,
Société des bibliophiles françois, 1875.

\emph{{~~~~~~~~}}, \emph{Oeuvres}, ed E. Hoepffner, Société des anciens
textes français, 2 vols., 1908--11.

\emph{{~~~~~~~~}}, \emph{Poésies lyriques}, ed. V. Chichmaref, Zapiski
istoritcheski fil. fakulteta imp. S. Peterb. univers., vol. XCII, 1909.

Magnien, Ch., \emph{Caxton à la cour de Charles le Téméraire}, Annuaire
de la société d'archéologie de Bruxelles, vol. XXIII, 1912.

Maillard, Olivier, \emph{Sermones dominicales, etc.}, Paris, Jean Petit,
1515.

Mâle, E., \emph{L'art religieux du treizième siècle en France}, Paris,
1902.

\emph{\protect\hypertarget{24_BIBLIOGRAPHY.xhtmlux5cux23page_447}{}{}{~~~~~~~~}},
\emph{L'art religieux à la fin du moyen-âge en France}, Paris, 1908.

Mangeart, J., \emph{Catalog des manuscrits de la bibliothèque de
Valenciennes}, 1860.

Martial (d'Auvergne), \emph{Les poésies de Martial de Paris dit
d'Auvergne}, 2 vols, Paris, 1724. See \emph{Amant rendu} .~.~.

\emph{Meschinot, Jean, sa vie et ses oeuvres}, par A. de la Borderie,
Bibliothèque de l'Ecole des chartes, vol. LVI, 1895.

Meyer, P., \emph{Les neuf preux}, Bulletin de la société des anciens
textes français, 1883, p. 45.

Michault, Pierre, \emph{La dance aux aveugles et autres poésies du XVe
siécle}, Lille, 1748.

Michel, André, \emph{Histoire de l'art}, vols. III and IV, Paris, 1907,
etc.

Molinet, Jean, \emph{Chronique}, ed. J. Buchon, Collection de chroniques
nationales, 5 vols., 1827--28.

\emph{{~~~~~~~~}}, \emph{Les faicts et dictz de messire Jehan}, Paris,
Jehan Petit, 1537.

Molinier, A., \emph{Les sources de l'histoire de France, des origines
aux guerres d'Italie} (1494), 6 vols., Paris, 1901--6.

Moll, W. \emph{Kerkgeschiedinis van Nederland vóór de hervorming}, 5
parts, Utrecht, 1864--69.

\emph{{~~~~~~~~}}, \emph{Johannes Brugman en het godsdienstig leven
onzer vaderen in de vijfiien eeuw, 2} vols., Amsterdam, 1854.

Monstrelet, Enguerrand de, \emph{Chroniques}, ed. Douet d'Arcq, Société
de l'histoire de France, 6 vols., 1857--62.

Montreuil, Jean de, \emph{Epistolae}, ed. Martène et Durand, Amplissima
collectio, II col., 1398.

Mougel, D. A., \emph{Denys le Chartreux, sa vie, etc.}, Montreuil, 1896.

Nys, E. \emph{Le droit de guerre et les précurseurs de Grotius},
Brussels and Leipzig, 1882.

\emph{{~~~~~~~~}}, \emph{Etudes de droit international et de droit
politique}, Brussels and Paris, 1896.

\emph{Ordonnances des rois de France}, Paris, 1723--77.

Orléans, Charles d', \emph{Poésies complètes, 2} vols., Paris, 1874.

Oulmont, Ch., \emph{Le verger, le temple et la cellule, Essai sur la
sensualité dans les oeuvres de mystique religieuse}, Paris, 1912.

Pannier, L., \emph{Les joyaux du duc de Guyenne, recherches sur les
goûts artistiques et la vie privée du dauphin Louis}, Revue
archéologique, 1873.

\emph{Pastoralet, Le}, ed. Kervyn de Lettenhove, \emph{Chron. rel. à
l'hist. de la Belgique sous la dom. des ducs de Bourgogne}, vol. II.

Pauli, Theodericus, \emph{De rebus actis sub ducibus Burgundiae
compendium}, ed. id., ibid., vol. III.

Petit Dutaillis, Ch., \emph{Charles VII, Louis XI et les premières
années de Charles VIII (1422--1492)}, in Lavisse, \emph{Histoire de
France}, vol. IV, part 2.

\emph{{~~~~~~~~}}, \emph{Documents nouveaux sur les moeurs populaires et
le droit de vengeance dans les Pays-bas au XVe siècle}, Bibliothèque de
XVe siècle, Paris, 1908.

Petrarca, Francesco, \emph{Opera}, Basle edition, 1581.

Piaget, A., \emph{Oton de Granson et ses poésies}, Romania, vol. XIX,
1890.

\emph{{~~~~~~~~}}, \emph{Chronologie des épistres sur le Roman de la
rose}, Etudes romanes dédiées à Gaston Paris, 1891, p. 113.

\emph{{~~~~~~~~}}, \emph{La cour amoureuse dite de Charles VI}, Romania,
vol. XX, 1891; XXI, 1892.

\emph{{~~~~~~~~}}, \emph{Le livre messire Geoffroy de Charney}, Romania,
vol. XXVI, 1897.

\emph{{~~~~~~~~}}, \emph{Le chapel des fleurs de lis, par Philippe de
Vitri}, Romania, vol. XXVII, 1898.

\protect\hypertarget{24_BIBLIOGRAPHY.xhtmlux5cux23page_448}{}{}Pierre de
Luxembourg, the Blessed, \emph{Acta sanctorum Julii}, vol. I, pp.
509--628.

Pierre Thomas, Carmelite Saint, \emph{Acta sanctorum Januarii}, vol. II
(his life by Philippe de Mézières).

Pirenne, H., \emph{Histoire de Belgique}, 5 vols., Brussels, 1902--21.

Pisan, Christine de, Oeuvres poétiques, ed. M. Roy, Société des anciens
textes français, 3 vols., 1886--96.

{~~~~~~~~}, \emph{Epitre d'Othéa à Hector}, Manuscrit 9392, de Jean
Miélot, ed. J. van den Gheyn, Brussels, 1913.

\emph{Poésies françoises des XVe et XVIe siècles, Recueil de}, ed. A. de
Montaiglon, Bibliothèque elzévirienne, Paris, 1856.

Polydorus Vergilius, \emph{Anglicae historiae libri XXVI}, Basle, 1546.

Pool, J. C, \emph{Frederik van Heilo en zijne schrifien}, Amsterdam,
1866.

Portiers, Alienor de, \emph{Les honneurs de la cour}, ed. La Curne de
Sainte Palaye, \emph{Mémoires sur l'ancienne chevalerie}, 1781, II.

\emph{Quinze joyes de mariage, Les}, Paris, Marpon et Flammarion, no
date.

Ramsay, J. H., \emph{Lancaster and York, 1399--1485, 2} vols., Oxford,
1892.

Raynaldus, \emph{Annales ecclesiastici}, vol. III (= Baronius, vol.
XXII).

Raynaud, G., \emph{Rondeaux, etc., du XVe siècle}, Société des anciens
textes français, 1889.

\emph{Religieux de Saint Denis, Chronique du}, ed. Bellaguet, Collection
des documents inédits, 6 vols., 1839--52.

Renaudet, A., \emph{Préréforme et humanisme à Paris, 1494--1517}, Paris,
1916.

\emph{René, Oeuvres du roi}, ed. Quatrebarbes, 4 vols., Angers, 1845.

\emph{Roman de la rose, Le}, ed. M. Méon, 4 vols., Paris, 1814.

{~~~~~~~~}, ed. F. Michel, 2 vols., Paris, 1864.

{~~~~~~~~}, ed. E. Langlois, Société des anciens textes français, 1914,
I.

Rousselot, P., \emph{Pour l'histoire du problème de l'amour}, Beiträge
zur Geschichte der Philosophie im Mittelalter, ed. Bäumker and von
Hertling, vol. VI, 1908.

Roye, Jean de, \emph{Journal dite Chronique scandaleuse}, ed. B. de
Mandrot, Société de l'histoire de France, 2 vols., 1894--96.

Rozmital, Leo von, \emph{Reise durch die Abendlände, 1465--1467}, ed.
Schmeller, Biblio--thek des literarischen Verins zu Stuttgart, vol. VII,
1844.

Ruelens, Ch., \emph{Recueil de chansons, poèmes, etc. relatifs aux
Pays-Bas}, 1878.

Ruusbroec, Johannes, \emph{Werken}, ed. David and Snellaert,
Maetschappij der Vlaemsche bibliophilen, 1860--68. Especially II, Die
chierheit de gheesteleker brulocht, Spieghel de ewigher salicheit; IV,
Van seven trappen in den graet der gheestelicker minnen, Boec van der
hoechster waerheit, Dat boec van seven sloten, Dat boec van den rike der
ghelieven.

\emph{Ruysbroeck l'Admirable, Oeuvres de}, Translation from the Flemish
by the Bénédictines de Saint Paul de Wisques, vols. I--III, Brussels and
Paris, 1917-20.

Salmon, Pierre le Fruictier dit, \emph{Mémoires}, ed. Buchon, Collection
de chroniques nationales 3e supplément de Froissart, vol. XV.

Schäfer, D. \emph{Mittlealterliche Brauch bei der Ueberführung von
Leichen}, Sitzungsberichte der preussichen Akademie der Wissenschaften,
1920, p. 478.

Schmidt, C., \emph{Der Prediger Olivier Maillard}, Zeitschrift für
historische Theologie, 1856.

Seuse, Heinrich (Suso), \emph{Deutsche Schrifien}, ed. K. Bihlmeyer,
Stuttgart, 1907.

\protect\hypertarget{24_BIBLIOGRAPHY.xhtmlux5cux23page_449}{}{}Sicard,
\emph{Mitrale sive de officiis ecclesiasticis summa}, Migne, Patr. lat.,
vol. CCXIII.

Stavelot, Jean de, \emph{Chronique}, ed. Borgnet, Collection des
chroniques belges, Brussels, 1861.

Stein, H., \emph{Etude sur Olivier de la Marche}, Mémoires couronnés de
l'Academie royale de Belgique, vol. XLIX, 1888.

Tauler, Johannes, \emph{Predigten}, in Vetter, Deutsche Texte des
Mittelalters, vol. XI, Berlin, 1910.

Thomas Aquinas, Saint, \emph{Historia transltionis corporis sanctissimi
ecclesioe doctoris divi Th. de Aq. 1368}, auct. fr. Raymundo Hugonis O.
P., Acta sanctorum Martii, vol. I, p. 725.

Thomas, see Pierre.

\emph{Trahisons}, see \emph{Livre des}.

Upton, Nicolas, \emph{De officio militari}, ed. E. Bysshe, London, 1654.

Valois, Noël, \emph{La France et le grand schisme d'occident}, 4 vols.,
Paris, 1896--1902.

Varennes, Jean de, \emph{Responsiones ad capitula accusationum, etc.},
in Gerson, Opera, I, pp. 906-43.

Vigneulles, Philippe de, \emph{Mémoires}, ed. H. Michelant, Bibliothek
des lit. Verins zu Stuttgart, vol. XXIV, 1852.

Villon, François, \emph{Oeuvres}, ed. A. Longnon, Les classiques
français du moyen âge, vol. II, Paris, 1914.

Vincent Ferrer, Saint, \emph{Vita}, auct. Petro Ranzano O. P., 1455,
Acta sanctorum Aprilis, vol. I, pp. 82--512.

{~~~~~~~~}, \emph{Sermones quadragesimales}, Cologne, 1482.

Vitri, Philippe de, Le \emph{chapel des fleurs delis}, ed. A. Piaget,
Romania, vol. XXVII, 1898.

\emph{Voeux du héron, Les}, ed. Société des bibliophiles de Mons, no. 8,
1839.

Walsingham, Thomas, \emph{Historia Anglicana}, ed. H. T. Riley, in
\emph{Rer brit. medii aevi scriptores} (Rolls series), 3 vols., London,
1864.

Weale, W. H. J., \emph{Hubert and John van Eyck, their Life and Work},
London and New York, 1908.

Wielant, Philippe, \emph{Antiquites de Flandre}, ed. De Smet, Corpus
chronicorum Flandriae, vol. IV.

Wright, Th., \emph{The Anglo-Latin Satirical Poets and Epigrammatists of
the Twelfth Century}, in \emph{Rerum britannicarum medii aevi
scriptores} (Rolls series), 2 vols., London, 1872.

Zöckler, O., \emph{Dionys des Kartäusers Schrift De venustate mundi,
Beitrag zur vorgeschichte der Asthetick}, Theologische Studien und
Kritiken, 1881.

% \chapter{INDEX}

\emph{Abuzé en court, L'} (Charles Rochefort),
\protect\hyperlink{10_Chapter_Three__THE_HEROIC_DREAM.xhtmlux5cux23page_124}{124},
\protect\hyperlink{16_Chapter_Nine__THE_DECLINE_OF_SYM.xhtmlux5cux23page_245}{245}

Achatius, Saint,
\protect\hyperlink{13_Chapter_Six__THE_DEPICTION_OF_TH.xhtmlux5cux23page_198}{198}

Adam and Eve,
\protect\hyperlink{21_Chapter_Thirteen__IMAGE_AND_WORD.xhtmlux5cux23page_337}{337},
\protect\hyperlink{21_Chapter_Thirteen__IMAGE_AND_WORD.xhtmlux5cux23page_373}{373}

Adolf, Saint,
\protect\hyperlink{13_Chapter_Six__THE_DEPICTION_OF_TH.xhtmlux5cux23page_184}{184}

\emph{Adoration of the Lamb. See} Ghent Altarpiece

Adrian, Saint,
\protect\hyperlink{13_Chapter_Six__THE_DEPICTION_OF_TH.xhtmlux5cux23page_198}{198}

Agincourt, battle of: Boucicaut captured,
\protect\hyperlink{10_Chapter_Three__THE_HEROIC_DREAM.xhtmlux5cux23page_78}{78};
dead boiled,
\protect\hyperlink{12_Chapter_Five__THE_VISION_OF_DEAT.xhtmlux5cux23page_164}{164};
Henry V at,
\protect\hyperlink{10_Chapter_Three__THE_HEROIC_DREAM.xhtmlux5cux23page_111}{111},
\protect\hyperlink{10_Chapter_Three__THE_HEROIC_DREAM.xhtmlux5cux23page_114}{114};
in \emph{Le livre des quatre dames},
\protect\hyperlink{21_Chapter_Thirteen__IMAGE_AND_WORD.xhtmlux5cux23page_339}{339};
in ``Le Pastoralet,''
\protect\hyperlink{11_Chapter_Four__THE_FORMS_OF_LOVE.xhtmlux5cux23page_152}{152};
named,
\protect\hyperlink{10_Chapter_Three__THE_HEROIC_DREAM.xhtmlux5cux23page_114}{114};
pride at,
\protect\hyperlink{10_Chapter_Three__THE_HEROIC_DREAM.xhtmlux5cux23page_113}{113--}\protect\hyperlink{10_Chapter_Three__THE_HEROIC_DREAM.xhtmlux5cux23page_114}{114}

Agricola, Rudolph,
\protect\hyperlink{13_Chapter_Six__THE_DEPICTION_OF_TH.xhtmlux5cux23page_183}{183}

Ailly, Pierre d': against blasphemy,
\protect\hyperlink{13_Chapter_Six__THE_DEPICTION_OF_TH.xhtmlux5cux23page_187}{187};
beggars,
\protect\hyperlink{14_Chapter_Seven__THE_PIOUS_PERSONA.xhtmlux5cux23page_205}{205};
disapproval of courtly life,
\protect\hyperlink{10_Chapter_Three__THE_HEROIC_DREAM.xhtmlux5cux23page_124}{124};
music,
\protect\hyperlink{20_ILLUSTRATIONS_FOLLOW_PAGE.xhtmlux5cux23page_323}{323};
opposes elaborate ritual, etc.,
\protect\hyperlink{13_Chapter_Six__THE_DEPICTION_OF_TH.xhtmlux5cux23page_175}{175};
relic of St. Louis and,
\protect\hyperlink{13_Chapter_Six__THE_DEPICTION_OF_TH.xhtmlux5cux23page_192}{192};
scholastic argument,
\protect\hyperlink{17_Chapter_Ten__THE_FAILURE_OF_IMAG.xhtmlux5cux23page_249}{249}

Alain. \emph{See} La Roche

Alexander the Great,
\protect\hyperlink{10_Chapter_Three__THE_HEROIC_DREAM.xhtmlux5cux23page_75}{75}

allegory: church forms of,
\protect\hyperlink{11_Chapter_Four__THE_FORMS_OF_LOVE.xhtmlux5cux23page_130}{130--}\protect\hyperlink{11_Chapter_Four__THE_FORMS_OF_LOVE.xhtmlux5cux23page_131}{131};
emotional value,
\protect\hyperlink{16_Chapter_Nine__THE_DECLINE_OF_SYM.xhtmlux5cux23page_243}{243--}\protect\hyperlink{16_Chapter_Nine__THE_DECLINE_OF_SYM.xhtmlux5cux23page_244}{244};
eroticism,
\protect\hyperlink{11_Chapter_Four__THE_FORMS_OF_LOVE.xhtmlux5cux23page_132}{132};
explanation,
\protect\hyperlink{16_Chapter_Nine__THE_DECLINE_OF_SYM.xhtmlux5cux23page_238}{238};
fatigue of,
\protect\hyperlink{22_Chapter_Fourteen__THE_COMING_OF.xhtmlux5cux23page_382}{382};
Meschinot,
\protect\hyperlink{21_Chapter_Thirteen__IMAGE_AND_WORD.xhtmlux5cux23page_381}{381};
pastorale and,
\protect\hyperlink{11_Chapter_Four__THE_FORMS_OF_LOVE.xhtmlux5cux23page_150}{150}.
See also \emph{Roman de la rose}; symbolism

altarpieces: Autun (Jan van Eyck),
\protect\hyperlink{20_ILLUSTRATIONS_FOLLOW_PAGE.xhtmlux5cux23page_306}{306},
\protect\hyperlink{20_ILLUSTRATIONS_FOLLOW_PAGE.xhtmlux5cux23page_317}{317};
Ghent (van Eyck Brothers
\protect\hyperlink{19_Chapter_Twelve__ART_IN_LIFE.xhtmlux5cux23page_297}{297},
\protect\hyperlink{21_Chapter_Thirteen__IMAGE_AND_WORD.xhtmlux5cux23page_342}{342},
\protect\hyperlink{21_Chapter_Thirteen__IMAGE_AND_WORD.xhtmlux5cux23page_339}{339};
Beaune (van der Weyden),
\protect\hyperlink{20_ILLUSTRATIONS_FOLLOW_PAGE.xhtmlux5cux23page_306}{306},
\protect\hyperlink{20_ILLUSTRATIONS_FOLLOW_PAGE.xhtmlux5cux23page_317}{317};
Bladelyn (van der Weyden),
\protect\hyperlink{20_ILLUSTRATIONS_FOLLOW_PAGE.xhtmlux5cux23page_316}{316};
\emph{Seven Sacraments} (van der Weyden),
\protect\hyperlink{20_ILLUSTRATIONS_FOLLOW_PAGE.xhtmlux5cux23page_306}{306--}\protect\hyperlink{20_ILLUSTRATIONS_FOLLOW_PAGE.xhtmlux5cux23page_307}{307}.
\emph{See also} individual works

Amadis,
\protect\hyperlink{10_Chapter_Three__THE_HEROIC_DREAM.xhtmlux5cux23page_84}{84}

\emph{Amant rendu cordelier de l'observance d'amour},
\protect\hyperlink{11_Chapter_Four__THE_FORMS_OF_LOVE.xhtmlux5cux23page_132}{132},
\protect\hyperlink{21_Chapter_Thirteen__IMAGE_AND_WORD.xhtmlux5cux23page_370}{370},
\protect\hyperlink{22_Chapter_Fourteen__THE_COMING_OF.xhtmlux5cux23page_389}{389}

Ambapali,
\protect\hyperlink{12_Chapter_Five__THE_VISION_OF_DEAT.xhtmlux5cux23page_163}{163}

Andrew, Saint,
\protect\hyperlink{08_Chapter_One__THE_PASSIONATE_INTE.xhtmlux5cux23page_18}{18},
\protect\hyperlink{08_Chapter_One__THE_PASSIONATE_INTE.xhtmlux5cux23page_19}{19},
\protect\hyperlink{08_Chapter_One__THE_PASSIONATE_INTE.xhtmlux5cux23page_24}{24},
\protect\hyperlink{10_Chapter_Three__THE_HEROIC_DREAM.xhtmlux5cux23page_108}{108}

Andromeda,
\protect\hyperlink{21_Chapter_Thirteen__IMAGE_AND_WORD.xhtmlux5cux23page_375}{375}

angels: \emph{Ars morendi},
\protect\hyperlink{12_Chapter_Five__THE_VISION_OF_DEAT.xhtmlux5cux23page_167}{167};
chivalry and,
\protect\hyperlink{10_Chapter_Three__THE_HEROIC_DREAM.xhtmlux5cux23page_70}{70--}\protect\hyperlink{10_Chapter_Three__THE_HEROIC_DREAM.xhtmlux5cux23page_71}{71};
guardian,
\protect\hyperlink{13_Chapter_Six__THE_DEPICTION_OF_TH.xhtmlux5cux23page_181}{181},
\protect\hyperlink{13_Chapter_Six__THE_DEPICTION_OF_TH.xhtmlux5cux23page_201}{201--}\protect\hyperlink{13_Chapter_Six__THE_DEPICTION_OF_TH.xhtmlux5cux23page_202}{202};
saints and,
\protect\hyperlink{13_Chapter_Six__THE_DEPICTION_OF_TH.xhtmlux5cux23page_192}{192--}\protect\hyperlink{13_Chapter_Six__THE_DEPICTION_OF_TH.xhtmlux5cux23page_193}{193},
\protect\hyperlink{13_Chapter_Six__THE_DEPICTION_OF_TH.xhtmlux5cux23page_196}{196--}\protect\hyperlink{13_Chapter_Six__THE_DEPICTION_OF_TH.xhtmlux5cux23page_197}{197},
\protect\hyperlink{13_Chapter_Six__THE_DEPICTION_OF_TH.xhtmlux5cux23page_201}{201--}\protect\hyperlink{13_Chapter_Six__THE_DEPICTION_OF_TH.xhtmlux5cux23page_202}{202};
Sluter's,
\protect\hyperlink{20_ILLUSTRATIONS_FOLLOW_PAGE.xhtmlux5cux23page_309}{309};
Van Eyck,
\protect\hyperlink{20_ILLUSTRATIONS_FOLLOW_PAGE.xhtmlux5cux23page_317}{317--}\protect\hyperlink{20_ILLUSTRATIONS_FOLLOW_PAGE.xhtmlux5cux23page_318}{318},
\protect\hyperlink{21_Chapter_Thirteen__IMAGE_AND_WORD.xhtmlux5cux23page_335}{335},
\protect\hyperlink{21_Chapter_Thirteen__IMAGE_AND_WORD.xhtmlux5cux23page_337}{337},
\protect\hyperlink{21_Chapter_Thirteen__IMAGE_AND_WORD.xhtmlux5cux23page_342}{342}

Anjou, Louis of,
\protect\hyperlink{14_Chapter_Seven__THE_PIOUS_PERSONA.xhtmlux5cux23page_212}{212}

Annouciade, Order of the,
\protect\hyperlink{10_Chapter_Three__THE_HEROIC_DREAM.xhtmlux5cux23page_94}{94}

\emph{Annunciation} (Jan van Eyck),
\protect\hyperlink{21_Chapter_Thirteen__IMAGE_AND_WORD.xhtmlux5cux23page_335}{335--}\protect\hyperlink{21_Chapter_Thirteen__IMAGE_AND_WORD.xhtmlux5cux23page_336}{336}

Anthony, Saint,
\protect\hyperlink{13_Chapter_Six__THE_DEPICTION_OF_TH.xhtmlux5cux23page_198}{198},
\protect\hyperlink{13_Chapter_Six__THE_DEPICTION_OF_TH.xhtmlux5cux23page_200}{200}

Antonius, Order of,
\protect\hyperlink{10_Chapter_Three__THE_HEROIC_DREAM.xhtmlux5cux23page_94}{94--}\protect\hyperlink{10_Chapter_Three__THE_HEROIC_DREAM.xhtmlux5cux23page_95}{95}

\emph{Arbre des batailles L'} (Honoré Bonet),
\protect\hyperlink{18_Chapter_Eleven__THE_FORMS_OF_THO.xhtmlux5cux23page_277}{277--}\protect\hyperlink{18_Chapter_Eleven__THE_FORMS_OF_THO.xhtmlux5cux23page_278}{278}

Arc, Jeanne d',
\protect\hyperlink{08_Chapter_One__THE_PASSIONATE_INTE.xhtmlux5cux23page_4}{4},
\protect\hyperlink{10_Chapter_Three__THE_HEROIC_DREAM.xhtmlux5cux23page_77}{77},
\protect\hyperlink{10_Chapter_Three__THE_HEROIC_DREAM.xhtmlux5cux23page_79}{79--}\protect\hyperlink{10_Chapter_Three__THE_HEROIC_DREAM.xhtmlux5cux23page_80}{80},
\protect\hyperlink{18_Chapter_Eleven__THE_FORMS_OF_THO.xhtmlux5cux23page_283}{283},
\protect\hyperlink{19_Chapter_Twelve__ART_IN_LIFE.xhtmlux5cux23page_295}{295}

Areopagite, Pseudo--Dionysius,
\protect\hyperlink{14_Chapter_Seven__THE_PIOUS_PERSONA.xhtmlux5cux23page_218}{218},
\protect\hyperlink{17_Chapter_Ten__THE_FAILURE_OF_IMAG.xhtmlux5cux23page_258}{258},
\protect\hyperlink{17_Chapter_Ten__THE_FAILURE_OF_IMAG.xhtmlux5cux23page_261}{261}

Aristo, Ludovico,
\protect\hyperlink{10_Chapter_Three__THE_HEROIC_DREAM.xhtmlux5cux23page_85}{85},
\protect\hyperlink{21_Chapter_Thirteen__IMAGE_AND_WORD.xhtmlux5cux23page_329}{329}

Armagnac,
\protect\hyperlink{08_Chapter_One__THE_PASSIONATE_INTE.xhtmlux5cux23page_3}{3},
\protect\hyperlink{08_Chapter_One__THE_PASSIONATE_INTE.xhtmlux5cux23page_4}{4},
\protect\hyperlink{08_Chapter_One__THE_PASSIONATE_INTE.xhtmlux5cux23page_18}{18},
\protect\hyperlink{14_Chapter_Seven__THE_PIOUS_PERSONA.xhtmlux5cux23page_205}{205}

Armentières, Peronelle d',
\protect\hyperlink{11_Chapter_Four__THE_FORMS_OF_LOVE.xhtmlux5cux23page_142}{142},
\protect\hyperlink{11_Chapter_Four__THE_FORMS_OF_LOVE.xhtmlux5cux23page_144}{144},
\protect\hyperlink{11_Chapter_Four__THE_FORMS_OF_LOVE.xhtmlux5cux23page_147}{147--}\protect\hyperlink{11_Chapter_Four__THE_FORMS_OF_LOVE.xhtmlux5cux23page_150}{150}

Arnolfini, Giovanni,
\protect\hyperlink{20_ILLUSTRATIONS_FOLLOW_PAGE.xhtmlux5cux23page_312}{312},
\protect\hyperlink{21_Chapter_Thirteen__IMAGE_AND_WORD.xhtmlux5cux23page_331}{331}

Arras: reputation of,
\protect\hyperlink{18_Chapter_Eleven__THE_FORMS_OF_THO.xhtmlux5cux23page_289}{289};
treaty of,
\protect\hyperlink{08_Chapter_One__THE_PASSIONATE_INTE.xhtmlux5cux23page_16}{16}

\emph{Arrestz d'amour} (Martial d'Auvergne),
\protect\hyperlink{11_Chapter_Four__THE_FORMS_OF_LOVE.xhtmlux5cux23page_131}{131},
\protect\hyperlink{11_Chapter_Four__THE_FORMS_OF_LOVE.xhtmlux5cux23page_144}{144}

\emph{ars morendi},
\protect\hyperlink{12_Chapter_Five__THE_VISION_OF_DEAT.xhtmlux5cux23page_167}{167}

Art of Dying. See \emph{ars morendi}

Artevelde, Jacob van,
\protect\hyperlink{10_Chapter_Three__THE_HEROIC_DREAM.xhtmlux5cux23page_104}{104}

Artevelde, Phillip van,
\protect\hyperlink{09_Chapter_Two__THE_CRAVING_FOR_A_M.xhtmlux5cux23page_34}{34},
\protect\hyperlink{10_Chapter_Three__THE_HEROIC_DREAM.xhtmlux5cux23page_104}{104},
\protect\hyperlink{10_Chapter_Three__THE_HEROIC_DREAM.xhtmlux5cux23page_115}{115}

\emph{Artharva-Veda},
\protect\hyperlink{17_Chapter_Ten__THE_FAILURE_OF_IMAG.xhtmlux5cux23page_255}{255}

Artois, Robert d',
\protect\hyperlink{10_Chapter_Three__THE_HEROIC_DREAM.xhtmlux5cux23page_98}{98},
\protect\hyperlink{10_Chapter_Three__THE_HEROIC_DREAM.xhtmlux5cux23page_102}{102}

\protect\hypertarget{25_INDEX.xhtmlux5cux23page_452}{}{}Aubroit, Hugues,
\protect\hyperlink{13_Chapter_Six__THE_DEPICTION_OF_TH.xhtmlux5cux23page_179}{179},
\protect\hyperlink{13_Chapter_Six__THE_DEPICTION_OF_TH.xhtmlux5cux23page_189}{189},
\protect\hyperlink{18_Chapter_Eleven__THE_FORMS_OF_THO.xhtmlux5cux23page_274}{274}

Augustine, Saint,
\protect\hyperlink{08_Chapter_One__THE_PASSIONATE_INTE.xhtmlux5cux23page_25}{25},
\protect\hyperlink{17_Chapter_Ten__THE_FAILURE_OF_IMAG.xhtmlux5cux23page_266}{266},
\protect\hyperlink{18_Chapter_Eleven__THE_FORMS_OF_THO.xhtmlux5cux23page_293}{293},
\protect\hyperlink{20_ILLUSTRATIONS_FOLLOW_PAGE.xhtmlux5cux23page_321}{321}

Augustinian order,
\protect\hyperlink{10_Chapter_Three__THE_HEROIC_DREAM.xhtmlux5cux23page_64}{64}

Autun Altarpiece,
\protect\hyperlink{20_ILLUSTRATIONS_FOLLOW_PAGE.xhtmlux5cux23page_306}{306},
\protect\hyperlink{20_ILLUSTRATIONS_FOLLOW_PAGE.xhtmlux5cux23page_317}{317},
\protect\hyperlink{21_Chapter_Thirteen__IMAGE_AND_WORD.xhtmlux5cux23page_333}{333-}\protect\hyperlink{21_Chapter_Thirteen__IMAGE_AND_WORD.xhtmlux5cux23page_335}{335},
\protect\hyperlink{21_Chapter_Thirteen__IMAGE_AND_WORD.xhtmlux5cux23page_336}{336}

Averröes,
\protect\hyperlink{13_Chapter_Six__THE_DEPICTION_OF_TH.xhtmlux5cux23page_189}{189}

Baerze, Jacques de,
\protect\hyperlink{20_ILLUSTRATIONS_FOLLOW_PAGE.xhtmlux5cux23page_301}{301}

Bayazid, sultan,
\protect\hyperlink{10_Chapter_Three__THE_HEROIC_DREAM.xhtmlux5cux23page_78}{78},
\protect\hyperlink{10_Chapter_Three__THE_HEROIC_DREAM.xhtmlux5cux23page_86}{86}

``Ballade de Fougères'' (Alain Chartier),
\protect\hyperlink{18_Chapter_Eleven__THE_FORMS_OF_THO.xhtmlux5cux23page_274}{274}

``Ballade des dames du temps jadis'' (Villon),
\protect\hyperlink{12_Chapter_Five__THE_VISION_OF_DEAT.xhtmlux5cux23page_158}{158}

Bamborough, Robert,
\protect\hyperlink{10_Chapter_Three__THE_HEROIC_DREAM.xhtmlux5cux23page_74}{74}

Bandello,
\protect\hyperlink{10_Chapter_Three__THE_HEROIC_DREAM.xhtmlux5cux23page_108}{108}

Barante, Prosper de,
\protect\hyperlink{19_Chapter_Twelve__ART_IN_LIFE.xhtmlux5cux23page_294}{294}

Barbara, Saint,
\protect\hyperlink{13_Chapter_Six__THE_DEPICTION_OF_TH.xhtmlux5cux23page_197}{197},
\protect\hyperlink{13_Chapter_Six__THE_DEPICTION_OF_TH.xhtmlux5cux23page_198}{198},
\protect\hyperlink{16_Chapter_Nine__THE_DECLINE_OF_SYM.xhtmlux5cux23page_246}{246}

Basin, Thomas (bishop of Lisieux),
\protect\hyperlink{10_Chapter_Three__THE_HEROIC_DREAM.xhtmlux5cux23page_72}{72},
\protect\hyperlink{18_Chapter_Eleven__THE_FORMS_OF_THO.xhtmlux5cux23page_283}{283}

Bastard of Vauru,
\protect\hyperlink{21_Chapter_Thirteen__IMAGE_AND_WORD.xhtmlux5cux23page_373}{373}

``Bataille de Karesme et de charnage,''
\protect\hyperlink{16_Chapter_Nine__THE_DECLINE_OF_SYM.xhtmlux5cux23page_246}{246}

\emph{Bath of Women} (Jan van Eyck),
\protect\hyperlink{21_Chapter_Thirteen__IMAGE_AND_WORD.xhtmlux5cux23page_373}{373}

Bavaria, Albert of,
\protect\hyperlink{10_Chapter_Three__THE_HEROIC_DREAM.xhtmlux5cux23page_118}{118}

Bavaria, Isabella,
\protect\hyperlink{08_Chapter_One__THE_PASSIONATE_INTE.xhtmlux5cux23page_11}{11},
\protect\hyperlink{11_Chapter_Four__THE_FORMS_OF_LOVE.xhtmlux5cux23page_129}{129},
\protect\hyperlink{13_Chapter_Six__THE_DEPICTION_OF_TH.xhtmlux5cux23page_185}{185},
\protect\hyperlink{20_ILLUSTRATIONS_FOLLOW_PAGE.xhtmlux5cux23page_298}{298},
\protect\hyperlink{20_ILLUSTRATIONS_FOLLOW_PAGE.xhtmlux5cux23page_311}{311}

Bavaria, John of,
\protect\hyperlink{09_Chapter_Two__THE_CRAVING_FOR_A_M.xhtmlux5cux23page_50}{50},
\protect\hyperlink{14_Chapter_Seven__THE_PIOUS_PERSONA.xhtmlux5cux23page_206}{206}

Beaugrant, Madam de,
\protect\hyperlink{08_Chapter_One__THE_PASSIONATE_INTE.xhtmlux5cux23page_23}{23}

Beaumanoir, Robert de,
\protect\hyperlink{10_Chapter_Three__THE_HEROIC_DREAM.xhtmlux5cux23page_112}{112},
\protect\hyperlink{14_Chapter_Seven__THE_PIOUS_PERSONA.xhtmlux5cux23page_211}{211}

Beaumont, Jean de,
\protect\hyperlink{10_Chapter_Three__THE_HEROIC_DREAM.xhtmlux5cux23page_87}{87},
\protect\hyperlink{10_Chapter_Three__THE_HEROIC_DREAM.xhtmlux5cux23page_102}{102}

Beaune Altarpiece (van der Weyden),
\protect\hyperlink{20_ILLUSTRATIONS_FOLLOW_PAGE.xhtmlux5cux23page_306}{306},
\protect\hyperlink{20_ILLUSTRATIONS_FOLLOW_PAGE.xhtmlux5cux23page_317}{317}

Beauté, castle of,
\protect\hyperlink{21_Chapter_Thirteen__IMAGE_AND_WORD.xhtmlux5cux23page_352}{352}

Bedford, John of Lancaster, duke of,
\protect\hyperlink{09_Chapter_Two__THE_CRAVING_FOR_A_M.xhtmlux5cux23page_51}{51},
\protect\hyperlink{09_Chapter_Two__THE_CRAVING_FOR_A_M.xhtmlux5cux23page_57}{57},
\protect\hyperlink{10_Chapter_Three__THE_HEROIC_DREAM.xhtmlux5cux23page_93}{93},
\protect\hyperlink{10_Chapter_Three__THE_HEROIC_DREAM.xhtmlux5cux23page_112}{112},
\protect\hyperlink{14_Chapter_Seven__THE_PIOUS_PERSONA.xhtmlux5cux23page_206}{206}

Beggars,
\protect\hyperlink{08_Chapter_One__THE_PASSIONATE_INTE.xhtmlux5cux23page_27}{27},
\protect\hyperlink{13_Chapter_Six__THE_DEPICTION_OF_TH.xhtmlux5cux23page_199}{199},
\protect\hyperlink{13_Chapter_Six__THE_DEPICTION_OF_TH.xhtmlux5cux23page_200}{200},
\protect\hyperlink{21_Chapter_Thirteen__IMAGE_AND_WORD.xhtmlux5cux23page_365}{365}.
\emph{See also} mendicant orders

beguines,
\protect\hyperlink{08_Chapter_One__THE_PASSIONATE_INTE.xhtmlux5cux23page_6}{6}

Belon,
\protect\hyperlink{08_Chapter_One__THE_PASSIONATE_INTE.xhtmlux5cux23page_23}{23}

Benedict VIII, pope,
\protect\hyperlink{08_Chapter_One__THE_PASSIONATE_INTE.xhtmlux5cux23page_13}{13},
\protect\hyperlink{10_Chapter_Three__THE_HEROIC_DREAM.xhtmlux5cux23page_100}{100}

Bernard, Saint,
\protect\hyperlink{14_Chapter_Seven__THE_PIOUS_PERSONA.xhtmlux5cux23page_218}{218},
\protect\hyperlink{15_Chapter_Eight__RELIGIOUS_EXCITAT.xhtmlux5cux23page_220}{220},
\protect\hyperlink{15_Chapter_Eight__RELIGIOUS_EXCITAT.xhtmlux5cux23page_222}{222},
\protect\hyperlink{15_Chapter_Eight__RELIGIOUS_EXCITAT.xhtmlux5cux23page_229}{229},
\protect\hyperlink{15_Chapter_Eight__RELIGIOUS_EXCITAT.xhtmlux5cux23page_232}{232};
language and mysticism of,
\protect\hyperlink{17_Chapter_Ten__THE_FAILURE_OF_IMAG.xhtmlux5cux23page_263}{263},
\protect\hyperlink{20_ILLUSTRATIONS_FOLLOW_PAGE.xhtmlux5cux23page_318}{318};
Thomas à Kempis and,
\protect\hyperlink{17_Chapter_Ten__THE_FAILURE_OF_IMAG.xhtmlux5cux23page_266}{266};
treasury of merit,
\protect\hyperlink{17_Chapter_Ten__THE_FAILURE_OF_IMAG.xhtmlux5cux23page_256}{256}

Bernard of Morley,
\protect\hyperlink{12_Chapter_Five__THE_VISION_OF_DEAT.xhtmlux5cux23page_157}{157}

Bernardino of Siena, Saint,
\protect\hyperlink{16_Chapter_Nine__THE_DECLINE_OF_SYM.xhtmlux5cux23page_234}{234}

Berry, herald,
\protect\hyperlink{10_Chapter_Three__THE_HEROIC_DREAM.xhtmlux5cux23page_73}{73}

Berry, John, duke of: amorous debate,
\protect\hyperlink{11_Chapter_Four__THE_FORMS_OF_LOVE.xhtmlux5cux23page_137}{137};
artists and,
\protect\hyperlink{20_ILLUSTRATIONS_FOLLOW_PAGE.xhtmlux5cux23page_298}{298},
\protect\hyperlink{20_ILLUSTRATIONS_FOLLOW_PAGE.xhtmlux5cux23page_313}{313};
decorates chapel,
\protect\hyperlink{19_Chapter_Twelve__ART_IN_LIFE.xhtmlux5cux23page_297}{297};
Louis d'Orléans, murder of and,
\protect\hyperlink{18_Chapter_Eleven__THE_FORMS_OF_THO.xhtmlux5cux23page_271}{271};
Peter of Luxembourg and,
\protect\hyperlink{14_Chapter_Seven__THE_PIOUS_PERSONA.xhtmlux5cux23page_213}{213},
\protect\hyperlink{14_Chapter_Seven__THE_PIOUS_PERSONA.xhtmlux5cux23page_214}{214};
prohibits knightly duel,
\protect\hyperlink{10_Chapter_Three__THE_HEROIC_DREAM.xhtmlux5cux23page_112}{112};
relic of St. Louis and,
\protect\hyperlink{13_Chapter_Six__THE_DEPICTION_OF_TH.xhtmlux5cux23page_192}{192};
sculptures and,
\protect\hyperlink{12_Chapter_Five__THE_VISION_OF_DEAT.xhtmlux5cux23page_164}{164},
\protect\hyperlink{12_Chapter_Five__THE_VISION_OF_DEAT.xhtmlux5cux23page_170}{170};
wears satin armor,
\protect\hyperlink{10_Chapter_Three__THE_HEROIC_DREAM.xhtmlux5cux23page_115}{115}

Berthelemy, Jean,
\protect\hyperlink{15_Chapter_Eight__RELIGIOUS_EXCITAT.xhtmlux5cux23page_231}{231}

Bertulph, Saint,
\protect\hyperlink{13_Chapter_Six__THE_DEPICTION_OF_TH.xhtmlux5cux23page_193}{193}

Bétisac, Jean,
\protect\hyperlink{13_Chapter_Six__THE_DEPICTION_OF_TH.xhtmlux5cux23page_188}{188}

\emph{bibhatsa-rasa},
\protect\hyperlink{12_Chapter_Five__THE_VISION_OF_DEAT.xhtmlux5cux23page_159}{159}

\emph{Bien public}, war of the,
\protect\hyperlink{10_Chapter_Three__THE_HEROIC_DREAM.xhtmlux5cux23page_79}{79--}\protect\hyperlink{10_Chapter_Three__THE_HEROIC_DREAM.xhtmlux5cux23page_80}{80},
\protect\hyperlink{10_Chapter_Three__THE_HEROIC_DREAM.xhtmlux5cux23page_115}{115}

Bièvre, castle of,
\protect\hyperlink{21_Chapter_Thirteen__IMAGE_AND_WORD.xhtmlux5cux23page_352}{352}

Bladelin, Pieter,
\protect\hyperlink{20_ILLUSTRATIONS_FOLLOW_PAGE.xhtmlux5cux23page_316}{316}

Bladelin Altarpiece (van Eyck),
\protect\hyperlink{20_ILLUSTRATIONS_FOLLOW_PAGE.xhtmlux5cux23page_316}{316}

Blaise, Saint,
\protect\hyperlink{13_Chapter_Six__THE_DEPICTION_OF_TH.xhtmlux5cux23page_198}{198}

\emph{Blason des couleurs, Le} (Sizilien),
\protect\hyperlink{11_Chapter_Four__THE_FORMS_OF_LOVE.xhtmlux5cux23page_142}{142}

blasphemy: eroticism, mixed with,
\protect\hyperlink{21_Chapter_Thirteen__IMAGE_AND_WORD.xhtmlux5cux23page_370}{370};
Gerson,
\protect\hyperlink{13_Chapter_Six__THE_DEPICTION_OF_TH.xhtmlux5cux23page_187}{187};
\emph{Hansje in den Kelder},
\protect\hyperlink{13_Chapter_Six__THE_DEPICTION_OF_TH.xhtmlux5cux23page_179}{179};
heretical visions and beliefs,
\protect\hyperlink{13_Chapter_Six__THE_DEPICTION_OF_TH.xhtmlux5cux23page_176}{176--}\protect\hyperlink{13_Chapter_Six__THE_DEPICTION_OF_TH.xhtmlux5cux23page_177}{177};
root in strong faith,
\protect\hyperlink{13_Chapter_Six__THE_DEPICTION_OF_TH.xhtmlux5cux23page_186}{186};
sacred festivals debauched,
\protect\hyperlink{13_Chapter_Six__THE_DEPICTION_OF_TH.xhtmlux5cux23page_183}{183--}\protect\hyperlink{13_Chapter_Six__THE_DEPICTION_OF_TH.xhtmlux5cux23page_184}{184}.
See also \emph{Roman de la rose}

Blois, Charles de,
\protect\hyperlink{14_Chapter_Seven__THE_PIOUS_PERSONA.xhtmlux5cux23page_211}{211--}\protect\hyperlink{14_Chapter_Seven__THE_PIOUS_PERSONA.xhtmlux5cux23page_212}{212}

Blois, Jehan de,
\protect\hyperlink{21_Chapter_Thirteen__IMAGE_AND_WORD.xhtmlux5cux23page_355}{355}

Boccaccio, Giovanni,
\protect\hyperlink{22_Chapter_Fourteen__THE_COMING_OF.xhtmlux5cux23page_385}{385},
\protect\hyperlink{22_Chapter_Fourteen__THE_COMING_OF.xhtmlux5cux23page_386}{386}

Bois, Nansart du,
\protect\hyperlink{08_Chapter_One__THE_PASSIONATE_INTE.xhtmlux5cux23page_3}{3}

Bonaventura, Saint,
\protect\hyperlink{15_Chapter_Eight__RELIGIOUS_EXCITAT.xhtmlux5cux23page_231}{231},
\protect\hyperlink{16_Chapter_Nine__THE_DECLINE_OF_SYM.xhtmlux5cux23page_248}{248},
\protect\hyperlink{20_ILLUSTRATIONS_FOLLOW_PAGE.xhtmlux5cux23page_318}{318}

Bonet, Honoré,
\protect\hyperlink{18_Chapter_Eleven__THE_FORMS_OF_THO.xhtmlux5cux23page_277}{277--}\protect\hyperlink{18_Chapter_Eleven__THE_FORMS_OF_THO.xhtmlux5cux23page_278}{278}

Boniface, Jean de,
\protect\hyperlink{10_Chapter_Three__THE_HEROIC_DREAM.xhtmlux5cux23page_101}{101}

Boniface VIII, pope,
\protect\hyperlink{12_Chapter_Five__THE_VISION_OF_DEAT.xhtmlux5cux23page_163}{163}

Borgia, Cesare,
\protect\hyperlink{10_Chapter_Three__THE_HEROIC_DREAM.xhtmlux5cux23page_109}{109}

Borromeus, Karl, Saint,
\protect\hyperlink{14_Chapter_Seven__THE_PIOUS_PERSONA.xhtmlux5cux23page_210}{210}

Boucicaut, Jean le Meingre, maréchal: biography,
\protect\hyperlink{10_Chapter_Three__THE_HEROIC_DREAM.xhtmlux5cux23page_70}{70},
\protect\hyperlink{10_Chapter_Three__THE_HEROIC_DREAM.xhtmlux5cux23page_78}{78--}\protect\hyperlink{10_Chapter_Three__THE_HEROIC_DREAM.xhtmlux5cux23page_80}{80},
\protect\hyperlink{10_Chapter_Three__THE_HEROIC_DREAM.xhtmlux5cux23page_85}{85--}\protect\hyperlink{10_Chapter_Three__THE_HEROIC_DREAM.xhtmlux5cux23page_86}{86};
Cemetery of the Innocents and,
\protect\hyperlink{12_Chapter_Five__THE_VISION_OF_DEAT.xhtmlux5cux23page_170}{170};
Christine de Pisan and,
\protect\hyperlink{11_Chapter_Four__THE_FORMS_OF_LOVE.xhtmlux5cux23page_140}{140};
greed,
\protect\hyperlink{10_Chapter_Three__THE_HEROIC_DREAM.xhtmlux5cux23page_117}{117};
Ordre de la Dame blanche,
\protect\hyperlink{10_Chapter_Three__THE_HEROIC_DREAM.xhtmlux5cux23page_94}{94};
women, honor of,
\protect\hyperlink{13_Chapter_Six__THE_DEPICTION_OF_TH.xhtmlux5cux23page_174}{174}

Bouillon, Godfrey of,
\protect\hyperlink{10_Chapter_Three__THE_HEROIC_DREAM.xhtmlux5cux23page_76}{76}

Bourbon, Isabella de,
\protect\hyperlink{09_Chapter_Two__THE_CRAVING_FOR_A_M.xhtmlux5cux23page_56}{56--}\protect\hyperlink{09_Chapter_Two__THE_CRAVING_FOR_A_M.xhtmlux5cux23page_58}{58},
\protect\hyperlink{18_Chapter_Eleven__THE_FORMS_OF_THO.xhtmlux5cux23page_281}{281}

Bourbon, Jacques de, king,
\protect\hyperlink{14_Chapter_Seven__THE_PIOUS_PERSONA.xhtmlux5cux23page_209}{209}

Bourbon, Jean de, duke,
\protect\hyperlink{10_Chapter_Three__THE_HEROIC_DREAM.xhtmlux5cux23page_101}{101}

Bourbon, Louis de,
\protect\hyperlink{10_Chapter_Three__THE_HEROIC_DREAM.xhtmlux5cux23page_95}{95},
\protect\hyperlink{14_Chapter_Seven__THE_PIOUS_PERSONA.xhtmlux5cux23page_213}{213}

Bouts, Dirk,
\protect\hyperlink{19_Chapter_Twelve__ART_IN_LIFE.xhtmlux5cux23page_297}{297},
\protect\hyperlink{21_Chapter_Thirteen__IMAGE_AND_WORD.xhtmlux5cux23page_376}{376}

\protect\hypertarget{25_INDEX.xhtmlux5cux23page_453}{}{}Brabant, Anton
of,
\protect\hyperlink{11_Chapter_Four__THE_FORMS_OF_LOVE.xhtmlux5cux23page_140}{140}

Breughel, Peter,
\protect\hyperlink{20_ILLUSTRATIONS_FOLLOW_PAGE.xhtmlux5cux23page_300}{300},
\protect\hyperlink{21_Chapter_Thirteen__IMAGE_AND_WORD.xhtmlux5cux23page_363}{363},
\protect\hyperlink{21_Chapter_Thirteen__IMAGE_AND_WORD.xhtmlux5cux23page_364}{364}

Bridget of Sweden, Saint,
\protect\hyperlink{15_Chapter_Eight__RELIGIOUS_EXCITAT.xhtmlux5cux23page_225}{225}

Broderlam, Melchoir,
\protect\hyperlink{13_Chapter_Six__THE_DEPICTION_OF_TH.xhtmlux5cux23page_177}{177},
\protect\hyperlink{21_Chapter_Thirteen__IMAGE_AND_WORD.xhtmlux5cux23page_363}{363}

Brothers of the Common Life,
\protect\hyperlink{15_Chapter_Eight__RELIGIOUS_EXCITAT.xhtmlux5cux23page_223}{223}.
See also \emph{Devotio moderna}

Brothers of the Free Spirit (Turlupins),
\protect\hyperlink{12_Chapter_Five__THE_VISION_OF_DEAT.xhtmlux5cux23page_163}{163},
\protect\hyperlink{13_Chapter_Six__THE_DEPICTION_OF_TH.xhtmlux5cux23page_189}{189},
\protect\hyperlink{15_Chapter_Eight__RELIGIOUS_EXCITAT.xhtmlux5cux23page_229}{229}

Brugman, Jan,
\protect\hyperlink{14_Chapter_Seven__THE_PIOUS_PERSONA.xhtmlux5cux23page_218}{218},
\protect\hyperlink{15_Chapter_Eight__RELIGIOUS_EXCITAT.xhtmlux5cux23page_230}{230},
\protect\hyperlink{17_Chapter_Ten__THE_FAILURE_OF_IMAG.xhtmlux5cux23page_250}{250},
\protect\hyperlink{17_Chapter_Ten__THE_FAILURE_OF_IMAG.xhtmlux5cux23page_266}{266},
\protect\hyperlink{20_ILLUSTRATIONS_FOLLOW_PAGE.xhtmlux5cux23page_319}{319}

Buddhism,
\protect\hyperlink{09_Chapter_Two__THE_CRAVING_FOR_A_M.xhtmlux5cux23page_35}{35},
\protect\hyperlink{17_Chapter_Ten__THE_FAILURE_OF_IMAG.xhtmlux5cux23page_252}{252},
\protect\hyperlink{17_Chapter_Ten__THE_FAILURE_OF_IMAG.xhtmlux5cux23page_263}{263--}\protect\hyperlink{17_Chapter_Ten__THE_FAILURE_OF_IMAG.xhtmlux5cux23page_264}{264}

Bueil, Jean de (Le Jouvencel): biography \emph{(Le Jouvencel)},
\protect\hyperlink{10_Chapter_Three__THE_HEROIC_DREAM.xhtmlux5cux23page_79}{79--}\protect\hyperlink{10_Chapter_Three__THE_HEROIC_DREAM.xhtmlux5cux23page_82}{82};
casuistry,
\protect\hyperlink{18_Chapter_Eleven__THE_FORMS_OF_THO.xhtmlux5cux23page_277}{277--}\protect\hyperlink{18_Chapter_Eleven__THE_FORMS_OF_THO.xhtmlux5cux23page_278}{278};
Chastellain, in,
\protect\hyperlink{10_Chapter_Three__THE_HEROIC_DREAM.xhtmlux5cux23page_78}{78};
dictums,
\protect\hyperlink{18_Chapter_Eleven__THE_FORMS_OF_THO.xhtmlux5cux23page_275}{275};
Greek, errors in,
\protect\hyperlink{22_Chapter_Fourteen__THE_COMING_OF.xhtmlux5cux23page_386}{386};
knightly duels and,
\protect\hyperlink{10_Chapter_Three__THE_HEROIC_DREAM.xhtmlux5cux23page_112}{112--}\protect\hyperlink{10_Chapter_Three__THE_HEROIC_DREAM.xhtmlux5cux23page_113}{113},
\protect\hyperlink{14_Chapter_Seven__THE_PIOUS_PERSONA.xhtmlux5cux23page_206}{206};
\emph{Preux},
\protect\hyperlink{10_Chapter_Three__THE_HEROIC_DREAM.xhtmlux5cux23page_77}{77}

Burarius,
\protect\hyperlink{21_Chapter_Thirteen__IMAGE_AND_WORD.xhtmlux5cux23page_379}{379}

Burckhardt, Jacob,
\protect\hyperlink{08_Chapter_One__THE_PASSIONATE_INTE.xhtmlux5cux23page_15}{15},
\protect\hyperlink{09_Chapter_Two__THE_CRAVING_FOR_A_M.xhtmlux5cux23page_43}{43},
\protect\hyperlink{10_Chapter_Three__THE_HEROIC_DREAM.xhtmlux5cux23page_73}{73--}\protect\hyperlink{10_Chapter_Three__THE_HEROIC_DREAM.xhtmlux5cux23page_74}{74},
\protect\hyperlink{13_Chapter_Six__THE_DEPICTION_OF_TH.xhtmlux5cux23page_173}{173--}\protect\hyperlink{13_Chapter_Six__THE_DEPICTION_OF_TH.xhtmlux5cux23page_174}{174}

Burgher of Paris, the: allegory,
\protect\hyperlink{16_Chapter_Nine__THE_DECLINE_OF_SYM.xhtmlux5cux23page_244}{244};
Anne of Burgundy angers,
\protect\hyperlink{14_Chapter_Seven__THE_PIOUS_PERSONA.xhtmlux5cux23page_206}{206};
Bastard of Vauru,
\protect\hyperlink{21_Chapter_Thirteen__IMAGE_AND_WORD.xhtmlux5cux23page_373}{373};
Brother Richard,
\protect\hyperlink{08_Chapter_One__THE_PASSIONATE_INTE.xhtmlux5cux23page_5}{5};
Louis d'Orléans, funeral of,
\protect\hyperlink{14_Chapter_Seven__THE_PIOUS_PERSONA.xhtmlux5cux23page_205}{205};
modesty,
\protect\hyperlink{21_Chapter_Thirteen__IMAGE_AND_WORD.xhtmlux5cux23page_373}{373};
murders in Paris,
\protect\hyperlink{16_Chapter_Nine__THE_DECLINE_OF_SYM.xhtmlux5cux23page_244}{244};
procession of children,
\protect\hyperlink{12_Chapter_Five__THE_VISION_OF_DEAT.xhtmlux5cux23page_170}{170};
stories of lost crowns,
\protect\hyperlink{08_Chapter_One__THE_PASSIONATE_INTE.xhtmlux5cux23page_14}{14};
tournaments,
\protect\hyperlink{10_Chapter_Three__THE_HEROIC_DREAM.xhtmlux5cux23page_89}{89};
Tutetey's notes,
\protect\hyperlink{08_Chapter_One__THE_PASSIONATE_INTE.xhtmlux5cux23page_29}{29}

Burgundians, party of the: founding of dukedom,
\protect\hyperlink{10_Chapter_Three__THE_HEROIC_DREAM.xhtmlux5cux23page_104}{104--}\protect\hyperlink{10_Chapter_Three__THE_HEROIC_DREAM.xhtmlux5cux23page_105}{105};
Golden Fleece,
\protect\hyperlink{10_Chapter_Three__THE_HEROIC_DREAM.xhtmlux5cux23page_94}{94};
knightly courage,
\protect\hyperlink{10_Chapter_Three__THE_HEROIC_DREAM.xhtmlux5cux23page_104}{104--}\protect\hyperlink{10_Chapter_Three__THE_HEROIC_DREAM.xhtmlux5cux23page_105}{105};
Louis d'Orléans, murder of,
\protect\hyperlink{18_Chapter_Eleven__THE_FORMS_OF_THO.xhtmlux5cux23page_282}{282};
Mézières, Philippe de, suspicions of,
\protect\hyperlink{18_Chapter_Eleven__THE_FORMS_OF_THO.xhtmlux5cux23page_288}{288};
propaganda,
\protect\hyperlink{11_Chapter_Four__THE_FORMS_OF_LOVE.xhtmlux5cux23page_152}{152},
\protect\hyperlink{21_Chapter_Thirteen__IMAGE_AND_WORD.xhtmlux5cux23page_379}{379}

Burgundy, Anne of,
\protect\hyperlink{14_Chapter_Seven__THE_PIOUS_PERSONA.xhtmlux5cux23page_206}{206}

Burgundy, Anton of,
\protect\hyperlink{11_Chapter_Four__THE_FORMS_OF_LOVE.xhtmlux5cux23page_129}{129}

Burgundy, court of: Coquinet,
\protect\hyperlink{08_Chapter_One__THE_PASSIONATE_INTE.xhtmlux5cux23page_12}{12};
dwarves,
\protect\hyperlink{08_Chapter_One__THE_PASSIONATE_INTE.xhtmlux5cux23page_23}{23};
duel of burghers,
\protect\hyperlink{08_Chapter_One__THE_PASSIONATE_INTE.xhtmlux5cux23page_3}{3},
\protect\hyperlink{08_Chapter_One__THE_PASSIONATE_INTE.xhtmlux5cux23page_8}{8},
\protect\hyperlink{08_Chapter_One__THE_PASSIONATE_INTE.xhtmlux5cux23page_12}{12},
\protect\hyperlink{20_ILLUSTRATIONS_FOLLOW_PAGE.xhtmlux5cux23page_314}{314--}\protect\hyperlink{20_ILLUSTRATIONS_FOLLOW_PAGE.xhtmlux5cux23page_315}{315};
France compared to,
\protect\hyperlink{09_Chapter_Two__THE_CRAVING_FOR_A_M.xhtmlux5cux23page_50}{50--}\protect\hyperlink{09_Chapter_Two__THE_CRAVING_FOR_A_M.xhtmlux5cux23page_52}{52};
Louis XI at,
\protect\hyperlink{20_ILLUSTRATIONS_FOLLOW_PAGE.xhtmlux5cux23page_315}{315};
Luxembourg, dukes of and,
\protect\hyperlink{14_Chapter_Seven__THE_PIOUS_PERSONA.xhtmlux5cux23page_212}{212--}\protect\hyperlink{14_Chapter_Seven__THE_PIOUS_PERSONA.xhtmlux5cux23page_213}{213};
ritual forms,
\protect\hyperlink{09_Chapter_Two__THE_CRAVING_FOR_A_M.xhtmlux5cux23page_42}{42--}\protect\hyperlink{09_Chapter_Two__THE_CRAVING_FOR_A_M.xhtmlux5cux23page_45}{45}

Burgundy, David of,
\protect\hyperlink{13_Chapter_Six__THE_DEPICTION_OF_TH.xhtmlux5cux23page_182}{182--}\protect\hyperlink{13_Chapter_Six__THE_DEPICTION_OF_TH.xhtmlux5cux23page_183}{183}

Burgundy, dukes of: \emph{Cour d'amours},
\protect\hyperlink{08_Chapter_One__THE_PASSIONATE_INTE.xhtmlux5cux23page_24}{24};
\emph{Hansje in den Keldern},
\protect\hyperlink{13_Chapter_Six__THE_DEPICTION_OF_TH.xhtmlux5cux23page_179}{179};
saints and,
\protect\hyperlink{14_Chapter_Seven__THE_PIOUS_PERSONA.xhtmlux5cux23page_214}{214};
state forms,
\protect\hyperlink{09_Chapter_Two__THE_CRAVING_FOR_A_M.xhtmlux5cux23page_37}{37}.
\emph{See also} Charles the Bold; John the Fearless; Philip the Bold;
Philip the Good

Burgundy, Mary of,
\protect\hyperlink{09_Chapter_Two__THE_CRAVING_FOR_A_M.xhtmlux5cux23page_55}{55},
\protect\hyperlink{09_Chapter_Two__THE_CRAVING_FOR_A_M.xhtmlux5cux23page_58}{58},
\protect\hyperlink{13_Chapter_Six__THE_DEPICTION_OF_TH.xhtmlux5cux23page_181}{181}

Busnois, Antoine,
\protect\hyperlink{20_ILLUSTRATIONS_FOLLOW_PAGE.xhtmlux5cux23page_314}{314}

Bussy, Oudart de,
\protect\hyperlink{08_Chapter_One__THE_PASSIONATE_INTE.xhtmlux5cux23page_4}{4}

Byron, Byronianism,
\protect\hyperlink{09_Chapter_Two__THE_CRAVING_FOR_A_M.xhtmlux5cux23page_32}{32},
\protect\hyperlink{12_Chapter_Five__THE_VISION_OF_DEAT.xhtmlux5cux23page_157}{157}

Cachan, castle of,
\protect\hyperlink{21_Chapter_Thirteen__IMAGE_AND_WORD.xhtmlux5cux23page_352}{352}

Caesar, Julius: exemplar of knighthood,
\protect\hyperlink{10_Chapter_Three__THE_HEROIC_DREAM.xhtmlux5cux23page_75}{75};
former splendor,
\protect\hyperlink{12_Chapter_Five__THE_VISION_OF_DEAT.xhtmlux5cux23page_157}{157}

Campin, Robert (the Master of Flémalle),
\protect\hyperlink{21_Chapter_Thirteen__IMAGE_AND_WORD.xhtmlux5cux23page_361}{361--}\protect\hyperlink{21_Chapter_Thirteen__IMAGE_AND_WORD.xhtmlux5cux23page_362}{362}

Campo Santo (Pisa),
\protect\hyperlink{12_Chapter_Five__THE_VISION_OF_DEAT.xhtmlux5cux23page_164}{164}

Capeluche (hangman),
\protect\hyperlink{09_Chapter_Two__THE_CRAVING_FOR_A_M.xhtmlux5cux23page_50}{50}

Capistrano, John of,
\protect\hyperlink{14_Chapter_Seven__THE_PIOUS_PERSONA.xhtmlux5cux23page_210}{210}

Carmelites, Order of: Brother Thomas,
\protect\hyperlink{08_Chapter_One__THE_PASSIONATE_INTE.xhtmlux5cux23page_6}{6};
Paris monastery,
\protect\hyperlink{13_Chapter_Six__THE_DEPICTION_OF_TH.xhtmlux5cux23page_179}{179}

``Casibus virorum illustrium, De'' (Boccaccio),
\protect\hyperlink{18_Chapter_Eleven__THE_FORMS_OF_THO.xhtmlux5cux23page_272}{272},
\protect\hyperlink{22_Chapter_Fourteen__THE_COMING_OF.xhtmlux5cux23page_385}{385}

Catherine, Saint,
\protect\hyperlink{13_Chapter_Six__THE_DEPICTION_OF_TH.xhtmlux5cux23page_192}{192},
\protect\hyperlink{13_Chapter_Six__THE_DEPICTION_OF_TH.xhtmlux5cux23page_198}{198}

Catherine of Siena, Saint,
\protect\hyperlink{15_Chapter_Eight__RELIGIOUS_EXCITAT.xhtmlux5cux23page_225}{225},
\protect\hyperlink{15_Chapter_Eight__RELIGIOUS_EXCITAT.xhtmlux5cux23page_229}{229},
\protect\hyperlink{15_Chapter_Eight__RELIGIOUS_EXCITAT.xhtmlux5cux23page_232}{232}

Caxton, William,
\protect\hyperlink{20_ILLUSTRATIONS_FOLLOW_PAGE.xhtmlux5cux23page_311}{311}

Celestines, Order of the: and Louis d'Orleans,
\protect\hyperlink{14_Chapter_Seven__THE_PIOUS_PERSONA.xhtmlux5cux23page_208}{208};
and Peter of Luxembourg,
\protect\hyperlink{14_Chapter_Seven__THE_PIOUS_PERSONA.xhtmlux5cux23page_213}{213};
and Philippe de Mézières,
\protect\hyperlink{14_Chapter_Seven__THE_PIOUS_PERSONA.xhtmlux5cux23page_206}{206},
\protect\hyperlink{14_Chapter_Seven__THE_PIOUS_PERSONA.xhtmlux5cux23page_208}{208},
\protect\hyperlink{18_Chapter_Eleven__THE_FORMS_OF_THO.xhtmlux5cux23page_272}{272};
Avignon monastery, wall painting,
\protect\hyperlink{12_Chapter_Five__THE_VISION_OF_DEAT.xhtmlux5cux23page_161}{161};
in ``Le Pastoralet,''
\protect\hyperlink{22_Chapter_Fourteen__THE_COMING_OF.xhtmlux5cux23page_393}{393};
Paris monastery,
\protect\hyperlink{14_Chapter_Seven__THE_PIOUS_PERSONA.xhtmlux5cux23page_206}{206},
\protect\hyperlink{14_Chapter_Seven__THE_PIOUS_PERSONA.xhtmlux5cux23page_208}{208},
\protect\hyperlink{14_Chapter_Seven__THE_PIOUS_PERSONA.xhtmlux5cux23page_213}{213}

Cemetery of the Innocents: Brother Richard preaches in,
\protect\hyperlink{08_Chapter_One__THE_PASSIONATE_INTE.xhtmlux5cux23page_5}{5};
closed,
\protect\hyperlink{08_Chapter_One__THE_PASSIONATE_INTE.xhtmlux5cux23page_27}{27};
\emph{danse macabre} in,
\protect\hyperlink{12_Chapter_Five__THE_VISION_OF_DEAT.xhtmlux5cux23page_165}{165--}\protect\hyperlink{12_Chapter_Five__THE_VISION_OF_DEAT.xhtmlux5cux23page_166}{166},
\protect\hyperlink{12_Chapter_Five__THE_VISION_OF_DEAT.xhtmlux5cux23page_169}{169--}\protect\hyperlink{12_Chapter_Five__THE_VISION_OF_DEAT.xhtmlux5cux23page_170}{170};
nuns in,
\protect\hyperlink{13_Chapter_Six__THE_DEPICTION_OF_TH.xhtmlux5cux23page_179}{179}

\emph{Cent ballades, Livre des} (Boucicaut),
\protect\hyperlink{11_Chapter_Four__THE_FORMS_OF_LOVE.xhtmlux5cux23page_137}{137}

\emph{Cent nouvelles nouvelles, Les}: blasphemy,
\protect\hyperlink{13_Chapter_Six__THE_DEPICTION_OF_TH.xhtmlux5cux23page_187}{187};
eroticism,
\protect\hyperlink{11_Chapter_Four__THE_FORMS_OF_LOVE.xhtmlux5cux23page_130}{130},
\protect\hyperlink{11_Chapter_Four__THE_FORMS_OF_LOVE.xhtmlux5cux23page_131}{131},
\protect\hyperlink{11_Chapter_Four__THE_FORMS_OF_LOVE.xhtmlux5cux23page_150}{150};
obscenity,
\protect\hyperlink{13_Chapter_Six__THE_DEPICTION_OF_TH.xhtmlux5cux23page_181}{181--}\protect\hyperlink{13_Chapter_Six__THE_DEPICTION_OF_TH.xhtmlux5cux23page_182}{182},
\protect\hyperlink{21_Chapter_Thirteen__IMAGE_AND_WORD.xhtmlux5cux23page_370}{370}

Chaise-Dieu, La,
\protect\hyperlink{12_Chapter_Five__THE_VISION_OF_DEAT.xhtmlux5cux23page_166}{166}

Champion, Pierre,
\protect\hyperlink{08_Chapter_One__THE_PASSIONATE_INTE.xhtmlux5cux23page_29}{29}

``Champion des Dames, Le'' (Martin Lefranc),
\protect\hyperlink{18_Chapter_Eleven__THE_FORMS_OF_THO.xhtmlux5cux23page_290}{290}

Champol, monastery of,
\protect\hyperlink{14_Chapter_Seven__THE_PIOUS_PERSONA.xhtmlux5cux23page_209}{209};
Moses Fountain (Sluter),
\protect\hyperlink{20_ILLUSTRATIONS_FOLLOW_PAGE.xhtmlux5cux23page_308}{308--}\protect\hyperlink{20_ILLUSTRATIONS_FOLLOW_PAGE.xhtmlux5cux23page_311}{311}

\emph{\protect\hypertarget{25_INDEX.xhtmlux5cux23page_454}{}{}chansons
de geste},
\protect\hyperlink{21_Chapter_Thirteen__IMAGE_AND_WORD.xhtmlux5cux23page_356}{356}

\emph{Chapel des fleurs delis} (Philippe de Vitri),
\protect\hyperlink{10_Chapter_Three__THE_HEROIC_DREAM.xhtmlux5cux23page_70}{70}

Charlemagne,
\protect\hyperlink{10_Chapter_Three__THE_HEROIC_DREAM.xhtmlux5cux23page_76}{76}

Charles V (emperor): entry into Antwerp,
\protect\hyperlink{21_Chapter_Thirteen__IMAGE_AND_WORD.xhtmlux5cux23page_374}{374};
mignon,
\protect\hyperlink{09_Chapter_Two__THE_CRAVING_FOR_A_M.xhtmlux5cux23page_59}{59};
princely duel,
\protect\hyperlink{10_Chapter_Three__THE_HEROIC_DREAM.xhtmlux5cux23page_109}{109}

Charles V (king of France): and Oresme,
\protect\hyperlink{22_Chapter_Fourteen__THE_COMING_OF.xhtmlux5cux23page_384}{384};
confession for felons,
\protect\hyperlink{08_Chapter_One__THE_PASSIONATE_INTE.xhtmlux5cux23page_21}{21};
piety,
\protect\hyperlink{14_Chapter_Seven__THE_PIOUS_PERSONA.xhtmlux5cux23page_206}{206}

Charles VI (king of France): and Philippe de Mézières,
\protect\hyperlink{10_Chapter_Three__THE_HEROIC_DREAM.xhtmlux5cux23page_71}{71};
\emph{Arbre des batailles} dedicated to,
\protect\hyperlink{18_Chapter_Eleven__THE_FORMS_OF_THO.xhtmlux5cux23page_277}{277};
Boucicaut in Genoa for,
\protect\hyperlink{10_Chapter_Three__THE_HEROIC_DREAM.xhtmlux5cux23page_79}{79};
chooses wife from portrait,
\protect\hyperlink{18_Chapter_Eleven__THE_FORMS_OF_THO.xhtmlux5cux23page_277}{277};
coronation,
\protect\hyperlink{09_Chapter_Two__THE_CRAVING_FOR_A_M.xhtmlux5cux23page_50}{50};
\emph{cour d'amours},
\protect\hyperlink{11_Chapter_Four__THE_FORMS_OF_LOVE.xhtmlux5cux23page_140}{140};
demeans body of Arteveldt,
\protect\hyperlink{10_Chapter_Three__THE_HEROIC_DREAM.xhtmlux5cux23page_115}{115};
distributes relics of Saint Louis,
\protect\hyperlink{13_Chapter_Six__THE_DEPICTION_OF_TH.xhtmlux5cux23page_192}{192};
entry into Paris,
\protect\hyperlink{20_ILLUSTRATIONS_FOLLOW_PAGE.xhtmlux5cux23page_311}{311};
funeral,
\protect\hyperlink{09_Chapter_Two__THE_CRAVING_FOR_A_M.xhtmlux5cux23page_51}{51};
madman,
\protect\hyperlink{08_Chapter_One__THE_PASSIONATE_INTE.xhtmlux5cux23page_12}{12},
\protect\hyperlink{08_Chapter_One__THE_PASSIONATE_INTE.xhtmlux5cux23page_23}{23};
marriage,
\protect\hyperlink{11_Chapter_Four__THE_FORMS_OF_LOVE.xhtmlux5cux23page_129}{129},
\protect\hyperlink{13_Chapter_Six__THE_DEPICTION_OF_TH.xhtmlux5cux23page_185}{185};
penance for felons,
\protect\hyperlink{08_Chapter_One__THE_PASSIONATE_INTE.xhtmlux5cux23page_21}{21};
thrashed,
\protect\hyperlink{08_Chapter_One__THE_PASSIONATE_INTE.xhtmlux5cux23page_11}{11};
war with Armagnacs,
\protect\hyperlink{08_Chapter_One__THE_PASSIONATE_INTE.xhtmlux5cux23page_3}{3};
welcomed to Paris,
\protect\hyperlink{08_Chapter_One__THE_PASSIONATE_INTE.xhtmlux5cux23page_19}{19}

Charles VII (king of France): Chastellain's \emph{Mystère}; entry into
Reims,
\protect\hyperlink{20_ILLUSTRATIONS_FOLLOW_PAGE.xhtmlux5cux23page_311}{311};
funeral,
\protect\hyperlink{08_Chapter_One__THE_PASSIONATE_INTE.xhtmlux5cux23page_7}{7},
\protect\hyperlink{09_Chapter_Two__THE_CRAVING_FOR_A_M.xhtmlux5cux23page_51}{51}

Charles the Bold (duke of Burgundy): bombardment of Granson,
\protect\hyperlink{11_Chapter_Four__THE_FORMS_OF_LOVE.xhtmlux5cux23page_153}{153};
camp near Neuss,
\protect\hyperlink{10_Chapter_Three__THE_HEROIC_DREAM.xhtmlux5cux23page_114}{114},
\protect\hyperlink{18_Chapter_Eleven__THE_FORMS_OF_THO.xhtmlux5cux23page_286}{286};
court life,
\protect\hyperlink{09_Chapter_Two__THE_CRAVING_FOR_A_M.xhtmlux5cux23page_42}{42--}\protect\hyperlink{09_Chapter_Two__THE_CRAVING_FOR_A_M.xhtmlux5cux23page_45}{45};
at death of Philip the Good,
\protect\hyperlink{08_Chapter_One__THE_PASSIONATE_INTE.xhtmlux5cux23page_18}{18},
\protect\hyperlink{09_Chapter_Two__THE_CRAVING_FOR_A_M.xhtmlux5cux23page_54}{54--}\protect\hyperlink{09_Chapter_Two__THE_CRAVING_FOR_A_M.xhtmlux5cux23page_55}{55};
dress,
\protect\hyperlink{20_ILLUSTRATIONS_FOLLOW_PAGE.xhtmlux5cux23page_325}{325};
dwarves,
\protect\hyperlink{08_Chapter_One__THE_PASSIONATE_INTE.xhtmlux5cux23page_23}{23};
entry into Lille,
\protect\hyperlink{21_Chapter_Thirteen__IMAGE_AND_WORD.xhtmlux5cux23page_374}{374};
funeral,
\protect\hyperlink{22_Chapter_Fourteen__THE_COMING_OF.xhtmlux5cux23page_387}{387};
Golden Fleece,
\protect\hyperlink{10_Chapter_Three__THE_HEROIC_DREAM.xhtmlux5cux23page_94}{94},
\protect\hyperlink{10_Chapter_Three__THE_HEROIC_DREAM.xhtmlux5cux23page_96}{96};
\emph{guerre du bien public},
\protect\hyperlink{10_Chapter_Three__THE_HEROIC_DREAM.xhtmlux5cux23page_115}{115};
and Jacques Coeur,
\protect\hyperlink{10_Chapter_Three__THE_HEROIC_DREAM.xhtmlux5cux23page_104}{104};
jewels,
\protect\hyperlink{18_Chapter_Eleven__THE_FORMS_OF_THO.xhtmlux5cux23page_269}{269};
and knightly ideal,
\protect\hyperlink{08_Chapter_One__THE_PASSIONATE_INTE.xhtmlux5cux23page_28}{28},
\protect\hyperlink{09_Chapter_Two__THE_CRAVING_FOR_A_M.xhtmlux5cux23page_39}{39},
\protect\hyperlink{10_Chapter_Three__THE_HEROIC_DREAM.xhtmlux5cux23page_75}{75};
music,
\protect\hyperlink{20_ILLUSTRATIONS_FOLLOW_PAGE.xhtmlux5cux23page_306}{306},
\protect\hyperlink{20_ILLUSTRATIONS_FOLLOW_PAGE.xhtmlux5cux23page_314}{314},
\protect\hyperlink{20_ILLUSTRATIONS_FOLLOW_PAGE.xhtmlux5cux23page_323}{323};
order of the garter,
\protect\hyperlink{10_Chapter_Three__THE_HEROIC_DREAM.xhtmlux5cux23page_93}{93};
quarrel with Philip the Good,
\protect\hyperlink{21_Chapter_Thirteen__IMAGE_AND_WORD.xhtmlux5cux23page_343}{343--}\protect\hyperlink{21_Chapter_Thirteen__IMAGE_AND_WORD.xhtmlux5cux23page_346}{346};
Quintius Curtius translated for,
\protect\hyperlink{10_Chapter_Three__THE_HEROIC_DREAM.xhtmlux5cux23page_75}{75};
revenues cancelled,
\protect\hyperlink{08_Chapter_One__THE_PASSIONATE_INTE.xhtmlux5cux23page_9}{9};
and Saint Colette,
\protect\hyperlink{14_Chapter_Seven__THE_PIOUS_PERSONA.xhtmlux5cux23page_217}{217};
and Saint Sophia,
\protect\hyperlink{14_Chapter_Seven__THE_PIOUS_PERSONA.xhtmlux5cux23page_218}{218};
stubbornness,
\protect\hyperlink{08_Chapter_One__THE_PASSIONATE_INTE.xhtmlux5cux23page_25}{25};
virtue,
\protect\hyperlink{11_Chapter_Four__THE_FORMS_OF_LOVE.xhtmlux5cux23page_128}{128};
wasteful splendor,
\protect\hyperlink{20_ILLUSTRATIONS_FOLLOW_PAGE.xhtmlux5cux23page_302}{302}

Charney, Geoffroy de,
\protect\hyperlink{18_Chapter_Eleven__THE_FORMS_OF_THO.xhtmlux5cux23page_277}{277}

Charolais, Count of; \emph{See} Charles the Bold

Chartier, Alain: fame,
\protect\hyperlink{21_Chapter_Thirteen__IMAGE_AND_WORD.xhtmlux5cux23page_338}{338--}\protect\hyperlink{21_Chapter_Thirteen__IMAGE_AND_WORD.xhtmlux5cux23page_339}{339};
kissed by Marguereta of Scotland,
\protect\hyperlink{17_Chapter_Ten__THE_FAILURE_OF_IMAG.xhtmlux5cux23page_251}{251};
works: ``Ballade de Fougères,''
\protect\hyperlink{18_Chapter_Eleven__THE_FORMS_OF_THO.xhtmlux5cux23page_274}{274};
\emph{Le Curial},
\protect\hyperlink{10_Chapter_Three__THE_HEROIC_DREAM.xhtmlux5cux23page_124}{124};
\emph{Livre de quatre dames},
\protect\hyperlink{21_Chapter_Thirteen__IMAGE_AND_WORD.xhtmlux5cux23page_333}{333--}\protect\hyperlink{21_Chapter_Thirteen__IMAGE_AND_WORD.xhtmlux5cux23page_339}{339},
\protect\hyperlink{21_Chapter_Thirteen__IMAGE_AND_WORD.xhtmlux5cux23page_351}{351},
\protect\hyperlink{21_Chapter_Thirteen__IMAGE_AND_WORD.xhtmlux5cux23page_367}{367--}\protect\hyperlink{21_Chapter_Thirteen__IMAGE_AND_WORD.xhtmlux5cux23page_368}{368};
\emph{Quadriloge invectii,
\protect\hyperlink{10_Chapter_Three__THE_HEROIC_DREAM.xhtmlux5cux23page_66}{66--}\protect\hyperlink{10_Chapter_Three__THE_HEROIC_DREAM.xhtmlux5cux23page_67}{67}}

Chastellain, George: allegory,
\protect\hyperlink{21_Chapter_Thirteen__IMAGE_AND_WORD.xhtmlux5cux23page_377}{377--}\protect\hyperlink{21_Chapter_Thirteen__IMAGE_AND_WORD.xhtmlux5cux23page_378}{378},
\protect\hyperlink{21_Chapter_Thirteen__IMAGE_AND_WORD.xhtmlux5cux23page_379}{379};
bells,
\protect\hyperlink{08_Chapter_One__THE_PASSIONATE_INTE.xhtmlux5cux23page_2}{2};
bourgeois honor,
\protect\hyperlink{10_Chapter_Three__THE_HEROIC_DREAM.xhtmlux5cux23page_116}{116};
and Charles the Bold,
\protect\hyperlink{08_Chapter_One__THE_PASSIONATE_INTE.xhtmlux5cux23page_9}{9--}\protect\hyperlink{08_Chapter_One__THE_PASSIONATE_INTE.xhtmlux5cux23page_10}{10},
\protect\hyperlink{21_Chapter_Thirteen__IMAGE_AND_WORD.xhtmlux5cux23page_343}{343--}\protect\hyperlink{21_Chapter_Thirteen__IMAGE_AND_WORD.xhtmlux5cux23page_346}{346};
compared to Breugel,
\protect\hyperlink{21_Chapter_Thirteen__IMAGE_AND_WORD.xhtmlux5cux23page_364}{364};
compared to van Eyck,
\protect\hyperlink{21_Chapter_Thirteen__IMAGE_AND_WORD.xhtmlux5cux23page_342}{342};
court ritual,
\protect\hyperlink{09_Chapter_Two__THE_CRAVING_FOR_A_M.xhtmlux5cux23page_42}{42--}\protect\hyperlink{09_Chapter_Two__THE_CRAVING_FOR_A_M.xhtmlux5cux23page_43}{43},
\protect\hyperlink{09_Chapter_Two__THE_CRAVING_FOR_A_M.xhtmlux5cux23page_46}{46},
\protect\hyperlink{09_Chapter_Two__THE_CRAVING_FOR_A_M.xhtmlux5cux23page_47}{47};
Crusade,
\protect\hyperlink{10_Chapter_Three__THE_HEROIC_DREAM.xhtmlux5cux23page_106}{106};
De Barant's source,
\protect\hyperlink{19_Chapter_Twelve__ART_IN_LIFE.xhtmlux5cux23page_294}{294};
description of nature,
\protect\hyperlink{21_Chapter_Thirteen__IMAGE_AND_WORD.xhtmlux5cux23page_351}{351};
duel of Plouvier and Mafout,
\protect\hyperlink{10_Chapter_Three__THE_HEROIC_DREAM.xhtmlux5cux23page_109}{109--}\protect\hyperlink{10_Chapter_Three__THE_HEROIC_DREAM.xhtmlux5cux23page_111}{111};
errors,
\protect\hyperlink{22_Chapter_Fourteen__THE_COMING_OF.xhtmlux5cux23page_386}{386};
exaggerations,
\protect\hyperlink{18_Chapter_Eleven__THE_FORMS_OF_THO.xhtmlux5cux23page_282}{282};
French loyalty,
\protect\hyperlink{10_Chapter_Three__THE_HEROIC_DREAM.xhtmlux5cux23page_77}{77};
Golden Fleece,
\protect\hyperlink{10_Chapter_Three__THE_HEROIC_DREAM.xhtmlux5cux23page_93}{93};
greed,
\protect\hyperlink{10_Chapter_Three__THE_HEROIC_DREAM.xhtmlux5cux23page_117}{117};
King René's \emph{pas d'armes},
\protect\hyperlink{11_Chapter_Four__THE_FORMS_OF_LOVE.xhtmlux5cux23page_151}{151--}\protect\hyperlink{11_Chapter_Four__THE_FORMS_OF_LOVE.xhtmlux5cux23page_152}{152};
Louis XI weeps,
\protect\hyperlink{08_Chapter_One__THE_PASSIONATE_INTE.xhtmlux5cux23page_8}{8};
\emph{manière},
\protect\hyperlink{22_Chapter_Fourteen__THE_COMING_OF.xhtmlux5cux23page_388}{388};
mignon,
\protect\hyperlink{09_Chapter_Two__THE_CRAVING_FOR_A_M.xhtmlux5cux23page_58}{58--}\protect\hyperlink{09_Chapter_Two__THE_CRAVING_FOR_A_M.xhtmlux5cux23page_59}{59};
misfortunes of Margaret of Anjou,
\protect\hyperlink{08_Chapter_One__THE_PASSIONATE_INTE.xhtmlux5cux23page_14}{14};
on nature of princes,
\protect\hyperlink{08_Chapter_One__THE_PASSIONATE_INTE.xhtmlux5cux23page_15}{15},
\protect\hyperlink{13_Chapter_Six__THE_DEPICTION_OF_TH.xhtmlux5cux23page_183}{183--}\protect\hyperlink{13_Chapter_Six__THE_DEPICTION_OF_TH.xhtmlux5cux23page_184}{184};
paganism,
\protect\hyperlink{22_Chapter_Fourteen__THE_COMING_OF.xhtmlux5cux23page_394}{394};
``pas de mort,''
\protect\hyperlink{12_Chapter_Five__THE_VISION_OF_DEAT.xhtmlux5cux23page_167}{167--}\protect\hyperlink{12_Chapter_Five__THE_VISION_OF_DEAT.xhtmlux5cux23page_168}{168};
Philip and the dauphin,
\protect\hyperlink{09_Chapter_Two__THE_CRAVING_FOR_A_M.xhtmlux5cux23page_46}{46};
Philip and quarrel with Charles the Bold,
\protect\hyperlink{21_Chapter_Thirteen__IMAGE_AND_WORD.xhtmlux5cux23page_343}{343--}\protect\hyperlink{21_Chapter_Thirteen__IMAGE_AND_WORD.xhtmlux5cux23page_346}{346};
political symbolism,
\protect\hyperlink{16_Chapter_Nine__THE_DECLINE_OF_SYM.xhtmlux5cux23page_242}{242};
princes and bad news,
\protect\hyperlink{09_Chapter_Two__THE_CRAVING_FOR_A_M.xhtmlux5cux23page_54}{54--}\protect\hyperlink{09_Chapter_Two__THE_CRAVING_FOR_A_M.xhtmlux5cux23page_56}{56};
Robertet, correspondence with,
\protect\hyperlink{22_Chapter_Fourteen__THE_COMING_OF.xhtmlux5cux23page_389}{389--}\protect\hyperlink{22_Chapter_Fourteen__THE_COMING_OF.xhtmlux5cux23page_392}{392};
on Rolin,
\protect\hyperlink{20_ILLUSTRATIONS_FOLLOW_PAGE.xhtmlux5cux23page_317}{317};
style,
\protect\hyperlink{21_Chapter_Thirteen__IMAGE_AND_WORD.xhtmlux5cux23page_342}{342--}\protect\hyperlink{21_Chapter_Thirteen__IMAGE_AND_WORD.xhtmlux5cux23page_343}{343},
\protect\hyperlink{21_Chapter_Thirteen__IMAGE_AND_WORD.xhtmlux5cux23page_364}{364};
superstition,
\protect\hyperlink{18_Chapter_Eleven__THE_FORMS_OF_THO.xhtmlux5cux23page_288}{288};
view of society,
\protect\hyperlink{10_Chapter_Three__THE_HEROIC_DREAM.xhtmlux5cux23page_63}{63--}\protect\hyperlink{10_Chapter_Three__THE_HEROIC_DREAM.xhtmlux5cux23page_65}{65};
world weariness,
\protect\hyperlink{09_Chapter_Two__THE_CRAVING_FOR_A_M.xhtmlux5cux23page_34}{34};
works: \emph{Temple de Bocace},
\protect\hyperlink{22_Chapter_Fourteen__THE_COMING_OF.xhtmlux5cux23page_386}{386}

Chaucer, Geoffrey,
\protect\hyperlink{22_Chapter_Fourteen__THE_COMING_OF.xhtmlux5cux23page_387}{387}

Chevalier, Etienne,
\protect\hyperlink{13_Chapter_Six__THE_DEPICTION_OF_TH.xhtmlux5cux23page_182}{182}

\emph{Chevaliers Nostre Dame de la Noble Maison. See} Stars, Order of
the

Chevrot, Jean (bishop of Tourney): collection for crusade,
\protect\hyperlink{10_Chapter_Three__THE_HEROIC_DREAM.xhtmlux5cux23page_107}{107};
patron of \emph{Seven Sacraments} (van der Weyden),
\protect\hyperlink{20_ILLUSTRATIONS_FOLLOW_PAGE.xhtmlux5cux23page_306}{306--}\protect\hyperlink{20_ILLUSTRATIONS_FOLLOW_PAGE.xhtmlux5cux23page_307}{307},
\protect\hyperlink{20_ILLUSTRATIONS_FOLLOW_PAGE.xhtmlux5cux23page_316}{316}

childbirth customs,
\protect\hyperlink{09_Chapter_Two__THE_CRAVING_FOR_A_M.xhtmlux5cux23page_57}{57--}\protect\hyperlink{09_Chapter_Two__THE_CRAVING_FOR_A_M.xhtmlux5cux23page_58}{58}

Chopinel, Jean,
\protect\hyperlink{11_Chapter_Four__THE_FORMS_OF_LOVE.xhtmlux5cux23page_127}{127}

Christopher, Saint,
\protect\hyperlink{13_Chapter_Six__THE_DEPICTION_OF_TH.xhtmlux5cux23page_197}{197},
\protect\hyperlink{13_Chapter_Six__THE_DEPICTION_OF_TH.xhtmlux5cux23page_198}{198},
\protect\hyperlink{16_Chapter_Nine__THE_DECLINE_OF_SYM.xhtmlux5cux23page_246}{246}

Cicero: ideal of equality,
\protect\hyperlink{10_Chapter_Three__THE_HEROIC_DREAM.xhtmlux5cux23page_68}{68};
Jean de Montreuil imitates,
\protect\hyperlink{11_Chapter_Four__THE_FORMS_OF_LOVE.xhtmlux5cux23page_138}{138},
\protect\hyperlink{22_Chapter_Fourteen__THE_COMING_OF.xhtmlux5cux23page_384}{384};
and tournaments,
\protect\hyperlink{10_Chapter_Three__THE_HEROIC_DREAM.xhtmlux5cux23page_88}{88}

\protect\hypertarget{25_INDEX.xhtmlux5cux23page_455}{}{}Claiquin,
Bertram de. \emph{See} Guesclin, Bertrand du

\emph{Claris mulieribus, De} (Boccaccio),
\protect\hyperlink{22_Chapter_Fourteen__THE_COMING_OF.xhtmlux5cux23page_385}{385--}\protect\hyperlink{22_Chapter_Fourteen__THE_COMING_OF.xhtmlux5cux23page_386}{386}

Clémanges, Nicolas de: critique of courtly life,
\protect\hyperlink{10_Chapter_Three__THE_HEROIC_DREAM.xhtmlux5cux23page_124}{124};
humanist letters,
\protect\hyperlink{22_Chapter_Fourteen__THE_COMING_OF.xhtmlux5cux23page_383}{383};
on pilgrimages,
\protect\hyperlink{13_Chapter_Six__THE_DEPICTION_OF_TH.xhtmlux5cux23page_185}{185};
Petrarch,
\protect\hyperlink{22_Chapter_Fourteen__THE_COMING_OF.xhtmlux5cux23page_385}{385};
works: \emph{Liber de lapsu et reparatione},
\protect\hyperlink{10_Chapter_Three__THE_HEROIC_DREAM.xhtmlux5cux23page_66}{66}

Clement VI, pope,
\protect\hyperlink{17_Chapter_Ten__THE_FAILURE_OF_IMAG.xhtmlux5cux23page_255}{255}

Clercq, Jacques du: dark chronicle,
\protect\hyperlink{08_Chapter_One__THE_PASSIONATE_INTE.xhtmlux5cux23page_29}{29};
nobles refuse sacrament,
\protect\hyperlink{13_Chapter_Six__THE_DEPICTION_OF_TH.xhtmlux5cux23page_189}{189},
\protect\hyperlink{18_Chapter_Eleven__THE_FORMS_OF_THO.xhtmlux5cux23page_280}{280}

Clermont, castle of,
\protect\hyperlink{21_Chapter_Thirteen__IMAGE_AND_WORD.xhtmlux5cux23page_352}{352}

Clopinel, Jean. \emph{See} Chopinel, Jean

clothing,
\protect\hyperlink{20_ILLUSTRATIONS_FOLLOW_PAGE.xhtmlux5cux23page_325}{325--}\protect\hyperlink{20_ILLUSTRATIONS_FOLLOW_PAGE.xhtmlux5cux23page_328}{328}.
See also \emph{poulaines}

Coeur, Jacques: chivalry,
\protect\hyperlink{10_Chapter_Three__THE_HEROIC_DREAM.xhtmlux5cux23page_104}{104};
in \emph{Temple de Bocace},
\protect\hyperlink{10_Chapter_Three__THE_HEROIC_DREAM.xhtmlux5cux23page_65}{65}

Coimbra, John of,
\protect\hyperlink{08_Chapter_One__THE_PASSIONATE_INTE.xhtmlux5cux23page_8}{8}

Coitier, Jacques,
\protect\hyperlink{14_Chapter_Seven__THE_PIOUS_PERSONA.xhtmlux5cux23page_216}{216}

Col, Gontier and Pierre: attack on Gerson,
\protect\hyperlink{11_Chapter_Four__THE_FORMS_OF_LOVE.xhtmlux5cux23page_138}{138};
defense of \emph{Roman de la rose},
\protect\hyperlink{11_Chapter_Four__THE_FORMS_OF_LOVE.xhtmlux5cux23page_137}{137--}\protect\hyperlink{11_Chapter_Four__THE_FORMS_OF_LOVE.xhtmlux5cux23page_138}{138};
G. warned against court service,
\protect\hyperlink{10_Chapter_Three__THE_HEROIC_DREAM.xhtmlux5cux23page_124}{124};
humanist letters,
\protect\hyperlink{22_Chapter_Fourteen__THE_COMING_OF.xhtmlux5cux23page_383}{383--}\protect\hyperlink{22_Chapter_Fourteen__THE_COMING_OF.xhtmlux5cux23page_384}{384};
membership in \emph{Cour d'amours},
\protect\hyperlink{11_Chapter_Four__THE_FORMS_OF_LOVE.xhtmlux5cux23page_141}{141}

Colette, Saint: adviser to court of Burgundy,
\protect\hyperlink{14_Chapter_Seven__THE_PIOUS_PERSONA.xhtmlux5cux23page_217}{217},
\protect\hyperlink{20_ILLUSTRATIONS_FOLLOW_PAGE.xhtmlux5cux23page_314}{314};
art of van Eyck and,
\protect\hyperlink{20_ILLUSTRATIONS_FOLLOW_PAGE.xhtmlux5cux23page_314}{314};
influences Jacques de Bourbon,
\protect\hyperlink{14_Chapter_Seven__THE_PIOUS_PERSONA.xhtmlux5cux23page_209}{209};
and the Passion,
\protect\hyperlink{15_Chapter_Eight__RELIGIOUS_EXCITAT.xhtmlux5cux23page_220}{220};
theopathic state,
\protect\hyperlink{15_Chapter_Eight__RELIGIOUS_EXCITAT.xhtmlux5cux23page_226}{226};
type of saint,
\protect\hyperlink{14_Chapter_Seven__THE_PIOUS_PERSONA.xhtmlux5cux23page_210}{210}

Cologne, Hermann of,
\protect\hyperlink{20_ILLUSTRATIONS_FOLLOW_PAGE.xhtmlux5cux23page_310}{310}

\emph{Combat des trente}: agreement for,
\protect\hyperlink{10_Chapter_Three__THE_HEROIC_DREAM.xhtmlux5cux23page_74}{74};
Froissart on,
\protect\hyperlink{10_Chapter_Three__THE_HEROIC_DREAM.xhtmlux5cux23page_112}{112};
Crokart,
\protect\hyperlink{10_Chapter_Three__THE_HEROIC_DREAM.xhtmlux5cux23page_118}{118}

Commines, Philippe de,
\protect\hyperlink{21_Chapter_Thirteen__IMAGE_AND_WORD.xhtmlux5cux23page_349}{349},
\protect\hyperlink{22_Chapter_Fourteen__THE_COMING_OF.xhtmlux5cux23page_391}{391};
on Charles the Bold,
\protect\hyperlink{10_Chapter_Three__THE_HEROIC_DREAM.xhtmlux5cux23page_75}{75--}\protect\hyperlink{10_Chapter_Three__THE_HEROIC_DREAM.xhtmlux5cux23page_76}{76},
\protect\hyperlink{10_Chapter_Three__THE_HEROIC_DREAM.xhtmlux5cux23page_115}{115},
\protect\hyperlink{10_Chapter_Three__THE_HEROIC_DREAM.xhtmlux5cux23page_117}{117};
contempt for knighthood,
\protect\hyperlink{10_Chapter_Three__THE_HEROIC_DREAM.xhtmlux5cux23page_117}{117},
\protect\hyperlink{10_Chapter_Three__THE_HEROIC_DREAM.xhtmlux5cux23page_120}{120};
does not exaggerate,
\protect\hyperlink{18_Chapter_Eleven__THE_FORMS_OF_THO.xhtmlux5cux23page_283}{283};
inaccuracies,
\protect\hyperlink{18_Chapter_Eleven__THE_FORMS_OF_THO.xhtmlux5cux23page_283}{283};
mignon,
\protect\hyperlink{09_Chapter_Two__THE_CRAVING_FOR_A_M.xhtmlux5cux23page_59}{59},
\protect\hyperlink{10_Chapter_Three__THE_HEROIC_DREAM.xhtmlux5cux23page_72}{72};
and Saint Francis of Paula,
\protect\hyperlink{14_Chapter_Seven__THE_PIOUS_PERSONA.xhtmlux5cux23page_216}{216--}\protect\hyperlink{14_Chapter_Seven__THE_PIOUS_PERSONA.xhtmlux5cux23page_217}{217};
sober mind,
\protect\hyperlink{10_Chapter_Three__THE_HEROIC_DREAM.xhtmlux5cux23page_118}{118}

``Complaincte de Eco'' (Guillaume Coquillart),
\protect\hyperlink{18_Chapter_Eleven__THE_FORMS_OF_THO.xhtmlux5cux23page_274}{274}

``Contrediz Franz Gontier'' (Villon),
\protect\hyperlink{11_Chapter_Four__THE_FORMS_OF_LOVE.xhtmlux5cux23page_154}{154}

Constance, Council of,
\protect\hyperlink{15_Chapter_Eight__RELIGIOUS_EXCITAT.xhtmlux5cux23page_223}{223}

\emph{Contemptu mundi, De} (Innocent III),
\protect\hyperlink{12_Chapter_Five__THE_VISION_OF_DEAT.xhtmlux5cux23page_160}{160},
\protect\hyperlink{17_Chapter_Ten__THE_FAILURE_OF_IMAG.xhtmlux5cux23page_254}{254},
\protect\hyperlink{22_Chapter_Fourteen__THE_COMING_OF.xhtmlux5cux23page_385}{385}

\emph{Contra peregrinantes} (Frederick van Helio),
\protect\hyperlink{13_Chapter_Six__THE_DEPICTION_OF_TH.xhtmlux5cux23page_186}{186}

\emph{Convivio} (Dante),
\protect\hyperlink{09_Chapter_Two__THE_CRAVING_FOR_A_M.xhtmlux5cux23page_36}{36}

Coquillart, Guillaume,
\protect\hyperlink{11_Chapter_Four__THE_FORMS_OF_LOVE.xhtmlux5cux23page_142}{142};
``Complaincte de Eco,''
\protect\hyperlink{18_Chapter_Eleven__THE_FORMS_OF_THO.xhtmlux5cux23page_274}{274};
modern,
\protect\hyperlink{22_Chapter_Fourteen__THE_COMING_OF.xhtmlux5cux23page_389}{389}

Coucy, castle of,
\protect\hyperlink{10_Chapter_Three__THE_HEROIC_DREAM.xhtmlux5cux23page_77}{77},
\protect\hyperlink{21_Chapter_Thirteen__IMAGE_AND_WORD.xhtmlux5cux23page_352}{352}

Coucy, Enguerrand de,
\protect\hyperlink{10_Chapter_Three__THE_HEROIC_DREAM.xhtmlux5cux23page_94}{94},
\protect\hyperlink{10_Chapter_Three__THE_HEROIC_DREAM.xhtmlux5cux23page_111}{111}

Count of Nevers. See John the Fearless

Courtenay, Pierre de,
\protect\hyperlink{10_Chapter_Three__THE_HEROIC_DREAM.xhtmlux5cux23page_112}{112}

Coustain, Jean,
\protect\hyperlink{14_Chapter_Seven__THE_PIOUS_PERSONA.xhtmlux5cux23page_207}{207}

Cranach, Lucas,
\protect\hyperlink{21_Chapter_Thirteen__IMAGE_AND_WORD.xhtmlux5cux23page_373}{373}

Craon, Pierre de: confession for felons,
\protect\hyperlink{08_Chapter_One__THE_PASSIONATE_INTE.xhtmlux5cux23page_21}{21};
mignon,
\protect\hyperlink{09_Chapter_Two__THE_CRAVING_FOR_A_M.xhtmlux5cux23page_59}{59}

Crokart,
\protect\hyperlink{10_Chapter_Three__THE_HEROIC_DREAM.xhtmlux5cux23page_118}{118}

Croy: Antoine de,
\protect\hyperlink{20_ILLUSTRATIONS_FOLLOW_PAGE.xhtmlux5cux23page_305}{305};
family,
\protect\hyperlink{20_ILLUSTRATIONS_FOLLOW_PAGE.xhtmlux5cux23page_316}{316},
\protect\hyperlink{20_ILLUSTRATIONS_FOLLOW_PAGE.xhtmlux5cux23page_325}{325},
\protect\hyperlink{21_Chapter_Thirteen__IMAGE_AND_WORD.xhtmlux5cux23page_343}{343};
Philippe de,
\protect\hyperlink{10_Chapter_Three__THE_HEROIC_DREAM.xhtmlux5cux23page_87}{87}

\emph{Cuer d'amours espris} (King René),
\protect\hyperlink{21_Chapter_Thirteen__IMAGE_AND_WORD.xhtmlux5cux23page_347}{347},
\protect\hyperlink{21_Chapter_Thirteen__IMAGE_AND_WORD.xhtmlux5cux23page_368}{368}

\emph{Curial, Le} (Alain Chartier),
\protect\hyperlink{10_Chapter_Three__THE_HEROIC_DREAM.xhtmlux5cux23page_124}{124},
\protect\hyperlink{22_Chapter_Fourteen__THE_COMING_OF.xhtmlux5cux23page_392}{392}

Cyriac, Saint,
\protect\hyperlink{13_Chapter_Six__THE_DEPICTION_OF_TH.xhtmlux5cux23page_198}{198}

Damianus, Saint,
\protect\hyperlink{13_Chapter_Six__THE_DEPICTION_OF_TH.xhtmlux5cux23page_200}{200}

``Danse aux Aveugles'' (Pierre Michault),
\protect\hyperlink{21_Chapter_Thirteen__IMAGE_AND_WORD.xhtmlux5cux23page_361}{361}

\emph{danse macabre},
\protect\hyperlink{08_Chapter_One__THE_PASSIONATE_INTE.xhtmlux5cux23page_4}{4},
\protect\hyperlink{08_Chapter_One__THE_PASSIONATE_INTE.xhtmlux5cux23page_5}{5},
\protect\hyperlink{10_Chapter_Three__THE_HEROIC_DREAM.xhtmlux5cux23page_68}{68},
\protect\hyperlink{12_Chapter_Five__THE_VISION_OF_DEAT.xhtmlux5cux23page_157}{157},
\protect\hyperlink{12_Chapter_Five__THE_VISION_OF_DEAT.xhtmlux5cux23page_172}{172};
in art,
\protect\hyperlink{12_Chapter_Five__THE_VISION_OF_DEAT.xhtmlux5cux23page_165}{165};
Cemetery of the Innocents,
\protect\hyperlink{12_Chapter_Five__THE_VISION_OF_DEAT.xhtmlux5cux23page_166}{166},
\protect\hyperlink{12_Chapter_Five__THE_VISION_OF_DEAT.xhtmlux5cux23page_170}{170};
corpse in,
\protect\hyperlink{12_Chapter_Five__THE_VISION_OF_DEAT.xhtmlux5cux23page_166}{166};
etymology,
\protect\hyperlink{12_Chapter_Five__THE_VISION_OF_DEAT.xhtmlux5cux23page_164}{164};
Guyot Marchant,
\protect\hyperlink{12_Chapter_Five__THE_VISION_OF_DEAT.xhtmlux5cux23page_165}{165},
\protect\hyperlink{12_Chapter_Five__THE_VISION_OF_DEAT.xhtmlux5cux23page_167}{167};
La Chaise-Dieu,
\protect\hyperlink{12_Chapter_Five__THE_VISION_OF_DEAT.xhtmlux5cux23page_165}{165};
and literature,
\protect\hyperlink{12_Chapter_Five__THE_VISION_OF_DEAT.xhtmlux5cux23page_168}{168};
Martial d'Auvergne; of women,
\protect\hyperlink{12_Chapter_Five__THE_VISION_OF_DEAT.xhtmlux5cux23page_167}{167},
\protect\hyperlink{12_Chapter_Five__THE_VISION_OF_DEAT.xhtmlux5cux23page_171}{171};
related images,
\protect\hyperlink{12_Chapter_Five__THE_VISION_OF_DEAT.xhtmlux5cux23page_164}{164};
verses,
\protect\hyperlink{12_Chapter_Five__THE_VISION_OF_DEAT.xhtmlux5cux23page_171}{171}

Dante: curses greed,
\protect\hyperlink{08_Chapter_One__THE_PASSIONATE_INTE.xhtmlux5cux23page_26}{26};
\emph{Dolce stil nuovo},
\protect\hyperlink{11_Chapter_Four__THE_FORMS_OF_LOVE.xhtmlux5cux23page_126}{126},
\protect\hyperlink{11_Chapter_Four__THE_FORMS_OF_LOVE.xhtmlux5cux23page_131}{131};
dreams in,
\protect\hyperlink{21_Chapter_Thirteen__IMAGE_AND_WORD.xhtmlux5cux23page_379}{379};
hell,
\protect\hyperlink{17_Chapter_Ten__THE_FAILURE_OF_IMAG.xhtmlux5cux23page_252}{252};
honor,
\protect\hyperlink{10_Chapter_Three__THE_HEROIC_DREAM.xhtmlux5cux23page_74}{74};
literature of love,
\protect\hyperlink{11_Chapter_Four__THE_FORMS_OF_LOVE.xhtmlux5cux23page_126}{126},
\protect\hyperlink{21_Chapter_Thirteen__IMAGE_AND_WORD.xhtmlux5cux23page_372}{372};
Farinata and Ugolino,
\protect\hyperlink{17_Chapter_Ten__THE_FAILURE_OF_IMAG.xhtmlux5cux23page_252}{252};
works: \emph{Convivio},
\protect\hyperlink{09_Chapter_Two__THE_CRAVING_FOR_A_M.xhtmlux5cux23page_36}{36};
De \emph{monarchia},
\protect\hyperlink{16_Chapter_Nine__THE_DECLINE_OF_SYM.xhtmlux5cux23page_247}{247};
\emph{Vita nuova},
\protect\hyperlink{11_Chapter_Four__THE_FORMS_OF_LOVE.xhtmlux5cux23page_126}{126--}\protect\hyperlink{11_Chapter_Four__THE_FORMS_OF_LOVE.xhtmlux5cux23page_127}{127}

David: exemplar of knighthood,
\protect\hyperlink{10_Chapter_Three__THE_HEROIC_DREAM.xhtmlux5cux23page_75}{75},
\protect\hyperlink{10_Chapter_Three__THE_HEROIC_DREAM.xhtmlux5cux23page_76}{76},
\protect\hyperlink{10_Chapter_Three__THE_HEROIC_DREAM.xhtmlux5cux23page_96}{96};
in \emph{Annunciation} (van Eyck),
\protect\hyperlink{21_Chapter_Thirteen__IMAGE_AND_WORD.xhtmlux5cux23page_336}{336};
in Brugmann,
\protect\hyperlink{15_Chapter_Eight__RELIGIOUS_EXCITAT.xhtmlux5cux23page_230}{230};
on Moses Fountain (Sluter),
\protect\hyperlink{20_ILLUSTRATIONS_FOLLOW_PAGE.xhtmlux5cux23page_309}{309};
\emph{Psalms},
\protect\hyperlink{20_ILLUSTRATIONS_FOLLOW_PAGE.xhtmlux5cux23page_309}{309}

David (bishop of Utrecht),
\protect\hyperlink{20_ILLUSTRATIONS_FOLLOW_PAGE.xhtmlux5cux23page_314}{314}

\protect\hypertarget{25_INDEX.xhtmlux5cux23page_456}{}{}David, Gerard:
\emph{Judgement of Cambyses},
\protect\hyperlink{19_Chapter_Twelve__ART_IN_LIFE.xhtmlux5cux23page_297}{297};
judicial scenes,
\protect\hyperlink{21_Chapter_Thirteen__IMAGE_AND_WORD.xhtmlux5cux23page_376}{376};
paints \emph{Broodhuis},
\protect\hyperlink{20_ILLUSTRATIONS_FOLLOW_PAGE.xhtmlux5cux23page_299}{299};
sense of color,
\protect\hyperlink{20_ILLUSTRATIONS_FOLLOW_PAGE.xhtmlux5cux23page_328}{328}

``Débat dou cheval et dou levrier, Le'' (Froissart),
\protect\hyperlink{21_Chapter_Thirteen__IMAGE_AND_WORD.xhtmlux5cux23page_360}{360}

\emph{Débat des hérauts d'armes de France et d'Angleterre},
\protect\hyperlink{10_Chapter_Three__THE_HEROIC_DREAM.xhtmlux5cux23page_116}{116}

\emph{Débat du laboureur, du prestre, et du gendarme} (Robert Gaguin),
\protect\hyperlink{10_Chapter_Three__THE_HEROIC_DREAM.xhtmlux5cux23page_67}{67},
\protect\hyperlink{22_Chapter_Fourteen__THE_COMING_OF.xhtmlux5cux23page_393}{393}

\emph{Decameron} (Boccaccio),
\protect\hyperlink{22_Chapter_Fourteen__THE_COMING_OF.xhtmlux5cux23page_385}{385}

Denis, Saint,
\protect\hyperlink{13_Chapter_Six__THE_DEPICTION_OF_TH.xhtmlux5cux23page_198}{198}

Denis the Carthusian, Saint: and art of van Eycks,
\protect\hyperlink{20_ILLUSTRATIONS_FOLLOW_PAGE.xhtmlux5cux23page_319}{319};
on beauty,
\protect\hyperlink{20_ILLUSTRATIONS_FOLLOW_PAGE.xhtmlux5cux23page_321}{321};
and blasphemy,
\protect\hyperlink{13_Chapter_Six__THE_DEPICTION_OF_TH.xhtmlux5cux23page_184}{184};
classical style,
\protect\hyperlink{22_Chapter_Fourteen__THE_COMING_OF.xhtmlux5cux23page_385}{385};
on death,
\protect\hyperlink{12_Chapter_Five__THE_VISION_OF_DEAT.xhtmlux5cux23page_156}{156},
\protect\hyperlink{12_Chapter_Five__THE_VISION_OF_DEAT.xhtmlux5cux23page_158}{158};
holds name of Jesus aloft,
\protect\hyperlink{16_Chapter_Nine__THE_DECLINE_OF_SYM.xhtmlux5cux23page_234}{234};
Luther on,
\protect\hyperlink{16_Chapter_Nine__THE_DECLINE_OF_SYM.xhtmlux5cux23page_248}{248};
mystic expression,
\protect\hyperlink{17_Chapter_Ten__THE_FAILURE_OF_IMAG.xhtmlux5cux23page_256}{256},
\protect\hyperlink{17_Chapter_Ten__THE_FAILURE_OF_IMAG.xhtmlux5cux23page_258}{258};
personalization of expression,
\protect\hyperlink{16_Chapter_Nine__THE_DECLINE_OF_SYM.xhtmlux5cux23page_243}{243};
political adviser,
\protect\hyperlink{08_Chapter_One__THE_PASSIONATE_INTE.xhtmlux5cux23page_12}{12},
\protect\hyperlink{14_Chapter_Seven__THE_PIOUS_PERSONA.xhtmlux5cux23page_217}{217},
\protect\hyperlink{20_ILLUSTRATIONS_FOLLOW_PAGE.xhtmlux5cux23page_314}{314};
scholastic form,
\protect\hyperlink{17_Chapter_Ten__THE_FAILURE_OF_IMAG.xhtmlux5cux23page_250}{250};
type of saint,
\protect\hyperlink{14_Chapter_Seven__THE_PIOUS_PERSONA.xhtmlux5cux23page_211}{211},
\protect\hyperlink{14_Chapter_Seven__THE_PIOUS_PERSONA.xhtmlux5cux23page_218}{218},
\protect\hyperlink{14_Chapter_Seven__THE_PIOUS_PERSONA.xhtmlux5cux23page_219}{219},
\protect\hyperlink{17_Chapter_Ten__THE_FAILURE_OF_IMAG.xhtmlux5cux23page_265}{265};
and witchcraft,
\protect\hyperlink{18_Chapter_Eleven__THE_FORMS_OF_THO.xhtmlux5cux23page_292}{292};
works: \emph{De doctrina et regulis},
\protect\hyperlink{17_Chapter_Ten__THE_FAILURE_OF_IMAG.xhtmlux5cux23page_250}{250};
\emph{De vita et regimine},
\protect\hyperlink{17_Chapter_Ten__THE_FAILURE_OF_IMAG.xhtmlux5cux23page_250}{250}

Deschamps, Eustache,
\protect\hyperlink{21_Chapter_Thirteen__IMAGE_AND_WORD.xhtmlux5cux23page_363}{363};
allegory,
\protect\hyperlink{18_Chapter_Eleven__THE_FORMS_OF_THO.xhtmlux5cux23page_284}{284--}\protect\hyperlink{18_Chapter_Eleven__THE_FORMS_OF_THO.xhtmlux5cux23page_285}{285};
beauty,
\protect\hyperlink{20_ILLUSTRATIONS_FOLLOW_PAGE.xhtmlux5cux23page_324}{324},
\protect\hyperlink{21_Chapter_Thirteen__IMAGE_AND_WORD.xhtmlux5cux23page_362}{362};
beggars,
\protect\hyperlink{13_Chapter_Six__THE_DEPICTION_OF_TH.xhtmlux5cux23page_199}{199},
\protect\hyperlink{21_Chapter_Thirteen__IMAGE_AND_WORD.xhtmlux5cux23page_365}{365};
classicism,
\protect\hyperlink{22_Chapter_Fourteen__THE_COMING_OF.xhtmlux5cux23page_387}{387},
\protect\hyperlink{22_Chapter_Fourteen__THE_COMING_OF.xhtmlux5cux23page_393}{393};
color symbolism,
\protect\hyperlink{20_ILLUSTRATIONS_FOLLOW_PAGE.xhtmlux5cux23page_326}{326};
compared to Limburg brothers,
\protect\hyperlink{21_Chapter_Thirteen__IMAGE_AND_WORD.xhtmlux5cux23page_352}{352},
\protect\hyperlink{21_Chapter_Thirteen__IMAGE_AND_WORD.xhtmlux5cux23page_362}{362};
contradictory,
\protect\hyperlink{14_Chapter_Seven__THE_PIOUS_PERSONA.xhtmlux5cux23page_208}{208};
equality in death,
\protect\hyperlink{10_Chapter_Three__THE_HEROIC_DREAM.xhtmlux5cux23page_68}{68};
former splendor theme,
\protect\hyperlink{12_Chapter_Five__THE_VISION_OF_DEAT.xhtmlux5cux23page_158}{158};
hypocrisy,
\protect\hyperlink{11_Chapter_Four__THE_FORMS_OF_LOVE.xhtmlux5cux23page_154}{154};
images of saints,
\protect\hyperlink{13_Chapter_Six__THE_DEPICTION_OF_TH.xhtmlux5cux23page_199}{199--}\protect\hyperlink{13_Chapter_Six__THE_DEPICTION_OF_TH.xhtmlux5cux23page_200}{200};
impotent forms,
\protect\hyperlink{21_Chapter_Thirteen__IMAGE_AND_WORD.xhtmlux5cux23page_354}{354};
irony,
\protect\hyperlink{21_Chapter_Thirteen__IMAGE_AND_WORD.xhtmlux5cux23page_366}{366},
\protect\hyperlink{21_Chapter_Thirteen__IMAGE_AND_WORD.xhtmlux5cux23page_379}{379};
marriage,
\protect\hyperlink{18_Chapter_Eleven__THE_FORMS_OF_THO.xhtmlux5cux23page_284}{284};
melancholy,
\protect\hyperlink{09_Chapter_Two__THE_CRAVING_FOR_A_M.xhtmlux5cux23page_34}{34};
miserable assertions about life,
\protect\hyperlink{09_Chapter_Two__THE_CRAVING_FOR_A_M.xhtmlux5cux23page_36}{36};
misery of court life,
\protect\hyperlink{09_Chapter_Two__THE_CRAVING_FOR_A_M.xhtmlux5cux23page_51}{51};
mocks knights,
\protect\hyperlink{10_Chapter_Three__THE_HEROIC_DREAM.xhtmlux5cux23page_116}{116};
Nine Worthies, the,
\protect\hyperlink{10_Chapter_Three__THE_HEROIC_DREAM.xhtmlux5cux23page_76}{76}
(see also \emph{\protect\hyperlink{25_INDEX.xhtmlux5cux23id_2270}{Preux,
Les Neuf})}; pastoral life,
\protect\hyperlink{10_Chapter_Three__THE_HEROIC_DREAM.xhtmlux5cux23page_122}{122--}\protect\hyperlink{10_Chapter_Three__THE_HEROIC_DREAM.xhtmlux5cux23page_124}{124};
realism and naturalism,
\protect\hyperlink{11_Chapter_Four__THE_FORMS_OF_LOVE.xhtmlux5cux23page_133}{133};
and Saint Joseph,
\protect\hyperlink{13_Chapter_Six__THE_DEPICTION_OF_TH.xhtmlux5cux23page_177}{177},
\protect\hyperlink{13_Chapter_Six__THE_DEPICTION_OF_TH.xhtmlux5cux23page_194}{194};
style,
\protect\hyperlink{21_Chapter_Thirteen__IMAGE_AND_WORD.xhtmlux5cux23page_357}{357},
\protect\hyperlink{21_Chapter_Thirteen__IMAGE_AND_WORD.xhtmlux5cux23page_366}{366};
and swearing,
\protect\hyperlink{13_Chapter_Six__THE_DEPICTION_OF_TH.xhtmlux5cux23page_186}{186};
works: ``Temps de doleur'',
\protect\hyperlink{09_Chapter_Two__THE_CRAVING_FOR_A_M.xhtmlux5cux23page_32}{32};
``Le Miroir de marriage,''
\protect\hyperlink{18_Chapter_Eleven__THE_FORMS_OF_THO.xhtmlux5cux23page_284}{284}

devils. \emph{See} witchcraft

\emph{Devotio moderna}: mysticism,
\protect\hyperlink{15_Chapter_Eight__RELIGIOUS_EXCITAT.xhtmlux5cux23page_223}{223},
\protect\hyperlink{17_Chapter_Ten__THE_FAILURE_OF_IMAG.xhtmlux5cux23page_265}{265};
not French,
\protect\hyperlink{15_Chapter_Eight__RELIGIOUS_EXCITAT.xhtmlux5cux23page_221}{221},
\protect\hyperlink{15_Chapter_Eight__RELIGIOUS_EXCITAT.xhtmlux5cux23page_223}{223--}\protect\hyperlink{15_Chapter_Eight__RELIGIOUS_EXCITAT.xhtmlux5cux23page_224}{224},
\protect\hyperlink{20_ILLUSTRATIONS_FOLLOW_PAGE.xhtmlux5cux23page_315}{315};
piety,
\protect\hyperlink{14_Chapter_Seven__THE_PIOUS_PERSONA.xhtmlux5cux23page_203}{203--}\protect\hyperlink{14_Chapter_Seven__THE_PIOUS_PERSONA.xhtmlux5cux23page_205}{205},
\protect\hyperlink{15_Chapter_Eight__RELIGIOUS_EXCITAT.xhtmlux5cux23page_222}{222},
\protect\hyperlink{15_Chapter_Eight__RELIGIOUS_EXCITAT.xhtmlux5cux23page_224}{224--}\protect\hyperlink{15_Chapter_Eight__RELIGIOUS_EXCITAT.xhtmlux5cux23page_225}{225};
and pilgrimages,
\protect\hyperlink{13_Chapter_Six__THE_DEPICTION_OF_TH.xhtmlux5cux23page_185}{185};
socialization,
\protect\hyperlink{15_Chapter_Eight__RELIGIOUS_EXCITAT.xhtmlux5cux23page_222}{222};
sphere of life,
\protect\hyperlink{20_ILLUSTRATIONS_FOLLOW_PAGE.xhtmlux5cux23page_313}{313},
\protect\hyperlink{20_ILLUSTRATIONS_FOLLOW_PAGE.xhtmlux5cux23page_315}{315}

Dionysius. \emph{See} Areopagite

\emph{Diversis diaboli tentationibus, De} (Gerson),
\protect\hyperlink{15_Chapter_Eight__RELIGIOUS_EXCITAT.xhtmlux5cux23page_228}{228}

\emph{Dolce stil nuove} (Dante),
\protect\hyperlink{11_Chapter_Four__THE_FORMS_OF_LOVE.xhtmlux5cux23page_126}{126},
\protect\hyperlink{11_Chapter_Four__THE_FORMS_OF_LOVE.xhtmlux5cux23page_131}{131}

Dominican order,
\protect\hyperlink{13_Chapter_Six__THE_DEPICTION_OF_TH.xhtmlux5cux23page_176}{176},
\protect\hyperlink{13_Chapter_Six__THE_DEPICTION_OF_TH.xhtmlux5cux23page_178}{178}

\emph{Donatus moralisatus seu allegoriam traductus} (Gerson),
\protect\hyperlink{16_Chapter_Nine__THE_DECLINE_OF_SYM.xhtmlux5cux23page_242}{242}

Dufay, Guillaume,
\protect\hyperlink{13_Chapter_Six__THE_DEPICTION_OF_TH.xhtmlux5cux23page_180}{180},
\protect\hyperlink{19_Chapter_Twelve__ART_IN_LIFE.xhtmlux5cux23page_294}{294}

Dunois,
\protect\hyperlink{10_Chapter_Three__THE_HEROIC_DREAM.xhtmlux5cux23page_77}{77}

Durand-Gréville, E.,
\protect\hyperlink{21_Chapter_Thirteen__IMAGE_AND_WORD.xhtmlux5cux23page_334}{334}

Durandus, Guilielmus,
\protect\hyperlink{16_Chapter_Nine__THE_DECLINE_OF_SYM.xhtmlux5cux23page_248}{248}

Durer, Albrecht,
\protect\hyperlink{20_ILLUSTRATIONS_FOLLOW_PAGE.xhtmlux5cux23page_320}{320},
\protect\hyperlink{21_Chapter_Thirteen__IMAGE_AND_WORD.xhtmlux5cux23page_374}{374}

Eck, Johannes,
\protect\hyperlink{13_Chapter_Six__THE_DEPICTION_OF_TH.xhtmlux5cux23page_195}{195}

Eckhart, Master,
\protect\hyperlink{17_Chapter_Ten__THE_FAILURE_OF_IMAG.xhtmlux5cux23page_257}{257},
\protect\hyperlink{17_Chapter_Ten__THE_FAILURE_OF_IMAG.xhtmlux5cux23page_258}{258},
\protect\hyperlink{17_Chapter_Ten__THE_FAILURE_OF_IMAG.xhtmlux5cux23page_262}{262},
\protect\hyperlink{17_Chapter_Ten__THE_FAILURE_OF_IMAG.xhtmlux5cux23page_265}{265}

Edward II (king of England),
\protect\hyperlink{08_Chapter_One__THE_PASSIONATE_INTE.xhtmlux5cux23page_12}{12}

Edward III (king of England): attacks merchant vessels,
\protect\hyperlink{08_Chapter_One__THE_PASSIONATE_INTE.xhtmlux5cux23page_11}{11},
\protect\hyperlink{10_Chapter_Three__THE_HEROIC_DREAM.xhtmlux5cux23page_111}{111};
defense of Montfort,
\protect\hyperlink{14_Chapter_Seven__THE_PIOUS_PERSONA.xhtmlux5cux23page_211}{211};
knightly duel,
\protect\hyperlink{10_Chapter_Three__THE_HEROIC_DREAM.xhtmlux5cux23page_115}{115};
vows,
\protect\hyperlink{10_Chapter_Three__THE_HEROIC_DREAM.xhtmlux5cux23page_97}{97},
\protect\hyperlink{10_Chapter_Three__THE_HEROIC_DREAM.xhtmlux5cux23page_98}{98},
\protect\hyperlink{10_Chapter_Three__THE_HEROIC_DREAM.xhtmlux5cux23page_102}{102}

Edward IV (king of England): and court ritual,
\protect\hyperlink{09_Chapter_Two__THE_CRAVING_FOR_A_M.xhtmlux5cux23page_42}{42};
and Innocent's Day,
\protect\hyperlink{13_Chapter_Six__THE_DEPICTION_OF_TH.xhtmlux5cux23page_176}{176};
and Margaret of Anjou,
\protect\hyperlink{08_Chapter_One__THE_PASSIONATE_INTE.xhtmlux5cux23page_14}{14}

Edward of York,
\protect\hyperlink{12_Chapter_Five__THE_VISION_OF_DEAT.xhtmlux5cux23page_164}{164}

Elizabeth of Thuringia, Saint,
\protect\hyperlink{13_Chapter_Six__THE_DEPICTION_OF_TH.xhtmlux5cux23page_192}{192}

Elspeth, Saint,
\protect\hyperlink{13_Chapter_Six__THE_DEPICTION_OF_TH.xhtmlux5cux23page_198}{198}

emblems,
\protect\hyperlink{18_Chapter_Eleven__THE_FORMS_OF_THO.xhtmlux5cux23page_276}{276}

Emerson, R. W.,
\protect\hyperlink{09_Chapter_Two__THE_CRAVING_FOR_A_M.xhtmlux5cux23page_47}{47}

\emph{Emprise du Dragon},
\protect\hyperlink{10_Chapter_Three__THE_HEROIC_DREAM.xhtmlux5cux23page_90}{90}

Epicureans,
\protect\hyperlink{13_Chapter_Six__THE_DEPICTION_OF_TH.xhtmlux5cux23page_189}{189}

epithalamium,
\protect\hyperlink{11_Chapter_Four__THE_FORMS_OF_LOVE.xhtmlux5cux23page_129}{129}

\emph{Epitre d'Othea à Hector} (Christine de Pisan),
\protect\hyperlink{21_Chapter_Thirteen__IMAGE_AND_WORD.xhtmlux5cux23page_376}{376--}\protect\hyperlink{21_Chapter_Thirteen__IMAGE_AND_WORD.xhtmlux5cux23page_377}{377}

Erasmus, Desiderius: cult of saints,
\protect\hyperlink{13_Chapter_Six__THE_DEPICTION_OF_TH.xhtmlux5cux23page_200}{200};
and Robert Gaguin,
\protect\hyperlink{22_Chapter_Fourteen__THE_COMING_OF.xhtmlux5cux23page_392}{392};
hears sermon on Prodigal,
\protect\hyperlink{21_Chapter_Thirteen__IMAGE_AND_WORD.xhtmlux5cux23page_333}{333};
optimism and pessimism,
\protect\hyperlink{09_Chapter_Two__THE_CRAVING_FOR_A_M.xhtmlux5cux23page_31}{31--}\protect\hyperlink{09_Chapter_Two__THE_CRAVING_FOR_A_M.xhtmlux5cux23page_32}{32};
Petrarch compared to,
\protect\hyperlink{22_Chapter_Fourteen__THE_COMING_OF.xhtmlux5cux23page_385}{385};
rhetoricians,
\protect\hyperlink{22_Chapter_Fourteen__THE_COMING_OF.xhtmlux5cux23page_389}{389}

Erasmus, Saint: attribute,
\protect\hyperlink{13_Chapter_Six__THE_DEPICTION_OF_TH.xhtmlux5cux23page_198}{198};
\emph{Martydom of},
\protect\hyperlink{21_Chapter_Thirteen__IMAGE_AND_WORD.xhtmlux5cux23page_376}{376}

``Esclamacion des os Sainct Innocent,''
\protect\hyperlink{21_Chapter_Thirteen__IMAGE_AND_WORD.xhtmlux5cux23page_361}{361}

Escouchy, Mathieu d',
\protect\hyperlink{08_Chapter_One__THE_PASSIONATE_INTE.xhtmlux5cux23page_28}{28},
\protect\hyperlink{10_Chapter_Three__THE_HEROIC_DREAM.xhtmlux5cux23page_72}{72}

\protect\hypertarget{25_INDEX.xhtmlux5cux23page_457}{}{}Escu verd à la
dame blanche, Ordre de 1',
\protect\hyperlink{10_Chapter_Three__THE_HEROIC_DREAM.xhtmlux5cux23page_79}{79},
\protect\hyperlink{10_Chapter_Three__THE_HEROIC_DREAM.xhtmlux5cux23page_94}{94}

\emph{Espinette amoureuse, Le} (Froissart),
\protect\hyperlink{21_Chapter_Thirteen__IMAGE_AND_WORD.xhtmlux5cux23page_363}{363}

Estienne, Henry,
\protect\hyperlink{13_Chapter_Six__THE_DEPICTION_OF_TH.xhtmlux5cux23page_200}{200}

\emph{estremets},
\protect\hyperlink{20_ILLUSTRATIONS_FOLLOW_PAGE.xhtmlux5cux23page_302}{302},
\protect\hyperlink{20_ILLUSTRATIONS_FOLLOW_PAGE.xhtmlux5cux23page_315}{315}

Eustace, Saint,
\protect\hyperlink{13_Chapter_Six__THE_DEPICTION_OF_TH.xhtmlux5cux23page_198}{198}

Eutropius, Saint,
\protect\hyperlink{13_Chapter_Six__THE_DEPICTION_OF_TH.xhtmlux5cux23page_200}{200}

\emph{Exposition su vérité mal prise} (Chastellain),
\protect\hyperlink{21_Chapter_Thirteen__IMAGE_AND_WORD.xhtmlux5cux23page_377}{377}

Eyck, Brothers van: courtly life,
\protect\hyperlink{20_ILLUSTRATIONS_FOLLOW_PAGE.xhtmlux5cux23page_313}{313--}\protect\hyperlink{20_ILLUSTRATIONS_FOLLOW_PAGE.xhtmlux5cux23page_316}{316};
contemporary appreciation,
\protect\hyperlink{20_ILLUSTRATIONS_FOLLOW_PAGE.xhtmlux5cux23page_319}{319};
dominate our perception,
\protect\hyperlink{19_Chapter_Twelve__ART_IN_LIFE.xhtmlux5cux23page_294}{294};
naturalism,
\protect\hyperlink{20_ILLUSTRATIONS_FOLLOW_PAGE.xhtmlux5cux23page_319}{319};
work compared to literature,
\protect\hyperlink{21_Chapter_Thirteen__IMAGE_AND_WORD.xhtmlux5cux23page_332}{332--}\protect\hyperlink{21_Chapter_Thirteen__IMAGE_AND_WORD.xhtmlux5cux23page_333}{333};
works: Ghent Altarpiece,
\protect\hyperlink{19_Chapter_Twelve__ART_IN_LIFE.xhtmlux5cux23page_297}{297},
\protect\hyperlink{21_Chapter_Thirteen__IMAGE_AND_WORD.xhtmlux5cux23page_342}{342},
\protect\hyperlink{21_Chapter_Thirteen__IMAGE_AND_WORD.xhtmlux5cux23page_372}{372--}\protect\hyperlink{21_Chapter_Thirteen__IMAGE_AND_WORD.xhtmlux5cux23page_373}{373}

Eyck, Hubert van,
\protect\hyperlink{21_Chapter_Thirteen__IMAGE_AND_WORD.xhtmlux5cux23page_363}{363}

Eyck, Jan van: bathing scenes,
\protect\hyperlink{20_ILLUSTRATIONS_FOLLOW_PAGE.xhtmlux5cux23page_299}{299};
bride portrait,
\protect\hyperlink{19_Chapter_Twelve__ART_IN_LIFE.xhtmlux5cux23page_297}{297};
compared to poets,
\protect\hyperlink{21_Chapter_Thirteen__IMAGE_AND_WORD.xhtmlux5cux23page_342}{342--}\protect\hyperlink{21_Chapter_Thirteen__IMAGE_AND_WORD.xhtmlux5cux23page_343}{343},
\protect\hyperlink{21_Chapter_Thirteen__IMAGE_AND_WORD.xhtmlux5cux23page_347}{347};
compared with Sluter,
\protect\hyperlink{20_ILLUSTRATIONS_FOLLOW_PAGE.xhtmlux5cux23page_309}{309};
court position,
\protect\hyperlink{20_ILLUSTRATIONS_FOLLOW_PAGE.xhtmlux5cux23page_307}{307--}\protect\hyperlink{20_ILLUSTRATIONS_FOLLOW_PAGE.xhtmlux5cux23page_308}{308},
\protect\hyperlink{20_ILLUSTRATIONS_FOLLOW_PAGE.xhtmlux5cux23page_313}{313--}\protect\hyperlink{20_ILLUSTRATIONS_FOLLOW_PAGE.xhtmlux5cux23page_316}{316};
Fazio on,
\protect\hyperlink{20_ILLUSTRATIONS_FOLLOW_PAGE.xhtmlux5cux23page_319}{319--}\protect\hyperlink{20_ILLUSTRATIONS_FOLLOW_PAGE.xhtmlux5cux23page_320}{320};
medieval,
\protect\hyperlink{20_ILLUSTRATIONS_FOLLOW_PAGE.xhtmlux5cux23page_319}{319},
\protect\hyperlink{21_Chapter_Thirteen__IMAGE_AND_WORD.xhtmlux5cux23page_329}{329};
nudity,
\protect\hyperlink{21_Chapter_Thirteen__IMAGE_AND_WORD.xhtmlux5cux23page_373}{373};
patrons,
\protect\hyperlink{20_ILLUSTRATIONS_FOLLOW_PAGE.xhtmlux5cux23page_306}{306--}\protect\hyperlink{20_ILLUSTRATIONS_FOLLOW_PAGE.xhtmlux5cux23page_307}{307};
practical work,
\protect\hyperlink{20_ILLUSTRATIONS_FOLLOW_PAGE.xhtmlux5cux23page_299}{299},
\protect\hyperlink{20_ILLUSTRATIONS_FOLLOW_PAGE.xhtmlux5cux23page_306}{306--}\protect\hyperlink{20_ILLUSTRATIONS_FOLLOW_PAGE.xhtmlux5cux23page_307}{307};
Rogier van der Weyden compared to,
\protect\hyperlink{21_Chapter_Thirteen__IMAGE_AND_WORD.xhtmlux5cux23page_376}{376};
Rolin,
\protect\hyperlink{20_ILLUSTRATIONS_FOLLOW_PAGE.xhtmlux5cux23page_317}{317};
weaknesses,
\protect\hyperlink{21_Chapter_Thirteen__IMAGE_AND_WORD.xhtmlux5cux23page_375}{375--}\protect\hyperlink{21_Chapter_Thirteen__IMAGE_AND_WORD.xhtmlux5cux23page_376}{376};
works: \emph{Annunciation},
\protect\hyperlink{21_Chapter_Thirteen__IMAGE_AND_WORD.xhtmlux5cux23page_335}{335},
\protect\hyperlink{22_Chapter_Fourteen__THE_COMING_OF.xhtmlux5cux23page_386}{386};
\emph{Arnolfini Marriage},
\protect\hyperlink{20_ILLUSTRATIONS_FOLLOW_PAGE.xhtmlux5cux23page_312}{312},
\protect\hyperlink{20_ILLUSTRATIONS_FOLLOW_PAGE.xhtmlux5cux23page_313}{313};
\emph{Bath of Women},
\protect\hyperlink{21_Chapter_Thirteen__IMAGE_AND_WORD.xhtmlux5cux23page_373}{373};
\emph{Little Love Magic} (attributed),
\protect\hyperlink{21_Chapter_Thirteen__IMAGE_AND_WORD.xhtmlux5cux23page_373}{373};
\emph{Madonna of the Chancellor Rolin},
\protect\hyperlink{21_Chapter_Thirteen__IMAGE_AND_WORD.xhtmlux5cux23page_333}{333--}\protect\hyperlink{21_Chapter_Thirteen__IMAGE_AND_WORD.xhtmlux5cux23page_335}{335},
\emph{Madonna of the Canon van der Paele},
\protect\hyperlink{21_Chapter_Thirteen__IMAGE_AND_WORD.xhtmlux5cux23page_356}{356};
portraits,
\protect\hyperlink{21_Chapter_Thirteen__IMAGE_AND_WORD.xhtmlux5cux23page_331}{331}

Falstaff, Sir John,
\protect\hyperlink{12_Chapter_Five__THE_VISION_OF_DEAT.xhtmlux5cux23page_164}{164}

Farinata degli Uberti,
\protect\hyperlink{17_Chapter_Ten__THE_FAILURE_OF_IMAG.xhtmlux5cux23page_252}{252}

Fazio, Bartolomeo,
\protect\hyperlink{20_ILLUSTRATIONS_FOLLOW_PAGE.xhtmlux5cux23page_319}{319},
\protect\hyperlink{21_Chapter_Thirteen__IMAGE_AND_WORD.xhtmlux5cux23page_373}{373}

Ferrer, Vincent, Saint: and Brother Thomas,
\protect\hyperlink{15_Chapter_Eight__RELIGIOUS_EXCITAT.xhtmlux5cux23page_226}{226};
eloquence,
\protect\hyperlink{15_Chapter_Eight__RELIGIOUS_EXCITAT.xhtmlux5cux23page_222}{222};
impact of,
\protect\hyperlink{08_Chapter_One__THE_PASSIONATE_INTE.xhtmlux5cux23page_5}{5--}\protect\hyperlink{08_Chapter_One__THE_PASSIONATE_INTE.xhtmlux5cux23page_6}{6};
political adviser,
\protect\hyperlink{08_Chapter_One__THE_PASSIONATE_INTE.xhtmlux5cux23page_12}{12};
tears,
\protect\hyperlink{15_Chapter_Eight__RELIGIOUS_EXCITAT.xhtmlux5cux23page_223}{223};
type of saint,
\protect\hyperlink{14_Chapter_Seven__THE_PIOUS_PERSONA.xhtmlux5cux23page_210}{210}

Fiacrius, Saint,
\protect\hyperlink{13_Chapter_Six__THE_DEPICTION_OF_TH.xhtmlux5cux23page_200}{200}

Fillastre, Guillaume (bishop of Tournay),
\protect\hyperlink{10_Chapter_Three__THE_HEROIC_DREAM.xhtmlux5cux23page_94}{94},
\protect\hyperlink{10_Chapter_Three__THE_HEROIC_DREAM.xhtmlux5cux23page_96}{96},
\protect\hyperlink{10_Chapter_Three__THE_HEROIC_DREAM.xhtmlux5cux23page_107}{107},
\protect\hyperlink{18_Chapter_Eleven__THE_FORMS_OF_THO.xhtmlux5cux23page_274}{274}

Fismes, castle of,
\protect\hyperlink{21_Chapter_Thirteen__IMAGE_AND_WORD.xhtmlux5cux23page_362}{362}

Flémalle, Master of (Robert Campin),
\protect\hyperlink{21_Chapter_Thirteen__IMAGE_AND_WORD.xhtmlux5cux23page_362}{362}

\emph{Flight into Egypt} (Broderlam),
\protect\hyperlink{21_Chapter_Thirteen__IMAGE_AND_WORD.xhtmlux5cux23page_363}{363}

Foucquet, Jehan,
\protect\hyperlink{13_Chapter_Six__THE_DEPICTION_OF_TH.xhtmlux5cux23page_182}{182}

Fradin, Antoine,
\protect\hyperlink{08_Chapter_One__THE_PASSIONATE_INTE.xhtmlux5cux23page_5}{5}

``Franc Gontier, Le dit de'' (Philippe de Vitri),
\protect\hyperlink{10_Chapter_Three__THE_HEROIC_DREAM.xhtmlux5cux23page_121}{121},
\protect\hyperlink{10_Chapter_Three__THE_HEROIC_DREAM.xhtmlux5cux23page_124}{124},
\protect\hyperlink{11_Chapter_Four__THE_FORMS_OF_LOVE.xhtmlux5cux23page_153}{153--}\protect\hyperlink{11_Chapter_Four__THE_FORMS_OF_LOVE.xhtmlux5cux23page_154}{154}

France, Anatole,
\protect\hyperlink{21_Chapter_Thirteen__IMAGE_AND_WORD.xhtmlux5cux23page_363}{363}

France, House of,
\protect\hyperlink{08_Chapter_One__THE_PASSIONATE_INTE.xhtmlux5cux23page_13}{13},
\protect\hyperlink{08_Chapter_One__THE_PASSIONATE_INTE.xhtmlux5cux23page_16}{16},
\protect\hyperlink{09_Chapter_Two__THE_CRAVING_FOR_A_M.xhtmlux5cux23page_46}{46},
\protect\hyperlink{14_Chapter_Seven__THE_PIOUS_PERSONA.xhtmlux5cux23page_217}{217}

France, kings and queens of,
\protect\hyperlink{08_Chapter_One__THE_PASSIONATE_INTE.xhtmlux5cux23page_8}{8},
\protect\hyperlink{10_Chapter_Three__THE_HEROIC_DREAM.xhtmlux5cux23page_63}{63},
\protect\hyperlink{09_Chapter_Two__THE_CRAVING_FOR_A_M.xhtmlux5cux23page_54}{54},
\protect\hyperlink{10_Chapter_Three__THE_HEROIC_DREAM.xhtmlux5cux23page_106}{106};
birth and mourning customs,
\protect\hyperlink{09_Chapter_Two__THE_CRAVING_FOR_A_M.xhtmlux5cux23page_56}{56--}\protect\hyperlink{09_Chapter_Two__THE_CRAVING_FOR_A_M.xhtmlux5cux23page_57}{57}

France, military spirit,
\protect\hyperlink{10_Chapter_Three__THE_HEROIC_DREAM.xhtmlux5cux23page_80}{80}

Francis I (king of France),
\protect\hyperlink{10_Chapter_Three__THE_HEROIC_DREAM.xhtmlux5cux23page_77}{77},
\protect\hyperlink{10_Chapter_Three__THE_HEROIC_DREAM.xhtmlux5cux23page_109}{109}

Francis de Paula, Saint,
\protect\hyperlink{14_Chapter_Seven__THE_PIOUS_PERSONA.xhtmlux5cux23page_210}{210},
\protect\hyperlink{14_Chapter_Seven__THE_PIOUS_PERSONA.xhtmlux5cux23page_214}{214},
\protect\hyperlink{14_Chapter_Seven__THE_PIOUS_PERSONA.xhtmlux5cux23page_216}{216--}\protect\hyperlink{14_Chapter_Seven__THE_PIOUS_PERSONA.xhtmlux5cux23page_218}{218}

Francis of Assisi, Saint,
\protect\hyperlink{14_Chapter_Seven__THE_PIOUS_PERSONA.xhtmlux5cux23page_206}{206},
\protect\hyperlink{14_Chapter_Seven__THE_PIOUS_PERSONA.xhtmlux5cux23page_209}{209}

Francis Xavier, Saint,
\protect\hyperlink{14_Chapter_Seven__THE_PIOUS_PERSONA.xhtmlux5cux23page_210}{210}

Franciscan order,
\protect\hyperlink{11_Chapter_Four__THE_FORMS_OF_LOVE.xhtmlux5cux23page_131}{131},
\protect\hyperlink{12_Chapter_Five__THE_VISION_OF_DEAT.xhtmlux5cux23page_158}{158},
\protect\hyperlink{13_Chapter_Six__THE_DEPICTION_OF_TH.xhtmlux5cux23page_176}{176}

Frankenthal, vision of,
\protect\hyperlink{13_Chapter_Six__THE_DEPICTION_OF_TH.xhtmlux5cux23page_192}{192--}\protect\hyperlink{13_Chapter_Six__THE_DEPICTION_OF_TH.xhtmlux5cux23page_193}{193}

\emph{Fraterhouses},
\protect\hyperlink{14_Chapter_Seven__THE_PIOUS_PERSONA.xhtmlux5cux23page_203}{203},
\protect\hyperlink{15_Chapter_Eight__RELIGIOUS_EXCITAT.xhtmlux5cux23page_223}{223},
\protect\hyperlink{17_Chapter_Ten__THE_FAILURE_OF_IMAG.xhtmlux5cux23page_265}{265},
\protect\hyperlink{20_ILLUSTRATIONS_FOLLOW_PAGE.xhtmlux5cux23page_314}{314}.
See also \emph{Devotio moderna}; Windesheim convents

Frederick III (emperor): compared to God the Father,
\protect\hyperlink{13_Chapter_Six__THE_DEPICTION_OF_TH.xhtmlux5cux23page_181}{181},
\protect\hyperlink{15_Chapter_Eight__RELIGIOUS_EXCITAT.xhtmlux5cux23page_221}{221};
entry into Brussels,
\protect\hyperlink{13_Chapter_Six__THE_DEPICTION_OF_TH.xhtmlux5cux23page_181}{181};
and Philip the Good,
\protect\hyperlink{10_Chapter_Three__THE_HEROIC_DREAM.xhtmlux5cux23page_109}{109},
\protect\hyperlink{20_ILLUSTRATIONS_FOLLOW_PAGE.xhtmlux5cux23page_315}{315};
princely duel,
\protect\hyperlink{10_Chapter_Three__THE_HEROIC_DREAM.xhtmlux5cux23page_109}{109}

Froissart, Jean: and Albert of Bavaria,
\protect\hyperlink{10_Chapter_Three__THE_HEROIC_DREAM.xhtmlux5cux23page_118}{118};
allegory,
\protect\hyperlink{16_Chapter_Nine__THE_DECLINE_OF_SYM.xhtmlux5cux23page_247}{247};
Artevelde,
\protect\hyperlink{09_Chapter_Two__THE_CRAVING_FOR_A_M.xhtmlux5cux23page_34}{34},
\protect\hyperlink{10_Chapter_Three__THE_HEROIC_DREAM.xhtmlux5cux23page_115}{115};
battles,
\protect\hyperlink{18_Chapter_Eleven__THE_FORMS_OF_THO.xhtmlux5cux23page_283}{283};
beauty,
\protect\hyperlink{20_ILLUSTRATIONS_FOLLOW_PAGE.xhtmlux5cux23page_324}{324};
blasphemy,
\protect\hyperlink{13_Chapter_Six__THE_DEPICTION_OF_TH.xhtmlux5cux23page_179}{179};
bravery,
\protect\hyperlink{10_Chapter_Three__THE_HEROIC_DREAM.xhtmlux5cux23page_75}{75};
Charles de Blois,
\protect\hyperlink{14_Chapter_Seven__THE_PIOUS_PERSONA.xhtmlux5cux23page_212}{212};
children's games,
\protect\hyperlink{21_Chapter_Thirteen__IMAGE_AND_WORD.xhtmlux5cux23page_363}{363};
\emph{combat des trente},
\protect\hyperlink{10_Chapter_Three__THE_HEROIC_DREAM.xhtmlux5cux23page_112}{112};
crudity,
\protect\hyperlink{11_Chapter_Four__THE_FORMS_OF_LOVE.xhtmlux5cux23page_129}{129};
description,
\protect\hyperlink{21_Chapter_Thirteen__IMAGE_AND_WORD.xhtmlux5cux23page_349}{349};
dialogue,
\protect\hyperlink{21_Chapter_Thirteen__IMAGE_AND_WORD.xhtmlux5cux23page_347}{347--}\protect\hyperlink{21_Chapter_Thirteen__IMAGE_AND_WORD.xhtmlux5cux23page_348}{348};
duel of Edward III,
\protect\hyperlink{10_Chapter_Three__THE_HEROIC_DREAM.xhtmlux5cux23page_115}{115};
Ghent, burghers of,
\protect\hyperlink{10_Chapter_Three__THE_HEROIC_DREAM.xhtmlux5cux23page_115}{115--}\protect\hyperlink{10_Chapter_Three__THE_HEROIC_DREAM.xhtmlux5cux23page_116}{116};
heralds and kings of arms,
\protect\hyperlink{10_Chapter_Three__THE_HEROIC_DREAM.xhtmlux5cux23page_72}{72};
Jason,
\protect\hyperlink{10_Chapter_Three__THE_HEROIC_DREAM.xhtmlux5cux23page_95}{95};
knighthood,
\protect\hyperlink{10_Chapter_Three__THE_HEROIC_DREAM.xhtmlux5cux23page_72}{72};
lack of ideas,
\protect\hyperlink{21_Chapter_Thirteen__IMAGE_AND_WORD.xhtmlux5cux23page_354}{354--}\protect\hyperlink{21_Chapter_Thirteen__IMAGE_AND_WORD.xhtmlux5cux23page_355}{355};
melancholy,
\protect\hyperlink{09_Chapter_Two__THE_CRAVING_FOR_A_M.xhtmlux5cux23page_34}{34};
oath,
\protect\hyperlink{10_Chapter_Three__THE_HEROIC_DREAM.xhtmlux5cux23page_99}{99};
Order of the Stars,
\protect\hyperlink{10_Chapter_Three__THE_HEROIC_DREAM.xhtmlux5cux23page_111}{111};
painters and nobles,
\protect\hyperlink{20_ILLUSTRATIONS_FOLLOW_PAGE.xhtmlux5cux23page_299}{299--}\protect\hyperlink{20_ILLUSTRATIONS_FOLLOW_PAGE.xhtmlux5cux23page_300}{300};
Peter of Luxembourg,
\protect\hyperlink{14_Chapter_Seven__THE_PIOUS_PERSONA.xhtmlux5cux23page_213}{213};
profit in war,
\protect\hyperlink{10_Chapter_Three__THE_HEROIC_DREAM.xhtmlux5cux23page_75}{75},
\protect\hyperlink{10_Chapter_Three__THE_HEROIC_DREAM.xhtmlux5cux23page_117}{117};
rhetoric,
\protect\hyperlink{21_Chapter_Thirteen__IMAGE_AND_WORD.xhtmlux5cux23page_355}{355};
sea battles,
\protect\hyperlink{10_Chapter_Three__THE_HEROIC_DREAM.xhtmlux5cux23page_116}{116};
ships,
\protect\hyperlink{20_ILLUSTRATIONS_FOLLOW_PAGE.xhtmlux5cux23page_324}{324};
symbolism,
\protect\hyperlink{16_Chapter_Nine__THE_DECLINE_OF_SYM.xhtmlux5cux23page_242}{242};
witchcraft,
\protect\hyperlink{18_Chapter_Eleven__THE_FORMS_OF_THO.xhtmlux5cux23page_292}{292};
works: ``Débat dou cheval et dou levrier,''
\protect\hyperlink{21_Chapter_Thirteen__IMAGE_AND_WORD.xhtmlux5cux23page_360}{360};
\emph{Espinette amoureuse},
\protect\hyperlink{21_Chapter_Thirteen__IMAGE_AND_WORD.xhtmlux5cux23page_363}{363};
\emph{Meliador},
\protect\hyperlink{10_Chapter_Three__THE_HEROIC_DREAM.xhtmlux5cux23page_72}{72},
\protect\hyperlink{10_Chapter_Three__THE_HEROIC_DREAM.xhtmlux5cux23page_84}{84};
``Orloge amoureus,''
\protect\hyperlink{16_Chapter_Nine__THE_DECLINE_OF_SYM.xhtmlux5cux23page_242}{242};
\emph{Perceforest},
\protect\hyperlink{10_Chapter_Three__THE_HEROIC_DREAM.xhtmlux5cux23page_84}{84}

\protect\hypertarget{25_INDEX.xhtmlux5cux23page_458}{}{}Froment, Jean,
\protect\hyperlink{08_Chapter_One__THE_PASSIONATE_INTE.xhtmlux5cux23page_28}{28}

Fulco (bishop of Toulouse),
\protect\hyperlink{17_Chapter_Ten__THE_FAILURE_OF_IMAG.xhtmlux5cux23page_251}{251}

Fusil,
\protect\hyperlink{10_Chapter_Three__THE_HEROIC_DREAM.xhtmlux5cux23page_96}{96}

Gaguin, Robert: academic history of France,
\protect\hyperlink{22_Chapter_Fourteen__THE_COMING_OF.xhtmlux5cux23page_392}{392};
Erasmus,
\protect\hyperlink{22_Chapter_Fourteen__THE_COMING_OF.xhtmlux5cux23page_392}{392};
humanist,
\protect\hyperlink{10_Chapter_Three__THE_HEROIC_DREAM.xhtmlux5cux23page_124}{124},
\protect\hyperlink{22_Chapter_Fourteen__THE_COMING_OF.xhtmlux5cux23page_392}{392};
works: \emph{Le Curial} (translation),
\protect\hyperlink{10_Chapter_Three__THE_HEROIC_DREAM.xhtmlux5cux23page_124}{124};
\emph{Débat du laboureur, du preste et du gendarme},
\protect\hyperlink{10_Chapter_Three__THE_HEROIC_DREAM.xhtmlux5cux23page_67}{67};
``Le passe temps d'oysiveté,''
\protect\hyperlink{18_Chapter_Eleven__THE_FORMS_OF_THO.xhtmlux5cux23page_274}{274}

\emph{Galois},
\protect\hyperlink{10_Chapter_Three__THE_HEROIC_DREAM.xhtmlux5cux23page_98}{98}

Garter, Order of the,
\protect\hyperlink{10_Chapter_Three__THE_HEROIC_DREAM.xhtmlux5cux23page_93}{93}

Gaunt, John of (duke of Lancaster),
\protect\hyperlink{19_Chapter_Twelve__ART_IN_LIFE.xhtmlux5cux23page_297}{297}

Gavere, battle of,
\protect\hyperlink{18_Chapter_Eleven__THE_FORMS_OF_THO.xhtmlux5cux23page_283}{283}

Gawain, Sir,
\protect\hyperlink{10_Chapter_Three__THE_HEROIC_DREAM.xhtmlux5cux23page_75}{75}

Gelder, Adolf and Arnold of,
\protect\hyperlink{08_Chapter_One__THE_PASSIONATE_INTE.xhtmlux5cux23page_18}{18},
\protect\hyperlink{14_Chapter_Seven__THE_PIOUS_PERSONA.xhtmlux5cux23page_218}{218},
\protect\hyperlink{18_Chapter_Eleven__THE_FORMS_OF_THO.xhtmlux5cux23page_283}{283}

George, Saint,
\protect\hyperlink{10_Chapter_Three__THE_HEROIC_DREAM.xhtmlux5cux23page_78}{78},
\protect\hyperlink{10_Chapter_Three__THE_HEROIC_DREAM.xhtmlux5cux23page_95}{95},
\protect\hyperlink{13_Chapter_Six__THE_DEPICTION_OF_TH.xhtmlux5cux23page_198}{198},
\protect\hyperlink{21_Chapter_Thirteen__IMAGE_AND_WORD.xhtmlux5cux23page_356}{356}

Germain, Jean (bishop of Chalons),
\protect\hyperlink{08_Chapter_One__THE_PASSIONATE_INTE.xhtmlux5cux23page_8}{8},
\protect\hyperlink{10_Chapter_Three__THE_HEROIC_DREAM.xhtmlux5cux23page_95}{95},
\protect\hyperlink{16_Chapter_Nine__THE_DECLINE_OF_SYM.xhtmlux5cux23page_244}{244},
\protect\hyperlink{22_Chapter_Fourteen__THE_COMING_OF.xhtmlux5cux23page_387}{387}

Gerson, Jean de: angels,
\protect\hyperlink{13_Chapter_Six__THE_DEPICTION_OF_TH.xhtmlux5cux23page_181}{181},
\protect\hyperlink{13_Chapter_Six__THE_DEPICTION_OF_TH.xhtmlux5cux23page_201}{201};
blasphemy,
\protect\hyperlink{13_Chapter_Six__THE_DEPICTION_OF_TH.xhtmlux5cux23page_187}{187-}\protect\hyperlink{13_Chapter_Six__THE_DEPICTION_OF_TH.xhtmlux5cux23page_188}{188};
character,
\protect\hyperlink{15_Chapter_Eight__RELIGIOUS_EXCITAT.xhtmlux5cux23page_224}{224};
compared to Thomas à Kempis,
\protect\hyperlink{17_Chapter_Ten__THE_FAILURE_OF_IMAG.xhtmlux5cux23page_266}{266};
compared to van Eycks,
\protect\hyperlink{20_ILLUSTRATIONS_FOLLOW_PAGE.xhtmlux5cux23page_319}{319};
contemplative life,
\protect\hyperlink{15_Chapter_Eight__RELIGIOUS_EXCITAT.xhtmlux5cux23page_225}{225};
dread of life,
\protect\hyperlink{09_Chapter_Two__THE_CRAVING_FOR_A_M.xhtmlux5cux23page_36}{36};
Greet Groote,
\protect\hyperlink{15_Chapter_Eight__RELIGIOUS_EXCITAT.xhtmlux5cux23page_223}{223};
image of cross,
\protect\hyperlink{15_Chapter_Eight__RELIGIOUS_EXCITAT.xhtmlux5cux23page_220}{220};
irreligion,
\protect\hyperlink{13_Chapter_Six__THE_DEPICTION_OF_TH.xhtmlux5cux23page_175}{175--}\protect\hyperlink{13_Chapter_Six__THE_DEPICTION_OF_TH.xhtmlux5cux23page_179}{179},
\protect\hyperlink{13_Chapter_Six__THE_DEPICTION_OF_TH.xhtmlux5cux23page_188}{188};
misfortunes of poor,
\protect\hyperlink{10_Chapter_Three__THE_HEROIC_DREAM.xhtmlux5cux23page_66}{66};
modern devotion,
\protect\hyperlink{15_Chapter_Eight__RELIGIOUS_EXCITAT.xhtmlux5cux23page_228}{228};
morality,
\protect\hyperlink{18_Chapter_Eleven__THE_FORMS_OF_THO.xhtmlux5cux23page_281}{281};
mysticism,
\protect\hyperlink{15_Chapter_Eight__RELIGIOUS_EXCITAT.xhtmlux5cux23page_228}{228--}\protect\hyperlink{15_Chapter_Eight__RELIGIOUS_EXCITAT.xhtmlux5cux23page_230}{230};
Orléans, Louis d',
\protect\hyperlink{18_Chapter_Eleven__THE_FORMS_OF_THO.xhtmlux5cux23page_287}{287};
penance for felons,
\protect\hyperlink{08_Chapter_One__THE_PASSIONATE_INTE.xhtmlux5cux23page_21}{21};
popular devotion,
\protect\hyperlink{15_Chapter_Eight__RELIGIOUS_EXCITAT.xhtmlux5cux23page_224}{224--}\protect\hyperlink{15_Chapter_Eight__RELIGIOUS_EXCITAT.xhtmlux5cux23page_225}{225};
proverbs,
\protect\hyperlink{18_Chapter_Eleven__THE_FORMS_OF_THO.xhtmlux5cux23page_274}{274};
Quiricus, Saint,
\protect\hyperlink{17_Chapter_Ten__THE_FAILURE_OF_IMAG.xhtmlux5cux23page_254}{254};
and \emph{Roman de la rose},
\protect\hyperlink{11_Chapter_Four__THE_FORMS_OF_LOVE.xhtmlux5cux23page_138}{138--}\protect\hyperlink{11_Chapter_Four__THE_FORMS_OF_LOVE.xhtmlux5cux23page_140}{140},
\protect\hyperlink{11_Chapter_Four__THE_FORMS_OF_LOVE.xhtmlux5cux23page_154}{154};
Saint Joseph,
\protect\hyperlink{13_Chapter_Six__THE_DEPICTION_OF_TH.xhtmlux5cux23page_176}{176--}\protect\hyperlink{13_Chapter_Six__THE_DEPICTION_OF_TH.xhtmlux5cux23page_177}{177},
\protect\hyperlink{13_Chapter_Six__THE_DEPICTION_OF_TH.xhtmlux5cux23page_179}{179};
saints,
\protect\hyperlink{13_Chapter_Six__THE_DEPICTION_OF_TH.xhtmlux5cux23page_187}{187};
symbolism,
\protect\hyperlink{16_Chapter_Nine__THE_DECLINE_OF_SYM.xhtmlux5cux23page_241}{241--}\protect\hyperlink{16_Chapter_Nine__THE_DECLINE_OF_SYM.xhtmlux5cux23page_242}{242},
\protect\hyperlink{16_Chapter_Nine__THE_DECLINE_OF_SYM.xhtmlux5cux23page_248}{248};
unchastity of priests,
\protect\hyperlink{15_Chapter_Eight__RELIGIOUS_EXCITAT.xhtmlux5cux23page_228}{228--}\protect\hyperlink{15_Chapter_Eight__RELIGIOUS_EXCITAT.xhtmlux5cux23page_229}{229};
vanished glory theme,
\protect\hyperlink{12_Chapter_Five__THE_VISION_OF_DEAT.xhtmlux5cux23page_158}{158};
witchcraft,
\protect\hyperlink{18_Chapter_Eleven__THE_FORMS_OF_THO.xhtmlux5cux23page_292}{292};
works: \emph{De diversis diaboli tentationibus},
\protect\hyperlink{15_Chapter_Eight__RELIGIOUS_EXCITAT.xhtmlux5cux23page_228}{228};
\emph{Donatus moralisatus seu des allegoriam traductus},
\protect\hyperlink{16_Chapter_Nine__THE_DECLINE_OF_SYM.xhtmlux5cux23page_242}{242}

Ghent, defense of,
\protect\hyperlink{10_Chapter_Three__THE_HEROIC_DREAM.xhtmlux5cux23page_65}{65},
\protect\hyperlink{10_Chapter_Three__THE_HEROIC_DREAM.xhtmlux5cux23page_115}{115},
\protect\hyperlink{18_Chapter_Eleven__THE_FORMS_OF_THO.xhtmlux5cux23page_281}{281--}\protect\hyperlink{18_Chapter_Eleven__THE_FORMS_OF_THO.xhtmlux5cux23page_283}{283}

Ghent Altarpiece (van Eycks); analysis,
\protect\hyperlink{21_Chapter_Thirteen__IMAGE_AND_WORD.xhtmlux5cux23page_337}{337};
compared to literature,
\protect\hyperlink{21_Chapter_Thirteen__IMAGE_AND_WORD.xhtmlux5cux23page_339}{339},
\protect\hyperlink{21_Chapter_Thirteen__IMAGE_AND_WORD.xhtmlux5cux23page_342}{342};
landscape in,
\protect\hyperlink{21_Chapter_Thirteen__IMAGE_AND_WORD.xhtmlux5cux23page_339}{339};
purpose,
\protect\hyperlink{19_Chapter_Twelve__ART_IN_LIFE.xhtmlux5cux23page_297}{297};
rhythm,
\protect\hyperlink{21_Chapter_Thirteen__IMAGE_AND_WORD.xhtmlux5cux23page_376}{376};
Jodocus Vydt, donor,
\protect\hyperlink{20_ILLUSTRATIONS_FOLLOW_PAGE.xhtmlux5cux23page_316}{316};
Windesheimers and,
\protect\hyperlink{20_ILLUSTRATIONS_FOLLOW_PAGE.xhtmlux5cux23page_314}{314}

Gideon,
\protect\hyperlink{10_Chapter_Three__THE_HEROIC_DREAM.xhtmlux5cux23page_95}{95},
\protect\hyperlink{21_Chapter_Thirteen__IMAGE_AND_WORD.xhtmlux5cux23page_342}{342}

Giles, Saint,
\protect\hyperlink{13_Chapter_Six__THE_DEPICTION_OF_TH.xhtmlux5cux23page_198}{198},
\protect\hyperlink{17_Chapter_Ten__THE_FAILURE_OF_IMAG.xhtmlux5cux23page_254}{254}

Giotto,
\protect\hyperlink{21_Chapter_Thirteen__IMAGE_AND_WORD.xhtmlux5cux23page_375}{375}

Glasdale, William,
\protect\hyperlink{12_Chapter_Five__THE_VISION_OF_DEAT.xhtmlux5cux23page_164}{164}

Godefroy, Denis,
\protect\hyperlink{13_Chapter_Six__THE_DEPICTION_OF_TH.xhtmlux5cux23page_182}{182}

Goes, Hugo van der,
\protect\hyperlink{20_ILLUSTRATIONS_FOLLOW_PAGE.xhtmlux5cux23page_299}{299}

Goethe,
\protect\hyperlink{09_Chapter_Two__THE_CRAVING_FOR_A_M.xhtmlux5cux23page_47}{47},
\protect\hyperlink{16_Chapter_Nine__THE_DECLINE_OF_SYM.xhtmlux5cux23page_238}{238}

Gonzaga, Aloysius, Saint,
\protect\hyperlink{14_Chapter_Seven__THE_PIOUS_PERSONA.xhtmlux5cux23page_210}{210},
\protect\hyperlink{15_Chapter_Eight__RELIGIOUS_EXCITAT.xhtmlux5cux23page_226}{226}

Gonzaga, Francesco,
\protect\hyperlink{10_Chapter_Three__THE_HEROIC_DREAM.xhtmlux5cux23page_109}{109}

Golden Fleece, Order of the,
\protect\hyperlink{20_ILLUSTRATIONS_FOLLOW_PAGE.xhtmlux5cux23page_316}{316};
Jean Cevrot,
\protect\hyperlink{20_ILLUSTRATIONS_FOLLOW_PAGE.xhtmlux5cux23page_316}{316};
Gideon,
\protect\hyperlink{10_Chapter_Three__THE_HEROIC_DREAM.xhtmlux5cux23page_95}{95};
heralds and kings of arms,
\protect\hyperlink{10_Chapter_Three__THE_HEROIC_DREAM.xhtmlux5cux23page_72}{72--}\protect\hyperlink{10_Chapter_Three__THE_HEROIC_DREAM.xhtmlux5cux23page_73}{73};
importance,
\protect\hyperlink{10_Chapter_Three__THE_HEROIC_DREAM.xhtmlux5cux23page_94}{94--}\protect\hyperlink{10_Chapter_Three__THE_HEROIC_DREAM.xhtmlux5cux23page_95}{95};
Jason,
\protect\hyperlink{10_Chapter_Three__THE_HEROIC_DREAM.xhtmlux5cux23page_95}{95},
\protect\hyperlink{20_ILLUSTRATIONS_FOLLOW_PAGE.xhtmlux5cux23page_311}{311};
Kolchis,
\protect\hyperlink{10_Chapter_Three__THE_HEROIC_DREAM.xhtmlux5cux23page_95}{95};
pomp,
\protect\hyperlink{10_Chapter_Three__THE_HEROIC_DREAM.xhtmlux5cux23page_96}{96};
spiritual significance,
\protect\hyperlink{10_Chapter_Three__THE_HEROIC_DREAM.xhtmlux5cux23page_93}{93};
success,
\protect\hyperlink{10_Chapter_Three__THE_HEROIC_DREAM.xhtmlux5cux23page_95}{95--}\protect\hyperlink{10_Chapter_Three__THE_HEROIC_DREAM.xhtmlux5cux23page_96}{96}

Golden Shield, Order of the,
\protect\hyperlink{10_Chapter_Three__THE_HEROIC_DREAM.xhtmlux5cux23page_94}{94}

Granson, bombardment of,
\protect\hyperlink{11_Chapter_Four__THE_FORMS_OF_LOVE.xhtmlux5cux23page_153}{153}

Granson, Othe de,
\protect\hyperlink{21_Chapter_Thirteen__IMAGE_AND_WORD.xhtmlux5cux23page_367}{367}

greed, sin of,
\protect\hyperlink{08_Chapter_One__THE_PASSIONATE_INTE.xhtmlux5cux23page_25}{25--}\protect\hyperlink{08_Chapter_One__THE_PASSIONATE_INTE.xhtmlux5cux23page_27}{27}

Gregory the Great (pope),
\protect\hyperlink{10_Chapter_Three__THE_HEROIC_DREAM.xhtmlux5cux23page_68}{68}

Guernier, Laurent,
\protect\hyperlink{18_Chapter_Eleven__THE_FORMS_OF_THO.xhtmlux5cux23page_280}{280}

Guesclin, Bertrand du,
\protect\hyperlink{10_Chapter_Three__THE_HEROIC_DREAM.xhtmlux5cux23page_77}{77},
\protect\hyperlink{10_Chapter_Three__THE_HEROIC_DREAM.xhtmlux5cux23page_78}{78},
\protect\hyperlink{10_Chapter_Three__THE_HEROIC_DREAM.xhtmlux5cux23page_100}{100},
\protect\hyperlink{14_Chapter_Seven__THE_PIOUS_PERSONA.xhtmlux5cux23page_211}{211},
\protect\hyperlink{20_ILLUSTRATIONS_FOLLOW_PAGE.xhtmlux5cux23page_298}{298}

Gwendolyn of Bar, Saint,
\protect\hyperlink{13_Chapter_Six__THE_DEPICTION_OF_TH.xhtmlux5cux23page_198}{198}

Hacht, Hannequin de,
\protect\hyperlink{20_ILLUSTRATIONS_FOLLOW_PAGE.xhtmlux5cux23page_310}{310}

Hagenbach, Peter von,
\protect\hyperlink{08_Chapter_One__THE_PASSIONATE_INTE.xhtmlux5cux23page_11}{11}

Hales, Alexander of,
\protect\hyperlink{17_Chapter_Ten__THE_FAILURE_OF_IMAG.xhtmlux5cux23page_255}{255},
\protect\hyperlink{20_ILLUSTRATIONS_FOLLOW_PAGE.xhtmlux5cux23page_321}{321}

Hannibal,
\protect\hyperlink{10_Chapter_Three__THE_HEROIC_DREAM.xhtmlux5cux23page_75}{75}

Hans, acrobat,
\protect\hyperlink{08_Chapter_One__THE_PASSIONATE_INTE.xhtmlux5cux23page_23}{23}

\emph{Hansje in den Kelder},
\protect\hyperlink{13_Chapter_Six__THE_DEPICTION_OF_TH.xhtmlux5cux23page_179}{179}

Hautbourdin, Bastard of St. Pol,
\protect\hyperlink{10_Chapter_Three__THE_HEROIC_DREAM.xhtmlux5cux23page_89}{89}

Hauteville, Pierre de,
\protect\hyperlink{11_Chapter_Four__THE_FORMS_OF_LOVE.xhtmlux5cux23page_140}{140}

Hector,
\protect\hyperlink{10_Chapter_Three__THE_HEROIC_DREAM.xhtmlux5cux23page_76}{76}

Heilo, Frederick von: \emph{Contra peregrinantes},
\protect\hyperlink{13_Chapter_Six__THE_DEPICTION_OF_TH.xhtmlux5cux23page_186}{186}

Hennegowen, House of,
\protect\hyperlink{10_Chapter_Three__THE_HEROIC_DREAM.xhtmlux5cux23page_104}{104}

Hennegowen, Philippa, queen,
\protect\hyperlink{10_Chapter_Three__THE_HEROIC_DREAM.xhtmlux5cux23page_99}{99--}\protect\hyperlink{10_Chapter_Three__THE_HEROIC_DREAM.xhtmlux5cux23page_100}{100}

Hennegowen, Willem of,
\protect\hyperlink{10_Chapter_Three__THE_HEROIC_DREAM.xhtmlux5cux23page_113}{113},
\protect\hyperlink{10_Chapter_Three__THE_HEROIC_DREAM.xhtmlux5cux23page_118}{118}

\emph{henouars} (salt-weighers),
\protect\hyperlink{09_Chapter_Two__THE_CRAVING_FOR_A_M.xhtmlux5cux23page_51}{51}

Henry IV (king of England),
\protect\hyperlink{10_Chapter_Three__THE_HEROIC_DREAM.xhtmlux5cux23page_75}{75},
\protect\hyperlink{10_Chapter_Three__THE_HEROIC_DREAM.xhtmlux5cux23page_107}{107--}\protect\hyperlink{10_Chapter_Three__THE_HEROIC_DREAM.xhtmlux5cux23page_108}{108}

Henry V (king of England):
Agin\protect\hypertarget{25_INDEX.xhtmlux5cux23page_459}{}{}court,
battle of,
\protect\hyperlink{10_Chapter_Three__THE_HEROIC_DREAM.xhtmlux5cux23page_111}{111},
\protect\hyperlink{10_Chapter_Three__THE_HEROIC_DREAM.xhtmlux5cux23page_114}{114};
body boiled,
\protect\hyperlink{12_Chapter_Five__THE_VISION_OF_DEAT.xhtmlux5cux23page_164}{164};
chivalry,
\protect\hyperlink{10_Chapter_Three__THE_HEROIC_DREAM.xhtmlux5cux23page_111}{111};
crusade,
\protect\hyperlink{10_Chapter_Three__THE_HEROIC_DREAM.xhtmlux5cux23page_106}{106};
death,
\protect\hyperlink{10_Chapter_Three__THE_HEROIC_DREAM.xhtmlux5cux23page_106}{106};
seige of Meaux,
\protect\hyperlink{18_Chapter_Eleven__THE_FORMS_OF_THO.xhtmlux5cux23page_286}{286}

Henry VI (king of England): and Margaret of Anjou,
\protect\hyperlink{08_Chapter_One__THE_PASSIONATE_INTE.xhtmlux5cux23page_13}{13};
coronation,
\protect\hyperlink{09_Chapter_Two__THE_CRAVING_FOR_A_M.xhtmlux5cux23page_52}{52};
insane,
\protect\hyperlink{08_Chapter_One__THE_PASSIONATE_INTE.xhtmlux5cux23page_13}{13};
\emph{preux},
\protect\hyperlink{10_Chapter_Three__THE_HEROIC_DREAM.xhtmlux5cux23page_77}{77}

Hercules,
\protect\hyperlink{10_Chapter_Three__THE_HEROIC_DREAM.xhtmlux5cux23page_75}{75},
\protect\hyperlink{20_ILLUSTRATIONS_FOLLOW_PAGE.xhtmlux5cux23page_306}{306}

Herp, Hendrik van,
\protect\hyperlink{15_Chapter_Eight__RELIGIOUS_EXCITAT.xhtmlux5cux23page_229}{229}

Hesdin, castle of,
\protect\hyperlink{10_Chapter_Three__THE_HEROIC_DREAM.xhtmlux5cux23page_108}{108},
\protect\hyperlink{11_Chapter_Four__THE_FORMS_OF_LOVE.xhtmlux5cux23page_128}{128},
\protect\hyperlink{16_Chapter_Nine__THE_DECLINE_OF_SYM.xhtmlux5cux23page_244}{244},
\protect\hyperlink{20_ILLUSTRATIONS_FOLLOW_PAGE.xhtmlux5cux23page_299}{299},
\protect\hyperlink{20_ILLUSTRATIONS_FOLLOW_PAGE.xhtmlux5cux23page_310}{310}

\emph{Heures d'Ailly, Les belles},
\protect\hyperlink{21_Chapter_Thirteen__IMAGE_AND_WORD.xhtmlux5cux23page_376}{376}

\emph{Histoire des Ducs de Bourgogne} (De Barante),
\protect\hyperlink{19_Chapter_Twelve__ART_IN_LIFE.xhtmlux5cux23page_294}{294}

\emph{Hoecken},
\protect\hyperlink{08_Chapter_One__THE_PASSIONATE_INTE.xhtmlux5cux23page_18}{18}.
\emph{See also}
\protect\hyperlink{25_INDEX.xhtmlux5cux23id_2265}{political parties}

Holando, Francesco,
\protect\hyperlink{20_ILLUSTRATIONS_FOLLOW_PAGE.xhtmlux5cux23page_320}{320}

Holbein, Hans,
\protect\hyperlink{12_Chapter_Five__THE_VISION_OF_DEAT.xhtmlux5cux23page_165}{165},
\protect\hyperlink{12_Chapter_Five__THE_VISION_OF_DEAT.xhtmlux5cux23page_166}{166}

Holy Martyrs, Fourteen,
\protect\hyperlink{13_Chapter_Six__THE_DEPICTION_OF_TH.xhtmlux5cux23page_192}{192},
\protect\hyperlink{13_Chapter_Six__THE_DEPICTION_OF_TH.xhtmlux5cux23page_197}{197},
\protect\hyperlink{13_Chapter_Six__THE_DEPICTION_OF_TH.xhtmlux5cux23page_198}{198}

Host: inappropriate behavior and,
\protect\hyperlink{13_Chapter_Six__THE_DEPICTION_OF_TH.xhtmlux5cux23page_178}{178--}\protect\hyperlink{13_Chapter_Six__THE_DEPICTION_OF_TH.xhtmlux5cux23page_179}{179},
\protect\hyperlink{13_Chapter_Six__THE_DEPICTION_OF_TH.xhtmlux5cux23page_183}{183--}\protect\hyperlink{13_Chapter_Six__THE_DEPICTION_OF_TH.xhtmlux5cux23page_184}{184},
\protect\hyperlink{14_Chapter_Seven__THE_PIOUS_PERSONA.xhtmlux5cux23page_206}{206};
monstrance,
\protect\hyperlink{16_Chapter_Nine__THE_DECLINE_OF_SYM.xhtmlux5cux23page_234}{234};
symbolism,
\protect\hyperlink{16_Chapter_Nine__THE_DECLINE_OF_SYM.xhtmlux5cux23page_239}{239}.
\emph{See also} Sacraments

Hours of Turin,
\protect\hyperlink{20_ILLUSTRATIONS_FOLLOW_PAGE.xhtmlux5cux23page_315}{315}

Hugo of St. Victor,
\protect\hyperlink{14_Chapter_Seven__THE_PIOUS_PERSONA.xhtmlux5cux23page_218}{218},
\protect\hyperlink{20_ILLUSTRATIONS_FOLLOW_PAGE.xhtmlux5cux23page_321}{321}

Huguenin, squire,
\protect\hyperlink{10_Chapter_Three__THE_HEROIC_DREAM.xhtmlux5cux23page_79}{79}

Hundred Years War, the,
\emph{\protect\hyperlink{10_Chapter_Three__THE_HEROIC_DREAM.xhtmlux5cux23page_66}{66}}

\emph{hungerdocks},
\protect\hyperlink{16_Chapter_Nine__THE_DECLINE_OF_SYM.xhtmlux5cux23page_246}{246}

Hutten, Ulrich von,
\protect\hyperlink{09_Chapter_Two__THE_CRAVING_FOR_A_M.xhtmlux5cux23page_31}{31}

Huysmans, Joris Karl,
\protect\hyperlink{21_Chapter_Thirteen__IMAGE_AND_WORD.xhtmlux5cux23page_363}{363}

Ignatius, Saint. \emph{See} Loyola

I'Isle Adam,
\protect\hyperlink{09_Chapter_Two__THE_CRAVING_FOR_A_M.xhtmlux5cux23page_50}{50}

\emph{Imitatio Christi} (Thomas à Kempis),
\protect\hyperlink{17_Chapter_Ten__THE_FAILURE_OF_IMAG.xhtmlux5cux23page_266}{266},
\protect\hyperlink{18_Chapter_Eleven__THE_FORMS_OF_THO.xhtmlux5cux23page_275}{275}

Innocent III, pope,
\protect\hyperlink{12_Chapter_Five__THE_VISION_OF_DEAT.xhtmlux5cux23page_160}{160}

Isabella de Bourbon (countess of Charolais),
\protect\hyperlink{09_Chapter_Two__THE_CRAVING_FOR_A_M.xhtmlux5cux23page_56}{56--}\protect\hyperlink{09_Chapter_Two__THE_CRAVING_FOR_A_M.xhtmlux5cux23page_58}{58}

Isabella of Bavaria (queen of France),
\protect\hyperlink{08_Chapter_One__THE_PASSIONATE_INTE.xhtmlux5cux23page_11}{11},
\protect\hyperlink{11_Chapter_Four__THE_FORMS_OF_LOVE.xhtmlux5cux23page_129}{129},
\protect\hyperlink{20_ILLUSTRATIONS_FOLLOW_PAGE.xhtmlux5cux23page_298}{298},
\protect\hyperlink{20_ILLUSTRATIONS_FOLLOW_PAGE.xhtmlux5cux23page_311}{311}

Isabella of France (queen of England),
\protect\hyperlink{20_ILLUSTRATIONS_FOLLOW_PAGE.xhtmlux5cux23page_298}{298}

Isabella of Lorraine,
\protect\hyperlink{12_Chapter_Five__THE_VISION_OF_DEAT.xhtmlux5cux23page_166}{166}

Isolde,
\protect\hyperlink{21_Chapter_Thirteen__IMAGE_AND_WORD.xhtmlux5cux23page_358}{358}

Ives, Saint,
\protect\hyperlink{14_Chapter_Seven__THE_PIOUS_PERSONA.xhtmlux5cux23page_212}{212}

James, Saint,
\protect\hyperlink{13_Chapter_Six__THE_DEPICTION_OF_TH.xhtmlux5cux23page_177}{177},
\protect\hyperlink{13_Chapter_Six__THE_DEPICTION_OF_TH.xhtmlux5cux23page_191}{191--}\protect\hyperlink{13_Chapter_Six__THE_DEPICTION_OF_TH.xhtmlux5cux23page_192}{192}

James, William,
\protect\hyperlink{10_Chapter_Three__THE_HEROIC_DREAM.xhtmlux5cux23page_82}{82},
\protect\hyperlink{14_Chapter_Seven__THE_PIOUS_PERSONA.xhtmlux5cux23page_212}{212--}\protect\hyperlink{14_Chapter_Seven__THE_PIOUS_PERSONA.xhtmlux5cux23page_213}{213},
\protect\hyperlink{15_Chapter_Eight__RELIGIOUS_EXCITAT.xhtmlux5cux23page_226}{226}

James I (king of England),
\protect\hyperlink{09_Chapter_Two__THE_CRAVING_FOR_A_M.xhtmlux5cux23page_58}{58}

Jannequin,
\protect\hyperlink{20_ILLUSTRATIONS_FOLLOW_PAGE.xhtmlux5cux23page_324}{324}

Jason,
\protect\hyperlink{10_Chapter_Three__THE_HEROIC_DREAM.xhtmlux5cux23page_96}{96},
\protect\hyperlink{20_ILLUSTRATIONS_FOLLOW_PAGE.xhtmlux5cux23page_311}{311}

Jeanne of Arc, Saint. \emph{See} Arc, Jeanne d'

Jerome, Saint,
\protect\hyperlink{17_Chapter_Ten__THE_FAILURE_OF_IMAG.xhtmlux5cux23page_255}{255},
\protect\hyperlink{20_ILLUSTRATIONS_FOLLOW_PAGE.xhtmlux5cux23page_319}{319}

Jerusalem, liberation of,
\protect\hyperlink{10_Chapter_Three__THE_HEROIC_DREAM.xhtmlux5cux23page_105}{105--}\protect\hyperlink{10_Chapter_Three__THE_HEROIC_DREAM.xhtmlux5cux23page_106}{106}

Joab,
\protect\hyperlink{18_Chapter_Eleven__THE_FORMS_OF_THO.xhtmlux5cux23page_270}{270}

John, Saint,
\protect\hyperlink{13_Chapter_Six__THE_DEPICTION_OF_TH.xhtmlux5cux23page_177}{177},
\protect\hyperlink{13_Chapter_Six__THE_DEPICTION_OF_TH.xhtmlux5cux23page_200}{200},
\protect\hyperlink{18_Chapter_Eleven__THE_FORMS_OF_THO.xhtmlux5cux23page_271}{271--}\protect\hyperlink{18_Chapter_Eleven__THE_FORMS_OF_THO.xhtmlux5cux23page_272}{272},
\protect\hyperlink{20_ILLUSTRATIONS_FOLLOW_PAGE.xhtmlux5cux23page_308}{308}

John II (king of France): battle of Portiers,
\protect\hyperlink{10_Chapter_Three__THE_HEROIC_DREAM.xhtmlux5cux23page_104}{104};
dukedom of Burgundy,
\protect\hyperlink{10_Chapter_Three__THE_HEROIC_DREAM.xhtmlux5cux23page_104}{104--}\protect\hyperlink{10_Chapter_Three__THE_HEROIC_DREAM.xhtmlux5cux23page_105}{105};
Order of Stars,
\protect\hyperlink{10_Chapter_Three__THE_HEROIC_DREAM.xhtmlux5cux23page_93}{93},
\protect\hyperlink{10_Chapter_Three__THE_HEROIC_DREAM.xhtmlux5cux23page_94}{94},
\protect\hyperlink{10_Chapter_Three__THE_HEROIC_DREAM.xhtmlux5cux23page_111}{111},
\protect\hyperlink{18_Chapter_Eleven__THE_FORMS_OF_THO.xhtmlux5cux23page_277}{277}

John of Capistrano, Saint,
\protect\hyperlink{14_Chapter_Seven__THE_PIOUS_PERSONA.xhtmlux5cux23page_210}{210}

John of Lancaster,
\protect\hyperlink{10_Chapter_Three__THE_HEROIC_DREAM.xhtmlux5cux23page_107}{107}

John of Salisbury,
\protect\hyperlink{10_Chapter_Three__THE_HEROIC_DREAM.xhtmlux5cux23page_121}{121}

John the Baptist, Saint,
\protect\hyperlink{13_Chapter_Six__THE_DEPICTION_OF_TH.xhtmlux5cux23page_177}{177},
\protect\hyperlink{13_Chapter_Six__THE_DEPICTION_OF_TH.xhtmlux5cux23page_200}{200},
\protect\hyperlink{20_ILLUSTRATIONS_FOLLOW_PAGE.xhtmlux5cux23page_319}{319},
\protect\hyperlink{21_Chapter_Thirteen__IMAGE_AND_WORD.xhtmlux5cux23page_337}{337}

John the Fearless (duke of Burgundy),
\protect\hyperlink{08_Chapter_One__THE_PASSIONATE_INTE.xhtmlux5cux23page_3}{3},
\protect\hyperlink{08_Chapter_One__THE_PASSIONATE_INTE.xhtmlux5cux23page_4}{4},
\protect\hyperlink{08_Chapter_One__THE_PASSIONATE_INTE.xhtmlux5cux23page_25}{25},
\protect\hyperlink{09_Chapter_Two__THE_CRAVING_FOR_A_M.xhtmlux5cux23page_58}{58};
and Boucicaut; burial,
\protect\hyperlink{09_Chapter_Two__THE_CRAVING_FOR_A_M.xhtmlux5cux23page_57}{57};
and Capeluche,
\protect\hyperlink{09_Chapter_Two__THE_CRAVING_FOR_A_M.xhtmlux5cux23page_50}{50};
compared to Lamb of God,
\protect\hyperlink{13_Chapter_Six__THE_DEPICTION_OF_TH.xhtmlux5cux23page_181}{181};
and \emph{cours d'amours},
\protect\hyperlink{11_Chapter_Four__THE_FORMS_OF_LOVE.xhtmlux5cux23page_140}{140};
defense of,
\protect\hyperlink{18_Chapter_Eleven__THE_FORMS_OF_THO.xhtmlux5cux23page_271}{271--}\protect\hyperlink{18_Chapter_Eleven__THE_FORMS_OF_THO.xhtmlux5cux23page_273}{273};
emblem,
\protect\hyperlink{18_Chapter_Eleven__THE_FORMS_OF_THO.xhtmlux5cux23page_276}{276};
funeral,
\protect\hyperlink{09_Chapter_Two__THE_CRAVING_FOR_A_M.xhtmlux5cux23page_54}{54};
to Joab,
\protect\hyperlink{18_Chapter_Eleven__THE_FORMS_OF_THO.xhtmlux5cux23page_270}{270};
and Louis d'Orléans,
\protect\hyperlink{08_Chapter_One__THE_PASSIONATE_INTE.xhtmlux5cux23page_12}{12},
\protect\hyperlink{18_Chapter_Eleven__THE_FORMS_OF_THO.xhtmlux5cux23page_271}{271};
and Michelle de France,
\protect\hyperlink{09_Chapter_Two__THE_CRAVING_FOR_A_M.xhtmlux5cux23page_46}{46};
Nicopolis, battle of,
\protect\hyperlink{08_Chapter_One__THE_PASSIONATE_INTE.xhtmlux5cux23page_13}{13};
``Le Pastoralet,''
\protect\hyperlink{11_Chapter_Four__THE_FORMS_OF_LOVE.xhtmlux5cux23page_152}{152},
\protect\hyperlink{21_Chapter_Thirteen__IMAGE_AND_WORD.xhtmlux5cux23page_379}{379};
penance for murder,
\protect\hyperlink{08_Chapter_One__THE_PASSIONATE_INTE.xhtmlux5cux23page_16}{16};
tomb monument,
\protect\hyperlink{20_ILLUSTRATIONS_FOLLOW_PAGE.xhtmlux5cux23page_308}{308}

Joseph, Saint,
\protect\hyperlink{13_Chapter_Six__THE_DEPICTION_OF_TH.xhtmlux5cux23page_177}{177},
\protect\hyperlink{13_Chapter_Six__THE_DEPICTION_OF_TH.xhtmlux5cux23page_193}{193--}\protect\hyperlink{13_Chapter_Six__THE_DEPICTION_OF_TH.xhtmlux5cux23page_196}{196},
\protect\hyperlink{21_Chapter_Thirteen__IMAGE_AND_WORD.xhtmlux5cux23page_361}{361}

Joseph of Arimathia,
\protect\hyperlink{20_ILLUSTRATIONS_FOLLOW_PAGE.xhtmlux5cux23page_318}{318}

Joshua,
\protect\hyperlink{10_Chapter_Three__THE_HEROIC_DREAM.xhtmlux5cux23page_76}{76}

Josquin des Pres,
\protect\hyperlink{20_ILLUSTRATIONS_FOLLOW_PAGE.xhtmlux5cux23page_324}{324}

Jouvencel, Jean. \emph{See} Bueil, Jean de

\emph{Jouvencel, Le}: analysis,
\protect\hyperlink{10_Chapter_Three__THE_HEROIC_DREAM.xhtmlux5cux23page_79}{79--}\protect\hyperlink{10_Chapter_Three__THE_HEROIC_DREAM.xhtmlux5cux23page_81}{81},
\protect\hyperlink{10_Chapter_Three__THE_HEROIC_DREAM.xhtmlux5cux23page_112}{112};
casuistry of war,
\protect\hyperlink{18_Chapter_Eleven__THE_FORMS_OF_THO.xhtmlux5cux23page_278}{278};
etymology in,
\protect\hyperlink{22_Chapter_Fourteen__THE_COMING_OF.xhtmlux5cux23page_386}{386}

Judas Maccabaeus,
\protect\hyperlink{10_Chapter_Three__THE_HEROIC_DREAM.xhtmlux5cux23page_76}{76}

\emph{Judgment of Cambyses} (Gerard David),
\protect\hyperlink{19_Chapter_Twelve__ART_IN_LIFE.xhtmlux5cux23page_297}{297}

Judgment of Paris,
\protect\hyperlink{21_Chapter_Thirteen__IMAGE_AND_WORD.xhtmlux5cux23page_374}{374}

\emph{Judgment of the Emperor Otto} (Dirk Bouts),
\protect\hyperlink{19_Chapter_Twelve__ART_IN_LIFE.xhtmlux5cux23page_297}{297}

\emph{Kabeljauen},
\protect\hyperlink{08_Chapter_One__THE_PASSIONATE_INTE.xhtmlux5cux23page_18}{18}.
\emph{See also} political parties

\protect\hypertarget{25_INDEX.xhtmlux5cux23page_460}{}{}Kempis, Thomas
à, Saint: character,
\protect\hyperlink{17_Chapter_Ten__THE_FAILURE_OF_IMAG.xhtmlux5cux23page_265}{265--}\protect\hyperlink{17_Chapter_Ten__THE_FAILURE_OF_IMAG.xhtmlux5cux23page_267}{267};
ignorance of mundane things,
\protect\hyperlink{15_Chapter_Eight__RELIGIOUS_EXCITAT.xhtmlux5cux23page_222}{222};
\emph{Imitatio Christi},
\protect\hyperlink{17_Chapter_Ten__THE_FAILURE_OF_IMAG.xhtmlux5cux23page_266}{266};
and modern perceptions,
\protect\hyperlink{19_Chapter_Twelve__ART_IN_LIFE.xhtmlux5cux23page_294}{294};
music,
\protect\hyperlink{20_ILLUSTRATIONS_FOLLOW_PAGE.xhtmlux5cux23page_314}{314};
on pilgrimages,
\protect\hyperlink{13_Chapter_Six__THE_DEPICTION_OF_TH.xhtmlux5cux23page_186}{186};
separate sphere from court,
\protect\hyperlink{20_ILLUSTRATIONS_FOLLOW_PAGE.xhtmlux5cux23page_314}{314}.
\emph{See also} political parties

\emph{Kerelsleid},
\protect\hyperlink{10_Chapter_Three__THE_HEROIC_DREAM.xhtmlux5cux23page_65}{65}

Klip, van der. \emph{See} La Roche, Alain de

knighthood: ideal of,
\protect\hyperlink{10_Chapter_Three__THE_HEROIC_DREAM.xhtmlux5cux23page_82}{82--}\protect\hyperlink{10_Chapter_Three__THE_HEROIC_DREAM.xhtmlux5cux23page_86}{86},
\protect\hyperlink{10_Chapter_Three__THE_HEROIC_DREAM.xhtmlux5cux23page_119}{119--}\protect\hyperlink{10_Chapter_Three__THE_HEROIC_DREAM.xhtmlux5cux23page_120}{120};
orders of,
\protect\hyperlink{10_Chapter_Three__THE_HEROIC_DREAM.xhtmlux5cux23page_92}{92}.
\emph{See also} individual orders

La Curne de Sante Palaye,
\protect\hyperlink{10_Chapter_Three__THE_HEROIC_DREAM.xhtmlux5cux23page_101}{101}

La Hire,
\protect\hyperlink{10_Chapter_Three__THE_HEROIC_DREAM.xhtmlux5cux23page_77}{77},
\protect\hyperlink{20_ILLUSTRATIONS_FOLLOW_PAGE.xhtmlux5cux23page_325}{325}

Laiaine, Jacques de: biography,
\protect\hyperlink{10_Chapter_Three__THE_HEROIC_DREAM.xhtmlux5cux23page_78}{78};
compared to \emph{Le Jouvencel},
\protect\hyperlink{10_Chapter_Three__THE_HEROIC_DREAM.xhtmlux5cux23page_81}{81};
Froissart on,
\protect\hyperlink{10_Chapter_Three__THE_HEROIC_DREAM.xhtmlux5cux23page_116}{116};
and Jacques Coeur,
\protect\hyperlink{10_Chapter_Three__THE_HEROIC_DREAM.xhtmlux5cux23page_101}{101};
mignon,
\protect\hyperlink{09_Chapter_Two__THE_CRAVING_FOR_A_M.xhtmlux5cux23page_59}{59};
vows,
\protect\hyperlink{10_Chapter_Three__THE_HEROIC_DREAM.xhtmlux5cux23page_101}{101}

La Marche, Olivier,
\protect\hyperlink{22_Chapter_Fourteen__THE_COMING_OF.xhtmlux5cux23page_394}{394};
allegory,
\protect\hyperlink{21_Chapter_Thirteen__IMAGE_AND_WORD.xhtmlux5cux23page_381}{381};
beauty,
\protect\hyperlink{20_ILLUSTRATIONS_FOLLOW_PAGE.xhtmlux5cux23page_324}{324};
on Chastellain,
\protect\hyperlink{21_Chapter_Thirteen__IMAGE_AND_WORD.xhtmlux5cux23page_342}{342};
colors,
\protect\hyperlink{20_ILLUSTRATIONS_FOLLOW_PAGE.xhtmlux5cux23page_325}{325};
crusade,
\protect\hyperlink{20_ILLUSTRATIONS_FOLLOW_PAGE.xhtmlux5cux23page_305}{305};
on extravagance,
\protect\hyperlink{20_ILLUSTRATIONS_FOLLOW_PAGE.xhtmlux5cux23page_315}{315};
duel of burghers,
\protect\hyperlink{10_Chapter_Three__THE_HEROIC_DREAM.xhtmlux5cux23page_109}{109--}\protect\hyperlink{10_Chapter_Three__THE_HEROIC_DREAM.xhtmlux5cux23page_111}{111};
errors,
\protect\hyperlink{18_Chapter_Eleven__THE_FORMS_OF_THO.xhtmlux5cux23page_283}{283},
\protect\hyperlink{22_Chapter_Fourteen__THE_COMING_OF.xhtmlux5cux23page_386}{386};
generalizes,
\protect\hyperlink{18_Chapter_Eleven__THE_FORMS_OF_THO.xhtmlux5cux23page_282}{282};
glorification of knighthood,
\protect\hyperlink{10_Chapter_Three__THE_HEROIC_DREAM.xhtmlux5cux23page_72}{72},
\protect\hyperlink{10_Chapter_Three__THE_HEROIC_DREAM.xhtmlux5cux23page_93}{93};
on Jacques de Bourbon, king,
\protect\hyperlink{14_Chapter_Seven__THE_PIOUS_PERSONA.xhtmlux5cux23page_209}{209};
lack of empathy with lower orders,
\protect\hyperlink{10_Chapter_Three__THE_HEROIC_DREAM.xhtmlux5cux23page_116}{116};
\emph{manière},
\protect\hyperlink{22_Chapter_Fourteen__THE_COMING_OF.xhtmlux5cux23page_388}{388};
motto,
\protect\hyperlink{09_Chapter_Two__THE_CRAVING_FOR_A_M.xhtmlux5cux23page_34}{34};
music,
\protect\hyperlink{20_ILLUSTRATIONS_FOLLOW_PAGE.xhtmlux5cux23page_324}{324};
old-fashioned,
\protect\hyperlink{22_Chapter_Fourteen__THE_COMING_OF.xhtmlux5cux23page_389}{389},
\protect\hyperlink{22_Chapter_Fourteen__THE_COMING_OF.xhtmlux5cux23page_391}{391};
``Parement et triumphe des dames,''
\protect\hyperlink{12_Chapter_Five__THE_VISION_OF_DEAT.xhtmlux5cux23page_158}{158},
\protect\hyperlink{12_Chapter_Five__THE_VISION_OF_DEAT.xhtmlux5cux23page_161}{161},
\protect\hyperlink{16_Chapter_Nine__THE_DECLINE_OF_SYM.xhtmlux5cux23page_242}{242};
pageant,
\protect\hyperlink{16_Chapter_Nine__THE_DECLINE_OF_SYM.xhtmlux5cux23page_246}{246};
on past glory,
\protect\hyperlink{12_Chapter_Five__THE_VISION_OF_DEAT.xhtmlux5cux23page_158}{158},
\protect\hyperlink{12_Chapter_Five__THE_VISION_OF_DEAT.xhtmlux5cux23page_161}{161};
poetry,
\protect\hyperlink{21_Chapter_Thirteen__IMAGE_AND_WORD.xhtmlux5cux23page_381}{381};
on rage,
\protect\hyperlink{08_Chapter_One__THE_PASSIONATE_INTE.xhtmlux5cux23page_8}{8};
on revenge,
\protect\hyperlink{08_Chapter_One__THE_PASSIONATE_INTE.xhtmlux5cux23page_16}{16};
ritual at court,
\protect\hyperlink{09_Chapter_Two__THE_CRAVING_FOR_A_M.xhtmlux5cux23page_42}{42}

\emph{Lamb, Adoration of the. See} Ghent Altarpiece

Lamprecht,
\protect\hyperlink{17_Chapter_Ten__THE_FAILURE_OF_IMAG.xhtmlux5cux23page_250}{250}

Lancaster, Duke of, John of Gaunt,
\protect\hyperlink{19_Chapter_Twelve__ART_IN_LIFE.xhtmlux5cux23page_297}{297}

Lancaster, House of,
\protect\hyperlink{08_Chapter_One__THE_PASSIONATE_INTE.xhtmlux5cux23page_14}{14}

Lancelot, Sir,
\protect\hyperlink{10_Chapter_Three__THE_HEROIC_DREAM.xhtmlux5cux23page_75}{75},
\protect\hyperlink{10_Chapter_Three__THE_HEROIC_DREAM.xhtmlux5cux23page_85}{85},
\protect\hyperlink{10_Chapter_Three__THE_HEROIC_DREAM.xhtmlux5cux23page_91}{91}

Lannoy, Baudouin de,
\protect\hyperlink{20_ILLUSTRATIONS_FOLLOW_PAGE.xhtmlux5cux23page_307}{307},
\protect\hyperlink{21_Chapter_Thirteen__IMAGE_AND_WORD.xhtmlux5cux23page_331}{331}

Lannoy, Ghillebert de,
\protect\hyperlink{14_Chapter_Seven__THE_PIOUS_PERSONA.xhtmlux5cux23page_205}{205}

Lannoy, Hue de,
\protect\hyperlink{09_Chapter_Two__THE_CRAVING_FOR_A_M.xhtmlux5cux23page_50}{50}

Lannoy, Jean de,
\protect\hyperlink{20_ILLUSTRATIONS_FOLLOW_PAGE.xhtmlux5cux23page_305}{305}

Lannoy family,
\protect\hyperlink{20_ILLUSTRATIONS_FOLLOW_PAGE.xhtmlux5cux23page_316}{316}

La Noue, François de,
\protect\hyperlink{10_Chapter_Three__THE_HEROIC_DREAM.xhtmlux5cux23page_84}{84}

\emph{Lapsu et reparatione justitiae, Liber de} (Nicolas de Clémanges),
\emph{\protect\hyperlink{10_Chapter_Three__THE_HEROIC_DREAM.xhtmlux5cux23page_66}{66}}

La Roche, Alain de,
\protect\hyperlink{15_Chapter_Eight__RELIGIOUS_EXCITAT.xhtmlux5cux23page_232}{232},
\protect\hyperlink{16_Chapter_Nine__THE_DECLINE_OF_SYM.xhtmlux5cux23page_241}{241},
\protect\hyperlink{17_Chapter_Ten__THE_FAILURE_OF_IMAG.xhtmlux5cux23page_266}{266}

La Salle, Antoine,
\protect\hyperlink{12_Chapter_Five__THE_VISION_OF_DEAT.xhtmlux5cux23page_171}{171},
\protect\hyperlink{14_Chapter_Seven__THE_PIOUS_PERSONA.xhtmlux5cux23page_208}{208}

La Tour Landry, Chevalier de,
\protect\hyperlink{10_Chapter_Three__THE_HEROIC_DREAM.xhtmlux5cux23page_98}{98},
\protect\hyperlink{11_Chapter_Four__THE_FORMS_OF_LOVE.xhtmlux5cux23page_147}{147},
\protect\hyperlink{13_Chapter_Six__THE_DEPICTION_OF_TH.xhtmlux5cux23page_185}{185},
\protect\hyperlink{13_Chapter_Six__THE_DEPICTION_OF_TH.xhtmlux5cux23page_195}{195}

Laud, Saint,
\protect\hyperlink{14_Chapter_Seven__THE_PIOUS_PERSONA.xhtmlux5cux23page_215}{215}

Laval, Jeanne de,
\protect\hyperlink{21_Chapter_Thirteen__IMAGE_AND_WORD.xhtmlux5cux23page_351}{351}

Lazarus,
\protect\hyperlink{12_Chapter_Five__THE_VISION_OF_DEAT.xhtmlux5cux23page_167}{167}

\emph{Leal Souvenir} (Jan van Eyck),
\protect\hyperlink{21_Chapter_Thirteen__IMAGE_AND_WORD.xhtmlux5cux23page_331}{331}

Le Fèvre, Jean,
\protect\hyperlink{12_Chapter_Five__THE_VISION_OF_DEAT.xhtmlux5cux23page_164}{164},
\protect\hyperlink{12_Chapter_Five__THE_VISION_OF_DEAT.xhtmlux5cux23page_165}{165}

Le Fèvre de Saint Remy, Jean,
\protect\hyperlink{20_ILLUSTRATIONS_FOLLOW_PAGE.xhtmlux5cux23page_311}{311}

Lefranc, Martin,
\protect\hyperlink{18_Chapter_Eleven__THE_FORMS_OF_THO.xhtmlux5cux23page_290}{290--}\protect\hyperlink{18_Chapter_Eleven__THE_FORMS_OF_THO.xhtmlux5cux23page_291}{291}

Legris, Estienne,
\protect\hyperlink{11_Chapter_Four__THE_FORMS_OF_LOVE.xhtmlux5cux23page_140}{140}

Leo X, pope,
\protect\hyperlink{10_Chapter_Three__THE_HEROIC_DREAM.xhtmlux5cux23page_78}{78}

Leo of Lusiguan, king of Armenia,
\protect\hyperlink{09_Chapter_Two__THE_CRAVING_FOR_A_M.xhtmlux5cux23page_54}{54}

Liein, Saint,
\protect\hyperlink{13_Chapter_Six__THE_DEPICTION_OF_TH.xhtmlux5cux23page_184}{184}

life: enjoyment,
\protect\hyperlink{09_Chapter_Two__THE_CRAVING_FOR_A_M.xhtmlux5cux23page_40}{40--}\protect\hyperlink{09_Chapter_Two__THE_CRAVING_FOR_A_M.xhtmlux5cux23page_41}{41};
three paths,
\protect\hyperlink{09_Chapter_Two__THE_CRAVING_FOR_A_M.xhtmlux5cux23page_36}{36--}\protect\hyperlink{09_Chapter_Two__THE_CRAVING_FOR_A_M.xhtmlux5cux23page_42}{42}

Limburg, Paul van,
\protect\hyperlink{21_Chapter_Thirteen__IMAGE_AND_WORD.xhtmlux5cux23page_353}{353},
\protect\hyperlink{21_Chapter_Thirteen__IMAGE_AND_WORD.xhtmlux5cux23page_356}{356},
\protect\hyperlink{21_Chapter_Thirteen__IMAGE_AND_WORD.xhtmlux5cux23page_363}{363}

Limburg brothers,
\protect\hyperlink{20_ILLUSTRATIONS_FOLLOW_PAGE.xhtmlux5cux23page_313}{313},
\protect\hyperlink{20_ILLUSTRATIONS_FOLLOW_PAGE.xhtmlux5cux23page_315}{315},
\protect\hyperlink{21_Chapter_Thirteen__IMAGE_AND_WORD.xhtmlux5cux23page_352}{352}

``Little Mary of Nymwegen,''
\protect\hyperlink{08_Chapter_One__THE_PASSIONATE_INTE.xhtmlux5cux23page_18}{18}

\emph{Livre de crainte amoureuse, Le} (Jean Berthelemy),
\protect\hyperlink{15_Chapter_Eight__RELIGIOUS_EXCITAT.xhtmlux5cux23page_231}{231}

\emph{Livre de voir-dit, Le} (Guillaume de Machaut),
\protect\hyperlink{11_Chapter_Four__THE_FORMS_OF_LOVE.xhtmlux5cux23page_144}{144--}\protect\hyperlink{11_Chapter_Four__THE_FORMS_OF_LOVE.xhtmlux5cux23page_150}{150},
\protect\hyperlink{22_Chapter_Fourteen__THE_COMING_OF.xhtmlux5cux23page_384}{384}

\emph{Livre du Chevalier de la Tour Landry pour l'enseignement de ses
filles, Le},
\protect\hyperlink{11_Chapter_Four__THE_FORMS_OF_LOVE.xhtmlux5cux23page_147}{147}

Livy, Titus,
\protect\hyperlink{10_Chapter_Three__THE_HEROIC_DREAM.xhtmlux5cux23page_78}{78},
\protect\hyperlink{22_Chapter_Fourteen__THE_COMING_OF.xhtmlux5cux23page_386}{386}

Longuyon, Jacque de,
\protect\hyperlink{10_Chapter_Three__THE_HEROIC_DREAM.xhtmlux5cux23page_76}{76}

Lorraine, René, duke of,
\protect\hyperlink{20_ILLUSTRATIONS_FOLLOW_PAGE.xhtmlux5cux23page_298}{298},
\protect\hyperlink{22_Chapter_Fourteen__THE_COMING_OF.xhtmlux5cux23page_387}{387}

Lorris, Guillaume de,
\protect\hyperlink{11_Chapter_Four__THE_FORMS_OF_LOVE.xhtmlux5cux23page_133}{133--}\protect\hyperlink{11_Chapter_Four__THE_FORMS_OF_LOVE.xhtmlux5cux23page_134}{134}

Louis, Saint. See Louis IX, king of France

Louis IX (king of France),
\protect\hyperlink{09_Chapter_Two__THE_CRAVING_FOR_A_M.xhtmlux5cux23page_37}{37},
\protect\hyperlink{10_Chapter_Three__THE_HEROIC_DREAM.xhtmlux5cux23page_77}{77},
\protect\hyperlink{13_Chapter_Six__THE_DEPICTION_OF_TH.xhtmlux5cux23page_192}{192}

Louis XI (king of France): Church of the Innocents,
\protect\hyperlink{12_Chapter_Five__THE_VISION_OF_DEAT.xhtmlux5cux23page_169}{169};
and Philippe de Commines,
\protect\hyperlink{10_Chapter_Three__THE_HEROIC_DREAM.xhtmlux5cux23page_117}{117};
compared with Jesus,
\protect\hyperlink{15_Chapter_Eight__RELIGIOUS_EXCITAT.xhtmlux5cux23page_221}{221};
contempt for knighthood,
\protect\hyperlink{10_Chapter_Three__THE_HEROIC_DREAM.xhtmlux5cux23page_117}{117};
coronation,
\protect\hyperlink{09_Chapter_Two__THE_CRAVING_FOR_A_M.xhtmlux5cux23page_52}{52};
entry into Paris,
\protect\hyperlink{20_ILLUSTRATIONS_FOLLOW_PAGE.xhtmlux5cux23page_325}{325};
exhumes de Bussy,
\protect\hyperlink{08_Chapter_One__THE_PASSIONATE_INTE.xhtmlux5cux23page_4}{4};
foreign orders,
\protect\hyperlink{10_Chapter_Three__THE_HEROIC_DREAM.xhtmlux5cux23page_93}{93};
and Francis of Paula, Saint,
\protect\hyperlink{14_Chapter_Seven__THE_PIOUS_PERSONA.xhtmlux5cux23page_214}{214--}\protect\hyperlink{14_Chapter_Seven__THE_PIOUS_PERSONA.xhtmlux5cux23page_219}{219};\protect\hypertarget{25_INDEX.xhtmlux5cux23page_461}{}{}hatred
of luxury,
\protect\hyperlink{20_ILLUSTRATIONS_FOLLOW_PAGE.xhtmlux5cux23page_315}{315};
inherits estates of King René,
\protect\hyperlink{08_Chapter_One__THE_PASSIONATE_INTE.xhtmlux5cux23page_13}{13};
Innocents Day,
\protect\hyperlink{13_Chapter_Six__THE_DEPICTION_OF_TH.xhtmlux5cux23page_176}{176};
mignon,
\protect\hyperlink{09_Chapter_Two__THE_CRAVING_FOR_A_M.xhtmlux5cux23page_59}{59};
queen visits Burgundy,
\protect\hyperlink{09_Chapter_Two__THE_CRAVING_FOR_A_M.xhtmlux5cux23page_47}{47};
reaction to bad news,
\protect\hyperlink{09_Chapter_Two__THE_CRAVING_FOR_A_M.xhtmlux5cux23page_56}{56};
superstitious,
\protect\hyperlink{18_Chapter_Eleven__THE_FORMS_OF_THO.xhtmlux5cux23page_288}{288};
and symbolic punishment,
\protect\hyperlink{18_Chapter_Eleven__THE_FORMS_OF_THO.xhtmlux5cux23page_281}{281};
tears,
\protect\hyperlink{08_Chapter_One__THE_PASSIONATE_INTE.xhtmlux5cux23page_8}{8};
in work of Jean Meschinot,
\protect\hyperlink{21_Chapter_Thirteen__IMAGE_AND_WORD.xhtmlux5cux23page_348}{348}

Louis XII (king of France),
\protect\hyperlink{10_Chapter_Three__THE_HEROIC_DREAM.xhtmlux5cux23page_109}{109}

Loyola, Ignatius, Saint,
\protect\hyperlink{14_Chapter_Seven__THE_PIOUS_PERSONA.xhtmlux5cux23page_210}{210}

Luna, Peter of. \emph{See} Benedict VIII

``Lunettes des Princes, Les'' (Jean Meschinot),
\protect\hyperlink{11_Chapter_Four__THE_FORMS_OF_LOVE.xhtmlux5cux23page_153}{153},
\protect\hyperlink{21_Chapter_Thirteen__IMAGE_AND_WORD.xhtmlux5cux23page_381}{381}

Lusignan, castle of,
\protect\hyperlink{20_ILLUSTRATIONS_FOLLOW_PAGE.xhtmlux5cux23page_306}{306},
\protect\hyperlink{21_Chapter_Thirteen__IMAGE_AND_WORD.xhtmlux5cux23page_352}{352}

Lusignan, Pierre de,
\protect\hyperlink{10_Chapter_Three__THE_HEROIC_DREAM.xhtmlux5cux23page_94}{94}

Luther, Martin,
\protect\hyperlink{16_Chapter_Nine__THE_DECLINE_OF_SYM.xhtmlux5cux23page_248}{248}

Luxembourg, Peter of,
\protect\hyperlink{14_Chapter_Seven__THE_PIOUS_PERSONA.xhtmlux5cux23page_212}{212--}\protect\hyperlink{14_Chapter_Seven__THE_PIOUS_PERSONA.xhtmlux5cux23page_214}{214}

Machaut, Guillaume de: casuistry,
\protect\hyperlink{11_Chapter_Four__THE_FORMS_OF_LOVE.xhtmlux5cux23page_143}{143--}\protect\hyperlink{11_Chapter_Four__THE_FORMS_OF_LOVE.xhtmlux5cux23page_144}{144};
color symbolism,
\protect\hyperlink{11_Chapter_Four__THE_FORMS_OF_LOVE.xhtmlux5cux23page_142}{142};
and Deschamps,
\protect\hyperlink{10_Chapter_Three__THE_HEROIC_DREAM.xhtmlux5cux23page_76}{76},
\protect\hyperlink{21_Chapter_Thirteen__IMAGE_AND_WORD.xhtmlux5cux23page_357}{357};
music,
\protect\hyperlink{21_Chapter_Thirteen__IMAGE_AND_WORD.xhtmlux5cux23page_356}{356};
and Peronelle d'Armentières,
\protect\hyperlink{11_Chapter_Four__THE_FORMS_OF_LOVE.xhtmlux5cux23page_142}{142--}\protect\hyperlink{11_Chapter_Four__THE_FORMS_OF_LOVE.xhtmlux5cux23page_147}{147};
pilgrimage,
\protect\hyperlink{11_Chapter_Four__THE_FORMS_OF_LOVE.xhtmlux5cux23page_145}{145};
poetry,
\protect\hyperlink{21_Chapter_Thirteen__IMAGE_AND_WORD.xhtmlux5cux23page_356}{356};
works: \emph{Jugement d'amour},
\protect\hyperlink{11_Chapter_Four__THE_FORMS_OF_LOVE.xhtmlux5cux23page_144}{144};
\emph{Livre de voir-dit},
\protect\hyperlink{11_Chapter_Four__THE_FORMS_OF_LOVE.xhtmlux5cux23page_144}{144},
\protect\hyperlink{11_Chapter_Four__THE_FORMS_OF_LOVE.xhtmlux5cux23page_147}{147}

\emph{Madonna of the Canon van de Paele} (Jan van Eyck),
\protect\hyperlink{21_Chapter_Thirteen__IMAGE_AND_WORD.xhtmlux5cux23page_356}{356}

\emph{Madonna of the Chancellor Rolin} (Jan van Eyck). \emph{See} Autun
Altarpiece

Maelweel, Jean,
\protect\hyperlink{20_ILLUSTRATIONS_FOLLOW_PAGE.xhtmlux5cux23page_310}{310}

Maillard, Olivier,
\protect\hyperlink{08_Chapter_One__THE_PASSIONATE_INTE.xhtmlux5cux23page_6}{6},
\protect\hyperlink{08_Chapter_One__THE_PASSIONATE_INTE.xhtmlux5cux23page_12}{12},
\protect\hyperlink{13_Chapter_Six__THE_DEPICTION_OF_TH.xhtmlux5cux23page_177}{177},
\protect\hyperlink{15_Chapter_Eight__RELIGIOUS_EXCITAT.xhtmlux5cux23page_221}{221},
\protect\hyperlink{18_Chapter_Eleven__THE_FORMS_OF_THO.xhtmlux5cux23page_274}{274}

\emph{Mahâbhârata},
\protect\hyperlink{10_Chapter_Three__THE_HEROIC_DREAM.xhtmlux5cux23page_89}{89}

Mahuot, no

Mâle, Emile,
\protect\hyperlink{12_Chapter_Five__THE_VISION_OF_DEAT.xhtmlux5cux23page_165}{165}

\emph{Malleus Maleficorum} (Henry Institoris, Jacob Sprenger),
\protect\hyperlink{18_Chapter_Eleven__THE_FORMS_OF_THO.xhtmlux5cux23page_286}{286},
\protect\hyperlink{18_Chapter_Eleven__THE_FORMS_OF_THO.xhtmlux5cux23page_289}{289}

Margaret, Saint,
\protect\hyperlink{13_Chapter_Six__THE_DEPICTION_OF_TH.xhtmlux5cux23page_192}{192},
\protect\hyperlink{13_Chapter_Six__THE_DEPICTION_OF_TH.xhtmlux5cux23page_198}{198}

Margaret of Anjou (queen of England),
\protect\hyperlink{08_Chapter_One__THE_PASSIONATE_INTE.xhtmlux5cux23page_13}{13--}\protect\hyperlink{08_Chapter_One__THE_PASSIONATE_INTE.xhtmlux5cux23page_14}{14}

Margaret of Austria,
\protect\hyperlink{14_Chapter_Seven__THE_PIOUS_PERSONA.xhtmlux5cux23page_217}{217},
\protect\hyperlink{20_ILLUSTRATIONS_FOLLOW_PAGE.xhtmlux5cux23page_312}{312}

Margaret of York,
\protect\hyperlink{20_ILLUSTRATIONS_FOLLOW_PAGE.xhtmlux5cux23page_302}{302},
\protect\hyperlink{20_ILLUSTRATIONS_FOLLOW_PAGE.xhtmlux5cux23page_314}{314}

Marie-Antoinette,
\protect\hyperlink{11_Chapter_Four__THE_FORMS_OF_LOVE.xhtmlux5cux23page_150}{150}

Marmion, Colard,
\protect\hyperlink{20_ILLUSTRATIONS_FOLLOW_PAGE.xhtmlux5cux23page_307}{307}

Marmion, Simon,
\protect\hyperlink{20_ILLUSTRATIONS_FOLLOW_PAGE.xhtmlux5cux23page_307}{307}

Marot, Clément,
\protect\hyperlink{11_Chapter_Four__THE_FORMS_OF_LOVE.xhtmlux5cux23page_140}{140},
\protect\hyperlink{21_Chapter_Thirteen__IMAGE_AND_WORD.xhtmlux5cux23page_338}{338}

Martial d'Auvergne,
\protect\hyperlink{11_Chapter_Four__THE_FORMS_OF_LOVE.xhtmlux5cux23page_144}{144},
\protect\hyperlink{12_Chapter_Five__THE_VISION_OF_DEAT.xhtmlux5cux23page_171}{171},
\protect\hyperlink{21_Chapter_Thirteen__IMAGE_AND_WORD.xhtmlux5cux23page_370}{370--}\protect\hyperlink{21_Chapter_Thirteen__IMAGE_AND_WORD.xhtmlux5cux23page_372}{372}

Martianus Capella,
\protect\hyperlink{16_Chapter_Nine__THE_DECLINE_OF_SYM.xhtmlux5cux23page_238}{238}

Martin V, pope,
\protect\hyperlink{16_Chapter_Nine__THE_DECLINE_OF_SYM.xhtmlux5cux23page_234}{234}

\emph{Martyrdom of St. Erasmus} (Dirk Bouts),
\protect\hyperlink{21_Chapter_Thirteen__IMAGE_AND_WORD.xhtmlux5cux23page_376}{376}

\emph{Martydom of St. Hippolytus} (Dirk Bouts),
\protect\hyperlink{21_Chapter_Thirteen__IMAGE_AND_WORD.xhtmlux5cux23page_376}{376}

\emph{Marys at the Sepulchre} (van Eyck brothers),
\protect\hyperlink{21_Chapter_Thirteen__IMAGE_AND_WORD.xhtmlux5cux23page_363}{363}

Maupassant, Guy de,
\protect\hyperlink{11_Chapter_Four__THE_FORMS_OF_LOVE.xhtmlux5cux23page_151}{151}

Maur, Saint,
\protect\hyperlink{13_Chapter_Six__THE_DEPICTION_OF_TH.xhtmlux5cux23page_198}{198}

Medea,
\protect\hyperlink{20_ILLUSTRATIONS_FOLLOW_PAGE.xhtmlux5cux23page_311}{311}

Medici, Lorenzo de,
\protect\hyperlink{09_Chapter_Two__THE_CRAVING_FOR_A_M.xhtmlux5cux23page_39}{39},
\protect\hyperlink{11_Chapter_Four__THE_FORMS_OF_LOVE.xhtmlux5cux23page_127}{127},
\protect\hyperlink{14_Chapter_Seven__THE_PIOUS_PERSONA.xhtmlux5cux23page_215}{215}

\emph{Meditationes vitae Christi} (St. Bonaventura),
\protect\hyperlink{20_ILLUSTRATIONS_FOLLOW_PAGE.xhtmlux5cux23page_318}{318}

Meingre, Jean le. \emph{See} Boucicaut

melancholy,
\protect\hyperlink{09_Chapter_Two__THE_CRAVING_FOR_A_M.xhtmlux5cux23page_32}{32--}\protect\hyperlink{09_Chapter_Two__THE_CRAVING_FOR_A_M.xhtmlux5cux23page_34}{34}

\emph{Méliador} (Froissart),
\protect\hyperlink{10_Chapter_Three__THE_HEROIC_DREAM.xhtmlux5cux23page_72}{72},
\protect\hyperlink{10_Chapter_Three__THE_HEROIC_DREAM.xhtmlux5cux23page_84}{84}

\emph{Melun Madonna} (Foucquet),
\protect\hyperlink{13_Chapter_Six__THE_DEPICTION_OF_TH.xhtmlux5cux23page_182}{182}

Mélusine,
\protect\hyperlink{20_ILLUSTRATIONS_FOLLOW_PAGE.xhtmlux5cux23page_306}{306}

Memling, Hans,
\protect\hyperlink{19_Chapter_Twelve__ART_IN_LIFE.xhtmlux5cux23page_294}{294}

mendicant orders,
\protect\hyperlink{09_Chapter_Two__THE_CRAVING_FOR_A_M.xhtmlux5cux23page_34}{34},
\protect\hyperlink{12_Chapter_Five__THE_VISION_OF_DEAT.xhtmlux5cux23page_156}{156},
\protect\hyperlink{12_Chapter_Five__THE_VISION_OF_DEAT.xhtmlux5cux23page_170}{170};
complaints against Brother of the Common Life by,
\protect\hyperlink{15_Chapter_Eight__RELIGIOUS_EXCITAT.xhtmlux5cux23page_223}{223};
dissolute,
\protect\hyperlink{14_Chapter_Seven__THE_PIOUS_PERSONA.xhtmlux5cux23page_204}{204};
not true poor,
\protect\hyperlink{13_Chapter_Six__THE_DEPICTION_OF_TH.xhtmlux5cux23page_175}{175},
\protect\hyperlink{13_Chapter_Six__THE_DEPICTION_OF_TH.xhtmlux5cux23page_186}{186};
\emph{Piers Plowman},
\protect\hyperlink{14_Chapter_Seven__THE_PIOUS_PERSONA.xhtmlux5cux23page_205}{205};
symbolism of poverty outmoded,
\protect\hyperlink{14_Chapter_Seven__THE_PIOUS_PERSONA.xhtmlux5cux23page_205}{205}.
\emph{See also} beggars

``Mervilles du monde'' (Jean Molinet),
\protect\hyperlink{18_Chapter_Eleven__THE_FORMS_OF_THO.xhtmlux5cux23page_284}{284}

Meschinot, Jean,
\protect\hyperlink{10_Chapter_Three__THE_HEROIC_DREAM.xhtmlux5cux23page_67}{67},
\protect\hyperlink{10_Chapter_Three__THE_HEROIC_DREAM.xhtmlux5cux23page_124}{124--}\protect\hyperlink{10_Chapter_Three__THE_HEROIC_DREAM.xhtmlux5cux23page_125}{125};
allegory,
\protect\hyperlink{21_Chapter_Thirteen__IMAGE_AND_WORD.xhtmlux5cux23page_381}{381};
dialogue,
\protect\hyperlink{21_Chapter_Thirteen__IMAGE_AND_WORD.xhtmlux5cux23page_348}{348--}\protect\hyperlink{21_Chapter_Thirteen__IMAGE_AND_WORD.xhtmlux5cux23page_349}{349};
``Lunettes des princes,''
\protect\hyperlink{11_Chapter_Four__THE_FORMS_OF_LOVE.xhtmlux5cux23page_153}{153};
rondel by,
\protect\hyperlink{21_Chapter_Thirteen__IMAGE_AND_WORD.xhtmlux5cux23page_357}{357}

Metsys, Quintin,
\protect\hyperlink{20_ILLUSTRATIONS_FOLLOW_PAGE.xhtmlux5cux23page_320}{320--}\protect\hyperlink{20_ILLUSTRATIONS_FOLLOW_PAGE.xhtmlux5cux23page_321}{321}

Meun, Jean de. \emph{See} Chopinel, Jean de

Mézières, Philippe de: Celestine cloister, Paris,
\protect\hyperlink{14_Chapter_Seven__THE_PIOUS_PERSONA.xhtmlux5cux23page_208}{208};
crusade,
\protect\hyperlink{10_Chapter_Three__THE_HEROIC_DREAM.xhtmlux5cux23page_106}{106};
funeral plans,
\protect\hyperlink{14_Chapter_Seven__THE_PIOUS_PERSONA.xhtmlux5cux23page_209}{209};
Louis d'Orléans,
\protect\hyperlink{18_Chapter_Eleven__THE_FORMS_OF_THO.xhtmlux5cux23page_272}{272},
\protect\hyperlink{18_Chapter_Eleven__THE_FORMS_OF_THO.xhtmlux5cux23page_287}{287};
Ordre de la Passion,
\protect\hyperlink{10_Chapter_Three__THE_HEROIC_DREAM.xhtmlux5cux23page_71}{71},
\protect\hyperlink{10_Chapter_Three__THE_HEROIC_DREAM.xhtmlux5cux23page_92}{92},
\protect\hyperlink{10_Chapter_Three__THE_HEROIC_DREAM.xhtmlux5cux23page_100}{100},
\protect\hyperlink{14_Chapter_Seven__THE_PIOUS_PERSONA.xhtmlux5cux23page_208}{208};
ostentation,
\protect\hyperlink{14_Chapter_Seven__THE_PIOUS_PERSONA.xhtmlux5cux23page_208}{208--}\protect\hyperlink{14_Chapter_Seven__THE_PIOUS_PERSONA.xhtmlux5cux23page_209}{209};
peace, England and France,
\protect\hyperlink{10_Chapter_Three__THE_HEROIC_DREAM.xhtmlux5cux23page_72}{72};
penance for felons,
\protect\hyperlink{08_Chapter_One__THE_PASSIONATE_INTE.xhtmlux5cux23page_21}{21};
Peter of Luxembourg,
\protect\hyperlink{14_Chapter_Seven__THE_PIOUS_PERSONA.xhtmlux5cux23page_213}{213};
Petrarch,
\protect\hyperlink{22_Chapter_Fourteen__THE_COMING_OF.xhtmlux5cux23page_384}{384};
\emph{Songe du vieil pélerin},
\protect\hyperlink{10_Chapter_Three__THE_HEROIC_DREAM.xhtmlux5cux23page_71}{71};
witchcraft,
\protect\hyperlink{18_Chapter_Eleven__THE_FORMS_OF_THO.xhtmlux5cux23page_287}{287}

Michael Archangel, Saint,
\protect\hyperlink{10_Chapter_Three__THE_HEROIC_DREAM.xhtmlux5cux23page_70}{70--}\protect\hyperlink{10_Chapter_Three__THE_HEROIC_DREAM.xhtmlux5cux23page_71}{71},
\protect\hyperlink{13_Chapter_Six__THE_DEPICTION_OF_TH.xhtmlux5cux23page_192}{192}

Michault, Pierre,
\protect\hyperlink{10_Chapter_Three__THE_HEROIC_DREAM.xhtmlux5cux23page_94}{94},
\protect\hyperlink{21_Chapter_Thirteen__IMAGE_AND_WORD.xhtmlux5cux23page_361}{361}

\protect\hypertarget{25_INDEX.xhtmlux5cux23page_462}{}{}Michelangelo,
\protect\hyperlink{20_ILLUSTRATIONS_FOLLOW_PAGE.xhtmlux5cux23page_309}{309},
\protect\hyperlink{20_ILLUSTRATIONS_FOLLOW_PAGE.xhtmlux5cux23page_320}{320--}\protect\hyperlink{20_ILLUSTRATIONS_FOLLOW_PAGE.xhtmlux5cux23page_321}{321},
\protect\hyperlink{21_Chapter_Thirteen__IMAGE_AND_WORD.xhtmlux5cux23page_335}{335},
\protect\hyperlink{21_Chapter_Thirteen__IMAGE_AND_WORD.xhtmlux5cux23page_375}{375}

Michelle de France (duchess of Burgundy),
\protect\hyperlink{09_Chapter_Two__THE_CRAVING_FOR_A_M.xhtmlux5cux23page_46}{46},
\protect\hyperlink{09_Chapter_Two__THE_CRAVING_FOR_A_M.xhtmlux5cux23page_54}{54}

Midas,
\protect\hyperlink{21_Chapter_Thirteen__IMAGE_AND_WORD.xhtmlux5cux23page_377}{377}

mignon,
\protect\hyperlink{09_Chapter_Two__THE_CRAVING_FOR_A_M.xhtmlux5cux23page_58}{58--}\protect\hyperlink{09_Chapter_Two__THE_CRAVING_FOR_A_M.xhtmlux5cux23page_59}{59}

Miliis, Ambrosius de,
\protect\hyperlink{13_Chapter_Six__THE_DEPICTION_OF_TH.xhtmlux5cux23page_189}{189},
\protect\hyperlink{22_Chapter_Fourteen__THE_COMING_OF.xhtmlux5cux23page_383}{383}

Minims, Order of,
\protect\hyperlink{14_Chapter_Seven__THE_PIOUS_PERSONA.xhtmlux5cux23page_216}{216}

Minorites, Order of,
\protect\hyperlink{08_Chapter_One__THE_PASSIONATE_INTE.xhtmlux5cux23page_21}{21},
\protect\hyperlink{14_Chapter_Seven__THE_PIOUS_PERSONA.xhtmlux5cux23page_216}{216},
\protect\hyperlink{18_Chapter_Eleven__THE_FORMS_OF_THO.xhtmlux5cux23page_278}{278}

Minos,
\protect\hyperlink{21_Chapter_Thirteen__IMAGE_AND_WORD.xhtmlux5cux23page_377}{377}

Mirabeau, marquis de,
\protect\hyperlink{10_Chapter_Three__THE_HEROIC_DREAM.xhtmlux5cux23page_67}{67}

\emph{Miroir de manage} (Eustace Deschamps),
\protect\hyperlink{18_Chapter_Eleven__THE_FORMS_OF_THO.xhtmlux5cux23page_284}{284}

Modern Devotion. See \emph{Devotio moderna}

Molinet, Jean: compares nobles to divinities,
\protect\hyperlink{13_Chapter_Six__THE_DEPICTION_OF_TH.xhtmlux5cux23page_181}{181};
glorification of knighthood,
\protect\hyperlink{10_Chapter_Three__THE_HEROIC_DREAM.xhtmlux5cux23page_72}{72};
judgment of Paris,
\protect\hyperlink{21_Chapter_Thirteen__IMAGE_AND_WORD.xhtmlux5cux23page_347}{347};
\emph{manière},
\protect\hyperlink{22_Chapter_Fourteen__THE_COMING_OF.xhtmlux5cux23page_388}{388};
mocks mendicants,
\protect\hyperlink{14_Chapter_Seven__THE_PIOUS_PERSONA.xhtmlux5cux23page_204}{204};
music,
\protect\hyperlink{20_ILLUSTRATIONS_FOLLOW_PAGE.xhtmlux5cux23page_323}{323};
old-fashioned,
\protect\hyperlink{22_Chapter_Fourteen__THE_COMING_OF.xhtmlux5cux23page_391}{391};
paganism,
\protect\hyperlink{22_Chapter_Fourteen__THE_COMING_OF.xhtmlux5cux23page_394}{394};
\emph{preux},
\protect\hyperlink{10_Chapter_Three__THE_HEROIC_DREAM.xhtmlux5cux23page_77}{77};
profane work,
\protect\hyperlink{14_Chapter_Seven__THE_PIOUS_PERSONA.xhtmlux5cux23page_208}{208};
proverbs,
\protect\hyperlink{16_Chapter_Nine__THE_DECLINE_OF_SYM.xhtmlux5cux23page_247}{247};
\emph{Roman de la rose},
\protect\hyperlink{11_Chapter_Four__THE_FORMS_OF_LOVE.xhtmlux5cux23page_140}{140},
\protect\hyperlink{21_Chapter_Thirteen__IMAGE_AND_WORD.xhtmlux5cux23page_380}{380};
symbolism,
\protect\hyperlink{21_Chapter_Thirteen__IMAGE_AND_WORD.xhtmlux5cux23page_378}{378--}\protect\hyperlink{21_Chapter_Thirteen__IMAGE_AND_WORD.xhtmlux5cux23page_381}{381};
works: \emph{Faictz et Dictz},
\protect\hyperlink{16_Chapter_Nine__THE_DECLINE_OF_SYM.xhtmlux5cux23page_247}{247};
``Mervilles du monde,''
\protect\hyperlink{18_Chapter_Eleven__THE_FORMS_OF_THO.xhtmlux5cux23page_284}{284};
\emph{Resource du petit peu-ple},
\protect\hyperlink{10_Chapter_Three__THE_HEROIC_DREAM.xhtmlux5cux23page_67}{67}

Monstrelet, Enguerrend de: Brother Thomas,
\protect\hyperlink{08_Chapter_One__THE_PASSIONATE_INTE.xhtmlux5cux23page_6}{6};
casuistry,
\protect\hyperlink{10_Chapter_Three__THE_HEROIC_DREAM.xhtmlux5cux23page_114}{114};
inaccurate,
\protect\hyperlink{18_Chapter_Eleven__THE_FORMS_OF_THO.xhtmlux5cux23page_283}{283};
knighthood,
\protect\hyperlink{10_Chapter_Three__THE_HEROIC_DREAM.xhtmlux5cux23page_72}{72};
and modern vision,
\protect\hyperlink{19_Chapter_Twelve__ART_IN_LIFE.xhtmlux5cux23page_294}{294};
superficial,
\protect\hyperlink{18_Chapter_Eleven__THE_FORMS_OF_THO.xhtmlux5cux23page_283}{283}

Montaigu, Jean de,
\protect\hyperlink{08_Chapter_One__THE_PASSIONATE_INTE.xhtmlux5cux23page_4}{4}

Montereau,
\protect\hyperlink{08_Chapter_One__THE_PASSIONATE_INTE.xhtmlux5cux23page_12}{12},
\protect\hyperlink{08_Chapter_One__THE_PASSIONATE_INTE.xhtmlux5cux23page_16}{16},
\protect\hyperlink{10_Chapter_Three__THE_HEROIC_DREAM.xhtmlux5cux23page_105}{105}

Montfort, Jean de,
\protect\hyperlink{14_Chapter_Seven__THE_PIOUS_PERSONA.xhtmlux5cux23page_211}{211}

Montlhéry, battle of,
\protect\hyperlink{08_Chapter_One__THE_PASSIONATE_INTE.xhtmlux5cux23page_28}{28},
\protect\hyperlink{10_Chapter_Three__THE_HEROIC_DREAM.xhtmlux5cux23page_117}{117}

Montreuil, Jean de,
\protect\hyperlink{10_Chapter_Three__THE_HEROIC_DREAM.xhtmlux5cux23page_124}{124},
\protect\hyperlink{11_Chapter_Four__THE_FORMS_OF_LOVE.xhtmlux5cux23page_137}{137--}\protect\hyperlink{11_Chapter_Four__THE_FORMS_OF_LOVE.xhtmlux5cux23page_138}{138},
\protect\hyperlink{11_Chapter_Four__THE_FORMS_OF_LOVE.xhtmlux5cux23page_141}{141}

\emph{Morgante} (Pulci),
\protect\hyperlink{10_Chapter_Three__THE_HEROIC_DREAM.xhtmlux5cux23page_85}{85}

Moses,
\protect\hyperlink{21_Chapter_Thirteen__IMAGE_AND_WORD.xhtmlux5cux23page_336}{336}

\emph{Moses Fountain} (Sluter),
\protect\hyperlink{20_ILLUSTRATIONS_FOLLOW_PAGE.xhtmlux5cux23page_301}{301},
\protect\hyperlink{20_ILLUSTRATIONS_FOLLOW_PAGE.xhtmlux5cux23page_308}{308--}\protect\hyperlink{20_ILLUSTRATIONS_FOLLOW_PAGE.xhtmlux5cux23page_310}{310}

mottoes,
\protect\hyperlink{18_Chapter_Eleven__THE_FORMS_OF_THO.xhtmlux5cux23page_275}{275--}\protect\hyperlink{18_Chapter_Eleven__THE_FORMS_OF_THO.xhtmlux5cux23page_276}{276};
Boucicaut,
\protect\hyperlink{10_Chapter_Three__THE_HEROIC_DREAM.xhtmlux5cux23page_79}{79};
at duel of the two burghers, no; La Marche,
\protect\hyperlink{09_Chapter_Two__THE_CRAVING_FOR_A_M.xhtmlux5cux23page_34}{34};
Mézières,
\protect\hyperlink{14_Chapter_Seven__THE_PIOUS_PERSONA.xhtmlux5cux23page_210}{210};
Rabelais and,
\protect\hyperlink{11_Chapter_Four__THE_FORMS_OF_LOVE.xhtmlux5cux23page_142}{142}

Moulins, Denis de,
\protect\hyperlink{08_Chapter_One__THE_PASSIONATE_INTE.xhtmlux5cux23page_27}{27}

mourning customs,
\protect\hyperlink{09_Chapter_Two__THE_CRAVING_FOR_A_M.xhtmlux5cux23page_53}{53--}\protect\hyperlink{09_Chapter_Two__THE_CRAVING_FOR_A_M.xhtmlux5cux23page_57}{57},
\protect\hyperlink{20_ILLUSTRATIONS_FOLLOW_PAGE.xhtmlux5cux23page_302}{302}

Murillo,
\protect\hyperlink{21_Chapter_Thirteen__IMAGE_AND_WORD.xhtmlux5cux23page_365}{365}

Najera, battle of,
\protect\hyperlink{10_Chapter_Three__THE_HEROIC_DREAM.xhtmlux5cux23page_113}{113}

Nancy, battle of,
\protect\hyperlink{18_Chapter_Eleven__THE_FORMS_OF_THO.xhtmlux5cux23page_284}{284}

\emph{Nativity} (Geertgen tot sint Jans),
\protect\hyperlink{21_Chapter_Thirteen__IMAGE_AND_WORD.xhtmlux5cux23page_347}{347}

Naugrete, battle of. \emph{See} Najera

neo-Platonism,
\protect\hyperlink{16_Chapter_Nine__THE_DECLINE_OF_SYM.xhtmlux5cux23page_237}{237},
\protect\hyperlink{18_Chapter_Eleven__THE_FORMS_OF_THO.xhtmlux5cux23page_268}{268}

Neuss, siege of,
\protect\hyperlink{10_Chapter_Three__THE_HEROIC_DREAM.xhtmlux5cux23page_87}{87},
\protect\hyperlink{10_Chapter_Three__THE_HEROIC_DREAM.xhtmlux5cux23page_114}{114},
\protect\hyperlink{18_Chapter_Eleven__THE_FORMS_OF_THO.xhtmlux5cux23page_283}{283},
\protect\hyperlink{18_Chapter_Eleven__THE_FORMS_OF_THO.xhtmlux5cux23page_286}{286},
\protect\hyperlink{20_ILLUSTRATIONS_FOLLOW_PAGE.xhtmlux5cux23page_314}{314},
\protect\hyperlink{20_ILLUSTRATIONS_FOLLOW_PAGE.xhtmlux5cux23page_323}{323}

Nicholas, Saint,
\protect\hyperlink{17_Chapter_Ten__THE_FAILURE_OF_IMAG.xhtmlux5cux23page_254}{254}

Nicopolis, battle of: Boucicaut at,
\protect\hyperlink{10_Chapter_Three__THE_HEROIC_DREAM.xhtmlux5cux23page_78}{78},
\protect\hyperlink{10_Chapter_Three__THE_HEROIC_DREAM.xhtmlux5cux23page_86}{86};
destruction of French knighthood,
\protect\hyperlink{08_Chapter_One__THE_PASSIONATE_INTE.xhtmlux5cux23page_13}{13},
\protect\hyperlink{10_Chapter_Three__THE_HEROIC_DREAM.xhtmlux5cux23page_86}{86};
poor planning,
\protect\hyperlink{10_Chapter_Three__THE_HEROIC_DREAM.xhtmlux5cux23page_105}{105},
\protect\hyperlink{10_Chapter_Three__THE_HEROIC_DREAM.xhtmlux5cux23page_118}{118};
\emph{poulaines},
\protect\hyperlink{20_ILLUSTRATIONS_FOLLOW_PAGE.xhtmlux5cux23page_302}{302}

Nieppe, castle of,
\protect\hyperlink{21_Chapter_Thirteen__IMAGE_AND_WORD.xhtmlux5cux23page_352}{352}

Nietzsche, Friedrich,
\protect\hyperlink{18_Chapter_Eleven__THE_FORMS_OF_THO.xhtmlux5cux23page_282}{282}

Niles, Saint,
\protect\hyperlink{14_Chapter_Seven__THE_PIOUS_PERSONA.xhtmlux5cux23page_216}{216}

nominalism,
\protect\hyperlink{16_Chapter_Nine__THE_DECLINE_OF_SYM.xhtmlux5cux23page_237}{237}

Noroy, castle of,
\protect\hyperlink{21_Chapter_Thirteen__IMAGE_AND_WORD.xhtmlux5cux23page_352}{352}

\emph{Notre Dame de Paris} (Victor Hugo),
\protect\hyperlink{19_Chapter_Twelve__ART_IN_LIFE.xhtmlux5cux23page_294}{294}

Or, Madam d',
\protect\hyperlink{08_Chapter_One__THE_PASSIONATE_INTE.xhtmlux5cux23page_23}{23}

Orange, William of,
\protect\hyperlink{09_Chapter_Two__THE_CRAVING_FOR_A_M.xhtmlux5cux23page_58}{58},
\protect\hyperlink{21_Chapter_Thirteen__IMAGE_AND_WORD.xhtmlux5cux23page_375}{375}

Oresme, Nicholas,
\protect\hyperlink{22_Chapter_Fourteen__THE_COMING_OF.xhtmlux5cux23page_384}{384}

Orgemont, Nicholas d',
\protect\hyperlink{08_Chapter_One__THE_PASSIONATE_INTE.xhtmlux5cux23page_4}{4};
family,
\protect\hyperlink{08_Chapter_One__THE_PASSIONATE_INTE.xhtmlux5cux23page_27}{27}

Orléans, Charles d': \emph{amoureux de l'observance},
\protect\hyperlink{21_Chapter_Thirteen__IMAGE_AND_WORD.xhtmlux5cux23page_370}{370};
captivity,
\protect\hyperlink{20_ILLUSTRATIONS_FOLLOW_PAGE.xhtmlux5cux23page_316}{316};
modern,
\protect\hyperlink{22_Chapter_Fourteen__THE_COMING_OF.xhtmlux5cux23page_389}{389};
piety and sinfulness,
\emph{\protect\hyperlink{14_Chapter_Seven__THE_PIOUS_PERSONA.xhtmlux5cux23page_206}{206}};
poetry,
\protect\hyperlink{11_Chapter_Four__THE_FORMS_OF_LOVE.xhtmlux5cux23page_131}{131},
\protect\hyperlink{21_Chapter_Thirteen__IMAGE_AND_WORD.xhtmlux5cux23page_330}{330},
\protect\hyperlink{21_Chapter_Thirteen__IMAGE_AND_WORD.xhtmlux5cux23page_367}{367--}\protect\hyperlink{21_Chapter_Thirteen__IMAGE_AND_WORD.xhtmlux5cux23page_370}{370}

Orléans, House of,
\protect\hyperlink{08_Chapter_One__THE_PASSIONATE_INTE.xhtmlux5cux23page_3}{3},
\protect\hyperlink{08_Chapter_One__THE_PASSIONATE_INTE.xhtmlux5cux23page_12}{12}

Orléans, Louis d': and Burgundians,
\protect\hyperlink{18_Chapter_Eleven__THE_FORMS_OF_THO.xhtmlux5cux23page_283}{283};
and Mézières,
\protect\hyperlink{10_Chapter_Three__THE_HEROIC_DREAM.xhtmlux5cux23page_71}{71},
\protect\hyperlink{18_Chapter_Eleven__THE_FORMS_OF_THO.xhtmlux5cux23page_287}{287},
\protect\hyperlink{18_Chapter_Eleven__THE_FORMS_OF_THO.xhtmlux5cux23page_289}{289--}\protect\hyperlink{18_Chapter_Eleven__THE_FORMS_OF_THO.xhtmlux5cux23page_290}{290};
and Peter of Luxembourg,
\protect\hyperlink{14_Chapter_Seven__THE_PIOUS_PERSONA.xhtmlux5cux23page_213}{213};
and \emph{Preux},
\protect\hyperlink{10_Chapter_Three__THE_HEROIC_DREAM.xhtmlux5cux23page_77}{77};
casuistry,
\protect\hyperlink{11_Chapter_Four__THE_FORMS_OF_LOVE.xhtmlux5cux23page_143}{143};
cloister of Celestines,
\protect\hyperlink{14_Chapter_Seven__THE_PIOUS_PERSONA.xhtmlux5cux23page_208}{208--}\protect\hyperlink{14_Chapter_Seven__THE_PIOUS_PERSONA.xhtmlux5cux23page_209}{209};
court case,
\protect\hyperlink{18_Chapter_Eleven__THE_FORMS_OF_THO.xhtmlux5cux23page_270}{270--}\protect\hyperlink{18_Chapter_Eleven__THE_FORMS_OF_THO.xhtmlux5cux23page_273}{273},
\protect\hyperlink{18_Chapter_Eleven__THE_FORMS_OF_THO.xhtmlux5cux23page_285}{285};
device,
\protect\hyperlink{10_Chapter_Three__THE_HEROIC_DREAM.xhtmlux5cux23page_95}{95},
\protect\hyperlink{18_Chapter_Eleven__THE_FORMS_OF_THO.xhtmlux5cux23page_276}{276};
knightly duel,
\protect\hyperlink{10_Chapter_Three__THE_HEROIC_DREAM.xhtmlux5cux23page_107}{107};
mignon,
\protect\hyperlink{09_Chapter_Two__THE_CRAVING_FOR_A_M.xhtmlux5cux23page_59}{59};
murder of,
\protect\hyperlink{08_Chapter_One__THE_PASSIONATE_INTE.xhtmlux5cux23page_12}{12},
\protect\hyperlink{11_Chapter_Four__THE_FORMS_OF_LOVE.xhtmlux5cux23page_152}{152},
\protect\hyperlink{13_Chapter_Six__THE_DEPICTION_OF_TH.xhtmlux5cux23page_181}{181},
\protect\hyperlink{18_Chapter_Eleven__THE_FORMS_OF_THO.xhtmlux5cux23page_270}{270--}\protect\hyperlink{18_Chapter_Eleven__THE_FORMS_OF_THO.xhtmlux5cux23page_271}{271},
\protect\hyperlink{18_Chapter_Eleven__THE_FORMS_OF_THO.xhtmlux5cux23page_282}{282};
order of the Porcupine,
\protect\hyperlink{10_Chapter_Three__THE_HEROIC_DREAM.xhtmlux5cux23page_95}{95};
``Le Pastoralet,''
\protect\hyperlink{21_Chapter_Thirteen__IMAGE_AND_WORD.xhtmlux5cux23page_379}{379};
piety and sinfulness,
\protect\hyperlink{14_Chapter_Seven__THE_PIOUS_PERSONA.xhtmlux5cux23page_206}{206},
\protect\hyperlink{14_Chapter_Seven__THE_PIOUS_PERSONA.xhtmlux5cux23page_213}{213},
\protect\hyperlink{20_ILLUSTRATIONS_FOLLOW_PAGE.xhtmlux5cux23page_317}{317};
witchcraft,
\protect\hyperlink{18_Chapter_Eleven__THE_FORMS_OF_THO.xhtmlux5cux23page_271}{271--}\protect\hyperlink{18_Chapter_Eleven__THE_FORMS_OF_THO.xhtmlux5cux23page_273}{273},
\protect\hyperlink{18_Chapter_Eleven__THE_FORMS_OF_THO.xhtmlux5cux23page_287}{287--}\protect\hyperlink{18_Chapter_Eleven__THE_FORMS_OF_THO.xhtmlux5cux23page_288}{288}

``Orloge amoureus, Le'' (Froissart),
\protect\hyperlink{16_Chapter_Nine__THE_DECLINE_OF_SYM.xhtmlux5cux23page_242}{242}

Ovid,
\protect\hyperlink{21_Chapter_Thirteen__IMAGE_AND_WORD.xhtmlux5cux23page_372}{372},
\protect\hyperlink{22_Chapter_Fourteen__THE_COMING_OF.xhtmlux5cux23page_384}{384}

Paele George van de, canon,
\protect\hyperlink{20_ILLUSTRATIONS_FOLLOW_PAGE.xhtmlux5cux23page_316}{316},
\protect\hyperlink{21_Chapter_Thirteen__IMAGE_AND_WORD.xhtmlux5cux23page_331}{331},
\protect\hyperlink{21_Chapter_Thirteen__IMAGE_AND_WORD.xhtmlux5cux23page_356}{356}

\emph{\protect\hypertarget{25_INDEX.xhtmlux5cux23page_463}{}{}pais, la},
\protect\hyperlink{09_Chapter_Two__THE_CRAVING_FOR_A_M.xhtmlux5cux23page_48}{48--}\protect\hyperlink{09_Chapter_Two__THE_CRAVING_FOR_A_M.xhtmlux5cux23page_49}{49},
\protect\hyperlink{11_Chapter_Four__THE_FORMS_OF_LOVE.xhtmlux5cux23page_147}{147}

Palamedes,
\protect\hyperlink{10_Chapter_Three__THE_HEROIC_DREAM.xhtmlux5cux23page_91}{91}

Pantaleon, Saint,
\protect\hyperlink{13_Chapter_Six__THE_DEPICTION_OF_TH.xhtmlux5cux23page_198}{198}

``Parement et triumph des dames, Le'' (Olivier de la Marche),
\protect\hyperlink{12_Chapter_Five__THE_VISION_OF_DEAT.xhtmlux5cux23page_158}{158},
\protect\hyperlink{12_Chapter_Five__THE_VISION_OF_DEAT.xhtmlux5cux23page_161}{161}

Paris, Geoffroi de,
\protect\hyperlink{18_Chapter_Eleven__THE_FORMS_OF_THO.xhtmlux5cux23page_275}{275}

Parlement of Paris,
\protect\hyperlink{08_Chapter_One__THE_PASSIONATE_INTE.xhtmlux5cux23page_3}{3},
\protect\hyperlink{08_Chapter_One__THE_PASSIONATE_INTE.xhtmlux5cux23page_28}{28},
\protect\hyperlink{09_Chapter_Two__THE_CRAVING_FOR_A_M.xhtmlux5cux23page_52}{52},
\protect\hyperlink{10_Chapter_Three__THE_HEROIC_DREAM.xhtmlux5cux23page_65}{65}

\emph{pas d'armes: de la Bergère},
\protect\hyperlink{22_Chapter_Fourteen__THE_COMING_OF.xhtmlux5cux23page_394}{394};
\emph{de la pélerine},
\protect\hyperlink{10_Chapter_Three__THE_HEROIC_DREAM.xhtmlux5cux23page_89}{89};
\emph{Empris du dragon},
\protect\hyperlink{10_Chapter_Three__THE_HEROIC_DREAM.xhtmlux5cux23page_91}{91};
\emph{Joyesse garde},
\protect\hyperlink{10_Chapter_Three__THE_HEROIC_DREAM.xhtmlux5cux23page_90}{90--}\protect\hyperlink{10_Chapter_Three__THE_HEROIC_DREAM.xhtmlux5cux23page_91}{91};
\emph{La fontaine des pleurs, l'arbre Charlemagne},
\protect\hyperlink{10_Chapter_Three__THE_HEROIC_DREAM.xhtmlux5cux23page_89}{89},
\protect\hyperlink{10_Chapter_Three__THE_HEROIC_DREAM.xhtmlux5cux23page_90}{90--}\protect\hyperlink{10_Chapter_Three__THE_HEROIC_DREAM.xhtmlux5cux23page_91}{91},
\protect\hyperlink{10_Chapter_Three__THE_HEROIC_DREAM.xhtmlux5cux23page_97}{97},
\protect\hyperlink{10_Chapter_Three__THE_HEROIC_DREAM.xhtmlux5cux23page_109}{109},
\protect\hyperlink{13_Chapter_Six__THE_DEPICTION_OF_TH.xhtmlux5cux23page_181}{181}

\emph{Pas de la mort} (Chastellain),
\protect\hyperlink{12_Chapter_Five__THE_VISION_OF_DEAT.xhtmlux5cux23page_158}{158},
\protect\hyperlink{12_Chapter_Five__THE_VISION_OF_DEAT.xhtmlux5cux23page_167}{167}

``Passe temps d'oysiveté, Le'' (Robert Gaguin),
\protect\hyperlink{18_Chapter_Eleven__THE_FORMS_OF_THO.xhtmlux5cux23page_274}{274}

Passion, Order of the,
\protect\hyperlink{10_Chapter_Three__THE_HEROIC_DREAM.xhtmlux5cux23page_71}{71},
\protect\hyperlink{10_Chapter_Three__THE_HEROIC_DREAM.xhtmlux5cux23page_92}{92--}\protect\hyperlink{10_Chapter_Three__THE_HEROIC_DREAM.xhtmlux5cux23page_93}{93},
\protect\hyperlink{10_Chapter_Three__THE_HEROIC_DREAM.xhtmlux5cux23page_100}{100},
\protect\hyperlink{14_Chapter_Seven__THE_PIOUS_PERSONA.xhtmlux5cux23page_208}{208}

``Pastorelet, Le'' (Buriarius),
\protect\hyperlink{11_Chapter_Four__THE_FORMS_OF_LOVE.xhtmlux5cux23page_152}{152},
\protect\hyperlink{21_Chapter_Thirteen__IMAGE_AND_WORD.xhtmlux5cux23page_379}{379},
\protect\hyperlink{22_Chapter_Fourteen__THE_COMING_OF.xhtmlux5cux23page_386}{386--}\protect\hyperlink{22_Chapter_Fourteen__THE_COMING_OF.xhtmlux5cux23page_387}{387},
\protect\hyperlink{22_Chapter_Fourteen__THE_COMING_OF.xhtmlux5cux23page_393}{393--}\protect\hyperlink{22_Chapter_Fourteen__THE_COMING_OF.xhtmlux5cux23page_394}{394}

``Pastoure, Le dit de la'' (Christine de Pisan),
\protect\hyperlink{11_Chapter_Four__THE_FORMS_OF_LOVE.xhtmlux5cux23page_151}{151}

Paul, Saint,
\protect\hyperlink{17_Chapter_Ten__THE_FAILURE_OF_IMAG.xhtmlux5cux23page_265}{265}

Pelias,
\protect\hyperlink{22_Chapter_Fourteen__THE_COMING_OF.xhtmlux5cux23page_386}{386}

Penthesilea,
\protect\hyperlink{10_Chapter_Three__THE_HEROIC_DREAM.xhtmlux5cux23page_77}{77}

Penthièvre, Jeanne de,
\protect\hyperlink{14_Chapter_Seven__THE_PIOUS_PERSONA.xhtmlux5cux23page_211}{211}

\emph{Perceforest} (Froissart),
\protect\hyperlink{10_Chapter_Three__THE_HEROIC_DREAM.xhtmlux5cux23page_84}{84}

Péronne, treaty of,
\protect\hyperlink{10_Chapter_Three__THE_HEROIC_DREAM.xhtmlux5cux23page_93}{93}

Peter, Saint,
\protect\hyperlink{13_Chapter_Six__THE_DEPICTION_OF_TH.xhtmlux5cux23page_177}{177},
\protect\hyperlink{14_Chapter_Seven__THE_PIOUS_PERSONA.xhtmlux5cux23page_215}{215},
\protect\hyperlink{16_Chapter_Nine__THE_DECLINE_OF_SYM.xhtmlux5cux23page_247}{247},
\protect\hyperlink{17_Chapter_Ten__THE_FAILURE_OF_IMAG.xhtmlux5cux23page_255}{255}

Peter of Luxembourg,
\protect\hyperlink{14_Chapter_Seven__THE_PIOUS_PERSONA.xhtmlux5cux23page_210}{210},
\protect\hyperlink{14_Chapter_Seven__THE_PIOUS_PERSONA.xhtmlux5cux23page_213}{213},
\protect\hyperlink{14_Chapter_Seven__THE_PIOUS_PERSONA.xhtmlux5cux23page_214}{214}

Petit, Jean,
\protect\hyperlink{18_Chapter_Eleven__THE_FORMS_OF_THO.xhtmlux5cux23page_270}{270--}\protect\hyperlink{18_Chapter_Eleven__THE_FORMS_OF_THO.xhtmlux5cux23page_274}{274},
\protect\hyperlink{18_Chapter_Eleven__THE_FORMS_OF_THO.xhtmlux5cux23page_285}{285},
\protect\hyperlink{18_Chapter_Eleven__THE_FORMS_OF_THO.xhtmlux5cux23page_287}{287}

\emph{Petit Jehan de Saintré} (Antoine de la Salle),
\protect\hyperlink{10_Chapter_Three__THE_HEROIC_DREAM.xhtmlux5cux23page_101}{101}

Petrarch: Alain Chartier compared to,
\protect\hyperlink{21_Chapter_Thirteen__IMAGE_AND_WORD.xhtmlux5cux23page_338}{338};
in Holy Orders,
\protect\hyperlink{11_Chapter_Four__THE_FORMS_OF_LOVE.xhtmlux5cux23page_146}{146};
humanists and,
\protect\hyperlink{22_Chapter_Fourteen__THE_COMING_OF.xhtmlux5cux23page_384}{384--}\protect\hyperlink{22_Chapter_Fourteen__THE_COMING_OF.xhtmlux5cux23page_385}{385};
love in poetry,
\protect\hyperlink{11_Chapter_Four__THE_FORMS_OF_LOVE.xhtmlux5cux23page_127}{127};
on tournaments,
\protect\hyperlink{10_Chapter_Three__THE_HEROIC_DREAM.xhtmlux5cux23page_88}{88--}\protect\hyperlink{10_Chapter_Three__THE_HEROIC_DREAM.xhtmlux5cux23page_89}{89};
praises Philippe de Vitri,
\protect\hyperlink{10_Chapter_Three__THE_HEROIC_DREAM.xhtmlux5cux23page_121}{121};
Robert Gaguin and,
\protect\hyperlink{22_Chapter_Fourteen__THE_COMING_OF.xhtmlux5cux23page_392}{392}

Petrus Christus,
\protect\hyperlink{21_Chapter_Thirteen__IMAGE_AND_WORD.xhtmlux5cux23page_376}{376}

Phébus, Gaston (count of Foix),
\protect\hyperlink{14_Chapter_Seven__THE_PIOUS_PERSONA.xhtmlux5cux23page_206}{206}

Phébus, Gaston (son),
\protect\hyperlink{09_Chapter_Two__THE_CRAVING_FOR_A_M.xhtmlux5cux23page_59}{59},
\protect\hyperlink{21_Chapter_Thirteen__IMAGE_AND_WORD.xhtmlux5cux23page_349}{349}

Philip le Beau (archduke of Austria),
\protect\hyperlink{09_Chapter_Two__THE_CRAVING_FOR_A_M.xhtmlux5cux23page_47}{47},
\protect\hyperlink{13_Chapter_Six__THE_DEPICTION_OF_TH.xhtmlux5cux23page_181}{181},
\protect\hyperlink{21_Chapter_Thirteen__IMAGE_AND_WORD.xhtmlux5cux23page_374}{374},
\protect\hyperlink{13_Chapter_Six__THE_DEPICTION_OF_TH.xhtmlux5cux23page_181}{181}

Philip the Bold (duke of Burgundy),
\protect\hyperlink{08_Chapter_One__THE_PASSIONATE_INTE.xhtmlux5cux23page_25}{25};
campaign against England,
\protect\hyperlink{10_Chapter_Three__THE_HEROIC_DREAM.xhtmlux5cux23page_107}{107};
cloistered orders and,
\protect\hyperlink{14_Chapter_Seven__THE_PIOUS_PERSONA.xhtmlux5cux23page_208}{208};
\emph{Cour d'amours} and,
\protect\hyperlink{11_Chapter_Four__THE_FORMS_OF_LOVE.xhtmlux5cux23page_140}{140};
Froissart's characterization,
\protect\hyperlink{21_Chapter_Thirteen__IMAGE_AND_WORD.xhtmlux5cux23page_355}{355};
Portiers, battle of,
\protect\hyperlink{10_Chapter_Three__THE_HEROIC_DREAM.xhtmlux5cux23page_104}{104}

Philip the Good (duke of Burgundy),
\protect\hyperlink{08_Chapter_One__THE_PASSIONATE_INTE.xhtmlux5cux23page_8}{8},
\protect\hyperlink{08_Chapter_One__THE_PASSIONATE_INTE.xhtmlux5cux23page_25}{25},
\protect\hyperlink{10_Chapter_Three__THE_HEROIC_DREAM.xhtmlux5cux23page_63}{63},
\protect\hyperlink{14_Chapter_Seven__THE_PIOUS_PERSONA.xhtmlux5cux23page_207}{207},
\protect\hyperlink{18_Chapter_Eleven__THE_FORMS_OF_THO.xhtmlux5cux23page_281}{281},
\protect\hyperlink{19_Chapter_Twelve__ART_IN_LIFE.xhtmlux5cux23page_297}{297},
\protect\hyperlink{20_ILLUSTRATIONS_FOLLOW_PAGE.xhtmlux5cux23page_311}{311},
\protect\hyperlink{20_ILLUSTRATIONS_FOLLOW_PAGE.xhtmlux5cux23page_315}{315},
\protect\hyperlink{21_Chapter_Thirteen__IMAGE_AND_WORD.xhtmlux5cux23page_377}{377};
bad news and,
\protect\hyperlink{09_Chapter_Two__THE_CRAVING_FOR_A_M.xhtmlux5cux23page_55}{55};
brewer of Lille and,
\protect\hyperlink{08_Chapter_One__THE_PASSIONATE_INTE.xhtmlux5cux23page_11}{11},
\protect\hyperlink{10_Chapter_Three__THE_HEROIC_DREAM.xhtmlux5cux23page_64}{64};
Charles the Bold, quarrel with,
\protect\hyperlink{21_Chapter_Thirteen__IMAGE_AND_WORD.xhtmlux5cux23page_343}{343--}\protect\hyperlink{21_Chapter_Thirteen__IMAGE_AND_WORD.xhtmlux5cux23page_346}{346};
Chastellain's portrayal of,
\protect\hyperlink{21_Chapter_Thirteen__IMAGE_AND_WORD.xhtmlux5cux23page_343}{343};
colors,
\protect\hyperlink{20_ILLUSTRATIONS_FOLLOW_PAGE.xhtmlux5cux23page_326}{326};
crusade,
\protect\hyperlink{10_Chapter_Three__THE_HEROIC_DREAM.xhtmlux5cux23page_97}{97},
\protect\hyperlink{10_Chapter_Three__THE_HEROIC_DREAM.xhtmlux5cux23page_101}{101},
\protect\hyperlink{10_Chapter_Three__THE_HEROIC_DREAM.xhtmlux5cux23page_106}{106};
emblem,
\protect\hyperlink{10_Chapter_Three__THE_HEROIC_DREAM.xhtmlux5cux23page_96}{96};
entry into Ghent,
\protect\hyperlink{21_Chapter_Thirteen__IMAGE_AND_WORD.xhtmlux5cux23page_374}{374};
head shaved,
\protect\hyperlink{08_Chapter_One__THE_PASSIONATE_INTE.xhtmlux5cux23page_11}{11};
hears Remonstrance,
\protect\hyperlink{16_Chapter_Nine__THE_DECLINE_OF_SYM.xhtmlux5cux23page_244}{244};
John the Fearless and,
\protect\hyperlink{08_Chapter_One__THE_PASSIONATE_INTE.xhtmlux5cux23page_16}{16};
life of,
\protect\hyperlink{09_Chapter_Two__THE_CRAVING_FOR_A_M.xhtmlux5cux23page_34}{34};
middle class and,
\protect\hyperlink{08_Chapter_One__THE_PASSIONATE_INTE.xhtmlux5cux23page_11}{11},
\protect\hyperlink{10_Chapter_Three__THE_HEROIC_DREAM.xhtmlux5cux23page_64}{64};
mourning dress,
\protect\hyperlink{20_ILLUSTRATIONS_FOLLOW_PAGE.xhtmlux5cux23page_302}{302};
piety and worldliness,
\protect\hyperlink{14_Chapter_Seven__THE_PIOUS_PERSONA.xhtmlux5cux23page_207}{207},
\protect\hyperlink{14_Chapter_Seven__THE_PIOUS_PERSONA.xhtmlux5cux23page_212}{212};
political fantasies,
\protect\hyperlink{10_Chapter_Three__THE_HEROIC_DREAM.xhtmlux5cux23page_106}{106};
politically astute,
\protect\hyperlink{08_Chapter_One__THE_PASSIONATE_INTE.xhtmlux5cux23page_10}{10};
precedence and,
\protect\hyperlink{09_Chapter_Two__THE_CRAVING_FOR_A_M.xhtmlux5cux23page_46}{46};
princely duel,
\protect\hyperlink{10_Chapter_Three__THE_HEROIC_DREAM.xhtmlux5cux23page_107}{107--}\protect\hyperlink{10_Chapter_Three__THE_HEROIC_DREAM.xhtmlux5cux23page_109}{109};
Saint Colette consults,
\protect\hyperlink{14_Chapter_Seven__THE_PIOUS_PERSONA.xhtmlux5cux23page_217}{217};
Saint Denis the Carthusian consults,
\protect\hyperlink{14_Chapter_Seven__THE_PIOUS_PERSONA.xhtmlux5cux23page_217}{217};
superstition,
\protect\hyperlink{18_Chapter_Eleven__THE_FORMS_OF_THO.xhtmlux5cux23page_288}{288};
strategy,
\protect\hyperlink{10_Chapter_Three__THE_HEROIC_DREAM.xhtmlux5cux23page_113}{113};
taxes,
\protect\hyperlink{10_Chapter_Three__THE_HEROIC_DREAM.xhtmlux5cux23page_106}{106};
van Eyck and,
\protect\hyperlink{20_ILLUSTRATIONS_FOLLOW_PAGE.xhtmlux5cux23page_313}{313};
vow,
\protect\hyperlink{10_Chapter_Three__THE_HEROIC_DREAM.xhtmlux5cux23page_101}{101}

Physiocrats,
\protect\hyperlink{11_Chapter_Four__THE_FORMS_OF_LOVE.xhtmlux5cux23page_150}{150}

\emph{Piers Plowman, The Vision of William Concerning},
\protect\hyperlink{14_Chapter_Seven__THE_PIOUS_PERSONA.xhtmlux5cux23page_205}{205}

Pisan, Christine de: Boucicaut praised by,
\protect\hyperlink{10_Chapter_Three__THE_HEROIC_DREAM.xhtmlux5cux23page_79}{79},
\protect\hyperlink{11_Chapter_Four__THE_FORMS_OF_LOVE.xhtmlux5cux23page_140}{140};
Celestines, cloister of,
\protect\hyperlink{14_Chapter_Seven__THE_PIOUS_PERSONA.xhtmlux5cux23page_208}{208};
classicism,
\protect\hyperlink{16_Chapter_Nine__THE_DECLINE_OF_SYM.xhtmlux5cux23page_247}{247},
\protect\hyperlink{21_Chapter_Thirteen__IMAGE_AND_WORD.xhtmlux5cux23page_376}{376},
\protect\hyperlink{22_Chapter_Fourteen__THE_COMING_OF.xhtmlux5cux23page_387}{387};
color symbolism,
\protect\hyperlink{20_ILLUSTRATIONS_FOLLOW_PAGE.xhtmlux5cux23page_326}{326--}\protect\hyperlink{20_ILLUSTRATIONS_FOLLOW_PAGE.xhtmlux5cux23page_327}{327};
irreligion,
\protect\hyperlink{13_Chapter_Six__THE_DEPICTION_OF_TH.xhtmlux5cux23page_185}{185};
\emph{Roman de la rose} and,
\protect\hyperlink{11_Chapter_Four__THE_FORMS_OF_LOVE.xhtmlux5cux23page_137}{137},
\protect\hyperlink{11_Chapter_Four__THE_FORMS_OF_LOVE.xhtmlux5cux23page_141}{141},
\protect\hyperlink{11_Chapter_Four__THE_FORMS_OF_LOVE.xhtmlux5cux23page_155}{155};
talent,
\protect\hyperlink{21_Chapter_Thirteen__IMAGE_AND_WORD.xhtmlux5cux23page_358}{358--}\protect\hyperlink{21_Chapter_Thirteen__IMAGE_AND_WORD.xhtmlux5cux23page_360}{360};
works: ``Epistre d'Othéa,''
\protect\hyperlink{16_Chapter_Nine__THE_DECLINE_OF_SYM.xhtmlux5cux23page_247}{247},
\protect\hyperlink{21_Chapter_Thirteen__IMAGE_AND_WORD.xhtmlux5cux23page_376}{376};
``Le dit de la pastoure,''
\protect\hyperlink{11_Chapter_Four__THE_FORMS_OF_LOVE.xhtmlux5cux23page_151}{151}

Pius, Saint,
\protect\hyperlink{13_Chapter_Six__THE_DEPICTION_OF_TH.xhtmlux5cux23page_200}{200}

Platonic idealism,
\protect\hyperlink{16_Chapter_Nine__THE_DECLINE_OF_SYM.xhtmlux5cux23page_236}{236}

\emph{plourants},
\protect\hyperlink{09_Chapter_Two__THE_CRAVING_FOR_A_M.xhtmlux5cux23page_55}{55},
\protect\hyperlink{20_ILLUSTRATIONS_FOLLOW_PAGE.xhtmlux5cux23page_301}{301},
\protect\hyperlink{20_ILLUSTRATIONS_FOLLOW_PAGE.xhtmlux5cux23page_310}{310}

Plouvier, Jacotin,
\protect\hyperlink{10_Chapter_Three__THE_HEROIC_DREAM.xhtmlux5cux23page_107}{107},
\protect\hyperlink{10_Chapter_Three__THE_HEROIC_DREAM.xhtmlux5cux23page_109}{109--}\protect\hyperlink{10_Chapter_Three__THE_HEROIC_DREAM.xhtmlux5cux23page_111}{111}

Polignac, house of,
\protect\hyperlink{20_ILLUSTRATIONS_FOLLOW_PAGE.xhtmlux5cux23page_298}{298}

\protect\hypertarget{25_INDEX.xhtmlux5cux23id_2265}{}{}political
parties,
\protect\hyperlink{08_Chapter_One__THE_PASSIONATE_INTE.xhtmlux5cux23page_18}{18--}\protect\hyperlink{08_Chapter_One__THE_PASSIONATE_INTE.xhtmlux5cux23page_19}{19}

Porcupine, Order of the,
\protect\hyperlink{10_Chapter_Three__THE_HEROIC_DREAM.xhtmlux5cux23page_94}{94},
\protect\hyperlink{18_Chapter_Eleven__THE_FORMS_OF_THO.xhtmlux5cux23page_276}{276}

Porete, Marguerite,
\protect\hyperlink{15_Chapter_Eight__RELIGIOUS_EXCITAT.xhtmlux5cux23page_229}{229}

Portiers, Alienor de,
\protect\hyperlink{09_Chapter_Two__THE_CRAVING_FOR_A_M.xhtmlux5cux23page_57}{57},
\protect\hyperlink{18_Chapter_Eleven__THE_FORMS_OF_THO.xhtmlux5cux23page_268}{268--}\protect\hyperlink{18_Chapter_Eleven__THE_FORMS_OF_THO.xhtmlux5cux23page_269}{269}

Pot, Philippe,
\protect\hyperlink{08_Chapter_One__THE_PASSIONATE_INTE.xhtmlux5cux23page_11}{11},
\protect\hyperlink{10_Chapter_Three__THE_HEROIC_DREAM.xhtmlux5cux23page_102}{102}

Pouchier, Etienne (bishop of Paris),
\protect\hyperlink{08_Chapter_One__THE_PASSIONATE_INTE.xhtmlux5cux23page_21}{21}

\emph{poulaines},
\protect\hyperlink{20_ILLUSTRATIONS_FOLLOW_PAGE.xhtmlux5cux23page_302}{302}

\protect\hypertarget{25_INDEX.xhtmlux5cux23page_464}{}{}precedence,
\protect\hyperlink{09_Chapter_Two__THE_CRAVING_FOR_A_M.xhtmlux5cux23page_44}{44--}\protect\hyperlink{09_Chapter_Two__THE_CRAVING_FOR_A_M.xhtmlux5cux23page_50}{50},
\protect\hyperlink{18_Chapter_Eleven__THE_FORMS_OF_THO.xhtmlux5cux23page_277}{277--}\protect\hyperlink{18_Chapter_Eleven__THE_FORMS_OF_THO.xhtmlux5cux23page_278}{278}

prefiguration,
\protect\hyperlink{16_Chapter_Nine__THE_DECLINE_OF_SYM.xhtmlux5cux23page_239}{239}

Prés, Josquin de,
\protect\hyperlink{20_ILLUSTRATIONS_FOLLOW_PAGE.xhtmlux5cux23page_324}{324}

\protect\hypertarget{25_INDEX.xhtmlux5cux23id_2270}{}{}Preux, Les Neuf,
\protect\hyperlink{10_Chapter_Three__THE_HEROIC_DREAM.xhtmlux5cux23page_76}{76--}\protect\hyperlink{10_Chapter_Three__THE_HEROIC_DREAM.xhtmlux5cux23page_78}{78},
\protect\hyperlink{18_Chapter_Eleven__THE_FORMS_OF_THO.xhtmlux5cux23page_285}{285},
\protect\hyperlink{22_Chapter_Fourteen__THE_COMING_OF.xhtmlux5cux23page_387}{387}

pride, sin of,
\protect\hyperlink{08_Chapter_One__THE_PASSIONATE_INTE.xhtmlux5cux23page_26}{26--}\protect\hyperlink{08_Chapter_One__THE_PASSIONATE_INTE.xhtmlux5cux23page_27}{27}

\emph{Proverbes del vilain},
\protect\hyperlink{10_Chapter_Three__THE_HEROIC_DREAM.xhtmlux5cux23page_65}{65}

proverbs,
\protect\hyperlink{12_Chapter_Five__THE_VISION_OF_DEAT.xhtmlux5cux23page_165}{165},
\protect\hyperlink{15_Chapter_Eight__RELIGIOUS_EXCITAT.xhtmlux5cux23page_227}{227},
\protect\hyperlink{16_Chapter_Nine__THE_DECLINE_OF_SYM.xhtmlux5cux23page_246}{246},
\protect\hyperlink{18_Chapter_Eleven__THE_FORMS_OF_THO.xhtmlux5cux23page_273}{273--}\protect\hyperlink{18_Chapter_Eleven__THE_FORMS_OF_THO.xhtmlux5cux23page_275}{275}.
See \emph{also} mottoes

Prudentius,
\protect\hyperlink{16_Chapter_Nine__THE_DECLINE_OF_SYM.xhtmlux5cux23page_238}{238}

Pulci, Luigi,
\protect\hyperlink{10_Chapter_Three__THE_HEROIC_DREAM.xhtmlux5cux23page_85}{85}

\emph{Purification of the Virgin} (Limburg brothers),
\protect\hyperlink{21_Chapter_Thirteen__IMAGE_AND_WORD.xhtmlux5cux23page_363}{363}

Pyramus and Thisby,
\protect\hyperlink{11_Chapter_Four__THE_FORMS_OF_LOVE.xhtmlux5cux23page_126}{126}

\emph{Quadriloge invectif} (Alain Chartier),
\protect\hyperlink{10_Chapter_Three__THE_HEROIC_DREAM.xhtmlux5cux23page_67}{67}

\emph{Quatre Dames, Le livre des} (Alain Chartier),
\protect\hyperlink{21_Chapter_Thirteen__IMAGE_AND_WORD.xhtmlux5cux23page_339}{339}

\emph{Quinze joyes de mariage, Les},
\protect\hyperlink{11_Chapter_Four__THE_FORMS_OF_LOVE.xhtmlux5cux23page_155}{155},
\protect\hyperlink{13_Chapter_Six__THE_DEPICTION_OF_TH.xhtmlux5cux23page_182}{182},
\protect\hyperlink{13_Chapter_Six__THE_DEPICTION_OF_TH.xhtmlux5cux23page_185}{185},
\protect\hyperlink{21_Chapter_Thirteen__IMAGE_AND_WORD.xhtmlux5cux23page_372}{372}

Quiricus, Saint,
\protect\hyperlink{17_Chapter_Ten__THE_FAILURE_OF_IMAG.xhtmlux5cux23page_254}{254}

Rabelais, François de,
\protect\hyperlink{11_Chapter_Four__THE_FORMS_OF_LOVE.xhtmlux5cux23page_142}{142},
\protect\hyperlink{13_Chapter_Six__THE_DEPICTION_OF_TH.xhtmlux5cux23page_200}{200},
\protect\hyperlink{21_Chapter_Thirteen__IMAGE_AND_WORD.xhtmlux5cux23page_329}{329},
\protect\hyperlink{22_Chapter_Fourteen__THE_COMING_OF.xhtmlux5cux23page_388}{388}

Rais, Gilles de,
\protect\hyperlink{10_Chapter_Three__THE_HEROIC_DREAM.xhtmlux5cux23page_65}{65},
\protect\hyperlink{14_Chapter_Seven__THE_PIOUS_PERSONA.xhtmlux5cux23page_206}{206},
\protect\hyperlink{18_Chapter_Eleven__THE_FORMS_OF_THO.xhtmlux5cux23page_289}{289}

Rallant, Gaultier,
\protect\hyperlink{09_Chapter_Two__THE_CRAVING_FOR_A_M.xhtmlux5cux23page_45}{45}

\emph{Rationale divinorum officiorum} (Durandus),
\protect\hyperlink{16_Chapter_Nine__THE_DECLINE_OF_SYM.xhtmlux5cux23page_248}{248}

Ravenstein, Beatrix of,
\protect\hyperlink{20_ILLUSTRATIONS_FOLLOW_PAGE.xhtmlux5cux23page_314}{314}

Ravenstein, Phillip of,
\protect\hyperlink{11_Chapter_Four__THE_FORMS_OF_LOVE.xhtmlux5cux23page_153}{153}

Raynaud, Gaston,
\protect\hyperlink{10_Chapter_Three__THE_HEROIC_DREAM.xhtmlux5cux23page_124}{124}

realism,
\protect\hyperlink{16_Chapter_Nine__THE_DECLINE_OF_SYM.xhtmlux5cux23page_237}{237--}\protect\hyperlink{16_Chapter_Nine__THE_DECLINE_OF_SYM.xhtmlux5cux23page_239}{239}

Rebreviettes, Jennet de,
\protect\hyperlink{10_Chapter_Three__THE_HEROIC_DREAM.xhtmlux5cux23page_103}{103}

``Reconfort de Madam du Fresne, Le'' (Antoine de la Salle),
\protect\hyperlink{12_Chapter_Five__THE_VISION_OF_DEAT.xhtmlux5cux23page_171}{171}

\emph{Reformatione, De} (Pierre d'Ailly),
\protect\hyperlink{13_Chapter_Six__THE_DEPICTION_OF_TH.xhtmlux5cux23page_175}{175}

Regnault d'Arincourt,
\protect\hyperlink{11_Chapter_Four__THE_FORMS_OF_LOVE.xhtmlux5cux23page_141}{141}

``Regnault et Jehanneton'' (King René),
\protect\hyperlink{21_Chapter_Thirteen__IMAGE_AND_WORD.xhtmlux5cux23page_351}{351}

Rembrandt,
\protect\hyperlink{20_ILLUSTRATIONS_FOLLOW_PAGE.xhtmlux5cux23page_312}{312},
\protect\hyperlink{21_Chapter_Thirteen__IMAGE_AND_WORD.xhtmlux5cux23page_365}{365}

René, king of Anjou: chivalry and,
\protect\hyperlink{10_Chapter_Three__THE_HEROIC_DREAM.xhtmlux5cux23page_74}{74};
colors,
\protect\hyperlink{10_Chapter_Three__THE_HEROIC_DREAM.xhtmlux5cux23page_91}{91},
\protect\hyperlink{20_ILLUSTRATIONS_FOLLOW_PAGE.xhtmlux5cux23page_326}{326};
Day of Innocents and,
\protect\hyperlink{13_Chapter_Six__THE_DEPICTION_OF_TH.xhtmlux5cux23page_176}{176};
\emph{Emprise du dragon},
\protect\hyperlink{10_Chapter_Three__THE_HEROIC_DREAM.xhtmlux5cux23page_91}{91};
fortune and,
\protect\hyperlink{08_Chapter_One__THE_PASSIONATE_INTE.xhtmlux5cux23page_13}{13};
\emph{Joyesse garde},
\protect\hyperlink{10_Chapter_Three__THE_HEROIC_DREAM.xhtmlux5cux23page_91}{91};
Laval, Jeanne de and,
\protect\hyperlink{10_Chapter_Three__THE_HEROIC_DREAM.xhtmlux5cux23page_71}{71},
\protect\hyperlink{21_Chapter_Thirteen__IMAGE_AND_WORD.xhtmlux5cux23page_351}{351};
painting by,
\protect\hyperlink{12_Chapter_Five__THE_VISION_OF_DEAT.xhtmlux5cux23page_161}{161};
\emph{Pas d'armes de bergère},
\protect\hyperlink{10_Chapter_Three__THE_HEROIC_DREAM.xhtmlux5cux23page_91}{91},
\protect\hyperlink{11_Chapter_Four__THE_FORMS_OF_LOVE.xhtmlux5cux23page_151}{151--}\protect\hyperlink{11_Chapter_Four__THE_FORMS_OF_LOVE.xhtmlux5cux23page_152}{152};
piety,
\protect\hyperlink{14_Chapter_Seven__THE_PIOUS_PERSONA.xhtmlux5cux23page_209}{209};
tomb monument,
\protect\hyperlink{12_Chapter_Five__THE_VISION_OF_DEAT.xhtmlux5cux23page_166}{166};
worldly,
\protect\hyperlink{14_Chapter_Seven__THE_PIOUS_PERSONA.xhtmlux5cux23page_206}{206};
works: \emph{Cuer d'amours espris},
\protect\hyperlink{21_Chapter_Thirteen__IMAGE_AND_WORD.xhtmlux5cux23page_347}{347},
\protect\hyperlink{21_Chapter_Thirteen__IMAGE_AND_WORD.xhtmlux5cux23page_368}{368};
\emph{L'abuzé} (attributed),
\protect\hyperlink{10_Chapter_Three__THE_HEROIC_DREAM.xhtmlux5cux23page_124}{124};
``Regnault et Jehanneton,''
\protect\hyperlink{21_Chapter_Thirteen__IMAGE_AND_WORD.xhtmlux5cux23page_351}{351}

\emph{Resource du petit peuple} (Molinet),
\protect\hyperlink{10_Chapter_Three__THE_HEROIC_DREAM.xhtmlux5cux23page_67}{67}

rhetoricians,
\protect\hyperlink{22_Chapter_Fourteen__THE_COMING_OF.xhtmlux5cux23page_389}{389}

Richard, Brother,
\protect\hyperlink{08_Chapter_One__THE_PASSIONATE_INTE.xhtmlux5cux23page_4}{4},
\protect\hyperlink{08_Chapter_One__THE_PASSIONATE_INTE.xhtmlux5cux23page_7}{7},
\protect\hyperlink{08_Chapter_One__THE_PASSIONATE_INTE.xhtmlux5cux23page_18}{18},
\protect\hyperlink{18_Chapter_Eleven__THE_FORMS_OF_THO.xhtmlux5cux23page_292}{292}

Richard II (king of England),
\protect\hyperlink{08_Chapter_One__THE_PASSIONATE_INTE.xhtmlux5cux23page_12}{12};
Charles VI and,
\protect\hyperlink{10_Chapter_Three__THE_HEROIC_DREAM.xhtmlux5cux23page_71}{71};
Isabella of France and,
\protect\hyperlink{20_ILLUSTRATIONS_FOLLOW_PAGE.xhtmlux5cux23page_298}{298};
mignon,
\protect\hyperlink{09_Chapter_Two__THE_CRAVING_FOR_A_M.xhtmlux5cux23page_59}{59};
princely duel,
\protect\hyperlink{10_Chapter_Three__THE_HEROIC_DREAM.xhtmlux5cux23page_107}{107}

Richard of St. Victor,
\protect\hyperlink{20_ILLUSTRATIONS_FOLLOW_PAGE.xhtmlux5cux23page_321}{321}

Robertet, Jean,
\protect\hyperlink{21_Chapter_Thirteen__IMAGE_AND_WORD.xhtmlux5cux23page_366}{366},
\protect\hyperlink{22_Chapter_Fourteen__THE_COMING_OF.xhtmlux5cux23page_389}{389--}\protect\hyperlink{22_Chapter_Fourteen__THE_COMING_OF.xhtmlux5cux23page_391}{391}

Roch, Saint,
\protect\hyperlink{13_Chapter_Six__THE_DEPICTION_OF_TH.xhtmlux5cux23page_191}{191},
\protect\hyperlink{13_Chapter_Six__THE_DEPICTION_OF_TH.xhtmlux5cux23page_197}{197},
\protect\hyperlink{13_Chapter_Six__THE_DEPICTION_OF_TH.xhtmlux5cux23page_198}{198}

Rochefort, Charles de: allegory,
\protect\hyperlink{16_Chapter_Nine__THE_DECLINE_OF_SYM.xhtmlux5cux23page_245}{245};
\emph{L'abuzé},
\protect\hyperlink{10_Chapter_Three__THE_HEROIC_DREAM.xhtmlux5cux23page_124}{124}

Rolin, Nicholas: artists, patron of,
\protect\hyperlink{20_ILLUSTRATIONS_FOLLOW_PAGE.xhtmlux5cux23page_306}{306};
Chevot, Jean and,
\protect\hyperlink{20_ILLUSTRATIONS_FOLLOW_PAGE.xhtmlux5cux23page_316}{316};
death,
\protect\hyperlink{09_Chapter_Two__THE_CRAVING_FOR_A_M.xhtmlux5cux23page_55}{55};
festivity at Lille,
\protect\hyperlink{20_ILLUSTRATIONS_FOLLOW_PAGE.xhtmlux5cux23page_305}{305};
John the Fearless, murder of and,
\protect\hyperlink{08_Chapter_One__THE_PASSIONATE_INTE.xhtmlux5cux23page_17}{17};
\emph{Madonna} (van Eyck),
\protect\hyperlink{20_ILLUSTRATIONS_FOLLOW_PAGE.xhtmlux5cux23page_317}{317},
\protect\hyperlink{21_Chapter_Thirteen__IMAGE_AND_WORD.xhtmlux5cux23page_334}{334--}\protect\hyperlink{21_Chapter_Thirteen__IMAGE_AND_WORD.xhtmlux5cux23page_335}{335},
\protect\hyperlink{21_Chapter_Thirteen__IMAGE_AND_WORD.xhtmlux5cux23page_336}{336};
nouveau riche,
\protect\hyperlink{20_ILLUSTRATIONS_FOLLOW_PAGE.xhtmlux5cux23page_316}{316--}\protect\hyperlink{20_ILLUSTRATIONS_FOLLOW_PAGE.xhtmlux5cux23page_317}{317};
piety,
\protect\hyperlink{20_ILLUSTRATIONS_FOLLOW_PAGE.xhtmlux5cux23page_317}{317}

\emph{Roman de la rose: L'abuzé} and,
\protect\hyperlink{10_Chapter_Three__THE_HEROIC_DREAM.xhtmlux5cux23page_124}{124};
allegory,
\protect\hyperlink{10_Chapter_Three__THE_HEROIC_DREAM.xhtmlux5cux23page_96}{96},
\protect\hyperlink{11_Chapter_Four__THE_FORMS_OF_LOVE.xhtmlux5cux23page_135}{135--}\protect\hyperlink{11_Chapter_Four__THE_FORMS_OF_LOVE.xhtmlux5cux23page_136}{136},
\protect\hyperlink{11_Chapter_Four__THE_FORMS_OF_LOVE.xhtmlux5cux23page_141}{141--}\protect\hyperlink{11_Chapter_Four__THE_FORMS_OF_LOVE.xhtmlux5cux23page_142}{142},
\protect\hyperlink{16_Chapter_Nine__THE_DECLINE_OF_SYM.xhtmlux5cux23page_244}{244--}\protect\hyperlink{16_Chapter_Nine__THE_DECLINE_OF_SYM.xhtmlux5cux23page_245}{245},
\protect\hyperlink{21_Chapter_Thirteen__IMAGE_AND_WORD.xhtmlux5cux23page_355}{355};
\emph{Chastel d'amours} and,
\protect\hyperlink{11_Chapter_Four__THE_FORMS_OF_LOVE.xhtmlux5cux23page_143}{143};
Christine de Pisan, works compared to,
\protect\hyperlink{21_Chapter_Thirteen__IMAGE_AND_WORD.xhtmlux5cux23page_360}{360};
classicism and,
\protect\hyperlink{16_Chapter_Nine__THE_DECLINE_OF_SYM.xhtmlux5cux23page_247}{247};
contempt for women in,
\protect\hyperlink{11_Chapter_Four__THE_FORMS_OF_LOVE.xhtmlux5cux23page_136}{136};
conventional,
\protect\hyperlink{21_Chapter_Thirteen__IMAGE_AND_WORD.xhtmlux5cux23page_358}{358};
debate over,
\protect\hyperlink{11_Chapter_Four__THE_FORMS_OF_LOVE.xhtmlux5cux23page_137}{137--}\protect\hyperlink{11_Chapter_Four__THE_FORMS_OF_LOVE.xhtmlux5cux23page_141}{141},
\protect\hyperlink{11_Chapter_Four__THE_FORMS_OF_LOVE.xhtmlux5cux23page_153}{153};
dominates aristocracy,
\protect\hyperlink{11_Chapter_Four__THE_FORMS_OF_LOVE.xhtmlux5cux23page_127}{127},
\protect\hyperlink{11_Chapter_Four__THE_FORMS_OF_LOVE.xhtmlux5cux23page_133}{133};
humanists and,
\protect\hyperlink{22_Chapter_Fourteen__THE_COMING_OF.xhtmlux5cux23page_384}{384};
literary history,
\protect\hyperlink{21_Chapter_Thirteen__IMAGE_AND_WORD.xhtmlux5cux23page_332}{332};
obcenity,
\protect\hyperlink{13_Chapter_Six__THE_DEPICTION_OF_TH.xhtmlux5cux23page_182}{182};
pagan,
\protect\hyperlink{11_Chapter_Four__THE_FORMS_OF_LOVE.xhtmlux5cux23page_135}{135};
prose version,
\protect\hyperlink{21_Chapter_Thirteen__IMAGE_AND_WORD.xhtmlux5cux23page_380}{380};
reality and,
\protect\hyperlink{11_Chapter_Four__THE_FORMS_OF_LOVE.xhtmlux5cux23page_154}{154}

Romuald, Saint,
\protect\hyperlink{13_Chapter_Six__THE_DEPICTION_OF_TH.xhtmlux5cux23page_192}{192},
\protect\hyperlink{14_Chapter_Seven__THE_PIOUS_PERSONA.xhtmlux5cux23page_216}{216}

Ronsard, Pierre,
\protect\hyperlink{11_Chapter_Four__THE_FORMS_OF_LOVE.xhtmlux5cux23page_140}{140}

Rosa of Viterbo, Saint,
\protect\hyperlink{12_Chapter_Five__THE_VISION_OF_DEAT.xhtmlux5cux23page_163}{163}

Rosary, the,
\protect\hyperlink{13_Chapter_Six__THE_DEPICTION_OF_TH.xhtmlux5cux23page_175}{175},
\protect\hyperlink{15_Chapter_Eight__RELIGIOUS_EXCITAT.xhtmlux5cux23page_232}{232},
\protect\hyperlink{15_Chapter_Eight__RELIGIOUS_EXCITAT.xhtmlux5cux23page_233}{233},
\protect\hyperlink{16_Chapter_Nine__THE_DECLINE_OF_SYM.xhtmlux5cux23page_241}{241}

Rosebeke, battle of,
\protect\hyperlink{08_Chapter_One__THE_PASSIONATE_INTE.xhtmlux5cux23page_19}{19}

Round Table,
\protect\hyperlink{09_Chapter_Two__THE_CRAVING_FOR_A_M.xhtmlux5cux23page_39}{39},
\protect\hyperlink{10_Chapter_Three__THE_HEROIC_DREAM.xhtmlux5cux23page_75}{75}

Roye, Jean de,
\protect\hyperlink{21_Chapter_Thirteen__IMAGE_AND_WORD.xhtmlux5cux23page_374}{374}

Rupe, Alanus de. See La Roche, Alain de

Ruusbroec, Jan,
\protect\hyperlink{17_Chapter_Ten__THE_FAILURE_OF_IMAG.xhtmlux5cux23page_260}{260},
\protect\hyperlink{17_Chapter_Ten__THE_FAILURE_OF_IMAG.xhtmlux5cux23page_261}{261},
\protect\hyperlink{17_Chapter_Ten__THE_FAILURE_OF_IMAG.xhtmlux5cux23page_265}{265}

Sacraments,
\protect\hyperlink{13_Chapter_Six__THE_DEPICTION_OF_TH.xhtmlux5cux23page_175}{175}:
and art,
\protect\hyperlink{19_Chapter_Twelve__ART_IN_LIFE.xhtmlux5cux23page_296}{296};
baptism,
\protect\hyperlink{09_Chapter_Two__THE_CRAVING_FOR_A_M.xhtmlux5cux23page_58}{58},
\protect\hyperlink{11_Chapter_Four__THE_FORMS_OF_LOVE.xhtmlux5cux23page_140}{140},
\protect\hyperlink{13_Chapter_Six__THE_DEPICTION_OF_TH.xhtmlux5cux23page_178}{178--}\protect\hyperlink{13_Chapter_Six__THE_DEPICTION_OF_TH.xhtmlux5cux23page_179}{179},
\protect\hyperlink{14_Chapter_Seven__THE_PIOUS_PERSONA.xhtmlux5cux23page_206}{206},
\protect\hyperlink{15_Chapter_Eight__RELIGIOUS_EXCITAT.xhtmlux5cux23page_222}{222},\protect\hypertarget{25_INDEX.xhtmlux5cux23page_465}{\protect\hyperlink{15_Chapter_Eight__RELIGIOUS_EXCITAT.xhtmlux5cux23page_227}{227}};
differ from charms, etc.,
\protect\hyperlink{18_Chapter_Eleven__THE_FORMS_OF_THO.xhtmlux5cux23page_292}{292--}\protect\hyperlink{18_Chapter_Eleven__THE_FORMS_OF_THO.xhtmlux5cux23page_293}{293};
disrespect for,
\protect\hyperlink{13_Chapter_Six__THE_DEPICTION_OF_TH.xhtmlux5cux23page_178}{178},
\protect\hyperlink{13_Chapter_Six__THE_DEPICTION_OF_TH.xhtmlux5cux23page_183}{183--}\protect\hyperlink{13_Chapter_Six__THE_DEPICTION_OF_TH.xhtmlux5cux23page_185}{185},
\protect\hyperlink{14_Chapter_Seven__THE_PIOUS_PERSONA.xhtmlux5cux23page_206}{206};
Extreme Unction,
\protect\hyperlink{08_Chapter_One__THE_PASSIONATE_INTE.xhtmlux5cux23page_21}{21};
Eckhart,
\protect\hyperlink{17_Chapter_Ten__THE_FAILURE_OF_IMAG.xhtmlux5cux23page_257}{257};
Eucharist,
\protect\hyperlink{09_Chapter_Two__THE_CRAVING_FOR_A_M.xhtmlux5cux23page_44}{44},
\protect\hyperlink{13_Chapter_Six__THE_DEPICTION_OF_TH.xhtmlux5cux23page_178}{178},
\protect\hyperlink{13_Chapter_Six__THE_DEPICTION_OF_TH.xhtmlux5cux23page_183}{183--}\protect\hyperlink{13_Chapter_Six__THE_DEPICTION_OF_TH.xhtmlux5cux23page_184}{184},
\protect\hyperlink{16_Chapter_Nine__THE_DECLINE_OF_SYM.xhtmlux5cux23page_234}{234},
\protect\hyperlink{16_Chapter_Nine__THE_DECLINE_OF_SYM.xhtmlux5cux23page_239}{239};
Festival of the,
\protect\hyperlink{16_Chapter_Nine__THE_DECLINE_OF_SYM.xhtmlux5cux23page_234}{234--}\protect\hyperlink{16_Chapter_Nine__THE_DECLINE_OF_SYM.xhtmlux5cux23page_235}{235};
invalid,
\protect\hyperlink{15_Chapter_Eight__RELIGIOUS_EXCITAT.xhtmlux5cux23page_227}{227--}\protect\hyperlink{15_Chapter_Eight__RELIGIOUS_EXCITAT.xhtmlux5cux23page_228}{228};
marriage,
\protect\hyperlink{11_Chapter_Four__THE_FORMS_OF_LOVE.xhtmlux5cux23page_129}{129};
mysticism and,
\protect\hyperlink{17_Chapter_Ten__THE_FAILURE_OF_IMAG.xhtmlux5cux23page_264}{264};
Penance,
\protect\hyperlink{08_Chapter_One__THE_PASSIONATE_INTE.xhtmlux5cux23page_21}{21},
\protect\hyperlink{15_Chapter_Eight__RELIGIOUS_EXCITAT.xhtmlux5cux23page_227}{227};
Seven,
\protect\hyperlink{16_Chapter_Nine__THE_DECLINE_OF_SYM.xhtmlux5cux23page_241}{241};
\emph{Seven Sacraments} (van der Weyden),
\protect\hyperlink{20_ILLUSTRATIONS_FOLLOW_PAGE.xhtmlux5cux23page_307}{307},
\protect\hyperlink{20_ILLUSTRATIONS_FOLLOW_PAGE.xhtmlux5cux23page_316}{316};
symbolism,
\protect\hyperlink{11_Chapter_Four__THE_FORMS_OF_LOVE.xhtmlux5cux23page_140}{140}

Saint Pol, \emph{connétable} of,
\protect\hyperlink{10_Chapter_Three__THE_HEROIC_DREAM.xhtmlux5cux23page_89}{89},
\protect\hyperlink{14_Chapter_Seven__THE_PIOUS_PERSONA.xhtmlux5cux23page_215}{215}

Saint Pol, Louis de (count of Luxembourg),
\protect\hyperlink{20_ILLUSTRATIONS_FOLLOW_PAGE.xhtmlux5cux23page_325}{325}

Saint Victor, monastery of. \emph{See} Hugo of St. Victor; Richard of
St. Victor

Salazar, Jean de,
\protect\hyperlink{20_ILLUSTRATIONS_FOLLOW_PAGE.xhtmlux5cux23page_325}{325}

Salisbury, William Montague, duke of,
\protect\hyperlink{10_Chapter_Three__THE_HEROIC_DREAM.xhtmlux5cux23page_99}{99}

Salmon, Pierre,
\protect\hyperlink{08_Chapter_One__THE_PASSIONATE_INTE.xhtmlux5cux23page_19}{19},
\protect\hyperlink{14_Chapter_Seven__THE_PIOUS_PERSONA.xhtmlux5cux23page_213}{213}

Salutati, Coluccio,
\protect\hyperlink{22_Chapter_Fourteen__THE_COMING_OF.xhtmlux5cux23page_384}{384}

Samson,
\protect\hyperlink{21_Chapter_Thirteen__IMAGE_AND_WORD.xhtmlux5cux23page_336}{336}

Samur, castle of,
\protect\hyperlink{21_Chapter_Thirteen__IMAGE_AND_WORD.xhtmlux5cux23page_352}{352}

Sancerre, Louis de,
\protect\hyperlink{09_Chapter_Two__THE_CRAVING_FOR_A_M.xhtmlux5cux23page_52}{52}

Saracens,
\protect\hyperlink{10_Chapter_Three__THE_HEROIC_DREAM.xhtmlux5cux23page_71}{71},
\protect\hyperlink{10_Chapter_Three__THE_HEROIC_DREAM.xhtmlux5cux23page_102}{102}.
\emph{See also} Turks

Saturn,
\protect\hyperlink{21_Chapter_Thirteen__IMAGE_AND_WORD.xhtmlux5cux23page_377}{377}

Savanarola,
\protect\hyperlink{08_Chapter_One__THE_PASSIONATE_INTE.xhtmlux5cux23page_7}{7}

Savoy, Amadeus of,
\protect\hyperlink{10_Chapter_Three__THE_HEROIC_DREAM.xhtmlux5cux23page_94}{94}

Savoy, duke of,
\protect\hyperlink{14_Chapter_Seven__THE_PIOUS_PERSONA.xhtmlux5cux23page_209}{209}

Savoy, House of,
\protect\hyperlink{14_Chapter_Seven__THE_PIOUS_PERSONA.xhtmlux5cux23page_217}{217}

Saxony, duke of,
\protect\hyperlink{10_Chapter_Three__THE_HEROIC_DREAM.xhtmlux5cux23page_108}{108}

Scipio,
\protect\hyperlink{10_Chapter_Three__THE_HEROIC_DREAM.xhtmlux5cux23page_88}{88}

Scorel, Jan van,
\protect\hyperlink{20_ILLUSTRATIONS_FOLLOW_PAGE.xhtmlux5cux23page_321}{321}

Sebastian, Saint,
\protect\hyperlink{13_Chapter_Six__THE_DEPICTION_OF_TH.xhtmlux5cux23page_198}{198}

Selonnet, prior of. \emph{See} Bonet, Honoré

Semiramis,
\protect\hyperlink{10_Chapter_Three__THE_HEROIC_DREAM.xhtmlux5cux23page_77}{77}

Sempy. \emph{See} Croy, Philippe de

Seneca,
\protect\hyperlink{10_Chapter_Three__THE_HEROIC_DREAM.xhtmlux5cux23page_68}{68}

Shakespeare,
\protect\hyperlink{21_Chapter_Thirteen__IMAGE_AND_WORD.xhtmlux5cux23page_379}{379}

Sint Jans, Geertgen of: \emph{Nativity},
\protect\hyperlink{21_Chapter_Thirteen__IMAGE_AND_WORD.xhtmlux5cux23page_347}{347};
rhythm,
\protect\hyperlink{21_Chapter_Thirteen__IMAGE_AND_WORD.xhtmlux5cux23page_376}{376}

Sluter, Claus: dominates our impression of age,
\protect\hyperlink{19_Chapter_Twelve__ART_IN_LIFE.xhtmlux5cux23page_294}{294};
duke of Burgundy and,
\protect\hyperlink{20_ILLUSTRATIONS_FOLLOW_PAGE.xhtmlux5cux23page_307}{307--}\protect\hyperlink{20_ILLUSTRATIONS_FOLLOW_PAGE.xhtmlux5cux23page_308}{308};
\emph{Moses Fountain},
\protect\hyperlink{20_ILLUSTRATIONS_FOLLOW_PAGE.xhtmlux5cux23page_308}{308--}\protect\hyperlink{20_ILLUSTRATIONS_FOLLOW_PAGE.xhtmlux5cux23page_311}{311};
medieval,
\protect\hyperlink{21_Chapter_Thirteen__IMAGE_AND_WORD.xhtmlux5cux23page_329}{329};
\emph{Plourants},
\protect\hyperlink{20_ILLUSTRATIONS_FOLLOW_PAGE.xhtmlux5cux23page_310}{310};
sanctity of subject matter,
\protect\hyperlink{20_ILLUSTRATIONS_FOLLOW_PAGE.xhtmlux5cux23page_308}{308--}\protect\hyperlink{20_ILLUSTRATIONS_FOLLOW_PAGE.xhtmlux5cux23page_309}{309}

sodomy,
\protect\hyperlink{08_Chapter_One__THE_PASSIONATE_INTE.xhtmlux5cux23page_20}{20},
\protect\hyperlink{09_Chapter_Two__THE_CRAVING_FOR_A_M.xhtmlux5cux23page_59}{59}

\emph{Songe du vieil pélerin} (Philippe de Mézières),
\protect\hyperlink{10_Chapter_Three__THE_HEROIC_DREAM.xhtmlux5cux23page_71}{71}

Sorel, Agnes,
\protect\hyperlink{09_Chapter_Two__THE_CRAVING_FOR_A_M.xhtmlux5cux23page_58}{58},
\protect\hyperlink{13_Chapter_Six__THE_DEPICTION_OF_TH.xhtmlux5cux23page_182}{182},
\protect\hyperlink{21_Chapter_Thirteen__IMAGE_AND_WORD.xhtmlux5cux23page_352}{352}

Sprenger, Jacob,
\protect\hyperlink{15_Chapter_Eight__RELIGIOUS_EXCITAT.xhtmlux5cux23page_233}{233}

Standonck, Jan,
\protect\hyperlink{14_Chapter_Seven__THE_PIOUS_PERSONA.xhtmlux5cux23page_217}{217}

Stars, Order of the,
\protect\hyperlink{10_Chapter_Three__THE_HEROIC_DREAM.xhtmlux5cux23page_93}{93},
\protect\hyperlink{10_Chapter_Three__THE_HEROIC_DREAM.xhtmlux5cux23page_94}{94},
\protect\hyperlink{10_Chapter_Three__THE_HEROIC_DREAM.xhtmlux5cux23page_111}{111},
\protect\hyperlink{18_Chapter_Eleven__THE_FORMS_OF_THO.xhtmlux5cux23page_277}{277}

Steinlen,
\protect\hyperlink{21_Chapter_Thirteen__IMAGE_AND_WORD.xhtmlux5cux23page_365}{365}

Stephen, Saint,
\protect\hyperlink{13_Chapter_Six__THE_DEPICTION_OF_TH.xhtmlux5cux23page_182}{182}

Suffolk, Michael de la Pole, count of,
\protect\hyperlink{12_Chapter_Five__THE_VISION_OF_DEAT.xhtmlux5cux23page_164}{164}

\emph{Summis Desiderantes},
\protect\hyperlink{18_Chapter_Eleven__THE_FORMS_OF_THO.xhtmlux5cux23page_286}{286}

Suso, Henry, Saint,
\protect\hyperlink{15_Chapter_Eight__RELIGIOUS_EXCITAT.xhtmlux5cux23page_232}{232};
mysticism,
\protect\hyperlink{17_Chapter_Ten__THE_FAILURE_OF_IMAG.xhtmlux5cux23page_263}{263};
name of Jesus,
\protect\hyperlink{16_Chapter_Nine__THE_DECLINE_OF_SYM.xhtmlux5cux23page_234}{234};
symbolism,
\protect\hyperlink{13_Chapter_Six__THE_DEPICTION_OF_TH.xhtmlux5cux23page_174}{174};
transcendence,
\protect\hyperlink{13_Chapter_Six__THE_DEPICTION_OF_TH.xhtmlux5cux23page_174}{174}

Sword, Order of the,
\protect\hyperlink{10_Chapter_Three__THE_HEROIC_DREAM.xhtmlux5cux23page_94}{94}

symbolism,
\protect\hyperlink{16_Chapter_Nine__THE_DECLINE_OF_SYM.xhtmlux5cux23page_234}{234--}\protect\hyperlink{16_Chapter_Nine__THE_DECLINE_OF_SYM.xhtmlux5cux23page_248}{248};
Alain de la Roche,
\protect\hyperlink{15_Chapter_Eight__RELIGIOUS_EXCITAT.xhtmlux5cux23page_232}{232},
\protect\hyperlink{17_Chapter_Ten__THE_FAILURE_OF_IMAG.xhtmlux5cux23page_266}{266};
colors,
\protect\hyperlink{11_Chapter_Four__THE_FORMS_OF_LOVE.xhtmlux5cux23page_142}{142},
\protect\hyperlink{20_ILLUSTRATIONS_FOLLOW_PAGE.xhtmlux5cux23page_326}{326};
epithamalium,
\protect\hyperlink{11_Chapter_Four__THE_FORMS_OF_LOVE.xhtmlux5cux23page_129}{129};
love,
\protect\hyperlink{20_ILLUSTRATIONS_FOLLOW_PAGE.xhtmlux5cux23page_326}{326};
medieval thought, life of,
\protect\hyperlink{17_Chapter_Ten__THE_FAILURE_OF_IMAG.xhtmlux5cux23page_249}{249};
mysticism,
\protect\hyperlink{15_Chapter_Eight__RELIGIOUS_EXCITAT.xhtmlux5cux23page_231}{231},
\protect\hyperlink{17_Chapter_Ten__THE_FAILURE_OF_IMAG.xhtmlux5cux23page_257}{257--}\protect\hyperlink{17_Chapter_Ten__THE_FAILURE_OF_IMAG.xhtmlux5cux23page_258}{258};
phallic,
\protect\hyperlink{11_Chapter_Four__THE_FORMS_OF_LOVE.xhtmlux5cux23page_130}{130};
poverty,
\protect\hyperlink{14_Chapter_Seven__THE_PIOUS_PERSONA.xhtmlux5cux23page_205}{205};
princes,
\protect\hyperlink{13_Chapter_Six__THE_DEPICTION_OF_TH.xhtmlux5cux23page_181}{181};
ritual,
\protect\hyperlink{17_Chapter_Ten__THE_FAILURE_OF_IMAG.xhtmlux5cux23page_264}{264}.
\emph{See also} allegory; \emph{Roman de la rose}

Tacitus,
\protect\hyperlink{10_Chapter_Three__THE_HEROIC_DREAM.xhtmlux5cux23page_101}{101}

Taine, Hippolyte,
\protect\hyperlink{10_Chapter_Three__THE_HEROIC_DREAM.xhtmlux5cux23page_73}{73}

\emph{Temple de Bocace, Le} (Chastellain),
\protect\hyperlink{10_Chapter_Three__THE_HEROIC_DREAM.xhtmlux5cux23page_65}{65},
\protect\hyperlink{22_Chapter_Fourteen__THE_COMING_OF.xhtmlux5cux23page_386}{386}

Tewkesbury, battle of,
\protect\hyperlink{08_Chapter_One__THE_PASSIONATE_INTE.xhtmlux5cux23page_14}{14}

Theocritus,
\protect\hyperlink{10_Chapter_Three__THE_HEROIC_DREAM.xhtmlux5cux23page_120}{120}

Thistle, Order of the,
\protect\hyperlink{10_Chapter_Three__THE_HEROIC_DREAM.xhtmlux5cux23page_94}{94}

Thomas, Brother,
\protect\hyperlink{08_Chapter_One__THE_PASSIONATE_INTE.xhtmlux5cux23page_6}{6--}\protect\hyperlink{08_Chapter_One__THE_PASSIONATE_INTE.xhtmlux5cux23page_7}{7},
\protect\hyperlink{14_Chapter_Seven__THE_PIOUS_PERSONA.xhtmlux5cux23page_208}{208},
\protect\hyperlink{15_Chapter_Eight__RELIGIOUS_EXCITAT.xhtmlux5cux23page_226}{226}

Thomas, Pierre,
\protect\hyperlink{14_Chapter_Seven__THE_PIOUS_PERSONA.xhtmlux5cux23page_209}{209}

Thomas Aquinas, Saint,
\protect\hyperlink{13_Chapter_Six__THE_DEPICTION_OF_TH.xhtmlux5cux23page_192}{192},
\protect\hyperlink{17_Chapter_Ten__THE_FAILURE_OF_IMAG.xhtmlux5cux23page_256}{256},
\protect\hyperlink{18_Chapter_Eleven__THE_FORMS_OF_THO.xhtmlux5cux23page_293}{293},
\protect\hyperlink{20_ILLUSTRATIONS_FOLLOW_PAGE.xhtmlux5cux23page_322}{322}

\emph{Three Kings, The} (Limburg brothers),
\protect\hyperlink{21_Chapter_Thirteen__IMAGE_AND_WORD.xhtmlux5cux23page_356}{356}

Thucydides,
\protect\hyperlink{10_Chapter_Three__THE_HEROIC_DREAM.xhtmlux5cux23page_72}{72}

Tomyris,
\protect\hyperlink{10_Chapter_Three__THE_HEROIC_DREAM.xhtmlux5cux23page_77}{77}

Tonnerre, count of,
\protect\hyperlink{11_Chapter_Four__THE_FORMS_OF_LOVE.xhtmlux5cux23page_141}{141}

Toraine, Jean de (dauphin of France). \emph{See} Louis d'Orléans

\emph{Totentanz}. See \emph{danse macabre}

tournaments. See \emph{pas d'armes}

\protect\hypertarget{25_INDEX.xhtmlux5cux23page_466}{}{}Tournay, Bishop,
\protect\hyperlink{09_Chapter_Two__THE_CRAVING_FOR_A_M.xhtmlux5cux23page_54}{54--}\protect\hyperlink{09_Chapter_Two__THE_CRAVING_FOR_A_M.xhtmlux5cux23page_55}{55}.
\emph{See also} Chevrot, Jean.

Trastamara, Heinrich de,
\protect\hyperlink{10_Chapter_Three__THE_HEROIC_DREAM.xhtmlux5cux23page_113}{113}

Trazegnies, Gilles de,
\protect\hyperlink{10_Chapter_Three__THE_HEROIC_DREAM.xhtmlux5cux23page_78}{78}

Treasury of Merit,
\protect\hyperlink{17_Chapter_Ten__THE_FAILURE_OF_IMAG.xhtmlux5cux23page_255}{255}

Trémoille, Guy de la,
\protect\hyperlink{10_Chapter_Three__THE_HEROIC_DREAM.xhtmlux5cux23page_112}{112},
\protect\hyperlink{20_ILLUSTRATIONS_FOLLOW_PAGE.xhtmlux5cux23page_300}{300}

Trent, Council of,
\protect\hyperlink{13_Chapter_Six__THE_DEPICTION_OF_TH.xhtmlux5cux23page_198}{198}

\emph{Très-riches heures de Chantilly, Les} (Limburg brothers),
\protect\hyperlink{21_Chapter_Thirteen__IMAGE_AND_WORD.xhtmlux5cux23page_350}{350},
\protect\hyperlink{21_Chapter_Thirteen__IMAGE_AND_WORD.xhtmlux5cux23page_352}{352}

Tristan,
\protect\hyperlink{10_Chapter_Three__THE_HEROIC_DREAM.xhtmlux5cux23page_91}{91},
\protect\hyperlink{21_Chapter_Thirteen__IMAGE_AND_WORD.xhtmlux5cux23page_358}{358}

Troilus,
\protect\hyperlink{10_Chapter_Three__THE_HEROIC_DREAM.xhtmlux5cux23page_75}{75}

Turks,
\protect\hyperlink{08_Chapter_One__THE_PASSIONATE_INTE.xhtmlux5cux23page_13}{13},
\protect\hyperlink{14_Chapter_Seven__THE_PIOUS_PERSONA.xhtmlux5cux23page_217}{217},
\protect\hyperlink{20_ILLUSTRATIONS_FOLLOW_PAGE.xhtmlux5cux23page_302}{302},
\protect\hyperlink{20_ILLUSTRATIONS_FOLLOW_PAGE.xhtmlux5cux23page_305}{305},
\protect\hyperlink{20_ILLUSTRATIONS_FOLLOW_PAGE.xhtmlux5cux23page_316}{316};
Bajasid, sultan,
\protect\hyperlink{10_Chapter_Three__THE_HEROIC_DREAM.xhtmlux5cux23page_78}{78};
chivalric orders and,
\protect\hyperlink{10_Chapter_Three__THE_HEROIC_DREAM.xhtmlux5cux23page_92}{92},
\protect\hyperlink{10_Chapter_Three__THE_HEROIC_DREAM.xhtmlux5cux23page_105}{105};
expulsion of,
\protect\hyperlink{10_Chapter_Three__THE_HEROIC_DREAM.xhtmlux5cux23page_71}{71};
Great Turk,
\protect\hyperlink{10_Chapter_Three__THE_HEROIC_DREAM.xhtmlux5cux23page_102}{102},
\protect\hyperlink{10_Chapter_Three__THE_HEROIC_DREAM.xhtmlux5cux23page_108}{108},
\protect\hyperlink{14_Chapter_Seven__THE_PIOUS_PERSONA.xhtmlux5cux23page_215}{215};
Nicopolis, battle of,
\protect\hyperlink{10_Chapter_Three__THE_HEROIC_DREAM.xhtmlux5cux23page_118}{118}

Turlupins,
\protect\hyperlink{12_Chapter_Five__THE_VISION_OF_DEAT.xhtmlux5cux23page_163}{163},
\protect\hyperlink{13_Chapter_Six__THE_DEPICTION_OF_TH.xhtmlux5cux23page_189}{189},
\protect\hyperlink{15_Chapter_Eight__RELIGIOUS_EXCITAT.xhtmlux5cux23page_229}{229}

Tutetey, A.,
\protect\hyperlink{08_Chapter_One__THE_PASSIONATE_INTE.xhtmlux5cux23page_29}{29}

\emph{Twelfth Night} (Shakespeare),
\protect\hyperlink{09_Chapter_Two__THE_CRAVING_FOR_A_M.xhtmlux5cux23page_59}{59}

``typism'' (Lamprecht),
\protect\hyperlink{17_Chapter_Ten__THE_FAILURE_OF_IMAG.xhtmlux5cux23page_250}{250}

Ugolino della Gherardesca,
\protect\hyperlink{17_Chapter_Ten__THE_FAILURE_OF_IMAG.xhtmlux5cux23page_252}{252}

\emph{Unigenitus},
\protect\hyperlink{17_Chapter_Ten__THE_FAILURE_OF_IMAG.xhtmlux5cux23page_255}{255}

universals,
\protect\hyperlink{16_Chapter_Nine__THE_DECLINE_OF_SYM.xhtmlux5cux23page_237}{237--}\protect\hyperlink{16_Chapter_Nine__THE_DECLINE_OF_SYM.xhtmlux5cux23page_238}{238}

Urban, pope,
\protect\hyperlink{08_Chapter_One__THE_PASSIONATE_INTE.xhtmlux5cux23page_19}{19}

Valentine, Saint,
\protect\hyperlink{13_Chapter_Six__THE_DEPICTION_OF_TH.xhtmlux5cux23page_198}{198}

\emph{Validorum per francium mendicantium varia astucia} (Robert
Gaguin),
\protect\hyperlink{13_Chapter_Six__THE_DEPICTION_OF_TH.xhtmlux5cux23page_199}{199--}\protect\hyperlink{13_Chapter_Six__THE_DEPICTION_OF_TH.xhtmlux5cux23page_200}{200}

Valois, House of,
\protect\hyperlink{08_Chapter_One__THE_PASSIONATE_INTE.xhtmlux5cux23page_16}{16},
\protect\hyperlink{14_Chapter_Seven__THE_PIOUS_PERSONA.xhtmlux5cux23page_211}{211},
\protect\hyperlink{21_Chapter_Thirteen__IMAGE_AND_WORD.xhtmlux5cux23page_345}{345}

Varennes, Jean de,
\protect\hyperlink{15_Chapter_Eight__RELIGIOUS_EXCITAT.xhtmlux5cux23page_226}{226--}\protect\hyperlink{15_Chapter_Eight__RELIGIOUS_EXCITAT.xhtmlux5cux23page_228}{228},
\protect\hyperlink{18_Chapter_Eleven__THE_FORMS_OF_THO.xhtmlux5cux23page_274}{274},
\protect\hyperlink{22_Chapter_Fourteen__THE_COMING_OF.xhtmlux5cux23page_385}{385}

\emph{Vauderie d'Arras},
\protect\hyperlink{18_Chapter_Eleven__THE_FORMS_OF_THO.xhtmlux5cux23page_289}{289}

Velazquez, Diego,
\protect\hyperlink{08_Chapter_One__THE_PASSIONATE_INTE.xhtmlux5cux23page_23}{23}

Venus,
\protect\hyperlink{11_Chapter_Four__THE_FORMS_OF_LOVE.xhtmlux5cux23page_134}{134--}\protect\hyperlink{11_Chapter_Four__THE_FORMS_OF_LOVE.xhtmlux5cux23page_136}{136},
\protect\hyperlink{11_Chapter_Four__THE_FORMS_OF_LOVE.xhtmlux5cux23page_139}{139},
\protect\hyperlink{16_Chapter_Nine__THE_DECLINE_OF_SYM.xhtmlux5cux23page_247}{247},
\protect\hyperlink{21_Chapter_Thirteen__IMAGE_AND_WORD.xhtmlux5cux23page_374}{374},
\protect\hyperlink{22_Chapter_Fourteen__THE_COMING_OF.xhtmlux5cux23page_395}{395}

Vere, Robert de,
\protect\hyperlink{09_Chapter_Two__THE_CRAVING_FOR_A_M.xhtmlux5cux23page_59}{59}

Vignolles, Etienne de. See La Hire

Villiers, George (duke of Buckingham),
\protect\hyperlink{09_Chapter_Two__THE_CRAVING_FOR_A_M.xhtmlux5cux23page_58}{58}

Villon, François: beauty,
\protect\hyperlink{19_Chapter_Twelve__ART_IN_LIFE.xhtmlux5cux23page_295}{295};
``La belle heaulmière,''
\protect\hyperlink{10_Chapter_Three__THE_HEROIC_DREAM.xhtmlux5cux23page_112}{112};
Chastellain compared to,
\protect\hyperlink{12_Chapter_Five__THE_VISION_OF_DEAT.xhtmlux5cux23page_168}{168--}\protect\hyperlink{12_Chapter_Five__THE_VISION_OF_DEAT.xhtmlux5cux23page_169}{169};
Latinisms, scorn of,
\protect\hyperlink{22_Chapter_Fourteen__THE_COMING_OF.xhtmlux5cux23page_388}{388};
love poetry,
\protect\hyperlink{21_Chapter_Thirteen__IMAGE_AND_WORD.xhtmlux5cux23page_367}{367};
modern,
\protect\hyperlink{22_Chapter_Fourteen__THE_COMING_OF.xhtmlux5cux23page_389}{389};
paganism,
\protect\hyperlink{22_Chapter_Fourteen__THE_COMING_OF.xhtmlux5cux23page_393}{393};
compared with painter,
\protect\hyperlink{21_Chapter_Thirteen__IMAGE_AND_WORD.xhtmlux5cux23page_330}{330};
prayer for mother,
\protect\hyperlink{13_Chapter_Six__THE_DEPICTION_OF_TH.xhtmlux5cux23page_191}{191};
proverbs,
\protect\hyperlink{18_Chapter_Eleven__THE_FORMS_OF_THO.xhtmlux5cux23page_274}{274};
testament,
\protect\hyperlink{18_Chapter_Eleven__THE_FORMS_OF_THO.xhtmlux5cux23page_278}{278};
will,
\protect\hyperlink{08_Chapter_One__THE_PASSIONATE_INTE.xhtmlux5cux23page_29}{29};
works: ``Ballade des dames du temps jadis,''
\protect\hyperlink{12_Chapter_Five__THE_VISION_OF_DEAT.xhtmlux5cux23page_158}{158};
``Les contrediz Franc Gontier,''
\protect\hyperlink{11_Chapter_Four__THE_FORMS_OF_LOVE.xhtmlux5cux23page_154}{154}

Vincennes, castle of,
\protect\hyperlink{21_Chapter_Thirteen__IMAGE_AND_WORD.xhtmlux5cux23page_352}{352}

Virgil,
\protect\hyperlink{22_Chapter_Fourteen__THE_COMING_OF.xhtmlux5cux23page_384}{384}

virginity: allegory,
\protect\hyperlink{16_Chapter_Nine__THE_DECLINE_OF_SYM.xhtmlux5cux23page_239}{239};
clergy and,
\protect\hyperlink{15_Chapter_Eight__RELIGIOUS_EXCITAT.xhtmlux5cux23page_225}{225--}\protect\hyperlink{15_Chapter_Eight__RELIGIOUS_EXCITAT.xhtmlux5cux23page_227}{227};
Colette, Saint, and,
\protect\hyperlink{15_Chapter_Eight__RELIGIOUS_EXCITAT.xhtmlux5cux23page_225}{225--}\protect\hyperlink{15_Chapter_Eight__RELIGIOUS_EXCITAT.xhtmlux5cux23page_227}{227};
estate of,
\protect\hyperlink{10_Chapter_Three__THE_HEROIC_DREAM.xhtmlux5cux23page_62}{62};
Gerson on,
\protect\hyperlink{09_Chapter_Two__THE_CRAVING_FOR_A_M.xhtmlux5cux23page_36}{36};
heroic dream and,
\protect\hyperlink{10_Chapter_Three__THE_HEROIC_DREAM.xhtmlux5cux23page_83}{83};
Mary of Burgundy,
\protect\hyperlink{13_Chapter_Six__THE_DEPICTION_OF_TH.xhtmlux5cux23page_181}{181}.
See also \emph{Roman de la rose}

\emph{Viris illustribus, Liber de} (Petrarch),
\protect\hyperlink{22_Chapter_Fourteen__THE_COMING_OF.xhtmlux5cux23page_385}{385}

Vitri, Philippe de (bishop of Meaux): Petrarch and,
\protect\hyperlink{22_Chapter_Fourteen__THE_COMING_OF.xhtmlux5cux23page_384}{384};
works: \emph{Chapel des fleurs de lis},
\protect\hyperlink{10_Chapter_Three__THE_HEROIC_DREAM.xhtmlux5cux23page_70}{70};
``Dit du Franc Gontier,''
\protect\hyperlink{10_Chapter_Three__THE_HEROIC_DREAM.xhtmlux5cux23page_120}{120--}\protect\hyperlink{10_Chapter_Three__THE_HEROIC_DREAM.xhtmlux5cux23page_123}{123}

\emph{Visitation} (Limburg brothers),
\protect\hyperlink{21_Chapter_Thirteen__IMAGE_AND_WORD.xhtmlux5cux23page_363}{363--}\protect\hyperlink{21_Chapter_Thirteen__IMAGE_AND_WORD.xhtmlux5cux23page_364}{364}

\emph{Vita et regimine episcoporum, etc}. (Denis the Carthusian),
\protect\hyperlink{17_Chapter_Ten__THE_FAILURE_OF_IMAG.xhtmlux5cux23page_250}{250}

\emph{Vita nuova, La} (Dante),
\protect\hyperlink{11_Chapter_Four__THE_FORMS_OF_LOVE.xhtmlux5cux23page_126}{126--}\protect\hyperlink{11_Chapter_Four__THE_FORMS_OF_LOVE.xhtmlux5cux23page_127}{127}

\emph{Vita solitaria},
\protect\hyperlink{22_Chapter_Fourteen__THE_COMING_OF.xhtmlux5cux23page_385}{385}

Vitus, Saint,
\protect\hyperlink{13_Chapter_Six__THE_DEPICTION_OF_TH.xhtmlux5cux23page_198}{198}

``Vivat rex'' (Gerson),
\emph{\protect\hyperlink{10_Chapter_Three__THE_HEROIC_DREAM.xhtmlux5cux23page_66}{66}}

\emph{Voeu du Héron},
\protect\hyperlink{10_Chapter_Three__THE_HEROIC_DREAM.xhtmlux5cux23page_99}{99--}\protect\hyperlink{10_Chapter_Three__THE_HEROIC_DREAM.xhtmlux5cux23page_100}{100},
\protect\hyperlink{10_Chapter_Three__THE_HEROIC_DREAM.xhtmlux5cux23page_102}{102--}\protect\hyperlink{10_Chapter_Three__THE_HEROIC_DREAM.xhtmlux5cux23page_103}{103}

\emph{Voeux des Faisan},
\protect\hyperlink{10_Chapter_Three__THE_HEROIC_DREAM.xhtmlux5cux23page_97}{97},
\protect\hyperlink{10_Chapter_Three__THE_HEROIC_DREAM.xhtmlux5cux23page_101}{101--}\protect\hyperlink{10_Chapter_Three__THE_HEROIC_DREAM.xhtmlux5cux23page_103}{103},
\protect\hyperlink{20_ILLUSTRATIONS_FOLLOW_PAGE.xhtmlux5cux23page_302}{302}

\emph{Voeux du Paon,
\protect\hyperlink{10_Chapter_Three__THE_HEROIC_DREAM.xhtmlux5cux23page_76}{76}}

Vydt, Judocus,
\protect\hyperlink{20_ILLUSTRATIONS_FOLLOW_PAGE.xhtmlux5cux23page_316}{316}

Watteau, Antoine,
\protect\hyperlink{21_Chapter_Thirteen__IMAGE_AND_WORD.xhtmlux5cux23page_370}{370}

Wenzel of Luxembourg,
\protect\hyperlink{21_Chapter_Thirteen__IMAGE_AND_WORD.xhtmlux5cux23page_363}{363}

Werve, Claus de,
\protect\hyperlink{20_ILLUSTRATIONS_FOLLOW_PAGE.xhtmlux5cux23page_308}{308}

Weyden, Rogier van der: aristocracy and,
\protect\hyperlink{20_ILLUSTRATIONS_FOLLOW_PAGE.xhtmlux5cux23page_306}{306--}\protect\hyperlink{20_ILLUSTRATIONS_FOLLOW_PAGE.xhtmlux5cux23page_307}{307};
bathing scenes,
\protect\hyperlink{20_ILLUSTRATIONS_FOLLOW_PAGE.xhtmlux5cux23page_299}{299};
Bladelin Altarpiece,
\protect\hyperlink{20_ILLUSTRATIONS_FOLLOW_PAGE.xhtmlux5cux23page_316}{316};
Beaune Altarpiece,
\protect\hyperlink{20_ILLUSTRATIONS_FOLLOW_PAGE.xhtmlux5cux23page_306}{306--}\protect\hyperlink{20_ILLUSTRATIONS_FOLLOW_PAGE.xhtmlux5cux23page_307}{307},
\protect\hyperlink{20_ILLUSTRATIONS_FOLLOW_PAGE.xhtmlux5cux23page_317}{317};
dominates our impression of age,
\protect\hyperlink{19_Chapter_Twelve__ART_IN_LIFE.xhtmlux5cux23page_294}{294};
images of justice,
\protect\hyperlink{19_Chapter_Twelve__ART_IN_LIFE.xhtmlux5cux23page_297}{297};
nudity,
\protect\hyperlink{21_Chapter_Thirteen__IMAGE_AND_WORD.xhtmlux5cux23page_373}{373};
rhythm,
\protect\hyperlink{21_Chapter_Thirteen__IMAGE_AND_WORD.xhtmlux5cux23page_376}{376};
\emph{Seven Sacraments},
\protect\hyperlink{20_ILLUSTRATIONS_FOLLOW_PAGE.xhtmlux5cux23page_306}{306--}\protect\hyperlink{20_ILLUSTRATIONS_FOLLOW_PAGE.xhtmlux5cux23page_307}{307};
\emph{Pietà},
\protect\hyperlink{21_Chapter_Thirteen__IMAGE_AND_WORD.xhtmlux5cux23page_376}{376}

wills,
\protect\hyperlink{08_Chapter_One__THE_PASSIONATE_INTE.xhtmlux5cux23page_29}{29},
\protect\hyperlink{18_Chapter_Eleven__THE_FORMS_OF_THO.xhtmlux5cux23page_278}{278}

Windesheim convents: character not French,
\protect\hyperlink{15_Chapter_Eight__RELIGIOUS_EXCITAT.xhtmlux5cux23page_223}{223};
cult of saints and,
\protect\hyperlink{13_Chapter_Six__THE_DEPICTION_OF_TH.xhtmlux5cux23page_200}{200};
Denis the Carthusian and,
\protect\hyperlink{14_Chapter_Seven__THE_PIOUS_PERSONA.xhtmlux5cux23page_218}{218};
hell, fear of,
\protect\hyperlink{17_Chapter_Ten__THE_FAILURE_OF_IMAG.xhtmlux5cux23page_253}{253};
Gerson and,
\protect\hyperlink{15_Chapter_Eight__RELIGIOUS_EXCITAT.xhtmlux5cux23page_228}{228};
Hendrik van Herp,
\protect\hyperlink{15_Chapter_Eight__RELIGIOUS_EXCITAT.xhtmlux5cux23page_229}{229};
Jan-I-Don't-Know,
\protect\hyperlink{15_Chapter_Eight__RELIGIOUS_EXCITAT.xhtmlux5cux23page_222}{222};
moderation,\protect\hypertarget{25_INDEX.xhtmlux5cux23page_467}{\protect\hyperlink{17_Chapter_Ten__THE_FAILURE_OF_IMAG.xhtmlux5cux23page_265}{265}};
music and,
\protect\hyperlink{20_ILLUSTRATIONS_FOLLOW_PAGE.xhtmlux5cux23page_314}{314};
religious tension and,
\protect\hyperlink{14_Chapter_Seven__THE_PIOUS_PERSONA.xhtmlux5cux23page_203}{203};
separate sphere of life,
\protect\hyperlink{19_Chapter_Twelve__ART_IN_LIFE.xhtmlux5cux23page_295}{295},
\protect\hyperlink{20_ILLUSTRATIONS_FOLLOW_PAGE.xhtmlux5cux23page_314}{314};
witchcraft craze,
\protect\hyperlink{15_Chapter_Eight__RELIGIOUS_EXCITAT.xhtmlux5cux23page_233}{233}

witchcraft,
\protect\hyperlink{08_Chapter_One__THE_PASSIONATE_INTE.xhtmlux5cux23page_20}{20},
\protect\hyperlink{08_Chapter_One__THE_PASSIONATE_INTE.xhtmlux5cux23page_24}{24},
\protect\hyperlink{13_Chapter_Six__THE_DEPICTION_OF_TH.xhtmlux5cux23page_193}{193},
\protect\hyperlink{14_Chapter_Seven__THE_PIOUS_PERSONA.xhtmlux5cux23page_218}{218};
Arras and,
\protect\hyperlink{18_Chapter_Eleven__THE_FORMS_OF_THO.xhtmlux5cux23page_288}{288--}\protect\hyperlink{18_Chapter_Eleven__THE_FORMS_OF_THO.xhtmlux5cux23page_290}{290};
belief in,
\protect\hyperlink{18_Chapter_Eleven__THE_FORMS_OF_THO.xhtmlux5cux23page_286}{286};
d'Es-couchy and false accusation,
\protect\hyperlink{08_Chapter_One__THE_PASSIONATE_INTE.xhtmlux5cux23page_28}{28};
\emph{invultare},
\protect\hyperlink{18_Chapter_Eleven__THE_FORMS_OF_THO.xhtmlux5cux23page_288}{288};
Louis d'Orléans,
\protect\hyperlink{14_Chapter_Seven__THE_PIOUS_PERSONA.xhtmlux5cux23page_206}{206},
\protect\hyperlink{18_Chapter_Eleven__THE_FORMS_OF_THO.xhtmlux5cux23page_287}{287--}\protect\hyperlink{18_Chapter_Eleven__THE_FORMS_OF_THO.xhtmlux5cux23page_288}{288};
\emph{Malleus malificarum},
\protect\hyperlink{15_Chapter_Eight__RELIGIOUS_EXCITAT.xhtmlux5cux23page_233}{233},
\protect\hyperlink{18_Chapter_Eleven__THE_FORMS_OF_THO.xhtmlux5cux23page_286}{286},
\protect\hyperlink{18_Chapter_Eleven__THE_FORMS_OF_THO.xhtmlux5cux23page_287}{287--}\protect\hyperlink{18_Chapter_Eleven__THE_FORMS_OF_THO.xhtmlux5cux23page_288}{288};
persecution of,
\protect\hyperlink{18_Chapter_Eleven__THE_FORMS_OF_THO.xhtmlux5cux23page_286}{286--}\protect\hyperlink{18_Chapter_Eleven__THE_FORMS_OF_THO.xhtmlux5cux23page_293}{293};
reformation and,
\protect\hyperlink{13_Chapter_Six__THE_DEPICTION_OF_TH.xhtmlux5cux23page_202}{202};
\emph{Summis desiderans},
\protect\hyperlink{18_Chapter_Eleven__THE_FORMS_OF_THO.xhtmlux5cux23page_286}{286};
Windesheimers and,
\protect\hyperlink{15_Chapter_Eight__RELIGIOUS_EXCITAT.xhtmlux5cux23page_233}{233}

Württemberg, Henry of,
\protect\hyperlink{20_ILLUSTRATIONS_FOLLOW_PAGE.xhtmlux5cux23page_328}{328}

Xaintrailles,
\protect\hyperlink{10_Chapter_Three__THE_HEROIC_DREAM.xhtmlux5cux23page_79}{79}

Xavier. \emph{See} Francis Xavier, Saint

York, Edward of,
\protect\hyperlink{12_Chapter_Five__THE_VISION_OF_DEAT.xhtmlux5cux23page_164}{164}

York, House of,
\protect\hyperlink{08_Chapter_One__THE_PASSIONATE_INTE.xhtmlux5cux23page_14}{14}

York, Margaret of (duchess of Burgundy),
\protect\hyperlink{08_Chapter_One__THE_PASSIONATE_INTE.xhtmlux5cux23page_23}{23},
\protect\hyperlink{18_Chapter_Eleven__THE_FORMS_OF_THO.xhtmlux5cux23page_283}{283},
\protect\hyperlink{20_ILLUSTRATIONS_FOLLOW_PAGE.xhtmlux5cux23page_302}{302},
\protect\hyperlink{20_ILLUSTRATIONS_FOLLOW_PAGE.xhtmlux5cux23page_314}{314}

Zanobi, Saint,
\protect\hyperlink{14_Chapter_Seven__THE_PIOUS_PERSONA.xhtmlux5cux23page_215}{215}

Zola, Émile,
\protect\hyperlink{21_Chapter_Thirteen__IMAGE_AND_WORD.xhtmlux5cux23page_363}{363}
