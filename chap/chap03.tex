\chapter{THE HEROIC DREAM}

AT THE END OF THE EIGHTEENTH CENTURY WHEN medieval cultural forms were
absorbed as new values of the eighteenth century itself, in other words
at the beginning of the Romantic era, the medieval world was seen, first
and foremost, as the world of knighthood. The Romantics were inclined to
think the term ``medieval'' simply meant ``when knighthood was in
flower.'' More than anything else, they saw in the time the nodding of
plumed helmets. As paradoxical as this may sound today, they were in
many respects correct. By now, more thorough studies have taught us that
chivalry was only a part of the culture of the period and that political
and social development took place, for the most part, outside of that
form. The period of genuine feudality and the flourishing of knighthood
ended during the thirteenth century. What follows is the urban-princely
period of the medieval era, during which the dominant factors in state
and society are the commercial power of the bourgeoisie and, based on
it, the monetary powers of the princes. With the advantage of our
hindsight we are accustomed, and justifiably so, to look much more to
Ghent and Augsburg, much more to emerging capitalism and the newly
arising forms of the state, than to the nobility, whose power, here more
so, there less so, was already ``broken'' everywhere. Historical
research itself has been democratized since the days of Romanticism. But
those who are accustomed to look at late medieval times under their
political-economic aspects cannot help but notice over and over that the
sources, particularly the narrative sources, give much more attention to
the nobility and their bustle than fits our understanding. This is true
not only for later medieval times, but also for the seventeenth century.

The reason for this is the fact that the life form of the nobility still
retains its relevance over society long after the nobility as social
structure had lost its dominant meaning. The nobility undoubtedly
\protect\hypertarget{10_Chapter_Three__THE_HEROIC_DREAM.xhtmlux5cux23page_62}{}{}still
occupied the first place as a social element in the mind of the
fifteenth century, but contemporaries are considered to have placed its
importance much too high and that of the bourgeoisie much too low. They
failed to realize that the real impetus for social development was
located somewhere else than in the life and actions of a warring
nobility. This kind of reasoning would blame contemporaries for that
mistake and the age of Romanticism for uncritically adopting their view
while claiming that modern historical research has unearthed the true
facts of late medieval life. This is true as far as political and
economic life is concerned. But if we desire to understand cultural
life, we have to be aware that the illusion itself retained its value as
truth for those who lived it. Even if the noble life forms had been
nothing more than the mere surface veneer of life, the task of the
historian would still be to understand life in terms of the luster of
that finish.

But it was much more than a mere veneer. During medieval times the
concept of the division of society into estates permeates all the fibers
of theological and political reflections. This concept was by no means
limited to the well-known three: clergy, nobility, and third estate. The
term estate has not only a greater value, but also a more far-reaching
meaning. Generally speaking, any group, any function, any profession was
regarded as an estate, which meant that in addition to the division of
society into three estates, another such division into twelve estates
was
possible.\textsuperscript{\protect\hypertarget{10_Chapter_Three__THE_HEROIC_DREAM.xhtmlux5cux23id_1929}{\protect\hyperlink{23_NOTES.xhtmlux5cux23id_1930}{1}}}
Because estate is ``state,'' or ``ordo,'' it contains the notion of an
entity willed by God. During the Middle Ages the word estate or
``ordre'' includes a large number of human groupings that, to our
understanding, are rather unalike: the estates that we understand, the
professions, stand next to the married state as well as to that of
virginity; of the state of sin, \emph{estat de péchié}; of the four
\emph{estats de corps et de
bouche\protect\hypertarget{10_Chapter_Three__THE_HEROIC_DREAM.xhtmlux5cux23id_2465}{\protect\hyperlink{23_NOTES.xhtmlux5cux23id_2466}{*\textsuperscript{1}}}}
at court (panetiers, wine handlers, meat cutters, and kitchen chefs); of
the sacerdotal orders (priest, deacon, subdeacon, etc.); of the monastic
orders; and of the orders of knights. In medieval thought the term
``estate'' or ``order'' is held together in all these cases by an
awareness that each of these groups represents a divine institution,
that it is an organ of the world edifice that is just as indispensable
and just as hierarchically dignified as the heavenly throne and the
powers of the angelic ranks.

\protect\hypertarget{10_Chapter_Three__THE_HEROIC_DREAM.xhtmlux5cux23page_63}{}{}In
this beautiful image of state and society every estate was assigned a
function corresponding, not to its proven utility, but to its degree of
holiness or to its splendor. The degeneration of spirituality, the decay
of chivalric virtue, could thus be lamented without abandoning even a
small part of the ideal image; the sins of men may prevent the
realization of the ideal, but the ideal remains the basis and the guide
for social thought. The medieval image of society is static, not
dynamic.

Chastellain, the court historiographer of Philip the Good and Charles
the Bold, whose rich work is here again the best mirror of the thought
of the time, sees the society of his day in a wondrous glow. In him we
meet a man, born among the meadows of Flanders, who had before his eyes
the most splendid unfolding of bourgeois power in the Netherlands, but
who was nonetheless so blinded by the external splendor of ostentatious
Burgundian life that he regarded knightly courage and knightly virtue as
the sources of all the strength within the state.

God created the common people to work, to till the soil, to sustain life
through commerce; he created the clergy for works of faith; but he
created the nobility to extol virtue, administer justice, and so that
the beautiful members of this estate may, through their deeds and
customs, be a model for others. Chastellain assigns to the nobility the
highest tasks of the state, the protection of the church, the spreading
of the faith, the defense of the people against oppression, the
supervision of the general welfare, the struggle against violence and
tyranny, the stabilization of peace. Their qualities are truth, bravery,
integrity, and kindness. The nobility of France, says this pompous
panegyrist, meet this ideal
standard.\textsuperscript{\protect\hypertarget{10_Chapter_Three__THE_HEROIC_DREAM.xhtmlux5cux23id_1927}{\protect\hyperlink{23_NOTES.xhtmlux5cux23id_1928}{2}}}
We sense in Chastellain's entire work that he actually looked at the
events of his time through these rose-colored glasses.

This underestimation of the bourgeoisie resulted from the fact that the
stereotype usually associated with the third estate had not been
corrected by reality. This stereotype was still as simple and as summary
in nature as a calendar picture or a bas-relief depicting the labors of
the season: the toiling worker in the field, the industrious craftsman,
or the busy merchant. The figure of the powerful patrician who was
pushing the nobility from its place, the fact that the nobility
constantly renewed itself with the blood and the strength of the
bourgeoisie, had as little room in this lapidary type as the figure of
the combative guild brother and his ideal of
free\protect\hypertarget{10_Chapter_Three__THE_HEROIC_DREAM.xhtmlux5cux23page_64}{}{}dom.
In the concept of the third estate, the bourgeoisie and the workers
remained undifferentiated up until the time of the French Revolution.
The figure of the poor farmer or of the indolent and wealthy
burgher\textsuperscript{\protect\hypertarget{10_Chapter_Three__THE_HEROIC_DREAM.xhtmlux5cux23id_1925}{\protect\hyperlink{23_NOTES.xhtmlux5cux23id_1926}{3}}}
take turns in dominating the foreground of the picture of the third
estate but they do not attain a depiction in accord with their real
economic and political functions. A reform program conceived in 1412 by
an Augustine monk put forth in all seriousness that every person outside
of the nobility should be forced to do manual or field labor or be
ordered to leave the
country.\textsuperscript{\protect\hypertarget{10_Chapter_Three__THE_HEROIC_DREAM.xhtmlux5cux23id_1923}{\protect\hyperlink{23_NOTES.xhtmlux5cux23id_1924}{4}}}

This alone explains why someone like Chastellain, whose susceptibility
to ethical illusions equals his political naiveté, would assign, next to
the lofty qualities of the nobility, only low and slavish qualities to
the third estate: ``Pour venir au tiers membre qui fait le royaume
entier, c'est l'estat des bonnes villes, des marchans et des gens de
labeur, des quels ils ne convient faire si longue exposition que des
autres, pour cause que de soy il n'est gaires capable de hautes
attributions, parce qu'il est au degré servile. {[}O kevels van
Vlaanderen!{]}''\protect\hypertarget{10_Chapter_Three__THE_HEROIC_DREAM.xhtmlux5cux23id_2467}{\protect\hyperlink{23_NOTES.xhtmlux5cux23id_2468}{*\textsuperscript{2}}}
The virtues of this estate are humility and industry, obedience to their
king and a willingness to please their
lords.\textsuperscript{\protect\hypertarget{10_Chapter_Three__THE_HEROIC_DREAM.xhtmlux5cux23id_1921}{\protect\hyperlink{23_NOTES.xhtmlux5cux23id_1922}{5}}}

Is it possible that this total lack of insight into a future of
bourgeois liberties and power contributes to the pessimism of
Chastellain and kindred spirits whose expectations focused entirely on
the nobility?

Even wealthy burghers are still summarily called ``villains'' by
Chastellain.\textsuperscript{\protect\hypertarget{10_Chapter_Three__THE_HEROIC_DREAM.xhtmlux5cux23id_1919}{\protect\hyperlink{23_NOTES.xhtmlux5cux23id_1920}{6}}}
He has not the faintest notion of bourgeoisie honor. Philip the Good
habitually abused his powers in order to marry off his ``archers,''
usually members of the lower nobility, or other servants of his house,
to wealthy bourgeois widows or daughters. Parents had their daughters
marry as early as possible to thwart such advances. One widow is known
to have married only two days after her husband's funeral for the same
reason.\textsuperscript{\protect\hypertarget{10_Chapter_Three__THE_HEROIC_DREAM.xhtmlux5cux23id_1917}{\protect\hyperlink{23_NOTES.xhtmlux5cux23id_1918}{7}}}
The duke at one time encountered the stubborn resistance of a Lille
brewer who refused to let his daughter submit to such a union. The duke
had the girl abducted to a safe place whereupon the enraged father
\protect\hypertarget{10_Chapter_Three__THE_HEROIC_DREAM.xhtmlux5cux23page_65}{}{}moved,
lock, stock, and barrel, to Tournay, where, outside of the duke's
territory, he was able, without impediment, to put the matter before the
parliament in Paris. He reaps only pain and sorrow for his efforts and
falls ill. The end of the story is highly characteristic of Philip's
impulsive character and does not, according to our standards, do him
proud.\textsuperscript{\protect\hypertarget{10_Chapter_Three__THE_HEROIC_DREAM.xhtmlux5cux23id_1915}{\protect\hyperlink{23_NOTES.xhtmlux5cux23id_1916}{8}}}
He returns the girl to her mother, who had come to him as a supplicant,
but grants her request only after having mocked and humiliated her.
Chastellain, usually not afraid to criticize his lord, in this case
throws his sympathies entirely on the side of the duke. For the offended
father he can only find words such as ``ce rebelle brasseur
rustique''\protect\hypertarget{10_Chapter_Three__THE_HEROIC_DREAM.xhtmlux5cux23id_2469}{\protect\hyperlink{23_NOTES.xhtmlux5cux23id_2470}{*\textsuperscript{3}}}
and ``et encore si meschant
vilain.''\textsuperscript{\protect\hypertarget{10_Chapter_Three__THE_HEROIC_DREAM.xhtmlux5cux23id_1913}{\protect\hyperlink{23_NOTES.xhtmlux5cux23id_1914}{9}}}\protect\hypertarget{10_Chapter_Three__THE_HEROIC_DREAM.xhtmlux5cux23id_2471}{\protect\hyperlink{23_NOTES.xhtmlux5cux23id_2472}{†\textsuperscript{4}}}

Chastellain admits the great financier Jacques Coeur to his \emph{Temple
de Bocace}, a kind of hall of honor for the fame and the misfortunes of
the nobility, but not without a few words of explanation, in contrast to
Gilles de
Rais,\textsuperscript{\protect\hypertarget{10_Chapter_Three__THE_HEROIC_DREAM.xhtmlux5cux23id_1911}{\protect\hyperlink{23_NOTES.xhtmlux5cux23id_1912}{10}}}
who is admitted on account of his high birth without much ado in spite
of the horrible misdeeds he had
committed.\textsuperscript{\protect\hypertarget{10_Chapter_Three__THE_HEROIC_DREAM.xhtmlux5cux23id_1909}{\protect\hyperlink{23_NOTES.xhtmlux5cux23id_1910}{11}}}
Chastellain considers it superfluous to list the names of the burghers
who lost their lives in the bitter fighting in the defense of
Ghent.\textsuperscript{\protect\hypertarget{10_Chapter_Three__THE_HEROIC_DREAM.xhtmlux5cux23id_1907}{\protect\hyperlink{23_NOTES.xhtmlux5cux23id_1908}{12}}}

Despite this disdain of the third estate, there is in the ideal of
knighthood itself, and in the cultivation of virtue and duties held up
to the nobility, an ambiguous element revealing a less arrogant
aristocratic attitude towards the people. Side by side with the mockery
of peasants, full of hatred and contempt, as we encounter it in the
Flemish \emph{Kerelslied} and the \emph{Proverbes del vilain} there
exists during the Middle Ages a countercurrent of empathy with poor
people and their miserable lot.

\emph{Si fault de faim perir les innocens}

\emph{Dont les grans loups font chacun jour ventrée}.

\emph{Qui amassent a milliers et a cens}

\emph{Les faulx tresors; c'est le grain, c'est la blée},

\emph{Le sang, les os qui ont la terre arée}

\emph{Des povres gens, dont leur esperit crie}

\emph{Vengence à Dieu, vé à seignourie} .~.~.
\textsuperscript{\protect\hypertarget{10_Chapter_Three__THE_HEROIC_DREAM.xhtmlux5cux23id_1905}{\protect\hyperlink{23_NOTES.xhtmlux5cux23id_1906}{13}}}\protect\hypertarget{10_Chapter_Three__THE_HEROIC_DREAM.xhtmlux5cux23id_2473}{\protect\hyperlink{23_NOTES.xhtmlux5cux23id_2474}{‡\textsuperscript{5}}}

\protect\hypertarget{10_Chapter_Three__THE_HEROIC_DREAM.xhtmlux5cux23page_66}{}{}There
is always this same note of sorrow: the poor people are visited by wars
and drained of their wealth by officialdom, they live in scarcity and
misery. Everyone feeds off the peasants while they suffer patiently,
``le prince n'en sçait
riens,''\protect\hypertarget{10_Chapter_Three__THE_HEROIC_DREAM.xhtmlux5cux23id_2475}{\protect\hyperlink{23_NOTES.xhtmlux5cux23id_2476}{*\textsuperscript{6}}}
and, if they occasionally grumble and denounce the authorities, ``povres
brebis, povre fol
peuple,''\protect\hypertarget{10_Chapter_Three__THE_HEROIC_DREAM.xhtmlux5cux23id_2477}{\protect\hyperlink{23_NOTES.xhtmlux5cux23id_2478}{†\textsuperscript{7}}}
with one word the lord will restore them to calm and reason. In France
where the entire country was gradually dragged into the pitiful
devastations and uncertainties of the Hundred Years War, one theme of
the lament rises to the surface: the peasants are plundered, burned out
of their homes, and mistreated at the hands of their own and enemy war
parties. They are robbed of their draft animals and are chased from
their homes and their lands. The lament, couched in these terms, never
ends. It is echoed by the great reform-minded clerics around 1400: by
Nicolas de Clémanges in his \emph{Liber de lapsu et reparatione
justitiae},\textsuperscript{\protect\hypertarget{10_Chapter_Three__THE_HEROIC_DREAM.xhtmlux5cux23id_1904}{\protect\hyperlink{23_NOTES.xhtmlux5cux23page_405}{14}}}
by Gerson in the courageous and moving political sermon that he gave on
November 7, 1405, at the palace of the queen in Paris on the theme
``vivat rex'': ``Le pauvre homme n'aura pain à manger, sinon par
advanture aucun peu de seigle ou d'orge; sa pauvre femme gerra, et
auront quatre au six petits enfans au fouyer, ou au four, qui par
advanture sera chauld; demanderont du pain, crieront à la rage de faim.
La pauvre mère si n'aura que bouter es dens que un peu de pain ou il y
ait du sel. Or, devroit bien suffire cette misere:---viendront ces
paillars que chergeront tout .~.~. tout sera prins, et happé; et querez
qui
paye.''\textsuperscript{\protect\hypertarget{10_Chapter_Three__THE_HEROIC_DREAM.xhtmlux5cux23id_1902}{\protect\hyperlink{23_NOTES.xhtmlux5cux23id_1903}{15}}}\protect\hypertarget{10_Chapter_Three__THE_HEROIC_DREAM.xhtmlux5cux23id_2479}{\protect\hyperlink{23_NOTES.xhtmlux5cux23id_2480}{‡\textsuperscript{8}}}
Jean Jouvenel, bishop of Beauvaix, holds up the misery of the people to
the estates in bitter laments at Blois in 1433 and at Orléans in
1439.\textsuperscript{\protect\hypertarget{10_Chapter_Three__THE_HEROIC_DREAM.xhtmlux5cux23id_1900}{\protect\hyperlink{23_NOTES.xhtmlux5cux23id_1901}{16}}}
Together with the laments of the other estates about their difficulties,
presented in form of a debate, the theme of the people's misery reoccurs
in Alain
Char\protect\hypertarget{10_Chapter_Three__THE_HEROIC_DREAM.xhtmlux5cux23page_67}{}{}tier's
\emph{Quadriloge
invectif}\textsuperscript{\protect\hypertarget{10_Chapter_Three__THE_HEROIC_DREAM.xhtmlux5cux23id_1898}{\protect\hyperlink{23_NOTES.xhtmlux5cux23id_1899}{17}}}
and in Robert Gaguin's \emph{Debat du laboureur, du prestre et du
gendarme},\textsuperscript{\protect\hypertarget{10_Chapter_Three__THE_HEROIC_DREAM.xhtmlux5cux23id_1896}{\protect\hyperlink{23_NOTES.xhtmlux5cux23id_1897}{18}}}
which was inspired by Chartier's work. The chroniclers could not help
resuming the topic time and again; their subject matter demanded
it.\textsuperscript{\protect\hypertarget{10_Chapter_Three__THE_HEROIC_DREAM.xhtmlux5cux23id_1894}{\protect\hyperlink{23_NOTES.xhtmlux5cux23id_1895}{19}}}
Molinet composed a \emph{Resource du petit
peuple.\textsuperscript{\protect\hypertarget{10_Chapter_Three__THE_HEROIC_DREAM.xhtmlux5cux23id_1892}{\protect\hyperlink{23_NOTES.xhtmlux5cux23id_1893}{20}}}}
Meschinot, a serious-minded man, repeats over and over again his warning
about the growing devastations of the people.

O \emph{Dieu voyez du commun l'indigence},

\emph{Pourvoyez-y à toute diligence}:

\emph{Las! par faim, froid, paour et misere tremble}.

\emph{S'il a peché ou commis négligence}

\emph{Encontre vous, il demande indulgence}.

\emph{N'est-ce pitieé des biens que l'on lui emble?}

\emph{Il n'a plus bled pour porter au molin},

\emph{Ou lui oste draps de laine et de lin},

\emph{L'eaue, sans plus, lui demeure pour
boire}.\textsuperscript{\protect\hypertarget{10_Chapter_Three__THE_HEROIC_DREAM.xhtmlux5cux23id_1890}{\protect\hyperlink{23_NOTES.xhtmlux5cux23id_1891}{21}}}\emph{\protect\hypertarget{10_Chapter_Three__THE_HEROIC_DREAM.xhtmlux5cux23id_2309}{\protect\hyperlink{23_NOTES.xhtmlux5cux23id_2310}{*\textsuperscript{9}}}}

In a volume of complaints handed to the king on the occasion of the
assembly of the estates at Tours in 1484, the lament even assumes the
character of a political
treatise.\textsuperscript{\protect\hypertarget{10_Chapter_Three__THE_HEROIC_DREAM.xhtmlux5cux23id_1888}{\protect\hyperlink{23_NOTES.xhtmlux5cux23id_1889}{22}}}
However, everything remains on the level of a completely stereotyped and
negative pity and never becomes a political program. There is still no
indication of any well-thought-out ideas of social reform in it, and
therefore the same theme will continue to be sung by La Bruyère and by
Fénelon until deep into the eighteenth century, and, because there is no
reform, the laments of the older Mirabeau, ``l'ami des hommes,'' are
little different even though they already sound the note of future
resistance.

It was to be expected that those glorifiers of the late medieval ideal
of knighthood chimed in with these confessions of pity for the people;
this was demanded by the knight's duty to protect the weak. The idea
that true nobility is based only on virtue and that, basically, all men
are equal also was a part of the ideal of
knight\protect\hypertarget{10_Chapter_Three__THE_HEROIC_DREAM.xhtmlux5cux23page_68}{}{}hood
and was equally stereotyped and theoretical. The historical-cultural
significance of both these sentiments may well be overestimated on
occasion. Recognition of the nobility of heart is celebrated as a
triumph of the Renaissance. There are plenty of references to the fact
that Poggio expresses this idea in his \emph{De nobilitate}. We used to
find this old egalitarianism echoed in the revolutionary tenor of John
Ball's ``when Adam delved and Eve span, where was then the gentleman?''
and imagined that noblemen would tremble when they heard this text.

Both ideas were long commonplaces of courtly literature itself just as
they were in the salons of the \emph{ancien régime}. The idea of the
true nobility of the heart originated with the glorification of courtly
love in the poetry of the troubadours. It remains an ethical reflection
without any socially active reality.

\emph{Dont vient a tous souveraine noblesce?}

\emph{Du gentil cuer, paré de nobles mours}.

.\emph{~.~. Nulz n'est villains se du cuer ne lui
muet}.\textsuperscript{\protect\hypertarget{10_Chapter_Three__THE_HEROIC_DREAM.xhtmlux5cux23id_1886}{\protect\hyperlink{23_NOTES.xhtmlux5cux23id_1887}{23}}}\emph{\protect\hypertarget{10_Chapter_Three__THE_HEROIC_DREAM.xhtmlux5cux23id_2481}{\protect\hyperlink{23_NOTES.xhtmlux5cux23id_2482}{*\textsuperscript{10}}}}

The idea of equality had been borrowed by the church fathers from Cicero
and Seneca. Gregory the Great had left for the approaching medieval age
the dictum that ``Omnes namque homines natura aequales sumus.'' This
adage had been repeated in the most varied colors and shades without,
however, diminishing true inequality, since, to medieval man, the
central point of this idea was the imminent equality of death and not a
hopelessly distant equality in life. In Eustache Deschamps this idea is
expressed in close linkage with the notion of the \emph{danse macabre},
which must have been a source of solace for the injustice of the world
during late medieval times. Adam himself addresses his descendants as
follows:

\emph{Enfans, enfans de moy, Adam, venuz},

\emph{Qui après Dieu suis peres premerain}

\emph{Creé de lui, tous estes descenduz}

\emph{Naturelment de ma coste et d'Evain};

\emph{Vo mere fut. Comment est l'un villain}

\emph{Et l'autre prant le nom de gentillesce}

\emph{\protect\hypertarget{10_Chapter_Three__THE_HEROIC_DREAM.xhtmlux5cux23page_69}{}{}De
vous freres? dont vient tele noblesce?}

\emph{Je ne le sçay, se ce n'est des vertus},

\emph{Et les villains de tout vice qui blesce};

\emph{Vous estes tous d'une pel revestus}.

\emph{Quant Dieu me fist de la boe ou je fus},

\emph{Homme mortel, faible, pesant et vain},

\emph{Eve de moy, il nous crea tous nuz},

\emph{Mais l'esperit nous inspira a plain}

\emph{Perpetuel, puis eusmes soif et faim},

\emph{Labour, dolour, et enfans en tristesce};

\emph{Pour noz pechiez enfantent a destresce}

\emph{Toutes femmes; vilment estes conçuz}.

\emph{Dont vient ce nom: villain, qui les cuers blesce?}

\emph{Vous estes tous d'une pel revestuz}.

\emph{Les roys puissans, les contes et les dus},

\emph{Li gouverneur du peuple et souverain},

\emph{Quant ilz naissent, de quoi sont ilz vestuz?}

\emph{D'une orde pel}.

.~.~. \emph{Prince, pensez, sanz avoir en desdain}

\emph{Les povres gens, que la mort tient le
frain}.\textsuperscript{\protect\hypertarget{10_Chapter_Three__THE_HEROIC_DREAM.xhtmlux5cux23id_1884}{\protect\hyperlink{23_NOTES.xhtmlux5cux23id_1885}{24}}}\emph{\protect\hypertarget{10_Chapter_Three__THE_HEROIC_DREAM.xhtmlux5cux23id_2483}{\protect\hyperlink{23_NOTES.xhtmlux5cux23id_2484}{*\textsuperscript{11}}}}

In conformity with this idea, passionate defenders of the ideal of
knighthood at times intentionally list the deeds of peasant heroes in
order to point out to the nobility ``that sometimes those whom they
regard as peasants possess the greatest
courage.''\textsuperscript{\protect\hypertarget{10_Chapter_Three__THE_HEROIC_DREAM.xhtmlux5cux23id_1882}{\protect\hyperlink{23_NOTES.xhtmlux5cux23id_1883}{25}}}

\protect\hypertarget{10_Chapter_Three__THE_HEROIC_DREAM.xhtmlux5cux23page_70}{}{}Here
is the basis of all of these ideas: the nobility is called to sustain
and purify the world by fulfilling the ideal of knighthood. The true
life of nobility and the true virtue of nobility are the remedy for evil
times: the well-being and tranquility of church and kingdom and the
strength of justice depend on
it.\textsuperscript{\protect\hypertarget{10_Chapter_Three__THE_HEROIC_DREAM.xhtmlux5cux23id_1880}{\protect\hyperlink{23_NOTES.xhtmlux5cux23id_1881}{26}}}
War entered the world with Cain arid Abel and since then has
proliferated between the good and the bad. To start it is bad. The very
noble and very distinguished state of knighthood is therefore instituted
in order to protect, defend, and preserve tranquility for the people
upon whom the misery of war is usually visited most
painfully.\textsuperscript{\protect\hypertarget{10_Chapter_Three__THE_HEROIC_DREAM.xhtmlux5cux23id_1878}{\protect\hyperlink{23_NOTES.xhtmlux5cux23id_1879}{27}}}
Two things are put in the world by the will of God, we are told in the
biography of one of the purest representatives of the late medieval
ideal of knighthood, Boucicaut, like two pillars in order to support the
order of divine and human laws; without them the world would be nothing
but confusion. These two pillars are knighthood and scholarship,
``chevalerie et science, qui moult bien conviennent
ensemble.''\textsuperscript{\protect\hypertarget{10_Chapter_Three__THE_HEROIC_DREAM.xhtmlux5cux23id_1876}{\protect\hyperlink{23_NOTES.xhtmlux5cux23id_1877}{28}}}\protect\hypertarget{10_Chapter_Three__THE_HEROIC_DREAM.xhtmlux5cux23id_2485}{\protect\hyperlink{23_NOTES.xhtmlux5cux23id_2486}{*\textsuperscript{12}}}
\emph{Science, Foy, et Chevalerie} are the three lilies \emph{of Le
Chapel des fleurs de lis} of Philippe de Vitri; they represent the three
estates. Knighthood is called to protect and guard the other
two.\textsuperscript{\protect\hypertarget{10_Chapter_Three__THE_HEROIC_DREAM.xhtmlux5cux23id_1874}{\protect\hyperlink{23_NOTES.xhtmlux5cux23id_1875}{29}}}
The equivalence of knighthood and scholarship, also revealed by the
tendency to attach to the doctor's degree the same privileges as to the
title of knighthood, demonstrates the high ethical substance of the
ideal of knighthood. It places the veneration of higher aspirations and
daring next to a higher knowledge and ability. There is a need to see in
man a higher potentiality and a need to express this in the fixed forms
of two, mutually equal consecrations for higher tasks in life. But of
these two, the ideal of knighthood was much more generally and strongly
effective, because it combined, together with the ethical element, many
aesthetic elements that were intelligible to everyone.

Medieval thought in general is permeated in all regards by elements of
faith: in a similar manner the thought of that more limited group that
moves in the spheres of the court and the nobility is saturated by the
ideal of knighthood. Even notions of faith themselves are incorporated
and succumb to the spell of the idea of knighthood: the feat of arms of
the Archangel Michael was ``la première milicie et prouesse
chevaleureuse qui oncques fut mis en
\protect\hypertarget{10_Chapter_Three__THE_HEROIC_DREAM.xhtmlux5cux23page_71}{}{}exploict.''\protect\hypertarget{10_Chapter_Three__THE_HEROIC_DREAM.xhtmlux5cux23id_2487}{\protect\hyperlink{23_NOTES.xhtmlux5cux23id_2488}{*\textsuperscript{13}}}
The Archangel is the ancestor of knighthood; the ``milicie terrienne et
chevalerie
humaine''\protect\hypertarget{10_Chapter_Three__THE_HEROIC_DREAM.xhtmlux5cux23id_2489}{\protect\hyperlink{23_NOTES.xhtmlux5cux23id_2490}{†\textsuperscript{14}}}
is an earthly replication of the host of angels surrounding God's
throne.\textsuperscript{\protect\hypertarget{10_Chapter_Three__THE_HEROIC_DREAM.xhtmlux5cux23id_1872}{\protect\hyperlink{23_NOTES.xhtmlux5cux23id_1873}{30}}}

Does this high expectation respecting the fulfillment of duty by the
nobility lead to any more precise definition of the political ideas
concerning its duties? One thing is certain: the aspirations for
universal peace are based on harmony among the kings, the conquest of
Jerusalem, and the expulsion of the Turks. Philippe de Mézières, never
tiring of devising ever new schemes, dreamed of an order of knights that
would surpass the old power of the Templars and Hospitaliers. In his
\emph{Songe du vieil pélerin} he worked out a plan that seemed to
guarantee the bliss of the entire world for the foreseeable future. The
young King of France---the tract was written in 1388 when there were
still high hopes attached to the hapless Charles VI---will readily make
peace with Richard of England, just as young and as innocent of the
current quarrels as he. They would have to negotiate personally about
this peace, and should tell each other of the wondrous revelations that
had foretold the peace to them. They would have to discard all the petty
concerns that would raise obstacles if the negotiations were to be
trusted to the clergy, legal scholars, or military leaders. The King of
France should generously give up a few border towns and castles.
Immediately following the conclusion of peace, preparations for a
crusade should be made. All disputes and feuds should be ended
everywhere and the tyrannical administration of the territories be
reformed; a general council should arouse the princes of Christendom to
go to war in case sermons were not sufficient to convert the Tartars,
Turks, Jews, and
Saracens.\textsuperscript{\protect\hypertarget{10_Chapter_Three__THE_HEROIC_DREAM.xhtmlux5cux23id_1870}{\protect\hyperlink{23_NOTES.xhtmlux5cux23id_1871}{31}}}
These far-reaching plans were possibly the subject of conversations
during the friendly meetings between Mézières and the young Louis of
Orléans in the Celestine monastery in Paris. Louis of Orléans himself
also entertained such dreams of peace and crusades, though they were
tempered by concerns of practical and self-serving
politics.\textsuperscript{\protect\hypertarget{10_Chapter_Three__THE_HEROIC_DREAM.xhtmlux5cux23id_1868}{\protect\hyperlink{23_NOTES.xhtmlux5cux23id_1869}{32}}}

The image of a society sustained by the ideal of knighthood coats the
world with a peculiar color. This color peels off rather easily. If one
consults the familiar French chroniclers of the fourteenth and
\protect\hypertarget{10_Chapter_Three__THE_HEROIC_DREAM.xhtmlux5cux23page_72}{}{}fifteenth
centuries, such as the keen Froissart, the matter-of-fact Monstrelet and
d'Escouchy, the deliberate Chastellain, the courtly Olivier de la
Marche, the bombastic Molinet, they all---with the exception of Commines
and Thomas Basin---begin with high-sounding declarations that they write
for the glorification of knightly virtue and glorious feats of
arms.\textsuperscript{\protect\hypertarget{10_Chapter_Three__THE_HEROIC_DREAM.xhtmlux5cux23id_1866}{\protect\hyperlink{23_NOTES.xhtmlux5cux23id_1867}{33}}}
But none of them can stick to it, although Chastellain manages to keep
it up longest. While Froissart, himself the author of a hyperromantic
knightly epic, \emph{Méliador}, indulges his spirit in the ideal
\emph{prouesse} and \emph{grand opertises d'armes}, his journalistic pen
continuously writes a record of treason and cruelty, crafty greed and
dominance, of a profession of arms that had become entirely devoted to
the making of profit. Molinet, disregarding for the moment style and
language, constantly forgets his chivalrous intention and reports events
clearly and simply; he only occasionally recalls the noble, uplifting
task he has set for himself. The knightly tenor is even more superficial
in the writings of Monstrelet.

It is as if the spirit of these writers---a superficial spirit, one has
to admit---employed the fiction of knighthood as a corrective for the
incomprehensibility their own time had for them. It was the only form
that allowed for even an imperfect understanding of events. In reality,
wars and politics in those days were extremely formless and seemed
disconnected. War appeared in most instances as a chronic process of
isolated campaigns scattered over large areas, diplomacy as a verbose
and deficient instrument that was in one respect dominated by very
general traditional ideas and in another by a hopelessly tangled complex
of individual petty legal questions. Incapable of discerning in all this
a real social development, historiography employed the fiction of the
ideal of knighthood and thus traced everything back to a beautiful image
of princely honor and knightly virtue, to a pretty game of noble rules
that created the illusion of order. If we compare this historical
standard to the insight of a historian like Thucydides, we find it to be
a rather low vantage point. History becomes a dry report of beautiful or
seemingly beautiful feats of arms and ceremonial state occasions. Who,
then, given this vantage point, are the proper witnesses of history? The
heralds and kings of
arms,\textsuperscript{\protect\hypertarget{10_Chapter_Three__THE_HEROIC_DREAM.xhtmlux5cux23id_1864}{\protect\hyperlink{23_NOTES.xhtmlux5cux23id_1865}{34}}}
in Froissart's opinion: they are present at those noble events and are
officially called upon to judge them; they are experts in matters of
fame and honor, and fame and honor are the motifs of
historiography.\textsuperscript{\protect\hypertarget{10_Chapter_Three__THE_HEROIC_DREAM.xhtmlux5cux23id_1863}{\protect\hyperlink{23_NOTES.xhtmlux5cux23page_406}{35}}}
The statutes of the Golden
\protect\hypertarget{10_Chapter_Three__THE_HEROIC_DREAM.xhtmlux5cux23page_73}{}{}Fleece
mandated the recording of knightly feats of arms; Lefèvre de Saint Remy,
called Toison
d'or\textsuperscript{\protect\hypertarget{10_Chapter_Three__THE_HEROIC_DREAM.xhtmlux5cux23id_1861}{\protect\hyperlink{23_NOTES.xhtmlux5cux23id_1862}{36}}}
or the Herald Berry, is the model for king of arms historiographers.

As the ideal of the beautiful life the idea of knighthood has a
particular form. In its essence, it is an aesthetic ideal, built out of
colorful fantasies and uplifting sentiments. But it aspires to be an
ethical ideal: medieval thought could only turn it into an ideal of life
by linking it with piety and virtue. Knighthood always fails in that
ethical function because it is dragged down by its sinful origin; the
core of the ideal is pride elevated into beauty. Chastellain completely
understands this when he says, ``La gloire des princes pend en orgueil
et en haut péril emprendre; toutes principales puissances conviengnent
en un point estroit qui se dit
orgueil.''\textsuperscript{\protect\hypertarget{10_Chapter_Three__THE_HEROIC_DREAM.xhtmlux5cux23id_1859}{\protect\hyperlink{23_NOTES.xhtmlux5cux23id_1860}{37}}}\protect\hypertarget{10_Chapter_Three__THE_HEROIC_DREAM.xhtmlux5cux23id_2491}{\protect\hyperlink{23_NOTES.xhtmlux5cux23id_2492}{*\textsuperscript{15}}}
Taine says that honor is born from pride---stylized and elevated---the
pole of the noble life. While the essential impetus for middle or
subordinate social relationships comes from advantage, pride is the
great motivating power of the aristocracy: ``or parmi les sentiments
profonds de l'homme, il n'en est pas qui soit plus proper à se
transformer en probité, patriotisme et conscience, car l'homme fier à
besoin de son propre respect, et, pour l'obtenir il est tenté de le
mériter.''\textsuperscript{\protect\hypertarget{10_Chapter_Three__THE_HEROIC_DREAM.xhtmlux5cux23id_1857}{\protect\hyperlink{23_NOTES.xhtmlux5cux23id_1858}{38}}}\protect\hypertarget{10_Chapter_Three__THE_HEROIC_DREAM.xhtmlux5cux23id_2493}{\protect\hyperlink{23_NOTES.xhtmlux5cux23id_2494}{†\textsuperscript{16}}}
Taine undoubtedly tends to view the aristocracy in too favorable a
light. The real history of aristocracies reveals a picture in which
pride and unabashed self-aggrandizement go together very well. In spite
of all this, Taine's words remain a valid definition of the aristocratic
ideal of life. They have a certain kinship with Burckhardt's definition
of the Renaissance sense of honor: ``This is that enigmatic mixture of
conscience and egotism which still is left to modern man after he has
lost, whether by his own fault or not, everything else, faith, love, and
hope. This sense of honor is compatible with much selfishness and great
vices, and is capable of incredible deceites; but likewise,
nevertheless, everything noble that has been left in a personality can
take this feeling of honor as a point of departure and gain new strength
from this
source.''\textsuperscript{\protect\hypertarget{10_Chapter_Three__THE_HEROIC_DREAM.xhtmlux5cux23id_1855}{\protect\hyperlink{23_NOTES.xhtmlux5cux23id_1856}{39}}}

\protect\hypertarget{10_Chapter_Three__THE_HEROIC_DREAM.xhtmlux5cux23page_74}{}{}The
preoccupation with personal honor and fame---seemingly arising from a
high sense of honor at one time and from unrefined pride at
another---has been posited by Burckhardt to be the characteristic
quality of Renaissance
man.\textsuperscript{\protect\hypertarget{10_Chapter_Three__THE_HEROIC_DREAM.xhtmlux5cux23id_1853}{\protect\hyperlink{23_NOTES.xhtmlux5cux23id_1854}{40}}}
In contrast to the particular honor and fame appropriate to a given
estate, which still inspired genuinely medieval societies outside Italy,
he describes the general-human honor and fame that the Italian mind,
strongly influenced by classical antiquity, had aspired to since Dante.
It seems to me that this is one of the points where Burckhardt has
judged the distance between medieval and Renaissance times and between
western Europe and Italy to be too great. That Renaissance love of fame
and the preoccupation with honor is at the core of the knightly vision
and is of French origin. The honor of a particular estate has broadened
into a more general application, has been freed from the feudal
sensibilities and fertilized by ideas from classical antiquity. The
passionate desire to be praised by posterity is just as well known to
the courtly knight of the twelfth century and the unrefined French and
German mercenaries of the fourteenth century as it is to the beautiful
minds of the Quattrocento. The agreement for the \emph{Combat des
Trente} (March 27, 1351) between Robert de Beaumanoir and the English
captain Robert Bamborough is concluded by the latter with the words,
``and let us so act, that people in times to come will speak of it in
halls and palaces, in markets and elsewhere throughout the
world.''\textsuperscript{\protect\hypertarget{10_Chapter_Three__THE_HEROIC_DREAM.xhtmlux5cux23id_1851}{\protect\hyperlink{23_NOTES.xhtmlux5cux23id_1852}{41}}}
Chastellain, whose esteem for the ideal of knighthood is entirely
medieval, nonetheless gives complete expression to the Renaissance
spirit when he says,

\emph{Honneur semont toute noble nature}

\emph{D'aimer tout ce qui noble est en son estre}.

\emph{Noblesse aussi y adjoint sa
droiture}.\textsuperscript{\protect\hypertarget{10_Chapter_Three__THE_HEROIC_DREAM.xhtmlux5cux23id_1849}{\protect\hyperlink{23_NOTES.xhtmlux5cux23id_1850}{42}}}''\emph{\protect\hypertarget{10_Chapter_Three__THE_HEROIC_DREAM.xhtmlux5cux23id_2495}{\protect\hyperlink{23_NOTES.xhtmlux5cux23id_2496}{*\textsuperscript{17}}}}

Elsewhere he states that honor was more precious to Jews and heathens
and was observed more carefully among them for its own sake because of
the expectation of earthly praise, while Christians, through faith and
the Light, are honored in the expectation of heavenly
rewards.\textsuperscript{\protect\hypertarget{10_Chapter_Three__THE_HEROIC_DREAM.xhtmlux5cux23id_1847}{\protect\hyperlink{23_NOTES.xhtmlux5cux23id_1848}{43}}}

\protect\hypertarget{10_Chapter_Three__THE_HEROIC_DREAM.xhtmlux5cux23page_75}{}{}Froissart
is one of the earliest to recommend bravery, without any religious or
direct ethical motivation, for the sake of fame and honor and---being
the \emph{enfant terrible} that he is---for the sake of one's
career.\textsuperscript{\protect\hypertarget{10_Chapter_Three__THE_HEROIC_DREAM.xhtmlux5cux23id_1845}{\protect\hyperlink{23_NOTES.xhtmlux5cux23id_1846}{44}}}

The quest for knighthood and honor is inseparably tied to a hero
veneration in which medieval and Renaissance elements are intertwined.
Knightly life is a life without historical dimensions. It makes little
difference whether its heroes are those of the Round Table or those of
classical antiquity. Alexander had already been fully incorporated into
the ideal world of knighthood by the time when chivalrous romances
flourished. The phantasmagoric realm of classical antiquity was not yet
separated from that of the Round Table. King René describes in a poem a
colorful combination. How he has seen the gravestones of Lancelot,
Caesar, David, Hercules, Paris, Troilus, among others, all marked with
their particular coat of
arms.\textsuperscript{\protect\hypertarget{10_Chapter_Three__THE_HEROIC_DREAM.xhtmlux5cux23id_1843}{\protect\hyperlink{23_NOTES.xhtmlux5cux23id_1844}{45}}}
Knighthood itself was considered to be Roman. ``Et bien entretenoit,''
it is said of Henry IV of England, ``la discipline de chevalerie, comme
jadis faisoient les
Rommains.''\textsuperscript{\protect\hypertarget{10_Chapter_Three__THE_HEROIC_DREAM.xhtmlux5cux23id_1841}{\protect\hyperlink{23_NOTES.xhtmlux5cux23id_1842}{46}}}\protect\hypertarget{10_Chapter_Three__THE_HEROIC_DREAM.xhtmlux5cux23id_2497}{\protect\hyperlink{23_NOTES.xhtmlux5cux23id_2498}{*\textsuperscript{18}}}
The rise of classicism brings some clarity to the historical picture of
antiquity. The Portuguese nobleman Vasco de Lucena, who translated
Quintus Curtius for Charles the Bold, explains to Charles that he is
presenting an authentic Alexander, just as Maerlant had done a century
and a half earlier, an Alexander whose story had been stripped of the
lies with which all the ordinary histories had disfigured
it.\textsuperscript{\protect\hypertarget{10_Chapter_Three__THE_HEROIC_DREAM.xhtmlux5cux23id_1839}{\protect\hyperlink{23_NOTES.xhtmlux5cux23id_1840}{47}}}
But the intent to offer to the king a model worthy of emulation is
stronger than ever and few princes are as self-conscious in their desire
to equal the ancients through great and splendid deeds as is Charles the
Bold. From his youth he had the heroic deeds of Gawain and Lancelot read
to him; later, classical antiquity gained the upper hand. There were
regularly a few hours of reading in ``les haultes histories de
Romme''\protect\hypertarget{10_Chapter_Three__THE_HEROIC_DREAM.xhtmlux5cux23id_2501}{\protect\hyperlink{23_NOTES.xhtmlux5cux23id_2502}{†\textsuperscript{19}}}
before going to
sleep.\textsuperscript{\protect\hypertarget{10_Chapter_Three__THE_HEROIC_DREAM.xhtmlux5cux23id_1837}{\protect\hyperlink{23_NOTES.xhtmlux5cux23id_1838}{48}}}
Most pleasing to him were the heroes of antiquity: Caesar, Hannibal, and
Alexander, ``lesques il vouloit ensuyre et
contrefaire.''\textsuperscript{\protect\hypertarget{10_Chapter_Three__THE_HEROIC_DREAM.xhtmlux5cux23id_1835}{\protect\hyperlink{23_NOTES.xhtmlux5cux23id_1836}{49}}}\protect\hypertarget{10_Chapter_Three__THE_HEROIC_DREAM.xhtmlux5cux23id_2499}{\protect\hyperlink{23_NOTES.xhtmlux5cux23id_2500}{†\textsuperscript{20}}}
All his contemporaries place great emphasis on these deliberate
emulations as the impetus for his own deeds, ``Il désiroit grand
\protect\hypertarget{10_Chapter_Three__THE_HEROIC_DREAM.xhtmlux5cux23page_76}{}{}gloire''---says
Commines---``qui estoit ce qui plus le mettoit en ses guerres que nulle
autre chose; et eust bien voulu ressembler à ses anciens princes dont il
a esté tant parlé après leur
mort.''\textsuperscript{\protect\hypertarget{10_Chapter_Three__THE_HEROIC_DREAM.xhtmlux5cux23id_1833}{\protect\hyperlink{23_NOTES.xhtmlux5cux23id_1834}{50}}}\protect\hypertarget{10_Chapter_Three__THE_HEROIC_DREAM.xhtmlux5cux23id_2503}{\protect\hyperlink{23_NOTES.xhtmlux5cux23id_2504}{*\textsuperscript{21}}}
Chastellain saw him put to use for the first time that high feeling for
great deeds and beautiful gestures in the ancient style. This occasion
was provided when he made his first entry as duke into Mechelen in 1467.
He went there to punish a rebellion. The matter was formally
investigated and handled by the court. One of the rebels was sentenced
to death while others were exiled forever. The scaffold is erected on
the town square; the duke takes his seat opposite; the condemned man has
already knelt down; the executioner bares his sword; at that moment
Charles, who has kept his intentions secret up to this point, calls out:
``Stop! Take off his blindfold and let him stand up.''

``Et me parcus de lors''---says Chastellain---``que le coeur luy estoit
en haut singulier propos pour le temps à venir et pour acquérir gloire
et renommée en singulière
oeuvre.''\textsuperscript{\protect\hypertarget{10_Chapter_Three__THE_HEROIC_DREAM.xhtmlux5cux23id_1831}{\protect\hyperlink{23_NOTES.xhtmlux5cux23id_1832}{51}}}\protect\hypertarget{10_Chapter_Three__THE_HEROIC_DREAM.xhtmlux5cux23id_2505}{\protect\hyperlink{23_NOTES.xhtmlux5cux23id_2506}{†\textsuperscript{22}}}

The example of Charles the Bold is quite suitable to convince us that
the spirit of the Renaissance and its yearning for the beautiful life of
antiquity has its direct roots in the ideal of knighthood. If compared
to the Italian virtuoso, there is merely a difference in degrees of
literacy and in taste. Charles still reads his classics in translations,
and his style of life is still flamboyantly Gothic.

The same inseparability of knightly and Renaissance elements can be
found in the cult of the Nine Worthies, ``les neuf preux.'' This group
of nine heroes---three pagans, three Jews, three Christians---appears
first in chivalric literature: the earliest account is found around 1312
in the ``Voeux du paon'' by Jacques de
Longuyon.\textsuperscript{\protect\hypertarget{10_Chapter_Three__THE_HEROIC_DREAM.xhtmlux5cux23id_1829}{\protect\hyperlink{23_NOTES.xhtmlux5cux23id_1830}{52}}}
The choice of heroes betrays the close linkage with knightly
romanticism: Hector, Caesar, Alexander; Joshua, David, Judas Maccabaeus;
Arthur, Charlemagne, and Godfrey of Bouillon. Eustache Deschamps adopts
this idea from his teacher, Guillaume de Machaut, and devotes numerous
poems to
it.\textsuperscript{\protect\hypertarget{10_Chapter_Three__THE_HEROIC_DREAM.xhtmlux5cux23id_1827}{\protect\hyperlink{23_NOTES.xhtmlux5cux23id_1828}{53}}}
It is likely that the
\protect\hypertarget{10_Chapter_Three__THE_HEROIC_DREAM.xhtmlux5cux23page_77}{}{}taste
for symmetry that was so characteristic of the late medieval mind
accounts for the fact that he added nine brave women to the list of
brave men. For this purpose he chose a number of classical figures, some
rather peculiar, from Justin and other literary sources. He included
Penthesilea, Tomyris, Semiramis, and mangled most of the names
considerably. This did not hinder the popularity of the idea and so
\emph{preux} and \emph{preuses} can be found in later works, such as
\emph{Le Jouvencel}. They are depicted on tapestries, coats of arms are
designed for them, and all eighteen lead the procession when Henry VI of
England makes his entry into Paris in
1431.\textsuperscript{\protect\hypertarget{10_Chapter_Three__THE_HEROIC_DREAM.xhtmlux5cux23id_1825}{\protect\hyperlink{23_NOTES.xhtmlux5cux23id_1826}{54}}}

What demonstrates how very much alive these notions remained during the
fifteenth century and later is the fact that they became the object of
parody. Molinet has fun with nine \emph{preux de
gourmandise};\textsuperscript{\protect\hypertarget{10_Chapter_Three__THE_HEROIC_DREAM.xhtmlux5cux23id_1823}{\protect\hyperlink{23_NOTES.xhtmlux5cux23id_1824}{55}}}
Francis I dresses occasionally \emph{à l'antique} in order to represent
one of the
\emph{preux}.\textsuperscript{\protect\hypertarget{10_Chapter_Three__THE_HEROIC_DREAM.xhtmlux5cux23id_1821}{\protect\hyperlink{23_NOTES.xhtmlux5cux23id_1822}{56}}}

But Deschamps has expanded this notion in yet another way than merely by
adding female pendants. By adding to the nine a contemporary Frenchman,
Bertrand du Guesclin, as the tenth \emph{preux}, he tied the veneration
of heroic virtue to the here and now, and thus transposed the
\emph{preux} into the sphere of rising French military
patriotism.\textsuperscript{\protect\hypertarget{10_Chapter_Three__THE_HEROIC_DREAM.xhtmlux5cux23id_1819}{\protect\hyperlink{23_NOTES.xhtmlux5cux23id_1820}{57}}}
This idea, too, was successful: Louis of Orléans saw to it that the
image of the courageous \emph{connétable} was included as the tenth
\emph{preux} in the grand hall of
Coucy.\textsuperscript{\protect\hypertarget{10_Chapter_Three__THE_HEROIC_DREAM.xhtmlux5cux23id_1817}{\protect\hyperlink{23_NOTES.xhtmlux5cux23id_1818}{58}}}
There were good reasons for the special attention Louis devoted to the
memory of du Guesclin; the \emph{connétable} had held him during his
baptism and had at that time placed a sword in his hand. The figure of
this brave and calculating Breton warrior came to be venerated as a
national military hero. It should be noted that during the fifteenth
century this veneration did not give first place to Jeanne d'Arc. Any
number of military leaders who had fought either side by side with her
or against her held a much larger and more honored place in the
imagination of their contemporaries than did the peasant girl from
Domrémy. People spoke of her without emotion or veneration, and rather
as a curiosity. Chastellain, who managed to shift from his Burgundian
sentiments to a pathetic French loyalty whenever the occasion demanded,
composed a \emph{mystère} on the death of Charles VII in which all the
leaders who had fought for him against the English---Dunois, Jean de
Bueil, Xaintrailles, la Hire, and a large number of less well known
individuals---like a hall of fame
\protect\hypertarget{10_Chapter_Three__THE_HEROIC_DREAM.xhtmlux5cux23page_78}{}{}for
the brave, recite a verse recalling their
deeds.\textsuperscript{\protect\hypertarget{10_Chapter_Three__THE_HEROIC_DREAM.xhtmlux5cux23id_1815}{\protect\hyperlink{23_NOTES.xhtmlux5cux23id_1816}{59}}}
They remind one, for a moment at least, of a gallery of Napoleonic
generals. But the Maid is not among them.

The Burgundian princes kept in their treasure rooms a number of relics
of a romantic sort that were linked to heroes: a sword of St. George,
decorated with his coat of arms; a sword that had belonged to ``messire
Bertram de Claiquin'' (du Guesclin); a tooth of the boar of Garin le
Loherain; the psalter from which St. Louis studied during
childhood.\textsuperscript{\protect\hypertarget{10_Chapter_Three__THE_HEROIC_DREAM.xhtmlux5cux23id_1813}{\protect\hyperlink{23_NOTES.xhtmlux5cux23id_1814}{60}}}
How much the fantastic aspects of knighthood and religion are merging
here! One more step, and we have arrived at the collarbone of Livy that
was received by Pope Leo X with all solemnity as if it were a
relic.\textsuperscript{\protect\hypertarget{10_Chapter_Three__THE_HEROIC_DREAM.xhtmlux5cux23id_1811}{\protect\hyperlink{23_NOTES.xhtmlux5cux23id_1812}{61}}}

The literary form of late medieval hero veneration is the biography of
the perfect knight. Some, like Gilles de Trazegnies, had already become
legendary figures; but the most important biographies deal with
contemporaries, as, for example, Boucicaut, Jean de Bueil, and Jacques
de Lalaing.

Jean le Meingre, usually called \emph{Maréchal} Boucicaut, had served
his country during a serious crisis. He was with John the Fearless at
Nicopolis in 1396 when the French knightly nobility had carelessly
ventured forth to drive the Turks out of Europe and were annihilated by
Sultan Bayazid. He was captured again in 1415 at Agincourt and died in
captivity six years later. One of his admirers recorded his deeds in
1409, while he was still alive. This account was based on very good
information and
documentation;\textsuperscript{\protect\hypertarget{10_Chapter_Three__THE_HEROIC_DREAM.xhtmlux5cux23id_1809}{\protect\hyperlink{23_NOTES.xhtmlux5cux23id_1810}{62}}}
however, it is not like a piece of contemporary history, but rather like
the depiction of an ideal knight. The reality of a life of sudden
reversals disappears under the beautiful gloss of the knightly image.
The terrible catastrophe at Nicopolis appears only in muted colors in
this \emph{Livre des faicts}. Boucicaut is presented as the type of the
simple and pious and yet courtly and well-read knight. The contempt for
wealth, mandatory for a true knight, is revealed in the words of
Boucicaut's father, who did not intend either to enlarge or to reduce
the size of his inherited estate when he said: my children, be honest
and brave and you will therefore not lack anything; and if you are
worthless it would be a pity to leave you too
much.\textsuperscript{\protect\hypertarget{10_Chapter_Three__THE_HEROIC_DREAM.xhtmlux5cux23id_1807}{\protect\hyperlink{23_NOTES.xhtmlux5cux23id_1808}{63}}}
Boucicaut's piety is of a strictly puritan nature. He gets up early and
spends about three hours in prayer. No matter how much pressed for time
or how busy, he kneels to hear mass twice a day. On Fridays he wears
black, on Sundays and holy days he makes a pilgrimage on
\protect\hypertarget{10_Chapter_Three__THE_HEROIC_DREAM.xhtmlux5cux23page_79}{}{}foot
or has someone read to him from the lives of the saints or from the
histories of ``des vaillans trepassez, soit Romains ou
autres,''\protect\hypertarget{10_Chapter_Three__THE_HEROIC_DREAM.xhtmlux5cux23id_2507}{\protect\hyperlink{23_NOTES.xhtmlux5cux23id_2508}{*\textsuperscript{23}}}
or he engages in pious conversation. He is temperate and frugal, speaks
little, and if he does, mostly about God, the saints, the virtues, or
chivalry. He inspires all his servants to be devout and above reproach
and he makes them give up
cursing.\textsuperscript{\protect\hypertarget{10_Chapter_Three__THE_HEROIC_DREAM.xhtmlux5cux23id_1805}{\protect\hyperlink{23_NOTES.xhtmlux5cux23id_1806}{64}}}
He is an active proponent of the noble and chaste service to women; he
honors all women for the sake of one and founds the Ordre de l'écu verd
à la dame blanche for the defense of women that earned him the praise of
Christine de
Pisan.\textsuperscript{\protect\hypertarget{10_Chapter_Three__THE_HEROIC_DREAM.xhtmlux5cux23id_1803}{\protect\hyperlink{23_NOTES.xhtmlux5cux23id_1804}{65}}}
In Genoa, where he had gone in 1401 to run the government for Charles
VI, when at one time he politely bowed to two ladies who had greeted
him, his page boy said, ``\,'Monseigneur, qui sont ces deux femmes à qui
vous avez si grans reverences faictes?'---'Huguenin,' dit-il, `je ne
scay.' Lors luy dist: `Monseigneur, elles sont filles
communes.'---'Filles communes,' dist-il, `Huguenin, j'ayme trop mieulx
faire reverence à dix filles communes que avoir failly à une femme de
bien.'\,''\textsuperscript{\protect\hypertarget{10_Chapter_Three__THE_HEROIC_DREAM.xhtmlux5cux23id_1802}{\protect\hyperlink{23_NOTES.xhtmlux5cux23page_407}{66}}}\protect\hypertarget{10_Chapter_Three__THE_HEROIC_DREAM.xhtmlux5cux23id_2509}{\protect\hyperlink{23_NOTES.xhtmlux5cux23id_2510}{†\textsuperscript{24}}}
\protect\hypertarget{10_Chapter_Three__THE_HEROIC_DREAM.xhtmlux5cux23id_2511}{\protect\hyperlink{23_NOTES.xhtmlux5cux23id_2512}{‡\textsuperscript{25}}}
His motto read, ``Ce que vous vouldrez''{}'66‡ deliberately kept
mysterious as befits a slogan. Does he have in mind the surrender of his
will to the lady to whom he is truly dedicated? Or should we view it as
a generally relaxed attitude towards life such as we would expect to
encounter only in much later times?

The beautiful portrait of the ideal knight was painted in these colors
of piety and restraint, simplicity and loyalty. It is only to be
expected that the real Boucicaut did not conform to this image in every
respect. Violence and greed for gold, the usual concerns for his
estate---these were no strangers even to this noble
figure.\textsuperscript{\protect\hypertarget{10_Chapter_Three__THE_HEROIC_DREAM.xhtmlux5cux23id_1800}{\protect\hyperlink{23_NOTES.xhtmlux5cux23id_1801}{67}}}

But the model knight came also to be seen in an entirely different hue.
The biographic novel about Jean de Bueil, called Le Jouvencel, was
written about half a century later than the life of Boucicaut and this
explains in part the difference in perception. Jean de Bueil was a
captain who had fought under the flag of Jeanne d'Arc, later
participated in the Praguerie uprising (1440) and in the war ``du
\protect\hypertarget{10_Chapter_Three__THE_HEROIC_DREAM.xhtmlux5cux23page_80}{}{}bien
public,'' and died in 1477. While out of favor with the king (about
1465), he had suggested that three of his servants write the story of
his life, to be entitled \emph{Le
Jouvencel}.\textsuperscript{\protect\hypertarget{10_Chapter_Three__THE_HEROIC_DREAM.xhtmlux5cux23id_1798}{\protect\hyperlink{23_NOTES.xhtmlux5cux23id_1799}{68}}}
In contrast to the life of Boucicaut, where the historical form has a
romantic spirit, \emph{Le Jouvencel} reveals, in its invented form, real
facts, at least in its first part. It is probably the result of its
multiple authorship that the story continues to lose itself in a
sugarcoated romanticism. There is found the story of the terrifying
campaign of the French marauders in Swiss territory in 1444 and that of
the battle of St. Jacob on the Birs, where the peasants of the Basel
region met their Thermopylae, stories adorned with the phony
embellishments of hackneyed pastoral
\emph{Minnelieder}.\textsuperscript{\protect\hypertarget{10_Chapter_Three__THE_HEROIC_DREAM.xhtmlux5cux23id_1796}{\protect\hyperlink{23_NOTES.xhtmlux5cux23id_1797}{69}}}

In stark contrast, the first part \emph{of Le Jouvencel} offers a simple
and genuine picture of the reality of war in those days such as is
rarely found anywhere else. Incidentally, these authors, too, do not
mention Jeanne d'Arc, who had been a comrade-in-arms of their lord. It
is his heroic deeds they glorify. How well he must have told them his
war stories. Here we find the announcement of the early stirring of
France's military spirit that was later to bring forth the figure of the
musketeer, the \emph{grognuard}, and the \emph{poilu}. The attempt to
glorify knighthood only betrays itself in the opening passages, where
young people are exhorted to become acquainted through this story with a
life at arms and are warned against the follies of pride, envy, and
greed. Both the pious and the \emph{Minne} elements, so strong in
Boucicaut, are absent in the first part of \emph{Le Jouvencel}. What we
do encounter here is the misery of war, its deprivations and monotony
and the brash courage needed to endure those deprivations and face its
dangers. A castellan musters his garrison and counts only fifteen
horses, all emaciated nags; most are not shoed. On each horse he puts
two men, most of these one-eyed or crippled. To mend the captain's
clothing, attempts are made to capture the enemy's laundry. A stolen cow
is returned to the enemy captain, upon his request, with all civilities.
A description of a nightly patrol across the fields lets us breathe the
night air and sense the mighty
quiet.\textsuperscript{\protect\hypertarget{10_Chapter_Three__THE_HEROIC_DREAM.xhtmlux5cux23id_1794}{\protect\hyperlink{23_NOTES.xhtmlux5cux23id_1795}{70}}}
\emph{Le Jouvencel} marks the transition from the type of knight to the
type of national military man. The hero of the book releases his
unfortunate prisoners on condition that they become good Frenchmen.
Having attained high honors, he yearns back to that life of adventure
and freedom.

Such a realistic knightly figure (which, as already mentioned, is
\protect\hypertarget{10_Chapter_Three__THE_HEROIC_DREAM.xhtmlux5cux23page_81}{}{}not
consistently presented as such to the end of the story) could as yet not
be fashioned by Burgundian literature, which was too old-fashioned, too
solemn, and too much more a captive of feudal ideas than pure French
literature to be ready for such a task. Jacques de Lalaing, compared to
Le Jouvencel, is an old-fashioned curiosity, described in terms of the
clichés of earlier knight-errants such as Gillon de Trazegnies. The book
about the deeds of this venerated Burgundian hero tells more of romantic
tournaments than about real
war.\textsuperscript{\protect\hypertarget{10_Chapter_Three__THE_HEROIC_DREAM.xhtmlux5cux23id_1792}{\protect\hyperlink{23_NOTES.xhtmlux5cux23id_1793}{71}}}

The psychology of wartime bravery has perhaps never been expressed,
earlier or later, as simply and as truly as in the following words from
\emph{Le
Jouvencel:\textsuperscript{\protect\hypertarget{10_Chapter_Three__THE_HEROIC_DREAM.xhtmlux5cux23id_1790}{\protect\hyperlink{23_NOTES.xhtmlux5cux23id_1791}{72}}}}

C'est joyeuse chose que la guerre .~.~. On s'entr'ayme tant à la guerre.
Quant on voit sa querelle bonne et son sang bien combatre, la larme en
vient à l'ueil. Il vient une doulceur au cueur de loyaulté et de pitié
de veoir son amy, qui si vaillamment expose son corps pour faire et
accomplir le commandement de nostre createur. Et puis on se dispose
d'aller mourir ou vivre avec luy, et pour amour ne l'abandonner point.
En cela vient une délectation telle que, qui ne l'a essaiié, il n'est
homme qui sceust dire quel bien c'est. Pensez-vous que homme qui face
cela craigne la mort? Nennil; car il est tant reconforté il est si ravi,
qu'il ne scet où il est. Vraiement il n'a paour de
rien.\protect\hypertarget{10_Chapter_Three__THE_HEROIC_DREAM.xhtmlux5cux23id_2513}{\protect\hyperlink{23_NOTES.xhtmlux5cux23id_2514}{*\textsuperscript{26}}}

This could just as well have come from a modern soldier as from a knight
of the fifteenth century. It has nothing to do with the knightly ideal
\emph{per se}, but reflects the emotions constituting the background of
pure fighting courage itself: the trembling stepping away from narrow
egoism into the excitement of facing mortal danger, the deeply touching
experience of the bravery of one's comrades,
\protect\hypertarget{10_Chapter_Three__THE_HEROIC_DREAM.xhtmlux5cux23page_82}{}{}the
enjoyment of loyalty and self-sacrifice. This primitive ascetic
excitement is the basis on which the ideal of knighthood was built into
a noble fantasy of male perfection, a close kin of the Greek
\emph{kalokagathia}, a purposeful striving for the beautiful life that
energetically inspired a number of centuries---but also a mask behind
which a world of greed and violence could hide.

Wherever the ideal of knighthood is professed in its purest form,
emphasis is placed on the ascetic element. In its first flowering it was
paired naturally, or even necessarily, with the monkish ideal in the
spiritual orders of knighthood at the time of the Crusades. But as
reality time and again gave a cruel lie to the ideal, it sank more and
more back into the realm of imagination, where it was able to preserve
features of noble asceticism that were rarely evident in the midst of
social realities. The knight-errant, as well as the Templar, is poor and
free of earthly ties. That ideal of the noble propertyless warrior, says
William James, still dominates, ``sentimentally if not practically, the
military and aristocratic view of life. We glorify the soldier as the
man absolutely unencumbered. Owning nothing but his bare life, and
willing to toss that up at any moment when the cause commands him, he is
the representative of unhampered freedom in ideal
directions.''\textsuperscript{\protect\hypertarget{10_Chapter_Three__THE_HEROIC_DREAM.xhtmlux5cux23id_1788}{\protect\hyperlink{23_NOTES.xhtmlux5cux23id_1789}{73}}}

Linking the knightly ideal with the higher elements of religious
consciousness, compassion, justice, and fidelity is therefore by no
means artificial or superficial. Yet, on the other hand, they are also
not that which turns knighthood into the beautiful form of life
\emph{kat'exochen} {[}par excellence{]}. Neither could knighthood's
immediate roots in the manly lust for combat have been elevated if love
for women had not been the burning passion that bestowed the warmth of
life on that complex of emotion and idea.

The profound ascetic element of courageous self-sacrifice that is
characteristic of the knightly ideal is most intimately tied to the
erotic base of this view of life and is perhaps merely the ethical
transformation of an unsatisfied desire. It is not only in literature
and the fine arts that the yearning for love receives its shaping and
its stylization. The desire to give love a noble style and noble form
finds also a broad arena for its unfolding in the forms of life
themselves; in courtly intimacy, social games, jokes and sport. Here,
too, love is continuously sublimated and romanticized: in this, life
imitates literature, but in the final analysis, it is literature that
learns everything from life. The knightly view of love is not based in
\protect\hypertarget{10_Chapter_Three__THE_HEROIC_DREAM.xhtmlux5cux23page_83}{}{}literature
but rather in life. The motif of the knight and his beloved is rooted in
the real conditions of life.

The knight and his beloved, the hero for the sake of love, constitute
the most primary and unchanging romantic motif that arises and must
arise everywhere anew. It is the direct transformation of sensual
passion into an ethical or quasi-ethical self-denial. It arises directly
from the need, known to every sixteen-year-old male, to display his
courage before a woman, to expose himself to dangers and to be strong,
to suffer and to shed his blood. The expression and fulfillment of this
desire, which seem to be unobtainable, are replaced and
elevated\textsuperscript{\protect\hypertarget{10_Chapter_Three__THE_HEROIC_DREAM.xhtmlux5cux23id_1786}{\protect\hyperlink{23_NOTES.xhtmlux5cux23id_1787}{74}}}
to the dream of heroic deeds for love. This immediately posits death as
an alternative to fulfillment, and satisfaction is, so to speak, thus
guaranteed in either direction.

But the dream of a heroic deed for love, a deed that now fills and
infatuates the heart, grows and grows like a luxuriant plant. The
initially simple theme has soon spent its force and the mind craves new
settings of the same theme. Passion itself imposes stronger colors on
the dream of suffering and renunciation. The heroic deed has to consist
of freeing or rescuing the woman from even the gravest of danger. A
stronger stimulus is thus added to the original motif. At first it is
the subject himself who wants to suffer for his woman, but soon this
motif is joined by that of the wish to rescue the very object of his
desires from suffering. I wonder if at base we can always trace the
rescue back to the act of preserving virginity, of fending off another
and securing the woman for the rescuer himself? In any event, this is
the highest knightly-erotic motif: the young hero who liberates the
virgin. Even if the enemy occasionally is an unsuspecting dragon, the
sexual element remains just beneath the surface.

Liberating the virgin is the most original romantic motif, forever
young. How is it possible that a nowadays outdated explanation of myth
saw in this the image of a natural phenomenon while the directness of
the thought could be tested daily by
everyone!\textsuperscript{\protect\hypertarget{10_Chapter_Three__THE_HEROIC_DREAM.xhtmlux5cux23id_1784}{\protect\hyperlink{23_NOTES.xhtmlux5cux23id_1785}{75}}}
Although in literature the motif may be avoided for a time because of
excessive repetition, it always comes back again in new forms, as, for
instance, in the romance of the cinematic cowboy. There is no doubt that
in the individual conception of love outside of literature it has always
remained strong.

It is difficult to ascertain to what extent the conception of the
hero-lover reveals the masculine or how far the feminine view of
\protect\hypertarget{10_Chapter_Three__THE_HEROIC_DREAM.xhtmlux5cux23page_84}{}{}love.
Is it in the image of willful suffering that a male wishes to see
himself, or is it the will of the female that he show himself this way?
The former is more likely. In general, the depiction of love as a
cultural form expresses the male conception almost exclusively, at least
until most recent times. The view of love held by woman always remains
hidden and veiled. It is a tender and deep mystery. And it does not even
need the romantic elevation into the heroic. Through its character of
self-sacrifice and its unbreakable link to motherhood, this view extols
itself without heroic fantasy and subservience to the egotistically
erotic. Womanly expressions of love are missing from literature not only
because literature originated primarily among men, but also because for
women, as far as love is concerned, the literary element is much less
indispensable.

The figure of the noble savior who willingly suffers for the sake of his
beloved is primarily a product of the male imagination, showing man as
he wishes to see himself. The tension in his dream of the liberator
increases whenever he appears with his true identity hidden and is only
recognized after the heroic deed is done. The romantic motif of the
hidden identity of the hero is most certainly rooted in the female
conception of love. In the ultimate realization of the image of manly
strength and courage in the form of the warrior on horseback, female
yearning to worship strength and masculine physical pride flow together.

Medieval society cultivated these primitive romantic motifs with boyish
insatiability. While the higher literary forms were refined into more
ethereal, reserved or spiritual and titillating expressions of desire,
the knightly novel repeated, time and again, examples of a fascination
that is not always intelligible to us. We frequently are of the opinion
that the age should have long outgrown these childish imaginations and
take Froissart's \emph{Méliador} or \emph{Perceforest} to be late
flowers of the knightly adventure story and anachronisms in their own
time. But this is as little the case then as it is in the case of the
sensational novels of our own time; however, all this is not pure
literature, but, so to speak, applied art. It is the need for models for
the erotic imagination that keeps this literature alive and continuously
renews it. There is a revival in the middle of the Renaissance in the
Amadis novels. If La Noue can still assure us in the latter part of the
sixteenth century that the Amadis novels caused an ``esprit de vertige''
among the same generation that had undergone the tempering of the
Renaissance and humanism, how
\protect\hypertarget{10_Chapter_Three__THE_HEROIC_DREAM.xhtmlux5cux23page_85}{}{}great
the romantic receptiveness must have been among the entirely
unsophisticated generation of 1400!

The enchantment of the romance of love was not only to be experienced in
reading, but also in games and performances. There are two forms in
which the game may appear: dramatic representations and sport. The
latter form was by far the most important during medieval times. Drama
was still, to a great extent, filled with other, pious, subject matter:
romantic issues were only exceptions. Medieval sport, on the other hand,
and first of all the tournament, was by itself dramatic to a high degree
and possessed at the same time a highly erotic ambiance. Sports retain
at all times such a dramatic and erotic element; today's rowing or
soccer contests contain much more of the emotional qualities of a
medieval tournament than athletes and spectators themselves are perhaps
conscious of. But while modern sports have returned to a natural, almost
Greek simplicity and beauty, medieval, or at least late medieval,
tournaments were a sport overladen with embellishments and heavily
elaborated, in which the dramatic and romantic element was so
deliberately worked out that it virtually came to serve the function of
drama itself.

The late Middle Ages is one of the end periods in which the cultural
life of the higher circles has become, almost in its entirety, social
play. Reality is crass, hard, and cruel; one turns back to the beautiful
dream of the knightly ideal and builds the game of life on this
foundation. One plays masked as Lancelot. All this is a tremendous
self-deception, the glaring unreality of which is only bearable because
the lie is denied by faint mockery. The entire knightly culture of the
fifteenth century is dominated by a precarious balance between
sentimental seriousness and easy derision. All those knightly terms of
honor and fidelity and noble
\emph{Minne\textsuperscript{\protect\hypertarget{10_Chapter_Three__THE_HEROIC_DREAM.xhtmlux5cux23id_1782}{\protect\hyperlink{23_NOTES.xhtmlux5cux23id_1783}{76}}}}
are handled with perfect seriousness, but the rigid face occasionally
relaxes for a moment into a smile. Where else but in Italy could this
mood first turn into deliberate parody: in Pulci's \emph{Morgante} and
Bonardo's \emph{Orlando Innamorato}. But even then and there, the
knightly-romantic sentiment emerges victorious again because, in
Ariosto, open mockery gives way to a wondrous transcendence of pain and
seriousness. The knightly fantasy has found its most classical
expression.

How can we doubt the seriousness of the knightly ideal in French society
around 1400? In the noble Boucicaut, the literary type of
\protect\hypertarget{10_Chapter_Three__THE_HEROIC_DREAM.xhtmlux5cux23page_86}{}{}the
model knight, the romantic foundation of the knightly ideal of life is
still as strong as anywhere. It is love, he says, which is strongest in
making young hearts avid for noble knightly struggles. He himself serves
his lady in the old courtly forms: ``Toutes servoit, toutes honnoroit
pour l'amour d'un. Son parler estoit gracieux, courtois et craintif
devant sa
dame.''\textsuperscript{\protect\hypertarget{10_Chapter_Three__THE_HEROIC_DREAM.xhtmlux5cux23id_1780}{\protect\hyperlink{23_NOTES.xhtmlux5cux23id_1781}{77}}}\protect\hypertarget{10_Chapter_Three__THE_HEROIC_DREAM.xhtmlux5cux23id_2515}{\protect\hyperlink{23_NOTES.xhtmlux5cux23id_2516}{*\textsuperscript{27}}}

The contrast between the literary vision of the life of a man like
Boucicaut and the bitter reality of his career is almost
incomprehensible for us. As a participant and a leader, he was
constantly involved in the roughest politics of his time. In 1388 he
made his first political journey to the East. He passes the time during
that journey by engaging two or three of his comrades-in-arms, Philippe
d'Artois, his seneschal, and a certain Creseque, in a poetic defense of
the noble true \emph{Minne} that is proper for the perfect knight:
\emph{Le livre des cents
ballades}.\textsuperscript{\protect\hypertarget{10_Chapter_Three__THE_HEROIC_DREAM.xhtmlux5cux23id_1778}{\protect\hyperlink{23_NOTES.xhtmlux5cux23id_1779}{78}}}
Well, why not? But seven years later, when he served as mentor to the
young Count of Nevers (the later John the Fearless) in the ill-conceived
knightly adventure of the military campaign against Sultan Bayazid, when
he witnessed the terrible catastrophe of Nicopolis where his three
fellow poets lost their lives, when the noble youth of France, taken
prisoner, were butchered before his very eyes, would not one assume that
a serious warrior would have turned cool towards that courtly game and
that knightly fancy? It had to teach him, we are inclined to believe, to
no longer see the world through colored glasses. But no, his mind
remains dedicated to the cult of antique knighthood, as evidenced by his
founding of the Ordre de l'écu verd à la dame blanche for the protection
of oppressed women. This was his way of taking his position in the
artful literary quarrel between the strict and the frivolous ideals of
love that in the French court circles of 1400 was an exciting pastime.

The entire presentation of noble love in literature and social life
frequently strikes us as intolerably stale and ridiculous. That is the
fate of any romantic form that has lost its power as an instrument of
passion. In the works of many of the artful poets, passion has vanished
from the expensively arranged tournaments; it can only be heard in very
rare voices. But the importance of all this, given that it was inferior
as literature or art, as a beautification of life or
\protect\hypertarget{10_Chapter_Three__THE_HEROIC_DREAM.xhtmlux5cux23page_87}{}{}as
an expression of sentiment can only be fathomed if one can again fill
the literature itself with living passion. What use is there in reading
\emph{Minne} poetry and descriptions of tournaments for facts and
historical detail without seeing the gull-like arches of the brows, the
dark shining eyes and delicate foreheads, now dust for centuries, but
which once were more important than the whole of that literature which
remains piled up like rubble?

Only an occasional glimmer allows us to clearly realize exactly the
passionate importance of this cultural form. In the poem ``Le voeu du
Heron,'' Jean de Beaumont, urged to take his knightly vow of combat,
says:

\emph{Quant sommes ès tavernes, de ces fors vins buvant},

\emph{Et ces dames delès qui nous vont regardant},

\emph{A ces gorgues polies, ces coliés triant},

\emph{Chil oeil vair resplendissent de biauté souriant},

\emph{Nature nous semont d'avoir coeur désirant},

.~.~. \emph{Adonc conquerons-nous Yaumont et
Agoulant}\textsuperscript{\protect\hypertarget{10_Chapter_Three__THE_HEROIC_DREAM.xhtmlux5cux23id_1776}{\protect\hyperlink{23_NOTES.xhtmlux5cux23id_1777}{79}}}

\emph{Et li autre conquierrent Olivier et Rollant}.

\emph{Mais, quant sommes as camps sus nos destriers courants},

\emph{Nos escus à no col et nos lansses bais(s)ans},

\emph{Et le froidure grande nous va tout engelant},

\emph{Li membres nous effondrent, et derrière et devant},

\emph{Et nos ennemies sont envers nous approchant},

\emph{Adonc vorrièmes estre en un chélier si grant}

\emph{Que jamais ne fussions veu tant ne
quanti}.\textsuperscript{\protect\hypertarget{10_Chapter_Three__THE_HEROIC_DREAM.xhtmlux5cux23id_1774}{\protect\hyperlink{23_NOTES.xhtmlux5cux23id_1775}{80}}}\emph{\protect\hypertarget{10_Chapter_Three__THE_HEROIC_DREAM.xhtmlux5cux23id_2517}{\protect\hyperlink{23_NOTES.xhtmlux5cux23id_2518}{*\textsuperscript{28}}}}

``Helas,'' Philippe de Croy writes from the headquarters of Charles the
Bold near Neuss, ``où sont dames pour nous entretenir,
\protect\hypertarget{10_Chapter_Three__THE_HEROIC_DREAM.xhtmlux5cux23page_88}{}{}pour
nous amonester de bien faire, ne pour nous enchargier emprinses,
devises, volets ne
guimpes!''\textsuperscript{\protect\hypertarget{10_Chapter_Three__THE_HEROIC_DREAM.xhtmlux5cux23id_1772}{\protect\hyperlink{23_NOTES.xhtmlux5cux23id_1773}{81}}}\protect\hypertarget{10_Chapter_Three__THE_HEROIC_DREAM.xhtmlux5cux23id_2519}{\protect\hyperlink{23_NOTES.xhtmlux5cux23id_2520}{*\textsuperscript{29}}}

The erotic element of the knightly tournament is most directly revealed
in such customs as the wearing of the beloved's veil or other garment
that carries the fragrance of her hair or of her body. Caught up in the
excitement of combat, women offer one piece of jewelry after another;
when the game is over, they sit there bareheaded with their arms
stripped of their
sleeves.\textsuperscript{\protect\hypertarget{10_Chapter_Three__THE_HEROIC_DREAM.xhtmlux5cux23id_1770}{\protect\hyperlink{23_NOTES.xhtmlux5cux23id_1771}{82}}}
This becomes a symbol of keen attraction in the poem from the second
half of the thirteenth century about the three knights and the
shirt.\textsuperscript{\protect\hypertarget{10_Chapter_Three__THE_HEROIC_DREAM.xhtmlux5cux23id_1768}{\protect\hyperlink{23_NOTES.xhtmlux5cux23id_1769}{83}}}
A lady whose husband is not fond of fighting but is otherwise full of
noble gentility sends her chemise to the three knights, who serve her in
\emph{Minne}. They are to wear it, as battle dress, in the tournament
that her husband is about to hold, without any armor or other protection
than helmet and greaves. The first and second knight shy away from this.
The third, who is poor, holds the shirt in his arms throughout the night
and kisses it passionately. He appears in the tournament wearing the
shirt as his battle dress without any armor underneath it; the shirt
becomes torn and soiled with his blood; he is seriously wounded. His
extraordinary bravery is noticed and the prize is awarded him; the lady
gives her heart to him. Now her beloved asks a favor in return. He sends
the bloody shirt back to her so that she can wear it, bloody and torn,
over her dress during the feast that concludes the tournament. She
embraces it tenderly and attends the feast in her bloodied piece of
clothing; most of those in attendance criticize her, her husband is
embarrassed, and the narrator asks: which of the lovers has done more
for the other?

This passionate sphere in which alone the tournament had significance
explains why the church fought the custom for such a long time with such
determination. That tournaments actually became the cause of sensational
cases of adultery is testified to, for example, in 1389 by the monk of
Saint Denis and, based on his authority, Jean Juvenal des
Ursins.\textsuperscript{\protect\hypertarget{10_Chapter_Three__THE_HEROIC_DREAM.xhtmlux5cux23id_1766}{\protect\hyperlink{23_NOTES.xhtmlux5cux23id_1767}{84}}}
Canon law had long before prohibited tournaments; originally useful as
training for combat, it was said, they could no longer be tolerated
because of numerous
abuses.\textsuperscript{\protect\hypertarget{10_Chapter_Three__THE_HEROIC_DREAM.xhtmlux5cux23id_1764}{\protect\hyperlink{23_NOTES.xhtmlux5cux23id_1765}{85}}}
They drew criticism from the
moralists.\textsuperscript{\protect\hypertarget{10_Chapter_Three__THE_HEROIC_DREAM.xhtmlux5cux23id_1762}{\protect\hyperlink{23_NOTES.xhtmlux5cux23id_1763}{86}}}
Petrarch asked pedantically, where do we read that Cicero and Scipio
held
tourna\protect\hypertarget{10_Chapter_Three__THE_HEROIC_DREAM.xhtmlux5cux23page_89}{}{}ments?
And the Burgher of Paris shrugged his shoulders. ``Prindrent par ne scay
quelle folle entreprinse champ de
bataille,''\protect\hypertarget{10_Chapter_Three__THE_HEROIC_DREAM.xhtmlux5cux23id_2521}{\protect\hyperlink{23_NOTES.xhtmlux5cux23id_2522}{*\textsuperscript{30}}}
he says about a famous
tournament.\textsuperscript{\protect\hypertarget{10_Chapter_Three__THE_HEROIC_DREAM.xhtmlux5cux23id_1760}{\protect\hyperlink{23_NOTES.xhtmlux5cux23id_1761}{87}}}

The world of the nobility, on the other hand, gives everything related
to tournaments and knightly contests an importance that is not even
granted to modern sports. It was a very old custom to have a memorial
stone placed on the site where a famous duel had been fought. Adam of
Bremen knew of one such stone at the border between Holstein and Vargia
where a German warrior had once killed the leader of the
Vends.\textsuperscript{\protect\hypertarget{10_Chapter_Three__THE_HEROIC_DREAM.xhtmlux5cux23id_1758}{\protect\hyperlink{23_NOTES.xhtmlux5cux23id_1759}{88}}}
During the fifteenth century such memorials were still dedicated in
commemoration of famous knightly duels. Near Saint Orner La croix
Pélerine remembered the fight between Hautbourdin, the bastard of Saint
Pol, with a Spanish knight during the time of the famous Pas d'armes de
la Pélerine. Half a century later, Bayard takes time prior to a
tournament for a pilgrimage to that
cross.\textsuperscript{\protect\hypertarget{10_Chapter_Three__THE_HEROIC_DREAM.xhtmlux5cux23id_1756}{\protect\hyperlink{23_NOTES.xhtmlux5cux23id_1757}{89}}}
The decor and garments that had been used during the Pas d'armes de la
Fontaine des Pleurs were dedicated after the tournament to our Beloved
Lady of Boulogne and displayed in the
church.\textsuperscript{\protect\hypertarget{10_Chapter_Three__THE_HEROIC_DREAM.xhtmlux5cux23id_1754}{\protect\hyperlink{23_NOTES.xhtmlux5cux23id_1755}{90}}}

Medieval swordplay differs, as already indicated above, from Greek and
from modern athletics by its much reduced degree of naturalness. To
increase its warlike tone it relies on the excitement of aristocratic
pride and aristocratic honor, on its romantic-erotic and artistic
splendor. It is overladen with splendor and ornamentation, and
overfilled with colorful fantasy. In addition to being play and exercise
it is also applied literature. The desires and the dreams of poetic
hearts seek a dramatic representation, a staged fulfillment in life
itself. Real life was not beautiful enough; it was harsh, cruel, and
treacherous. There was little room in courtly and military careers for
feelings of courage that arose out of love, but the soul is filled with
such sentiments, and people want to experience them and to create a more
beautiful life in precious play. The element of genuine courage is most
certainly of no less value in a knightly tournament than in a pentathon
competition. Its explicitly erotic character was the cause of its bloody
intensity. In its motives the tournament is closest to the contests of
the Indian epics; in the \emph{Mahâbhârata}, too, fighting over a woman
is the central idea.

\protect\hypertarget{10_Chapter_Three__THE_HEROIC_DREAM.xhtmlux5cux23page_90}{}{}The
fantasy in which the tournament was dressed was that of the Arthur
novels, that is, the childish conceptions of the fairy tale: the dream
adventure with its shifting of dimensions into giants and dwarfs is
joined to the sentimentality of courtly love.

For \emph{a pas d'armes} of the fifteenth century a freely invented
romantic circumstance was artificially constructed. It was centered in a
novel-like setting given a fitting name: \emph{la fontaine des pleurs,
l'arbre
Charlemagne\protect\hypertarget{10_Chapter_Three__THE_HEROIC_DREAM.xhtmlux5cux23id_2523}{\protect\hyperlink{23_NOTES.xhtmlux5cux23id_2524}{*\textsuperscript{31}}}}
The fountain is especially
constructed.\textsuperscript{\protect\hypertarget{10_Chapter_Three__THE_HEROIC_DREAM.xhtmlux5cux23id_1752}{\protect\hyperlink{23_NOTES.xhtmlux5cux23id_1753}{91}}}
For an entire year an unknown knight on the first of each month will
pitch a tent in front of the fountain. Inside the tent a lady (only a
painting) sits and holds a unicorn that carries three shields. Any
knight touching one of the shields or having them touched by his herald
obligates himself to take part in a certain duel. The conditions of this
duel are precisely described in the detailed ``chapitres'' that are at
the same time invitations and rules for the
tournament.\textsuperscript{\protect\hypertarget{10_Chapter_Three__THE_HEROIC_DREAM.xhtmlux5cux23id_1750}{\protect\hyperlink{23_NOTES.xhtmlux5cux23id_1751}{92}}}
The shields have to be touched while on horseback and for this reason
horses are always available for the knights. In another example: at the
\emph{Emprise du dragon} four knights wait at a crossroads; no lady may
pass this crossroads without having one knight break two lances for her.
Otherwise she has to leave a
keepsake.\textsuperscript{\protect\hypertarget{10_Chapter_Three__THE_HEROIC_DREAM.xhtmlux5cux23id_1749}{\protect\hyperlink{23_NOTES.xhtmlux5cux23page_408}{93}}}
Actually, this childish game of forfeits is nothing but a lower form of
the usual age-old warrior and \emph{Minne} plays. This relationship is
clearly shown by a provision such as the following article from the
\emph{Chapitres de la Fontaine des pleurs}: Anyone thrown to the ground
during combat has to wear for a whole year a golden bracelet with a lock
attached until he finds the lady who has the small key fitting the lock
and can free him when he offers his services to her. In another conceit
the case is based on a giant who has been captured by a dwarf, complete
with a golden tree and a \emph{dame de l'isle
celée},\protect\hypertarget{10_Chapter_Three__THE_HEROIC_DREAM.xhtmlux5cux23id_2526}{\protect\hyperlink{23_NOTES.xhtmlux5cux23id_2525}{†\textsuperscript{32}}}
or on a ``noble chevalier esclave et serviteur à la belle géande à la
blonde perruque, la plus grande du
monde.''\textsuperscript{\protect\hypertarget{10_Chapter_Three__THE_HEROIC_DREAM.xhtmlux5cux23id_1747}{\protect\hyperlink{23_NOTES.xhtmlux5cux23id_1748}{94}}}\protect\hypertarget{10_Chapter_Three__THE_HEROIC_DREAM.xhtmlux5cux23id_2527}{\protect\hyperlink{23_NOTES.xhtmlux5cux23id_2528}{‡\textsuperscript{33}}}
The anonymity of the knight is a standard feature. He is called \emph{le
blanc chevalier, le chevalier mesconnu, le chevalier à la
pélerine},\protect\hypertarget{10_Chapter_Three__THE_HEROIC_DREAM.xhtmlux5cux23id_2529}{\protect\hyperlink{23_NOTES.xhtmlux5cux23id_2530}{§\textsuperscript{34}}}
or he may even appear as a hero
\protect\hypertarget{10_Chapter_Three__THE_HEROIC_DREAM.xhtmlux5cux23page_91}{}{}from
a novel and be called Swan Knight; or he may carry the arms of Lancelot,
Tristan, or
Palamedes.\textsuperscript{\protect\hypertarget{10_Chapter_Three__THE_HEROIC_DREAM.xhtmlux5cux23id_1745}{\protect\hyperlink{23_NOTES.xhtmlux5cux23id_1746}{95}}}

In most instances an extra touch of melancholy is spread over the scene:
this is already seen in the name \emph{Fontaine des pleurs}; the shields
are white, violet and black---all dotted with white tears; they are
touched out of compassion for the \emph{Dame de pleurs}. King René
appears at the \emph{Emprise du dragon} in the black of mourning---and
not without reason!---because he has just bid farewell to his daughter
Margaret, who has become Queen of England. The horse is black, draped
with a mourning saddlecloth; the lance is black; the shield black and
dotted with silver
tears.\textsuperscript{\protect\hypertarget{10_Chapter_Three__THE_HEROIC_DREAM.xhtmlux5cux23id_1743}{\protect\hyperlink{23_NOTES.xhtmlux5cux23id_1744}{96}}}
In \emph{l'arbre Charlemagne} the shield is black and violet with gold
and black tears. This somber key does not always prevail; in another
instance the insatiable lover of beauty King René holds the
\emph{Joyesse garde} near Saumur. For forty days he celebrates feasts in
the wooden castle ``de la joyesse garde'' with his wife and daughter and
with Jeanne de Laral, who was to become his second wife. The feast is
secretly prepared for her. The castle has been put up, painted, and hung
with tapestry specifically for that purpose. Everything is red and
white. For his \emph{pas d'armes de la bergère} everything is kept in
the style of shepherds, the knights and ladies as shepherds and
shepherdesses complete with staff and bagpipe. All in gray with touches
of gold and
silver.\textsuperscript{\protect\hypertarget{10_Chapter_Three__THE_HEROIC_DREAM.xhtmlux5cux23id_1741}{\protect\hyperlink{23_NOTES.xhtmlux5cux23id_1742}{97}}}

The great game of the beautiful life played as the dream of noble
courage and fidelity had another form than that of the tournament. The
second form, equally important, was that of the knightly orders. While
it may not be easy to show a direct link, no one even casually familiar
with the customs of primitive people will have any doubt that the roots
of knightly orders, just as those of the tournaments and the chivalric
initiations themselves, go back to the sacred customs of a distant past.
The ceremony conferring knighthood is an ethically and socially
elaborated puberty ritual, granting arms to the young warrior. The
staged combat itself is of ancient origin and was once full of sacred
meaning. The chivalric orders cannot be separated from the male bands of
primitive peoples.

But this link can only be suggested here as an unproven thesis; we are
not concerned at this moment with confirming an ethnological hypothesis,
but rather with envisioning the ideal value of fully
\protect\hypertarget{10_Chapter_Three__THE_HEROIC_DREAM.xhtmlux5cux23page_92}{}{}developed
knighthood. Who would deny that in all this some of the primitive still
survives?

To be sure, the Christian element in the idea is so strong that an
explanation founded on purely medieval ecclesiastical and political
conditions alone could also be convincing provided one did not realize
that universal primitive parallels furnish still more basic
explanations.

The first knightly orders, the three great orders of the Holy Land and
the three Spanish
orders,\textsuperscript{\protect\hypertarget{10_Chapter_Three__THE_HEROIC_DREAM.xhtmlux5cux23id_1739}{\protect\hyperlink{23_NOTES.xhtmlux5cux23id_1740}{98}}}
arose as the purest embodiment of the medieval spirit from a combination
of the monastic and knightly ideals at a time when the fight against
Islam had become a wondrous reality. They had grown into large political
and economic institutions, vast conglomerates of wealth and financial
power. Their political usefulness had pushed both their spiritual
character and the chivalric play element into the background while their
economic success, in turn, had eaten away their political usefulness. As
long as the Templars and Hospitalers flourished and were still active in
the Holy Land itself the knightly way of life had served a real
political function and the knightly orders really were practical
organizations serving functions of great significance.

In the fourteenth and fifteenth centuries, however, knightly practice
was only an elevated form of life and as a result the element of noble
play that was at its very heart had again come to the foreground in the
newer chivalric orders. Not that they had become only play. As idea, the
orders are still filled with ethical and political aspiration. But this
is now illusion and dream, vain scheming. The peculiar idealist Philippe
de Mézières saw the remedy for his age in a new knightly order that he
called the \emph{Ordre de la
\textsuperscript{\protect\hypertarget{10_Chapter_Three__THE_HEROIC_DREAM.xhtmlux5cux23id_1737}{\protect\hyperlink{23_NOTES.xhtmlux5cux23id_1738}{Passion.''99}}}}
He wanted all estates included in it. Incidentally, the great chivalric
orders of the Crusades had already made use of warriors without noble
status. The grand master and the knights should come from the ranks of
the nobility, the clergy should provide the patriarch and his
suffragans; burghers should become brothers; and rural people and
craftsmen servants. The order will thus be a solid amalgamation of the
estates for the great struggle against the Turks. There should be four
vows. Two are traditional, shared by the monks and the spiritual
knights: poverty and obedience. But in place of absolute celibacy
Philippe de Mézières put conjugal chastity. He wanted to permit marriage
for the practical reason that the oriental climate required it and that
it would make the order more
\protect\hypertarget{10_Chapter_Three__THE_HEROIC_DREAM.xhtmlux5cux23page_93}{}{}desirable.
The fourth vow, unknown to earlier orders, is the \emph{summa
perfectio}, the highest personal ethical perfection. Here is the
colorful picture of a knightly order in which all ideals come together
in actions ranging from the making of political plans all the way to the
struggle for salvation.

The word \emph{Ordre} mixed a number of meanings without distinguishing
among them, encompassing highest holiness as well as the most pragmatic
cooperatives. It could mean social status just as well as priestly
consecration, or refer to monastic or chivalric orders. That the word
``ordre'' in the sense of knightly order still retained some spiritual
significance is shown by the fact that the word ``religion'' was used in
its place, a usage that normally would perhaps be restricted only to the
cloistered orders. Chastellain calls the Golden Fleece \emph{une
religion} as if it were a cloistered order and speaks of it with the
kind of awe reserved for a holy
mystery.\textsuperscript{\protect\hypertarget{10_Chapter_Three__THE_HEROIC_DREAM.xhtmlux5cux23id_1735}{\protect\hyperlink{23_NOTES.xhtmlux5cux23id_1736}{100}}}
Olivier de la Marche speaks of a Portuguese as a ``chevalier de la
religion de
Avys.''\textsuperscript{\protect\hypertarget{10_Chapter_Three__THE_HEROIC_DREAM.xhtmlux5cux23id_1733}{\protect\hyperlink{23_NOTES.xhtmlux5cux23id_1734}{101}}}
But there is not only the reverential awe of that pompous Polonius
Chastellain to testify to the pious meaning of the Golden Fleece; church
attendance and the Mass occupy a dominant position within the entire
ritual of the order: the knights sit on the seats of the lords of the
cathedral, the memorial services for members who have passed away are
conducted in the strictest ecclesiastical style.

Small wonder therefore that membership in a knightly order was felt to
be a strong, sacred bond. The knights of the Order of Stars of King John
II are obligated, if possible, to abandon membership in all other
orders.\textsuperscript{\protect\hypertarget{10_Chapter_Three__THE_HEROIC_DREAM.xhtmlux5cux23id_1731}{\protect\hyperlink{23_NOTES.xhtmlux5cux23id_1732}{102}}}
The duke of Bedford, intending to tie young Philip of Burgundy closer to
England, wants to foist the Order of the Garter on him but the
Burgundian, fully realizing that this would bind him forever to the
English king, finds a way to politely evade the
honor.\textsuperscript{\protect\hypertarget{10_Chapter_Three__THE_HEROIC_DREAM.xhtmlux5cux23id_1729}{\protect\hyperlink{23_NOTES.xhtmlux5cux23id_1730}{103}}}
When Charles the Bold accepted the garter and even wore it, Louis XI
regarded this as a breach of the treaty of Péronne, which enjoined the
duke not to enter into an alliance with England without the king's
assent.\textsuperscript{\protect\hypertarget{10_Chapter_Three__THE_HEROIC_DREAM.xhtmlux5cux23id_1727}{\protect\hyperlink{23_NOTES.xhtmlux5cux23id_1728}{104}}}
The English custom of not accepting foreign orders may be regarded as a
traditional reminder of the notion that the honor obligates the
recipient to remain faithful to the prince who awards it.

That touch of sanctity notwithstanding, we may assume that among the
princely circles of the fourteenth and fifteenth centuries there was a
feeling that many regarded these artfully contrived new
\protect\hypertarget{10_Chapter_Three__THE_HEROIC_DREAM.xhtmlux5cux23page_94}{}{}knightly
orders as empty pastimes. Why else the endlessly repeated, insistent
assurances that all this was in aid of higher, most important purposes?
Philip of Burgundy, the noble duke, founded his Toison d'or, says the
poet Michault:

\emph{Non point pour jeu ne pour esbatement},

\emph{Mais à la fin que soit attribuée}

\emph{Loenge à Dieu trestout premièrement}

\emph{Et aux bons gloire et haulte
renommée}.\textsuperscript{\protect\hypertarget{10_Chapter_Three__THE_HEROIC_DREAM.xhtmlux5cux23id_1725}{\protect\hyperlink{23_NOTES.xhtmlux5cux23id_1726}{105}}}\emph{\protect\hypertarget{10_Chapter_Three__THE_HEROIC_DREAM.xhtmlux5cux23id_2531}{\protect\hyperlink{23_NOTES.xhtmlux5cux23id_2532}{*\textsuperscript{35}}}}

Guillaume Fillastre, too, promises in the preface of his work about the
Golden Fleece to explain its importance so that one would realize that
the order was not a matter of vanity or a matter of trifling importance.
Your father, he addresses Charles the Bold, ``n'a pas comme dit est, en
vain instituée ycelle
ordre.''\textsuperscript{\protect\hypertarget{10_Chapter_Three__THE_HEROIC_DREAM.xhtmlux5cux23id_1723}{\protect\hyperlink{23_NOTES.xhtmlux5cux23id_1724}{106}}}\protect\hypertarget{10_Chapter_Three__THE_HEROIC_DREAM.xhtmlux5cux23id_2533}{\protect\hyperlink{23_NOTES.xhtmlux5cux23id_2534}{†\textsuperscript{36}}}

It became necessary to emphasize the high intentions of the order if the
Golden Fleece were to take the first place Philip's pride craved for it.
Since the middle of the fourteenth century, the founding of chivalric
orders had become almost a fashion. Every prince simply had to have an
order of his own and even noble houses of high status did not want to be
left behind. There is Boucicaut with his Ordre de l'écu verd à la dame
blanche for the defense of courtly \emph{Minne} and oppressed women.
There is King John with his Chevaliers Nostre Dame de la Noble Maison
(1351), usually called the Order of the Stars after their insignia. In
the noble house at Saint Ouen near Saint Denis they had a \emph{table
d'oneur} at which the three bravest princes, the three bravest
bannerets,\textsuperscript{\protect\hypertarget{10_Chapter_Three__THE_HEROIC_DREAM.xhtmlux5cux23id_1721}{\protect\hyperlink{23_NOTES.xhtmlux5cux23id_1722}{107}}}
and the three bravest knights had to sit during their festivities. There
was further Pierre de Lusignan with his Order of the Sword, which
demanded of its members a pure life and put around their necks as witty
symbol a golden chain with its links formed in the shape of the letter
S, which signified ``silence.'' Amadeus of Savoy founded the Annouciade;
Louis de Bourbon the Golden Shield and the Thistle; Enguerrand de Coucy,
who had hoped for an imperial crown, the crown reversed; Louis of
Orléans the Order of the Porcupine. The Bavarian dukes of
Holland-Henegowen had their Order of Antonious,
\protect\hypertarget{10_Chapter_Three__THE_HEROIC_DREAM.xhtmlux5cux23page_95}{}{}complete
with the T-shaped cross and little bell that attract our attention in
numerous
portraits.\textsuperscript{\protect\hypertarget{10_Chapter_Three__THE_HEROIC_DREAM.xhtmlux5cux23id_1719}{\protect\hyperlink{23_NOTES.xhtmlux5cux23id_1720}{108}}}

The founding of such orders was frequently used to celebrate important
events, such as happened in the case of Louis Bourbon's return from his
term as an English prisoner of war, or, in other cases to make a
political point as, for example, with Orléans's \emph{porc-epic}, which
turned its quills towards Burgundy. Sometimes the pious character,
always significant, very strongly prevailed, as in the founding of an
order of St. George in the Franche-Comté when Philibert de Miolans
returned from the East with relics of that saint. At times the orders
are not much more than ordinary brotherhoods of mutual protection, such
as that of the Hazewind, founded by the nobles of the dukedom of Bar in
1416.

The reason for the success of the Golden Fleece, surpassing that of all
other newer orders, is the wealth of the Burgundians. The special
splendor of the order may have contributed just as much as the
fortuitous choice of the symbol itself. Initially the name of the Golden
Fleece evoked only the legend of Colchis. The legend of Jason was
generally known; Froissart had it told by a shepherd in a
pastorale.\textsuperscript{\protect\hypertarget{10_Chapter_Three__THE_HEROIC_DREAM.xhtmlux5cux23id_1717}{\protect\hyperlink{23_NOTES.xhtmlux5cux23id_1718}{109}}}
But Jason as a hero of legend was suspect; he had broken his vow of
fidelity and this theme was bound to trigger unwelcome insinuations
concerning the policy of the Burgundians towards France. Alain Charrier
put it this way in a poem:

\emph{A Dieu et aux gens detestables}

\emph{Est menterie et trahison}.

\emph{Pour ce n'est point mis à la table}

\emph{Des preux l'image de Jason},

\emph{Qui pour emporter la toison}

\emph{De Colcos se veult parjurer}.

\emph{Larrecin ne se peult
celer}.\textsuperscript{\protect\hypertarget{10_Chapter_Three__THE_HEROIC_DREAM.xhtmlux5cux23id_1715}{\protect\hyperlink{23_NOTES.xhtmlux5cux23id_1716}{110}}}\emph{\protect\hypertarget{10_Chapter_Three__THE_HEROIC_DREAM.xhtmlux5cux23id_2535}{\protect\hyperlink{23_NOTES.xhtmlux5cux23id_2536}{*\textsuperscript{37}}}}

Jean Germain, the learned bishop of Chalons and chancellor of the order,
brought to Philip's attention the fleece that Gideon had spread on the
ground and on which the heavenly dew fell. This was an especially good
idea because this Fleece of Gideon was one
\protect\hypertarget{10_Chapter_Three__THE_HEROIC_DREAM.xhtmlux5cux23page_96}{}{}of
the most fitting symbols of the fertilization of Mary's womb. The
biblical hero thus came to replace the heathen as patron of the Golden
Fleece. This enabled Jacques de Clercq to claim that Philip had
deliberately refrained from selecting Jason because he had broken his
vow of
fidelity.\textsuperscript{\protect\hypertarget{10_Chapter_Three__THE_HEROIC_DREAM.xhtmlux5cux23id_1713}{\protect\hyperlink{23_NOTES.xhtmlux5cux23id_1714}{111}}}
A court poet of Charles the Bold called the order ``Gedeonis
signa.''\textsuperscript{\protect\hypertarget{10_Chapter_Three__THE_HEROIC_DREAM.xhtmlux5cux23id_1711}{\protect\hyperlink{23_NOTES.xhtmlux5cux23id_1712}{112}}}
But others, such as the chronicler Theodericus Pauli, continue to speak
of the ``Vellus Jasonis.'' Jean Germains's successor as chancellor of
the order, Bishop Guillaume Fillastre, went further than his predecessor
and discovered four additional fleeces in the Holy Scripture: one of
Jacob, one of King Mesa of Moab, one of Job, and one of
David.\textsuperscript{\protect\hypertarget{10_Chapter_Three__THE_HEROIC_DREAM.xhtmlux5cux23id_1709}{\protect\hyperlink{23_NOTES.xhtmlux5cux23id_1710}{113}}}
He said that each of these represented a virtue and that he intended to
devote a book to each of the six. This was obviously too much of a good
thing. Fillastre had the spotted sheep of Jacob serve as symbol of
\emph{justifia};\textsuperscript{\protect\hypertarget{10_Chapter_Three__THE_HEROIC_DREAM.xhtmlux5cux23id_1707}{\protect\hyperlink{23_NOTES.xhtmlux5cux23id_1708}{114}}}
he had simply taken all instances where the Vulgate uses the word
``Vellus''---a rather peculiar demonstration of the flexibility of
allegory. There is no indication that his idea met with sustained
applause.

The pomp and festivities of the Golden Fleece have been described often
enough; to mention them here would only add further material to what has
been said above in
\protect\hyperlink{09_Chapter_Two__THE_CRAVING_FOR_A_M.xhtmlux5cux23page_30}{chapter
2} about the pomp of courtly life. One single feature of the order's
customs deserves to be cited here because it reveals so clearly the
character of a primitive and sacred play. The order counts among its
members next to its knights, its officers: the chancellor, the
treasurer, the secretary, and, further, the king of arms with his staff
of heralds and pursuivants. The latter group, specifically charged with
the service of the noble knightly game, are given symbolic names. The
king of arms himself has the name Toison d'or, as for example, Jean
Lefèvre de Saint Remy and Nicolas of Hames, the latter known from the
union of Dutch nobles in 1565. The heralds are given territorial names:
Charolais, Zélande. The First of the Pursuivants is named Fusil, after
the flint stone in the insignia chain of the order, the emblem of Philip
the Good. Others have names with romantic flavor, like Montreal, or of
virtues, like Persévérance; or names borrowed from the allegory of the
\emph{Roman de la rose}, for example, Humble Requeste, Doulce Pensée,
Léal Poursuite. During the great festival such pursuivants were solemnly
baptized with these names by the grand master, who sprinkled wine over
them.
\protect\hypertarget{10_Chapter_Three__THE_HEROIC_DREAM.xhtmlux5cux23page_97}{}{}He
also changed their names on the occasion of their elevation to higher
rank.\textsuperscript{\protect\hypertarget{10_Chapter_Three__THE_HEROIC_DREAM.xhtmlux5cux23id_1705}{\protect\hyperlink{23_NOTES.xhtmlux5cux23id_1706}{115}}}

The vows imposed by the chivalric orders are merely a firm collective
form of the personal knightly vows to perform some kind of heroic deed.
This is perhaps the point where the foundations of the knightly ideal
can best be viewed in their interlocking relationships. Those who might
be inclined to regard the connection between the act of being dubbed a
knight, the tournament, knightly orders, and primitive customs as a mere
suggestion will find that the barbaric character of the knightly vow
lurks so close to the surface that doubt is no longer possible. We are
dealing with genuine survivals, which have parallels in the ancient
Indian \emph{vratam}, in the \emph{Nasoräerschaft} of the Jews, and,
perhaps most directly, in the practices of the Vikings during their
legendary period.

The ethnological problem is not at issue here, but rather the question
of what significance the vows had in late medieval spiritual life. Three
values are possible. The knightly vows may have a religious-ethical
meaning that places them at the same level as clerical vows; their
content and meaning can also be of a romantic-erotic sort; and, finally,
the vows may have degenerated into a courtly game without any
significance other than that of a pastime. Actually, all these existed
together at the same time; the idea of the vow vacillates between the
highest dedication of life in the service of the most solemn ideal and
the most conceited mockery of the elaborate social game that found only
amusement in courage, love, and concerns of state. The play element
predominates; the vows became, for the most part, embellishments of
court festivities. But they always remained tied to the most serious
military undertakings: the invasion of France by Edward III, Philip the
Good's envisioned crusade.

It is as in the case of the tournaments: as tasteless and as worn as the
ready-made romanticism of the \emph{pas d'armes} may appear to us, so
too, the vow ``of the pheasant,'' ``of the peacock,'' and ``of the
egret'' seem to be equally vain and insincere, if we are not sensitive
to the passion that permeated all this. It is the dream of the more
beautiful life just as the festivities and forms of the Florentines of
Cosimo, Lorenzo, and Giuliana were this dream. In Italy it attained
eternal beauty, but here the dream's magic vanished with those who
dreamed it.

\protect\hypertarget{10_Chapter_Three__THE_HEROIC_DREAM.xhtmlux5cux23page_98}{}{}The
link between the ascetic and the erotic that is at the base of the
fantasy of the hero who frees the virgin or sheds his blood for her, the
central motif of tournament romanticism, reveals itself in another and
perhaps more striking aspect in the knightly vow. In his instructions
for his daughter, the knight De la Tour Landry tells us of a peculiar
order of noblemen and noblewomen given to the practice of \emph{Minne}
that had existed during his days of youth in Poitou and elsewhere. They
called themselves ``Galois et Galoises'' and observed ``une ordonnance
moult
sauvaige,''\protect\hypertarget{10_Chapter_Three__THE_HEROIC_DREAM.xhtmlux5cux23id_2537}{\protect\hyperlink{23_NOTES.xhtmlux5cux23id_2538}{*\textsuperscript{38}}}
the most important element of which was that they had to keep a fire
burning in the fireplace and dress themselves warmly in furs and padded
hoods during the summer while during the winter they were permitted to
wear nothing but a furless coat. They were not allowed any cloak or
other protection, hat, gloves or mittens, no matter how freezing the
temperature. During winter they scattered green leaves on the floor and
hid the chimney behind green branches, and on their bed they had only a
thin blanket. This wonderful aberration, so peculiar that the writer is
not likely to have invented it, can hardly be regarded as anything but
as an ascetic intensification of erotic attraction. Though not perfectly
clear in all details, and most likely strongly exaggerated, only minds
completely lacking in ethnological knowledge would take all this to be
the invention of a chatty old
man.\textsuperscript{\protect\hypertarget{10_Chapter_Three__THE_HEROIC_DREAM.xhtmlux5cux23id_1703}{\protect\hyperlink{23_NOTES.xhtmlux5cux23id_1704}{116}}}
The primitive character of the Galois and Galoises is further emphasized
by the rule of their order that the husband had to leave his entire
house and his wife to the Galois who was his guest in order to go to the
Galoise of his visitor; failure to do so meant total disgrace. According
to the knight De la Tour Landry, many members of the order had died of
cold: ``Si doubte moult que ces Galois et Galoises qui moururent en cest
etat et en cestes amouretes furent martirs
d'amours.''\textsuperscript{\protect\hypertarget{10_Chapter_Three__THE_HEROIC_DREAM.xhtmlux5cux23id_1701}{\protect\hyperlink{23_NOTES.xhtmlux5cux23id_1702}{117}}}\protect\hypertarget{10_Chapter_Three__THE_HEROIC_DREAM.xhtmlux5cux23id_2539}{\protect\hyperlink{23_NOTES.xhtmlux5cux23id_2540}{†\textsuperscript{39}}}

There are more examples that betray the primitive character of the
knightly vows. As, for example, the poem describing the vows that Robert
of Artois urged on the King of England, Edward III, and his noblemen in
order to start the war against France: ``Le voeu de héron.'' It is a
story of little historical value but the spirit of
\protect\hypertarget{10_Chapter_Three__THE_HEROIC_DREAM.xhtmlux5cux23page_99}{}{}barbarian
crudeness that it reveals is well suited to acquaint us with the nature
of the knightly vows.

The duke of Salisbury is sitting at the feet of his lady during a feast.
When his turn to take a vow has arrived, he asks his beloved to put a
finger on his right eye. Even two, she answers and closes the right eye
of the knight with two fingers. ``Belle, est-il bien clos?'' he asks.
``Oyl,
certainement.''\protect\hypertarget{10_Chapter_Three__THE_HEROIC_DREAM.xhtmlux5cux23id_2541}{\protect\hyperlink{23_NOTES.xhtmlux5cux23id_2542}{*\textsuperscript{40}}}
``Well, then,'' says Salisbury, ``then I vow to God the Almighty and his
sweet mother, not to open this eye again, no matter what pain and
suffering this may cause, until I have lit the flame in France, the
country of the enemy, and have fought the men of King Philip:''

\emph{Or aviegne qu'aviegne, car il n'est autrement}.

---\emph{Adonc osta son doit la puchelle au cors gent},

\emph{Et li iex clos demeure, si que virent la
gent}.\textsuperscript{\protect\hypertarget{10_Chapter_Three__THE_HEROIC_DREAM.xhtmlux5cux23id_1699}{\protect\hyperlink{23_NOTES.xhtmlux5cux23id_1700}{118}}}\protect\hypertarget{10_Chapter_Three__THE_HEROIC_DREAM.xhtmlux5cux23id_2543}{\protect\hyperlink{23_NOTES.xhtmlux5cux23id_2544}{†\textsuperscript{41}}}

In Froissart we can read of the reality reflected by this literary
motif; Froissart tells us that he actually saw English gentlemen who had
one eye covered with a piece of cloth so that they could fulfill their
vow of seeing with only one eye until they had performed heroic deeds in
France.\textsuperscript{\protect\hypertarget{10_Chapter_Three__THE_HEROIC_DREAM.xhtmlux5cux23id_1697}{\protect\hyperlink{23_NOTES.xhtmlux5cux23id_1698}{119}}}

This primitive crudeness of the ``voeu du héron'' is still more evident
in the vow of Jehan de Faukemont, who will not spare monastery or altar,
pregnant woman or child, friends or relatives, in order to serve King
Edward. At the end the queen, Philippa of Hennegowen, asks her husband
for permission to be also allowed to take a vow.

\emph{Adonc, dist la roine, je sai bien, que piecha}

\emph{Que sui grosse d'enfant, que mon corps senti l'a}.

\emph{Encore n'a il gaires, qu'en mon corps se tourna}.

\emph{Et je voue et prometh à Dieu qui me créa .~.~}.

\emph{Qui la li fruis de moi de mon corps n'istera},

\emph{Si m'en arés menée au païs par de-là}

\emph{Pour avanchier le veu que vo corps voué a};

\emph{\protect\hypertarget{10_Chapter_Three__THE_HEROIC_DREAM.xhtmlux5cux23page_100}{}{}Et
s'il en voelh isir, quant besoins n'en sera},

\emph{D'un grant coutel d'achier li miens corps s'ochira};

\emph{Serai m'asme perdue et li fruis perira!
\protect\hypertarget{10_Chapter_Three__THE_HEROIC_DREAM.xhtmlux5cux23id_2871}{\protect\hyperlink{23_NOTES.xhtmlux5cux23id_2872}{*\textsuperscript{42}}}}

This blasphemous vow is met with a chilled silence. The poet only says:

\emph{Et quant li rois l'entent, moult forment l'en pensa},

\emph{Et dist: certainement, nul plues ne
vouera}.\protect\hypertarget{10_Chapter_Three__THE_HEROIC_DREAM.xhtmlux5cux23id_2873}{\protect\hyperlink{23_NOTES.xhtmlux5cux23id_2874}{†\textsuperscript{43}}}

Hair and beard, everywhere bearers of magical power have a special
meaning in medieval vows. Benedict XIII, pope of Avignon but actually a
prisoner there, swore not to have his beard cut as a sign of his travail
until his freedom was
restored.\textsuperscript{\protect\hypertarget{10_Chapter_Three__THE_HEROIC_DREAM.xhtmlux5cux23id_1695}{\protect\hyperlink{23_NOTES.xhtmlux5cux23id_1696}{120}}}
When Lumey takes the same vow with respect to taking revenge for the
count of Egmont, we encounter one of the last remnants of a custom that
had sacred meaning in the distant past.

The meaning of a vow, as a rule, is that someone imposes on himself an
austerity as a stimulant to the completion of the vow. In most cases the
austerity is linked to food. The first to be taken as knight into his
Chevalerie de la Passion by Philippe de Mézières was a Pole who had for
nine years not eaten or drunk while sitting
down.\textsuperscript{\protect\hypertarget{10_Chapter_Three__THE_HEROIC_DREAM.xhtmlux5cux23id_1693}{\protect\hyperlink{23_NOTES.xhtmlux5cux23id_1694}{121}}}
Bertrand du Guesclin is very hasty with respect to such vows. Once there
was a challenge from an English warrior: Bertrand declared that he would
only have three wine soups in the name of the Trinity until he had
fought the challenger. In another instance he had pledged not to eat
meat or take off his clothes until he had taken Montcontour, or even
that he would not eat until he had clashed with the
English.\textsuperscript{\protect\hypertarget{10_Chapter_Three__THE_HEROIC_DREAM.xhtmlux5cux23id_1691}{\protect\hyperlink{23_NOTES.xhtmlux5cux23id_1692}{122}}}

Naturally, the nobleman of the fourteenth century was no longer
conscious of the magical significance of this fasting. To us, the
\protect\hypertarget{10_Chapter_Three__THE_HEROIC_DREAM.xhtmlux5cux23page_101}{}{}underlying
motif is very evident from the manifold use of bonds as emblems of a
vow. On January 1, 1415, Duke Jean de Bourbon, ``désirant eschiver
oisiveté, pensant y acquerir bonne renommée et la grâce de la très-belle
de qui nous sommes serviteurs,''
\protect\hypertarget{10_Chapter_Three__THE_HEROIC_DREAM.xhtmlux5cux23id_2875}{\protect\hyperlink{23_NOTES.xhtmlux5cux23id_2876}{*\textsuperscript{44}}}
takes the vow, together with sixteen other knights and page boys, to
wear every Sunday for two years a bond like that of a prisoner on his
left leg---the knights' in gold, the pageboys' in silver---until he had
found sixteen knights ready to fight the band in a battle on foot ``à
outrance.''\textsuperscript{\protect\hypertarget{10_Chapter_Three__THE_HEROIC_DREAM.xhtmlux5cux23id_1689}{\protect\hyperlink{23_NOTES.xhtmlux5cux23id_1690}{123}}}\protect\hypertarget{10_Chapter_Three__THE_HEROIC_DREAM.xhtmlux5cux23id_2877}{\protect\hyperlink{23_NOTES.xhtmlux5cux23id_2878}{†\textsuperscript{45}}}
Jacques de Lalaing in 1445 meets a Sicilian knight in Antwerp, Jean de
Boniface, who as ``chevalier aventureux'' has come from the court of
Aragon. On his left leg he has an iron, just like slaves used to wear,
and, hanging on a golden bracelet, an ``emprise'' that signifies his
readiness to
fight.\textsuperscript{\protect\hypertarget{10_Chapter_Three__THE_HEROIC_DREAM.xhtmlux5cux23id_1687}{\protect\hyperlink{23_NOTES.xhtmlux5cux23id_1688}{124}}}
In the novel of the Petit Jehan de Saintré the knight Loiselench wears
two golden rings on arm and leg, each on a golden chain, until he finds
a knight who ``liberates'' him from his
enterprise.\textsuperscript{\protect\hypertarget{10_Chapter_Three__THE_HEROIC_DREAM.xhtmlux5cux23id_1685}{\protect\hyperlink{23_NOTES.xhtmlux5cux23id_1686}{125}}}
This is what is called ``délivrer''; thus the sign is touched when one
goes ``pour chevalier''; it is torn off if mortal combat is intended. La
Curne de Sainte Palaye noticed that, according to Tacitus, the very same
custom was found among the ancient
Chatten.\textsuperscript{\protect\hypertarget{10_Chapter_Three__THE_HEROIC_DREAM.xhtmlux5cux23id_1683}{\protect\hyperlink{23_NOTES.xhtmlux5cux23id_1684}{126}}}
The bonds worn by the penitent on their pilgrimages or those that pious
ascetics put on themselves are related to these ``enterprises'' of the
late medieval knights.

The most famous solemn vow of the fifteenth century, the \emph{Voeux du
Faisan}, was taken in 1454 in Lille during a court festival given by
Philip the Good in preparation for the crusade. What it still reveals of
all this is not much more than a beautiful courtly form. Not that the
custom of taking a spontaneous vow during an emergency or moment of
strong emotion had lost any of its power. This custom has such deep
psychological roots that it is bound neither to education nor faith. The
knightly vow as cultural form, however, as a custom elevated to an
embellishment of life, reaches its last phase in the splendid
extravagances of the Burgundian court.

The theme of the action is still always the unmistakable old theme. Vows
are taken during feasts, an oath is made in the name of a bird that is
served and later eaten. The Vikings, too, knew the
\protect\hypertarget{10_Chapter_Three__THE_HEROIC_DREAM.xhtmlux5cux23page_102}{}{}competition
in vows taken during drunken feasts; one of the forms is to touch the
wild boar as it is being
served.\textsuperscript{\protect\hypertarget{10_Chapter_Three__THE_HEROIC_DREAM.xhtmlux5cux23id_1681}{\protect\hyperlink{23_NOTES.xhtmlux5cux23id_1682}{127}}}
The pheasant of the famous feast at Lille seems to have been
alive.\textsuperscript{\protect\hypertarget{10_Chapter_Three__THE_HEROIC_DREAM.xhtmlux5cux23id_1679}{\protect\hyperlink{23_NOTES.xhtmlux5cux23id_1680}{128}}}
The vow was taken in the name of God and his Mother, of ladies and the
bird.\textsuperscript{\protect\hypertarget{10_Chapter_Three__THE_HEROIC_DREAM.xhtmlux5cux23id_1677}{\protect\hyperlink{23_NOTES.xhtmlux5cux23id_1678}{129}}}
It is not too daring to assume that the Deity in this instance was not
the original recipient of the vow: actually many vows are taken only in
the name of the ladies or of birds. There is little variety in the
austerities the oath takers imposed upon themselves. Most are related to
sleep or food. This knight is not allowed to sleep in a bed on Sundays
until he has fought a Saracen, nor may he stay for fourteen consecutive
days in the same city. Another may not eat meat on Friday until he has
touched the banner of the great Turk; yet another piles ascetic practice
on top of ascetic practice: he is not allowed to wear any armor at all,
drink wine on Sundays, sleep in a bed, sit at a table, and he has to
wear a hair shirt. The manner in which the heroic deed required by the
vow is to be carried out is described in precise
detail.\textsuperscript{\protect\hypertarget{10_Chapter_Three__THE_HEROIC_DREAM.xhtmlux5cux23id_1675}{\protect\hyperlink{23_NOTES.xhtmlux5cux23id_1676}{130}}}

How serious is this all? When messire Philippe Pot takes the vow to keep
his right arm bare of any armor during the campaign against the Turks,
the duke has the following comment added below the vow (which was
registered in writing): ``Ce n'est pas le plaisir de mon très redoubté
seigneur, que messire Phelippe Pot voise en sa compaignie ou saint
voyage qu'il a voué, le bras désarmé; mais il est content qu'il voist
aveuc lui armé bien et soufisamment ainsi qu'il
appartient.''\textsuperscript{\protect\hypertarget{10_Chapter_Three__THE_HEROIC_DREAM.xhtmlux5cux23id_1673}{\protect\hyperlink{23_NOTES.xhtmlux5cux23id_1674}{131}}}\protect\hypertarget{10_Chapter_Three__THE_HEROIC_DREAM.xhtmlux5cux23id_2879}{\protect\hyperlink{23_NOTES.xhtmlux5cux23id_2880}{*\textsuperscript{46}}}
Obviously a vow was still regarded as serious and dangerous. The vow by
the duke himself stirs emotions
everywhere.\textsuperscript{\protect\hypertarget{10_Chapter_Three__THE_HEROIC_DREAM.xhtmlux5cux23id_1671}{\protect\hyperlink{23_NOTES.xhtmlux5cux23id_1672}{132}}}

Others take cautiously conditioned vows that testify both to serious
intent and to self-satisfaction with a beautiful
pretense.\textsuperscript{\protect\hypertarget{10_Chapter_Three__THE_HEROIC_DREAM.xhtmlux5cux23id_1669}{\protect\hyperlink{23_NOTES.xhtmlux5cux23id_1670}{133}}}
On some occasions the vows are addressed to the ``much beloved'' who is
but a pale remnant of
herself.\textsuperscript{\protect\hypertarget{10_Chapter_Three__THE_HEROIC_DREAM.xhtmlux5cux23id_1667}{\protect\hyperlink{23_NOTES.xhtmlux5cux23id_1668}{134}}}
A mocking element is not lacking even in the grim \emph{Voeu du héron}:
Robert of Artois offers the king, pictured here as not very belligerent,
the heron as the most timid of birds. After Edward has taken his vow,
all break out in laughter. Jean de Beaumont took the \emph{Voeu du
héron} in the words
\protect\hypertarget{10_Chapter_Three__THE_HEROIC_DREAM.xhtmlux5cux23page_103}{}{}already
mentioned
earlier,\textsuperscript{\protect\hypertarget{10_Chapter_Three__THE_HEROIC_DREAM.xhtmlux5cux23id_1665}{\protect\hyperlink{23_NOTES.xhtmlux5cux23id_1666}{135}}}
which reveal with faint mockery the passionate nature of vows made under
the influence of wine and under the eyes of the ladies. According to
another story, he loudly took a cynical vow, in the name of the heron,
that he would serve that lord from whom he could expect to get the most.
Whereupon the English lords
laughed.\textsuperscript{\protect\hypertarget{10_Chapter_Three__THE_HEROIC_DREAM.xhtmlux5cux23id_1663}{\protect\hyperlink{23_NOTES.xhtmlux5cux23id_1664}{136}}}
What mood, in spite of all the solemn importance with which the
\emph{Voeux du Faisan} were received, must have prevailed at the table
when Jennet de Rebreviette took the vow that he, in case he did not
receive the favor of his lady before the campaign started, would upon
his return from the East marry the first woman or maiden who had 20,000
crowns---``se elle
veult.''\textsuperscript{\protect\hypertarget{10_Chapter_Three__THE_HEROIC_DREAM.xhtmlux5cux23id_1661}{\protect\hyperlink{23_NOTES.xhtmlux5cux23id_1662}{137}}}\protect\hypertarget{10_Chapter_Three__THE_HEROIC_DREAM.xhtmlux5cux23id_2881}{\protect\hyperlink{23_NOTES.xhtmlux5cux23id_2882}{*\textsuperscript{47}}}
Yet the same Rebreviette as ``pouvre escuier''
\protect\hypertarget{10_Chapter_Three__THE_HEROIC_DREAM.xhtmlux5cux23id_2883}{\protect\hyperlink{23_NOTES.xhtmlux5cux23id_2884}{†\textsuperscript{48}}}
ventures forth and fights against the Moors at Ceuta and Granada.

So the exhausted aristocracy laughs at its own ideal. Having dressed and
painted their passionate dream of a beautiful life with all their powers
of imagination and artfulness and wealth and molded it into a plastic
form, they then pondered and realized that life was really not so
beautiful---and then laughed.

It was only a vain illusion, that knightly glory, only style and
ceremony, a beautiful and insincere play! The real history of the late
medieval period, we are told by the researcher who traces the
development of the state and of economics in the documents, has little
to do with the phony knightly renaissance; it was old varnish that had
begun to peel off. The men who made history were by no means dreamers
but were very calculating, sober politicians or merchants, be they
princes, noblemen, prelates or burghers.

This they certainly were. But the history of culture has just as much to
do with dreams of beauty and the illusions of a noble life as with
population figures and statistics. A more recent scholar, having studied
today's society in terms of the growth of banks and traffic, of
political and military conflicts, would be able to state at the end of
his studies: ``I have noticed very little about music, which obviously
had little meaning for this culture.''

It seems to be that way if the history of the Middle Ages is described
for us from political and economic documents. But it may well be that
the knightly ideal, artificial and worn-out as it
\protect\hypertarget{10_Chapter_Three__THE_HEROIC_DREAM.xhtmlux5cux23page_104}{}{}may
have been, still continued to exert a more powerful influence on the
purely political history of the late Middle Ages than is usually
imagined.

The charm of the noble life form is so great that even burghers succumb
to it wherever they can. We imagine the Flemish heroes Jacob and Philipp
van Artevelde to be true men of the third estate---proud of their
bourgeois stature and simplicity. On the contrary: Philipp van Artevelde
lived in princely splendor, every day he had musicians perform in front
of his lodging, every meal he had served on silver dishes as if he were
the count of Flanders. He dressed in purple, red, and ``menu vair'' like
a duke of Brabant or count of Hennegowen. He rode on horseback in the
style of a prince, an unfurled banner carried ahead of him to display
his coat of arms, a sable with three silver
hats.\textsuperscript{\protect\hypertarget{10_Chapter_Three__THE_HEROIC_DREAM.xhtmlux5cux23id_1659}{\protect\hyperlink{23_NOTES.xhtmlux5cux23id_1660}{138}}}
Who appears to be more modern to us than the leading financier of the
fifteenth century, Jacques Coeur, the outstanding banker of Charles VII?
If we are to believe his biographer, Jacques de Lalaing, this great
banker took a lively interest in the old-fashioned knight-errantry of
the Hennegowen hero Philipp van
Artevelde.\textsuperscript{\protect\hypertarget{10_Chapter_Three__THE_HEROIC_DREAM.xhtmlux5cux23id_1657}{\protect\hyperlink{23_NOTES.xhtmlux5cux23id_1658}{139}}}

All higher forms of the bourgeois life of modern times are based on
imitations of noble life forms. Just as the bread served on a
``serviette'' (napkin) and the word ``serviette'' itself have their
origin in medieval courtly
stateliness,\textsuperscript{\protect\hypertarget{10_Chapter_Three__THE_HEROIC_DREAM.xhtmlux5cux23id_1655}{\protect\hyperlink{23_NOTES.xhtmlux5cux23id_1656}{140}}}
the most bourgeois of the prenuptial pranks are offsprings of the
grandiose ``entremets'' of Lille. In order to fully understand the
meaning of the knightly ideal in cultural-historical terms one would
have to trace it to Shakespeare's and Moliere's time, or even to the
modern gentleman. But in this instance we are concerned with exploring
the effect of that ideal on real life during the waning Middle
Ages\textsuperscript{\protect\hypertarget{10_Chapter_Three__THE_HEROIC_DREAM.xhtmlux5cux23id_1653}{\protect\hyperlink{23_NOTES.xhtmlux5cux23id_1654}{141}}}
themselves. Could politics and warfare actually be controlled by the
knightly idea? Undoubtedly yes, if not by its merits then by its
weaknesses. Just as the tragic blunders of today arise from the frenzy
of nationalism and cultural arrogance, those of the medieval period
arose more than once from chevaleresque notions. Did not the motive for
the creation of the new Burgundian state, the gravest mistake France
could have committed, rise from a knightly impulse? King John, that
knightly maniac, hands the dukedom in 1363 to his young son who had
stood with him at Poitiers when the elder son fled. The same holds true
for the conscious notion that was intended to justify the later
anti-French policy of the Burgundians to their
contem\protect\hypertarget{10_Chapter_Three__THE_HEROIC_DREAM.xhtmlux5cux23page_105}{}{}poraries:
vengeance for Montereau, the defense of knightly honor. I am well aware
that all this could also be explained as the results of calculating or
even farsighted politics, but this does not keep the contemporaries from
regarding the value and lesson of the facts of 1363 as a case of
knightly courage that had received princely rewards. The Burgundian
state in its rapid unfolding is an edifice of political insight and
purposefully sober calculation. But what one may call the Burgundian
idea always takes on the forms of the knightly ideal. The nicknames of
the dukes---\emph{Sans Peur, Le Hardi, Qui qu'en Hongue}, which was
replaced in the case of Philip with \emph{Le Bon}---are all deliberate
inventions of court literature so that the prince can be seen in the
light of the knightly
ideal.\textsuperscript{\protect\hypertarget{10_Chapter_Three__THE_HEROIC_DREAM.xhtmlux5cux23id_1651}{\protect\hyperlink{23_NOTES.xhtmlux5cux23id_1652}{142}}}

One great political quest was inseparably tied to the knightly ideal:
the crusade, Jerusalem! The thought of Jerusalem was constantly before
the eyes of all the princes of Europe as the most noble political idea
and continued to spur them into action. There was here a peculiar
contradiction between practical political interest and political idea.
Christendom of the fourteenth and fifteenth centuries faced an Oriental
question of the highest urgency: defense against the Turks, who had
already taken Adrianopolis (1378) and destroyed the Serbian Empire
(1389). Danger loomed in the Balkans. But Europe's first and most
imperative political task could not yet be separated from the idea of a
crusade. The Turkish question could only be viewed as part of the great
holy task that earlier times had failed to accomplish: the liberation of
Jerusalem.

This conception put the knightly ideal in the foreground. In this
context it could and was bound to have a particularly powerful effect.
The religious content of the knightly ideal found its highest expression
in this quest, and the liberation of Jerusalem could be nothing but holy
and noble knightly work. The limited success in combating the Turks may
be explained to a certain degree by the very fact that the
religious-knightly ideal was so prominent in shaping the political
response to the Orient. The expeditions that required, above all, exact
calculation and patient preparation were conceived and implemented under
a very high tension that led not to a calm consideration of that which
was attainable, but to a romanticizing of the plan: a tension that was
bound either to remain fruitless or to become fatal. The catastrophe of
Nicopolis in 1396 proved how dangerous it was to mount a serious
expedition against a very militant enemy in the old-fashioned style of
those knightly
\protect\hypertarget{10_Chapter_Three__THE_HEROIC_DREAM.xhtmlux5cux23page_106}{}{}jaunts
into Prussia or Lithuania where the objective was merely to put to death
a few poor heathens. Who was it who designed the plans for the Crusades?
Dreamers like Philippe de Mézières, who dedicated his life to them and
political fantasizers, one of whom was Philip the Good, all his clever
calculations notwithstanding.

The liberation of Jerusalem remained a compelling and vital task for all
kings. In 1422, Henry V of England lay dying. The young conqueror of
Rouen and Paris was taken away right in the middle of the work with
which he had caused France so much misery. The physicians told him that
he had less than two hours to live; the confessor and other clerics have
come, the seven penitential psalms are read. As the clergymen recite the
words ``Bénigne fac, Domine, in bona voluntate tue Sion, ut aedificentur
muri
Jerusalem,''\textsuperscript{\protect\hypertarget{10_Chapter_Three__THE_HEROIC_DREAM.xhtmlux5cux23id_1649}{\protect\hyperlink{23_NOTES.xhtmlux5cux23id_1650}{143}}}\protect\hypertarget{10_Chapter_Three__THE_HEROIC_DREAM.xhtmlux5cux23id_2885}{\protect\hyperlink{23_NOTES.xhtmlux5cux23id_2886}{*\textsuperscript{49}}}
the king makes them stop and, with a loud voice, says that it had been
his intention to conquer Jerusalem once peace had been restored in
France, ``se ce eust été le plaisir de Dieu son createur de la laisser
vivre son
aage.''\protect\hypertarget{10_Chapter_Three__THE_HEROIC_DREAM.xhtmlux5cux23id_2887}{\protect\hyperlink{23_NOTES.xhtmlux5cux23id_2888}{†\textsuperscript{50}}}
Then he lets the reading of the penitential psalms be concluded and dies
shortly
thereafter.\textsuperscript{\protect\hypertarget{10_Chapter_Three__THE_HEROIC_DREAM.xhtmlux5cux23id_1647}{\protect\hyperlink{23_NOTES.xhtmlux5cux23id_1648}{144}}}

The Crusades had also for a long time been a pretext for imposing
special levies: even Philip the Good had generously availed himself of
that opportunity. Yet this could hardly be said to have been only a
hypocritical use of the planned crusade for the sake of financial
gain.\textsuperscript{\protect\hypertarget{10_Chapter_Three__THE_HEROIC_DREAM.xhtmlux5cux23id_1645}{\protect\hyperlink{23_NOTES.xhtmlux5cux23id_1646}{145}}}
It seems to have been a mixture of serious concerns and of the intent to
secure for himself higher fame than that of the Kings of France and
England, whose rank was superior to his own, by pursuing this
particularly useful and, at the same time, especially knightly plan, to
be the savior of Christendom. ``Le voyage de Turquie''
\protect\hypertarget{10_Chapter_Three__THE_HEROIC_DREAM.xhtmlux5cux23id_2889}{\protect\hyperlink{23_NOTES.xhtmlux5cux23id_2890}{‡\textsuperscript{51}}}
remained his trump card that he did not play. Chastellain takes pains to
stress that the duke was serious about this but that there were
important considerations: the times were not ripe, influential people
were shaking their heads that the prince intended to undertake such a
dangerous campaign given his age; territories and dynasty would both be
in peril. While the pope sent him the flag of the cross that was
received by Philip in the Hague with all humility and respect in a
solemn procession, while vows to take
\protect\hypertarget{10_Chapter_Three__THE_HEROIC_DREAM.xhtmlux5cux23page_107}{}{}the
journey were made during the festivities in Lille and afterwards, while
Joffrey de Toisy reconnoitered the Syrian ports and Jean Chevrot, bishop
of Tournay, supervised collections, and Guillaume Fillastre held his
entire train in readiness and had already confiscated ships for the
campaign, there prevailed, in the midst of all this, a vague premonition
that the campaign might not take place in spite of
everything.\textsuperscript{\protect\hypertarget{10_Chapter_Three__THE_HEROIC_DREAM.xhtmlux5cux23id_1643}{\protect\hyperlink{23_NOTES.xhtmlux5cux23id_1644}{146}}}
The duke's own vow had a somewhat qualified ring to it; he would venture
out if the territories, which God had entrusted to be governed by him,
would enjoy peace and
security.\textsuperscript{\protect\hypertarget{10_Chapter_Three__THE_HEROIC_DREAM.xhtmlux5cux23id_1641}{\protect\hyperlink{23_NOTES.xhtmlux5cux23id_1642}{147}}}

Announcing military campaigns, excepting the ideal of the crusade,
seemed to have been a popular technique in the clamor for political
prestige. These noisily proclaimed campaigns were prepared in great
detail, but failed to materialize or had very little consequence, as,
for example, the English expedition against Flanders in 1383; or the
campaign of Philip the Bold against England in 1387, in which a splendid
fleet was assembled and made ready to sail from the port of Sluis; or
the campaign of Charles VI against Italy in 1391.

A very special form of knightly fiction used as political propaganda was
the repeatedly announced but never accomplished princely duel. I have
elsewhere detailed how the quarrels between the states of the fifteenth
century were still regarded as quarrels between parties, as personal
``querelles.''\textsuperscript{\protect\hypertarget{10_Chapter_Three__THE_HEROIC_DREAM.xhtmlux5cux23id_1639}{\protect\hyperlink{23_NOTES.xhtmlux5cux23id_1640}{148}}}
The cause one served was called ``la querelle des Bourguignons.'' What
was more natural than that the princes themselves should fight it out
just as still proposed in casual political rhetoric? This solution that
arose from both a primitive sense of justice and from the knightly
imagination actually appeared time and again on the agenda. Reading
about the detailed preparations for the princely duels, one wonders if
this was only a beautiful game of conscious hypocrisy, again the search
for a beautiful life, or whether the knightly adversaries really
expected to do battle against each other. There is no doubt that the
historians of that period took such challenges just as seriously as the
belligerent princes themselves. In 1383 Richard II commissioned his
uncle, John of Lancaster, to negotiate peace with the King of France
and, as a proper means thereto, a duel between the two kings or between
Richard and his three uncles and Charles and his
uncles.\textsuperscript{\protect\hypertarget{10_Chapter_Three__THE_HEROIC_DREAM.xhtmlux5cux23id_1637}{\protect\hyperlink{23_NOTES.xhtmlux5cux23id_1638}{149}}}
Monstrelet devotes considerable space, right at the beginning of his
chronicle, to the challenge by Louis of Orléans to King Henry
\protect\hypertarget{10_Chapter_Three__THE_HEROIC_DREAM.xhtmlux5cux23page_108}{}{}IV
of
England.\textsuperscript{\protect\hypertarget{10_Chapter_Three__THE_HEROIC_DREAM.xhtmlux5cux23id_1635}{\protect\hyperlink{23_NOTES.xhtmlux5cux23id_1636}{150}}}
To the impetuous and brilliant mind of Orléans, which had scope for
fiery devotion, the appreciation of the arts, fantastic ideals of
knightly combat and courtly love, side by side with debauchery,
cynicism, and the magical arts, such a duel might also have well been a
passionate undertaking. The same holds true for the pompous mind of
Philip the Good. He, in his turn, provided the most imposing elaboration
of the theme backed by all the resources of his wealth and his love of
splendor. It was Humphrey of Gloucester whom he challenged in the noble
manner in 1425. In the challenge there is clear reference to the motif
of \emph{noblesse oblige}: ``pour éviter effusion de sang chrestien et
la destruction du peuple, dont en mon cuer ay compacion .~.~. que par
mon corps sans plus ceste querelle soit menée à fin, sans y aler avant
par voies de guerres dont il convendroit mains gentilz hommes et
aultres, tant de vostre ost comme du mien, finer leurs jours
piteusement.''\textsuperscript{\protect\hypertarget{10_Chapter_Three__THE_HEROIC_DREAM.xhtmlux5cux23id_1633}{\protect\hyperlink{23_NOTES.xhtmlux5cux23id_1634}{151}}}\protect\hypertarget{10_Chapter_Three__THE_HEROIC_DREAM.xhtmlux5cux23id_2891}{\protect\hyperlink{23_NOTES.xhtmlux5cux23id_2892}{*\textsuperscript{52}}}
All the props for the battle were ready: the costly armor and the
splendid garments to be worn by the duke were prepared; work was in
progress on the tents, the standards and banners, the coats for the
heralds and pursuivants, all displaying in profusion the court of arms
of the ducal realm, the tinderbox and the cross of St. Andrew. Philip
was in training: ``tant en abstinence de sa bouche comme en prenant
painne pour luy mettre en
alainne.''\textsuperscript{\protect\hypertarget{10_Chapter_Three__THE_HEROIC_DREAM.xhtmlux5cux23id_1631}{\protect\hyperlink{23_NOTES.xhtmlux5cux23id_1632}{152}}}
In his park at Hesdin he practiced daily under experienced fencing
masters.\textsuperscript{\protect\hypertarget{10_Chapter_Three__THE_HEROIC_DREAM.xhtmlux5cux23id_1629}{\protect\hyperlink{23_NOTES.xhtmlux5cux23id_1630}{153}}}
The bills inform us of the cost of all of this. The expensive tent
fashioned for the purpose could be seen in Lille as late as
1460.\textsuperscript{\protect\hypertarget{10_Chapter_Three__THE_HEROIC_DREAM.xhtmlux5cux23id_1627}{\protect\hyperlink{23_NOTES.xhtmlux5cux23id_1628}{154}}}
But the duel never took place.

This did not stop Philip from later issuing a new challenge to the duke
of Saxony during their quarrel over Luxembourg. At the feast at Lille
when Philip was almost sixty years old, his vow to launch a crusade
stated that he would be only too willing to do battle with the great
Turk \emph{corps à
corps}\protect\hypertarget{10_Chapter_Three__THE_HEROIC_DREAM.xhtmlux5cux23id_2893}{\protect\hyperlink{23_NOTES.xhtmlux5cux23id_2894}{†\textsuperscript{53}}}
if the latter wanted it that
way.\textsuperscript{\protect\hypertarget{10_Chapter_Three__THE_HEROIC_DREAM.xhtmlux5cux23id_1626}{\protect\hyperlink{23_NOTES.xhtmlux5cux23page_410}{155}}}
The stubborn combative spirit of Philip the Good still echoes in a short
story by Bandello about how Philip had once been
\protect\hypertarget{10_Chapter_Three__THE_HEROIC_DREAM.xhtmlux5cux23page_109}{}{}kept
by great effort on the part of his noblemen from a duel of
honor.\textsuperscript{\protect\hypertarget{10_Chapter_Three__THE_HEROIC_DREAM.xhtmlux5cux23id_1624}{\protect\hyperlink{23_NOTES.xhtmlux5cux23id_1625}{156}}}

This form still survived in the Italy of the high Renaissance. Francesco
Gonzaga challenged Cesare Borgia to a duel. With sword and dagger he
intended to free Italy from the feared and hated enemy. The duel was
averted through the mediation of King Louis XII of France and the case
ended with a moving
reconciliation.\textsuperscript{\protect\hypertarget{10_Chapter_Three__THE_HEROIC_DREAM.xhtmlux5cux23id_1622}{\protect\hyperlink{23_NOTES.xhtmlux5cux23id_1623}{157}}}
Even Charles V at least twice formally proposed that his quarrels with
Francis I be settled by a personal duel, the first time after Francis
had returned from captivity and, in the opinion of the Emperor, had
broken his word, and then again in
1536.\textsuperscript{\protect\hypertarget{10_Chapter_Three__THE_HEROIC_DREAM.xhtmlux5cux23id_1620}{\protect\hyperlink{23_NOTES.xhtmlux5cux23id_1621}{158}}}

Duels arranged to settle a point of law, judicial duels, and those that
were spontaneous all had a strong survival in custom and thought
particularly in Burgundy and in the quarrelsome north of France. Both
high and low hailed duels as producing truly decisive results. These
concepts, taken by themselves, had little to do with the knightly ideal;
they were much older. Knightly culture bestowed on the duel a certain
respectability, but duels were also favored outside the circles of
nobility. In cases not involving the nobility, duels immediately reveal
the full brutality of the age. The knights themselves enjoyed the
spectacle much more if their code of honor was not involved in it.

Most remarkable, in this connection, is the concern displayed by
noblemen and historians for a judicial duel between two burghers at
Valenciennes in
1455.\textsuperscript{\protect\hypertarget{10_Chapter_Three__THE_HEROIC_DREAM.xhtmlux5cux23id_1618}{\protect\hyperlink{23_NOTES.xhtmlux5cux23id_1619}{159}}}
This was a great rarity, since nothing like it had taken place for about
a hundred years. The citizens of Valenciennes wanted to see it happen at
any cost because to them it meant the maintenance of an old privilege;
but the count of Charolais who was in charge of the administration
during Philip's absence (in Germany) felt differently and managed to
have it postponed month by month while the two litigants, Jacotin
Plouvier and Mahuot, were held back like two expensive fighting cocks.
As soon as the aging count had returned from his trip to see the
Emperor, the decision was made that the battle should take place. Philip
was anxious to witness it himself; it was only for this reason that he
chose to travel via Valenciennes on his trip from Bruges to Louvain.
While knightly spirits like Chastellain and La Marche usually do not
provide a realistic account of the festive \emph{pas d'armes} of knights
and noblemen in spite of all their efforts to do so, in this instance
they record the most clearly seen picture. The crude
Flem\protect\hypertarget{10_Chapter_Three__THE_HEROIC_DREAM.xhtmlux5cux23page_110}{}{}ing
whom Chastellain was is revealed here under his enveloping
houpelande,\textsuperscript{\protect\hypertarget{10_Chapter_Three__THE_HEROIC_DREAM.xhtmlux5cux23id_1616}{\protect\hyperlink{23_NOTES.xhtmlux5cux23id_1617}{160}}}
which was splendid in gold with a pattern of red squares. No detail of
the ``moult belle serimonie''
\protect\hypertarget{10_Chapter_Three__THE_HEROIC_DREAM.xhtmlux5cux23id_2895}{\protect\hyperlink{23_NOTES.xhtmlux5cux23id_2896}{*\textsuperscript{54}}}
escapes him; his description of the circles of the barriers and benches
at the scene is precise.

Each of the poor sacrificial victims has his fencing master at his side.
Jacotin as plaintiff appears first, bareheaded with his hair cut short
and looking very pale. His entire body has been sewn into a dress of
cordovan leather, all just one piece, and he wears nothing underneath.
After a few pious obeisances and the welcoming of the duke, who is
seated behind a latticed screen, the two combatants are seated opposite
one another on two chairs draped in black, and wait until the
preparations are completed. The notables in the circle make their
comments in subdued voices about the chances of the opponents; nothing
escapes them: Mahuot was pale as snow when he kissed the New Testament!
Two servants come in and cover the warriors with fat from their necks
down to their ankles. In the case of Jacotin the fat is immediately
absorbed into the leather, but not in the case of Mahuot; for which of
the two is this a favorable sign? Their hands are covered with ashes,
they put sugar into their mouths. Then they are given clubs and shields
on which there are painted images of saints, which are kissed by the
combatants. They hold their shields with the points upward and have in
their hands ``une bannerolle de devocion,'' a ribbon with a pious motto.

Mahuot, who is short, opens the duel by scooping up sand with the tip of
his shield and flipping it into the eyes of Jacotin. This is followed by
intense club fighting that ends with Mahuot's fall; his opponent throws
himself on top of Mahuot and rubs sand in his mouth and eyes. But Mahuot
manages to get his enemy's finger between his teeth. To free himself,
Jacotin presses his thumb into his tormentor's eye and, in spite of
Mahuot's cries for mercy, twists his arms behind him and turns Mahuot on
his back and proceeds to break his spine. Mahuot, in his death throes,
pleads in vain to be allowed to confess; then he cries out,''O
monseigneur de Bourgogne, je vou ay si bien servi en vostre guerre de
Gand! O monsigneur, pour Dieu, je vous prie mercy, sauvez-moy la
vie!''\protect\hypertarget{10_Chapter_Three__THE_HEROIC_DREAM.xhtmlux5cux23id_2897}{\protect\hyperlink{23_NOTES.xhtmlux5cux23id_2898}{†\textsuperscript{55}}}
\protect\hypertarget{10_Chapter_Three__THE_HEROIC_DREAM.xhtmlux5cux23page_111}{}{}At
this point Chastellain's report breaks off; some pages are missing. From
other sources we know how the half-dead Mahuot is hanged by the
executioner.

Did Chastellain, after his energetic description of these revolting
cruelties end his account with noble knightly contemplations? La Marche
did. He tells us about the shame felt by the noblemen after the event
for having seen such a thing. Thereupon, this incorrigible court poet
continues, God allowed a knightly duel to follow that ended without
injuries.

The conflict between the chivalric spirit and reality is most clearly
revealed when the knightly ideal attempts to establish its validity in
the midst of real war. No matter how much the knightly ideal may have
infused fighting courage with form and vigor, as a rule it had a more
retarding than promoting effect on the conduct of war because it
sacrificed the demands of strategy for those of the beautiful life.
Repeatedly the best leaders, on occasion even the kings, exposed
themselves to the dangers of a romantic war adventure. Edward III risks
his life in a questionable raid on some Spanish naval
transports.\textsuperscript{\protect\hypertarget{10_Chapter_Three__THE_HEROIC_DREAM.xhtmlux5cux23id_1614}{\protect\hyperlink{23_NOTES.xhtmlux5cux23id_1615}{161}}}
The knights of King John's Order of Stars have to take an oath that they
will not retreat in battle more than four ``arpents''; failing that they
must either die or surrender, a peculiar rule of the game that,
according to Froissart, immediately cost about ninety knights their
lives.\textsuperscript{\protect\hypertarget{10_Chapter_Three__THE_HEROIC_DREAM.xhtmlux5cux23id_1612}{\protect\hyperlink{23_NOTES.xhtmlux5cux23id_1613}{162}}}
When Henry V of England in 1415 moved towards the enemy on the eve of
the battle of Agincourt, he mistakenly advanced one evening past the
village that his officials had designated as his quarters for the night.
Now the king, ``comme celuy qui gardoit le plus les cérimonies d'honneur
très loable,''
\protect\hypertarget{10_Chapter_Three__THE_HEROIC_DREAM.xhtmlux5cux23id_2899}{\protect\hyperlink{23_NOTES.xhtmlux5cux23id_2900}{*\textsuperscript{56}}}
had just issued the order that the knights sent out on reconnaissance
missions should take off their battle dress so as to spare them the
shame of retreating in armor on their way back to camp. Since in this
instance he himself had advanced too far in his battle dress he could
not turn back, he therefore spent the night at the place he had reached
and had his advanced troops move forward
accordingly.\textsuperscript{\protect\hypertarget{10_Chapter_Three__THE_HEROIC_DREAM.xhtmlux5cux23id_1610}{\protect\hyperlink{23_NOTES.xhtmlux5cux23id_1611}{163}}}

During the deliberations over the great French invasion of Flanders in
1382 the knightly spirit continuously resisted the requirements of
strategy. ``Se nous querons autres chemins que le droit'' it is argued
against the advice given by Clisson and Coucy to invade
\protect\hypertarget{10_Chapter_Three__THE_HEROIC_DREAM.xhtmlux5cux23page_112}{}{}along
unexpected detours, ``nous ne monsterons pas que soions droites gens
d'armes.''\textsuperscript{\protect\hypertarget{10_Chapter_Three__THE_HEROIC_DREAM.xhtmlux5cux23id_1608}{\protect\hyperlink{23_NOTES.xhtmlux5cux23id_1609}{164}}}\protect\hypertarget{10_Chapter_Three__THE_HEROIC_DREAM.xhtmlux5cux23id_2901}{\protect\hyperlink{23_NOTES.xhtmlux5cux23id_2902}{*\textsuperscript{57}}}
The same holds for a raid by the French on the English coast near
Dartmouth in 1404. The leader, Guillaume du Châtel, plans to attack the
English on their flank because they have protected themselves on the
beach by a trench. But the Sire de Jaille calls the defenders a troop of
peasants; it would be shameful to avoid meeting such opponents head-on;
he urges the others not to be afraid. These words hit home with Du
Châtel: ``It is unknown to the noble heart of a Breton that he be
afraid; now I shall challenge vagrant fortune even though I see death
rather than victory ahead.'' He adds the vow that he will not beg for
mercy, then goes on the attack. He is killed and his troops are
completely
defeated.\textsuperscript{\protect\hypertarget{10_Chapter_Three__THE_HEROIC_DREAM.xhtmlux5cux23id_1606}{\protect\hyperlink{23_NOTES.xhtmlux5cux23id_1607}{165}}}
During the campaign in Flanders there is constant shuffling for
positions in the advance guard; a knight put in charge of the rear guard
stubbornly resists such
duties.\textsuperscript{\protect\hypertarget{10_Chapter_Three__THE_HEROIC_DREAM.xhtmlux5cux23id_1604}{\protect\hyperlink{23_NOTES.xhtmlux5cux23id_1605}{166}}}

The actual application of the knightly ideal to warfare consisted of
agreed-upon
\emph{aristies},\textsuperscript{\protect\hypertarget{10_Chapter_Three__THE_HEROIC_DREAM.xhtmlux5cux23id_1602}{\protect\hyperlink{23_NOTES.xhtmlux5cux23id_1603}{167}}}
be they of two combatants or of groups of equal numbers. The best-known
case is the famous \emph{Combat des Trente} that was fought in 1351 near
Ploërnel in Brittany between thirty Frenchmen led by Beaumanoir and a
group of Englishmen, Germans, and Bretons. Froissart found it to be
extraordinarily beautiful but comments at the end, ``Li aucun le
tenoient à proèce, et li aucun à outrage et grant
outrecuidance.''\textsuperscript{\protect\hypertarget{10_Chapter_Three__THE_HEROIC_DREAM.xhtmlux5cux23id_1600}{\protect\hyperlink{23_NOTES.xhtmlux5cux23id_1601}{168}}}\protect\hypertarget{10_Chapter_Three__THE_HEROIC_DREAM.xhtmlux5cux23id_2903}{\protect\hyperlink{23_NOTES.xhtmlux5cux23id_2904}{†\textsuperscript{58}}}
A duel between Guy de la Tremoille and the English nobleman Pierre de
Courtenay in 1386 that was intended to prove the superiority of either
the English or the French was prohibited by the French regents Burgundy
and Berry and only stopped at the very last
moment.\textsuperscript{\protect\hypertarget{10_Chapter_Three__THE_HEROIC_DREAM.xhtmlux5cux23id_1598}{\protect\hyperlink{23_NOTES.xhtmlux5cux23id_1599}{169}}}
Le Jouvencel shares in this disapproval of such a useless form of
demonstrating bravery. We had already emphasized earlier how in his case
the knight gave way to the commander. When the duke of Bedford proposes
a fight of twelve against twelve, \emph{Le Jouvencel's} chronicler has
the French leader respond: ``There is a general dictum not to do
anything proposed by your enemy. We are here to drive you out of your
positions and that is work enough.'' And the challenge is refused.
Elsewhere Le Jouvencel had
\protect\hypertarget{10_Chapter_Three__THE_HEROIC_DREAM.xhtmlux5cux23page_113}{}{}one
of his officers prohibit such a duel by explaining (he resumes this
explanation at the end) that he would never give permission for
something like this to happen. These are forbidden things. Those who
demand such a duel intend to take something away from their opponent;
that is, their honor, and to claim for themselves vainglory, which is of
little value, while in the meantime they are negligent in their service
to their king and the public
good.\textsuperscript{\protect\hypertarget{10_Chapter_Three__THE_HEROIC_DREAM.xhtmlux5cux23id_1596}{\protect\hyperlink{23_NOTES.xhtmlux5cux23id_1597}{170}}}

This sounds like the voice of the new age. Yet the custom of fighting
duels between opposing forces survives until after the Middle Ages. We
know of the \emph{Sfida de Barletta}, the fight between Bayard and
Sotomaya in 1501; during the Netherlands war we have the fight between
Bréauté and Lekkerbeetje on the heather near Vught in 1600 and that of
Lodewijk van de Kethulle against an Albanesian knight at Deventer in
1591.

In most instances, knightly notions are pushed into the background by
considerations of warfare and tactics. But the idea that even battles in
open field are nothing but honestly arranged duels for justice always
comes to the fore, though it is seldom given its due vis-à-vis the
demands of the necessities of war. Heinrich of Trastamara wants to fight
it out with the enemy in open field at any price. He voluntarily
abandons his advantageous position and loses the battle at Najera (or
Navarete). An English contingent proposes to the Scots in 1333 that they
come down from their advantageous position onto the plain so that they
may fight each other there. Failing to gain access to Calais in order to
liberate the town, the French king politely proposes to the English that
they should designate a site for battle somewhere else; Willem of
Hennegowen goes one step further. He proposes to the French king a
three-day armistice so that there would be time to build a bridge that
would allow the armies to get close to each other for
battle.\textsuperscript{\protect\hypertarget{10_Chapter_Three__THE_HEROIC_DREAM.xhtmlux5cux23id_1594}{\protect\hyperlink{23_NOTES.xhtmlux5cux23id_1595}{171}}}
In all these instances, the knightly offers were declined; strategic
interests retained the upper hand, as was the case with Philip the Good,
who faced a serious conflict with his knightly honor when he was offered
battle three times on the same day but declined to accept.
\textsuperscript{\protect\hypertarget{10_Chapter_Three__THE_HEROIC_DREAM.xhtmlux5cux23id_1592}{\protect\hyperlink{23_NOTES.xhtmlux5cux23id_1593}{172}}}

Yet there remained plenty of opportunities to beautify warfare even in
cases where the knightly ideal had to give way to reality. What an aura
of pride must have been exuded by the colorful and boastful battle armor
itself. On the eve of the battle of Agincourt
\protect\hypertarget{10_Chapter_Three__THE_HEROIC_DREAM.xhtmlux5cux23page_114}{}{}the
armies, encamped opposite each other, stirred up their courage in the
darkness with the music of trumpets and trombones. There were serious
complaints that the French did not have enough of them ``pour eux
resjouyr'' and therefore remained in a subdued
mood.\textsuperscript{\protect\hypertarget{10_Chapter_Three__THE_HEROIC_DREAM.xhtmlux5cux23id_1590}{\protect\hyperlink{23_NOTES.xhtmlux5cux23id_1591}{173}}}

Towards the end of the fifteenth century mercenaries with large drums
based on oriental models made their
appearance.\textsuperscript{\protect\hypertarget{10_Chapter_Three__THE_HEROIC_DREAM.xhtmlux5cux23id_1588}{\protect\hyperlink{23_NOTES.xhtmlux5cux23id_1589}{174}}}
The drum with its hypnotic, unmusical effect is a fitting sign of the
transition from the chivalric to the modern-military period; it is an
element in the mechanization of war. In 1400 the entire beautiful and
half-playful suggestion of personal competition for fame and honor is
still in full bloom. By means of individualized helmet insignia,
weapons, banners, and battle cries, combat retains its individual
character and an element of sport. Throughout the entire day, a man
could hear different individuals raise their cries in a competitive game
of arrogant
pride.\textsuperscript{\protect\hypertarget{10_Chapter_Three__THE_HEROIC_DREAM.xhtmlux5cux23id_1586}{\protect\hyperlink{23_NOTES.xhtmlux5cux23id_1587}{175}}}
Prior to and after the battle the creation of new knights and the
raising of others in rank seal the game: knights are promoted to the
rank of banneret by having the tails of their banner cut
off.\textsuperscript{\protect\hypertarget{10_Chapter_Three__THE_HEROIC_DREAM.xhtmlux5cux23id_1584}{\protect\hyperlink{23_NOTES.xhtmlux5cux23id_1585}{176}}}
The famous camp of Charles the Bold near Neuss had all the festive
splendor of the stateliness of a court: some had their tents ``par
plaisance'' in the form of a castle complete with surrounding galleries
and
gardens.\textsuperscript{\protect\hypertarget{10_Chapter_Three__THE_HEROIC_DREAM.xhtmlux5cux23id_1582}{\protect\hyperlink{23_NOTES.xhtmlux5cux23id_1583}{177}}}

The feats of war had to be recorded within the frame of reference
provided by knightly notions. Attempts were made to distinguish between
battles and mere engagements on technical grounds because each battle
had to have its fixed location and name in the annals of fame.
Monstrelet says, ``Si fut de ce jour en avant ceste besongne appellée la
recontre de Mons en Vimeu. Et ne fu declairée à estre bataille, pour ce
que les parties rencontrèrent l'un l'autre aventureusement, et qu'il n'y
avoit comme nulles bannières
desploiées.''\textsuperscript{\protect\hypertarget{10_Chapter_Three__THE_HEROIC_DREAM.xhtmlux5cux23id_1580}{\protect\hyperlink{23_NOTES.xhtmlux5cux23id_1581}{178}}}\protect\hypertarget{10_Chapter_Three__THE_HEROIC_DREAM.xhtmlux5cux23id_2905}{\protect\hyperlink{23_NOTES.xhtmlux5cux23id_2906}{*\textsuperscript{59}}}
Henry V of England solemnly christens his great victory ``pour tant que
toutes batailles doivent porter le nom de la prochaine forteresse où
elles sont
faictes,''\protect\hypertarget{10_Chapter_Three__THE_HEROIC_DREAM.xhtmlux5cux23id_2907}{\protect\hyperlink{23_NOTES.xhtmlux5cux23id_2908}{†\textsuperscript{60}}}
as the battle of
Agincourt.\textsuperscript{\protect\hypertarget{10_Chapter_Three__THE_HEROIC_DREAM.xhtmlux5cux23id_1578}{\protect\hyperlink{23_NOTES.xhtmlux5cux23id_1579}{179}}}
Remaining for the night on the battlefield was regarded as the accepted
sign of
victory.\textsuperscript{\protect\hypertarget{10_Chapter_Three__THE_HEROIC_DREAM.xhtmlux5cux23id_1576}{\protect\hyperlink{23_NOTES.xhtmlux5cux23id_1577}{180}}}

\protect\hypertarget{10_Chapter_Three__THE_HEROIC_DREAM.xhtmlux5cux23page_115}{}{}The
personal bravery of the prince in battle occasionally has a rather
artificial character. Froissart's description of a fight between Edward
III and a French nobleman near Calais contains expressions that allow us
to assume that they were not bitterly serious, ``Là se combati li rois à
monsigneur Ustasse moult longuement et messires Ustasse à lui, et tant
que il les faisoit moult plaisant veoir.''
\protect\hypertarget{10_Chapter_Three__THE_HEROIC_DREAM.xhtmlux5cux23id_2909}{\protect\hyperlink{23_NOTES.xhtmlux5cux23id_2910}{*\textsuperscript{61}}}
The Frenchman finally surrenders and the fight is concluded with a
supper offered to his prisoner by the
king.\textsuperscript{\protect\hypertarget{10_Chapter_Three__THE_HEROIC_DREAM.xhtmlux5cux23id_1574}{\protect\hyperlink{23_NOTES.xhtmlux5cux23id_1575}{181}}}
In the battle of Saint Richier, Philip of Burgundy had somebody else
wear his splendid armor because of the danger it attracted, but it was
explained that this was done so that he could prove himself better as an
ordinary
combatant.\textsuperscript{\protect\hypertarget{10_Chapter_Three__THE_HEROIC_DREAM.xhtmlux5cux23id_1572}{\protect\hyperlink{23_NOTES.xhtmlux5cux23id_1573}{182}}}
When the young dukes of Berry and Brittany follow Charles the Bold in
his ``guerre du bien public'' they wear, as Commines was told, fake
armor of satin with gilded
nails.\textsuperscript{\protect\hypertarget{10_Chapter_Three__THE_HEROIC_DREAM.xhtmlux5cux23id_1570}{\protect\hyperlink{23_NOTES.xhtmlux5cux23id_1571}{183}}}

Everywhere lies shine through the holes in the stately knightly dress.
Reality continuously denies the ideal. Therefore it withdraws further
and further back into the sphere of literature, festival, and play; only
here the illusion of the beautiful knightly life remains. Here one is
with the caste among whom such feelings have their only validity.

It is astonishing how instantly the knightly ideal fails whenever it has
to assert itself in confrontations with unequals. Whenever the lower
classes are confronted, any need for knightly loftiness disappears.
Noble Chastellain does not have the least understanding of the stubborn
bourgeois honor of the wealthy brewer who does not want to give his
daughter to a soldier of the duke and who risks his life and wealth to
resist the
duke.\textsuperscript{\protect\hypertarget{10_Chapter_Three__THE_HEROIC_DREAM.xhtmlux5cux23id_1568}{\protect\hyperlink{23_NOTES.xhtmlux5cux23id_1569}{184}}}
Froissart reports, without any respect, how Charles VI asks to see the
body of Philipp van Artevelde. ``Quant on l'eust regardé un espasse on
le osta de là et fu pendus à un arbre. Velà le darraine fin de che
Phillippe
d'Artevelle.''\textsuperscript{\protect\hypertarget{10_Chapter_Three__THE_HEROIC_DREAM.xhtmlux5cux23id_1566}{\protect\hyperlink{23_NOTES.xhtmlux5cux23id_1567}{185}}}\protect\hypertarget{10_Chapter_Three__THE_HEROIC_DREAM.xhtmlux5cux23id_2911}{\protect\hyperlink{23_NOTES.xhtmlux5cux23id_2912}{†\textsuperscript{62}}}
The king was not above kicking the body, ``en le traitant de
vilain.''\textsuperscript{\protect\hypertarget{10_Chapter_Three__THE_HEROIC_DREAM.xhtmlux5cux23id_1564}{\protect\hyperlink{23_NOTES.xhtmlux5cux23id_1565}{186}}}\protect\hypertarget{10_Chapter_Three__THE_HEROIC_DREAM.xhtmlux5cux23id_2913}{\protect\hyperlink{23_NOTES.xhtmlux5cux23id_2914}{‡\textsuperscript{63}}}
The most cruel atrocities of the nobility were committed against the
burghers of Ghent during the war of 1382. They sent to the city forty
grain merchants with their limbs cut
\protect\hypertarget{10_Chapter_Three__THE_HEROIC_DREAM.xhtmlux5cux23page_116}{}{}off
and their eyes gouged out. This did not for a moment lessen Froissart's
passion for
knighthood.\textsuperscript{\protect\hypertarget{10_Chapter_Three__THE_HEROIC_DREAM.xhtmlux5cux23id_1562}{\protect\hyperlink{23_NOTES.xhtmlux5cux23id_1563}{187}}}
Chastellain, who revels in the heroic deeds of Jacques de Lalaing and
the like, mentions without showing any sympathy, that an unknown
apprentice from Ghent had dared to attack Lalaing all by
himself.\textsuperscript{\protect\hypertarget{10_Chapter_Three__THE_HEROIC_DREAM.xhtmlux5cux23id_1560}{\protect\hyperlink{23_NOTES.xhtmlux5cux23id_1561}{188}}}
La Marche comments somewhat naively about the heroic deeds of a commoner
from Ghent that would have been important if they had been accomplished
by ``un homme de
bien.''\textsuperscript{\protect\hypertarget{10_Chapter_Three__THE_HEROIC_DREAM.xhtmlux5cux23id_1559}{\protect\hyperlink{23_NOTES.xhtmlux5cux23page_411}{189}}}\protect\hypertarget{10_Chapter_Three__THE_HEROIC_DREAM.xhtmlux5cux23id_2915}{\protect\hyperlink{23_NOTES.xhtmlux5cux23id_2916}{*\textsuperscript{64}}}

Reality pressed the mind in every which way to deny the knightly ideal.
Military strategy had long ago abandoned the tournament element; the
wars of the fourteenth and fifteenth centuries resorted to stealth and
surprise. They were wars of raids and predatory attacks. The English had
first introduced the practice of having the knights dismount during
battle, and this was adopted by the
French.\textsuperscript{\protect\hypertarget{10_Chapter_Three__THE_HEROIC_DREAM.xhtmlux5cux23id_1557}{\protect\hyperlink{23_NOTES.xhtmlux5cux23id_1558}{190}}}
Eustache Deschamps comments mockingly that this was done to keep them
from
fleeing.\textsuperscript{\protect\hypertarget{10_Chapter_Three__THE_HEROIC_DREAM.xhtmlux5cux23id_1555}{\protect\hyperlink{23_NOTES.xhtmlux5cux23id_1556}{191}}}
It is useful to fight at sea, says Froissart, because there the men
cannot run away and
vanish.\textsuperscript{\protect\hypertarget{10_Chapter_Three__THE_HEROIC_DREAM.xhtmlux5cux23id_1553}{\protect\hyperlink{23_NOTES.xhtmlux5cux23id_1554}{192}}}
The extraordinary naiveté of the knightly notions as military principles
manifests itself in the \emph{Débat des hérauts d'armes de France et
d'Angleterre}, a tract from about 1455, in which the supremacy of France
or England is contested in the form of a debate. The English herald
asked his French counterpart why his king, in contrast to the English
king, does not maintain a great fleet. The French herald answers that
his king does not need to do that, and, moreover, that the French
nobility likes war on land more than that at sea for various reasons,
``car il y a danger et perdicion de vie, et Dieu scet quelle pitié quant
il fait une tourmente, et si est la malladie de la mer forte à endurer à
plusieurs gens. Item, et la dure vie dont il fault vivre, qui n'est pas
bien consonante à
noblesse.''\textsuperscript{\protect\hypertarget{10_Chapter_Three__THE_HEROIC_DREAM.xhtmlux5cux23id_1551}{\protect\hyperlink{23_NOTES.xhtmlux5cux23id_1552}{193}}}\protect\hypertarget{10_Chapter_Three__THE_HEROIC_DREAM.xhtmlux5cux23id_2917}{\protect\hyperlink{23_NOTES.xhtmlux5cux23id_2918}{†\textsuperscript{65}}}
Though still of only negligible effect, the use of cannons already
foreshadowed future changes in warfare. It is like an ironic symbolism
that the pride of knight-errantry, ``à la mode de bourgogne,''
\protect\hypertarget{10_Chapter_Three__THE_HEROIC_DREAM.xhtmlux5cux23id_2919}{\protect\hyperlink{23_NOTES.xhtmlux5cux23id_2920}{‡\textsuperscript{66}}}
Jacques de Lalaing, was killed by a fiery
cannonball.\textsuperscript{\protect\hypertarget{10_Chapter_Three__THE_HEROIC_DREAM.xhtmlux5cux23id_1549}{\protect\hyperlink{23_NOTES.xhtmlux5cux23id_1550}{194}}}

\protect\hypertarget{10_Chapter_Three__THE_HEROIC_DREAM.xhtmlux5cux23page_117}{}{}The
noble-military career had a financial side to it that was often openly
admitted. Every page of the histories of late medieval warfare gives us
to understand how important prominent prisoners were for the sake of
exacting ransom. Froissart does not fail to mention how much the
originator of a successful surprise raid gained financially as a result
of
it.\textsuperscript{\protect\hypertarget{10_Chapter_Three__THE_HEROIC_DREAM.xhtmlux5cux23id_1547}{\protect\hyperlink{23_NOTES.xhtmlux5cux23id_1548}{195}}}
But in addition to the immediate advantages of war, pensions and rents
and government posts played a major role in the lives of the knights.
Career advancement is publicly acknowledged as a goal. ``Je sui uns
povres horns qui desire mon avancement,''
\protect\hypertarget{10_Chapter_Three__THE_HEROIC_DREAM.xhtmlux5cux23id_2921}{\protect\hyperlink{23_NOTES.xhtmlux5cux23id_2922}{*\textsuperscript{67}}}
says Eustache de Ribeumont. Froissart endlessly explains \emph{his fait
diverse} of knightly warfare among others as example of those brave men
``qui se désirent à avanchier par
armes.''\textsuperscript{\protect\hypertarget{10_Chapter_Three__THE_HEROIC_DREAM.xhtmlux5cux23id_1545}{\protect\hyperlink{23_NOTES.xhtmlux5cux23id_1546}{196}}}\protect\hypertarget{10_Chapter_Three__THE_HEROIC_DREAM.xhtmlux5cux23id_2923}{\protect\hyperlink{23_NOTES.xhtmlux5cux23id_2924}{†\textsuperscript{68}}}

Deschamps has a ballad in which the knights, pages, and sergeants of the
Burgundian court yearn with great anticipation for payday with the
refrain

\emph{Et quant venra le
tresorier?}\textsuperscript{\protect\hypertarget{10_Chapter_Three__THE_HEROIC_DREAM.xhtmlux5cux23id_1543}{\protect\hyperlink{23_NOTES.xhtmlux5cux23id_1544}{197}}}\protect\hypertarget{10_Chapter_Three__THE_HEROIC_DREAM.xhtmlux5cux23id_2925}{\protect\hyperlink{23_NOTES.xhtmlux5cux23id_2926}{‡\textsuperscript{69}}}

To Chastellain it is natural and fitting that someone striving for
earthly fame is stingy and calculating ``fort veillant et entendant à
grand somme de deniers, soit en pensions, soits en rentes, soit en
governemens ou en
pratiques.''\textsuperscript{\protect\hypertarget{10_Chapter_Three__THE_HEROIC_DREAM.xhtmlux5cux23id_1541}{\protect\hyperlink{23_NOTES.xhtmlux5cux23id_1542}{198}}}
As a matter of fact, the noble Boucicaut himself, who was the model of
all knights, seems not to have been entirely free from a certain greed
for
money.\textsuperscript{\protect\hypertarget{10_Chapter_Three__THE_HEROIC_DREAM.xhtmlux5cux23id_1539}{\protect\hyperlink{23_NOTES.xhtmlux5cux23id_1540}{199}}}
The sober Commines ranks a nobleman according to his stipend as ``un
gentilhomme de vingt
escuz.''\textsuperscript{\protect\hypertarget{10_Chapter_Three__THE_HEROIC_DREAM.xhtmlux5cux23id_1537}{\protect\hyperlink{23_NOTES.xhtmlux5cux23id_1538}{200}}}\protect\hypertarget{10_Chapter_Three__THE_HEROIC_DREAM.xhtmlux5cux23id_2927}{\protect\hyperlink{23_NOTES.xhtmlux5cux23id_2928}{§\textsuperscript{70}}}

Among the loud voices glorifying knightly warfare there can be heard
occasional voices rejecting the knightly ideal. Sometimes they are sober
voices, sometimes they are derisive. Noblemen on occasion recognize the
dressed-up misery and falsity of such a life of war and
tournaments.\textsuperscript{\protect\hypertarget{10_Chapter_Three__THE_HEROIC_DREAM.xhtmlux5cux23id_1535}{\protect\hyperlink{23_NOTES.xhtmlux5cux23id_1536}{201}}}
It does not come as a surprise that Louis XI and Philippe de Commines,
two sarcastic minds who had nothing but scorn and contempt for
knighthood, found each other. Commines's description of the battle of
Montlhéry is entirely
mod\protect\hypertarget{10_Chapter_Three__THE_HEROIC_DREAM.xhtmlux5cux23page_118}{}{}ern
in its sober realism. There are no beautiful heroic deeds, no invented
dramatic events, but only the report of continual advance and retreat,
hesitation and fear, all told with light sarcasm. He delights in
reporting shameful flights with bravery restored when the moment of
danger had passed. He rarely uses the word ``honneur'' and treats honor
almost like a necessary evil. ``Mon advis est que s'il eust voulu s'en
aller ceste nuyt, il eust bien faict. .~.~. Mais sans doubte là où il
avoit de l'honneur, il n'eust point voulu estre reprins de
couardise.''\protect\hypertarget{10_Chapter_Three__THE_HEROIC_DREAM.xhtmlux5cux23id_3081}{\protect\hyperlink{23_NOTES.xhtmlux5cux23id_3082}{*\textsuperscript{71}}}
Even where he reports bloody encounters, one searches in vain for the
vocabulary of knighthood; he does not know the words bravery or
chivalry.\textsuperscript{\protect\hypertarget{10_Chapter_Three__THE_HEROIC_DREAM.xhtmlux5cux23id_1533}{\protect\hyperlink{23_NOTES.xhtmlux5cux23id_1534}{202}}}

Does Commines inherit his sober mind from his Zealand mother, Margretha
of Arnemuiden? It appears that in Holland, the presence of the vain
adventurer William IV of Hennegouw notwithstanding, the knightly spirit
died away quite early, while the Hennegouw with which it was united had
always been the true land of knightly nobility. During the \emph{Combat
des Trente} the best man on the English side was a certain Crokart,
formerly a servant of the Lords of Arkel. He had acquired a large
fortune during the war, estimated to be worth about sixty thousand
crowns, and a stable of thirty horses; he had also acquired considerable
fame for bravery, which had prompted the King of France to offer him a
knighthood and a respectable marriage in the event that he would become
French. This Crokart returned to Holland with his fame and fortune and
held forth in grand style. But the Dutch notables knew well who he was
and ignored him. He finally returned to the country where knightly fame
was more
favored.\textsuperscript{\protect\hypertarget{10_Chapter_Three__THE_HEROIC_DREAM.xhtmlux5cux23id_1531}{\protect\hyperlink{23_NOTES.xhtmlux5cux23id_1532}{203}}}

When Jean de
Nevers\textsuperscript{\protect\hypertarget{10_Chapter_Three__THE_HEROIC_DREAM.xhtmlux5cux23id_1529}{\protect\hyperlink{23_NOTES.xhtmlux5cux23id_1530}{204}}}
prepared himself for his journey to Turkey where he was to find
Nicopolis, Froissart had the Duke Albert of Bavaria, the duke of
Holland, Zealand, and Hennegouw, say to his son William, ``Guillemme
puisque tu as la voulenté de voyagier et aler en Honguerie et en Turquie
et quérir les armes sur gents et pays qui onques riens ne nous
foufirent, ne nul article de raison tu n'y as d'y aler fors que pour la
vayne gloire de ce monde, laisse Jean de Bourgoigne et nos cousins de
France faire leur emprises,
\protect\hypertarget{10_Chapter_Three__THE_HEROIC_DREAM.xhtmlux5cux23page_119}{}{}et
fay la tienne à par toy, et t'en va en Frise et conquiers nostre
héritage.''\textsuperscript{\protect\hypertarget{10_Chapter_Three__THE_HEROIC_DREAM.xhtmlux5cux23id_1527}{\protect\hyperlink{23_NOTES.xhtmlux5cux23id_1528}{205}}}\protect\hypertarget{10_Chapter_Three__THE_HEROIC_DREAM.xhtmlux5cux23id_3083}{\protect\hyperlink{23_NOTES.xhtmlux5cux23id_3084}{*\textsuperscript{72}}}

Of all the lands under the Burgundian duke the nobility of Holland had
by far the weakest representation during the Vows of the Cross taken at
the festivities in Lille. When after the festivities still more written
vows were collected in all territories, twenty-seven came from Artois,
fifty-four from Flanders, twenty-seven from Hennegouw, and four from
Holland, and even those sound quite conditional and
cautious.\textsuperscript{\protect\hypertarget{10_Chapter_Three__THE_HEROIC_DREAM.xhtmlux5cux23id_1525}{\protect\hyperlink{23_NOTES.xhtmlux5cux23id_1526}{206}}}

But knighthood could hardly have been the life ideal of centuries if it
had not contained high values for the development of society, if it had
not been socially, ethically, and aesthetically necessary. The power of
this ideal had once rested in its beautiful exaggeration. It seems as if
the medieval mind in all its bloody passions could only be guided by an
ideal that was fixed much too highly: this was done by the church, and
was done by the knightly spirit as well. ``Without this violence of
direction, which men and women have, without a spice of bigot and
fanatic, no excitement, no efficiency. We aim above the mark to hit the
mark. Every act hath some falsehood of exaggeration in
it.''\textsuperscript{\protect\hypertarget{10_Chapter_Three__THE_HEROIC_DREAM.xhtmlux5cux23id_1523}{\protect\hyperlink{23_NOTES.xhtmlux5cux23id_1524}{207}}}

But the more a cultural ideal is filled with the claims to the highest
virtues, the greater the disharmony between the life form and reality.
Only a time still able to close its eyes to gross reality and receptive
to the highest illusion could uphold the knightly ideal with its still
half-religious content. The unfolding new culture soon forced the
abandonment of the all too lofty aspirations of the old life forms. The
knight is transformed into the French \emph{gentilhomme} of the
seventeenth century, who, though still maintaining a number of concepts
of state and honor, no longer claims to be a warrior for matters of
faith or a defender of the weak and oppressed. The place of the type of
French nobleman is taken---modified and refined---by the ``gentleman,''
who is derived directly from the type of the old knight. During the
successive transformations of the
\protect\hypertarget{10_Chapter_Three__THE_HEROIC_DREAM.xhtmlux5cux23page_120}{}{}ideal
the outermost shells, each having become a lie, are peeled away time and
again.

The knightly life form was overburdened with ideals of beauty, virtue,
and utility. If viewed with a sober sense of reality, as does Commines,
all this highly praised chivalry appeared to be as useless and phony as
a fabricated, ridiculously anachronistic comedy. The true driving forces
that prompted human action and determined the fate of states and
communities lay elsewhere. As the social usefulness of the knightly
ideal had already become extremely weak, so it was that the ethical
aspect, the practice of virtue, which also had been claimed by the
knightly ideal, was even weaker. Seen from a truly spiritual point of
view, all that noble life was nothing but open sin and vanity. The ideal
failed also from a purely aesthetic point of view: even the beauty of
that life form could be denied in every respect. Though the knightly
ideal may on occasion appear to be desirable to some burghers, a great
feeling of fatigue and overindulgence arises among the nobility itself.
The beautiful play of courtly life was so colored, so false, so
paralyzing. Away from the painfully constructed art of life towards that
of secure simplicity and peace!

There were then two ways to preserve the knightly ideal: the one to move
towards real, active life and the modern spirit of inquiry, the other
that of denial of the world. But the latter, like the Y of
Pythagoras,\textsuperscript{\protect\hypertarget{10_Chapter_Three__THE_HEROIC_DREAM.xhtmlux5cux23id_1521}{\protect\hyperlink{23_NOTES.xhtmlux5cux23id_1522}{208}}}
split into two: the main line was that of the genuinely spiritual life,
the secondary line kept close to the edge of the world and its
pleasures. The yearning for the beautiful life was so strong that even
in places where the vanity and degeneration of courtly and combative
life were recognized, there still seemed to be a path to a beautiful
earthly life, to a sweeter and brighter dream. The old illusion of the
pastoral life still radiated its promise of natural bliss with the full
glow it had possessed since Theocritus. It seemed to be possible to
achieve the great liberation without a struggle through a flight from
the hate- and envy-filled scramble for vain honors and vain rank, from
oppressive, overburdened luxury and splendor, and from cruel, dangerous
war.

The praise of the simple life was a theme that medieval literature had
already adopted from antiquity. It is not identical with the pastorale:
the two forms are a positive and a negative expression of one and the
same emotion. The pastorale describes a positive contrast to courtly
life. The negative expression describes a flight
\protect\hypertarget{10_Chapter_Three__THE_HEROIC_DREAM.xhtmlux5cux23page_121}{}{}from
the court, from the praise of the \emph{aurea mediocritas} (the Golden
Mean); it denies the aristocratic life ideal, a denial expressed through
scholarship, solitary quietude, or work. The motifs are continuously
fusing. As early as the twelfth century John of Salisbury and Walter
Mapes had written their tracts ``de nugis curialium'' on the theme of
the shortcomings of courtly life. In fourteenth-century France the
classic expression of this theme is found in a poem by Philippe de
Vitri, bishop of Meaux, who was both a composer and a poet and was
praised by Petrarch. In this poem, ``Le dit de Franc
Gontier,''\textsuperscript{\protect\hypertarget{10_Chapter_Three__THE_HEROIC_DREAM.xhtmlux5cux23id_1519}{\protect\hyperlink{23_NOTES.xhtmlux5cux23id_1520}{209}}}
the fusion with the pastorale is perfect:

\emph{Soubz feuille vert, sur herbe delitable}

\emph{Lez ru bruiant et prez clere fontaine}

\emph{Trouvay fichee une borde portable},

\emph{Ilec mengeoit Gontier o dame Helayne}

\emph{Fromage frais, laict, burre fromaigee},

\emph{Craime, matton, pomme, nois, prune, poire},

\emph{aulx et oignons, escaillogne froyee}

\emph{Sur crouste bise, a gros sel, pour mieulx boire.
\protect\hypertarget{10_Chapter_Three__THE_HEROIC_DREAM.xhtmlux5cux23id_3085}{\protect\hyperlink{23_NOTES.xhtmlux5cux23id_3086}{*\textsuperscript{73}}}}

After the meal they kiss one another, ``et bouche et nez, polie et bien
barbue'';\protect\hypertarget{10_Chapter_Three__THE_HEROIC_DREAM.xhtmlux5cux23id_3087}{\protect\hyperlink{23_NOTES.xhtmlux5cux23id_3088}{†\textsuperscript{74}}}
thereupon Gontier goes to the forest to chop down a tree while Lady
Helayne does the wash.

\emph{J'oy Gontier en abatant son arbe}

\emph{Dieu mercier de sa vie seüre};

\emph{``Ne scay''}---\emph{dit-il}---\emph{``que sont pilliers de
marbre},

\emph{Pommeaux lusisans, murs vestus de paincture};

\emph{Je n'ay paour de traïson tissue}

\emph{Soubz beau semblant, ne qu'empoisonné soye}

\emph{En vaisseau d'or. Je n'ay la teste nue}

\emph{Devant thirant, ne genoil qui s'i ploye}.

\emph{Verge d'ussier jamais ne me deboute},

\emph{\protect\hypertarget{10_Chapter_Three__THE_HEROIC_DREAM.xhtmlux5cux23page_122}{}{}Car
jusques la ne m'esprent convoitise},

\emph{Ambition, ne lescherie gloute}.

\emph{Labour me paist en joieuse franchise};

\emph{Moult j'ame Helayne et elle moy sans faille},

\emph{Et c'est assez. De tombel n'avons cure.''}

\emph{Lors je dy: ``Las! serf de court ne vault maille},

\emph{Mais Franc Gontier vault en or jame
pure.''\protect\hypertarget{10_Chapter_Three__THE_HEROIC_DREAM.xhtmlux5cux23id_3089}{\protect\hyperlink{23_NOTES.xhtmlux5cux23id_3090}{*\textsuperscript{75}}}}

For coming generations this poem remained the classic expression of the
ideal of the simple life replete with security and independence, its
enjoyment of moderation, good health, work, and natural, uncomplicated
love in marriage.

Eustache Deschamps sang the praise of the simple life and rejection of
the court in a number of ballads. Among others he presents a faithful
imitation of Franc Gontier:

\emph{En retounant d'un court souveraine}

\emph{Où j'avoie longuement sejourné},

\emph{En un bosquet, dessus une fontaine}

\emph{Trouvay Robin le franc, enchapelé},

\emph{Chapeauls de flours avoit cilz ajublé}

\emph{Dessus son chief et Marion sa drue} .~.~.
\textsuperscript{\protect\hypertarget{10_Chapter_Three__THE_HEROIC_DREAM.xhtmlux5cux23id_1517}{\protect\hyperlink{23_NOTES.xhtmlux5cux23id_1518}{210}}}\protect\hypertarget{10_Chapter_Three__THE_HEROIC_DREAM.xhtmlux5cux23id_3091}{\protect\hyperlink{23_NOTES.xhtmlux5cux23id_3092}{†\textsuperscript{76}}}

He expands the theme by ridiculing military life and knighthood. In
simple seriousness he bewails the misery and cruelty of war; there is no
estate worse than that of the warrior; the seven cardinal sins are his
daily work; greed and the vain quest for fame constitute the essence of
war:

.~.~.
\protect\hypertarget{10_Chapter_Three__THE_HEROIC_DREAM.xhtmlux5cux23page_123}{}{}\emph{Je
vueil mener d'or en avant}

\emph{Estat moien, c'est mon oppinion},

\emph{Guerre laissier et vivre en labourant}:

\emph{guerre mener n'est que
dampnacion}.\textsuperscript{\protect\hypertarget{10_Chapter_Three__THE_HEROIC_DREAM.xhtmlux5cux23id_1515}{\protect\hyperlink{23_NOTES.xhtmlux5cux23id_1516}{211}}}\emph{\protect\hypertarget{10_Chapter_Three__THE_HEROIC_DREAM.xhtmlux5cux23id_3093}{\protect\hyperlink{23_NOTES.xhtmlux5cux23id_3094}{*\textsuperscript{77}}}}

Or he mockingly curses those who might want to challenge him, or has a
lady expressedly order him not to fight a duel that has been forced on
him for her
sake.\textsuperscript{\protect\hypertarget{10_Chapter_Three__THE_HEROIC_DREAM.xhtmlux5cux23id_1513}{\protect\hyperlink{23_NOTES.xhtmlux5cux23id_1514}{212}}}

But mostly the poem is about the theme of the \emph{aurea mediocritas}
itself.

\emph{Je ne requier à Dieu fors qu'il me doint}

\emph{En ce monde lui server et loer},

\emph{Vivre pour moy, cote entiere ou pourpoint},

\emph{Aucun cheval pour mon labour porter},

\emph{Et qui je puisse mon estat gouverner}

\emph{Moiennement, en grace, sanz envie},

\emph{Sanz trop avoir et sanz pain demander},

\emph{Car au jour d'ui est la plus seure
vie}.\textsuperscript{\protect\hypertarget{10_Chapter_Three__THE_HEROIC_DREAM.xhtmlux5cux23id_1511}{\protect\hyperlink{23_NOTES.xhtmlux5cux23id_1512}{213}}}\protect\hypertarget{10_Chapter_Three__THE_HEROIC_DREAM.xhtmlux5cux23id_3095}{\protect\hyperlink{23_NOTES.xhtmlux5cux23id_3096}{†\textsuperscript{78}}}

Seeking fame and fortune brings nothing but misery. The poor man is
satisfied and happy and lives an undisturbed and long life:

.~.~. \emph{Un ouvrier et uns povres chartons}

\emph{Va mauvestuz, deschirez et deschaulz}

\emph{Mais en ouvrant prant en gré ses travaulz}

\emph{Et liement fait son euvre fenir}.

\emph{Par nuit dort bien; pour ce uns telz cueurs loiaulx}

\emph{Voit quatre roys et leur regne
fenir}.\textsuperscript{\protect\hypertarget{10_Chapter_Three__THE_HEROIC_DREAM.xhtmlux5cux23id_1509}{\protect\hyperlink{23_NOTES.xhtmlux5cux23id_1510}{214}}}\protect\hypertarget{10_Chapter_Three__THE_HEROIC_DREAM.xhtmlux5cux23id_3097}{\protect\hyperlink{23_NOTES.xhtmlux5cux23id_3098}{‡\textsuperscript{79}}}

\protect\hypertarget{10_Chapter_Three__THE_HEROIC_DREAM.xhtmlux5cux23page_124}{}{}The
poet liked the idea that the simple laborer outlived four kings so much
that he made repeated use of
it.\textsuperscript{\protect\hypertarget{10_Chapter_Three__THE_HEROIC_DREAM.xhtmlux5cux23id_1507}{\protect\hyperlink{23_NOTES.xhtmlux5cux23id_1508}{215}}}

The editor of Deschamps's poetry, Gaston Raynaud, argues that all the
poems with this
tendency,\textsuperscript{\protect\hypertarget{10_Chapter_Three__THE_HEROIC_DREAM.xhtmlux5cux23id_1505}{\protect\hyperlink{23_NOTES.xhtmlux5cux23id_1506}{216}}}
usually among the best Deschamps wrote, should be assigned to the late
period when he, removed from office, abandoned and disappointed, had
gained insight into the vanity of courtly
life.\textsuperscript{\protect\hypertarget{10_Chapter_Three__THE_HEROIC_DREAM.xhtmlux5cux23id_1503}{\protect\hyperlink{23_NOTES.xhtmlux5cux23id_1504}{217}}}
This would mean that he had also turned inward, but might it not also be
a reaction, an expression of general fatigue? It seems to me that the
nobility itself favored and demanded these productions, in the midst of
their lives of driving passion and splendor, from a court poet who at
other times prostituted his talents to satisfy their crudest need for
laughter.

Around 1400 the theme of the disapproval of courtly life is further
elaborated within the circle of the earliest French humanists, who were
in part identical with the reform party of the great church councils.
Pierre d'Ailly himself, a great theologian and church politician,
composed, as a companion piece to ``Franc Gontier,'' a picture of the
tyrant whose slavish life is filled with anxiety. His brothers-in-spirit
used for the purpose of their critiques of courtly life the newly
rediscovered form of letters, as in the case of Nicholas de
Clémanges\textsuperscript{\protect\hypertarget{10_Chapter_Three__THE_HEROIC_DREAM.xhtmlux5cux23id_1501}{\protect\hyperlink{23_NOTES.xhtmlux5cux23id_1502}{218}}}
and his correspondent Jean de
Montreuil.\textsuperscript{\protect\hypertarget{10_Chapter_Three__THE_HEROIC_DREAM.xhtmlux5cux23id_1499}{\protect\hyperlink{23_NOTES.xhtmlux5cux23id_1500}{219}}}
The Milanese Ambrosius de Millis, secretary to the duke of Orléans,
belonged to this circle and wrote a literary letter to a Gontier Col in
which he has a courtier warn his friend against entering court
service.\textsuperscript{\protect\hypertarget{10_Chapter_Three__THE_HEROIC_DREAM.xhtmlux5cux23id_1497}{\protect\hyperlink{23_NOTES.xhtmlux5cux23id_1498}{220}}}
This letter, itself long forgotten, was translated by Alain Chartier,
the famous court poet, or was at least published in its translated
version with the title \emph{Le Curial} under his
name.\textsuperscript{\protect\hypertarget{10_Chapter_Three__THE_HEROIC_DREAM.xhtmlux5cux23id_1495}{\protect\hyperlink{23_NOTES.xhtmlux5cux23id_1496}{221}}}
\emph{Le Curial} was later retranslated into Latin by the humanist
Robert
Gaguin.\textsuperscript{\protect\hypertarget{10_Chapter_Three__THE_HEROIC_DREAM.xhtmlux5cux23id_1493}{\protect\hyperlink{23_NOTES.xhtmlux5cux23id_1494}{222}}}

A certain Charles de Rochefort handled the theme in the form of an
allegorical poem in the style of the \emph{Roman de la rose}. His
\emph{L'abuzé} was ascribed to King
René.\textsuperscript{\protect\hypertarget{10_Chapter_Three__THE_HEROIC_DREAM.xhtmlux5cux23id_1491}{\protect\hyperlink{23_NOTES.xhtmlux5cux23id_1492}{223}}}
Jean Meschinot composed poems like those of all his predecessors:

\emph{La cour est un mer, dort sourt}

\emph{Vagues d'orgueil, d'envie orages}

. \emph{.~.~. . .~.~. . .~.~. . .~.~. . .~.~. . .~.~. .} .

\emph{Ire esmeut debats et outrages},

\emph{que les nefs jettent souvent bas};

\emph{\protect\hypertarget{10_Chapter_Three__THE_HEROIC_DREAM.xhtmlux5cux23page_125}{}{}Traison
y fait son personnage}

\emph{nage aultre part pour tes
ebats}.\textsuperscript{\protect\hypertarget{10_Chapter_Three__THE_HEROIC_DREAM.xhtmlux5cux23id_1490}{\protect\hyperlink{23_NOTES.xhtmlux5cux23page_412}{224}}}\emph{\protect\hypertarget{10_Chapter_Three__THE_HEROIC_DREAM.xhtmlux5cux23id_3099}{\protect\hyperlink{23_NOTES.xhtmlux5cux23id_3100}{*\textsuperscript{80}}}}

The old theme had not lost its fascination as late as the sixteenth
century.\textsuperscript{\protect\hypertarget{10_Chapter_Three__THE_HEROIC_DREAM.xhtmlux5cux23id_1488}{\protect\hyperlink{23_NOTES.xhtmlux5cux23id_1489}{225}}}

Security, quietude, and independence are the good things of life and for
their sake people want to flee the court in order to lead a simple life
of work and moderation in the midst of nature. This is the negative side
of the ideal. But the positive side is not so much the enjoyment of work
and simplicity itself, but the comfort of natural love. The pastoral
ideal leads us directly to the forms of erotic culture.
