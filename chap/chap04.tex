\chapter{THE FORMS OF LOVE}

EVER SINCE THE PROVENÇAL TROUBADOURS OF THE twelfth century first gave
voice to the melody of unsatisfied desire, the violins of love had sung
ever higher until only Dante could play the instrument purely.

The medieval mind took one of its most important turns when it developed
for the first time an ideal of love with a negative groundtone. To be
sure, antiquity had also sung of the yearning and pain of love, but did
that yearning not merely imply delay and the titillation of the
certainty of fulfillment? And in the love stories of antiquity that did
end sadly, the unavailability of the beloved was not at issue, but
rather a previously satisfied love that was dramatically ended by death
itself, as in the case of Cephalus and Procris or Pyramus and Thisbe.
The feeling of pain in those stories lay not in erotic frustration, but
in the sadness of fate. It is first in the courtly \emph{Minne} of the
troubadours that frustration itself becomes the vital concern. An
intellectual form of erotic thought had been created that was able to
encompass a superabundance of ethical content, without having, on
account of it, to entirely abandon the connection with the natural love
of women. The courtly service of women that idealized itself by never
demanding fulfillment had arisen from sensual love itself. In
\emph{Minne}, love became the field in which all aesthetic and ethical
perfection was allowed to blossom. The noble lover, according to the
theory of courtly \emph{Minne}, was made virtuous and pure by his love.
In lyric poetry the spiritual element more and more gained the upper
hand until, finally, the effect of love is a state of sacred insight and
piety: \emph{La vita
nuova}.\textsuperscript{\protect\hypertarget{11_Chapter_Four__THE_FORMS_OF_LOVE.xhtmlux5cux23id_1486}{\protect\hyperlink{23_NOTES.xhtmlux5cux23id_1487}{1}}}

This had to be followed by a new turn of direction. In the \emph{dolce
stil
nuovo,\protect\hypertarget{11_Chapter_Four__THE_FORMS_OF_LOVE.xhtmlux5cux23id_3101}{\protect\hyperlink{23_NOTES.xhtmlux5cux23id_3102}{*\textsuperscript{1}}}}
Dante and his contemporaries had reached a point
be\protect\hypertarget{11_Chapter_Four__THE_FORMS_OF_LOVE.xhtmlux5cux23page_127}{}{}yond
which one cannot pass. Petrarch stands hesitatingly between the ideal of
spiritual love and the new inspiration of antiquity. And from Petrarch
to Lorenzo de'Medici the love song in Italy returns to the path of
natural sensuality that had permeated the admired models of antiquity.
The artificially elaborated system of the courtly \emph{Minne} was
abandoned.

In France, and in those lands that were under the spell of the French
spirit, a different turn was taken. In those countries the development
of erotic thought after the high flowering of the courtly lyric was not
as simple. The forms of the old system remain, but they are filled with
a new spirit. There, even before the \emph{Vita nuova} had found the
true harmonies of a spiritualized passion, the \emph{Roman de la rose}
had partially filled the forms of courtly \emph{Minne} with new content.
For about two centuries this work by Guillaume de Lorris and Jean
Clopinel (or
Chopinel),\textsuperscript{\protect\hypertarget{11_Chapter_Four__THE_FORMS_OF_LOVE.xhtmlux5cux23id_1484}{\protect\hyperlink{23_NOTES.xhtmlux5cux23id_1485}{2}}}
begun before 1240 and completed before 1280, not only completely
dominated the forum of aristocratic love, but, because of its wealth of
encyclopedic digressions into all sorts of other arenas, was also the
treasure house from which educated people drew the most lively elements
of their intellectual development. It is impossible to overestimate the
importance of the fact that the ruling class of an entire period
obtained, in this manner, its view of life and its erudition in the form
of an \emph{ars
amandi.\protect\hypertarget{11_Chapter_Four__THE_FORMS_OF_LOVE.xhtmlux5cux23id_3103}{\protect\hyperlink{23_NOTES.xhtmlux5cux23id_3104}{*\textsuperscript{2}}}}
During no other age did the ideal of worldly erudition enter into such
intimate union with the love of women than from the twelfth to the
fifteenth centuries. All Christian and social virtues, the entire
structure of the forms of life, were fitted into the framework of true
love by the system of \emph{Minne}. The erotic view of life, either in
its older, purely courtly form or in its embodiment in the \emph{Roman
de la rose}, can be placed on the same level with its contemporary,
scholasticism. Both represent a great effort by the medieval mind to
comprehend everything that pertains to life from a single point of view.

The entire struggle to beautify life is concentrated in the colorful
presentation of the forms of love. Those who sought beauty in honor and
rank, or who endeavored to embellish their lives with splendor and
stateliness, in short, those who sought the beauty of life in pride,
were constantly reminded of the vanity of these things. In love,
however, there appeared to be a purpose and reality for
\protect\hypertarget{11_Chapter_Four__THE_FORMS_OF_LOVE.xhtmlux5cux23page_128}{}{}all
those who had not entirely taken leave of that earthly bliss, which was
the enjoyment of beauty itself. In this there was no need to create a
beautiful life from noble forms or to emphasize high status. Here
dwelled the most profound beauty, the highest bliss itself, which needed
only to be given color and form. Every beautiful object, every flower,
and every sound could contribute something to the building of love's
life form.

The effort to stylize love was more than a vain game. The power of
passion itself required that late medieval society transform the life of
love into a beautiful play with noble rules. Here above all, if men were
not to fall into crude barbarism, there was a need to frame emotions
within fixed forms. Among the lower estates it was left to the church to
tame unrestrained outbursts, and the church met its task as well as
could be managed under the circumstances. The aristocracy, which felt
itself to be somewhat independent of the church since it possessed a
modicum of culture from outside the ecclesiastical realm, fashioned an
obstacle to disorder out of refined eroticism itself; literature,
fashion, and the forms of etiquette exercised in this way a normative
influence on the life of love.

Or at least, these three created a beautiful illusion within which
people could imagine themselves to live, in spite of the fact that even
among the upper classes life remained extraordinarily crude. Ordinary
behavior had a character of free-spirited insolence that later times
have lost. The duke of Burgundy had the bathhouses of Valenciennes put
in order for the English envoys expected there ``pour eux et pour
quiconque avoient de famille, voire bains estorés de tout ce qu'il faut
au mestier de Vénus, à prendre par choix et par élection ce que on
désiroit mieux, et tout aux frais du
duc.''\textsuperscript{\protect\hypertarget{11_Chapter_Four__THE_FORMS_OF_LOVE.xhtmlux5cux23id_1482}{\protect\hyperlink{23_NOTES.xhtmlux5cux23id_1483}{3}}}\protect\hypertarget{11_Chapter_Four__THE_FORMS_OF_LOVE.xhtmlux5cux23id_3105}{\protect\hyperlink{23_NOTES.xhtmlux5cux23id_3106}{*\textsuperscript{3}}}
The virtuous behavior of his son, Charles the Bold, was suspected by
many to be inappropriate for a
prince.\textsuperscript{\protect\hypertarget{11_Chapter_Four__THE_FORMS_OF_LOVE.xhtmlux5cux23id_1480}{\protect\hyperlink{23_NOTES.xhtmlux5cux23id_1481}{4}}}
Among the mechanical pranks of the pleasure house at Hesdin the bills
mention ``ung engien pour moullier les dames en marchant par
dessoubz.''\textsuperscript{\protect\hypertarget{11_Chapter_Four__THE_FORMS_OF_LOVE.xhtmlux5cux23id_1478}{\protect\hyperlink{23_NOTES.xhtmlux5cux23id_1479}{5}}}\protect\hypertarget{11_Chapter_Four__THE_FORMS_OF_LOVE.xhtmlux5cux23id_3107}{\protect\hyperlink{23_NOTES.xhtmlux5cux23id_3108}{†\textsuperscript{4}}}

Yet this crudity is not simply a failure of the ideal. Even as
\protect\hypertarget{11_Chapter_Four__THE_FORMS_OF_LOVE.xhtmlux5cux23page_129}{}{}ennobled
love had a style of its own, so too did license itself, and a much older
one at that. It may be called the epithalamic style. In matters of
notions of love, a refined society, such as that of the waning Middle
Ages, inherits so many ancient motifs that the erotic styles must
compete or merge with one another. The style of courtly \emph{Minne} was
confronted by the primitive form of eroticism, with much older roots and
an equally vital significance, which glorified the sexual union itself.
Although in Christian culture its value was replaced by sacred mystery,
eroticism remained as alive as \emph{Minne}.

The entire epithalamic apparatus with its shameless laughter and its
phallic symbolism had once been a part of the sacred rites of the
wedding festivity itself. The consummation of marriage and the wedding
ceremony had once been inseparable: a great mystery that focused on
copulation. Then came the church and claimed sanctity and mystery for
itself by transposing both the marriage and its consummation into the
sacrament of a solemn union. The secondary aspects of the mystery, such
as the procession, the song, and the shout of jubilation, were left to
the wedding festivities. But, stripped of their sacred power, they were
expressed with even more lascivious abandon and the church was never
able to tame them. No churchly ethic could repress the exuberant cry of
life in the ``Hymen, O Hymenäe''! No puritanical mind could banish from
custom the shamelessly open character of the wedding night. Even the
seventeenth century still knew this open character in its full bloom.
Only modern individual sensitivity, which desires to hide in stillness
and darkness that which belongs to the two individuals alone, has broken
with these public displays.

If we remember that as late as 1641 at the wedding of the young Prince
of Orange with Mary of England, the practical jokes rendered the
bridegroom, a boy still, nearly incapable of consummating the marriage,
we will not be astonished at the frivolous abandon with which princely
and noble marriages used to be celebrated around 1400. The obscene grin
with which Froissart describes the marriage of Charles VI with Isabella
of
Bavaria,\textsuperscript{\protect\hypertarget{11_Chapter_Four__THE_FORMS_OF_LOVE.xhtmlux5cux23id_1476}{\protect\hyperlink{23_NOTES.xhtmlux5cux23id_1477}{6}}}
or the Epithalamium that Deschamps dedicated to Anton of Burgundy are
examples for
us.\textsuperscript{\protect\hypertarget{11_Chapter_Four__THE_FORMS_OF_LOVE.xhtmlux5cux23id_1474}{\protect\hyperlink{23_NOTES.xhtmlux5cux23id_1475}{7}}}
The \emph{Cent nouvelles nouvelles} tell us, as something quite
ordinary, of a couple who were married during early mass and, after a
light meal, immediately went to
bed.\textsuperscript{\protect\hypertarget{11_Chapter_Four__THE_FORMS_OF_LOVE.xhtmlux5cux23id_1472}{\protect\hyperlink{23_NOTES.xhtmlux5cux23id_1473}{8}}}
All the jokes
concern\protect\hypertarget{11_Chapter_Four__THE_FORMS_OF_LOVE.xhtmlux5cux23page_130}{}{}ing
weddings or sex in general were considered suitable for gatherings of
ladies. The \emph{Cent nouvelles
nouvelles\protect\hypertarget{11_Chapter_Four__THE_FORMS_OF_LOVE.xhtmlux5cux23id_3109}{\protect\hyperlink{23_NOTES.xhtmlux5cux23id_3110}{*\textsuperscript{5}}}}
introduce themselves, even though with some irony, as ``glorieuse et
édificant
euvre,''\protect\hypertarget{11_Chapter_Four__THE_FORMS_OF_LOVE.xhtmlux5cux23id_3111}{\protect\hyperlink{23_NOTES.xhtmlux5cux23id_3112}{†\textsuperscript{6}}}
as stories ``moult plaisants à reconter en toute bonne
compagnie,''\protect\hypertarget{11_Chapter_Four__THE_FORMS_OF_LOVE.xhtmlux5cux23id_3113}{\protect\hyperlink{23_NOTES.xhtmlux5cux23id_3114}{‡\textsuperscript{7}}}
A noble versesmith composed a lascivious ballade at the request of
Madame de Bourgogne and all of the ladies and maidens of her
court.\textsuperscript{\protect\hypertarget{11_Chapter_Four__THE_FORMS_OF_LOVE.xhtmlux5cux23id_1470}{\protect\hyperlink{23_NOTES.xhtmlux5cux23id_1471}{9}}}

It is clear that things such as these were not regarded as violations of
the high and rigid ideals of honor and propriety. This is a
contradiction that should not be explained by imagining the noble forms
and the high degree of prudishness displayed by the Middle Ages in other
areas to be hypocritical. Just as little as we can call their
shamelessness a saturnalian throwing off of restraints. Still further
off the mark is the impression that the epithalamic obscenities are a
sign of decadence or aristocratic overrefinement. The double meanings,
the indecencies, the lascivious dissimulations are at home in the
epithalamic style because they originated there. They become
understandable if seen against their ethnological background: as the
weakened remnants of the phallic symbolism of primitive culture, as
debased mysteries. What once, at a time when the borders between play
and seriousness had not been drawn by culture, joined the sacredness of
ritual to the exuberance of the joy of life could only be handled, in a
Christian society, as titillating mockery and stimulating jest. In
direct contradiction to piety and \emph{courteoisie} sexual notions
survived in nuptial customs with their vitality intact.

One may, if so inclined, regard the whole comic-erotic genre as wild
sprouts from the stem of the epithalamium---the story, the farce, the
ditty. The link to the source, however, has long been lost: the literary
genre has become independent, the comic effect an end in itself. The
comic art remains the same as that of the epithalamium; it depends
throughout on a symbolic representation of sexual matters or the
depiction of the sexual act in the image of a profession. Almost any
craft, any occupation, yielded its form to erotic metaphor, then just as
well as now. It is obvious that during the fourteenth and fifteenth
centuries, the tournament, the hunt, and music provided the subject
matter for this
purpose.\textsuperscript{\protect\hypertarget{11_Chapter_Four__THE_FORMS_OF_LOVE.xhtmlux5cux23id_1468}{\protect\hyperlink{23_NOTES.xhtmlux5cux23id_1469}{10}}}
Both the
\protect\hypertarget{11_Chapter_Four__THE_FORMS_OF_LOVE.xhtmlux5cux23page_131}{}{}treatment
of love stories in the form of legal disputes and the \emph{arrestz
d'amour} are to be understood from the vantage point of this category of
parody. There was still another domain favored to provide a garb for
sexual matters; this was the church. The Middle Ages were
extraordinarily open in expressing sexual matters in technical
ecclesiastical terminology. In the \emph{Cent nouvelles nouvelles}, the
use of words such as \emph{bénir} or \emph{confesser} in an indecent
sense or the play of words like \emph{saints} or \emph{seins} is
untiringly repeated. In refined examples, however, the
ecclesiastical-erotic allegory becomes a literary form in itself. The
poetic circle around Charles d'Orléans veils the lamentations of love in
the forms of monastic asceticism, liturgy, and martyrdom. Echoing the
recently successful reform of the Franciscan monastic life around 1400,
these poets call themselves \emph{Les amoureux de l'observance}. This is
like an ironic byplay to the sacred seriousness of the \emph{dolce stil
nuovo}. The desecrating tendency is halfway atoned for by the intensity
of the amorous sentiments.

\emph{Ce sont idi les dix commandemens},

\emph{Vray Dieu d'amours .~.~}.
\protect\hypertarget{11_Chapter_Four__THE_FORMS_OF_LOVE.xhtmlux5cux23id_3115}{\protect\hyperlink{23_NOTES.xhtmlux5cux23id_3116}{*\textsuperscript{8}}}

So the poet profanes the Ten Commandments, or here, the oath taken on
the New Testament:

\emph{Lors m'appella, et me fist les mains mettre}

\emph{Sur ung livre, en me faisant promettre}

\emph{Que feroye loyaument mon devoir}

\emph{Des points
d'amour}.\textsuperscript{\protect\hypertarget{11_Chapter_Four__THE_FORMS_OF_LOVE.xhtmlux5cux23id_1466}{\protect\hyperlink{23_NOTES.xhtmlux5cux23id_1467}{11}}}\protect\hypertarget{11_Chapter_Four__THE_FORMS_OF_LOVE.xhtmlux5cux23id_3117}{\protect\hyperlink{23_NOTES.xhtmlux5cux23id_3118}{†\textsuperscript{9}}}

Of a dead lover, he says:

\emph{Et j'ay espoir que brief ou {[}au{]} paradis}

\emph{Des amoureux sera moult hault assis},

\emph{Comme martir et très honnoré
saint}.\protect\hypertarget{11_Chapter_Four__THE_FORMS_OF_LOVE.xhtmlux5cux23id_3119}{\protect\hyperlink{23_NOTES.xhtmlux5cux23id_3120}{‡\textsuperscript{10}}}

\protect\hypertarget{11_Chapter_Four__THE_FORMS_OF_LOVE.xhtmlux5cux23page_132}{}{}And
of his own dead beloved:

\emph{J'ay fait l'obseque de ma dame}

\emph{Dedens le moustier amoureux},

\emph{Et le service pour son ame}

\emph{A chanté Penser doloreux}.

\emph{Mains sierges de soupirs piteux}

\emph{Ont esté en son luminaire},

\emph{Aussi j'ay fait la tombe faire}

\emph{De regrets} .~.~.
\textsuperscript{\protect\hypertarget{11_Chapter_Four__THE_FORMS_OF_LOVE.xhtmlux5cux23id_1464}{\protect\hyperlink{23_NOTES.xhtmlux5cux23id_1465}{12}}}\protect\hypertarget{11_Chapter_Four__THE_FORMS_OF_LOVE.xhtmlux5cux23id_3121}{\protect\hyperlink{23_NOTES.xhtmlux5cux23id_3122}{*\textsuperscript{11}}}

In the candid poem ``L'amant rendu cordelier de l'observance
d'amour,''\protect\hypertarget{11_Chapter_Four__THE_FORMS_OF_LOVE.xhtmlux5cux23id_3123}{\protect\hyperlink{23_NOTES.xhtmlux5cux23id_3124}{†\textsuperscript{12}}}
which describes the admittance of a despairing lover into the Monastery
of the Martyrs of Love, the entire comic effect promised by the
ecclesiastical parody is worked out to the last detail. Does this not
indicate that the erotic, time and again and no matter how perversely,
is drawn towards reestablishing that contact with the holy that it lost
a long time ago?

Eroticism, in order to be culture, had to find at any price a style, a
form, which could hold it in bounds, an expression that could veil it.
And even where it rejected that form and lowered itself from
questionable allegory to realistic and unveiled treatment of sexual
activities, eroticism still remained, though unintentionally, stylized.
An unsophisticated mind may easily mistake the entire genre for erotic
naturalism. This genre, where men never tire and women are always
willing, is just as much a romantic fiction as the most noble courtly
\emph{Minne}. What other than romanticism is the cowardly neglect of all
the natural and social complications of love, the beautiful gloss of
undisturbed pleasure as cover for all the false, self-seeking, and
tragic elements in sexual activities? Here again we encounter that great
cultural motive: the craving for a beautiful life, the need to make life
appear more beautiful than it is revealed by reality. Here the life of
love is forced into a form that conforms to a fantastic desire but does
it now by emphasizing the animal side of humanity. Here is another life
ideal: the ideal of chastelessness.

\protect\hypertarget{11_Chapter_Four__THE_FORMS_OF_LOVE.xhtmlux5cux23page_133}{}{}Reality
is at any time more wretched and cruder than the refined literary ideal
of love sees it, but it is also purer and more ethical than it is
represented by that shallow eroticism which is usually regarded as
naturalistic. Eustache Deschamps, the professional poet, lowers himself
in many ballades, in which he has a speaking part, to the most debased
transgressions. But he is not the real hero of those indecent scenes,
and amongst them we suddenly find a tender poem in which he points out
to his daughter the virtues of her dead
mother.\textsuperscript{\protect\hypertarget{11_Chapter_Four__THE_FORMS_OF_LOVE.xhtmlux5cux23id_1462}{\protect\hyperlink{23_NOTES.xhtmlux5cux23id_1463}{13}}}

As a source of literature and culture the whole epithalamic genre, with
all its facets and ramifications, remains of secondary importance. It
has as its theme full and complete satisfaction. It is overtly erotic.
But that which can serve to shape and adorn life is the covertly erotic,
whose theme is the possibility of satisfaction, the promise, the
longing, the deprivation, anticipated happiness. Here, the greatest
satisfaction is found in that which is unexpressed, disguised by the
thin veils of expectation. Because of this, indirect eroticism is much
more viable and embraces a much wider sphere of life. And it knows love
not only in its major key, or in its laughing mask, but is also capable
of transforming the pain of love into beauty and has, therefore, an
infinitely higher value for life. It can embrace the ethical elements of
faithfulness, of courage, of noble gentility, and being thus bonded with
virtues in addition to love, may strive for the ideal.

Completely in agreement with the the general spirit of the later Middle
Ages, which desired all thought to be captured in the most detailed
images and systems, the \emph{Roman de la rose} succeeded in bestowing
on erotic culture in its entirety such a colorful, self-contained, and
rich form that it was like a treasury of profane liturgy, doctrine, and
legend. The hermaphroditism of the \emph{Roman de la rose}, a work of
two authors of vastly different nature and perception, rendered it even
more usable as a bible of erotic culture. Texts for the most diverse
usages can be found in it.

Guillaume de Lorris, the first poet, paid homage to all the old courtly
ideals. The graceful plan and the gay, charming imagination of the work
have to be attributed to him. The theme of the dream frequently
reoccurs. Early on, the poet sees himself awake on a May morning so that
he might hear the nightingale and the lark. His path leads him along a
stream to the wall of the mysterious garden of love. On the wall he sees
the images of hatred, betrayal,
\protect\hypertarget{11_Chapter_Four__THE_FORMS_OF_LOVE.xhtmlux5cux23page_134}{}{}perfidity,
rapaciousness, greed, melancholy, false piety, poverty, envy, and age.
The anti-courtly qualities. But Dame Oiseuse (laziness), the friend of
Déduit (amusement), opens the gate for him. Inside, Liesse (gaiety)
leads the dance. The God of Love dances with Beauty in the round dance,
and Wealth, Charity, Frankness (Franchise), Courtly Manners
(Courteoisie), and Youth take part. While the poet is absorbed in
admiration of the Rosebud that he has noticed near the Narcissus
Fountain, the God of Love shoots him with his arrows: Beauté, Simplesse,
Courteoisie, Compagnie, and Beau-Semblant. The poet declares himself to
be the liegeman \emph{(homme lige)} of Love. Amour locks his heart with
a key and explains the Commandments of Love, the pains of Love and its
comforts \emph{(biens)}; these last are Espérance, Doux-Penser,
Doux-Parler, and Doux-Regard.

Bel-Accueil, the son of Courteoisie, summons him to the Rose, but at
that momemt the guardians of the Rose appear, Danger, Male-Bouche, Peur,
and Honte, and drive him away. Now the complications begin. Raison
descends from his high tower to plead with the lover; Ami consoles him,
and Venus turns all her charms against Chasteté. Franchise and Pitié
bring him back to Bel-Accueil, who permits him to kiss the Rose.
However, Male-Bouche spreads the alarm, Jalousie comes running, and a
strong wall is built around the Rose. Bel-Accueil is imprisoned in a
tower. Danger and his servants guard the gates. With the lament of the
lover, the work of Guillaume de Lorris comes to an end.

Then, most likely a good time later, Jean de Meun {[}also known as Jean
Clopinel{]} enters with a much more voluminous sequel. The further
course of events---the attack and conquest of the castle of the Rose by
Amour and all his allies, that is the courtly virtues assisted by Bien
Celer and Faux-Semblant---nearly drowns in a flood of diversions,
contemplations, and narrations by means of which the second poet turns
the work into a veritable encyclopedia. But most importantly, here
speaks a mind so unself-conscious, so coolly skeptical and cynically
hardened, such as the Middle Ages rarely produced. And at the same time,
a mind with a command of the French language equaled by few. The naive
and easy idealism of Guillaume de Lorris is tarnished by the negating
spirit of Jean de Meun. De Meun did not believe in ghosts or magicians,
in true love or feminine honor, but he had a sense of pathological
problems
\protect\hypertarget{11_Chapter_Four__THE_FORMS_OF_LOVE.xhtmlux5cux23page_135}{}{}and
he put in the mouths of Venus, Nature, and Genius the most daring
defense of life's sensual urges.

When Amour fears that he and his army may be defeated, he dispatches
Franchise and Doux-Regard to his mother, Venus, who answers his call and
comes to him riding her chariot drawn by doves. Told by Amour how things
stand, she vows that she will no longer tolerate any woman remaining
chaste and urges Amour to take the same oath in respect to men. He does,
and the entire army takes the oath with him.

In the meantime, Nature is at her forge, busy with her task of
preserving the species in her eternal struggle with Death. She bitterly
complains that of all her creatures only humanity disobeys her
commandment and refrains from procreation. On her orders, Genius, her
priest, after a long confession during which she explains her works to
him, joins the army of Love to impose Nature's curse on all those who
defy her commandments. Amour dresses Genius in a sacramental gown, with
a ring, a staff, and a miter; Venus, laughing loudly, puts a burning
candle in his hand

\emph{Qui ne fu pas de cire
vierge\protect\hypertarget{11_Chapter_Four__THE_FORMS_OF_LOVE.xhtmlux5cux23id_3125}{\protect\hyperlink{23_NOTES.xhtmlux5cux23id_3126}{*\textsuperscript{13}}}}

The excommunication begins with a rejection of virginity, the audacious
symbolism of which amounts to a wondrous mysticism. Hell for those who
fail to observe the laws of Nature and Love! For the others, the
flowering fields where the Son of the Virgin tends his white sheep that
graze in eternal bliss on the flowers and plants that bloom there to all
eternity.

After Genius has tossed the candle, whose flame sets all the world on
fire, into the fortress, the final battle for the tower begins. Venus
herself tosses her torch, Honte and Peur flee, and Bel-Accueil allows
the lover to pluck the Rose.

Here, anew, the sexual motive is with full consciousness placed in the
center of things and dressed in such artificial mystery, indeed, with so
much sanctity, that a more pronounced challenge to the Christian ideal
of life is not possible. In its perfectly pagan tendency, the
\emph{Roman de la rose} may be regarded as a step towards the
Renaissance. In its external form, it is seemingly genuinely
medi\protect\hypertarget{11_Chapter_Four__THE_FORMS_OF_LOVE.xhtmlux5cux23page_136}{}{}eval.
What can be more medieval than the personification of emotional
reactions and the circumstances of love taken to their extremes? The
figures of the \emph{Roman de la rose}, Bel-Accueil, Doux-Regard,
Faux-Semblant, Male-Bouche, Danger, Honte, Peur, are at the same level
as the truly medieval representations of the virtues and sins in human
form. They are allegories, or something more, half believed in
mythologems. Where is the division between these representations and the
nymphs, satyrs, and ghosts that awaken to new life in the Renaissance?
They are taken from another sphere, but their value to the imagination
is the same. The external character of the figures of the \emph{Rose}
are occasionally reminiscent of the fantastic flowery figures of
Botticelli.

Here the dream of love was depicted in a form that was both artificial
and passionate. The detailed allegory satisfied all the needs of the
medieval imagination. Without the personifications, the mind would not
have been able either to express or to follow the shifts in emotion. The
whole colorful fabric and elegant lines of this incomparable puppet show
were necessary in order to form a conceptual system of love that people
could use to communicate with one another. The figures of Danger, Nouvel
Penser, Male-Bouche were used like the handy terms of a scientific
psychology. The basic theme carried throughout the poem is in a
passionate key since in the place of the pale service to a married woman
who was elevated to the clouds by the troubadours as the unreachable
object of their longing, there now comes again the most natural erotic
motif: the potent attraction of the secret of virginity, symbolized as
the Rose, which can only be won by art and endurance.

In theory, love in the \emph{Roman de la rose} remained courtly and
noble. The Garden of the Joy of Life is open only to the chosen and only
through Love. Whoever wishes to enter must be free of hate,
unfaithfulness, perfidy, rapaciousness, greed, envy, old age, and
hypocrisy. The positive virtues, however, which must be mustered against
all these prove the ideal is no longer ethical as it was in courtly
\emph{Minne}, but rather is purely aristocratic. The virtues are:
carefreeness, receptability to enjoyment, gaiety of spirit, love,
beauty, wealth, gentleness, freedom of spirit \emph{(franchise)}, and
\emph{Courteoisie}. These are no longer changes in the lover who is
ennobled by the reflected glory of the beloved, but are the appropriate
means used to win her. And it is no longer the veneration of the woman,
misguided as it may have been, which inspires the work, but rather,
\protect\hypertarget{11_Chapter_Four__THE_FORMS_OF_LOVE.xhtmlux5cux23page_137}{}{}at
least in the case of the second poet, Jean Clopinel, the mocking
contempt of her weaknesses, a contempt that has its sources in the
sensual nature of this mode of love itself.

But in spite of its strong hold on the minds of the time, the
\emph{Roman de la rose} was unable to completely replace the older
conception of love. Next to the glorification of flirtation, the idea of
pure, knightly, faithful, and self-denying love held its own because it
was an essential component of the knightly life ideal. It became a
subject of courtly debate, in that colorful circle of abundant,
aristocratic life around the French king and his uncles of Berry and
Burgundy, which idea of love should have priority in the life of the
true nobleman: that of genuine \emph{Courteoisie}, with its yearning
faithfulness and service dedicated in honor of a lady, or that of the
\emph{Roman de la rose}, where faithfulness was only a means in the
service of the hunt for a woman. The noble knight Boucicaut and his
comrades had made themselves the advocates of knightly faithfulness
during a journey to the East in 1388 and had passed the time in the
composition of the \emph{Livre des cent ballades}. The decision between
flirtation and faithfulness they left to the \emph{beaux-esprits} of the
courts.

The words with which Christine de Pisan entered the fray a few years
later sprang from a deeper seriousness. This courageous defender of
female honor and female rights turned to the God of Love in a poetic
letter that contained the complaint of womankind against all the
betrayal and dishonor by the world of
men.\textsuperscript{\protect\hypertarget{11_Chapter_Four__THE_FORMS_OF_LOVE.xhtmlux5cux23id_1460}{\protect\hyperlink{23_NOTES.xhtmlux5cux23id_1461}{14}}}
She rejected with outrage the lessons of the \emph{Roman de la rose}. A
few agreed with her, but the work of Jean de Meun continued to have its
share of passionate admirers and defenders. In the ensuing literary
controversy a number of attackers and defenders had their say. The
champions who upheld the \emph{Rose} were of no mean stature. Many wise,
scientific, highly learned men---we are assured by the provost of Lille,
Jean de Montreuil---held the \emph{Roman de la rose} so highly as to pay
it almost divine reverence \emph{(paene ut colerent)} and would rather
have rent their shirt than that book!

It is not easy for us to understand the intellectual and emotional
conditions that gave rise to this defense. It was not frivolous court
pages but earnest high-ranking officials, some even clerics such as the
above-mentioned provost of Lille, Jean de Montreuil, secretary to the
Dauphin (later duke of Burgundy), who corresponded about this issue with
his friends Gontier and Pierre Col in poetic letters written in Latin
and who urged others to take up the burden of
\protect\hypertarget{11_Chapter_Four__THE_FORMS_OF_LOVE.xhtmlux5cux23page_138}{}{}defending
Jean de Meun. What is most peculiar is that this circle that appointed
itself defenders of that colorful, abundant medieval work is the same in
which the first growth of French humanism was cultivated. Jean de
Montreuil is the author of a large number of Ciceronian letters full of
humanist attitudes, humanist rhetoric, and humanist vanity. He and his
friends Gontier and Pierre Col carry on a correspondence with that
earnest, reform-minded theologian Nicolas de Clémanges.

Jean de Montreuil was certainly serious about his literary point of
view. The more I study the mysteries of importance and the importance of
the mysteries of this deep and famous work of the master Jean de Meun,
he wrote to an unknown legal scholar who had attacked the \emph{Rose},
the more I am astonished by your disapproval.---Until his last breath he
will defend the book, and there are many who will do the same with their
pens, voices, and
hands.\textsuperscript{\protect\hypertarget{11_Chapter_Four__THE_FORMS_OF_LOVE.xhtmlux5cux23id_1458}{\protect\hyperlink{23_NOTES.xhtmlux5cux23id_1459}{15}}}

In order to prove that this controversy was more than only a part of the
great social game of courtly life, I avail myself, finally, of a man who
said that when he spoke he did so for the sake of the highest morality
and the purest doctrine: the famous theologian and Chancellor of the
University of Paris, Jean de Gerson. From his library, on the evening of
May 18, 1402, he wrote a tract against the \emph{Roman de la rose}. The
tract is an answer to the attack on an earlier treatise by Gerson
launched by Pierre
Col,\textsuperscript{\protect\hypertarget{11_Chapter_Four__THE_FORMS_OF_LOVE.xhtmlux5cux23id_1456}{\protect\hyperlink{23_NOTES.xhtmlux5cux23id_1457}{16}}}
and even that had not been the first writing of Gerson on the subject of
the \emph{Rose}. The book seemed to him to be a dangerous plague, the
source of all immorality; he intended to attack it at every opportunity.
Repeatedly he mounted a campaign against the corrupting influence ``du
vicieux romant de la
rose.''\textsuperscript{\protect\hypertarget{11_Chapter_Four__THE_FORMS_OF_LOVE.xhtmlux5cux23id_1454}{\protect\hyperlink{23_NOTES.xhtmlux5cux23id_1455}{17}}}\protect\hypertarget{11_Chapter_Four__THE_FORMS_OF_LOVE.xhtmlux5cux23id_3127}{\protect\hyperlink{23_NOTES.xhtmlux5cux23id_3128}{*\textsuperscript{14}}}
If he had a copy of the \emph{Rose}---he said---which was the only one
in the world and worth a thousand pounds, he would rather burn it than
sell it and turn it over to the public.

Gerson took the form of his argument from his opponent: an allegorical
vision. When he awakes one morning, he feels his heart flee from him,
``moyennant les plumes et les eles de diverses pensees, d'un lieu en
autre jusques à la court saincte de
crestienté.''\protect\hypertarget{11_Chapter_Four__THE_FORMS_OF_LOVE.xhtmlux5cux23id_3129}{\protect\hyperlink{23_NOTES.xhtmlux5cux23id_3130}{†\textsuperscript{15}}}
There he meets Justice, Conscience, and Knowledge and hears how
\protect\hypertarget{11_Chapter_Four__THE_FORMS_OF_LOVE.xhtmlux5cux23page_139}{}{}Chasteté
accuses Fol amoureux (namely, Jean de Meun) of having banished her and
all her disciples from the earth. Her ``bonnes gardes'' have been
represented as the evil figures of the \emph{Rose}: ``Honte, Paour, et
Dangier le bon portier, qui ne oseroit ne daigneroit ottroyer neïs (pas
même) un vilain baisier ou dissolu regart ou ris attraiant ou parole
legiere.''\protect\hypertarget{11_Chapter_Four__THE_FORMS_OF_LOVE.xhtmlux5cux23id_3131}{\protect\hyperlink{23_NOTES.xhtmlux5cux23id_3132}{*\textsuperscript{16}}}
Chastity continues to direct a number of charges against Fol amoureux:
that he spreads, with the help of the damnable Old
Woman,\textsuperscript{\protect\hypertarget{11_Chapter_Four__THE_FORMS_OF_LOVE.xhtmlux5cux23id_1452}{\protect\hyperlink{23_NOTES.xhtmlux5cux23id_1453}{18}}}
the doctrine ``comment toutes jeunes filles doivent vendre leurs corps
tost et chierement sans paour et sans vergoigne, et qu'elles ne
tiengnent compte de decevoir ou
parjurer,''\protect\hypertarget{11_Chapter_Four__THE_FORMS_OF_LOVE.xhtmlux5cux23id_3133}{\protect\hyperlink{23_NOTES.xhtmlux5cux23id_3134}{†\textsuperscript{17}}}
He mocks marriage and the monastic life; he turns all imaginations to
carnal desire, and, worst of all, has Venus, Nature, and even lady
Raison mingle the notions of Paradise and the Christian mysteries with
those of sensual enjoyment.

This is indeed where danger was lurking. The great work with its linking
of sensuality, derisive cynicism, and elegant symbolism awakened in the
mind a sensuous mysticism that was bound to appear to the serious
theologian as an abyss of sinfulness. How daring Gerson's adversary had
been in his
claims!\textsuperscript{\protect\hypertarget{11_Chapter_Four__THE_FORMS_OF_LOVE.xhtmlux5cux23id_1450}{\protect\hyperlink{23_NOTES.xhtmlux5cux23id_1451}{19}}}
Only the Fol amoureux himself can judge the value of unrestrained
passion. Those who do not know it see it only in a mirror and as a dark
mystery. That is to say that he borrowed for earthly love the holy word
from the letter to the Corinthians so that he could speak of earthly
love as the mystic speaks of his ecstasy! He dared to claim that
Solomon's high song had been composed to praise Pharaoh's daughter.
Those who had denounced the book of the \emph{Rose} had bent their knee
before Baal since Nature did not intend that one man would be enough for
a woman, and the genius of nature is God. Verily, he even dares to
misuse Luke
2:23\textsuperscript{\protect\hypertarget{11_Chapter_Four__THE_FORMS_OF_LOVE.xhtmlux5cux23id_1448}{\protect\hyperlink{23_NOTES.xhtmlux5cux23id_1449}{20}}}
to prove with the help of the gospel itself that formerly the female
sexual organ, the Rose of the novel, had been sacred. And, fully
confident in all these blasphemies, he calls on the defenders of this
work, on a number of witnesses, and threatens that Gerson himself will
fall victim to an irrational love as had happened to other theologians
before him.

\protect\hypertarget{11_Chapter_Four__THE_FORMS_OF_LOVE.xhtmlux5cux23page_140}{}{}The
power of the \emph{Roman de la rose} was not broken by Gerson's attack.
In 1444 a canon of Liseux, Estienne Legris, offered Jean Lebégue,
secretary of the chamber of accounts in Paris, a \emph{Répertoire du la
roman de la Rose} that he had
written.\textsuperscript{\protect\hypertarget{11_Chapter_Four__THE_FORMS_OF_LOVE.xhtmlux5cux23id_1446}{\protect\hyperlink{23_NOTES.xhtmlux5cux23id_1447}{21}}}
As late as the end of the fifteenth century Jean Molinet could claim
that quotations from the \emph{Rose} were as familiar as common
proverbs.\textsuperscript{\protect\hypertarget{11_Chapter_Four__THE_FORMS_OF_LOVE.xhtmlux5cux23id_1444}{\protect\hyperlink{23_NOTES.xhtmlux5cux23id_1445}{22}}}
He felt called to offer a moralizing commentary on the entire work in
which the well at the beginning of the poem becomes the symbol of
baptism, the nightingale calling to love becomes the voice of the
preacher and theologian, and the Rose, Jesus himself. Clement Marot made
a modernized version of the \emph{Rose}, and Ronsard himself still uses
allegorical figures such as Belacueil, Fausdanger,
etc.\textsuperscript{\protect\hypertarget{11_Chapter_Four__THE_FORMS_OF_LOVE.xhtmlux5cux23id_1442}{\protect\hyperlink{23_NOTES.xhtmlux5cux23id_1443}{23}}}

While dignified scholars fought their literary battles, the aristocrats
took the controversy as a welcome occasion for staging entertaining
festivities and pompous amusements. Boucicaut, who was praised by
Christine de Pisan for his defense of the old idea of knightly
faithfulness, may have found in her work the inspiration for the
founding of his Ordre de l'écu verd à la dame blanche for the defense of
unfortunate women. But he could not compete with the Duke of Burgundy,
and his order immediately found itself overshadowed by the grandiose
inception of the Cour d'amours, which was founded on Feburary 14, 1401,
in the Hotel d'Artois in Paris. The Cour d'amours was a splendidly
furnished literary salon. Philip the Bold, duke of Burgundy, that crafty
old statesman, whom one would not expect to have been interested in such
matters, had requested that the king found the Cour d'amours in order to
distract people from the epidemic of plague then visited upon Paris,
``pour passer partie du tempz plus gracieusement et affin de trouver
esveil de nouvelle
joye.''\textsuperscript{\protect\hypertarget{11_Chapter_Four__THE_FORMS_OF_LOVE.xhtmlux5cux23id_1440}{\protect\hyperlink{23_NOTES.xhtmlux5cux23id_1441}{24}}}\protect\hypertarget{11_Chapter_Four__THE_FORMS_OF_LOVE.xhtmlux5cux23id_3135}{\protect\hyperlink{23_NOTES.xhtmlux5cux23id_3136}{*\textsuperscript{18}}}
The Cour d'amours was based upon the virtues of humility and
faithfulness, ``à l'onneur, loenge et recommandacion et service de
toutes dames et damoiselles.'' The numerous members were graced with the
most glorious titles: both the founders and Charles VI were \emph{Grands
conservateurs}; among the \emph{conservateurs} were John the Fearless,
his brother Anton of Brabant, and his younger son Philip. There was a
Prince d'amour, Pierre de Hauteville from Hennegouw; then there were
Ministres, Auditeurs, Chevaliers d'honneur, Conseillers, Chevaliers
trésoriers,
\protect\hypertarget{11_Chapter_Four__THE_FORMS_OF_LOVE.xhtmlux5cux23page_141}{}{}Grands
Veneurs, Ecuyers d'amour, Maîtres des requêtes, Secrétaires; in short,
the entire apparatus of the court and the government was imitated
therein. Princes and prelates could be found in it, along with burghers
and the lower clergy. The functions and ceremonies were minutely
regulated. It was like a Toastmasters' club. The members were given the
task of responding with refrains in all the existing verse forms:
``ballades couronnés ou chapelées,'' chansons, sirventois, complaintes,
rondeaux, lais, virelais, etc. Debates were to be carried out, ``en
forme d'amoureux procès, pour différentes opinions soustenir.''
\protect\hypertarget{11_Chapter_Four__THE_FORMS_OF_LOVE.xhtmlux5cux23id_3137}{\protect\hyperlink{23_NOTES.xhtmlux5cux23id_3138}{*\textsuperscript{19}}}
Ladies would award the prizes, and it was forbidden to compose verses
that dishonored the female gender.

How truly Burgundian this pompous endeavor, solemn forms for light
amusement. It is striking, and yet understandable, that the court
preserves the strict ideal of noble fidelity. But if we were to suppose
that the nearly seven hundred members of which we know during the
fifteen years we hear of the existence of the society were all like
Boucicaut, honest followers of Christine de Pisan and therefore enemies
of the \emph{Roman de la rose}, we would be in conflict with the facts.
Whatever is known of the behavior of Anton of Brabant and other high
officials of the order renders them unsuitable to be defenders of female
honor. One of the members, a certain Regnault d'Azincourt, is the
instigator of a failed attempt to kidnap, in the grand style, a young
merchant widow, using twenty horses and bringing a priest with
him.\textsuperscript{\protect\hypertarget{11_Chapter_Four__THE_FORMS_OF_LOVE.xhtmlux5cux23id_1438}{\protect\hyperlink{23_NOTES.xhtmlux5cux23id_1439}{25}}}
Another member, the count of Tonnerre, is guilty of a similar offense.
And, just to prove conclusively that the order was nothing but a
beautiful social game, the adversaries of Christine de Pisan, in the
literary battle over the \emph{Roman de la rose}, themselves were
members: Jean de Montreuil and Gontier and Pierre
Col.\textsuperscript{\protect\hypertarget{11_Chapter_Four__THE_FORMS_OF_LOVE.xhtmlux5cux23id_1436}{\protect\hyperlink{23_NOTES.xhtmlux5cux23id_1437}{26}}}

The forms of love of that time can be learned from literature, but we
have to try to understand how the forms of love operated in life itself.
A complete system of prescribed forms was available to fill a young life
with aristocratic conventions. How many signs and symbols of love have
later centuries gradually surrendered! In place of Amour alone there was
the entire peculiar personal mythology of the \emph{Roman de la rose}.
Doubtlessly, Bel-Accueil, Doux-Penser, Faux-Semblant, and the others
also lived in the
\protect\hypertarget{11_Chapter_Four__THE_FORMS_OF_LOVE.xhtmlux5cux23page_142}{}{}imagination
outside of literary works. There were also the whole range of tender
meanings of colors in clothing, in flowers, and in decorations. Color
symbolism, which has not yet been entirely forgotten, had a very
important place in the life of love in the Middle Ages. Those who could
not understand it found a guide in \emph{Le blason des couleurs} that
was written around 1458 by the Herald Sizilien, turned into verse in the
sixteenth century and ridiculed by Rabelais, not so much because he
despised the subject matter, but because he had given some thought to
writing about it
himself.\textsuperscript{\protect\hypertarget{11_Chapter_Four__THE_FORMS_OF_LOVE.xhtmlux5cux23id_1435}{\protect\hyperlink{23_NOTES.xhtmlux5cux23page_413}{27}}}

When Guillaume de Machaut sees his unknown beloved for the first time,
he is delighted that she is wearing a hood of a sky-blue material,
trimmed with green parrots, to go with her white dress, because green is
the color of new love and blue that of love which is true. Later, when
the beautiful time of his poetic love is over, he dreams that her
likeness which hangs over his bed has its head turned away and is
completely dressed in green, ``qui nouvelleté
signifie.''\protect\hypertarget{11_Chapter_Four__THE_FORMS_OF_LOVE.xhtmlux5cux23id_3139}{\protect\hyperlink{23_NOTES.xhtmlux5cux23id_3140}{*\textsuperscript{20}}}
He composes a ballade of reproach:

\emph{En lieu de bleu, dame, vous vestez
vert}.\textsuperscript{\protect\hypertarget{11_Chapter_Four__THE_FORMS_OF_LOVE.xhtmlux5cux23id_1433}{\protect\hyperlink{23_NOTES.xhtmlux5cux23id_1434}{28}}}\protect\hypertarget{11_Chapter_Four__THE_FORMS_OF_LOVE.xhtmlux5cux23id_3141}{\protect\hyperlink{23_NOTES.xhtmlux5cux23id_3142}{†\textsuperscript{21}}}

Rings, veils, all the treasures and little gifts of love have their
special functions and their mysterious devices and emblems that
frequently degrade into artfully contrived rebuses. The Dauphin went
into battle in 1414 with a banner that had on it in gold a ``K,'' a swan
\emph{(cygne)}, and an ``L,'' which stood for the name of a
lady-in-waiting called Cassinelle who served his mother
Isabeau.\textsuperscript{\protect\hypertarget{11_Chapter_Four__THE_FORMS_OF_LOVE.xhtmlux5cux23id_1431}{\protect\hyperlink{23_NOTES.xhtmlux5cux23id_1432}{29}}}
Rabelais, a century later, mocks the ``glorieux de court de
transporteurs de
noms,''\protect\hypertarget{11_Chapter_Four__THE_FORMS_OF_LOVE.xhtmlux5cux23id_3143}{\protect\hyperlink{23_NOTES.xhtmlux5cux23id_3144}{‡\textsuperscript{22}}}
who in their mottoes represent \emph{espoir} by a sphere, \emph{peine}
by \emph{pennes d'oiseaux}, and \emph{melancholie} by a columbine
\emph{(ancholie)}.\textsuperscript{\protect\hypertarget{11_Chapter_Four__THE_FORMS_OF_LOVE.xhtmlux5cux23id_1429}{\protect\hyperlink{23_NOTES.xhtmlux5cux23id_1430}{30}}}
Coquillart speaks of a

\emph{Mignonne de haulte entreprise}

\emph{Qui porte des devises à
tas}.\textsuperscript{\protect\hypertarget{11_Chapter_Four__THE_FORMS_OF_LOVE.xhtmlux5cux23id_1427}{\protect\hyperlink{23_NOTES.xhtmlux5cux23id_1428}{31}}}\protect\hypertarget{11_Chapter_Four__THE_FORMS_OF_LOVE.xhtmlux5cux23id_3145}{\protect\hyperlink{23_NOTES.xhtmlux5cux23id_3146}{§\textsuperscript{23}}}

Then there were for keenly infatuated minds games such as \emph{Le
\protect\hypertarget{11_Chapter_Four__THE_FORMS_OF_LOVE.xhtmlux5cux23page_143}{}{}roi
qui ne ment, Le chastel d'amours, Ventes d'amour}, and \emph{Jeux à
vendre.\protect\hypertarget{11_Chapter_Four__THE_FORMS_OF_LOVE.xhtmlux5cux23id_3147}{\protect\hyperlink{23_NOTES.xhtmlux5cux23id_3148}{*\textsuperscript{24}}}}
The girl would call out the name of a flower or something else. The boy
had to respond with a rhyme that contained a compliment:

\emph{Je vous vens la passerose},

---\emph{Belle, dire ne vous ose}

\emph{Comment Amours vers vous me tire},

\emph{Si l'apercevez tout sanz
dire}.\textsuperscript{\protect\hypertarget{11_Chapter_Four__THE_FORMS_OF_LOVE.xhtmlux5cux23id_1425}{\protect\hyperlink{23_NOTES.xhtmlux5cux23id_1426}{32}}}\protect\hypertarget{11_Chapter_Four__THE_FORMS_OF_LOVE.xhtmlux5cux23id_3149}{\protect\hyperlink{23_NOTES.xhtmlux5cux23id_3150}{†\textsuperscript{25}}}

The \emph{chastel d'amours} was such a question-and-answer game based on
the figures of the \emph{Roman de la rose}:

\emph{Du chastel d'Amours vous demant}:

\emph{Dites le premier fondement!}

---\emph{Amer loyaument}.

\emph{Or me nommez le mestre mur}

\emph{Qui joli le font, fort et seur!}

---\emph{Celer sagement}.

\emph{Dites moy qui sont li crenel},

\emph{Les fenestres et li carrel!}

---\emph{Regart atraiant}.

\emph{Amis, nommez moy le portier!}

---\emph{Dangier mauparlant}.

\emph{Qui est la clef qui le puet deffermer?}

---\emph{Prier
courtoisement}.\textsuperscript{\protect\hypertarget{11_Chapter_Four__THE_FORMS_OF_LOVE.xhtmlux5cux23id_1423}{\protect\hyperlink{23_NOTES.xhtmlux5cux23id_1424}{33}}}\protect\hypertarget{11_Chapter_Four__THE_FORMS_OF_LOVE.xhtmlux5cux23id_3151}{\protect\hyperlink{23_NOTES.xhtmlux5cux23id_3152}{‡\textsuperscript{26}}}

A great part of courtly conversation had been taken up since the time of
the troubadours by questions of the casuistry of love. It may be
regarded as raising nosiness and slander to a literary form. Mealtime at
the court of Louis d'Orléans was enlivened by ``beaulx livres, dits,
ballads'' and ``demandes
gracieuses.''\textsuperscript{\protect\hypertarget{11_Chapter_Four__THE_FORMS_OF_LOVE.xhtmlux5cux23id_1421}{\protect\hyperlink{23_NOTES.xhtmlux5cux23id_1422}{34}}}
These last were
\protect\hypertarget{11_Chapter_Four__THE_FORMS_OF_LOVE.xhtmlux5cux23page_144}{}{}preferably
put before poets for a decision. A company of ladies and gentlemen came
to Machaut with a number of ``partures d'amour et de ses
aventures.''\textsuperscript{\protect\hypertarget{11_Chapter_Four__THE_FORMS_OF_LOVE.xhtmlux5cux23id_1419}{\protect\hyperlink{23_NOTES.xhtmlux5cux23id_1420}{35}}}\protect\hypertarget{11_Chapter_Four__THE_FORMS_OF_LOVE.xhtmlux5cux23id_3153}{\protect\hyperlink{23_NOTES.xhtmlux5cux23id_3154}{*\textsuperscript{27}}}
He had defended, in his \emph{Jugement d'amour}, the thesis that a lady
who lost her lover to death should be less pitied than the lover whose
beloved is unfaithful. Every love affair was judged in this way,
according to strict norms.---``Beau sire, what would you prefer, that
evil things were said about your beloved and you found her to be
faithful, or that people praised her and you found her to be
unfaithful?''---Whereupon the highly formal concept of honor and the
strict duty of the lover to guard the public honor of the beloved,
required the answer: ``Dame, j'aroie plus chier que j'en oïsse bien dire
et y trouvasse
mal.''\protect\hypertarget{11_Chapter_Four__THE_FORMS_OF_LOVE.xhtmlux5cux23id_3155}{\protect\hyperlink{23_NOTES.xhtmlux5cux23id_3156}{†\textsuperscript{28}}}
If a lady is neglected by her first lover, does she act unfaithfully if
she takes a second, who is more forthcoming? May a knight, who has
abandoned all hope of ever seeing his lady, who is kept under lock and
key by a jealous husband, finally seek a new love? If a knight turning
from his beloved to a lady of higher birth, and spurned by the latter,
returns to the former, may her honor permit her to forgive
him?\textsuperscript{\protect\hypertarget{11_Chapter_Four__THE_FORMS_OF_LOVE.xhtmlux5cux23id_1417}{\protect\hyperlink{23_NOTES.xhtmlux5cux23id_1418}{36}}}
It is only a small step from this kind of casuistry to dealing with
questions of love entirely in a legal format as is done by Martial
d'Auvergne in the \emph{Arrestz d'amour}.

All of these conventions of love are known only through the way they are
reflected in literature, but they were at home in real life. The code of
courtly terms, rules, and forms did not seek only to turn conventions
into poems, but also to apply them in aristocratic life or, at least, in
conversation. It is, however, difficult to sense the life of that time
behind the veil of poetry, because even where real love is described as
exactly as possible, the description is made under the influence of the
technical apparatus of the readymade illusion of the conventions of
love, and presented within the format of the literary stylization. Such
is the case in Guillaume de Machaut's fourteenth-century story, overly
long and tedious, of the poetic love of the aged poet and a certain
Bettina,\textsuperscript{\protect\hypertarget{11_Chapter_Four__THE_FORMS_OF_LOVE.xhtmlux5cux23id_1415}{\protect\hyperlink{23_NOTES.xhtmlux5cux23id_1416}{37}}}
called \emph{Le livre de voir-dit} (``The Book of the True
Event'').\textsuperscript{\protect\hypertarget{11_Chapter_Four__THE_FORMS_OF_LOVE.xhtmlux5cux23id_1413}{\protect\hyperlink{23_NOTES.xhtmlux5cux23id_1414}{38}}}
He must have been nearly sixty years old when Péronelle
d'Armentières,\textsuperscript{\protect\hypertarget{11_Chapter_Four__THE_FORMS_OF_LOVE.xhtmlux5cux23id_1411}{\protect\hyperlink{23_NOTES.xhtmlux5cux23id_1412}{39}}}
about eighteen and of a noble family of Champagne, sent him her first
\protect\hypertarget{11_Chapter_Four__THE_FORMS_OF_LOVE.xhtmlux5cux23page_145}{}{}rondel.
He was very famous and knew nothing of her. Nevertheless, she offered
him her heart and requested that he begin a poetic correspondence about
love with her. The poor poet, sickly, blind in one eye and troubled by
gout, is immediately inflamed. He answers her rondel, and an exchange of
letters and poems begins. Péronelle is proud of this literary
connection; at first she makes no effort to keep it a secret. She
insists that the poems tell the whole truth about their love and that
her letters and poems be included in his account. He fulfills these
requests with pleasure: ``je feray à vostre gloire et loenge, chose dont
il sera bon
mémoire.''\textsuperscript{\protect\hypertarget{11_Chapter_Four__THE_FORMS_OF_LOVE.xhtmlux5cux23id_1409}{\protect\hyperlink{23_NOTES.xhtmlux5cux23id_1410}{40}}}
``Et mon très-dous cuer,'' he writes to her, ``vous estes courrecié de
ce nous avons si tart commencié? {[}How could she have started
earlier?{]} Par Dieu aussi suis-je (with more justification); mais
ves-cy le remède; menons si bonne vie que nous porrons, en lieu et en
temps, que nous recompensons le temps que nous avons perdu; et qu'on
parle de nos amours jusques à cent ans cy après, en tout bien et en tout
honneur; car s'il y avoit mal, vous le celeriés à Dieu, se vous
poviés.''\textsuperscript{\protect\hypertarget{11_Chapter_Four__THE_FORMS_OF_LOVE.xhtmlux5cux23id_1407}{\protect\hyperlink{23_NOTES.xhtmlux5cux23id_1408}{41}}}\protect\hypertarget{11_Chapter_Four__THE_FORMS_OF_LOVE.xhtmlux5cux23id_2311}{\protect\hyperlink{23_NOTES.xhtmlux5cux23id_2312}{*\textsuperscript{29}}}

But what was within the bounds of an honorable love in those days we
learn from the narrative passages inserted by Machaut to string the
letters and poems together. The poet receives her painted portrait,
which he has requested, and he venerates it like his god on earth. He
looks forward to their first encounter with great trepidation because of
his physical handicaps. His joy knows no bounds when his young beloved
is not horrified by his appearance. She lies down under a cherry tree to
sleep, or pretend to sleep, in his lap. She grants him greater favors. A
pilgrimage to St. Denis and the Foire du Lendit offers the opportunity
to spend a few days together. By noon of one day, the party is dead
tired because of the throngs and the heat; it is the middle of June.
They find shelter in the overcrowded town with a man who offers them a
room with two beds. In the darkened room, Péronelle's sister-in-law lies
down for a nap. Péronelle and her chambermaid lie down on the
\protect\hypertarget{11_Chapter_Four__THE_FORMS_OF_LOVE.xhtmlux5cux23page_146}{}{}other
bed. She makes the shy poet lie down between them; he lies there still
as death for fear of disturbing them. When she awakes, she orders him to
embrace her. As the end of their short journey approaches and she
becomes aware of his sadness, she permits him to come to her and make
his farewells. Though he continues to speak of ``onneur'' and
``onnesté'' on this occasion too, his rather blunt account does not make
it clear what else she could have denied him. She gives him the small
golden key of her honor, her treasure, to guard carefully, but what was
left to guard should perhaps be understood as her reputation before her
fellowmen.\textsuperscript{\protect\hypertarget{11_Chapter_Four__THE_FORMS_OF_LOVE.xhtmlux5cux23id_1405}{\protect\hyperlink{23_NOTES.xhtmlux5cux23id_1406}{42}}}

The poet was not destined to have any more such luck and, lacking any
turns of fate, filled the second half of his book with endless tales
from mythology. Finally, Péronelle tells him that their relationship
must come to an end, probably because of her impending marriage. He,
however, decides to remain in love with her and to venerate her. After
their deaths, his spirit will ask God to continue to call her beatified
soul Toute-belle.

The \emph{Voir-Dit} tells us more about the customs, and also about the
emotions, than most of the amorous literature of that time. There is,
first, the extraordinary liberties the young girl could take without
causing a scandal. Next, there is the naive imperturbability with which
everything down to the most intimate acts takes place in the presence of
others, be it sister-in-law, lady-in-waiting, or secretary. During the
tryst under the cherry tree, the secretary even devises a charming
trick: while Péronelle is asleep, he places a green leaf on her mouth
and tells Machaut that he should kiss the leaf. When the poet finally
dares to do so, the secretary pulls the leaf away so that their lips
meet.\textsuperscript{\protect\hypertarget{11_Chapter_Four__THE_FORMS_OF_LOVE.xhtmlux5cux23id_1403}{\protect\hyperlink{23_NOTES.xhtmlux5cux23id_1404}{43}}}
Equally remarkable is the congruence of amorous and religious duties.
The fact that Machaut, as canon of the cathedral of Rheims, was a member
of the clergy, should not be taken too seriously. The lower orders of
clergy, which sufficed for canonical duties, did not at that time take
the vows of celibacy very seriously. Even Petrarch was a canon. That a
pilgrimage was chosen for a \emph{rendez-vous} was also not unusual.
Amorous adventures while on pilgrimage were very popular. But that
pilgrimage carried out by Machaut and Péronelle was done with great
seriousness, ``très
devotement.''\textsuperscript{\protect\hypertarget{11_Chapter_Four__THE_FORMS_OF_LOVE.xhtmlux5cux23id_1401}{\protect\hyperlink{23_NOTES.xhtmlux5cux23id_1402}{44}}}\protect\hypertarget{11_Chapter_Four__THE_FORMS_OF_LOVE.xhtmlux5cux23id_3157}{\protect\hyperlink{23_NOTES.xhtmlux5cux23id_3158}{*\textsuperscript{30}}}
At an earlier get-together, they hear mass; he sits behind her:

.~.~.
\protect\hypertarget{11_Chapter_Four__THE_FORMS_OF_LOVE.xhtmlux5cux23page_147}{}{}\emph{Quant
on dist: Agnus dei},

\emph{Foy que je doy à Saint Crepais},

\emph{Doucement me donna la pais},

\emph{entre deux pilers du moustier}

\emph{Et j'en avoie bien mestier},

\emph{Car mes cuers amoureus estoit}

\emph{Troublés, quant si tost se
partoit}.\textsuperscript{\protect\hypertarget{11_Chapter_Four__THE_FORMS_OF_LOVE.xhtmlux5cux23id_1399}{\protect\hyperlink{23_NOTES.xhtmlux5cux23id_1400}{45}}}\emph{\protect\hypertarget{11_Chapter_Four__THE_FORMS_OF_LOVE.xhtmlux5cux23id_3159}{\protect\hyperlink{23_NOTES.xhtmlux5cux23id_3160}{*\textsuperscript{31}}}}

The ``pais'' was a small plate that was passed around to be kissed in
place of the ``Kiss of Peace'' that was given mouth to
mouth.\textsuperscript{\protect\hypertarget{11_Chapter_Four__THE_FORMS_OF_LOVE.xhtmlux5cux23id_1397}{\protect\hyperlink{23_NOTES.xhtmlux5cux23id_1398}{46}}}
In this case, the meaning is that Péronelle offered him her own lips. He
awaits her in the garden, reciting his breviary. Upon beginning a novena
(a nine-day sequence of certain prayers) he takes a silent vow while
entering the church that he will on each of the nine days compose a new
poem about love. This does not keep him from speaking of the great
devotion with which he
prays.\textsuperscript{\protect\hypertarget{11_Chapter_Four__THE_FORMS_OF_LOVE.xhtmlux5cux23id_1395}{\protect\hyperlink{23_NOTES.xhtmlux5cux23id_1396}{47}}}

We should not assume that there were frivolous or profane intentions
behind all of this. Guillaume de Machaut, all else being said, is an
earnest and high-minded poet. We are encountering here the almost
incomprehensible way in which, in pre-Tridentine days, the exercise of
faith was interwoven with daily life. Soon we will have to say more
about this.

The emotion revealed by the letters and the description of this historic
love affair is soft, sweetish, and a little sickly. The expression of
emotion remains veiled in the narrative flow of words, rationalizing and
deliberating, and in the garb of allegorical phantasies and dreams.
There is something touching about the deep emotions with which the
graying poet describes his own glorious good fortune and the outstanding
qualities of Toute-Belle while failing to realize that she is only
playing with him and with her own heart.

At almost the same time as Machaut's \emph{Voir-Dit} there appears
another work that is in certain ways comparable: \emph{Le livre du
chevalier de la Tour Landry pour l'enseignement de ses
filles}.\textsuperscript{\protect\hypertarget{11_Chapter_Four__THE_FORMS_OF_LOVE.xhtmlux5cux23id_1393}{\protect\hyperlink{23_NOTES.xhtmlux5cux23id_1394}{48}}}
It is an aristocratic work just as much as the romance of Machaut and
Péronelle d'Armentières, which was played out in Champagne and in Paris.
\protect\hypertarget{11_Chapter_Four__THE_FORMS_OF_LOVE.xhtmlux5cux23page_148}{}{}The
Knight de la Tour Landry takes us to Anjou and Poitou. Here, though,
there is no aged poet in love, but a somewhat prosaic father who offers
reminiscences of his youth, anecdotes and stories ``pour mes filles
aprandre à roumancier.'' We would say, to teach them the civilized forms
of love. But the instructions are far from being romantic. Rather the
examples and admonishments that the careful nobleman holds up to his
daughters tend to be warnings against romantic flirtations. Be on guard
against silver-tongued people who are always ready with ``faux regars
longs et pensifs et petits soupirs et de merveilleuses contenances
affectées et ont plus de paroles à main que autres
gens.''\textsuperscript{\protect\hypertarget{11_Chapter_Four__THE_FORMS_OF_LOVE.xhtmlux5cux23id_1391}{\protect\hyperlink{23_NOTES.xhtmlux5cux23id_1392}{49}}}\protect\hypertarget{11_Chapter_Four__THE_FORMS_OF_LOVE.xhtmlux5cux23id_3161}{\protect\hyperlink{23_NOTES.xhtmlux5cux23id_3162}{*\textsuperscript{32}}}
Don't be too accommodating. As a youth he had once been taken by his
father to a castle to make the acquaintance of the daughter of the lord
of the manor with a view to a prospective engagement. The girl had
received him with particular kindness. To find out her true qualities he
had spoken with her about all kinds of things. The talk turned to
prisoners, and the youth paid the girl a dignified compliment, ``\,'Ma
demoiselle, il vaudroit mieulx cheoir à estre vostre prisonnier que à
tout plain d'autres, et pense que vostre prison ne seroit pas si dure
comme celle des Angloys.'---Se me respondit, qu'elle avoyt vue nagaires
cel qu'elle vouldroit bien qu'il feust son prisonnier. Et lors je luy
demanday se elle luy feroit male prison, et elle ne dit que nennil dt
qu'elle le tandroit ainsi chier comme son propre corps, et je lui dis
que celui estoit bien eureux d'avoir si doulce et si noble prison. Que
vous dirai-je? Elle avoit assez de langaige et lui sambloit bien, selon
ses parolles, qu'elle savoit assez, et si avoit l'ueil bien vif et
legier.'' Upon taking leave she asked him two or three times to come
again as if she had already known him for a long time. ``Et quant nous
fumes partis, mon seigneur de père me dist: Que te samble de celle que
tu as veue. Dy m'en ton avis.'' But her all too ready encouragement had
cooled any ardor for a closer acquaintanceship. ``Mon seigneur, elle me
samble belle et bonne, maiz je ne luy seray jà plus de près que je suis,
si vous
plaist.''\protect\hypertarget{11_Chapter_Four__THE_FORMS_OF_LOVE.xhtmlux5cux23id_3163}{\protect\hyperlink{23_NOTES.xhtmlux5cux23id_3164}{†\textsuperscript{33}}}
So the engagement did not take place, and the knight later
\protect\hypertarget{11_Chapter_Four__THE_FORMS_OF_LOVE.xhtmlux5cux23page_149}{}{}naturally
found out things that gave him no cause for
regret.\textsuperscript{\protect\hypertarget{11_Chapter_Four__THE_FORMS_OF_LOVE.xhtmlux5cux23id_1389}{\protect\hyperlink{23_NOTES.xhtmlux5cux23id_1390}{50}}}
Similar little bits taken directly from life that would inform us how
customs adapted themselves to the ideal are unfortunately exceedingly
rare for the centuries with which we are concerned. If only the Knight
de la Tour Landry had told us still more about his life! Most of his
reminiscences are of a general nature. He desires for his daughters most
of all a good marriage. And marriage had little to do with love. He
presents a detailed ``debat'' between himself and his wife about what is
permissible in matters of love, ``le fait d'amer par
amours.''\protect\hypertarget{11_Chapter_Four__THE_FORMS_OF_LOVE.xhtmlux5cux23id_2313}{\protect\hyperlink{23_NOTES.xhtmlux5cux23id_2314}{*\textsuperscript{34}}}
He believes that in certain circumstances a girl may well find honorable
love, for example, ``en esperance de
mariage.''\protect\hypertarget{11_Chapter_Four__THE_FORMS_OF_LOVE.xhtmlux5cux23id_2316}{\protect\hyperlink{23_NOTES.xhtmlux5cux23id_2315}{†\textsuperscript{35}}}
His wife is opposed. It is better for a girl not to fall in love at all,
not even with her husband, as it keeps her from true piety. ``Car j'ay
ouy dire à plusieurs, qui avoient esté amoureuses en leur juenesce, que
quant elles estoient à l'eglise, que la pensée et la
merencollie\textsuperscript{\protect\hypertarget{11_Chapter_Four__THE_FORMS_OF_LOVE.xhtmlux5cux23id_1387}{\protect\hyperlink{23_NOTES.xhtmlux5cux23id_1388}{51}}}
leur faisoit plus souvent penser a ces estrois pensiers et dliz de leurs
amours que ou (au) service de
Dieu,\textsuperscript{\protect\hypertarget{11_Chapter_Four__THE_FORMS_OF_LOVE.xhtmlux5cux23id_1385}{\protect\hyperlink{23_NOTES.xhtmlux5cux23id_1386}{52}}}
et est l'art d'amours de telle nature que quant l'en (on) est plus au
divin office, c'est tant comme le prestre tient nostre seigneur sur
l'autel, lors leur venoit plus de menus
pensiers.''\textsuperscript{\protect\hypertarget{11_Chapter_Four__THE_FORMS_OF_LOVE.xhtmlux5cux23id_1383}{\protect\hyperlink{23_NOTES.xhtmlux5cux23id_1384}{53}}}\protect\hypertarget{11_Chapter_Four__THE_FORMS_OF_LOVE.xhtmlux5cux23id_2317}{\protect\hyperlink{23_NOTES.xhtmlux5cux23id_2318}{‡\textsuperscript{36}}}
With this deep psychological observation, Machaut and Péronelle would be
in agreement. But aside from that, what a difference in perception
between the poet and the knight! But how do we reconcile the strictness
of the father with the fact that in order to instruct his daughters he
repeatedly
\protect\hypertarget{11_Chapter_Four__THE_FORMS_OF_LOVE.xhtmlux5cux23page_150}{}{}uses
stories that, given their salacious content, would not be misplaced
among those of the \emph{Cent nouvelles nouvelles?}

This loose fit between the beautiful forms of the courtly ideal of love
and the reality of engagement and marriage means that the element of
play, of conversation, of literary conventions could unfold with little
restraint in anything having to do with the refined art of love. There
was no room for the ideal of love, for the fiction of faithfulness and
sacrifice, in the very material considerations that enter into a
marriage, above all an aristocratic marriage. They could only be
experienced in the form of beguiling or heart-thrilling play. The
tournament provided the game of romantic love its heroic form, the
pastoral, the idyllic.

The pastoral in its real significance is something more than a mere
literary genre. We are not dealing here with a description of the real
life of the shepherd and its simple and natural enjoyments, but rather
with its echoed life. The pastoral is an \emph{imitatio}. There is a
fiction that in pastoral life the undisturbed naturalness of love finds
its essential expression. There is where one can escape, if not in
reality, then in dreams. Time and again the pastoral serves as the means
to liberate the spirit from the clutch of a highly pressured, dogmatic,
and formalized view of love. There is a yearning for deliverance from
the oppressive requirements of knightly faithfulness and veneration and
from the colorful apparatus of allegory as well as from the crudity, the
greed, the social sins, of the life of love in reality. An easy,
satisfied, and simple love amidst the innocent enjoyments of nature
seems most desirable. That is what appeared to be the lot of Robin and
Marion and of Gontier and Helayne; they were the lucky ones, worthy of
envy. The much maligned peasant, in his turn, became the ideal.

But the late Middle Ages are still so genuinely aristocratic and
vulnerable to beautiful illusions that passion for the life of nature
could not yet lead to a vigorous realism except that it be linked in
practice to an artful ornamentation of courtly customs. When the
aristocracy of the fifteenth century played shepherd and shepherdess the
genuine veneration of nature and the admiration of simplicity and work
are still very weak. When, three centuries later, Marie-Antoinette milks
cows and makes butter in the Trianon, the ideal is already filled with
the seriousness of the physiocrats. Nature and work have already become
the great sleeping deities of the time, yet aristocratic culture still
managed to make a game of it all.
\protect\hypertarget{11_Chapter_Four__THE_FORMS_OF_LOVE.xhtmlux5cux23page_151}{}{}When
the intellectual Russian youth around 1870 placed themselves among the
people, to live like peasants for the sake of the peasants, at that
point the ideal became bitterly serious. But then too, it turned out
that its realization was a delusion.

There is a poetic form that represents the middle ground between the
pastoral proper and reality. This is the pastorelle, the short poem that
sings of the opportune adventure between the knight and the country
girl. In these, the overtly erotic found a fresh and elegant form, which
raised it above the obscene and yet still managed to retain all the
charm of naturalism. They bring to mind certain scenes from Guy de
Maupassant.

The sentiment is truly pastoral only at the moment when the lover begins
to feel himself to be a shepherd. In this, any contact with reality
vanishes. All the elements of the courtly system of love are merely
transported into a rural setting; a sunny dreamland engulfs yearning in
a mist of flute tunes and bird twitters. It is a gay sound; even the
sorrows of love, yearning, and lamentation, even the agony of those who
are abandoned dissolve in the lovely sound. In the pastoral, the erotic
always finds that indispensable contact with the joys of nature. Thus,
the pastoral became the field wherein the literary feeling for nature
developed. In the beginning it was not yet concerned with the
description of the beauty of nature, but rather with the immediate
enjoyment of sun and summer, shade and fresh water, flowers and birds.
The observation of nature and its description is only a secondary
consideration, the main concern is the dream of love. As a by-product,
nature poetry offers quite a bit of charming realism. The description of
life on the land in a poem such as ``Le dit de la pastoure'' by
Christine de Pisan creates a new genre.

Once it has taken its place as a courtly ideal, the simple life becomes
a mask. Everything can be put in a country costume. The imaginative
spheres of the pastoral and the knightly romance merge. Tournaments were
held in pastoral dress. King René calls his the \emph{Pas d'armes de la
bergère}.

His contemporaries seem to have seen in this comedy something really
genuine; Chastellain gives René's pastoral vision a place among the
wonders of the world:

\emph{J'ay un roi de Cécille}

\emph{Vu devenir berger}

\emph{\protect\hypertarget{11_Chapter_Four__THE_FORMS_OF_LOVE.xhtmlux5cux23page_152}{}{}Et
sa femme gentille}

\emph{De se mesme mestier},

\emph{Portant la pannetière},

\emph{La houlette et chappeau},

\emph{Logeans sur la bruyère}

\emph{Auprès le leur
trouppeau}.\textsuperscript{\protect\hypertarget{11_Chapter_Four__THE_FORMS_OF_LOVE.xhtmlux5cux23id_1381}{\protect\hyperlink{23_NOTES.xhtmlux5cux23id_1382}{54}}}\emph{\protect\hypertarget{11_Chapter_Four__THE_FORMS_OF_LOVE.xhtmlux5cux23id_2929}{\protect\hyperlink{23_NOTES.xhtmlux5cux23id_2930}{*\textsuperscript{37}}}}

In another instance, the pastoral served to dress a slanderous political
satire. There is no stranger work of art than the long shepherd poem
``Le
pastoralet,''\textsuperscript{\protect\hypertarget{11_Chapter_Four__THE_FORMS_OF_LOVE.xhtmlux5cux23id_1379}{\protect\hyperlink{23_NOTES.xhtmlux5cux23id_1380}{55}}}
in which a partisan of the Burgundians tells in charming guise of the
murder of Louis de Orléans so that the misdeed of John the Fearless is
excused and all the Burgundian hatred of the Duke of Orléans is vented.
Léonet is John's shepherd name, Tristifer that of Orléans; the fantasy
of dance and floral decoration is done in a strange manner. Even the
battle of Agincourt is done in pastoral
guise.\textsuperscript{\protect\hypertarget{11_Chapter_Four__THE_FORMS_OF_LOVE.xhtmlux5cux23id_1377}{\protect\hyperlink{23_NOTES.xhtmlux5cux23id_1378}{56}}}

The pastoral element was never missing from court festivities. It was
exceptionally suited for the masquerades that as ``entremets'' provided
glamor for festive meals and that were especially suited for political
allegories. The picture of the prince as shepherd and the people as his
flock had already been presented from another side: from the
representation of the original form of the state by the church fathers.
The patriarchs had lived as herdsmen; the proper role of authority, for
the secular as well as the spiritual, was not to rule, but to guard.

\emph{Seigneur, tu es de Dieu bergier};

\emph{garde ses bestes loyaument},

\emph{Mets les en champ ou en vergier},

\emph{Mais ne les perds aucunement},

\emph{Pour ta peine auras bon paiement}

\emph{En bien le gardant, et se non},

\emph{A male heure reçus ce
nom}.\textsuperscript{\protect\hypertarget{11_Chapter_Four__THE_FORMS_OF_LOVE.xhtmlux5cux23id_1375}{\protect\hyperlink{23_NOTES.xhtmlux5cux23id_1376}{57}}}\protect\hypertarget{11_Chapter_Four__THE_FORMS_OF_LOVE.xhtmlux5cux23id_2931}{\protect\hyperlink{23_NOTES.xhtmlux5cux23id_2932}{†\textsuperscript{38}}}

\protect\hypertarget{11_Chapter_Four__THE_FORMS_OF_LOVE.xhtmlux5cux23page_153}{}{}In
these verses from Jean Meschinot's ``Lunettes des princes'' there is no
mention of a truly pastoral image. However, as soon as there is an
attempt to represent something like this visually, the two notions, the
prince as caretaker and the simple shepherd, automatically merge. One
\emph{entremet} at a wedding fest in Bruges in 1468 glorified earlier
princesses as the ``nobles bergieres qui par cy devant ont esté
pastoures et gardes des brebis de
pardeça.''\textsuperscript{\protect\hypertarget{11_Chapter_Four__THE_FORMS_OF_LOVE.xhtmlux5cux23id_1373}{\protect\hyperlink{23_NOTES.xhtmlux5cux23id_1374}{58}}}\protect\hypertarget{11_Chapter_Four__THE_FORMS_OF_LOVE.xhtmlux5cux23id_2933}{\protect\hyperlink{23_NOTES.xhtmlux5cux23id_2934}{*\textsuperscript{39}}}
A play in Valenciennes in 1493 to celebrate the return of Marguerite of
Austria from France showed how the country had recovered from its
devastations ``le tout en
bergerie.''\textsuperscript{\protect\hypertarget{11_Chapter_Four__THE_FORMS_OF_LOVE.xhtmlux5cux23id_1371}{\protect\hyperlink{23_NOTES.xhtmlux5cux23id_1372}{59}}}\protect\hypertarget{11_Chapter_Four__THE_FORMS_OF_LOVE.xhtmlux5cux23id_2935}{\protect\hyperlink{23_NOTES.xhtmlux5cux23id_2936}{†\textsuperscript{40}}}
We all know the political pastoral in \emph{De
Leeuwendalers}.\textsuperscript{\protect\hypertarget{11_Chapter_Four__THE_FORMS_OF_LOVE.xhtmlux5cux23id_1369}{\protect\hyperlink{23_NOTES.xhtmlux5cux23id_1370}{60}}}
The note of prince as shepherd is also audible in the
\emph{Wilhelmus}:\textsuperscript{\protect\hypertarget{11_Chapter_Four__THE_FORMS_OF_LOVE.xhtmlux5cux23id_1367}{\protect\hyperlink{23_NOTES.xhtmlux5cux23id_1368}{61}}}

\emph{Oirlof myn arme schapen}

\emph{Die syt in groter noot},

\emph{Uw herder sal niet slapen},

\emph{Al syt gy nu
verstroyt}.\protect\hypertarget{11_Chapter_Four__THE_FORMS_OF_LOVE.xhtmlux5cux23id_2937}{\protect\hyperlink{23_NOTES.xhtmlux5cux23id_2938}{‡\textsuperscript{41}}}

Even in real war, people played with the notion of the pastoral. Charles
the Bold's bombardment of Granson was called ``le berger et la
bergère.''\protect\hypertarget{11_Chapter_Four__THE_FORMS_OF_LOVE.xhtmlux5cux23id_2939}{\protect\hyperlink{23_NOTES.xhtmlux5cux23id_2940}{§\textsuperscript{42}}}
When the French mocked the Flemings, calling them shepherds unfit for
war, Phillip of Ravenstein showed up on the field with twenty-four
nobles dressed as shepherds with crooks and bread
baskets.\textsuperscript{\protect\hypertarget{11_Chapter_Four__THE_FORMS_OF_LOVE.xhtmlux5cux23id_1365}{\protect\hyperlink{23_NOTES.xhtmlux5cux23id_1366}{62}}}

Even as true knightly devotion set against the ideas of the \emph{Roman
de la rose} provided the material for an elegant literary war, so too
the pastoral ideal became the subject of such a struggle. Here too, the
falsity was too obvious and had to be masked. How little did the
hyperbolically contrived, wastefully colored life of the late Middle
Ages resemble the ideal of simplicity, freedom, and carefree true love
in the midst of nature! The theme of Philippe de Vitri's ``Franc
Gontier,'' the type of the simplicity of the Golden Age, was
\protect\hypertarget{11_Chapter_Four__THE_FORMS_OF_LOVE.xhtmlux5cux23page_154}{}{}given
endless variations. Everyone claimed to hunger for Franc Gontier's meal
in the shade with Lady Helayne, for his menu of cheese, butter, cream,
apples, onions, and brown bread, for his lusty wood chopping work, his
sense of freedom and lack of care:

\emph{Mon pain est bon; ne faut que nulz me veste};

\emph{L'eaue est saine qu'à boire sui enclin},

\emph{je ne doubte ne tirant ne
venin}.\textsuperscript{\protect\hypertarget{11_Chapter_Four__THE_FORMS_OF_LOVE.xhtmlux5cux23id_1363}{\protect\hyperlink{23_NOTES.xhtmlux5cux23id_1364}{63}}}\protect\hypertarget{11_Chapter_Four__THE_FORMS_OF_LOVE.xhtmlux5cux23id_2941}{\protect\hyperlink{23_NOTES.xhtmlux5cux23id_2942}{*\textsuperscript{43}}}

Sometimes the poets temporarily misstep. The same Eustache Deschamps who
repeatedly sang the life of Robin and Marion and the praise of natural
simplicity and a life filled with work regrets that the court dances to
the music of the cornemuse, ``cet instrument des hommes
bestiaulx.''\textsuperscript{\protect\hypertarget{11_Chapter_Four__THE_FORMS_OF_LOVE.xhtmlux5cux23id_1361}{\protect\hyperlink{23_NOTES.xhtmlux5cux23id_1362}{64}}}\protect\hypertarget{11_Chapter_Four__THE_FORMS_OF_LOVE.xhtmlux5cux23id_2943}{\protect\hyperlink{23_NOTES.xhtmlux5cux23id_2944}{†\textsuperscript{44}}}
But it took the much deeper sensitivity and sharp skepticism of François
Villon to see through all the falsity of the beautiful dream. There is a
merciless mockery in the ballade ``Les contrediz Franc Gontier.''
Cynically, Villon compares the lightheartedness of that ideal countryman
with his meal of onions ``qui causent fort
alaine''\protect\hypertarget{11_Chapter_Four__THE_FORMS_OF_LOVE.xhtmlux5cux23id_2945}{\protect\hyperlink{23_NOTES.xhtmlux5cux23id_2946}{‡\textsuperscript{45}}}
and his love under the roses with the comfortable life of the fat priest
who has comfort and joy in a well-furnished room with a fire in the
fireplace, good wine, and a soft bed. The brown bread and the water of
Franc Gontier? ``Tous les oyseaulx d'ici en
Babiloine''\protect\hypertarget{11_Chapter_Four__THE_FORMS_OF_LOVE.xhtmlux5cux23id_2947}{\protect\hyperlink{23_NOTES.xhtmlux5cux23id_2948}{§\textsuperscript{46}}}
would not be able to make Villon suffer such fare even one
morning.\textsuperscript{\protect\hypertarget{11_Chapter_Four__THE_FORMS_OF_LOVE.xhtmlux5cux23id_1359}{\protect\hyperlink{23_NOTES.xhtmlux5cux23id_1360}{65}}}

Even as did the beautiful dream of the knightly ideal, the other forms
in which sex tried to become culture had to be recognized as false and
full of lies. Neither the infatuated ideal of noble, chaste, knightly
faithfulness, nor the refined lust of the \emph{Roman de la rose}, nor
the sweet, comfortable fantasy of the pastoral could hold their own
against the storm of life itself. The storm blew from all sides. From
the spiritual side came the curse on everything, since sex is the sin
that ruins the world. At the bottom of the chalice of the \emph{Roman de
la rose}, the moralist sees all the bitter sediment. ``Whence,'' cries
Gerson, ``whence the bastards, whence the murder
\protect\hypertarget{11_Chapter_Four__THE_FORMS_OF_LOVE.xhtmlux5cux23page_155}{}{}of
children, the abortions, whence the hatred and poisoning in
marriage?''\textsuperscript{\protect\hypertarget{11_Chapter_Four__THE_FORMS_OF_LOVE.xhtmlux5cux23id_1357}{\protect\hyperlink{23_NOTES.xhtmlux5cux23id_1358}{66}}}

From the side of woman, another charge rings out. All these conventional
forms of love are the work of men. Even when it is enjoyed in idealized
forms, erotic culture is through and through the product of male
self-seeking. What else is the constantly repeated mocking of marriage
and the weaknesses of women, their unfaithfulness and conceit, but a
cover for male self-centeredness? To all this defamation I only respond,
says Christine de Pisan, that it is not women who wrote these
books.\textsuperscript{\protect\hypertarget{11_Chapter_Four__THE_FORMS_OF_LOVE.xhtmlux5cux23id_1355}{\protect\hyperlink{23_NOTES.xhtmlux5cux23id_1356}{67}}}

Actually, in the entire erotic as well as the pious literature of the
Middle Ages there is hardly a trace of genuine pity for women, for their
weakness and the pain and danger that love causes them. Pity had
formalized itself into the fiction of the liberation of the virgin that
was really only sensual stimulation and self-satisfaction. After the
author of the \emph{Quinze joyes de manage} had listed all the
weaknesses of women in a mutely toned and finely colored satire, he
offers to describe the neglect of
women,\textsuperscript{\protect\hypertarget{11_Chapter_Four__THE_FORMS_OF_LOVE.xhtmlux5cux23id_1353}{\protect\hyperlink{23_NOTES.xhtmlux5cux23id_1354}{68}}}
but he does not do it. For the expression of a tender womanly voice, one
has to turn to the poetry of Christine herself:

\emph{Dolce chose est que mariage},

\emph{Je le puis bien par moy prouver} .~.~.
\textsuperscript{\protect\hypertarget{11_Chapter_Four__THE_FORMS_OF_LOVE.xhtmlux5cux23id_1351}{\protect\hyperlink{23_NOTES.xhtmlux5cux23id_1352}{69}}}\protect\hypertarget{11_Chapter_Four__THE_FORMS_OF_LOVE.xhtmlux5cux23id_2949}{\protect\hyperlink{23_NOTES.xhtmlux5cux23id_2950}{*\textsuperscript{47}}}

But how weak the voice of a single woman sounds against the choir of
ridicule in which the flat voices of licentiousness and pious morality
come together. There is only a small distance between the homiletic
contempt of women and the coarse denial of ideal love by prosaic
sensuality.

The beautiful play of love as a form of life continued to be played in
the knightly style, and in the pastoral, and in the artificial dress of
the rose allegory and even though, from all sides, there could be heard
the sound of the denial of all these conventions, they retained their
value for life and culture until long after the Middle Ages because
there are only a few forms in which the ideal of love can dress itself
in any age.
